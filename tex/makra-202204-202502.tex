\shyph % use csplain
\input opmac
\margins/1 a5 (1.2, 1.2, 1.2, 1.7)cm
\parindent 1em % odstavcova zarazka
\emergencystretch=2em % roztiahnutie medzislovnych medzier v riadku

\hyperlinks{\Blue}{\Blue} % farba hyperlinkov - vnutorne aj URL modre

% Fonty

% Rodina fontov
%\input lmfonts % Latin Modern (Unicode)
\input cs-pagella % font Palatino
%\font\tenrm="Lucida Sans Unicode:mapping=tex-text"
%\font\tenit="Lucida Sans Unicode:mapping=tex-text,slant=0.2"
%\font\tenbf="Lucida Sans Unicode:mapping=tex-text,embolden=1.5"
%\font\tenbi="Lucida Sans Unicode:mapping=tex-text,slant=0.2,embolden=1.5"
%\font\tenrm="Linux Libertine O:mapping=tex-text"
%\font\tenit="Linux Libertine O/I:mapping=tex-text"
%\font\tenbf="Linux Libertine O/B:mapping=tex-text"
%\font\tenbi="Linux Libertine O/BI:mapping=tex-text"
% Poznamka k "Butler" fontom: Fonty stiahnute z webstranky tvorcu nie su priamo pouzitelne, je potrebne ich otvorit v programe FontForge a vyexportovat (File/Generate Fonts...).
% Dovod: niektore pismena s diakritikou nie su vo fonte ulozene korektne (neviem co to presne znamena) a XeTeX ich nevie pouzit, FontForge zobrazuje pre nich nejaky warning.
% Reexport fontu odstranuje problem (nevie preco), warning zmizne a XeTeX je tiez spokojny. Tento postup som nasiel na internete (v suvislosti s inym fontom).
%\font\tenrm="[Butler_Regular-reexported]:mapping=tex-text"
%\font\tenit="[Butler_Regular-reexported]:mapping=tex-text,slant=0.2"
%\font\tenbf="[Butler_Bold-reexported]:mapping=tex-text"
%\font\tenbi="[Butler_Bold-reexported]:mapping=tex-text,slant=0.2"

% Zakladna velkost fontu - vyzaduje existenciu makier \velkostpisma a \velkostriadku
% 9/12 - standard
% 10/12.5 - trochu vacsie, vhodne ak napr. "standard" vychadza na neparny pocet stran
\typosize[\velkostpisma/\velkostriadku]

% Fonty pre specificke pouzitie
\def\ftitle{\typosize[42/]\bf}
\def\ffootertitle{\typosize[9.5/]}
\def\fissueyear{\typosize[20/]\bf}
\def\fissuemonth{\ftitle}

\def\monthname#1{%
\ifcase#1
\or január% 1
\or február% 2
\or marec% 3
\or apríl% 4
\or máj% 5
\or jún% 6
\or júl% 7
\or august% 8
\or september% 9
\or október% 10
\or november% 11
\or december% 12
\fi}
\def\nazovmesiaca{\monthname\cislomesiaca}

% Titulka
\newcount\issueyear
\def\rom#1{\uppercase\expandafter{\romannumeral#1}}

% Titulka - spravodaj
\def\spravodaj#1#2{\gdef\cislomesiaca{{#1}}\gdef\rokvydania{{#2}}%
\def\footertext{{\ffootertitle spravodaj}\kern0.8em #1/#2}%
%v oboch nasledujucich riadkoch musi byt ciselny rozmer rovnaky ale s opacnym znamienkom (napr. 4cm a -4cm)
\vbox{\picheight=5.7cm\hfill\inspic kostol.pdf
\vskip-5.7cm\vskip0.2cm
{\noindent\ftitle Spravodaj\hfill#1}\par
\vskip-0.3cm
{\noindent\hfill{\fissueyear\ #2}}\par
\issueyear = #2
\advance \issueyear by -1997
\vskip-0.1cm
\noindent\hfill ročník \rom{\the\issueyear}\par
\vskip-0.53cm
\noindent {\typosize[11/15]\bf cirkevného zboru BJB Palisády}\par
\noindent{\typosize[9/12]\ulink[https://www.bjbpalisady.sk/spravodaj]{www.bjbpalisady.sk/spravodaj}}\par
\vskip 1.4cm
}}

% Titulka - vyrocne spravy
\def\vyrocnespravy#1{\gdef\rokvydania{#1}%
\def\footertext{{\ffootertitle \bf výročné správy za rok #1}}%
%v oboch nasledujucich riadkoch musi byt ciselny rozmer rovnaky ale s opacnym znamienkom (napr. 4cm a -4cm)
\null
\vskip1.5cm
\picheight=4cm\inspic kostol.pdf
\vskip-4cm
\vskip-1.5cm
{\noindent\ftitle Výročné správy\par}
{\noindent\ftitle\hfill za rok #1\par}
\vskip0.6cm
\noindent {\hfill\typosize[11/15]\bf cirkevný zbor BJB Palisády}\par
\noindent{\hfill\typosize[9/12]\ulink[https://www.bjbpalisady.sk/spravodaj]{www.bjbpalisady.sk/spravodaj}}\par
%\vskip 0.1cm
\null}

\footline={\ifnum\pageno=1 \hfil \else \ifodd\pageno \footertext\hfil\the\pageno \else \the\pageno\hfil\footertext\fi\fi}%

% Tiraz
\long\def\tiraz{
\footline={\hfil} % na poslednej strane nechceme paticku
\vfill % spolu s \vfil\break na konci makra to sposobi umiestnenie textu na koniec stranky
\centerline{\typosize[8.1/11.5]\table{|c|}{\crl
e-mail: \email{zbor@bjbpalisady.sk} webstránka: \ulink[http://www.bjbpalisady.sk/]{www.bjbpalisady.sk} YouTube: \ulink[https://bit.ly/3aL34co]{https://bit.ly/3aL34co} \cr
vydal Cirkevný zbor Bratskej jednoty baptistov Bratislava I Palisády (Zrínskeho 2, 81103)\cr
bankové spojenie (IBAN): SK36 0900 0000 0000 1147 1836\cr
redakcia: P. Šrankota, K. Kerekréty, obálka: M. Kohút, sadzba: I. Kohút\crl
}}
\vfil\break
}

\def\clanok#1{\nonum\secc{#1}}
\def\cast#1{\vskip1ex\noindent{\bf #1}\par\nobreak\firstnoindent}
\def\autor#1{\nobreak{\hfill\tenit #1}\par}

% Narodeniny

\newcount\pocetoslavencov \pocetoslavencov=0
\def\oslavencimeno[#1]{\csname oslavencimeno:#1\endcsname}
\def\pridajoslavenca#1{\advance\pocetoslavencov by 1
\expandafter\def\csname oslavencimeno:\the\pocetoslavencov\endcsname{#1}
}
\def\n #1;{\pridajoslavenca{#1}}

\newdimen\halfhsize
\halfhsize=\hsize
\divide\halfhsize by 2

\def\printnarodeniny#1{\hskip0.5em #1\hfil\vrule}
\def\printoslavenec#1{\printnarodeniny{\oslavencimeno[#1]}}
\def\printprazdnyoslavenec{\printnarodeniny{}{}{}}
\def\printriadoknarodenin#1#2{\hbox to\hsize{\hbox to\halfhsize{\vrule height2.5ex depth1ex #1}#2}\par\hrule}

\newcount\oslavenciindex \oslavenciindex=0
\newcount\oslavenciindexleft
\newcount\oslavenciindexright
\newcount\polovicapoctuoslavencov

\def\narodeniny{
% hlavicka
\vfill
\vbox{
\centerline{\bf Výročie príchodu na tento svet oslavujú}\nobreak
\frame{\vbox{
\hrule\nobreak
% oslavenci
\polovicapoctuoslavencov=\pocetoslavencov
\divide\polovicapoctuoslavencov by 2
\ifodd \pocetoslavencov \advance\polovicapoctuoslavencov by 1 \fi
\loop\ifnum \polovicapoctuoslavencov>\oslavenciindex{
  \oslavenciindexleft=\oslavenciindex
  \advance\oslavenciindexleft by 1
  \oslavenciindexright=\oslavenciindexleft
  \advance\oslavenciindexright by \polovicapoctuoslavencov
  \parindent=0pt
  \printriadoknarodenin{\printoslavenec{\the\oslavenciindexleft}}{\ifnum\oslavenciindexright>\pocetoslavencov\printprazdnyoslavenec\else\printoslavenec{\the\oslavenciindexright}\fi}
}
\advance\oslavenciindex by 1
\repeat
}}}
\vfil\break
}

% Program

\newcount\hasmultieventrowtoday % 0 - default, netlacime zvislu ciaru za prvym eventom v danom riadku, 1 - tlacime medzeru, lebo v dany den uz boli dva eventy v jednom riadku
\hasmultieventrowtoday=0
\newcount\issunday
\def\textcell#1#2{\hskip2pt\vtop{\noindent\hsize#2cm\baselineskip=1pt#1{\emergencystretch=2em\par}\vskip1pt}\hskip2pt}
\def\programevent#1#2#3#4{\vrule\vbox{
  \hbox{%
    \vtop{\vbox to0pt{}\hrule height0pt\hbox to.4cm{\hfil #1}}%
    \vtop{\vbox to0pt{}\hrule height0pt\hbox to.4cm{\hskip0.5pt\typoscale[700/]#2\hfil}}%
  }
  \hrule height0pt
  \vbox to0pt{}
}%
\textcell{#3}{#4}%
}
\def\pp#1.#2;#3;#4.#5;#6;#7.#8;#9;{{\typosize[8.1/12]\if^#1^\hrule height0pt\else\hrule\global\hasmultieventrowtoday=0\gdef\dayofweek{#3}\fi\hbox to\hsize{%
\ifnum\strcmp{ ne }{\dayofweek}=0\issunday=1\else\issunday=0\fi%
\vrule height2.4ex%
{%
%ciselny rozmer width a kern musi byt rovnaky, ale s opacnym znamienkom (napr. 1.5cm a -1.5cm)
\ifnum 1=\issunday{\localcolor \LightGrey \vrule width1.52cm}\kern-1.52cm\bf\fi%
\hbox to.35cm{\vrule width0pt depth0.9ex  \hfil #1\unskip}%
\hbox to.55cm{\if^#1^\ \else .~#2.\fi\hfil}%
\vrule\hbox to.6cm{#3\hfil}%
}%
\if^#7^\ifnum 0=\hasmultieventrowtoday\programevent{#4}{#5}{#6}{9.5}\else\programevent{#4}{#5}{#6}{4.45}\programevent{}{}{}{4.45}\fi\else%
\programevent{#4}{#5}{#6}{4.45}%
\programevent{#7}{#8}{#9}{4.45}%
\global\hasmultieventrowtoday=1
\fi%
\hfil\vrule\par}}}
\def\p#1;#2;#3.#4;#5;#6.#7;#8;{\pp#1.\cislomesiaca;#2;#3.#4;#5;#6.#7;#8;}
\def\programna#1#2{\def\cislomesiaca{#1}\centerline{\typosize[9/12]\bf Program na \monthname{#1}}\nobreak\frame{\vbox{#2\hrule}}}
\def\program#1{\expandafter\programna\cislomesiaca{#1}}
\def\riadokkoncaprogramu#1{\noindent\centerline{\typosize[9/12]\it#1}\par}

% Linky
\def\email#1{\ulink[mailto:#1]{\hbox{#1}}}

\rm
