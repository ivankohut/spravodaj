\def\velkostpisma{10}
\def\velkostriadku{12.5}
\input makra.tex % nacitanie Ivanom pripravenych nastaveni a prikazov
\hyphenation{star-šov-stvo} % rozdelenie slov na konci riadku, treba tu uviest slova, ktore sam nepozna

\spravodaj{5}{2024}

\def\sekcia#1{\vskip0.5em\noindent #1}

\clanok {EL OLÁM je VEČNÝ BOH}

Μeno El Olám je hebrejské meno pre Boha, ktorý nemá začiatok ani koniec, pre ktorého je jeden deň ako tisíc rokov a tisíc rokov ako jeden deň. Jeho zámery sú pevné a neotrasiteľné. Ak si aj ty vložil svoj život do Božích rúk, potom takýmto zámerom je aj tvoja budúcnosť, ktorá je plná nádeje. Keď sa modlíš k~večnému Bohu, modlíš sa k~tomu, ktorého Syn sa nazýva Alfa a Omega. Modlíš sa k~Bohu, ktorý je od vekov na veky, ktorého láska trvá naveky.

\sekcia{KĽÚČOVÝ VERŠ}

„Takto uzavreli zmluvu v~Beér-Šebe. Potom sa Abimelech a veliteľ jeho vojska Pichól pobrali a vrátili sa na územie Filištíncov. Abrahám zasadil v~Beér-Šebe tamarišku a vzýval tam meno Hospodina, Večného Boha.“ (Gn~21,32-33)

\sekcia{ZAMYSLI SA}
\vskip-1ex\begitems
* Čo ti príde na myseľ, keď rozmýšľaš nad „večným, alebo nekonečným Bohom“?
* Čo ti tieto mená napovedajú o~Bohu samom a o~povahe Jeho zasľúbení?
\enditems

{\em Pane, ako Ty, tak aj Tvoja láska a~vernosť trvajú naveky. Preto mi pomôž žiť môj život každý deň v~sile, ktorá vychádza z~vedomia, že s~Tebou budem žiť naveky.}

\sekcia{EL OLÁM je VEČNÝ BOH}
\vskip-1ex\begitems
* „Ja som až do vašej staroby ten istý, až do šedín, ja vás budem nosiť. Ja som konal, ja vás budem niesť, ja vás budem nosiť a vyslobodím.“ (Iz 46,4)
* „Či nevieš? Nepočul si? Hospodin je večný Boh, ktorý stvoril končiny zeme; neunaví sa, neustane, jeho múdrosť nemožno preskúmať. On dáva silu unavenému a bezmocnému veľkú udatnosť.“ (Iz 40,28-29)
\enditems

\vskip-1ex\begitems
* {\it Chváľ ho}: Za jeho stálosť, večnosť a vytrvalosť s~človekom (teda aj so mnou).
* {\it Ďakuj mu}: Za to, že Boh sľubuje silu bezmocnému počas celých dejín ľudstva.
* {\it Vyznaj mu}: Každý sklon sťažovať sa na nepriaznivé okolnosti a problémy namiesto toho, aby sme ich predkladali Bohu, ktorý môže priniesť zmenu.
* {\it Pros ho}: Aby ti dal silu a obnovil tvoju dôveru v~jeho moc v~jeho vernosť.
\enditems

\sekcia{PRISĽÚBENIA SPOJENÉ S~MENOM EL OLÁM}

Nestály, náladový, nevyrovnaný, rozporuplný, nerozhodný - my môžeme byť len vďační Bohu za to, že ani jedna z~týchto vlastností neplatí pre večného Boha a neovplyvňuje to ani Jeho zámery. Pevnou skalou pod nohami pre nás je, že nič vo vesmíre nie je také pevné ako Boží plán. To môžeme vnímať v~textoch Božieho slova, ktoré hovoria, že zámery Božieho srdca trvajú naveky. Práve preto si môžeme byť istí, že keď tento stvorený svet jedného dňa zanikne, Ježiš nás vzkriesi. Všetci budeme zobratí pred Jeho trón. No jedni k~odsúdeniu, druhí k~získaniu večného života.

\sekcia{PRISĽÚBENIA V~PÍSME}
\vskip-1ex\begitems
* „Hospodinove zámery sú večné, úmysly jeho srdca pretrvávajú pokolenia.“ (Ž~33,11)
* „Vstúpte do Jeho brán s~vďakou a s~chválospevom do Jeho nádvorí, ďakujte mu, dobrorečte Jeho menu, lebo Hospodin je dobrý, Jeho milosť trvá naveky a Jeho vernosť z~pokolenia na pokolenie!“ (Ž~100,4-5)
\enditems

\autor{inšpirované knihou Božie mená, Peter Šrankota}
\vfill\break


\clanok {Správy zo staršovstva}

Staršovstvo zboru sa aj v~apríli stretlo dvakrát, 9.~4. a 23.~4.~2024. Pokračujeme v~podrobnejšom analyzovaní výsledkov prieskumu NCD. Cieľom je porozumenie ako ešte viac posilniť silné stránky a náprava tých oblastí, v~ktorých zaostávame.

Príprava na voľby ďalšieho kazateľa bola predmetom oboch našich stretnutí.

Na prvom stretnutí sme sa pripravovali na DKDZ, riešili plánovanie a organizačné zabezpečenie zborového tábora a jarnej víkendovky, ako aj ďalších aktivít zboru a hľadali sme efektívne spôsoby komunikácie v~oblasti misijných projektov ako aj plnenia našich záväzkov vyplývajúcich z~nášho rozpočtu.

Druhé stretnutie bolo venované mimo iného zhodnoteniu uznesení DKDZ a príprave prezentácie týchto unesení pre členov zboru počas ZČZ, ktoré sa má uskutočniť 5.~5.~2024. Vyhodnotili sme doterajší priebeh nahlasovania záväzkov vo veci rekonštrukcie fasády kostola a diskutovali o~ďalšom postupe. Boli sme informovaní o~plánovanej rekonštrukcii zborovej kancelárie a Pali Šrankota nám odprezentoval vizualizáciu tohto projektu.

Za členov staršovstva chcem poďakovať všetkým, ktorí sa pravidelne prihovárate za nás a celý zbor vo svojich modlitbách.

\autor {za staršovstvo Radislav Nemec}


\clanok{Sestry}

Sestry sa prvý májový víkend zúčastnia Konferencie sestier z~ČR a SR v~Litoměřiciach s~názvom Srdce ženy. Ich pravidelné stredajšie stretnutie nebude. Môžu si však rezervovať termín pre sobotu 25.~5.~2024 na Dámske raňajky. Podrobnosti o~raňajkách vám prinesieme v~e-mailových oznamoch.

\n  1. 5. Milica	MALÁ;
\n  1. 5. Andrea	ČURILLOVÁ;
\n  3. 5. Dárius	KRÁĽ;
\n  4. 5. Peter	BUZÁŠ ml.;
\n  8. 5. Vladimír	KRAJČÍ;
\n  9. 5. Ján	ŠTEFKO;
\n 11. 5. Želmíra	PRAŽENICOVÁ;
\n 16. 5. Ján	SZŐLLŐS;
\n 17. 5. Lenka	KOVÁČOVÁ;
\n 18. 5. Anna	DANTEROVÁ;
\n 19. 5. Oľga	VALCHÁŘOVÁ;
\n 20. 5. Rastislav	PAULEN;
\narodeniny


\program{
\p  1 ; st ;.;;.;;
\p  2 ; št ; 18.00 ; Biblická hodina (J.~Szőllős, Zrínskeho 2) ;.;;
\p  3 ; pi ; 17.30 ; Dorast (Súľovská 2) ;.;;
\p  4 ; so ; 18.00 ; Mládež (Súľovská 2) ;.;;
\p  5 ; ne ;  9.30 ; Bohoslužby (P. Šrankota + VP) ;.;;
\p    ;    ; 17.00 ; Zborové členské zhromaždenie ;.;;
\p  6 ; po ;.;;.;;
\p  7 ; ut ; 15.15 ; Biblická hodina pre seniorov (P. Pivka, Zrínskeho 2) ;.;;
\p  8 ; st ;.;;.;;
\p  9 ; št ; 18.00 ; Biblická hodina (J.~Szőllős, Zrínskeho 2) ;.;;
\p 10 ; pi ; 17.30 ; Dorast (Súľovská 2) ;.;;
\p 11 ; so ; 18.00 ; Mládež (Súľovská 2) ;.;;
\p 12 ; ne ;  9.30 ; Bohoslužby (J.~Szőllős) ;.;;
\p 13 ; po ;.;;.;;
\p 14 ; ut ; 15.15 ; Biblická hodina pre seniorov (P. Pivka, Zrínskeho 2) ;.;;
\p 15 ; st ;.;;.;;
\p 16 ; št ; 18.00 ; Biblická hodina (J.~Szőllős, Zrínskeho 2) ;.;;
\p 17 ; pi ; 17.30 ; Dorast (Súľovská 2) ;.;;
\p 18 ; so ; 18.00 ; Mládež (Súľovská 2) ;.;;
\p 19 ; ne ;  9.30 ; Bohoslužby (P. Šrankota) ;.;;
\p 20 ; po ;.;;.;;
\p 21 ; ut ; 15.15 ; Biblická hodina pre seniorov (P. Pivka, Zrínskeho 2) ;.;;
\p 22 ; st ;.;;.;;
\p 23 ; št ; 18.00 ; Biblická hodina (J.~Szőllős, Zrínskeho 2) ;.;;
\p 24 ; pi ; 17.30 ; Dorast (Súľovská 2) ;.;;
\p 25 ; so ;  9.00 ; Dámske raňajky (Partizánska 2) ;.;;
\p    ;    ; 18.00 ; Mládež (Súľovská 2) ;.;;
\p 26 ; ne ;  9.30 ; Bohoslužby (P. Pribula) ;.;;
\p    ;    ; 17.00 ; Rodinný koncert ;.;;
\p 27 ; po ;.;;.;;
\p 28 ; ut ; 15.15 ; Biblická hodina pre seniorov (J.~Szőllős, Zrínskeho 2) ;.;;
\p 29 ; st ;.;;.;;
\p 30 ; št ; 18.00 ; Biblická hodina (J.~Szőllős, Zrínskeho 2) ;.;;
\p 31 ; pi ; 17.30 ; Dorast (Súľovská 2) ;.;;
}


\tiraz
\bye
