%\typosize[10/12.5]% - pouzita velkost pisma/riadku - trochu vacsie
\input makra.tex % nacitanie Ivanom pripravenych nastaveni a prikazov
\hyphenation{star-šov-stvo} % rozdelenie slov na konci riadku, treba tu uviest slova, ktore sam nepozna

\spravodaj{3}{2021}


\clanok {V Ježišovi máme všetko}
V angličtine je výrok: „Nezáleží na tom, čo vieš, ale koho poznáš.“ Mám kamaráta Mira, ktorý je druhý najvyššie postavený Slovák v~Európskej komisii v~Bruseli. Pred odchodom do Belgicka v~roku~1997 mi občas v~Bratislave pomohol vybaviť zopár vecí. Po našom príchode na Slovensko v~roku~1993 ma colníci požiadali o~84~000-korunovú zálohu za husle, ktoré sme sem so sebou priviezli. O~rok neskôr, keď vypršala lehota na zálohu, mi úradníčka povedala, že ešte musím jeden rok počkať. Keď sa to Miro dozvedel, povedal: „Ideme spolu na colnicu.“ Keď sme prišli do tej kancelárie, úradníčka ma hneď spoznala a vedela, o~čo ide. Miro jej položil svoju vizitku na stôl a „poprosil“ ju (vtedy boli iné časy), aby mi okamžite vrátila peniaze. O~pár minút som odišiel so svojimi peniazmi. Vedel som, že tá žena čakala na úplatok, no nevedel som, ako to riešiť. Ešte dobre, že som poznal človeka, ktorý to dokázal vyriešiť.

Milí moji, máme Niekoho omnoho schopnejšieho ako Mira. Máme živého Ježiša, prítomného a aktívneho v~nás a okolo nás počas celého dňa.{\it „Jeho božská moc nám darovala všetko potrebné pre život a nábožnosť, keď sme poznali toho, ktorý nás povolal vlastnou slávou a účinnou mocou. Tým nám daroval vzácne a veľmi veľké prisľúbenia, aby sme prostredníctvom nich mali účasť na Božej prirodzenosti a unikli skaze, ktorú vo svete spôsobuje žiadostivosť. Preto.... milosť a pokoj nech sa rozhojňuje medzi vami v~poznávaní Boha a Ježiša, nášho Pána“} (2Pt 1,2-4). Boh nám daroval túto pandemickú dobu, aby sme Ho spoznali inak a konečne prišli na to, že Ježiš postačuje. Svojou vizitkou vyrieši úplne všetko. Verme tomu!

\autor{Danny Jones}


\clanok {Správy zo staršovstva}
V minulom mesiaci sme sa stretli dva razy. Z~dôvodu dodržiavania protipandemických opatrení sa stretávame prostredníctvom Zoom-u.

Témy našich stretnutí sú vo veľkej miere poznačené pandémiou a opatreniami prijatými na našu ochranu. Hľadáme spôsoby, ako si môžeme navzájom pomáhať. V~tomto období sú medzi nami aj v~našom okolí tí, ktorí potrebujú fyzickú aj duchovnú pomoc. Uvedomili sme si, že málo vieme o~tom, kto je medzi nami chorý na COVID--19. Preto, aby sme sa mohli navzájom za seba modliť, vedieme si zoznam tých, ktorí sú chorí.

Ani v~tomto období sa nám nevyhýbajú problémy, a preto sa potrebujeme venovať aj pastoračným otázkam. Spolu s~tým súvisí aj to, že sa o~vnútrozborových otázkach neprimeranou formou diskutuje na sociálnych sieťach. Považujeme to za nesprávne.

Blíži sa termín, kedy zvykneme mať výročné zborové členské zhromaždenie. Pre jeho realizáciu pripravujeme podklady. K~nim patrí aj rozpočet pre rok 2021. Ak niekto z~členov zboru má návrh na to, čo by sme mali v~tomto roku finančne podporiť, má možnosť ešte podať návrh staršovstvu do 16.~3.~2021.

Máme novú webovú stránku, na ktorej zverejňujeme informácie, ktorými sa chceme priblížiť členom zboru, priateľom, ale aj náhodným návštevníkom.

{\it „Modlíme sa, aby vaša láska čoraz väčšmi rástla v~poznaní a v~každej skúsenosti, aby ste boli schopní rozoznávať, čo je podstatné, a aby ste boli čistí a bez úhony pre Kristov deň, naplnení ovocím spravodlivosti, ktoré je skrze Ježiša Krista na Božiu slávu a chválu"} (Flp 1,9-11)

\autor {za staršovstvo Peter Pribula st.}


\clanok {Výzva k~modlitbám}
Zdravotná situácia na Slovensku sa naďalej nezlepšuje. Veríme však, že keď sa dvaja alebo traja spolu modlia, Pán je uprostred nich a koná. Spoločná modlitba slúži aj nám na povzbudenie a budovanie. Už mesiace sme izolovaní a možností na spoločné modlitby je čoraz menej. Popritom sledujeme zhoršovanie zdravotnej situácie na Slovensku.

Z toho dôvodu sa každý týždeň v~pondelok medzi 18.00~hod. a 20.00~hod. chceme ako zbor naďalej spolu modliť. Zavolajte niekomu a modlite sa spolu za Slovensko, za zdravotný stav ľudí v~našom zbore a v~našom okolí, za situáciu a stav nemocníc, za vládu a múdrosť pri rozhodovaní, za duševný a duchovný stav ľudí okolo nás. Je celkom na vás, s~kým sa spojíte a za čo sa budete modliť. Netreba sa modliť celý čas ani dlho. Nič oficiálne nebudeme organizovať. Je to na vás. Bude však dobre, keď sa celý zbor zjednotíme pred Bohom v~modlitbe a spoločne Ho budeme prosiť o~pomoc. Každý týždeň to môže byť s~niekým iným, alebo aj vždy s~tým istým modlitebným partnerom. Modlitbou dokážeme ovplyvniť veľa. Nech je to na Jeho slávu!
\vfill\break


\clanok {Rozpočet zboru na rok 2021}
Členovia zboru budú mať možnosť zapojiť sa do online diskusie k~príprave rozpočtu 2. a 16.~marca o~19.00 hod. Záujemcovia o~túto diskusiu sa môžu prihlásiť na e-mailovej adrese \email {starsovstvo@bjbpalisady.sk}.


\clanok {Seminár pre pracovníkov s~deťmi a dorastom}
V dňoch 11.~--~14.~marca sa uskutoční seminár pre tých, ktorí sa venujú práci s~deťmi a dorastencami. Toto školenie sa uskutoční prostredníctvom štúdia biblickej knihy Jonáš. Viac informácií o~tejto službe i o~tomto tréningu nájdete na webovej stránke: \ulink [https://preceptslovakia.estranky.sk]{preceptslovakia.estranky.sk} (vzdelávací program PLA).

Koordinátorom tejto služby na Slovensku je br. kazateľ Darko Kraljik.


\clanok {Cyklus prednášok pre rodičov teenagerov}
Chceme vám dať do pozornosti cyklus 4 prednášok pre rodičov teenagerov, ktoré budú v~marci a apríli prostredníctvom Zoom-u.

Prednášajúci: Martina Vagačová, Danny a Clara Jones, Zoltán Mátyus.

Témy a dátumy nájdete v~ozname nižšie.

Tiež je možné aby si rodičia pozvali svojich známych rodičov teenagerov, ktorých by tieto témy zaujímali – aj takto môže dôjsť k~prirodzenej misii a kontaktu s~ľuďmi, ktorí možno do cirkvi neprídu, ale takúto prednášku by si prišli vypočuť. Prednášky budú samozrejme interaktívne s~možnosťou klásť otázky.

Všetky informácie s~možnosťou prihlásenia nájdete v~našom ozname na webovej stránke \ulink [https://bjbviera.sk/cyklus-prednasok-pre-rodicov-teenagerov/]{bjbviera.sk/cyklus-prednasok-pre-rodicov-teenagerov} a na facebookovej stránke \ulink [https://www.facebook.com/bjbviera/posts/2706252349499832]{facebook.com/bjbviera/posts/2706252349499832}.

Poplatok za tento cyklus je 20~€.
\vfill\break


\clanok {Motivačné stretnutia pre manželov a stretnutia so Zolim Mátyusom}
Vzhľadom na nepriaznivú situáciu v~súvislosti s~koronavírusom nie je možné organizovať motivačné stretnutia pre manželov v~Račkovej doline. Z~toho dôvodu odbor pastorácie a poradenstva v~spolupráci s~NTM pripravili niekoľko aktivít, ktoré je možné absolvovať v~online priestore.

Na webovej stránke \ulink [https://bit.ly/2OnKAbK]{bit.ly/2OnKAbK} môžete nájsť prihlasovací formulár na Motivačné stretnutia pre manželov 1. Obsah je totožný s~víkendami, ktoré sa konajú v~Račkovej doline. Tí, ktorí sa zúčastnia týchto stretnutí, budú môcť potom pokračovať Motivačnými víkendmi 2, 3 a 4.

Počas Národného týždňa manželstva (NTM) bola odvysielaná naživo cez Facebook NTM diskusia s~br. Zolim Mátyusom na tému NTM 2021 {\it Bezpečne v~manželstve}. (Na facebookovej stránke NTM je ju možné nájsť z~8.~2.~2021.)

Vzhľadom na veľký záujem (online bolo pripojených takmer 400 účastníkov a dodnes má tento záznam cez 13~000 videní), sa odbor pastorácie a poradenstva rozhodol pripraviť osem rozhovorov na 4~témy, ktoré manželstvo ohrozujú a 4~témy, ktoré ho budujú.

Termíny živých vysielaní do letných prázdnin:

\vskip-1ex\begitems
* 19. 3. 2021 o~20.00 hod. -- téma Nevera
* 23. 4. 2021 o~20.00 hod. -- téma Priateľstvo
* 21. 5. 2021 o~20.00 hod. -- téma Egoizmus
* 18. 6. 2021 o~20.00 hod. -- téma Odpustenie

\enditems

Počas živého vysielania je možné klásť otázky cez aplikáciu Slido -- anonymne.

Na sledovanie stretnutí nie je potrebný žiaden link. Je potrebný len prístup na Facebook a ísť na stránku „Národný týždeň manželstva“.


\clanok {Ak potrebujete pomoc, napíšte nám!}
V našom zbore sme zriadili e-mailovú adresu \email{pomoc@bjbpalisady.sk}, na ktorú môžete napísať, ak ste sa dostali do zlej situácie alebo potrebujete nejakú pomoc. Takisto sa môžete ozvať, ak ste ochotní s~niečím pomôcť.

Ak ste boli pozitívne testovaní na koronavírus a potrebujete pomoc so zabezpečením nákupov potravín či liekov, dajte nám vedieť.

V zbore sme zakúpili niekoľko kusov oximetrov na meranie saturácie kyslíku v~krvi. V~prípade potreby je ich možné zapožičať.
\vfill\break


\clanok{Verš na zapamätanie}
Tento mesiac máme nový veršík, ktorý sa chceme spoločne učiť. Veríme, že poznanie Písma prospeje našej duši i našej mysli:

{\it „Milosť a pokoj nech sa rozhojňuje medzi vami v~poznávaní Boha a Ježiša, nášho Pána. Jeho božská moc nám darovala všetko potrebné pre život a nábožnosť, keď sme poznali toho, ktorý nás povolal vlastnou slávou a účinnou mocou.“}

\autor{2Pt~1,~2~--~3}


\clanok{Zbierky za uplynulé obdobie}
Milí bratia a sestry,

vo februári ste prispeli:

\vskip-1ex\begitems
* Misia: 258,50 €
* Investície: 258,50 €

\enditems

Ďakujeme vám, že napriek okolnostiam a neistým ekonomickým vyhliadkam do budúcnosti, ste mnohí prispeli na činnosť a službu zboru. Aj naďalej máte možnosť prispieť do „nedeľnej zbierky“, a to prevodom na účet zboru. Do poznámky pre prijímateľa, prosím, uveďte „zbierka“.

Bankové spojenie: SK36 0900 0000 0000 1147 1836, SWIFT: GIBASKBX

Ďakujeme!

\n 3.	3.	Elena	BUZÁŠOVÁ;
\n 10.	3.	Rada	BÁNOVÁ;
\n 20.	3.	Jana	MÁŤUŠOVÁ;
\n 21.	3.	Ladislav	TALIGA;
\n 23.	3.	Ľudmila	VIDA;
\n 25.	3.	Pavol	ŠKULEC;
\n 25.	3.	Filip	KOVÁČ;
\n 26.	3.	Matej	KOLÁŘIK;
\n 28.	3.	Marta	BARKÓCI;
\n 29.	3.	Marcel	MAĎAR;
\n 29.	3.	Peter	PRIBULA ml.;
\n 30.	3.	Marta	GULDANOVÁ;
\n 31.	3.	Judit	KOBZOVÁ;
\narodeniny


\program{
\p  1 ; po ;.;;.;;
\p  2 ; ut ;.;;.;;
\p  3 ; st ;.;;.;;
\p  4 ; št ;.;;.;;
\p  5 ; pi ;.;;.;;
\p  6 ; so ;.;;.;;
\p  7 ; ne ; 10.30 ; Bohoslužby (D. Jones, online) ; 14.00 ; Veľká besiedka (FB Messenger) ;
\p  8 ; po ; 18.00 ; Modlitby za Slovensko (individuálne, telefonicky) ;.;;
\p  9 ; ut ;.;;.;;
\p 10 ; st ;.;;.;;
\p 11 ; št ;.;;.;;
\p 12 ; pi ; 17.30 ; Dorast (Zoom) ;.;;
\p 13 ; so ;.;;.;;
\p 14 ; ne ; 10.30 ; Bohoslužby (P. Kolárovský, online) ;.;;
\p 15 ; po ; 18.00 ; Modlitby za Slovensko (individuálne, telefonicky) ;.;;
\p 16 ; ut ;.;;.;;
\p 17 ; st ;.;;.;;
\p 18 ; št ;.;;.;;
\p 19 ; pi ; 17.30 ; Dorast (Zoom) ;.;;
\p 20 ; so ;.;;.;;
\p 21 ; ne ; 10.30 ; Bohoslužby (D. Jones, online) ; 14.00 ; Veľká besiedka (FB Messenger) ;
\p 22 ; po ; 18.00 ; Modlitby za Slovensko (individuálne, telefonicky) ;.;;
\p 23 ; ut ;.;;.;;
\p 24 ; st ;.;;.;;
\p 25 ; št ;.;;.;;
\p 26 ; pi ; 17.30 ; Dorast (Zoom) ;.;;
\p 27 ; so ;.;;.;;
\p 28 ; ne ; 10.30 ; Bohoslužby (J. Szőllős, online) ;.;;
\p 29 ; po ; 18.00 ; Modlitby za Slovensko (individuálne, telefonicky) ;.;;
\p 30 ; ut ;.;;.;;
\p 31 ; st ;.;;.;;
}


\tiraz
\bye
