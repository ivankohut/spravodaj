\def\velkostpisma{10}
\def\velkostriadku{12.5}
\input makra.tex % nacitanie Ivanom pripravenych nastaveni a prikazov
\hyphenation{star-šov-stvo} % rozdelenie slov na konci riadku, treba tu uviest slova, ktore sam nepozna

\spravodaj{6}{2025}


\clanok {Správy zo staršovstva}
Staršovstvo sa v~máji stretlo dvakrát, 13.~5. a 27.~5. Spoločne sme začali študovať knihu Starší zboru (Jeremie Rinne).

Prvé stretnutie bolo otvorené pre záujemcov o~rozhovor v~otázke budúceho využitia chaty J.~A.~Komenského v~Račkovej doline, túto možnosť využil jeden člen zboru. Pozícia nášho zboru v~tejto otázke však zostáva nezmenená, tak ako sme sa dohodli na ostatnom ZČZ, náš zbor sa v~hlasovaní o~ďalšom prenájme chaty zdrží hlasovania. S~vďačnosťou sme prijali pozitívne ohlasy na organizáciu sesterskej konferencie, ktoré Peter Pribula aj komunikoval zboru. Tešíme sa z~ochoty a lásky všetkých, ktorí sa zapojili a pomáhali pri tejto službe.

Na druhom stretnutí sme sa rozprávali o~pastierskej zodpovednosti starších zboru a o~možnostiach prehlbovania vzájomných vzťahov v~našom zbore. Pozitívne sme zhodnotili Noc kostolov, do ktorej bol náš zbor tento rok opäť zapojený, premýšľali sme o~tom ako ešte efektívnejšie využiť túto príležitosť v~budúcom roku. Okrem toho sme sa na našich stretnutiach venovali pokrytiu služieb a plánovaniu zborových akcií v~budúcom období.

Svoje podnety a návrhy týkajúce sa života a služby zboru, či práce staršovstva môžete dávať ústne alebo písomne členom staršovstva, alebo využiť „bielu skrinku“, alebo zaslať aj e-mailom na adresu \email{starsovstvo@bjbpalisady.sk}.

\autor {za staršovstvo R.~Nemec}


\clanok {Výročná správa za mládež za rok 2024}
V~roku 2024 sme viedli mládež v~zložení: Naomi Dzuriak, Martin Hovorka, Ksenia Ionova, Dávid Pribula a ako pomocný vedúci (viac kouč ako riaditeľ) nám pomáhal Peter Antalík. Toto rozšírené zloženie, nielen kvantitou ale aj kvalitou, nám umožnilo priniesť viac nápadov a tvorivosti do pravidelných stretnutí, ale aj vytvárať výnimočné stretnutia, alebo napríklad aj letný tábor, ktorý sme opäť pripravovali spolu s~vedúcimi nášho dorastu a tiež spoločne tak pre dorastencov ako aj mládežníkov.

Nemennými vecami tohto roku opäť ostávajú miesto a čas stretnutia. Čo sa však snažíme meniť a zlepšovať je určite náplň našich stretnutí, kde sa nám pravidelne darí mať čas chvál a uctievania Pána, témy ktoré sú zamerané na štúdium Písma, ale aj osobné svedectvá ľudí z~nášho zboru. S~týmto nám pomáhajú vedúci dorastu, ktorí sú vždy ochotní nám prakticky pomôcť s~témami alebo pri príprave stretnutí. Okrem nich sme však na mládeži počuli aj iných ľudí z~nášho zboru, ktorí nás obohatili už spomínaným výkladom Božieho slova alebo osobným svedectvom.

Sme vďační za to, že naša mládež rastie v~počte a naši mládežníci v~poznaní Písma a vo vzájomnej láske. Mládež sa stáva miestom, kde sa noví ľudia cítia vítaní a prijatí, kam sa radi vracajú pravidelne alebo príležitostne, keď navštívia Bratislavu. Sme vďační Pánu Bohu, že našu prácu završuje, našu snahu odmeňuje a naše prípadné nedostatky prikrýva svojou láskou a milosťou.

\autor {za prípravný tím D.~Pribula}


\clanok {Verš na mesiac}
V~júni sa budeme učiť verš, ktorý dostal dorast pre rok 2025: „Podľa toho sme poznali lásku, že položil On za nás dušu; aj my máme duše klásť za bratov.“ (1J~3,16)


\clanok {Stretnutie sestier}
Posledné stretnutie sestier v~prvom polroku bude v~stredu 4.~6. o~17.30~hod. na Zrínskeho 2. Povieme si, čo pre nás sestry znamená verš, ktorý sme si vytiahli na rok 2025. Krátke slovo si pre vás pripravili aj sestry z~tímu. Budeme mať aj priestor na zdieľanie sa o~tom, čo ste v~poslednom čase prežili alebo prežívate s~naším Pánom.


\clanok {Senior klub}
Seniori sa stretnú posledný štvrtok v~júni a to vo štvrtok 26.~6.~2025 od~10.00~hod. na Súľovskej. Téma bude: „Zoslanie Ducha Svätého“.


\clanok {Biela skrinka}
Máme dobrú správu, hospodársky výbor sa postaral o~obnovu a montáž našej starej známej „bielej skrinky“. Nachádza sa za pravým (zvnútra) krídlom dverí. Skrinku môžeme využívať napr. pre kontakt so staršovstvom, pre písanie rôznych návrhov, podnetov, anonymne darovať financie a pod. Veríme, že bude užitočná.
\vfill\break


\clanok {Zbierky v~máji}
\table{lr}{
Na misiu			&   257~€ \cr
Na investičný fond 	&   554~€ \cr
Mjanmarsko (18.~5.)	& 1 150~€ \cr}
\vskip1em

Aj naďalej máte možnosť prispieť do „nedeľnej zbierky“, a to prevodom na účet zboru. Do poznámky pre prijímateľa, prosím, uveďte „zbierka“.

Bankové spojenie: SK36 0900 0000 0000 1147 1836, SWIFT: GIBASKBX



\n  2.	6.	Miriam	KEŠJAROVÁ;
\n  7.	6.	Pavel	KOHÚT;
\n  9.	6.	Samuel	PLETT;
\n 10.	6.	Ján	LAURENČÍK;
\n 15.	6.	Ľubica	HOVORKOVÁ;
\n 15.	6.	Peter	LICHANEC;
\n 15.	6.	Trey	ATKINS;
\n 17.	6.	Juraj	KVAČKA;
\n 19.	6.	Anna	ŠANDOROVÁ;
\n 19.	6.	Oľga	VALCHÁŘOVÁ;
\n 22.	6.	Kristína	HORVÁTIKOVÁ;
\n 25.	6.	Peter	ŽEMBERY;
\n 25.	6.	Marica	ŠČEVLÍKOVÁ;
\n 27.	6.	Sylvia	PRIBULOVÁ;
\n 28.	6.	Jana	PERKNOVSKÁ;
\narodeniny


\program{
\p  1 ; ne ;  9.30 ; Bohoslužby (J. Szőllős + VP);.;;
\p  2 ; po ; 18.00 ; ;.;;
\p  3 ; ut ; 13.30 ; Biblická hodina pre seniorov (P. Pivka);.;;
\p  4 ; st ; 17.30 ; Stretnutie sestier;.;;
\p  5 ; št ; 18.00 ; Biblická hodina (J. Szőllős);.;;
\p  6 ; pi ; 17.30 ; Dorast;.;;
\p  7 ; so ; 18.00 ; Mládež;.;;
\p  8 ; ne ;  9.30 ; Bohoslužby (T. Valchář);.;;
\p  9 ; po ; 18.00 ; Skupinka Základy viery;.;;
\p 10 ; ut ; 13.30 ; Biblická hodina pre seniorov (P. Pivka);.;;
\p 11 ; st ;.;;.;;
\p 12 ; št ; 18.00 ; Biblická hodina (J. Szőllős);.;;
\p 13 ; pi ; 17.30 ; Dorast;.;;
\p 14 ; so ; 18.00 ; Mládež;.;;
\p 15 ; ne ;  9.30 ; Bohoslužby (P. Kolárovský);.;;
\p 16 ; po ;.;;.;;
\p 17 ; ut ; 13.30 ; Biblická hodina pre seniorov (P. Pivka);.;;
\p 18 ; st ;.;;.;;
\p 19 ; št ; 18.00 ; Biblická hodina (J. Szőllős);.;;
\p 20 ; pi ; 17.30 ; Dorast;.;;
\p 21 ; so ; 18.00 ; Mládež;.;;
\p 22 ; ne ;  9.30 ; Bohoslužby (S. Baláž);.;;
\p 23 ; po ; 18.00 ; Skupinka Základy viery;.;;
\p 24 ; ut ; 13.30 ; Biblická hodina pre seniorov (P. Pivka);.;;
\p 25 ; st ;.;;.;;
\p 26 ; št ; 18.00 ; Biblická hodina (J. Szőllős);.;;
\p 27 ; pi ; 17.30 ; Dorast;.;;
\p 28 ; so ; 18.00 ; Mládež;.;;
\p 29 ; ne ;  9.30 ; Bohoslužby (F. Barkóczi);.;;
\p 30 ; po ;.;;.;;
}

\tiraz
\bye
