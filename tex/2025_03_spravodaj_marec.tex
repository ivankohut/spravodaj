\def\velkostpisma{10}
\def\velkostriadku{12.5}
\input makra.tex % nacitanie Ivanom pripravenych nastaveni a prikazov
\hyphenation{star-šov-stvo} % rozdelenie slov na konci riadku, treba tu uviest slova, ktore sam nepozna

\spravodaj{3}{2025}


\clanok {Čo znamená „prebývať v~úkryte Najvyššieho“}

V~žalme 91 sa hovorí, že Boh je naše útočisko a naša pevnosť. Zároveň tam nachádzame toto úžasné zasľúbenie: „Kto býva v~úkryte Najvyššieho, odpočíva v~tôni Všemohúceho.“ (Ž~91,1) Čo to znamená „prebývať v~úkryte Najvyššieho a odpočívať v~tôni Všemohúceho“? Existuje podobné zasľúbenie pre kresťanov aj v~Novom zákone? A~má s~tým niečo spoločné každodenné čítanie Písma?

Áno, podobné zasľúbenie máme aj v~Novom zákone a čítanie Písma je určite jedným zo spôsobov, ako prebývať v~úkryte Najvyššieho. Ale aby sme pochopili skutočný význam žalmu 91, prečítajme si ho a potom sa pozrime na udalosť zo života umučeného misionára Jima Elliota, ktorého životopis, napísaný jeho manželkou, má názov {\em Úkryt Všemohúceho}.

\cast{Jeho bezpečný úkryt}

Toto spojenie pochádza zo žalmu 91, ktorý sa začína takto:

„Kto býva v úkryte Najvyššieho, odpočíva v tôni Všemohúceho, nech povie Hospodinovi: ‚Moje útočisko a moja pevnosť je môj Boh, v ktorého dúfam.‘“

A potom v~siedmom verši pokračuje týmito nádhernými slovami:

„Keby ti po boku padli tisíce a desaťtisíce po pravici, teba to nezasiahne.
Len čo otvoríš oči, uzrieš odplatu bezbožných. Ak máš útočisko v~Hospodinovi, u~Najvyššieho svoj príbytok, nič zlé sa ti nestane, nijaká pohroma sa k~tvojmu stanu nepriblíži.“

Z~toho vyplýva, že prebývanie v~skrýši Všemohúceho a v~útočisku Najvyššieho znamená, že ak po tebe niekto hodí oštep, nezasiahne ťa.

\cast{Pre zisk}

Bola teda Elisabeth Elliotová naivná a nebiblická, keď životopis svojho manžela nazvala Úkryt Všemohúceho, hoci ho spolu s~ďalšími štyrmi mužmi 8.~1.~1956 v~Ekvádore ubodali na smrť indiáni kmeňa Huaorani, ktorým sa snažili zvestovať evanjelium? Túto otázku jej často kládli. Teraz je s~Pánom, ale osobne som mal možnosť s~ňou mnohokrát hovoriť. Väčšina ľudí považuje jej dôveru v~Božiu zvrchovanosť za trochu prehnanú. Na konci knihy však autorka podáva odpoveď. Môžete si ju prečítať na posledných stranách spomínaného životopisu Jima Elliota:

„Človek nie je blázon, ak dá to, čo si nemôže nechať, aby získal to, čo nemôže stratiť.“

Čo tým myslel? Čo tým myslela Elisabeth Elliotová, keď ho citovala? Obaja vyjadrovali nasledujúcu skutočnosť: Ak Boh uzná za vhodné, aby šíp alebo oštep indiána kmeňa Huaorani zabil jedno z~jeho detí, potom to robí pre ich zisk. Slovami Jima Elliota, aby získali to, čo nemôžu stratiť. Boh to urobil pre zisk, nie na škodu. Myslím si, že Elliotovci mali pravdu. Som presvedčený, že toto je správny výklad žalmu 91.

Prečo si to myslím? Pretože v~evanjeliu podľa Matúša 4,6 vidíme, že satan sa pokúsil zneužiť žalm~91, aby prinútil Ježiša vyskočiť z~chrámu, pretože tento žalm sľubuje, že nás anjeli ponesú. Ježiš si ho však takto nevykladal. Ani Štefan, keď bol ukameňovaný na smrť. Ani Jakub, keď bol sťatý. Ani Pavol, keď ho opakovane bili palicou. Ani Ježiš, keď trpel na kríži. Nikto z~nich nechápal žalm~91 tak, že Božie deti nikdy nebudú trpieť z~rúk svojich nepriateľov.

\cast{Všetko, čo potrebujeme}

Avšak satan sa ich snažil presvedčiť, aby to tak chápali. Ale čo potom znamená tento žalm? Čo vlastne znamená prebývať pod ochranou Všemohúceho, keď vás tam môžu zabiť? Pozrime sa do Nového zákona, kde nájdeme podobné texty. Niekoľko. Napríklad:
\begitems
* Júda v~21. verši hovorí: „Udržujte sa v~Božej láske a očakávajte milosrdenstvo nášho Pána Ježiša Krista pre večný život.“ Myslím si, že je to to isté, ako keď sa hovorí: „Zostaňte v~prístrešku Najvyššieho.“
* Ježiš v~Jánovi 15,9 hovorí: „Zostaňte v~mojej láske“, čo by sa dalo tiež preložiť ako: „Zostaňte v~prístrešku Najvyššieho“.
\enditems

Inými slovami, prebývať v~prístrešku Všemohúceho alebo Najvyššieho znamená bezvýhradne dôverovať, že Božia láska a moc vám dajú všetko, čo potrebujete, aby ste mohli plniť jeho vôľu a oslavovať jeho meno, či už v~živote alebo v~smrti. Alebo inak povedané: prebývať v~prístrešku Najvyššieho a zdržiavať sa v~Božej láske znamená veriť, že Božia láska, múdrosť a moc vás ochránia pred všetkým, čo by vás mohlo úplne zničiť.

\cast{Nikdy nie porazení}

Prečo si to myslím? Jeden z~najjasnejších dôvodov sa nachádza v~liste Rimanom~8,~32~--~39, čo je azda najúžasnejší odsek v~celej Biblii. Pavol tvrdí, že Božia láska k~jeho vyvoleným, adoptovaným deťom, ktorú dokazuje smrť jeho Syna Ježiša, „nám určite dá všetko“.

„Ako by nám ten, ktorý neušetril vlastného Syna, ale vydal ho za nás všetkých, nedaroval s~ním všetko?“ (R 8,32)

Áno, daroval. Čo však znamená {\em všetko}? Pavol to vysvetľuje ďalej a dokonca na to používa žalmy. Ukazuje, že ak zotrvávame v~Kristovej láske, v~prístrešku Všemohúceho, nič nás nemôže odlúčiť od Božej lásky v~Kristovi. Potom vymenúva niekoľko možností, čo by nás od nej mohlo oddeliť, čím ukazuje, že dobre pozná žalm~91. Píše: „Súženie alebo nešťastie, prenasledovanie alebo hlad, nahota, nebezpečenstvo alebo meč?“ A~mohli by sme dodať: „Alebo oštep Indiánov kmeňa Huaorani?“

Potom cituje Ž 44,23: „Veď kvôli tebe nás usmrcujú deň čo deň, pokladajú nás za ovce na zabitie“. Teda nie pre náš hriech, ale kvôli Kristovi. Už žalmisti vedeli, že Boží ľud zomiera, keď koná dobro. A~potom Pavol odpovedá: „Toto všetko však víťazne prekonávame skrze toho, ktorý si nás zamiloval.“ (R 8,36-37)

Pavol teda hovorí, že kresťania môžu zotrvávať v~Božej láske a v~prístrešku Všemohúceho, a napriek tomu môžu byť vydaní na porážku ako ovce. Napriek tomu v~tom všetkom dokonale víťazia. Ak teda vo dne priletí šíp, zasiahne vás priamo do hrude a vy zomriete pre Krista, nie ste porazení. Ste dokonalými víťazmi.

\cast{Vstúpte do večnosti}

V~akom zmysle ste dokonalými víťazmi? Práve ten šíp, ktorý nad vami zdanlivo zvíťazil, sa stáva vaším služobníkom, ktorý uskutočňuje zvrchovaný Boží zámer vo svete. A~Božím plánom spásy pre vás je večný život. Pozrite sa do knihy Zjavenia: „Ale oni nad ním zvíťazili pre Baránkovu krv a pre slovo svojho svedectva a nemilovali svoj život tak, aby sa zľakli smrti.“ (Zj 12,11)

Ako sa dá prebývať v~prístrešku Najvyššieho? Tak, že bezvýhradne verili, že Baránkova krv im zabezpečuje budúce večné víťazstvo. Otvorili svoje ústa a vydávali svedectvo. Nezastavil ich ani strach zo smrti. V~tej chvíli boli v~bezpečí v~prístrešku Všemohúceho. Zvíťazili nad diablom a vstúpili do nebeskej vlasti. Verím, že práve k~takémuto víťaznému pocitu bezpečia nás Boh vyzýva v~žalme~91.

\autor {podľa J.~Pipera, \ulink[https://www.radiologos.sk/otazky-a-odpovede/sermon/307-prebyvanie-v-ukryte-najvyssieho.html]{radiologos.sk}}


\clanok {Správy zo staršovstva}

Staršovstvo zboru sa vo februári stretlo v~plánovaných pravidelných termínoch 4.~2. a 14.~2. Na oboch stretnutiach bolo dominantnou témou hospodárenie zboru za rok 2024 a návrh rozpočtu na rok 2025.

Na prvom stretnutí bratia starší spolu s~ekonómkou zboru s.~Ľ.~Kohútovou zhodnotili hospodárenie zboru a dôsledky z~toho plynúce a podrobne prebrali ňou pripravený návrh rozpočtu. Pri dlhšej diskusii navrhli zmeny v~rozpočte a takto upravený rozpočet rozhodli zaslať členom zboru.

Na druhom stretnutí bola daná príležitosť členom zboru, aby osobne prišli na staršovstvo a kládli otázky k~hospodáreniu a rozpočtu a navrhovali zmeny. Na stretnutie prišla jedna členka zboru a tiež brat V. Potockij, s~ktorým boli prediskutované návrhy v~rozpočte, týkajúce sa cirkevného zboru Nádej, ktorý pôsobí v~našich priestoroch. Diskusia sa okrem konkrétnych položiek týkala aj zmeny celkovej štruktúry rozpočtu, ktorá by sa mohla udiať podľa rozhodnutia staršovstva v~rozpočte na rok 2026 a tiež otázky ako realizovať finančnú nezávislosť zboru od štátnej podpory a vyjadriť toto naše úsilie a smerovanie v~rozpočte. Na oboch stretnutiach sa pripravovalo aj výročné zborové členské zhromaždenie spojené s~voľbami staršovstva, starších v~zácviku, voľbami revíznej komisie a voľbami kazateľského asistenta, ktoré je plánované na nedeľu 9.~3.~2025 popoludní.

Na prvom stretnutí bolo na programe aj zhodnotenie ZČZ z~26.~1., na ktorom prebehol aj prieskum na voľby do revíznej komisie a implementácia jeho rozhodnutí. Bratia starší rozhodli aj o~zapojení zboru do pripravovaných podujatí Kresťanov v~meste, o~prenájme našej modlitebne na sobáše, termínoch našich zborových akcií, zreálnení počtu členov zboru a o~ďalších otázkach zborového života. Na druhom stretnutí bolo súčasťou bodu „rôzne“ aj overenie termínu zborovej víkendovky na jeseň.

Svoje podnety a návrhy týkajúce sa života a služby zboru, či práce staršovstva môžete dávať ústne alebo písomne členom staršovstva, alebo zaslať aj e-mailom na adresu \email{starsovstvo@bjbpalisady.sk}.

\autor {za staršovstvo J. Szőllős}


\clanok {Verš na mesiac}
„Kto býva v~úkryte Najvyššieho, odpočíva v~tôni Všemohúceho, nech povie Hospodinovi: ‚Moje útočisko a moja pevnosť je môj Boh, v~ktorého dúfam.‘“ (Ž~91,1-2)


\clanok {Stretnutie sestier}
Sestry sa stretnú v~stredu 5.~3.~2025 o~17.30~hod. na Zrínskeho 2. Hosťom bude s.~Dáša Leeder, téma je: „Milosť a pomoc v~pravý čas“.


\clanok {Výročné zborové členské zhromaždenie}
V~nedeľu 9.~3.~2025 o~16.30~hod. sa uskutoční výročné zborové členské zhromaždenie v~modlitebni na Palisádach.
\vfill\break


\clanok {Zbierky vo februári}
\table{lr}{
Všeobecne			& 1 708,00~€ \cr
Na misiu			&   321,00~€ \cr
Na investičný fond 	&   438,00~€ \cr}
\vskip1em

Aj naďalej máte možnosť prispieť do „nedeľnej zbierky“, a to prevodom na účet zboru. Do poznámky pre prijímateľa, prosím, uveďte „zbierka“.

Bankové spojenie: SK36 0900 0000 0000 1147 1836, SWIFT: GIBASKBX


\n 12.	3.	Alžbeta	SMOLKOVÁ;
\n 17.	3.	Tamara	SYČOVÁ;
\n 20.	3.	Jana	MÁŤUŠOVÁ;
\n 25.	3.	Filip	KOVÁČ;
\n 26.	3.	Matej	KOLÁŘIK;
\n 27.	3.	Marta	MAJEROVÁ;
\n 28.	3.	Marta	BARKÓCI;
\n 29.	3.	Marcel	MAĎAR;
\n 30.	3.	Marta	GULDANOVÁ;
\n 31.	3.	Judit	KOBZOVÁ;


\narodeniny


\program{
\p  1 ; so ; 18.00 ; Mládež;.;;
\p  2 ; ne ;  9.30 ; Bohoslužby (J. Szőllős + VP);.;;
\p  3 ; po ; 18.00 ; Skupinka „Základy viery“;.;;
\p  4 ; ut ; 13.30 ; Biblická hodina pre seniorov (P. Pivka);.;;
\p  5 ; st ; 17.30 ; Stretnutie sestier (D. Leeder);.;;
\p  6 ; št ; 18.00 ; Biblická hodina (J. Szőllős);.;;
\p  7 ; pi ; 17.30 ; Dorast;.;;
\p  8 ; so ; 18.00 ; Mládež;.;;
\p  9 ; ne ;  9.30 ; Bohoslužby (R. Nemec); 16.30 ; VZČZ;
\p 10 ; po ;.;;.;;
\p 11 ; ut ; 13.30 ; Biblická hodina pre seniorov (P. Pivka);.;;
\p 12 ; st ;.;;.;;
\p 13 ; št ; 18.00 ; Biblická hodina (J. Szőllős);.;;
\p 14 ; pi ; 17.30 ; Dorast;.;;
\p 15 ; so ; 18.00 ; Mládež;.;;
\p 16 ; ne ;  9.30 ; Bohoslužby (M. Mišinec);.;;
\p 17 ; po ;.;;.;;
\p 18 ; ut ; 13.30 ; Biblická hodina pre seniorov (P. Pivka);.;;
\p 19 ; st ;.;;.;;
\p 20 ; št ; 18.00 ; Biblická hodina (J. Szőllős);.;;
\p 21 ; pi ; 17.30 ; Dorast;.;;
\p 22 ; so ; 18.00 ; Mládež;.;;
\p 23 ; ne ;  9.30 ; Bohoslužby (V. Ira);.;;
\p 24 ; po ; 18.00 ; Skupinka „Základy viery“;.;;
\p 25 ; ut ; 13.30 ; Biblická hodina pre seniorov (P. Pivka);.;;
\p 26 ; st ;.;;.;;
\p 27 ; št ; 18.00 ; Biblická hodina (J. Szőllős);.;;
\p 28 ; pi ; 17.30 ; Dorast;.;;
\p 29 ; so ; 18.00 ; Mládež;.;;
\p 30 ; ne ;  9.30 ; Bohoslužby (T. Hanes);.;;
}

\tiraz
\bye
