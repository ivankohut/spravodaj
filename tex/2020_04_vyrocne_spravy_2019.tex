%\typosize[9/12]% - pouzita velkost pisma/riadku - standard
\input makra.tex % nacitanie Ivanom pripravenych nastaveni a prikazov
\hyphenation{star-šov-stvo} % rozdelenie slov na konci riadku, treba tu uviest slova, ktore sam nepozna

\vyrocnespravy{2019}

\clanok{Zbor}
Sedím dnes už týždne v~izolácii a teším sa, že píšem o~minulom roku. Výročná správa 2020 bude zaujímavá. Kým rozmýšľam nad vecami z~minulého roku, zisťujem, že mi chýbate ešte viac, než som si to uvedomoval. Som Bohu vďačný za ďalší rok služby a požehnania medzi vami.

Počas minulého roku bolo pre mňa niekoľko vecí veľmi veľkým požehnaním. {\bi Služba žien} a pravidelné stretávanie sa našich sestier ma veľmi povzbudilo. Počet žien hľadajúcich spoločenstvo a duchovný pokrm priniesol do nášho zboru požehnanie na rôznych frontoch. Verím, že kvôli tomu sú aj manželstvá a rodiny v~inom stave. Som vďačný Clare za jej vernosť a množstvo sily, čo do toho dáva.  Vďaka Bohu za {\bi seniorov} v~zbore. Sú pre nás skutočne zdrojom moci a povzbudenia a ich službou sme všetci obohatení. Sledujem cez stenu kancelárie pravidelné stretnutia seniorov s~Paľkom Pivkom a mám z~toho radosť, keď počujem ich spev aj smiech. Niektorí z~nich slúžia aj v~{\bi diakonii} a zo skupiny diakonov som povzbudený. Tento rok sme do tej služby prijali viac diakonov a za tých nových sme vďační, aj za ich ochotu slúžiť. Táto skupina veľa robí v~anonymite a stará sa o~mnohých v~zbore. Náš zbor by bol omnoho menej efektívny, keby sme týchto vzácnych ľudí nemali. Nebudem  vymenovávať všetko, čo robia, lebo je toho veľa. Ale iba to, že som povzbudený v~nedeľu, keď ich vidím, ako vítajú aj nás i nových návštevníkov a ako robia všetko preto, aby sme sa cítili prijatí a milovaní. Časť tejto diakonskej služby je {\bi hospodársky výbor} a počas roka som sledoval, ako títo bratia robia všetko možné, aby budovy, technika, aj iné praktické oblasti zboru zostali v~správnom a zdravom stave. Je to funkčný tím bratov, ktorí tú službu berú zodpovedne. Nezabudnime na nich a buďme za nich vďační.

{\bi Slávovi Kráľovi} chcem poďakovať za každotýždennú nedeľnú vernú službu, či pri organe alebo vo vedení {\bi spevokolu}. Aj v~čase Vianoc a Veľkej noci sme so spevokolom  poslúžili viacerým ľuďom z~okolia, a tak mali možnosť počuť evanjelium aj slovom, aj spevom. Sme požehnaní mimoriadnym spevokolom a každému som vďačný za jeho účasť aj spev. Zvlášť {\bi hudobníkom} ďakujem, že do toho dávajú ešte viac hodín prípravy. Vďaka Bohu za {\bi chválospevové skupiny}, ktoré verne slúžia. Tento rok sme privítali nových členov do tejto služby a slúžia Bohu aj zboru požehnanou hudbou a spevom. Ďakujem vám. Je tu skupina služobníkov, ktorú na pódiu nevidíme, a za ktorú som takisto vďačný. Títo majú na starosti {\bi zvuk}, {\bi projekciu} a {\bi premietanie} zhromaždenia na webe. „Zhromko“ by bez nich nešlo. Ďakujem vám, bratia. Rola {\bi moderátora} u~nás je dôležitá služba a tento rok sme do tejto služby  privítali pár nových bratov. Najviac sa v~nedeľu teším na {\bi vás všetkých}. Počet návštevníkov narastá a pomaly si ani nemám veľmi kde sadnúť. Ďakujem vám všetkým za vašu vernosť a modlitby za zbor Palisády. Boh u~nás koná. Je to odpoveď na vaše {\bi modlitby}. Vidím, že je náš zbor stále zdravší, a to aj podľa spevu, ktorý napĺňa našu modlitebňu každý týždeň. Spievajúci zbor je zdravý zbor.  Pre mňa by bolo ťažké zvládnuť všetko, čo treba,  bez {\bi kazateľského tímu}, ktorý nám verne slúži  Božím Slovom. Srdečne ďakujem Peťovi Kolárovskému, Tomášovi Valchářovi, Ľubošovi Dzuriakovi a Jankovi Szőllősovi. Verím, že nás Boh cez nich bohato požehnáva.

V rámci služby deťom a dorastu sme tento rok urobili  niekoľko zmien. Som vďačný za učiteľov {\bi besiedky}, za ich vernosť a ochotu vyskúšať zmeny. Na krátku dobu sme skúšali nové priestory, ale rýchlo sme sa vrátili na Zrínskeho. Tieto ochotné služobníčky majú kľúčovú rolu v~životoch detí v~zbore. Ich služba nie je samozrejmá. Ďakujem im za obetavosť a kreativitu. Celú službu {\bi dorastu} sme tento rok zmenili.  Do služby sa pridalo niekoľko nových pracovníkov a používame materiály PreTeens, čo vypracovala misijná organizácia YWAM. Verím, že je to dobrá zmena a prinesie ovocie. Som vďačný za tých, ktorí vedú  {\bi mládež}. Teším sa osobne z~toho, že raz za mesiac sa stretávajú u~nás a my s~Clarou ich môžeme spoznávať a povzbudiť ich. Mládež zorganizovala raz na Súľovskej požehnaný večer modlitieb a chvál, z~čoho sme boli veľmi povzbudení. {\bi Letný zborový tábor} v~júli bol zatiaľ najväčší a mnohí z~nás strávili ten týždeň v~Častej požehnané dni spoločenstva a radosti. Služba Riša Nagypala bola pre našich mládežníkov aj pre nás všetkých veľkým požehnaním.

Boh nám počas roku pridal do našich rodín nové {\bi bábätká} a tešíme sa z~nich a znovu si uvedomujeme, že sme všetci v~zbore zodpovední za to, aby sme pre nich boli vzorom aj povzbudením. Takisto sa dalo {\bi pokrstiť} sedem vzácnych bratov a sestier. S~nimi sa tešíme z~tohto kroku poslušnosti. Sme radi, že aj noví členovia zboru si našli svoje miesto medzi nami.

Pre mňa bolo veľkým povzbudením, keď sa toľko bratov a sestier zo zboru zapojilo a slúžilo v~akcii {\bi Milujem svoje mesto} na základnej škole Milana Hodžu. Ukázali sme im skutočnú Božiu lásku a svedčili o~tom, že smerujeme k~tomu, aby sme boli najmilším zborom v~meste. Je to veľká vec!

Za všetkých, ktorí slúžia na Chvojnici, som vďačný. Spolu s~bratom Paľkom Škulecom a sestrou Blaženkou tam mnohí z~vás z~Bratislavy pravidelne cestujú a slúžia. Vaša obetavosť a láska nie je márna a prinášate požehnanie do tejto oblasti. S~radosťou, ako je už zvykom, bola slúžiť na Chvojnici veľká skupina veriacich z~Palisád na sviatok zoslania Svätého Ducha. Spoločenstvo s~bratmi a sestrami, či z~Chvojnice alebo z~Bratislavy, bolo vzácne.

Urobili sme kľúčový krok poslušnosti a dôvery, keď sme sa stali materským zborom pri zakladaní misijnej stanice {\bi Connect}. Verím, že ako zdravé rodiny, aj zdravé zbory sa rozmnožujú. Je to prirodzené. S~veľkým povzbudením sme žehnali zakladajúcemu tímu a stojíme pri nich a podporujeme ich. Ďakujem vám za tú vieru a podporu, čo neberiem ako samozrejmosť. Takisto s~radosťou pozorujeme rast našej {\bi ukrajinskej služby}. Každý týždeň v~roku 2019 sme boli povzbudení účasťou týchto bratov a sestier na nedeľných bohoslužbách aj tým, ako sa do života zboru zapájajú. Stretávajú sa v~nedeľu na Zrínskeho o~11.30~hod. aj v~stredu o~19.30~hod. Pridáva sa stále viac ľudí a buduje sa jadro vedenia, čo je prvým krokom k~založeniu zboru. Máme z~nich veľa radosti.

Počas minulého roka svojou láskou a múdrosťou, čomu niekedy nerozumieme, náš dobrý Boh k~sebe povolal sestru Lynn Plett a brata Radka Hovorku. Ešte stále prežívam tú veľkú stratu, ale aj Božiu vernosť v~takýchto ťažkých situáciách. Sme vďační za roky ich vernosti a služby. Ďakujeme všetkým, ktorí týmto drahým rodinám rôznym spôsobom poslúžili.

Na záver vám úprimne poviem, že zboru na Palisádach slúži aj skupina vzácnych bratov na {\bi staršovstve}. Robil som počas rokov s~rôznymi skupinkami starších a som z~tejto skupiny veľmi povzbudený. Bratia sú ochotní aj ustúpiť a druhému dôverovať. Boh nám stále dáva múdrosť, vedenie aj jednotu, aj v~niektorých ťažkých situáciách. Do staršovstva sa zapojili Janko Szőllős a Daniel Plett. Ich prácou sme obohatení a sme za nich vďační.

Ak dovolíte, ešte raz chcem vyjadriť celým srdcom svoju vďaku za vás všetkých.  V~roku 2019 sme v~našich rodinách čelili mnohým ťažkým veciam, ale Boh nám stále dáva nádej, a za to sme vďační. Je mi cťou vám slúžiť a verím, že nás čaká veľa dobrého. Buďme citliví na Jeho vedenie a naučme sa na Jeho hlas okamžite s~poslušnosťou reagovať. Ak to robíme všetci, splníme to, čo je pre nás prichystané. Na to sa teším. {\it „Som presvedčený, že ten, čo vo vás začal dobré dielo, ho aj dokončí, až do dňa Krista Ježiša“ {\em (Fil 1,6)}.}

\autor{Danny Jones, kazateľ zboru}


\clanok{Staršovstvo}
Staršovstvo dostalo na rok 2019 verš z~listu Kolosenským 1,10: {\it „… aby ste žili hodní Pána a páčili sa mu vždy, keď budete prinášať ovocie v~každom dobrom skutku a rásť v~poznaní Boha.“} Toto slovo bolo v~roku 2019 pre nás výzvou, aby sme dávali pozor na to, ako my sami žijeme. Náš život sa nemá páčiť nám, ale nášmu Pánovi. Zároveň má náš život prinášať ovocie, ktoré je výsledkom dobrých skutkov, ktoré v~nás pôsobí Svätý Duch, a je aj výsledkom poznávania nášho nebeského Otca.

Staršovstvo zboru pracovalo v~roku 2019 v~niekoľkých zloženiach. Do volieb v~marci 2019 to boli kazateľ zboru Danny Jones a členovia Peter Antalík, Radovan Hovorka, Vladimír Ira, Pavel Kohút, Peter Kolárovský, Miroslav Kolářik, Peter Pribula. Po voľbách Ján Szőllős nahradil Pavla Kohúta. V~decembri bol na uvoľnené miesto po Radkovi Hovorkovi zvolený Daniel Plett.

Začiatok roka bol naplnený prípravou volieb. Volili sme si na ďalšie obdobie revíznu komisiu a staršovstvo. Hľadali sme aj ďalších pracovníkov medzi diakonov.

V priebehu roka sme sa venovali nasledujúcim témam:
\begitems
* rozhovory so záujemcami o~členstvo v~zbore;
* príprava a realizácia troch zborových členských zhromaždení;
* výročné zborové členské zhromaždenie, ktorého súčasťou boli voľby, hospodárenie v~roku 2018 a rozpočet na ďalší rok;
* zabezpečenie služieb v~zbore aj na Chvojnici;
* program Národného týždňa manželstva;
* stretnutia so záujemcami o~krst;
* príprava sederovej večere;
* pastoračné otázky;
* návšteva spevokolu v~zbore vo Viedni;
* práca v zakladajúcom sa zbore Connect;
* práca medzi ukrajinsky hovoriacimi návštevníkmi zboru;
* príprava krstu, vďakyvzdania, návštevy mimo nášho zboru.
\enditems

Sme vďační nášmu nebeskému Otcovi za Jeho vedenie a požehnanie pri všetkom, čomu sme sa počas minulého roku venovali. Zároveň sa tešíme na všetko, čo má pre nás pripravené v~tomto roku.

\autor{Peter Pribula st.}


\clanok{Diakonia}
Verš na rok 2019: {\it „Hospodin moja skala, hrad môj a môj vysloboditeľ, môj Boh je moje bralo, v~Neho dúfam, môj štít, roh mojej spásy, moja pevnosť.“}

\autor {Žalm 18,3}

\cast{Pracovné stretnutia}
Tím diakonov sa v~roku 2019 pravidelne stretával na pracovných stretnutiach raz mesačne na Zrínskeho ulici v~zborových priestoroch. (Zápisnice z~pracovných stretnutí boli pravidelne zaslané všetkým členom e-mailom, prípadne osobne odovzdané). Pozvánky na pracovné stretnutia pre členov tímu diakonov boli zasielané e-mailom spravidla tri dni vopred.

\def\aktivita#1{{\it #1\par}\firstnoindent}
\cast{I. Vnútrozborové aktivity}

\begitems \style n
* \aktivita{Návštevná služba}
Pravidelne pokračovala návšteva našich imobilných členov v~domácnostiach, ktorú vykonávajú jednotlivé sestry a bratia, ktorí majú s~navštevovanými bratmi a sestrami vytvorený niekoľkoročný blízky vzťah. Chcel by som aj menovite spomenúť aspoň niektorých členov tímu diakonov, ktorí pravidelne navštevovali našich imobilných seniorov, sú to sestry Lenka Gubová, Vladka Laurenčíková, Juditka Kolářiková, Gitka Kráľová a manželia Valentovci. Tiež bola vyslúžená aj pamiatka Večere Pánovej v~domácnostiach na požiadanie bratmi zo skupiny diakonov. Okrem našich seniorov boli navštevovaní aj naši chorí bratia a sestry, či už v~domácnostiach alebo v~nemocniciach. Sestry navštevovali aj mladé mamičky s~bábätkami z~nášho zboru.

* \aktivita{Dopravná služba}
Okrem služby návštevnej pokračovala aj služba dopravná málo mobilných bratov a sestier do nedeľného zhromaždenia. Verne slúžia bratia Ladislav Taliga a Vladimír Krajčí. Taktiež máme naďalej prenajaté parkovacie miesta v~čase nedeľného zhromaždenia v~areáli evanjelickej nemocnice na Partizánskej ulici.

* \aktivita{Zborové pohostenie}
Obed pre bratov a sestry z~Ukrajiny sa v~zbore uskutočnil v~nedeľu 24.~novembra~2019.

V nedeľu 1.~12.~2019 sme mali už tradičný „vianočný obed pre seniorov,“ kde im boli odovzdané aj vianočné darčeky.

* \aktivita{Svätodušné sviatky na Chvojnici}
Aj v~roku 2019 sme  na Letnice (9.~júna)  boli poslúžiť bratom a sestrám na Chvojnici. Bol to pre mnohých z~nás celodenný zborový výlet, na ktorý nás odviezol už tradične „Barnibus.“

* \aktivita{Vysluhovanie Večere Pánovej}
Večera Pánova sa vysluhovala pravidelne každú prvú nedeľu v~mesiaci (podľa rozpisu). Okrem toho sa Večera Pánova vysluhovala aj v~domácnostiach. V~tomto roku sa rozšírila aj skupina bratov vysluhujúcich Večeru Pánovu, z~čoho máme veľkú radosť, lebo sme to veľmi potrebovali.
\enditems

\cast{II. Aktivity zboru smerom von}

\begitems \style n
* \aktivita{Služba v~domovoch sociálnej starostlivosti}
Okrem služby v~našom zbore sa venujeme aj službe mimo zboru v~domovoch dôchodcov pod vedením brata P. Pivku za vernej pomoci sestier Lenky Gubovej a Vladky Laurenčíkovej. V~prípade potreby brata Pivku zastúpil „v Betánii“ brat kazateľ D.~ Jones. Pravidelne navštevujeme „Domovy sociálnej starostlivosti“ v~Starom meste a v~Dúbravke.

* \aktivita{Biblické vyučovanie}
V rámci služby seniorom máme aj duchovnú časť služby v~podobe pravidelných biblických hodín -- „Popoludnie pri Biblii“, kde postupne preberáme biblické knihy;  v~tomto roku to boli epištoly apoštola Pavla. Našich stretnutí sa zúčastňuje cca 10~--~12 bratov a sestier aj z~iných spoločenstiev (KZ Rača, Kresťanské spoločenstvo Christiana).

* \aktivita{Služba bezdomovcom}
Aj tento rok sme podporovali službu varenia pre bezdomovcov v~rámci združenia {\it Kresťania v~meste}.
\enditems

\autor{Pavel Pivka}


\clanok{Hospodársky výbor}
Veršom z~Prísl. 3,25-26 sme boli vedení k~vyproseniu si požehnania od nášho Pána pre prácu na Božej vinici v~roku 2019. Na našom prvom stretnutí sme rekapitulovali rok 2018 a diskutovali sme o~potrebách nášho zboru pre rok 2019. Práce, ktoré z~rôznych dôvodov neboli vykonané, boli prenesené do nastávajúceho obdobia (viď správa za rok 2018).

V apríli bol vyspravený a vymaľovaný celý objekt zborovej chalupy na Chvojnici, vymenený druhý bojler a príprava na prevádzku pre rok 2019. Zvažovali sme pristúpiť k~vyriešeniu zdroja vody navŕtaním novej studne (2x c.p). Pre budúcnosť prevádzkovania chalupy je to nutné realizovať. Pred prázdninami boli vykosené pozemky v~okolí kostolíka i chalupy, vykonané elektrické a protipožiarne revízie. Našim mládežníkom za prípravu stodoly a kostolíka (zaplaveného bahnom a vodou po búrke) pri príležitosti svätodušných sviatkov ďakujeme. Vďaka patrí nášmu Pánovi i za túto možnosť, ktorú máme na duchovnú ale aj telesnú regeneráciu našich životov v~Jeho jedinečnom stvoriteľskom diele.

V kostole na Palisádach a Zrínskeho boli taktiež urobené už spomínané revízie. Prácu, ktorú vykonáva tím pracovníkov (ozvučenie kostola, projekciu, živý prenos bohoslužieb) si nesmierne vážime. Ďakujeme aj všetkým účastníkom akcie {\it Milujem svoje mesto}.

Sme vďační rodine Plettovcov, ktorá sa podujala finančne zabezpečiť generálnu opravu klavíra v~modlitebni na Palisádach. Generálnu opravu vykonal klavírnik Pavol Tima.

Pán Boh nás obdaroval každého jedného darmi a očakáva, že ich budeme využívať na oslavu Jeho svätého mena a obohatenie nás všetkých. Prikročili sme spolu s~rodinami (s.~Liptáková, rod.~M.~Maďara) a BJB k~ideovému rozdeleniu priestorov domu na Zrínskeho. Konanie je momentálne v~procese dohody.

Ďakujeme aj za službu zabezpečenia fotodokumentácie pre súčasnú dobu, ale aj pre historické zhodnotenie zborového života.

Sme vďační Pánovi za povolanie, požehnanie, zdravie a silu k~práci na Jeho diele na tejto zemi. Nech je oslávený v~našich životoch.

\autor{Daniel Mikletič}


\clanok{Biblické a iné vzdelávanie}
Spoločné štúdium Svätého Písma prebiehalo vo viac--menej pravidelnom rytme a nezmeneným spôsobom. Takmer každý utorok popoludní preberal kazateľ Pavel Pivka so skupinou záujemcov, najmä z~radov seniorov, texty z~Božieho Slova. Správu o~náplni podal br. Pivka osobitne (viď správu za diakoniu).

Ja som viedol štvrtkové biblické hodiny, ktorých sa zúčastňovalo 10~--~15 bratov a sestier. Od začiatku roka sme pokračovali v~preberaní knihy Ezdráš a po nej sme preberali knihu Nehemiáš až do prázdnin. Od októbra sme potom začali preberať Apoštolské vyznanie viery. V~advente sme preberali aktuálne adventné a vianočné témy. Okrem uvedených celozborových príležitostí bolo zamyslenie nad Písmom a diskusia aj súčasťou stretávania viacerých skupiniek.

Napriek existencii širokej škály ponúk rôznych typov vzdelávania v~rámci našej cirkvi aj na naddenominačnej úrovni a napriek ponúkanej podpore zo strany zboru, sú tieto možnosti málo využívané našimi členmi, ale nemám o~tom podrobný prehľad. Na medzidenominačnej úrovni sa tím pracujúci s~mládežou zúčastnil na Konferencii pre pracovníkov s~mládežou (KPM) v~Žiline. Do kurzov Detskej misie sa mohli zapojiť niektorí jednotlivci spomedzi vedúcich besiedky.

\autor{Ján Szőllős}


\clanok{Sestry}
Keď som prijala vedenie služby sestier v~januári 2019, vyjadrila som, že som skutočne nemala víziu viesť službu sestier, ale bola som skôr inšpirovaná viesť službu sestrám. Preto som vedenie všetkej praktickej služby, ktorú vykonávajú sestry v~zbore (služba bezdomovcom, besiedka, dorast, chvály, moderovanie v~zhromaždení, obedy pre prichádzajúcich a seniorov atď.) odovzdala diakonii a staršovstvu. Dávalo mi zmysel, aby činnosť a služba kohokoľvek v~zbore, či je to muž alebo žena, boli pod dohľadom diakonie. Zatiaľ čo sestry slúžia všetkým, mojou snahou je slúžiť sestrám.

Vedela som, že som nemohla a ani nechcela viesť túto službu sestrám sama, a preto som oslovila tieto sestry s~prosbou o~pomoc pri vedení: Gitka Kráľová, Jarka Cihová, Mirka Hovorková, Barbi Antalíková, Angie Vráblová a Radka Bánová. Súhlasili, že mi pomôžu. Mali sme na srdci poskytnúť sestrám príležitosti, ktoré by povzbudili ich duchovný rast so zameraním na tieto oblasti:
\begitems
* študovať Božie Slovo spolu;
* učiť sa, čo znamená nájsť odpočinok v~Kristovi;
* posilňovať svoje manželstvá;
* stretávať sa so sestrami všetkých generácií, aby sme sa mohli učiť jedna od druhej.
\enditems

Od januára do júna sme sa stretávali každý druhý týždeň v~stredu spolu ako sestry nielen z~nášho zboru, ale aj zo zboru Viera. Do konca júna sme sa venovali biblickému štúdiu s~názvom {\it Milovaná: ako nájsť odpočinok}. Stretnutí sa zúčastňovalo spolu 69~žien všetkých vekových skupín. Na naše stretnutia ako aj štúdium som dostala pozitívnu spätnú väzbu. Sestry sa tešili, že sme mohli spolu študovať Božie Slovo a v~skupinkách mať diskusie o~tom, čo sme sa doma pri Slove naučili. Bola som povzbudená tým, že mnoho sestier vyjadrilo, že štúdium im pomohlo nájsť odpočinok v~Kristovom dokonalom diele a duchovne rásť.

S cieľom posilniť manželstvá v~našom zbore sme sestrám ponúkali posedenia so sestrou Ľubkou Hovorkovou, pri ktorých sa s~nami zdieľala zo svojej múdrosti 62 rokov manželstva. Mali sme s~ňou tri stretnutia a zúčastnilo sa ich okolo 25~sestier. Veľa sme získali z~jej skúseností v~manželstve a bolo veľmi povzbudzujúce počuť od nej o~tom, ako sa manželstvo s~Jurajom v~priebehu rokov stalo ešte krajším.

Mnoho z~nás sa stretlo 3.~--~5.~mája~2019 v~Brne na jubilejnej konferencii sestier BJB, kde sme si okrem iného spoločne pripomenuli 50.~výročie obnovenia spoločnej práce sestier z~Česka a Slovenska. Téma konferencie bola {\it Obdarované milosťou}.

Od septembra do decembra sa sestry znovu stretávali raz mesačne. Znovu sme spolu študovali Božie Slovo, ale popri tom sme sa učili aj o~tom, ako študovať samotnú Bibliu. Na stretnutiach sme pokračovali v~tom, že po prednáške sme sa rozdelili do diskusných skupiniek. Bola som povzbudená tým, keď som videla, že Boh pomáha našim sestrám, aby sa otvárali a zdieľali o~sebe, o~svojich zápasoch a o~tom, čo ich Boh učí. V~skupinkách sa povzbudzovali navzájom vo viere. Okrem týchto stretnutí sme mali ďalšie dve stretnutia určené pre manželky, na ktorých som sa zdieľala s~mojimi skúsenosťami v~manželstve. Na konci roka sme mali úžasnú príležitosť vyrábať adventné vence spolu so sestrami zo zboru Viera.

\autor{Clara Jones}


\clanok{Mládež}
Rok 2019, tak ako aj niektoré roky pred ním, bol pre mládež rokom, kedy sme sa zamýšľali nad zmyslom služby mladým. Nad tým, či úsilie, ktoré vynakladáme, je dostatočné, ale aj nad tým, či práca, ktorú robíme, nie je márna a či neostáva naša snaha nepovšimnutá.

V prvej polovici roka naša mládež neprechádzala veľkými zmenami oproti predošlému roku. Mládežníkov nepribúdalo, ale ani nikto neodchádzal. Naše stretnutia sa konali na Súľovskej, ale pravidelne sme sa stretávali aj u~Jonesovcov, kde brat kazateľ Danny viedol sériu tém.

Mládeže u~Jonesovcov nám pomohli nahliadnuť do rodiny brata kazateľa, ale aj spoznať sa bližšie navzájom. Vďaka kontaktom Dannyho Jonesa sme mohli stretnúť a počuť aj niekoľko nových zaujímavých ľudí zo zahraničia, ktorí pôsobia a pracujú ako misionári v~rôznych častiach sveta, napr. v~Afrike a Indii. Myslím si, že toto všetko nám pomohlo silno vnímať túžbu brata kazateľa, aby sa z~nášho zboru stal zbor najmilujúcejší, s~túžbou po misii, ale aj s~túžbou byť osobne blízko každému, kto to potrebuje.

S nástupom nového školského roka sme privítali nových, pravidelných aj občasných mládežníkov. Naša mládež sa opäť začala pomaly rozbiehať a rozrastať, ale pracovníkov, ktorí sú ochotní viesť a pomáhať organizovať mládež, je stále málo. Napriek tomu sme radi, keď sa do prípravy mládeže zapájajú práve mládežníci, ktorí tam chodia. Náplň našich stretnutí sa nijako zásadne nezmenila a stále nám leží na srdci vzťah mládežníkov k~sebe navzájom, ale aj to, aby našli svoje miesto v~mládeži a aby sa cítili prijatí.

Pravidelné stretnutia u~Jonesovcov sa v~druhej polovici roka vytratili kvôli pracovnej vyťaženosti brata kazateľa, ale podarilo sa nám pripojiť k~mládeži zboru Viera, kde mal Danny Jones jednu tému. S~mládežou z~Viery sme sa spojili aj pri príležitosti, keď sme privítali študentov z~Dánska, ktorí nám predstavili štúdium na biblickej škole, ale aj niečo zo svojho osobného života.

Okrem iného sa naši mládežníci zúčastnili na pravidelných akciách, ako je napr. mládežnícka konferencia alebo mládežnícka víkendovka. Našli sa aj takí, ktorí boli na tábore v~Muránskej Zdychave, pripravenom bratmi a sestrami z~okolia Revúcej. Na tieto tábory sú pozývaní aj neveriaci ľudia z~tejto oblasti, a preto je to dobrá príležitosť pre všetkých mladých, ktorí majú túžbu robiť misiu aj takýmto spôsobom.

Som vďačný, že nám Pán Boh dáva čas a príležitosti na to, aby sme sa venovali mladým ľuďom. Prial by som si, aby sme svojimi životmi mohli odzrkadľovať Božiu lásku v~každodennom živote, aby ľudia mali túžbu spoznať Toho, ktorý nám dáva milosť a pokoj.

Prosím vás o~modlitby za výbor mládeže, v~ktorom nás nie je dosť na to, aby sme sa dokázali naplno venovať tomu, k~čomu sme boli povolaní. Chcem vás prosiť o~modlitby za mladých z~nášho zboru, aby mohli byť svetlom svojmu okoliu a mohli odovzdávať ďalej to, čo prijali.  A~prosím, aby ste sa modlili aj za nových ľudí, ktorí by sa mohli pridať no našej mládeže, ale aj pomôcť s~prípravou a organizáciou.

\autor{Dávid Pribula}


\clanok{Dorast}
S dorastencami sme prebrali materiál Detskej misie {\it Kresťanský bojovník}, kde sme hovorili o~duchovnom boji, Božej výzbroji a modlitbe. Pokračovali sme v~príbehoch našich biblických predchodcov a tiež v~pravidelnom čítaní Biblie. Počas letných prázdnin sme boli na dorasteneckom tábore na Chvojnici, čiastočne na zborovom tábore v~Častej a na ďalších menších akciách počas roka.

Stretnutia dorastu sme viedli v~zložení L. Kamocsai, M. Simon, manželia Hovorkovci a manželia Halamičkovci. Od septembra pokračujeme v~zložení manželia Vráblovci, Halamičkovci, Rado Nemec a Martin Simon s~podporou brata kazateľa Dannyho Jonesa.

V centre Lučatín YWAM sme absolvovali seminár o~predtínedžerskom programe, ktorý s~dorastencami realizujeme od nového školského roka. Ide o~kombináciu teoretickej a praktickej prípravy na pubertu a nový spôsob, akým s~dorastencami pracujeme.

Našou snahou je:
\begitems
* podporiť vzťahy medzi dorastencami,
* pripraviť ich na výzvy, ktorým budú čeliť v~období dospievania,
* zapájať do programu aj rodičov našich dorastencov.
\enditems

Stretávame sa na Zrínskeho počas nedieľ od 16.00 do 18.00~hod., ale do budúcnosti by sme stretnutia radi presunuli na piatkové večery. Ani nám sa nevyhýbajú skúšky, ale veríme, že nad tým, o~čo sa s~dorastom snažíme, pevne stojí náš milujúci Ocko.

Keď mal Ježiš 12 rokov, išiel s~rodičmi na slávnosť Paschy do Jeruzalema. Po slávnosti sa vracali a mysleli si, že Ježiš je medzi ostatnými pútnikmi. Takto kráčali jeden deň, a keď ho nenašli medzi známymi, vrátili sa do Jeruzalema, kde ho našli až na ďalší deň. {\it „Čo si nám to urobil, skoro sme prišli o~rozum, keď sme ťa hľadali”, zaznelo, keď ho rodičia našli medzi učiteľmi, ktorí žasli nad jeho bystrosťou. Odpovedal im: „Prečo ste ma hľadali? Neviete, že ja musím byť tu a zaoberať sa záležitosťami môjho Otca?” Oni netušili, o~čom hovorí, no vrátil sa a poslušne žil s~nimi. Ježiš dozrieval, rástol na tele i duchu, požehnaný Bohom aj ľuďmi {\em (Lk 2,41-52)}}.

Ježiš bol dokonalý, ale tiež bol 12--ročný. Bol dokonale 12--ročný!

Túžime, aby naši dorastenci, Tamarka Syčová, Dara Plett, Oskar Kolárovský, Marek Syč,  Daniel a Lenka Vráblovci, Matej a Benko Maďarovci, Tomáš a Šimon Halamičkovci, Naomi a Tobi Dzuriakovci a ich kamaráti boli ako On.

Myslime na nich aj v~modlitbách.

\autor{Martin Simon}


\clanok{Besiedka}
{\it Naša besiedka, naša besiedka, tam my radi chodíme…}

Túto pesničku sme spievali kedysi dávno ešte ako malí besiedkári. Dnes sme už rodičia, dokonca niektorí z~nás sú starí rodičia. Je úžasné, keď vidíme svoje deti, ako kráčajú v~Božích šľapajách a privádzajú na pôdu zboru ďalšie generácie. Modlíme sa, aby deti, ktorých je dnes v~zbore požehnane, v~ňom zotrvali aj ako dospelí kresťania.

V roku 2019 navštevovalo nedeľnú besiedku vyše 20 detí vo veku od 3 do 11 rokov. Mladšie deti sú s~rodičmi v~zhromaždení, deti od 12 do 15 rokov chodia od septembra v~nedeľu popoludní na dorast.  Na začiatku šk. roka 2019/2020 sme vyskúšali priestory škôlky na Palisádach, neďaleko nášho zhromaždenia. Nevedeli sme totiž, či sa všetci na Zrínskeho zmestíme. Napokon sa dorast presunul na 16.00~hod. v~nedeľu poobede, takže pre besiedku zostalo dosť miesta v~pôvodných priestoroch. Malá aj veľká besiedka sa stretáva naďalej o~10.00~hod. na Zrínskeho. Je dobré, že deti môžu byť na začiatku zhromaždenia spolu s~ostatnými bratmi a sestrami. Cítia sa tak už od malička súčasťou spoločenstva, čo je pre ne veľmi dôležité. V~tíme učiteľov zostávajú v~malej besiedke -- Miriam Kešjarová, Janka Máťušová, Katka Kerekréty, Rada Bánová, Kristína Kešjarová; a vo veľkej besiedke -- Vierka Kolárovská, Barborka Pribulová a Ľubka Kráľová. Príležitostne počas uplynulého roka vypomáhala Kristína Matušeková. Spev stále vedie Dianka Dzuriaková. Ak by sa chcel niekto do nášho tímu pridať, s~radosťou ho privítame. Pokojne oslovte kohokoľvek z~nás a môžeme sa o~tejto jedinečnej službe porozprávať.

Pri vyučovaní detí sa snažíme nielen im porozprávať zaujímavé a vzrušujúce biblické príbehy. Túžime, aby pochopili, čo ich Pán Boh chce cez jednotlivé biblické udalosti naučiť. Božie Slovo je živé a mení ľudské mysle a srdcia. Modlíme sa, aby Pán Boh oslovil aj naše deti a pritiahol ich k~sebe čo najskôr. Prosím, modlite sa za mladú generáciu aj vy, pretože dnes je na ňu vyvíjaný enormný tlak a niekedy sa deti strácajú v~množstve možností, ktoré im tento svet ponúka. Nech vedia, kde je skutočná pravda a žijú v~perspektíve večnosti.

\autor{Miriam Kešjarová}


\clanok{Spevokol}
V roku 2019 k~nám pribudli noví speváci, ale niekoľkí aj odišli. Sme vďační Pánu Bohu za možnosť spevom oslavovať nášho Stvoriteľa.

Ako každý rok, tak aj tento sme začali službou na Novoročnom koncerte v~evanjelickom kostole v~Petržalke. Pravidelne sa tam stretávame s~viacerými spevokolmi i skupinami z~Bratislavy a blízkeho okolia. Tentoraz sme poslúžili nielen za sprievodu klavíra, ale aj nášho komorného orchestra.

Už tradične spievame na záver Aliančného modlitebného týždňa v~evanjelickom chráme pieseň Otčenáš aj s~miestnym spevokolom.

Takisto sme aj v~tomto roku spievali pri spomienke na zoslanie Sv. Ducha na Chvojnici.

V tomto roku sme sa snažili slúžiť na domácej pôde počas nedeľných bohoslužieb.  Ide hlavne o~spievanie na prvú nedeľu v~mesiaci pri slávnosti Večere Pánovej. Snažili sme sa počas roka slúžiť s~naším spevokolom aj na pohrebných spomienkových zhromaždeniach.

Hlavný dôraz našej služby cielene prikladáme na naše koncerty pre širokú verejnosť. Vianoce a Veľká noc sú vynikajúcou príležitosťou osloviť aj takých ľudí, ktorí by k~nám inokedy neprišli. Pre veľký záujem verejnosti už pravidelne pripravujeme vianočné koncerty dva, v~sobotu a v~nedeľu.

Aby sme sa na koncerty čo najlepšie pripravili, snažíme sa robiť generálne skúšky niekde mimo nášho zboru. Tento rok sme na generálku vianočného koncertu išli do nášho zboru v~Bernolákove.

Uvedomujeme si dôležitosť tejto služby, ktorá má dosah nielen na poslucháčov, ale aj na nás samotných, a preto prosíme o~silu a požehnanie na túto prácu.

\autor{Slávo Kráľ}


\clanok{Služba ľuďom v~núdzi}
V spolupráci s~občianskym združením {\it Kresťania v~meste} (ďalej ako „KvM“) sa náš zbor aj v~roku 2019 zapojil do pomoci ľuďom v~núdzi, a to hlavne varením polievok a dobrovoľnými finančnými darmi na nákup surovín na polievku. Mobilný výdaj stravy, t.~j. varenie a výdaje prebiehajú každý týždeň. Dvakrát týždenne od apríla do septembra a od októbra do marca sú výdaje trikrát týždenne pod mostom Lafranconi. Ročne navaria dobrovoľníci približne 4~000 litrov hustej výživnej polievky a vydá sa viac ako 10~000 porcií teplého jedla. Na jeden výdaj chodí 50~--~100 ľudí bez domova, do pomoci ktorým je zapojený cca 150-členný tím dobrovoľníkov z~rôznych denominácií a spoločenstiev, ku ktorým patrí a aktívne sa zapája aj náš zbor.

V júni 2019 prebehlo na výdaji očkovanie ľudí bez domova proti hepatitíde v~spolupráci s~občianskym združením {\it Equita}. Hlavní koordinátori z~KvM sa zúčastnili aj konferencie „Cesty k~ukončovaniu bezdomovectva“, ktorú v~Bratislave organizovalo občianske združenie {\it Proti prúdu}. V~októbri sa stretli s~prof. Krčmérym, lekármi a študentami medicíny a v~decembri sa obnovilo poskytovanie zdravotného ošetrenia priamo v~teréne. Naďalej sa tiež sociálna pracovníčka venuje jednému páru bývalých ľudí bez domova, s~ktorými sme boli v~kontakte pri riešení rôznych životných situácií a pomáhame im adaptovať sa z~prostredia bezdomoveckých návykov do bežného života.

Vo výdajovom tíme sa minulý rok veľmi aktívne zapájala s. Zuzka Pařízková.

Minulý rok sme sa do služby varenia zapojili nasledovne:
\begitems
* január 1x (Laurenčíkovci)
* február 1x (Pavel a Ľubka Kohútovci)
* marec 1x (Beata Bogárová s~mamou)
* apríl 1x (Rút Bednáriková so švagrinou Antóniou)
* máj 2x (1x Slávka Volentičová s~Majkou Bilickou a 1x Taligovci)
* jún 1x (Slávka Volentičová s~Majkou Bilickou)
* júl 1x (Slávka Volentičová s~Majkou Bilickou)
* august 1x (Slávka Volentičová s~Majkou Bilickou)
* september 2x (Slávka Volentičová s~Majkou Bilickou)
* október 2x (1x Taligovci a 1x Beata Bogárová s~mamou)
* november 2x (1x Laurenčíkovci a 1x Slávka Volentičová s~Majkou Bilickou)
* december 3x (1x Marcelka Krišková a s~nákupom pomohol Peter Pribula st. a 2x Slávka Volentičová s~Majkou Bilickou)
\enditems

Dobrovoľné finančné dary od darcov z~nášho zboru na nákup surovín na polievky boli v~roku 2019 celkovo vo výške 270~€. Zostatok z~roku 2018 bol 100,48~€. V~r. 2019 sa na varenie polievok použilo 258,15~€. Zostatok do ďalšieho roku je 112,33~€.

Všetkým Vám, ktorí ste ochotne a s~láskou darovali svoj čas, financie, schopnosti, poskytli ste svoje autá, spolupodieľali ste sa na prípravách polievok, či už nákupom surovín, krájaním zeleniny a mäsa, či samotným varením, prihovárali ste sa na modlitbách za núdznych ľudí, vyjadrujem srdečnú vďaku. Vďaka Pánu Bohu za Vás! Ak by ste mali chuť a túžbu pripojiť sa v~tomto ďalšom roku, budeme veľmi radi.

\autor{Beata Bogárová}


\clanok{Connect}

\cast{Spojenie s~Bohom a ľuďmi}

V týchto riadkoch chceme priblížiť službu nášho tímu, ktorej cieľom je založenie nového zboru v~Bratislave.

Sme stále na začiatku, mohli by sme povedať v~rozbehovej fáze. V~nej sa zameriavame na 5 hlavných oblastí:

\begitems \style n
* \aktivita{Rast v~milovaní Boha a ľudí}
V tejto oblasti ide o~vzájomné vystrojovanie a povzbudzovanie sa k~tomu, aby každý z~nás bol vždy a všade mocným a odovzdaným svedkom Ježiša Krista. K~tomu nám slúžia  stretnutia tímu raz za dva až tri týždne.

* \aktivita{Modlitebný zápas}
Máme určený čas medzi 8. a 9. hodinou doobeda, v~ktorom sa každý deň, každý tam, kde sa momentálne nachádza, modlí za spásu ľudí, za potreby Connectu.

* \aktivita{Činenie učeníkov ako náš životný štýl}
Snažíme sa priberať nových ľudí, ktorí nepoznajú Boha, do našich všedných životov s~citlivosťou na napĺňanie ich potrieb a osobne sprevádzať novoobrátených ľudí.

* \aktivita{Budovanie platformy, ktorú voláme Connect nedeľa}
\begitems
* Connect nedele sú nateraz naším jediným stretnutím otvoreným pre všetkých ľudí.
* Začali sme v~prvú júnovú nedeľu 2019 a pokračujeme na Súľovskej 2, vždy v~podvečer od 17. hodiny.
* Connect nedele majú základné prvky klasickej bohoslužby s~tým, že tieto prvky sú zaodiate do rôznych foriem. Nateraz používame najmä vyučovanie, recitály, rozhovory, diskusie, chvály a uctievanie.
* Osvedčuje sa nám sedenie pri stoloch, ktoré vytvárajú neformálne prostredie a umožňujú užšie spoločenstvo sediacim pri jednotlivých stoloch.
\enditems

* \aktivita{Príprava ľudí na založenie ďalšej novej misijnej práce}
V budúcnosti vyšleme ľudí na misijnú prácu z~nášho stredu.
\enditems

\cast{Ako je možné zapojiť sa do služby zakladania nového zboru?}

\begitems \style n
* \aktivita{Stať sa jej súčasťou}
Prijať od Boha túto službu za svoju a verne a oddane v~dobrom i zlom v~nej spolu s~ostatnými stáť a vytrvať.

* \aktivita{Stať sa jej podporovateľom}
Podporovať ju napríklad žehnaním, modlitbami, finančne, praktickou pomocou, aktívnou účasťou na nedeľných Connectoch, modlitebných prechádzkach a podobne.
\enditems

\autor{Tomáš Valchář}


\clanok{Ukrajinská služba}
Prvé stretnutie ukrajinskej skupiny v~Bratislave sa uskutočnilo v~decembri 2018 na Zrínskeho 2 a od marca 2019 sa tieto stretnutia začali konať každú nedeľu. Na naše stretnutia zvyčajne prichádza do 20~ľudí. Každú stredu na Zrínskeho o~19.00~hod. sa konalo stretnutie venované štúdiu Biblie. Začali sme aj so stretnutiami malej skupiny vo štvrtok, ktoré prebiehajú na Černyševského 9. 17. novembra 2019 bol prvý krst.

Začali sme ukrajinskú službu aj v~Nitre. Na jeseň roku 2019 bolo prvé stretnutie tejto skupiny. Skupina sa stretáva v~sobotu o~12.00~hod. dvakrát mesačne v~priestoroch zboru Cirkvi bratskej na Kráľovskej ceste~1. Našich stretnutí sa zúčastňuje do 10~ľudí.

\autor{Viktor Potocki}


\clanok{Správa o~revízii hospodárenia cirkevného zboru BJB Bratislava Palisády}
Revízna komisia v~zložení Miroslav Antalík, Helena Mikletičová, Barbora Antalíková za prítomnosti účtovníčky zboru Ľ. Kohútovej vykonali revíziu hospodárenia za rok 2019.
Boli prekontrolované nasledovné doklady:
\begitems \style -
* výpisy z~bežného účtu vedeného v~Slov. sporiteľni za mesiace 1,~2,~7,~8,~10~a~11;
* výdavkové pokladničné doklady za mesiace 2,~3,~5,~7,~10~a~12;
* príjmové pokladničné doklady za mesiace 2,~3,~5,~7,~10~a~12.
\enditems
Revízna komisia konštatuje, že uvedené doklady sú vedené prehľadne v~súlade s~účtovnými predpismi. Pokladničná kniha je vedená mesačne a~založená priamo pri pokladničných dokladoch.

Neboli zistené žiadne nedostatky.

Stav finančnej hotovosti ku dňu 31.~12.~2019 bol:

\vskip1em\hskip1cm\table{lr}{
pokladňa & 5~944,71~€ \cr
bankový účet & 56~894,03~€ \crl
spolu & 62~838,74~€ \cr
}\vskip1em

Tento stav súhlasí so stavom v~účtovnej evidencii k~uvedenému dátumu.

Revízna komisia konštatuje zvýšenie obetavosti zboru, ktoré sa prejavilo vyššími príjmami zo zbierok a~darov.

\autor{Miroslav Antalík, Helena Mikletičová a Barbora Antalíková}

\tiraz
\bye
