% DOKUMENTACIA:

% Prazdny riadok za textom znamena ukoncenie odstavca.
% Cierne obldzniky na konci riadku (v PDF) - to nechaj na mna (moze to o.i. znamenat, ze treba pridat nejake slovo do \hyphenation, lebo ho sam nevie rozdelit na konci riadku)

% Prikazy pre casti spravodaja:
% \spravodaj{<mesiac>}{<rok>}
% \clanok{<nazov clanku>}
% \autor{<autor clanku>}
% \n<den.mesiac.meno> - zadefinovanie oslavenca
% \narodeniny - vytvorenie tabulky s~narodeninami vsetkych zadefinovanych oslavencov
% \tiraz - ukoncenie spravodaja tirazou

% Styl fontu:
% \bf - bold, plati do konca aktualne skupiny, napr. ak mas {aaa \bf bbb} ccc, tak aaa bude normalne, bbb bude bold, ccc bude normalne
% \it - italic (pouzit rovnakym sposobom ako \bf)
% \bi - bold italic (pouzit rovnakym sposobom ako \bf)
% \rm - normalne (pouzit rovnakym sposobom ako \bf)

% Dalsie prikazy a znaky:
% \begitems - zoznam (odrazky), informacie najdes na stranke http://petr.olsak.net/ftp/olsak/opmac/opmac-u.pdf#toc%3A.5
% \ulink[<cielova adresa]{<zobrazena adresa>} - klikatelny odkaz na webstranku
% \email{<adresa>} - klikatelny odkaz na e-mailovu adresu
% ~ - nedelitelna medzera, napr. v~dome, 21.~6.~2018
% -- - pomlcka (dvakrát -)
% „ - zaciatocna uvodzovka
% “ - koncova uvodzovka
% \noindent - najblizsi odstavec nebude odsadeny
% \vskip<velkost> - vertikalna medzera, napr. \vskip3pt alebo \vskip-3ex (zaporna medzera, t.j. posun smerom hore)

%\typosize[9/12]% - pouzita velkost pisma/riadku
\input makra.tex % nacitanie Ivanom pripravenych nastaveni a prikazov
\hyphenation{star-šov-stvo} % rozdelenie slov na konci riadku, treba tu uviest slova, ktore sam nepozna

\spravodaj{3}{2019}


\clanok {Očakávaná žatva}
Zo všetkých ročných období mám najradšej jar. Po zime čakáme všetci na jar. Tešíme sa na jej príchod, ktorý so~sebou prináša nový život.  Nedávno som videl prvé snežienky v~tomto roku, a~tešil som sa. Vždy prídu. Napriek tomu, že vieme, že prídu, aj tak sa tešíme. Ale niektoré veci prídu, len keď do~nich vstúpime a~pripravíme sa na~ne. Musíme sa zámerne zúčastniť toho procesu. Paradajky alebo uhorky nebudú, ani muškáty alebo astry, ak nezasadíme semená. Ale vieme skoro isto, že ak zasejeme semená, niečo z~nich vzíde.

Nie je to len vo~sfére záhradkárstva. Je to aj duchovný princíp, o čom hovorí Písmo. Galaťanom 6,~7 hovorí: „Nemýľte sa! Boh sa nedá vysmievať, lebo čo človek rozsieva, bude aj žať.“ Väčšinou ten verš vidíme ako varovanie. Ale je tam napísané aj krásne zasľúbenie. Ôsmy verš nám hovorí: „Ale kto rozsieva pre Ducha, z~Ducha bude žať večný život.“ Možno si hovoríš: „No, raz som niekomu svedčil evanjelium, ale nič z~toho nebolo. Asi sa mi to neoplatí.“ Čo keby sme túto filozofiu preniesli do záhradky? „Raz som zasadil jedno paradajkové semienko, ale z~toho nebolo nič.“ Dobre vieme, že treba viac semienok zasadiť, lebo potom určite z~toho niečo bude. Platí to aj v~duchovnej oblasti, keď stále a~radostne rozsievame semená evanjelia, vediac, „že kto rosieva pre Ducha, z~Ducha bude žať večný život.“ Ježiš tak isto o~tom raz hovoril, že z~tých štyroch zasadených semienok, iba jedno vyrástlo k~zrelosti. Pavol to vedel, a~preto pokračuje v~deviatom verši: „Neúnavne konajme dobro, lebo ak neochabneme, v~určenom čase budeme žať.“

Túto jar dôverujme Bohu. Poďme zasadiť, nielen kvety a~zeleninu, ale aj semená evanjelia. Keby každý znovuzrodený veriaci každý týždeň aspoň jedno semeno evanjelia odovzdal niekomu, jednoznačne by z~toho bola úroda. Očakávajme tento rok inú žatvu, žatvu viery, lebo veríme, že ak neochabneme, v~určenom čase budeme žať. Modlím sa, aby sme tento rok v~rámci nášho zboru pokrstili ôsmych nových znovuzrodených. Len Boh to dokáže. Čakáme od Neho zázrak žatvy.

\autor{Danny Jones}


\clanok{Správy zo staršovstva}
Od posledného vydania nášho Spravodaja sa staršovstvo stretlo na jednom stretnutí. Podstatnú časť stretnutia sme venovali diskusii o~hospodárení zboru v~roku 2018 a~návrhom do rozpočtu pre rok 2019. Sme vďační za tých, ktorí nás prišli v~tejto diskusii podporiť. Je to dobré, keď sa stretne viacero pohľadov k~otázke ako je rozpočet. Aj my starší potrebujeme počuť, aké sú predstavy vás, členov zboru, na použitie financií, ktorými tiež slúžime nášmu Pánovi. Zároveň vás chcem pozvať na ešte jednu príležitosť tvoriť smerovanie nášho zboru aj cez rozpočet. Budeme veľmi radi, keď budeme počuť vaše návrhy a~požiadavky na použitie financií, ktoré dostávame do správy od nášho Pána.  Stretnutie sa uskutoční  {\bf 5.~marca~2019 o~18.00~hod.} v~kancelárii na Zrínskeho.

Je to radostná príležitosť, ak Pán Ježiš privádza k sebe ľudí, ktorí ho hľadajú s~úprimným srdcom. Sme preto vďační za to, že sme sa mohli rozprávať s~niekoľkými záujemcami o~členstvo v~zbore. Vidíme, že modlitby za tých, ktorí strácajú vieru a~potrebujú obživenie Svätým Duchom, prinášajú ovocie. Je potrebné, aby sme v~tomto zápase nepoľavovali, ale pevne stáli a~prosili za tých, ktorí potrebujú naše modlitby.

Tešíme sa na spoločné stretnutie 10. marca 2019, pri Výročnom zborovom členskom zhromaždení. Počas tohto stretnutia chceme spomínať na rok 2018, na Božie vedenie, požehnanie, Jeho vyučovanie a~naprávanie nás na Jeho ceste. Ale chceme aj prosiť za vedenie počas obdobia, ktoré je pred nami, aby sme rozumeli Božej vôli a~poslúchali Ho.

\autor{za staršovstvo Peter Pribula}


\clanok{Zborové skupinky}
Od marca začíname naše stretnutia skupiniek (presný termín treba overiť u~vedúdich skupiniek). Boli sme stvorení pre spoločenstvo a~život mimo komunity je nebezpečný. Každý z~nás potrebuje miesto bezpečia a~prijatia, kde môžeme svoj život spracovať a~napredovať. Našou túžbou je, aby si každý z~nás našiel takéto miesto v~spoločenstve. Ak nie si do skupinky zapojený, skontaktuj sa s~vedúcim niektorej z~týchto skupiniek pre viac informácií.
\vskip5pt
\noindent  {\bf Skupinky zamerané na diskusiu ku kázni z predchádzajúcej nedele:}
\vskip5pt
{\it Vedúci:} Paľo a Ľubka Kohútovci (0915 772 763, 0902 815 188)

{\it Miesto:} Petržalka
\vskip5pt

{\it Vedúci:} Miro a Štefka Antalíkovci (0902 112 669, 0904 403 871)

{\it Miesto:} Karlova Ves
\vskip5pt
\noindent  {\bf Skupinky zamerané na manželstvo~--~preberať sa bude kniha {\bi Láska a úcta} od Emersona Eggerichsa:}
\vskip5pt
{\it Vedúci:} Radko a Mirka Hovorkovci (0903 923 387, 0904 266 776)

{\it Miesto:} Staré mesto
\vskip5pt

{\it Vedúci:} Ľuboš a Mirka Kešjarovci (0903 740 340, 0904 400 833)

{\it Miesto:} Trnávka, Ružinov
\vskip5pt

{\it Vedúci:} Danny a Clara Jones (0948 410 777, 0948 288 879)

{\it Miesto:} Zrínskeho 2


\clanok{Kandidáti do staršovstva zboru}
Členovia zboru navrhli do staršovstva nasledujúcich bratov (menovaní prijali kandidatúru):
\vskip-1ex\begitems
* Peter Antalík
* Radovan Hovorka
* Vladimír Ira
* Pavel Kohút
* Peter Kolárovský
* Miroslav Kolářik
* Peter Pribula
* Ján Szőllős
\enditems


\clanok{Kandidáti do revíznej komisie zboru}
Členovia zboru navrhli do revíznej komisie nasledujúcich bratov a sestry (menovaní prijali kandidatúru):
\vskip-1ex\begitems
* Miroslav Antalík
* Barbora Antalíková
* Slavomír Máťuš
* Helena Mikletičová
* Ladislav Taliga
* Katarína Valentová
\enditems


\clanok{Kandidáti do diakonského tímu}
Do tímu diakonov prijali kandidatúru títo bratia:
\vskip-1ex\begitems
* Ján Laurenčík
* Marcel Maďar
* Slavomír Máťuš
* Daniel Mikletič
* Ondrej Škodák
* Ján Štefko
* Ladislav Taliga
\enditems


\clanok{Spoločné modlitby}
\vskip-1ex\begitems
* Muži -- streda {\bf od 6.00~hod. do 7.00~hod.}, kostol na Palisádach
* Ženy -- pondelok {\bf od 17.00~hod.}, Zrínskeho 2
\enditems

Priveďte na spoločné modlitby aj svojich priateľov a známych, ktorým leží na srdci naše mesto a ľudia v~ňom.


\clanok{Verše na zapamätanie}
Na mesiac marec máme nový veršík, ktorý sa chceme spoločne učiť. Veríme, že poznanie Písma prospeje našej duši i našej mysli:

{\it „Tak teda, bratia, stojte pevne a~držte sa podania, ktorému ste sa naučili, či naším slovom, či listom. Ale sám náš Pán Ježiš Kristus a~Boh, náš Otec, ktorý si nás zamiloval a~v~milosti nám dal večnú útechu a~dobrú nádej, nech poteší vaše srdcia a~upevní ich v~každom dobrom skutku a~slove.“}

\autor{2Tes~2,~15~--~17}


\clanok{Stretnutia sestier}
Najbližšie stretnutie sestier sa uskutoční  {\bf 6.~marca o~17.30~hod.} v modlitebni na Palisádach. Do najbližšieho stretnutia si treba naštudovať 2. lekciu po stranu 72. (Ak by ste nestíhali prečítať všetko, tak aspoň po stranu 53, aby ste sa vedeli zapojiť do diskusie.)

Ženy všetkých vekových kategórií sú srdečne vítané!
\vskip5pt

Naše ďalšie posedenie so sestrou Ľubicou Hovorkovou, určené predovšetkým pre manželsky, bude {\bf 8.~marca o~17.30~hod.} u~Clary Jonesovej na Zrínskeho~2. Tentokrát bude rozhovor o~manželstve, ktorý pre ňu pripravil br.~kazateľ Danny Jones. Príďte a~poučte sa aj vy z~jej životných skúseností v manželstve.


\clanok {Kurz (nielen) pre pracovníkov s~deťmi}
Detská misia pripravuje jednodňový kurz {\bf 16.~marca od~9.00 do~15.00~hod.} pre všetkých, ktorí by sa chceli posunúť ďalej vo svojej službe, a~ktorí by radi načerpali informácie, povzbudenie a~nové nápady pre prácu s~deťmi. Zúčastniť sa ho môžu všetci, ktorí sa o~službu deťom zaujímajú. Kurz je užitočný aj pre rodičov, ktorí slúžia svojim deťom každý deň.

Z programu:
\vskip-1ex\begitems
* Prečo a ako vyučovať deti?
* Ako viesť hry a kvízy?
* Ako jednoducho vysvetliť evanjelium?
* Diskusia a iné
\enditems

Viac informácií o~kurze ako aj o~možnostiach prihlásenia sa dostanete u~sestry Mirky Kešjarovej.


\clanok{Senior klub v marci}
Ak dá Pán zdravia a života, v~mesiaci marec sa  stretneme {\bf posledný štvrtok, t.~j.~dňa 28.~marca~2019 na Súľovskej ul. od 10.00~hod. do 14.00~hod.}

Téma stretnutia: „Moc slova“ na základe knižky od Dereka Princa {\it Jazyk -- kormidlo tvojho života.}

Všetci sú srdečne vítaní!

V láske Kristovej

\autor{Jana Makovíni}


\clanok{Služba ľuďom bez domova}
 {\bf Daruj lyžicu človeku bez domova!}  Prostredníctvom výdaja polievky chceme ľudí bez domova učiť zodpovednosti za malé veci. Chceli by sme preto každému človeku darovať „naozajstnú“ lyžicu, ktorá bude jeho a~s~ktorou bude na výdaj prichádzať. Okrem toho, že sa ľudia budú učiť zodpovednosti, vyprodukujeme menej plastového odpadu.  Lyžice môžete doniesť každý pondelok od~17.00 do~19.00~hod. na Ambroseho 6 v Petržalke, kontaktná osoba Sylvia Vaniherová, mobil 0905~484~675. Alternatívne je možné lyžice priniesť aj do zboru v~období od 24.~februára do 17.~marca. Krabica bude položená vo vstupnej hale na stolíku s~nápisom „Lyžice pre ľudí bez domova“. Samostatné lyžice sa dajú kúpiť napr.~v~Tescu. Prosím Vás, nenoste celé príbory. Vopred Vám ďakujem za ochotu pomôcť.

\autor {Beata Bogárová}
\vskip5pt

Aktuálne stale zbierame zimné pánske šatstvo, najmä bundy, čiapky, rukavice, spodné prádlo a topánky. Podobne ako lyžice, šatstvo je možné priniesť na Ambroseho v pondelky od~17.00 do~19.00~hod. po telefonickom dohovore so Sylviou Vaniherovou 0905~484~675.

\autor {Lenka Antalíková}


\clanok{Zbierky za január}
Milí bratia a sestry, ďakujeme za vašu obetavosť. V~mesiaci február ste prispeli:
\vskip-1ex\begitems
* misia: 520~€
* investičný fond: 267~€
\enditems


\clanok{Premietací tím hľadá nových členov}
Do premietacieho tímu hľadáme nových členov -- je to príležitosť pre tých, ktorí máte kladný vzťah k počítačom a chceli by ste sa zapojiť do nejakej služby v rámci zboru. Podrobnejšie informácie poskytnem osobne príp. e-mailom (\email{ivankohut@azet.sk}).

\autor {Ivan Kohút}


\n 10.	3.	Rada 	BÁNOVÁ;
\n 20.	3.	Jana	MÁŤUŠOVÁ;
\n 21.	3.	Ladislav	TALIGA;
\n 25.	3.	Alžbeta	PAULENOVÁ;
\n 25.	3.	Pavol	ŠKULEC;
\n 25.	3.	Filip	KOVÁČ;
\n 26.	3.	Matej	KOLÁŘIK;
\n 28.	3.	Marta	PRIBULOVÁ, ml.;
\n 29.	3.	Marcel	MAĎAR;
\n 29.	3.	Peter PRIBULA, ml.;
\n 30.	3.	Marta	GULDANOVÁ;
\n 31.	3.	Judit	KOBZOVÁ;
\narodeniny


\program{
\p 1  ; pi ;.;;.;;
\p 2  ; so ;.;Jarné prázdniny (mládež nebude);.;;
\p 3  ; ne ;  9.30 ; Bohoslužby (T. Valchár) ; 10.00 ; Chvojnica (M. Kolářik);
\p 4  ; po ; 17.00 ; Modlitby -- ženy (Zrínskeho 2);.;;
\p 5  ; ut ; 15.15 ; Stretnutie pri Biblii (P. Pivka, Zrínskeho 2);.;;
\p 6  ; st ;  6.00 ; Modlitby -- muži (kostol Palisády); 17.30 ; Stretnutie sestier;
\p 7  ; št ; 19.00 ; Biblická hodina (J. Szőllős, Zrínskeho 2);.;;
\p 8  ; pi ; 17.30 ; Stretnutie sestier s Ľ. Hovorkovou (Zrínskeho 2);.;;
\p 9  ; so ; 18.00 ; Mládež (Súľovská 2);.;;
\p 10 ; ne ;  9.30 ; Bohoslužby (D. Jones); 10.00 ; Chvojnica (P. Škulec);
\p    ;    ; 16.00 ; Výročné zborové členské zhromaždenie;.;;
\p 11 ; po ; 17.00 ; Modlitby -- ženy (Zrínskeho 2);.;;
\p 12 ; ut ; 15.15 ; Stretnutie pri Biblii (P. Pivka, Zrínskeho 2);.;;
\p 13 ; st ;  6.00 ; Modlitby -- muži (kostol Palisády);.;;
\p 14 ; št ; 19.00 ; Biblická hodina (J. Szőllős, Zrínskeho 2);.;;
\p 15 ; pi ;.;;.;;
\p 16 ; so ; 18.00 ; Mládež (Súľovská 2);.;;
\p 17 ; ne ;  9.30 ; Bohoslužby (D. Jones); 10.00 ; Chvojnica (J. Štefko);
\p 18 ; po ; 17.00 ; Modlitby -- ženy (Zrínskeho 2);.;;
\p 19 ; ut ; 15.15 ; Stretnutie pri Biblii (P. Pivka, Zrínskeho 2);.;;
\p 20 ; st ;  6.00 ; Modlitby -- muži (kostol Palisády) ; 17.00 ; Stretnutie sestier (kostol Palisády);
\p 21 ; št ; 19.00 ; Biblická hodina (J. Szőllős, Zrínskeho 2);.;;
\p 22 ; pi ;.;;.;;
\p 23 ; so ; 18.00 ; Mládež (Súľovská 2);.;;
\p 24 ; ne ;  9.30 ; Bohoslužby (T. Valchář); 10.00 ; Chvojnica (Vladimír Ira);
\p 25 ; po ; 17.00 ; Modlitby -- ženy (Zrínskeho 2);.;;
\p 26 ; ut ; 15.15 ; Stretnutie pri Biblii (P. Pivka, Zrínskeho 2);.;;
\p 27 ; st ;  6.00 ; Modlitby -- muži (kostol Palisády);.;;
\p 28 ; št ; 9.00 ; Senior klub (Súľovská 2); 19.00 ; Biblická hodina (J. Szőllős, Zrínskeho 2);
\p 29 ; pi ;.;;.;;
\p 30 ; so ; 18.00 ; Mládež (Súľovská 2);.;;
\p 31 ; ne ; 9.30 ; Bohoslužby (Ľ. Dzuriak) ; 10.00 ; Chvojnica (R. Hovorka);
}

\tiraz
\bye
