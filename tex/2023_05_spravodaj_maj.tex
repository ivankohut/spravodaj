\def\velkostpisma{10}
\def\velkostriadku{12.5}
\input makra.tex % nacitanie Ivanom pripravenych nastaveni a prikazov
\hyphenation{star-šov-stvo} % rozdelenie slov na konci riadku, treba tu uviest slova, ktore sam nepozna

\spravodaj{5}{2023}


\clanok {Službu srdcom predchádza pokora}
Ježišove vedomie svojho postavenia, vedomie o~tom, kde patrí, odkiaľ pochádza a kam sa vracia. Vedomie toho, že je tým, skrze ktorého povstalo všetko a skrze ktorého všetko existuje. Ježiš je BOH a jeho identitu mu nikto nevezme. Ježiš koná svoju službu s~týmto vedomím. „Ježiš vo vedomí toho, že Otec mu dal všetko do rúk a že od Boha vyšiel a k~Bohu odchádza...,“ (J~13,3).

Aj my máme o~sebe nejaké vedomie, kedy si uvedomujeme svoju hodnotu, svoju veľkosť, svoje schopnosti, zručnosti, jednoducho, že sme niekto. Čo je nám prirodzené, keď si uvedomujeme svoju hodnotu? Vyžadujeme od iných patričnú úctu a správanie sa voči nám, „ktorého sme hodní“… Do určitej miery naše vnímanie seba samých pripomína príbeh Námana Sýrskeho. Čo robí Náman, keď si je vedomý svojej hodnoty a svojho postavenia? „A Náman, veliteľ vojska sýrskeho kráľa, bol veľký muž pred svojím pánom a veľmi vážený, lebo skrze neho dal Hospodin záchranu Sýrii, a bol to udatný muž, ale bol malomocný“ (2.Kráľ.~5,1). Náman, po tom ako dostane malomocenstvo a vyčerpal svoje možnosti na uzdravenie, na popud izraelskej zajatkyne hľadá uzdravenie v~Izraeli. Kráľ ho posiela za prorokom Elizeom, no keď prichádza k~Elizeovi, hovorí k~Námanovi len sluha vyslaný Elizeom s~inštrukciou, že sa má ísť umyť sedemkrát v~Jordáne a že bude uzdravený. Jordán bola špinavá rieka. No Náman na popud sluhu sa premohol a poslúchol. Pokoril sa a bol uzdravený. Chce sa odvďačiť. No nakoniec Elizeus od Námana neprijíma ani peniaze, ani dary, lebo to neurobil on, ale Boh. Aká pokorujúca lekcia pre Námana. Lekcia o~tom, že to, čo dostal, nebolo pre to, kým je, ani pre to, čo má, dokonca ani pre to, že by si to zaplatil ako službu. Ale to, čo dostal, dostal ako MILOSŤ UZDRAVENIA.

Ako vplýva Ježišove uvedomenie si jeho hodnoty na Jeho službu? Nevedie ho to do pýchy, ale práve naopak: Ježišova veľkosť sa tu ukazuje v~sile pokoriť sa. Text v~Jánovi pokračuje: „vstal od večere, odložil si vrchný odev, vzal si zásteru a opásal sa. Potom nalial do umývadla vodu a začal umývať učeníkom nohy a utierať ich zásterou, ktorou bol opásaný“ (J~13,4-5). Ježiš vedomý si svojej hodnosti, vstupuje do služby nie hodnej jeho postavenia, ale naopak vstupuje do služby, ktorá bola určená sluhom ako jedna z~podradných úloh. Preto jeho službu predchádza pokora, ktorá vychádza z~lásky. „...pretože Ježiš vedel, že prišla jeho hodina, aby odišiel z~tohto sveta k~Otcovi, a preto, že miloval svojich, ktorí boli na svete, preukázal im dokonalú lásku“ (J~13,1).

Aj my, takí akí sme, so všetkým, čo sme dosiahli a čo máme, sme boli ako Náman odkázaní na milosť, ktorú sme si ničím nezaslúžili. Preto aj my preukazujme lásku nie len hodnú nášho postavenia, ale nechajme pokoru, aby predišla našu hodnotu. V~porovnaní s~Ježišom naša hodnota, či postavenie je zanedbateľné, o~to viac buďme ochotní sa pokoriť, aby sme si s~láskou slúžili navzájom. „Podľa toho všetci spoznajú, že ste moji učeníci, ak budete mať lásku jeden k~druhému“ (J~13,35).

\autor{Peter Šrankota}


\clanok {Správa za staršovstvo}
V~apríli sa uskutočnili dve stretnutia staršovstva, na ktorých sme sa venovali aktuálnym témam súvisiacim so životom nášho zboru. Sústredili sme sa najmä na úlohy, ktoré vyplynuli z~výsledkov hlasovania zborového členského zhromaždenia.

Plánovali sme sviatočné zhromaždenia pri príležitosti oslavy obeti nášho Spasiteľa Pána Ježiša Krista, keďže Veľká noc tento rok pripadla na apríl. Tiež sme začali pripravovať krst, riešili sme spôsob zviditeľnenia nášho zboru, pripravovali sa na diskusnú konferenciu delegátov zborov a pokračovali v~témach zefektívnenia platieb za energie v~našich priestoroch a hľadaní spôsobu ako budovať vzájomné vzťahy v~rámci zboru aj prostredníctvom plánovaného celo-zborového víkendového stretnutia na jeseň. Dňa 22.~4.~2023 sme sa zúčastnili na už spomínanej diskusnej konferencii delegátov zborov v~Banskej Bystrici.

Aj naďalej prosíme o~modlitebnú podporu.

\autor{Radislav Nemec}
\vfill\break


\clanok{Verš na zapamätanie}
Tento mesiac máme nový veršík, ktorý sa chceme spoločne učiť. Veríme, že poznanie Písma prospeje našej duši i našej mysli:

{\it „Dúfajte v~Hospodina po všetky veky, pretože Hospodin, len Hospodin je skala vekov.“}

\autor{Iz 26,4}


\clanok{Zbierky za uplynulé obdobie}
Milí bratia a sestry,

v apríli ste prispeli:

\vskip-1ex\begitems
* Misia: 328,00 €
* Investície: 85,00 €

\enditems

Ďakujeme vám, že napriek okolnostiam a neistým ekonomickým vyhliadkam do budúcnosti, ste mnohí prispeli na činnosť a službu zboru. Aj naďalej máte možnosť prispieť do „nedeľnej zbierky“, a to prevodom na účet zboru. Do poznámky pre prijímateľa, prosím, uveďte „zbierka“.

Bankové spojenie: SK36 0900 0000 0000 1147 1836, SWIFT: GIBASKBX

Ďakujeme!


\n 1.	5.	Milica	MALÁ;
\n 1.	5.	Andrea	ČURILLOVÁ;
\n 3.	5.	Dárius	KRÁĽ;
\n 4.	5.	Peter	BUZÁŠ ml.;
\n 8.	5.	Vladimír	KRAJČÍ;
\n 11.	5.	Želmíra	PRAŽENICOVÁ;
\n 16.	5.	Ján	SZŐLLŐS;
\n 17.	5.	Lenka	KOVÁČOVÁ;
\n 18.	5.	Anna	DANTEROVÁ;
\n 19.	5.	Oľga	VALCHÁŘOVÁ;
\n 20.	5.	Rastislav	PAULEN;

\narodeniny


\program{
\p  1 ; po ;.;;.;;
\p  2 ; ut ; 15.15 ; Biblická hodina pre seniorov (P. Pivka) ;.;;
\p  3 ; st ;.;;.;;
\p  4 ; št ; 18.00 ; Biblická hodina (J. Szőllős) ;.;;
\p  5 ; pi ; 17.30 ; Dorast ;.;;
\p  6 ; so ; 18.00 ; Mládež ;.;;
\p  7 ; ne ;  9.30 ; Bohoslužby (P. Šrankota, VP) ;.;;
\p  8 ; po ;.;;.;;
\p  9 ; ut ; 15.15 ; Biblická hodina pre seniorov (P. Pivka) ;.;;
\p 10 ; st ; 17.30 ; Stretnutie sestier (D. Hanesová) ;.;;
\p 11 ; št ; 18.00 ; Biblická hodina (J. Szőllős) ;.;;
\p 12 ; pi ; 17.30 ; Dorast ;.;;
\p 13 ; so ; 18.00 ; Mládež ;.;;
\p 14 ; ne ;  9.30 ; Bohoslužby (T. Valchář) ;.;;
\p    ;    ; 11.00 ; Zborové členské zhromaždenie ;.;;
\p 15 ; po ;.;;.;;
\p 16 ; ut ; 15.15 ; Biblická hodina pre seniorov (P. Pivka) ;.;;
\p 17 ; st ;.;;.;;
\p 18 ; št ; 18.00 ; Biblická hodina (J. Szőllős) ;.;;
\p 19 ; pi ; 17.30 ; Dorast ;.;;
\p 20 ; so ; 18.00 ; Mládež ;.;;
\p 21 ; ne ;  9.30 ; Bohoslužby (P. Šrankota) ;.;;
\p 22 ; po ;.;;.;;
\p 23 ; ut ; 15.15 ; Biblická hodina pre seniorov (P. Pivka) ;.;;
\p 24 ; st ;.;;.;;
\p 25 ; št ; 18.00 ; Biblická hodina (J. Szőllős) ;.;;
\p 26 ; pi ; 17.30 ; Dorast ;.;;
\p 27 ; so ; 18.00 ; Mládež ;.;;
\p 28 ; ne ;  9.30 ; Bohoslužby (F. Čurilla) ;.;;
\p 29 ; po ;.;;.;;
\p 30 ; ut ; 15.15 ; Biblická hodina pre seniorov (P. Pivka) ;.;;
\p 31 ; st ;.;;.;;
}
\vskip1ex
\riadokkoncaprogramu{Z bohoslužieb je zabezpečený online prenos.}


\tiraz
\bye
