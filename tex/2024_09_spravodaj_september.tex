\def\velkostpisma{10}
\def\velkostriadku{12.5}
\input makra.tex % nacitanie Ivanom pripravenych nastaveni a prikazov
\hyphenation{star-šov-stvo} % rozdelenie slov na konci riadku, treba tu uviest slova, ktore sam nepozna

\spravodaj{9}{2024}

\def\sekcia#1{\vskip0.5em\noindent #1}

\clanok {ADONAI je PÁN / MAJSTER}

Základ hebrejského mena Adonai je slovo, ktoré znamená Pán a viaže sa k~vzťahu: Boh je Pán a my sme jeho služobníci. Toto meno sa v~knihách Starého zákona objavuje viac než tristokrát. Keď sa modlíš k~Bohu Adonai, povedz mu, že chceš vydať každú oblasť svojho života Jemu. Modli sa za milosť byt takým služobníkom, ktorý je pohotový konať Božiu vôľu. Pamätaj tiež, že Pán je jediný, kto ťa môže zmocniť na naplnenie Božieho zámeru v~tvojom živote. Tým, že Ho spoznáš ako Pána, spoznáš aj pravý zmysel tohto zámeru. Nový zákon opisuje Ježiša ako Pána a Služobníka zároveň. V~tej druhej úlohe nám ukazuje, aký by mal byť náš vzťah k~Bohu, ktorého meno je Adonai.

\sekcia{KĽÚČOVÝ VERŠ}

„Ty si môj Pán! Bez teba niet šťastia pre mňa.“ (Ž 16,2)

\sekcia{ZAMYSLI SA}

Cítil si niekedy neochotu plniť Božiu vôľu, a pritom si ho oslovoval: „Pane“? Čo ti bránilo urobiť to, čo Pán od teba žiadal?

{\em Pane, odpusť mi, keď som vo svojom živote vyznával, že si Pán iba svojími slovami. Keď som nežil v~integrite a~niečo iné som vyznával v~modlitbe a~iné som robil. Prosím Ťa, pomôž mi od tejto chvíle už prežívať radosť zo služby Tebe a~viac Ťa spoznávať ako môjho Adonai. Môjho Pána a~Boha. Amen. }

\sekcia{PÁN MAJSTER -- ADONAI}

„Koho mám v~nebi? Keď som s~tebou, v~ničom na zemi nemám záľubu. Aj keď mi chradne telo i srdce, Boh mi naveky bude skalou srdca i údelom.“ (Ž 73,25-26)

\vskip-1ex\begitems
* {\it Chváľ ho}: Za to, že On je pre teba zdrojom všetkého, čo potrebuješ pre svoj život.
* {\it Ďakuj mu}: Za všetky veci, ktoré ti Boh do života dal, lebo On je tvojím Pánom. On vie, čo je pre teba dobré, aj hoc tomu v~tých okolnostiach ešte nerozumieš.
* {\it Vyznaj mu}: Všetky snahy nechať si niektoré oblasti svojho života, alebo ich časti, len pre seba. Tam, kde si si chcel ponechať právo rozhodovať sám.
* {\it Pros ho}: Aby sa ti dal do hĺbky spoznať, aký je On Pán -- aj spojitosť medzi tým, že On je Pán a Jeho požehnaním pre teba.
\enditems

{\em PÁN / MAJSTER -- ADONAI}

\sekcia{PRISLÚBENIA SPOJENÉ S~MENOM ADONAI}

Na rozdiel od politikov alebo obchodných lídrov, ktorí využijú svojich ľudí a potom ich odhodia, Boh nás nikdy neprestane podporovať. Potrebujeme spoznať, že Boh, ktorému slúžime, je Boh, ktorý nás miluje. Aby sme sa nemuseli báť, že nás využije, alebo že od nás bude žiadať niečo, čo nedokážeme splniť. Spoznanie toho, kto je Boh, nám pomôže prestať riešiť to, kto sme my alebo v~čom sme slabí.

Napokon budeme schopní vydať svoje životy Jemu vo viere, že dokážeme čokoľvek, o~čo nás požiada, s~pomocou milosti, ktorú od Neho dostaneme.

\sekcia{PRISĽÚBENIA V~PÍSME}

„Ale mne Boh pomáha a môj život udržiava Pán.“ (Ž 54,6)

„Raz prehovoril Boh, počul som toto dvoje: že Boh je mocný a Ty, Pane, milostivý: že Ty každému odplácaš podľa jeho skutkov.“ (Ž 62,12-13)

„No Ty, Pane, si Boh milosrdný a láskavý, zhovievavý, veľmi milostivý a verný.“ (Ž 86,15)


\autor{na základe knihy Božie mená, Peter Šrankota}


\clanok {Správy zo staršovstva}

Staršovstvo zboru sa počas prázdnin mimoriadne stretlo dvakrát -- 18.~7. a 20.~8. Na oboch stretnutiach bol ako hosť prizvaný predseda Rady BJB B. Uhrin. Hlavnou témou oboch stretnutí bolo riešenie situácie v~zbore v~súvislosti so službou kazateľa zboru P.~Šrankotu. Popri uvedenej téme staršovstvo riešilo aj niektoré otázky bežného života zboru, o~ktorých bolo potrebné rozhodnúť -- vyjadrenie sa k~sobášom pripravovaným v~rámci nášho zboru alebo v~našich priestoroch, neuskutočnenie zborového tábora, jesenná zborová víkendovka, oblastné vďakyvzdanie, inštalácia kazateľa zboru.

Staršovstvo prijalo nasledujúce uznesenie:

Na základe vnímania situácie v~zbore a súčasných potrieb členov zboru, reflektujúc podnety zo strany členov zboru, staršovstvo rozpoznáva, že brat kazateľ Peter Šrankota aktuálne nedisponuje potrebnými skúsenosťami, na základe ktorých by mohol situáciu v~zbore riešiť a napĺňať potreby členov zboru.

Preto sa staršovstvo rozhodlo navrhnúť kazateľovi Petrovi Šrankotovi odstúpenie z~funkcie kazateľa Cirkevného zboru BJB Bratislava -- Palisády. Kazateľ Peter Šrankota návrh prijal a
z~pozície kazateľa odstupuje k~31.~12.~2024. Detaily dohody a procesu rozviazania vzájomných záväzkov predloží staršovstvo členom zboru na najbližšom zborovom členskom zhromaždení.

Členovia, ale aj nečlenovia zboru budú mať priestor položiť otázky a vyjadriť svoje pohľady na situáciu v~zbore aj v~rámci diskusie na zborovej víkendovke 21.~9.~2024.

\autor {J. Szőllős}


\clanok {Abeceda starnutia od Juraja Pribulu}

Pri príležitosti nedožitých 85. narodenín kazateľa BJB, ale aj manžela a otca Juraja Pribulu, vám predstavujem publikáciu s~biblickými úvahami na tému ako správne žiť podľa Božieho slova.

Publikácia má názov Abeceda starnutia. Abeceda preto, lebo úvahy sú zoradené v~abecednom poradí. Starnutia preto, lebo sa prihovárajú vekovo starším, ale nie iba im. Veď kto z~nás môže povedať, že nestarne?  

Publikácia je zostavená z~materiálov, ktoré boli podkladom relácií vysielaných TWR rádiom na Slovensku a ktoré Juraj Pribula pripravoval.

Pri tejto príležitosti chcem poďakovať tým, ktorí pomáhali s~dosiahnutím konečnej podoby tejto publikácie. Výber textov a ich úpravu urobila Elena Pribulová, manželka kazateľa Juraja Pribulu. S~jazykovou korektúrou pomáhal Peter Žembery, zalamovanie – úpravu pred tlačou urobil Jaro Bán ml. a nakoniec obal, tlač a väzbu realizovali manželia Hanyusovci z~Košíc.

Predkladám vám túto publikáciu s~modlitbou za to, aby bola pre jej čitateľov pomôckou na porozumenie tomu, ako máme správne žiť podľa Božieho slova na česť a slávu nášho nebeského Otca.

\autor {P. Pribula}


\clanok {Rekonštrukcia fasády modlitebne}

V~apríli 2024 sa vykonal odber vzoriek z~povrchových vrstiev fasády našej modlitebne odborným reštaurátorom za účelom overenia pôvodného riešenia farebnosti objektu, zistenia stavu pohľadových vrstiev a za účelom navrhnutia optimálnych reštaurátorských technológií a postupov obnovy fasády. V~júli sme dostali vypracovanú správu z~tohto reštaurátorského výskumu. V~správe bol zahrnutý aj umelecko-historický popis a vyhodnotenie nášho objektu.

Z~výskumu vyplýva, že pôvodne boli použité dva odtiene. Na kamenné prvky (...okolo brán) bledookrový odtieň a na všetkých zvyšných omietkových vrstvách aj ozdobných profiláciách zlatookrový odtieň (o~málo svetlejší ako súčasný odtieň).

V~návrhu sú uvedené postupy a materiály. Z~celej omietky je nutné odstrániť staré nátery, a preklepaním zistiť a odstrániť nesúdržné plochy a nahradiť novou omietkou. Soklová časť bude omietnutá sanačnými omietkami. Poškodené plastické časti sa domodelujú do pôvodného tvaru. Následne sa pred finálnym farebným náterom (3 až 4 vrstvy) celá plocha natrie zjednocujúcim náterom (zjednotenie povrchu a savosti podkladu).

Výsledky reštaurátorského prieskumu a návrhu rekonštrukcie dostal aj Krajský pamiatkový úrad a 15.~7.~2024 nám zaslal rozhodnutie, v~ktorom schvaľuje predložený návrh rekonštrukcie. Súčasťou rozhodnutia sú podmienky, ktoré musíme pri rekonštrukcii dodržať.

Náš najbližší cieľ bude doplniť tieto zistenia a výsledky už osloveným firmám za účelom vypracovania aktualizovaných a relevantných cenových ponúk.

Sme vďační Pánovi, že sa v~tejto veci postupuje, že stále prebieha zbieranie financií na tento účel, a prosíme o~múdrosť pre ďalšie kroky.

\autor {za hospodársky výbor Ľ.~Syč}


\clanok {Verš na mesiac}

„Milosrdenstvo a vernosť nech ťa neopustia. Priviaž si ich na krk, napíš si ich na tabuľu srdca.“ (Pr 3,3)
 

\clanok {Začiatok činnosti zborových zložiek} 

Besiedka začína s~vyučovaním detí malej a veľkej besiedky v~nedeľu 15.~9. Dorast sa prvýkrát stretne v~piatok 6.~9., mládež v~sobotu 7.~9. Seniori sa na biblickej hodine s~br. Pivkom stretnú v~utorok 3.~9. Sestry majú svoje stretnutie v~stredu 18.~9. Podrobnosti sú v~kalendári. Biblická hodina s~br. Szőllősom začne až v~októbri -- 3.~10.


\clanok {Inštalácia br.~kazateľa J.~Szőllősa a spoločný obed}

V~nedeľu 8.~9.~2024 bude mať náš zbor slávnostnú inštaláciu br.~kazateľa Jána Szőllősa. Bohoslužba začína tradične o~9.30~hod. Slovom poslúžia bratia D.~Kraljik a B.~Uhrin. O~13.00~hod. pokračujeme spoločným obedom v~hoteli Plus v~Trnávke. Prihlásiť sa môžete online. Cena obeda je 7~€ a počet možných miest je 100. Podrobnejšie informácie získate u~br.~J.~Štefka.
\vfill\break


\clanok {Jesenná víkendovka v~Berei, 20. -- 22. september}

Opäť máme možnosť prežiť spoločné chvíle vrámci zborovej rodiny na jesennej víkendovke. Uskutoční sa počas víkendu 20. -- 22.~9.
v~penzióne Berea v~Modre Harmónii. Prihlasovanie bolo spustené online formou. Ak nemáte túto možnosť, prihláste sa osobne u~K.~Kerekréty.
Tešíme sa na hosťovanie V.~Boška, spoločné rozhovory, čas a oddych. Detailný program bude čoskoro oznámený.


\n 2.	9.	Radislav	NEMEC;
\n 5.	9.	Dušan	UHRIN;
\n 14.	9.	Štefan	SYNOVEC;
\n 16.	9.	Daniel	PLETT;
\n 19.	9.	Richard	HALAMÍČEK;
\n 21.	9.	Kvetoslava	MAĎAROVÁ;
\n 21.	9.	Miroslava	SIMONOVÁ;
\n 22.	9.	Viera	KOLÁROVSKÁ;
\n 25.	9.	Stanislav	KRÁĽ;
\narodeniny


\program{
\p  1 ; ne ;  9.30 ; Bohoslužby (J. Szőllős, VP);.;;
\p  2 ; po ;.;;.;;
\p  3 ; ut ; 15.15 ; Biblická hodina pre seniorov (P. Pivka);.;;
\p  4 ; st ;.;;.;;   
\p  5 ; št ;.;;.;;  
\p  6 ; pi ; 17.30 ; Dorast ;.;;
\p  7 ; so ; 18.00 ; Mládež ;.;; 
\p  8 ; ne ;  9.30 ; Bohoslužby (D. Kraljik, B. Uhrin);.;;
\p  9 ; po ;.;;.;;
\p 10 ; ut ; 15.15 ; Biblická hodina pre seniorov (P. Pivka);.;;
\p 11 ; st ;.;;.;; 
\p 12 ; št ;.;;.;;
\p 13 ; pi ; 17.30 ; Dorast ;.;;
\p 14 ; so ; 18.00 ; Mládež ;.;;  
\p 15 ; ne ;  9.30 ; Bohoslužby (P. Pribula), začína besiedka ;.;;
\p 16 ; po ;.;;.;;
\p 17 ; ut ; 15.15 ; Biblická hodina pre seniorov (P. Pivka);.;;
\p 18 ; st ; 17.30 ; Stretnutie sestier ;.;;
\p 19 ; št ;.;;.;; 
\p 20 ; pi ; 17.30 ; Dorast ;.;;
\p 21 ; so ; 18.00 ; Mládež ;.;; 
\p 22 ; ne ;  9.30 ; Bohoslužby (M. Ira) ;.;; 
\p 23 ; po ;.;;.;; 
\p 24 ; ut ; 15.15 ; Biblická hodina pre seniorov (P. Pivka) ;.;;
\p 25 ; st ;.;;.;;
\p 26 ; št ;.;;.;; 
\p 27 ; pi ; 17.30 ; Dorast ;.;;
\p 28 ; so ; 18.00 ; Mládež ;.;; 
\p 29 ; ne ;  9.30 ; Bohoslužby (P. Šrankota) ;.;;
\p 30 ; po ;.;;.;;
}



\tiraz
\bye
