\def\velkostpisma{9}
\def\velkostriadku{12}
\input makra.tex % nacitanie Ivanom pripravenych nastaveni a prikazov
\hyphenation{star-šov-stvo} % rozdelenie slov na konci riadku, treba tu uviest slova, ktore sam nepozna

\vyrocnespravy{2022}

\clanok{Zbor}
\cast{Úvod}

Rok 2022 bol pre náš zbor rokom veľkých zmien. Pre zbor je vždy veľkou zmenou a aj výzvou výmena kazateľa a pred potrebu tejto zmeny sme boli postavení hneď na začiatku roku, keď brat kazateľ Danny Jones 23.~1.~2022 oznámil, že sa musí vrátiť z~rodinných dôvodov späť do USA a ukončiť svoju službu v~našom zbore. Nakoľko proces voľby a inštalácie nového kazateľa trval až do októbra, požiadalo ma staršovstvo zboru, aby som správu kazateľa za zbor ako celok, spracoval ja. Táto „správa o~stave únie (zboru)“ ako som ju sám pre seba zvykol nazývať v~časoch mojej kazateľskej služby v~tomto zbore nebude taká komplexná, ako takáto správa má byť, nakoľko nemám už taký prehľad o~živote zboru a nebude obsahovať pohľady a vízie do budúcnosti, čo prenechám nášmu novému kazateľovi.

Ďalšou veľkou zmenou aj z~hľadiska histórie zboru a rozvoja našej služby na Božom kráľovstve bol zánik našej misijnej stanici na Chvojnici a zároveň vznik ukrajinskej misijnej stanice. Rok 2022 bol zároveň aj rokom, v~ktorom sa skončili mimoriadne opatrenia v~súvislosti s~pandémiou COVID-19 a zároveň aj vojna na Ukrajine zasiahla celú krajinu a priniesla energetickú krízu a infláciu. Medzi týmito udalosťami, ukončením mimoriadneho stavu kvôli covidu a začatím invázie Ruska na Ukrajinu, bol len jeden „normálny“ deň. Uprostred všetkých týchto zmienených udalostí, ktoré boli a sú veľkými výzvami pre náš zbor a pre naše životy a svedectvo pred ľuďmi o~Pánovi a jeho milosti sme mohli zažívať, čo znamená verš Božieho Slova pre náš zbor na rok 2022: „Vedzte, že Hospodin divne si vyznačil zbožného, Hospodin počuje, keď volám na Neho.“ (Žalm~4,4). Aj v~uplynulom roku sme v~mnohých zložitých a ťažkých situáciách mohli volať na nášho Pána a On napĺňal svoje zasľúbenie a venoval pozornosť nášmu volaniu a odpovedal v~pravý čas.

\cast{Najdôležitejšie udalosti v~živote zboru}

Ako som už zmienil v~úvode správy, najdôležitejšími udalosťami v~živote nášho zboru v~roku 2022 boli zmena na poste kazateľa zboru a vznik novej (ukrajinskej) misijnej stanice nášho zboru.

V roku 2021 sme začali v~spolupráci s~bratom kazateľom Dannym Jonesom hľadať kandidáta na ďalšieho kazateľa nášho zboru, ktorý by s~ním spolupracoval. Počas toho procesu hľadania sa rodinná situácia brata Dannyho vyvinula tak, že na začiatku roku 2022 musel ohlásiť ukončenie svojej služby v~našom zbore a návrat do USA. Z~kandidátov na ďalšieho kazateľa zboru sa tak stali kandidáti na kazateľa zboru. Brat kazateľ Danny Jones v~celom procese oznámenia odchodu a ukončenia svojej služby spolupracoval so staršovstvom zboru a veľmi uľahčil zvládnutie tejto náročnej situácie a zmien s~tým spojených. Z~nášho ľudského pohľadu sa jeho služba v~našom zbore skončila predčasne, ale Pán Boh mal svoj čas. Sme vďační Bohu za požehnanie, ktoré sme za štyri roky služby Dannyho v~našom zbore prijali. Sme vďační Dannymu a Clare za ich obetavú službu v~našom zbore, za ich lásku k~nám, ktorú sme mohli pociťovať. Svoju vďačnosť sme im mohli vyjadriť pri rozlúčke s~nimi začiatkom februára. Brat Danny potom ešte slúžil v~pozícii kazateľa zboru až do svojho odchodu koncom februára 2022 a ako štatutár a správca zboru až do konca mája.

V povinnostiach kazateľa zboru, najmä pri vysluhovaní Večere Pánovej, pri kázaní a vyučovaní Božieho Slova a pri sobášnych obradoch a pohrebných rozlúčkach ho zastupoval až do nástupu nového kazateľa zboru v~októbri brat Ján Szőllős a ďalší bratia a pri povinnostiach správcu a štatutára zboru predseda staršovstva, brat Peter Pribula.
Výber a voľbu nového kazateľa zboru komplikovala skutočnosť, že v~roku 2022 sa končilo aj funkčné obdobie staršovstva a na výročnom zborovom členskom zhromaždení bolo zvolené nové staršovstvo, kde sa vymenila polovica členov. Vďaka Pánovej milosti sa však podarilo prejsť procesom volieb -- návrhom kandidátov, rozhovormi s~nimi a pripraviť voľby kazateľa zboru, ktoré sa konali 29.~5. -- 19.~6. Kandidatúru na kazateľa zboru prijal brat Peter Šrankota, ktorý pôvodne chcel kandidovať na ďalšieho kazateľa zboru. Vo voľbách bol zvolený za kazateľa a správcu zboru a so staršovstvom bol dohodnutý termín jeho nástupu do služby koncom septembra a slávnostná inštalácia sa konala 9.~októbra. Sme vďační Pánovi, že povolal a poslal do služby v~našom zbore ďalšieho služobníka v~pozícii kazateľa zboru.

Božie vedenie a úžasné Božie plánovanie a cesty sme mohli zažívať aj pri vzniku ukrajinskej zborovej stanice v~našom zbore. Brat Viktor Potockij bol pokrstený v~našom zbore v~roku 2001, potom sa vrátil na Ukrajinu, kde vyštudoval zakladanie zborov a s~podporou nášho zboru zakladal zbory. V~roku 2018 vnímal povolanie vrátiť sa na Slovensko a založiť zbor pre ukrajinsky hovoriacich bratov a sestry. V~roku 2021 ukončil magisterské štúdium teológie v~Banskej Bystrici a popri tom skupina ukrajinských bratov a sestier v~našom zbore početne vzrástla. Na zborovom členskom zhromaždení 20.~2.~2022 sme odsúhlasili vznik ukrajinskej misijnej stanice „Nadija“. Bolo to štyri dni pred začiatkom vojny na Ukrajine. Vďaka Božiemu vedeniu sme mali pripravené štruktúry pre pomoc utečencom z~Ukrajiny, do ktorého sa náš zbor a najmä misijná stanica „Nadija“ plne zapojili. Počet členov tejto stanice, ktorá je už vlastne prakticky samostatným zborom ďalej počas roka vzrástol. Koncom júla sme mohli inštalovať brata Viktora Potockého v~spolupráci s~Radou BJB za kazateľa pre službu Ukrajincom na Slovensku a za správcu misijnej stanice „Nadija“. Na programe dňa je riešenie osamostatnenia tejto stanice na samostatný cirkevný zbor v~rámci BJB.

V~procese života cirkvi sa musíme učiť, že vznikajú aj zanikajú nové zbory. Neboli sme na to zvyknutí, stav sa dlhé desaťročia nemenil, preto sme museli novým skutočnostiam prispôsobiť aj náš zborový poriadok. Popri vzniku novej misijnej stanice sme museli prijať aj realitu zániku našej misijnej stanice na Chvojnici. V~samotnej Chvojnici ostali žiť len dvaja členovia nášho zboru a počet členov po úmrtiach a odsťahovaniach klesol na 5. Rozhodli sme sa aj formálne ukončiť existenciu misijnej stanice na Chvojnici. Zostávajúcich členov na Chvojnici sme zverili do starostlivosti miestneho zboru ECAV vo Vrbovciach, kde Chvojnica patrí. Zborovú chalupu na Chvojnici budeme ďalej využívať a príležitostne aj modlitebňu na Chvojnici.
Počas roku sme posilnili aj prácu na zakladaní nového zboru Connect, kde sme prijali praktikanta Filipa Barkócziho. Správu o~vývoji práce v~tejto skupine nájdete medzi ostatnými správami. Chceli by sme, aby sa už v~roku 2023 aj z~tejto skupiny stala samostatná misijná stanica nášho zboru.

\cast{Štatistika}

Čísla v~štatistike nášho zboru sa v~uplynulom roku výrazne hýbali najmä kvôli vzniku ukrajinskej misijnej stanice Nadija. Nie je možné ani celkom presne stanoviť počet, nakoľko po vzniku misijnej stanice si viedli už štatistiku ďalších, nových prijatých členov tejto stanice sami a v~štatistike nášho zboru figurujú už len členovia, ktorých sme prijali pred vznikom stanice.
Na začiatku roku 2022 sme mali v~zbore 173 členov. Do ukrajinskej stanice sme prijali ďalších štyroch členov a celkovo 12 členov sme vyčlenili do ukrajinskej misijnej stanice a 5 členov na Chvojnici sme včlenili priamo medzi členov materského zboru Palisády, ktorého počet tak bol 165 členov. Piati členovia zboru sa presťahovali a ukončili členstvo v~našom zbore (Danny a Clara Jones, Pavol a Blažeka Škulecovci, Zora Fedáková).

V~uplynulom roku sme sa rozlúčili s~celkovo 8 bratmi a sestrami a priateľmi, ktorí boli v~spojení s~našim zborom, a pripravili sme, alebo sme participovali na rozlúčkovom obrade. Z~členov nášho zboru sme sa rozlúčili ešte v~ januári s~Bohušom Kráľom (77), vo februári so sestrou Annou Kopčokovou (98), ktorá bola jednou z~najstarších členiek nášho zboru. V~marci nás do večnosti predišla sestra Milada Krejčová (80) a v~máji sestra Kristína Zlochová (85) z~Chvojnice. V~našej modlitebni mali rozlúčku v~auguste aj Peter Csütörtöki (42) a v~decembri Daniel Boledovič (71) a rozlúčili sme sa v~júli s~Petrom Rapošom (73) a v~októbri s~Ondrejom Dudášom (95), ktorí patrili do širšieho spoločenstva nášho zboru. Na konci roku 2022 sme mali v~materskom zbore 158 členov a 72 členov evidovala ukrajinská misijná stanica „Nadija“.

Uplynulý rok bol rokom, keď aj kvôli výmene kazateľa sa priamo v~materskom zbore neuskutočnil žiaden krst. V~ukrajinskej stanici Nadija sa uskutočnili 2 krsty, v~júli sa krstilo na vyznanie svojej viery 8 bratov a sestier a v~decembri 6 krstencov.

Pán Boh požehnal v~roku 2022 do rodín členov nášho zboru dve deti, pre ktoré sme vyprosili aj požehnanie, a to Jáchyma Syča a Leu Pelíškovú.
Na spoločnú cestu životom sa rozhodli vykročiť 4 páry spomedzi členov a priateľov nášho zboru, v~máji Marta Brnová a Ondrej Majer a Anna Plett a Andrew Rucin, v~júni Daniel Plett a Kirsten Paynter a Kristína Kešjarová a Juraj Horvátik.

Kým na začiatku roku 2022 bola účasť na zhromaždeniach ovplyvnená ešte pandemickými opatreniami, väčšinu roku (od konca februára) už boli možné stretnutia bez obmedzení. Pandémia ovplyvnila účasť na bohoslužbách, ale vďaka Bohu sa po pandémii účasť príliš neznížila a na nedeľných hlavných bohoslužbách sa zúčastňuje okolo 100 ľudí. Zároveň prebieha aj priame online vysielanie, ktoré sledujú najmä členovia, ktorí sa zo zdravotných alebo iných dôvodov nemôžu zúčastňovať našich bohoslužieb, a ľudia z~iných miest na Slovesnku aj v~zahraničí. Paralelne sa na ďalších zhromaždeniach v~nedeľu (na zhromaždeniach staníc a skupín) zúčastňovalo asi ďalších 100 ľudí. Okolo 50 členov zboru evidujeme ako vzdialených členov, ktorí sa dlhodobo z~dôvodu zdravotného stavu, veku, vzdialeného bydliska, alebo z~pastoračných dôvodov dlhodobo nezúčastňujú na bohoslužbách a neparticipujú na živote zboru. Ostávajú predmetom našich modlitieb spolu s~tými členmi našich rodín, ktorí ešte neprijali Krista za svojho Spasiteľa, alebo sa vzdialili od Pána.

Našich nedeľných bohoslužieb aj rôznych iných zborových aktivít sa viac-menej pravidelne zúčastňujú aj viacerí priatelia, ktorí nie sú, alebo sa nechcú stať z~rôznych dôvodov členmi nášho zboru. Nemáme ich spočítaných, ale môže ich byť okolo 30.

Za veľmi dôležité považujem, aby sme prehlbovali svoj záujem o~ľudí, ktorí prichádzajú do nášho spoločenstva ako nepravidelní, alebo noví návštevníci a majú záujem o~naše spoločenstvo. Nie je to len úloha uvítacej služby pri dverách, ktorej služba sa žiaľ od covidu neobnovila v~organizovanej forme, ale v~podstate každého z~nás „domácich“, aby sme prejavovali záujem a ľudí, ktorí k~nám prídu, aby si mohli u~nás nájsť priateľov, vybudovať kontakty a nájsť v~našom spoločenstve svoju duchovnú rodinu.

\cast{Modlitby}

Vážme si milosť, že môžeme predkladať naše chvály aj prosby všemocnému Bohu, a buďme vďační, že On viditeľne odpovedá. Modlitebné zázemie a podpora, príhovorné modlitby za našich bratov a sestry, za priateľov a príbuzných, ktorí ešte nepoznajú Pána Ježiša ako svojho Spasiteľa, ako aj modlitebná podpora všetkých aktivít, celého života nášho spoločenstva je jednou z~podstatných vecí, ktoré nás odlišujú od akéhokoľvek ľudského spolku.

Popri našich individuálnych modlitbách v~komôrke sú pridanou hodnotou spoločenstva spoločné modlitby, ktoré majú špeciálne prisľúbenie od Pána. Som rád, že súčasťou veľkej väčšiny našich stretnutí, či už skupín, pracovných výborov aj spoločných bohoslužieb sú aj hlasné spoločné modlitby.

Všetkých našich bratov a sestry, priateľov a aj neznovuzrodených členov našich rodín, o~ktorých som hovoril aj v~predchádzajúcom odseku, ako aj všetky problémy máme príležitosť prinášať neustále v~našich príhovorných modlitbách a prosbách pred nášho Pána. Modlitebný život je jedným zo základov života jednotlivca aj spoločenstva veriacich.

Špeciálne spoločné modlitebné stretnutia počas covidu zanikli a už sa neobnovili. O~to dôležitejšie je udržiavať modlitby na ostatných spoločných stretnutiach.

Celý uplynulý rok pokračovali v~našich priestoroch na Zrínskeho ulici aj pravidelné modlitby (približne v~dvojtýždňovom intervale) kresťanov z~rôznych spoločenstiev z~Bratislavy a okolia s~poslancami NR SR. Vytvorila sa skupina modlitebníkov, ktorí aj týmto spôsobom napĺňajú výzvu Božieho Slova (1Tim 2,2), aby sme sa modlili za tých v~moci postavených.

V modlitebni na Palisádach sa od septembra, raz mesačne konali modlitby spojené s~chválami pre a za ľudí pracujúcich vo verejnej službe (v~parlamente, na ministerstvách, rôznych celoštátnych úradoch, ...). Aj týmto spôsobom sme využívali výhodné umiestnenie našej modlitebne, aby sme poskytli priestor pre modlitby.

Nepodarilo sa uplynulý rok plne obnoviť pomôcku – modlitebný kalendár nášho zboru a nepokračovala ani jeho aktualizácia uverejňovaná v~týždenných oznamoch. Určite by bolo dobré nájsť vhodnú formu, ako pravidelne prinášať aktuálne modlitebné predmety do zboru.

Viaceré veľmi požehnané chvíle chvál a modlitieb boli v~nedeľu dopoludnia. Využívajme výsadu a príležitosť, ktorú máme, že sa môžeme modliť k~všemohúcemu Bohu, Stvoriteľovi neba a zeme a prichádzať k~Nemu kedykoľvek ako milované deti k~nášmu Nebeskému Otcovi.

\cast{Nedeľné bohoslužby}

Nedeľné dopoludňajšie bohoslužby boli aj v~roku 2022 hlavnou príležitosťou, kde sa mohla stretnúť k~spoločnej oslave nášho Pána celá zborová rodina. Vysoko si cením službu moderátorov, ktorí pripravia a zorganizujú nielen celé bohoslužby, ale väčšinou majú aj obohacujúce sprievodné slovo a úvody k~spoločným modlitbám. Túto dôležitú službu však potrebujeme doplniť o~ďalších bratov a sestry ochotných a schopných túto službu konať.

Veľkým obohatením nedeľných bohoslužieb je služba klasickými spoločnými piesňami, ktoré už roky verne pripravuje a doprevádza brat Slávo Kráľ ako aj služba chválami. Som rád, že po covide sa od septembra obnovila aj služba spevokolu zboru, ktorý okrem vianočného koncertu poslúžil aj v~nedeľu. V~roku 2022 sa striedali rôzne chválospevové skupiny pri službe, za čo som vďačný. Je to o~to cennejšie, že sa z~kapacitných dôvodov nedarí pokryť každú nedeľu službou chválospevmi.

Som vďačný Bohu, že máme v~našom zbore viacerých obdarovaných bratov a sestry, ktorí sú zároveň väčšinou aj ochotní uplatniť svoje obdarovanie a slúžia v~našom zbore kázaním Slova. Kázanie a výklad Božieho Slova je ústrednou časťou našich nedeľných bohoslužieb. Záujem o~službu Slovom v~našom zbore je aj zo strany hostí, čo nás teší.
Božím Slovom v~uplynulom roku slúžil v~úvode roku do februára ešte kazateľ zboru Danny Jones a od októbra nový kazateľ zboru brat Peter Šrankota v~dohodnutej periodicite dvoch nedieľ v~mesiaci. Zvyšné nedele slúžili bratia zo zboru a pozvaní hostia. Z~ordinovaných kazateľov, členov a návštevníkov nášho zboru slúžili najmä Ján Szőllős, Tomáš Valchář, Dušan Uhrin a tiež Stanislav Baláž. Z~členov zboru viackrát slúžili bratia Peter Pribula, Peter Kolárovský a Slavo Kráľ, ako aj kazateľ ukrajinskej stanice Viktor Potockij. Náš zbor trikrát navštívil a slúžil Slovom predseda Rady BJB Benjamin Uhrin, najmä pri inštalácii kazateľov. Z~ďalších kazateľov BJB nám prišli zvestovať Božie Slovo Timotej Hanes zo zboru Revúcka Lehota a Nick Gagnon z~Medzinárodného zboru ako aj misijní pracovníci Pavle Cekov a Pali Šrankota.
Veľkým prínosom je, že naše bohoslužby sú vysielané priamo cez internet a sú tam aj archivované. Patrí za to veľká vďaka našim technikom, ktorí prenosy a archiváciu zabezpečujú.

\cast{Vzťahy so zbormi v~západnej oblasti BJB a ostatnými cirkvami a medzinárodná spolupráca}

Je výborné, že sa po prestávke aj napriek covidovým obmedzeniam, ktoré ešte začiatkom roku 2022 pretrvávali, podarilo obnoviť modlitebné stretnutia v~rámci Aliančného modlitebného týždňa. Kvôli zložitosti organizovať tieto stretnutia v~rôznych spoločenstvách, ako tradične bývali, sme poskytli priestory našej modlitebne, kde mohli počas týždňa modlitieb prísť kresťania z~rôznych spoločenstiev a spoločne sa modliť.

Spoločné oblastné zhromaždenie zborov západnej oblasti BJB sa minulý rok nekonalo, ani spoločné dvojstranné zhromaždenia so zbormi sa nekonali. V~uplynulom roku sa neuskutočnili ani služby kazateľa nášho zboru v~iných zboroch oblasti a u~nás slúžil len kazateľ Medzinárodného baptistického zboru Nick Gagnon. Z~iných denominácií v~našom zbore slúžil brat farár Boris Mišina (ECAV) a brat Rado Krupa (CB). Návštevy nášho kazateľa v~iných denomináciách sa neuskutočnili.

Tak ako v~predchádzajúcich rokoch pokračovala aj v~roku 2016 spolupráca so zbormi v~rámci platformy Kresťania v~meste najmä pri varení polievky bezdomovcom. Spoločné ekumenické bohoslužby na Veľký Piatok sa nekonali.

\cast{Služba zborových zložiek}

Som veľmi vďačný Pánovi za obetavú službu viacerých bratov a sestier v~rôznych zložkách nášho zboru. O~ich službe nájdete informáciu v~správach za jednotlivé zložky. Služba aj za sťažených podmienok v~týchto zložkách bežala aj počas pandémie. Po covide sa obnovila práca vo väčšine z~nich. Nová služba, ktorú je potrebné rozvinúť, je služba na parkovisku, pri jeho otváraní a zatváraní. Som vďačný Bohu, že máme k~dispozícii toto parkovisko počas našich bohoslužieb.

Ak som zmienil vyššie, nevšimol som si, že by sa obnovila dôležitá uvítacia služba. Verím, že sa nájdu obdarovaní bratia a sestry, ktorí aj túto službu obnovia.

\cast{Záver}

Za všetko, čo sa aj v~uplynulom roku v~našej službe a v~živote nášho zboru udialo, patrí v~prvom rade vďaka nášmu Bohu, ktorý dával silu, zmocnenie, svoje požehnanie, a ktorý nás viedol aj v~roku 2022. Tieto výročné správy nemajú slúžiť na naše vychvaľovanie sa, ale majú povzbudzovať oslavu nášho Pána. Jemu jedinému patrí naša chvála za čokoľvek, čo sa podarilo uskutočniť. Bez neho by sme nič neboli schopní urobiť. Naša vďaka však patrí aj ľuďom, všetkým, ktorí ste sa aktívne zapojili do života a služby v~našom zbore, všetkým, ktorí ste túto službu podopierali svojimi modlitbami. Nech je povzbudením a výzvou do našej ďalšej spoločnej služby v~Pánovej prítomnosti a k~jeho nasledovaniu aj v~tomto roku verš so zasľúbením, ktorý sme dostali na rok 2023 pre náš zbor: „Ak mi niekto slúži, nech ma nasleduje a kde som ja, tam bude aj môj služobník. Ak mi niekto slúži, toho Otec poctí.“ (J~12,26)

\autor{Ján Szőllős}
\vfill\break


\clanok{Staršovstvo}

Pre rok 2022 sme dostali text zo Žalmu 50,23: „Kto vďaku obetuje, ten ma ctí, tomu, kto správnou cestou kráča, ukážem Božiu pomoc.“ Dve témy sú obsiahnuté v~tomto texte. Úcta voči Bohu a Božia pomoc. Sú navzájom prepojené. Ak si ctíme, vážime alebo milujeme nášho nebeského Otca a náš život smerujeme po Jeho cestách, On nám pomáha. Ako tá pomoc vyzerá, je individuálne pre každého z~nás. Mnohí z~nás by vedeli veľa hovoriť o~Božej pomoci aj počas roku 2022.

Staršovstvo na tom nie je inak. Aj my si uvedomujeme Božiu pomoc, Jeho vedenie, inšpiráciu a múdrosť v~službe, ktorou nás poveril.
Rok 2022 bol rokom volieb a nevyhlo sa to ani staršovstvu a kazateľovi.
Do výročného zborového členského zhromaždenia 2022 pracovalo staršovstvo v~zložení kazateľ Danny Jones a členovia Peter Antalík, Vladimír Ira, Peter Kolárovský, Miroslav Kolářik, Daniel Plett, Peter Pribula, Ján Szőllős a zástupca bratov a sestier z~Ukrajiny Viktor Potocký.

Po voľbách sa zloženie čiastočne zmenilo a pracujeme v~zložení Peter Antalík, Ľuboš Dzuriak, Miroslav Kolářik (v~rámci možností svojho zdravia), Marcel Maďar, Radislav Nemec, Viktor Potocký, Peter Pribula a Ján Szőllős. Po voľbách kazateľa je od polovice roka medzi nami aj Peter Šrankota.

Ak bol rok 2021 aj v~našom spoločenstve poznačený koronavírusom, tak rok 2022 bol poznačený vojnou na Ukrajine. Pomoc, ktorú sme poskytli utečencom pred vojnou, bola reakciou na slová „Jedni druhých bremená neste a tak naplňte zákon Kristov“ (Gal. 6,2). Znova chcem vyjadriť vďačnosť všetkým členom a priateľom zboru, ktorí sa akokoľvek zapojili do tejto pomoci. Tak ako sme vedeli a mohli, napĺňali sme zákon Kristov. Verím, že sme tým pomohli odľahčiť bremená tým, ktorí prešli cez náš zbor, naše domácnosti a naše srdcia.

V~priebehu tejto služby sa ukázalo, že pomoc utečencom potrebuje koordinátora na úrovni celej jednoty. Za tohto koordinátora bol ustanovený Viktor Potocký. V~súčasnosti má na starosti ukrajinskú misijnú stanicu nášho zboru a zároveň koordináciu pomoci utečencom v~rámci BJB na Slovensku.

Ako som naznačil, v~roku 2022 sme sa venovali aj voľbám. Volili sme členov staršovstva, revíznej komisie, diakonov a delegátov na konferencie zborov našej jednoty. Pripravované voľby ďalšieho kazateľa sa zmenili na voľby kazateľa.

Začiatkom roku Danny a Clara Jonesovci definitívne ukončili prácu v~našom zbore. Vrátili sa do USA, aby sa venovali starostlivosti o~rodinu a hlavne o~ich chorého syna. Zo správ, ktoré od nich máme, počujeme a sme vďační, že Pán Boh potvrdzuje správnosť ich rozhodnutia.
Namiesto Jonesovcov sme v~polovici roku dostali novú kazateľskú rodinu. Sme vďační za Šrankotovcov a službu, ktorú na seba zobrali. Potrebujú naše modlitby a podporu. Preto nás všetkých volám k~tomu, aby sme ich podporovali v~rozhodnutí, ktoré urobili, a sľube, ktorý dali aj nám aj Bohu.

Posledné roky so sebou prinášajú zmeny. Zmeny, na ktoré potrebujeme reagovať aj ako kresťania. A~to sú témy, ktorým venujeme našu pozornosť v~staršovstve. Vedenie zboru (pasenie stáda) v~meniacej sa dobe.
A do tejto meniacej sa doby sme dostali verš na rok 2023, ktorý nás volá k~stabilite, zodpovednosti, práci a vedomej závislosti na Pánu Ježišovi a našom Nebeskom Otcovi: „Nie vy ste si vyvolili mňa, ale ja som si vyvolil vás a ustanovil som vás, aby ste šli a prinášali ovocie, aby vaše ovocie zostávalo, a aby vám Otec dal všetko, o~čo ho budete prosiť v~mojom mene.“ (Ján 15,16).

Počas školského roku sa staršovstvo stretáva v~intervale 2 -- 3 týždne v~utorok o~18.30 hod. Diskutujeme, modlíme sa a hľadáme správnu cestu pre náš zbor a jednotlivcov v~ňom.

\autor{Peter Pribula}


\clanok{Diakonia}

Verš z~Božieho Slova na rok 2022: „Veru, veru Vám hovorím: kto počúva moje slovo a verí Tomu, ktorý ma poslal, má večný život a nejde na súd, ale prešiel zo smrti do života.“ (Ján~5,24).

Diakonia v~uplynulom roku mala pravidelne štvrťročné pracovné stretnutia, konkrétne 25.~4.~2022, 13.~6.~2022, 26.~9.~2022 a 14.~11.~2022. Z~týchto stretnutí boli spísané zápisnice, v~ktorých bol uvedený predmet rokovania, návrh jeho riešenia a následne aj vyhodnotenie plnenia riešenia. Zápisnice boli preukázateľne doručené členom diakonie vždy pred nasledujúcim stretnutím.

Diakonia zabezpečovala a vykonávala nasledovné činnosti:
\vskip-1ex\begitems
* kontaktovanie členov zboru, ktorí už dlhší čas z~neznámych dôvodov nenavštevujú pravidelné zhromaždenia, s~cieľom zistenia príčin a následne zaradenia do predmetov modlitieb
* pravidelná návšteva seniorov, chorých a imobilných členov zboru v~ich domácnostiach, v~niektorých prípadoch spojená aj vysluhovaním Večere Pánovej, s~cieľom zistenia ich potrieb a budovania a udržiavania prirodzených bratsko-sesterských vzťahov
* zabezpečovanie pravidelného mesačného vysluhovania Večere Pánovej, a to po stránke materiálnej (víno, chlieb, kalíšky, tácky) a personálneho obsadenia
* spolupráca na projekte výstavby Účelového zariadenia RODINA pre seniorov v~Bernolákove
* organizovanie a zabezpečenie spoločného zborového obedu pri príležitosti inštalácie nového kazateľa Petra Šrankotu (9.~10.~2022)
* organizovanie a zabezpečovanie obedu, a to pre tých, ktorí začali pravidelne navštevovať naše zhromaždenia (30.~10.~2022) a seniorov (11.~12.~2022)
* pravidelné návštevy v~domove seniorov Betánia so službou piesňami a Božím Slovom (P.~Pivka, L.~Gubová, V.~Laurenčíková)
* biblické vyučovanie pre seniorov z~nášho zboru a iných spoločenstiev s~názvom „Popoludnie pri Biblii“, ktoré vedie P.~Pivka
* finančná podpora varenia obedov pre bezdomovcov
\enditems

Touto cestou ďakujem všetkým členom diakonie za aktívnu a obetavú prácu a zároveň aj staršovstvu zboru za spoluprácu a podporu.

\autor{Ján Štefko}


\clanok{Hospodársky výbor}

Z Božej milosti môžeme pokračovať v ďalšom roku našej pozemskej púti a pracovať na Božej vinici. V roku 2022 nám náš Pán pripravil cestu a prostredníctvom svojho Slova, ktoré nám zanechal, nás povzbudzuje v Žalme 37,39. On, náš Pán je záchrana spravodlivých a je útočišťom v čase súženia. Tak sme to prežívali v časoch pandémie.

Vo Chvojnici naša stanica zanikla, ale zostala nádej. V~našej zborovej chalupe, ktorú navštevujú a využívajú rekreačne bratsko-sesterské spoločenstvá na víkendové a tábornícke pobyty, je možnosť priblížiť sa a byť svetlom a soľou domácim osadníkom. V~roku~2022 bolo v~zborovej chalupe 20 rekreačných pobytov, 2 brigády a 1 zborový výlet na svätodušné sviatky. Počas roka bola vykonávaná bežná údržba chalupy.

Na Zrínskeho po odchode br. D. Jonesa (vďaka všetkým, ktorí sa zúčastnili naloženia kontajneru), bol pred nasťahovaním br. Šrankotu vymaľovaný byt, (vďaka br. P.~Kohút). Vďaka i všetkým, ktorí sa podieľali pri nasťahovaní Šrankotovcov. Uskutočnila sa brigáda -- upratanie pôjdu a vyvezenie nepotrebných vecí; vďaka všetkým zúčastneným, najmä však dorastencom a br.~Synovcovi.

V~tomto roku nás čakajú nové výzvy, do ktorých sme boli povzbudení veršom z~Písma Svätého Ján~15,8-9.

Na záver by som chcel poďakovať všetkým, ktorí sa podieľajú na potrebách nášho spoločenstva, či už finančnými darmi alebo priložením rúk k dielu na Božej vinici.
Nech je meno Božie oslávené v našich srdciach a v našich životoch.

\autor{Daniel Mikletič}


\clanok{Biblické a iné vzdelávanie}

Správa o~biblickom a inom vzdelávaní obsahuje len vzdelávanie formou stretnutí na biblických hodinách organizovaných vo štvrtok večer na Palisádach (Zrínskeho) pod mojím vedením. Ostatné formy biblického vzdelávania organizované jednotlivými zložkami zboru môžu byť zahrnuté v~správach za zložky alebo v~správe kazateľa zboru. V~tejto správe nie sú zahrnuté ani vzdelávacie aktivity na nadzborovej úrovni, ktorých sa zúčastnili členovia nášho zboru, a ani vzdelávanie v~skupinkách.

Spoločné štúdium Svätého Písma prebiehalo po zrušení pandemických opatrení viac-menej pravidelne v~týždennej periodicite, okrem prázdnin. Pravidelnosť bola občas narušená, keď som kvôli pracovným povinnostiam bol nútený stretnutie zrušiť. Online vzdelávanie vzhľadom na charakter nášho vzdelávania na biblických hodinách a zloženie účastníkov a aj moje kapacitné možnosti sme nezrealizovali.

Počas celého uplynulého roku sme pokračovali vo štvrtky v~priestoroch na Palisádach (od novembra na Zrínskeho) v~preberaní Kázne na hore. Od januára sme pokračovali v~preberaní kresťanovej spravodlivosti (5,21) a do júna sme prebrali časti po Mt~6,10, po prosbu modlitby Pánovej „buď vôľa Tvoja“. Cez prázdniny som na zborovom tábore v~auguste preberal blahoslavenstvá zhrnuté do šiestich tém. Po prázdninách sme od konca septembra pokračovali v~preberaní modlitby Otče náš a v~polovici decembra sme skončili pri Mt 6,34 „hľadajte najprv kráľovstvo Božie...“ Vzdelávania vo štvrtky sa zúčastňovalo v~priemere 10 až 12 členov a priateľov nášho zboru. Približne toľkí sa zúčastnili aj prázdninových lekcií.

\autor{Ján Szőllős}


\clanok{Sestry}

Začiatkom roka 2022 sme sa dozvedeli, že naša drahá Clara Jonesová, ktorá viedla službu ženám na Palisádach od januára roku 2019, sa vracia do Spojených štátov. Cítili sme smútok, ale aj veľkú vďačnosť za Clarinu službu, za jej vyučovanie, ktoré nám otváralo novým spôsobom pravdy Písma, za jej lásku k~Pánovi Ježišovi i k~nám.
Clara mi odovzdala svoju službu, no od začiatku pri mne verne stáli sestry Gitka Kráľová, Mirka Hovorková, Barbi Antalíková a Barborka Pribulová, ktoré aj predtým tvorili Clarin užší pracovný tím a ktorým som veľmi vďačná za podporu a pomoc.

V~prvom polroku roku 2022 sme pokračovali v~štúdiu blahoslavenstiev podľa knihy Úvahy o~Kázni na hore od Martyna Lloyda-Jonesa, ktoré sme začali ešte s~Clarou v~októbri 2021. Úvod k~jednotlivým blahoslavenstvám si zakaždým pripravila iná sestra: Elenka Pribulová, Štefka Antalíková, Helenka Mikletičová, Mirka Hovorková, Baka Pribulová.

V~júni sme naše stretnutia zakončili posledným blahoslavenstvom: Blahoslavení prenasledovaní pre spravodlivosť, lebo ich je nebeské kráľovstvo (Mt 5,10). Toto štúdium nám prinieslo veľké požehnanie i radosť.
Boli sme vďačné za vzácnu možnosť stretnúť sa so sestrou Aničkou Plettovou pár dní pred jej svadbou a odchodom do Spojených štátov a vyprosiť pre ňu Božie požehnanie do manželského života. Využili sme aj príležitosť vyprosiť pred svadbou požehnanie pre sestru Kristínku Kešjarovú -- pre nás všetky to bola radostná a obohacujúca skúsenosť.
K~nášmu sesterskému životu samozrejme patrili aj návštevy novonarodených detičiek.

Konferencia Odboru sestier BJB v~SR a ČR sa konala v~Prahe 23.~--~25.~septembra a výnimočná bola tým, že jej témou boli problémy, o~ktorých sa väčšinou nehovorí, ako napr. komunikácia dnešnej doby, manipulácia a skryté násilie, rozbité vzťahy v~rodine. V~rovnakom termíne bolo aj stretnutie sestier Evanjelickej aliancie v~Račkovej doline, ktorého sa zúčastnili aj niektoré naše sestry.

Od jesene sme si začali pozývať na stretnutia sestry, ktoré si pre nás pripravili rozličné témy. Ester Jankovičová – Vzťahy, Rút Krajčiová – 5 kňazských modlitieb ženy, Aďka Šrankotová – Božia blízkosť, Rút Maďarová –  Nádhera pohostinnosti v~biblických dobách aj dnes.

9. novembra sme sa na Palisádach spojili v~modlitbách s~baptistickými ženami na celom svete pri príležitosti Svetového dňa modlitieb. Témou roka 2022 bol Víťazný život a ústredný verš, nad ktorým sme sa zamýšľali, bol z~2. listu apoštola Pavla Korinťanom 2,14 -- 15: „Ale vďaka Bohu, ktorý nám vždy dáva víťazstvo v~Kristovi a naším prostredníctvom zjavuje na každom mieste vôňu jeho poznania. Lebo sme Kristovou vôňou, príjemnou Bohu uprostred tých, čo získavajú spásu, aj medzi tými, čo hynú.“ Svojou zbierkou sme podporili prácu sestier po celom svete.

Celý rok sme sa stretávali pravidelne každý druhý týždeň, vždy prišlo najmenej pätnásť sestier, niekedy aj vyše tridsať, s~túžbou po duchovnom raste a zdieľaní sa, čo nás napĺňa veľkou vďačnosťou Bohu.

\autor{Jarmila Cihová}


\clanok{Mládež}

V~roku 2022 sme sa začali vracať do normálu. Rok sme začínali s~nádejou, že naše stretávanie sa už nebude obmedzované a my budeme môcť naplno prežívať spoločenstvo. Spoločenstvo jeden s~druhým, ale aj s~ našim Pánom pri spoločných stretnutiach.
Naše mládeže boli stretnutia pri zamysleniach, štúdiu Božieho slova a spievaných chválach.

Keď však začiatkom roka začala vojna na Ukrajine, naše stretnutia nabrali aj nový rozmer. Pri jednej príležitosti sme spoločne skúsili vyjadriť, aké pocity v~nás táto situácia vyvoláva. Počas mládeže sme spísali myšlienky, pocity a obavy, ktoré nás prepadli krátko po začiatku tohto konfliktu. Ale zozbierali sme aj verše, piesne a Božie zasľúbenia, ktoré nás napĺňajú pokojom, nádejou a vierou, že napriek všetkým okolnostiam je Pán Boh väčší a má moc nás ochrániť aj v~tejto situácii.

Modlili sme sa spolu za Ukrajinu, hľadali sme Božie vedenie pre túto situáciu a pre každého z~nás osobne, aby sme vedeli pomáhať tam, kde nás Pán Boh postaví.
Naďalej sme potom pokračovali aj v~nasledujúcich týždňoch v~modlitbách za našich bratov a sestry na Ukrajine, ale aj za tých, ktorí prichádzali na Slovensko a aj do nášho zboru.
Tento rok sme medzi nami privítali aj nových mládežníkov z~rôznych kútov Slovenska, ale aj zo zahraničia (Ukrajina, Švajčiarsko). Sme vďační za nových ľudí, ktorí chodia pravidelne, ale aj za tých ktorí prídu vtedy, keď sú práve v~Bratislave.
Tešíme sa, že mládež je miestom, kde môžeme spoločne hľadať odpovede na otázky, ktoré máme. Môžeme tu prijímať požehnanie aj byť požehnaním pre druhých. Chceme aj naďalej byť ponúkať priestor, kde sa mladí cítia vítaní a sú súčasťou živého spoločenstva.
Počas roka sme sa stretávali v~priestoroch na Súľovskej, za ktoré sme Bohu veľmi vďační. Toto miesto si obzvlášť vážime, lebo nám poskytuje nielen veľký vnútorný priestor na spoločenské hry a aktivity, ale aj vonkajší dvor kde môžeme počas teplejších dní grilovať a tráviť spolu požehnaný čas.

V~roku 2022 bol veršom pre mládež Žalm 46,2, kde sa píše: „Boh nám je útočiskom a silou, pomocou v~súžení vždy osvedčenou.“ Na začiatku roka sme nevedeli, akými súženiami budeme prechádzať. Ale Pán Boh nás nimi previedol a učil nás, že On je naším útočiskom a silou.
Sme vďační Pánu Bohu za požehnania tohto roka, ďakujeme za podporu zboru a kazateľov. Ďakujeme aj za nového spolupracovníka Filipa Barkóciho, ktorý sa pridal do našej mládeže a je ochotný pomáhať pri vyučovaní Božieho slova, ale aj pri rozvíjaní osobných vzťahov s~mládežníkmi.
Naďalej prosíme za modlitby za nás, ktorí vedieme mládeže, aby nám Boh kládol na srdce potreby mladých a dôležitosť evanjelia.
Prosíme aj o~modlitby za mládežníkov, aby mohli prijímať požehnanie, ktoré má Boh pre nich pripravené, a aby mohli nájsť správnu cestu pre ich životy.
A nakoniec prosíme aj o~modlitby za nových ľudí, ktorí sú zapálení pre službu mladým.

\autor{Dávid Pribula}


\clanok{Dorast}

Po dvoch rokoch obmedzení a oddelenia sme sa z~Božej milosti mohli začať stretávať naplno osobne a bez rúšok. Dorasty sme mali až na pár výnimiek na Súľovskej 2 každý piatok o~18.00 hod.

Naše prvé stretnutie sme mali netradične vonku pri Bratislavskom hrade. Ďalšie dorasty sme pokračovali v~štúdiu judských kráľov z~druhej knihy Kronickej (od Chizkiju po Cidkiju). Po veľkonočnej téme v~apríli sme začali preberať situáciu izraelského národa po babylonskom zajatí v~knihe Jeremiáš, kapitoly 40-44. Jeden dorast sme venovali základnému rozdeleniu kníh a dôvodu, prečo boli takto usporiadané. V~máji nás navštívil br. Martin Simon so svojím svedectvom pôsobenia na Ukrajine. Dorast sa koncom mája zúčastnil celodennej brigády na chate Chvojnica. Po nej sme privítali br.~Ivana Kohúta so zaujímavou témou o~dinosauroch a drakoch v~Biblii. Pred letnými prázdninami sme mali biblické kvízy, ktoré sa týkali prebratých tém. Dorastenci mali možnosť uviesť do praxe svoju rýchlosť a vedomosti pri hľadaní v~Biblii. Na prázdniny sme dostali letnú výzvu -- prečítať si knihu Genesis a naučiť sa niekoľko Žalmov.

V lete sme sa zúčastnili tábora na Chvojnici. Hlavná téma, ktorá celý náš pobyt viedla, bola o~tom, kto je a aký je Boh -- že Boh je svätý a spravodlivý, mal s~nami Zmluvu, že nás z~tejto zmluvy vytrhol hriech a prečo musel prísť Ježiš, aby zmluvu obnovil a hriech premohol. Večerné témy pri táboráku boli o~piatich jazykoch lásky - čo každý z~nich znamená a bolo nám vysvetlené, že každý z~nás hovorí a prijíma lásku inak. Počas tábora sme hrali hry a novinkou bol doobedný baseball a večerné výzvy, ktoré budovali tímovú spoluprácu. Prežili sme úžasný a hlavne požehnaný týždeň. Veríme, že sa tento rok stretneme znova. V~novom školskom roku sme pokračovali v~téme babylonského zajatia v~knihe Daniel (Daniel 1-6). Po každých troch kapitolách sme mali biblický kvíz, v~ktorom sme si témy zopakovali.

Minulý rok sme stretnutia dorastu viedli v~zložení: manželia Vráblovci, Janko Kováčik, Rado Nemec a hosťami Martinom Simonom, Ivanom Kohútom a Mirkou Hovorkovou. Na dorastoch sa pravidelne zúčastňovali: Marek Syč, Matej a Benko Maďarovci, Daniel a Lenka Vráblovci, Davyd a Iľja Potockí, Radko Nemec a Emma Čonková.

Cieľom nášho stretávania je budovanie vzťahu našich tínedžerov k~Bohu a povzbudzovanie v~hľadaní Ježiša Krista ako ich osobného spasiteľa. Takisto sme vďační, že máme možnosť využívať zborové priestory, či už ide o~budovu na Súľovskej alebo aj chatu na Chvojnici. Každé osobné stretnutie je možnosťou na budovanie vzťahov medzi dorastencami navzájom, k~zboru ako takému, ale aj prípravou na plnohodnotné začlenenie sa do cirkvi.
Prosíme vás o~modlitby na tento rok, a to najmä za našich dorastencov. Aby mohli odovzdať svoj život Kristovi a mohli náš zbor brať ako rodinu, ktorá ich miluje a prijíma. Takisto prosíme o~modlitby za dorastencov, ktorí stretnutia z~akéhokoľvek dôvodu nemôžu navštevovať, aby mohli tento rok byť s~nami. Modlite sa prosím aj za nás učiteľov, nech sme pri každej téme citliví na to, čo chce Duch Svätý hovoriť do ich sŕdc. Ďakujeme za vašu pomoc a modlitby. Veríme, že nás bude náš Pán viesť aj v~týchto časoch skúšok a neistoty vo svete.

„Pokoj vám zanechávam, svoj pokoj vám dávam, ale ja vám ho dávam, nie ako svet dáva. Nech sa vám srdce neznepokojuje a neľaká.” (Ján~14,27)

\autor{Ján Kováčik}
\vfill\break


\clanok{Besiedka}

V~uplynulom roku sme sa s~deťmi stretávali viac menej pravidelne. Na začiatku roka sme ešte prispôsobovali podmienky pandemickej situácii, ale postupne sa okolnosti zlepšovali, a tak sme mohli začať s~pravidelnými besiedkami. V~malej besiedke (deti od 3 do 7 rokov) evidujeme 14 detí a vo veľkej besiedke (od 7 do 11 rokov) 17 detí. Niektoré deti však dlhodobo na besiedku nechodia z~rôznych dôvodov (rodinná situácia, zmena bydliska a pod.). Napriek tomu, že ich nemáme na našich stretnutiach, modlíme sa za ne a myslíme aj na ich rodiny. Sme vďační, že niektoré deti stretávame aspoň na rodinnom tábore alebo cez sviatočné obdobie (Vianoce, Veľká noc).

Uvedomujeme si, že deti sú dnes pod veľkým tlakom. Dianie vo svete, ale aj v~rodinách je veľmi dynamické a nepokojné. O~to viac deti potrebujú nájsť pokoj v~Bohu. Modlíme sa, aby sme im vedeli ukázať cestu k~Pánovi Ježišovi, aby Mu čo najskôr mohli odovzdať svoje životy a nemuseli zažívať zranenia a sklamania.

Sme vďační, že v~uplynulom roku sme sa mohli v~hojnom počte stretnúť na rodinnom tábore a po dvojročnej pauze sme opäť pripravili vianočnú besiedku v~kostole. Je dobré, keď sa rodiny stretávajú nielen v~zhromaždení, ale aj na rôznych výletoch a osobných návštevách.
Povzbudzujeme rodičov, aby využívali možnosti vzájomného stretávania sa. Deti potrebujú vyrastať v~kolektíve a je dobré, keď majú okolo seba veriacich kamarátov. Pandémia spôsobila, že máme sklon k~izolácii, avšak virtuálny svet nikdy nenahradí osobný kontakt.

Ďakujeme vám za vaše modlitby!

Učiteľky v~malej besiedke: Miriam Kešjarová, Kika Horvátiková, Mirka Hovorková, pomocníci: Katka Kerekréty, Martina Javorníková, Martin Hovorka

Učiteľky vo veľkej besiedke: Ľubka Kováčiková, Baka Pribulová, Kvetka Maďarová, Slávka Volentičová.

\autor{Miriam Kešjarová}


\clanok{Spevokol}

Do roku 2022 sme vchádzali veľmi opatrne. Vírus nám bránil nielen spievanie, ale aj možnosť byť spolu. Pre zborový spev je jedna zo základných požiadaviek vzájomné stretávanie sa. Bez spoločných nácvikov dokáže fungovať len mládežnícky spevokol JAS, kde si jednotliví členovia všetko individuálne nacvičia doma a stretnú sa až na konečnom vystúpení pred verejnosťou. My sme sa na takúto úroveň ešte nevypracovali.

Celý rok sme čakali na pokyn zhora, či covid ustupuje a či prácu spevokolu môžeme opäť rozbehnúť. Pomaly, veľmi pomaly sa rozbriežďovalo, až nakoniec, asi dva mesiace pred koncom roka, vznikla možnosť opatrne začať spievať. Ja som si vtedy niekedy pripadal ako Mojžiš, ktorý mal vyviezť svoj národ z~púšte. Bol vedený Pánom Bohom, no z~druhej strany tlačený reptajúcim ľudom.

Dnes som vďačný za našich „reptajúcich“ spevákov, lebo som sa podvolil ich tlaku a využil prvú možnosť na rozbehnutie spevokolu. Hneď sme si určili prvý cieľ, vianočný koncert na pôde nášho zboru. Vďaka Božej milosti priniesol nové požehnanie nielen pre poslucháčov, ale aj pre samotných spevákov.
Všetko berieme z~Božích rúk, lebo vieme, že milujúcim Boha všetky veci slúžia na dobro.

\autor{Slávo Kráľ}


\clanok{Služba ľuďom v~núdzi}

Členovia a priatelia nášho zboru sa v~roku 2022 zapojili do služby varenia pre ľudí bez domova 31x, čo predstavuje 27~\% z~počtu všetkých dobrých polievok!
Intenzita zapojenia sa zvýšila v~druhom polroku a prijímame mnohé poďakovania od KvM, že v~posledných mesiacoch by bez našich sestier a bratov táto služba ťažko fungovala. V~januári tohto roku sme dokonca pokryli viac ako 50~\% varení!

Naše poďakovanie patrí: Katke Kerekréty, Hanke Šándorovej, Kohútovcom, Vrábľovcom, Kráľovcom, Rút Bednarikovej, Laurenčíkovcom, Želke Praženicovej, Janke Matúšovej a Ingrid Jančulovej, Brunckovcom, Cihovcom, Írovcom.
Okrem poďakovania za samotné varenie mi dovoľte poďakovať aj ochotným darcom, ktorý podporujú túto službu finančne. V~súčasnej dobe sa náklady na jedno varenie pohybujú okolo 70~€. pred pár rokmi to bolo do 30,-€.
Mnohí variči pokrývajú náklady sami zo svojho vrecka, niektorí, hlavne tí, ktorí varia často, potrebujú niekedy preplatiť náklady na ingrediencie. V~neposlednom rade mi dovoľte poďakovať Kvetke Maďarovej a Danielovi Mikletičovi, ktorí sa skvelo starajú o~kuchynku a priestor na Zrínskeho.
Verím tomu, že služba tým najzraniteľnejším je dôležitá. Pre mnohých môže byť naša teplá polievka jediným teplým jedlom za niekoľko dní. Táto služba je spojená aj s~výdajom šatstva a otvoreným priestorom pre zvestovanie Slova Božieho.

„Kráľ im odpovie: ‚Veru, hovorím vám: Čokoľvek ste urobili jednému z~týchto mojich najmenších bratov, mne ste urobili.‘“ (Mat.~25,40)

\autor{Slávka Kráľová}


\clanok{Ukrajinská misijná stanica}

Rok 2022 bol pre ukrajinskú službu náročný, ale zároveň plodný. Od začiatku vojny na Ukrajine sme sa aktívne zapájali do služby ukrajinským utečencom. V roku 2022 sme spolu s našimi partnermi pomohli približne 500 ľuďom na Slovensku a mnohým ďalším na Ukrajine.

Vďaka prílevu Ukrajincov sa naša služba niekoľkonásobne rozrástla. Na našich nedeľných bohoslužbách sa vždy zúčastňuje viac ako sto ľudí. Dnes máme 68 členov. V roku 2022 sme uskutočnili dva krsty a pokrstených bolo 14 ľudí. V roku 2023 sme začali s prípravou na krst a v súčasnosti pripravujeme 3 ľudí.

V minulom roku sme začali so službou pre mládež a službu študentom. Spustili sme malé skupinky, ktorých je v súčasnosti 5. V lete sa nám podarilo zorganizovať dva tábory -- mládežnícky a detský. Okrem toho sme sa zameriavame na misiu. Členovia ukrajinského zhromaždenia, rodina Volodymyra a Júlie Matsolovcov, odišli slúžiť do Nových Zámkov.

\autor{Viktor Pototskyi}
\vfill\break


\clanok{Connect}

Za rokom 2022 sa pozeráme s~vďačnosťou v~srdciach nášmu Bohu. Bol to rok, ktorý pre Connect znamenal veľké výzvy a súčasne kvalitatívny a kvantitatívny rast.
Po dobe covidovej sa znásobili naše kontakty s~ľuďmi, čoho dôsledkom je, že v~súčasnej dobe má svoj duchovný domov v~Connecte 25 dospelých ľudí a 9 detí. Z~tohto počtu sú 11 ľudia členmi zboru Palisády. Ak sa nič nezmení v~lete plánujeme mať prvý krst.

Sme vďační Bohu, že si všetky služby, s~občasnou externou výpomocou, vieme zabezpečovať z~vlastných ľudských zdrojov. Máme vytvorené platformy pre pravidelné nedeľné bohoslužby, online biblické štúdium, pre stretávania v~domácnostiach, pre mužov, ženy a deti. Tiež máme vlastnú web stránku.

Všetky platformy sa snažíme formovať tak, aby mali nie len vzdelávací, ale aj silný misijný, pastorálny a motivačný rozmer. Tiež sa nám podarilo vybudovať na Súľovskej petangové ihrisko, ktoré rozširuje možnosti využitia areálu nielen pre Connect.

Z~hľadiska celkového smerovania nášho spoločenstva sme sa rozhodli pre 8 pilierov Božej stavby – Connectu, ktoré nám formačne pomáhajú na spoločnej ceste. Sú nimi:
vrúcna zbožnosť, láskyplné vzťahy, inšpiratívne bohoslužby, služba v~intenciách obdarovaní, misia zameraná na potreby, cirkev po domoch, zmocňujúce vedenie, funkčná organizácia a štruktúry.

Začiatkom tohto roka sa staršovstvo zboru Palisády zhodlo s~vedením Connectu, že dozrel čas na to, aby sa Connect stal stanicou zboru. Sme za to vďační a súčasne pripravení urobiť tento krok, ktorý je nevyhnutný pre to, aby sa Connect mohol v~budúcnosti stať samostatným zborom.

Na záver chceme poďakovať za modlitby a všetku podporu, ktorej sa nám v~tejto službe zakladania nového zboru dostáva zo strany zboru.

\autor{Tomáš Valchář}
\vfill\break


\clanok{Revízia hospodárenia}

Revízna komisia v~zložení Miroslav Antalík, Helena Mikletičová, Barbora Antalíková za spolupráce účtovníčky zboru Ľubomíry Kohútovej vykonali revíziu hospodárenia za rok 2022.

Boli prekontrolované nasledovné doklady:
\vskip-1ex\begitems
* výpisy z~bežného účtu vedeného v~Slovenskej sporiteľni za mesiace 2, 3, 6, 8, 10 a 12
* výdavkové pokladničné doklady za mesiace 1 -- 12
* príjmové pokladničné doklady za mesiace 1 -- 12
\enditems
Revízna komisia konštatuje, že uvedené doklady sú vedené prehľadne v~súlade s~účtovnými predpismi. Pokladničná kniha je vedená mesačne a založená priamo pri pokladničných dokladoch.

Neboli zistené žiadne nedostatky.

Stav finančnej hotovosti ku dňu 31.~12.~2022 bol:

\vskip1em\hskip1cm\table{lr}{
pokladňa & 1~775,34~€ \cr
bankový účet & 116~137,90~€ \crl
spolu & 117~913,24~€ \cr
}\vskip1em

Tento stav súhlasí so stavom v~účtovnej evidencii k~uvedenému dátumu.

Revízna komisia chce poďakovať členom a priateľom zboru za obetavosť v~daroch na podporu bratov z~Ukrajiny. Celkovo za rok 2022 sa vyzbieralo 33~668~€. Na daroch sa podielali nadácia Integra, Summit church z~USA, Kompas a suma 14~753 € bola dar od priatelov a členov nášho zboru.

\autor{Miroslav Antalík}

\tiraz
\bye
