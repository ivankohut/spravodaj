%\typosize[10/12.5]% - pouzita velkost pisma/riadku - trochu vacsie
\input makra.tex % nacitanie Ivanom pripravenych nastaveni a prikazov
\hyphenation{star-šov-stvo} % rozdelenie slov na konci riadku, treba tu uviest slova, ktore sam nepozna

\spravodaj{6}{2021}


\clanok {Spoločenstvo ako miesto bezpečia a prijatia}
{\it „Aké milé sú tvoje príbytky, Hospodin zástupov! Moja duša sa umára túžbou po nádvoriach Hospodina; moje srdce i telo plesajú v~ústrety živému Bohu. Veď aj vrabec si nájde domov a lastovička hniezdo, kde si kladie mláďatá pri tvojich oltároch, Hospodin zástupov, môj Kráľ a môj Boh. Blahoslavení tí, čo bývajú v~tvojom dome a ustavične ťa chvália“ (Žalm 84, 2 -- 5).}

{\bf Túžba a radosť.} Dve silné emócie. To je to, čo na prvé prečítanie cítiť zo slov žalmu. Bolesť srdca, keď žalmista nie je v~Božej prítomnosti, jeho túžba po chráme, dome Hospodinovom, a radosť zo samotného Boha sú tak silné, že sa nedajú prehliadnuť. Nie je to iba nejaká hmlistá spomienka na niečo z~dávnej minulosti alebo predstava o~niečom nepoznanom. Oddelenie od chrámu a teda aj od Hospodina prežíva žalmista tak intenzívne, že ho bolí duša. Pre žalmistu je chrám a samotný Boh radosťou, ktorú prežíva emocionálne aj fyzicky. Chrám bol pre neho miestom, na ktorom sa stretával s~Bohom; miestom, na ktorom prežíval fyzické a duševné naplnenie.

{\bf Bezpečie.} Toto je ďalší rozmer chrámu, miesta Božej prítomnosti. Pri oltároch si vtáky stavali svoje hniezda, v~ktorých vychovávali mláďatá. Robili to preto, lebo sa na tomto mieste cítili v~bezpečí. Mali skúsenosť, že na tomto mieste nebudú zranené ani vyháňané preč.

Po období izolácie sa nám postupne „otvárajú dvere“ našej modlitebne. Kostol, modlitebňa pre nás nie je jediné možné miesto stretávania sa s~naším Spasiteľom. My vieme, že sa s~Ním môžeme a máme stretávať všade, kde sme. Je to však miesto, kde sa my môžeme spolu stretnúť, oslavovať Ho, zdieľať sa o~tom, ako pracuje v~našich životoch, a navzájom sa učiť ako žiť pre Neho. Je to miesto, na ktorom môžeme vytvárať spoločenstvo.

Túžime tak ako žalmista po vzájomnom stretnutí a spoločnej oslave Pána Ježiša? Prežívame bezpečie a istotu v~prítomnosti našich bratov a sestier?

Modlím sa o~to, aby v~našich srdciach bola takáto neodbytná túžba, až fyzická potreba po spoločenstve s~ľuďmi, s~ktorými budeme spoločne oslavovať Pána Ježiša Krista za Jeho obeť a očistenie Jeho krvou.

Vytvárajme spoločne miesto, kde sa všetci budeme cítiť bezpečne a prijatí. A~to nie preto, že sme dokonalí, ale preto, že si uvedomujeme omilostenie a očistenie od svojich hriechov.

Nech nám náš nebeský Otec dá milosť byť neustále v~Jeho prítomnosti skrze Ducha Svätého a radosť zo spoločenstva, s~ktorým Ho oslavujeme za očistenie od našich hriechov a záchranu pre večnosť v~Jeho prítomnosti.

S láskou,

\autor{Peter Pribula st.}


\clanok {Správy zo staršovstva}
Vedenie zboru je vecou spolupatričnosti a jednoty, čo sú dary od Pána. Prežívame tieto vzácne Božie dary, a to napriek výzvam tejto doby. Počas mája sme sa stretli dvakrát cez Zoom, čo umožnilo aj Dannyho účasť počas pobytu v~USA. Veľkou témou počas Covidu je starostlivosť o~členov zboru, ktorá prebieha najmä prostredníctvom telefónnych hovorov. Teší nás, že ľudia v~zbore sa navzájom o~seba starajú a zostávame v~kontakte napriek obmedzeniam. Chceme vás vyzvať k~tomu, aby ste tomu venovali ešte viac pozornosti počas leta, či telefonátmi alebo návštevami.

Rozhodli sme sa, že kvôli súčasnej situácii je potrebné zlepšiť internetové pripojenie v~kostole. Firma VNET zabezpečila pripojenie optickým káblom, čo rieši spoľahlivosť vysielania zhromaždení.

Traja bratia sa zúčastnili prostredníctvom Zoom--u Konferencie delegátov zborov BJB. Delegátmi zboru boli J.~Szőllős, M.~Kolářik a V.~Ira. Zakladajúce sa zbory Connect pod vedením T.~Valchářa a ukrajinský zbor pod vedením Viktora Potockého sa začali v~máji stretávať prezenčne v~modlitebni na Palisádach. V~období, keď sa nebolo možné schádzať, sa stretávali cez Zoom.

Tešíme sa spolu s~Viktorom, že úspešne ukončil štúdium a získal titul magister teológie. Staršovstvo schválilo požiadavku Rady BJB ohľadom jeho praxe na Palisádach pod Dannyho dohľadom. Takto bude jeho služba Ukrajincom v~Bratislave ďalej pokračovať.

Naďalej sa venujeme pastoračným otázkam, najmä v~rámci manželstiev a rozvodov členov zboru. Z~tohto dôvodu sme sa rozhodli pravidelne pripravovať recenzie a odporúčať knihy, ktoré sa venujú krízovým situáciám. Prvé dve knihy sú {\it Manželstvo} od Timothyho Kellera a {\it Mal by som ťa radšej, keby si bol trochu viac ako ja} od Johna Ortberga. Odporúčame si tieto dve knihy prečítať a aplikovať ich v~manželstve bez ohľadu na to, či sa manželský pár nachádza v~krízovej situácii alebo nie.

Veľa času sme venovali rozhovorom o~tom, ako pokračovať v~práci v~zbore ovplyvneným Covidom. V~budúcnosti chceme klásť väčší dôraz na budovanie spoločenstva. Plánujeme začať viac zborových skupiniek a každého povzbudiť, aby sa pripojil k~nejakej skupinke. Rovnako plánujeme pravidelné zborové dni a výlety. Samozrejme, všetko záleží od pandemickej situácie, ale veríme, že treba čo najviac času tráviť spolu.

Vnímame, že v~zbore je veľká skupina slobodných, ktorým chceme venovať viac pozornosti.

Svojimi priestormi sme ako zbor veľmi obmedzení, a preto sa modlíme za múdrosť ako čo najlepšie využívať existujúce priestory, aby sme zabezpečili a budovali zdravé vzťahy a spoločenstvo v~zbore. V~nadchádzajúcich mesiacoch budeme o~tom viac hovoriť.

\autor {Danny Jones}


\clanok {Recenzia na knihu {\it Manželstvo} od Timothyho Kellera}
Na základe série kázní Timothy Kellera na text z~Listu Efezanom 5, 22 -- 33 ponúka kniha {\it Manželstvo} dôkladné štúdium základov a vplyv evanjelia na manželstvo. Keller nepíše, ako v~iných knihách o~manželstve, o~ideálnom pohľade na manželstvo, napr. kde nájdeš dokonalého partnera a všetko určite dobre dopadne. Nerealistické očakávania moderného pohľadu na to, aby partneri boli kompatibilní, kladie dôraz na to, aby som našiel správneho partnera a aby boli splnené moje očakávania a potreby. Výsledkom býva sklamanie, ubíjajúci pocit beznádeje a prudký pád manželstva. Je to preto, že nikto nikdy nebude s~tebou kompatibilný. Kellerovci však pripomínajú čitateľom, že napriek tomu, že nám naša kultúra hovorí, aby sme hľadali „toho pravého“, tajomstvom trvalého manželstva nie je kompatibilita, ale služba. Manželia by mali robiť pre svoje manželky to, čo Ježiš urobil, aby nás spojil so sebou. Toto je tajomstvo -- že Ježišovo evanjelium a manželstvo sa navzájom vysvetľujú. Kellerovci sa zaoberajú dôsledkami tejto veľkej pravdy a pozerajú sa najskôr na zmocňujúcu moc Ducha Svätého, ktorá nám pomáha vzdať sa jeden pre druhého. Veľmi oceňujem, že autori citlivo zaobchádzali s~témou sexu v~manželstve. Zatiaľ čo mnohí kresťania majú pokrivený pohľad na sex (buď kvôli zlým učeniam, alebo traumatizujúcim zážitkom), iní idú príliš ďaleko opačným smerom a upadajú do pohľadu hraničiaceho s~pornografiou (človek vidí sex ako niečo hrubé, ako„boha“). Spôsob, akým Kellerovci rámcujú diskusiu, je však mimoriadne užitočný a uspokojivý. Po prvé, pripomínajú čitateľom, že sex je skutočne dar -- nie je sám o~sebe špinavý, zlý ani hriešny. Je to dobrá vec, ktorú vytvoril Boh. Po druhé, sex nie je len súkromná záležitosť v~tom zmysle, že je to všetko o~vašom vlastnom potešení, ale je to zjednocujúci akt, ktorý vám pomôže odovzdať celé svoje ja inej ľudskej bytosti. Ak intimita spočíva v~spoluzjednotení dvoch ľudí na slávu Božiu, potom sa náš pohľad na sex zmení. Kniha Manželstvo je veľmi silný a biblicky verný pohľad na to, čo robí manželstvo trvalým. Je to kniha, ktorú budem medzi prvými odporúčať každému páru alebo jednotlivcovi, ktorý sa snaží lepšie pochopiť Božie zámery v~manželstve a upevniť svoj vzťah k~sláve Božej. Verím, že kniha bude pre vás požehnaním.

\autor {Danny Jones}


\clanok {Besiedka počas mesiaca jún}
Do konca školského roka bude veľká besiedka na Zrínskeho~2 počas bohoslužieb (od 9.30~hod.). Malá besiedka už do konca školského roka nebude.

Prosíme rodičov, aby sa počas besiedky nezdržiavali v~priestoroch na Zrínskeho vzhľadom na bezpečnostné opatrenia.


\clanok {Zborový tábor}
Zborový tábor tento rok plánujeme v~čase od~15. do~21.~augusta v~stredisku Detskej misie v~Častej–Papierničke. Bližšie informácie nájdete v~elektronickej prihláške: \ulink[https://forms.gle/4kGK8pTsuQt5Sr7v7]{forms.gle/4kGK8pTsuQt5Sr7v7}.


\clanok {Rodinný koncert na Palisádach}
Na konci školského roka plánujeme už tradičný rodinný koncert na Palisádach, a to v~nedeľu 20.~júna o~15.00~hod.

Pozývame všetky deti, ktoré sa učia hrať na hudobný nástroj či spievať. Pozývame aj dospelých, aby boli vzorom a pomocou našim deťom.

V prípade akýchkoľvek otázok sa môžete obrátiť na br. Daniela Pletta (tel.: 0903~385~102, \email{thepletts@zoznam.sk}).


\clanok {Krst v~zbore}
Chceme osloviť všetkých, ktorí by sa chceli dať pokrstiť na vyznanie viery, aby sa prihlásili u~br. kazateľa Dannyho Jonesa alebo ktoréhokoľvek staršieho zboru.


\clanok {Pomoc ľuďom v~núdzi}
Ak by ste mali záujem zapojiť sa do služby varenia pre ľudí v~núdzi, v~mesiaci jún sú ešte voľné termíny, na ktoré je možné sa prihlásiť: 17.~6., 22.~6. a 29.~6.

{\bf Potravinová výdajňa} už funguje. Viac informácií nájdete na webovej stránke \ulink[https://www.krestaniavmeste.sk/vydajna/]{krestaniavmeste.sk/vydajna}. Ak poznáte ľudí v~núdzi, ktorým by sa zišla takáto pomoc, neváhajte dať nám o~nich vedieť!

{\bf Zbierka šatstva} sa presunula z~pondelkov na štvrtky od~16.00 do~19.00~hod. Viac informácií nájdete na webovej stránke \ulink[https://www.krestaniavmeste.sk/rozpis/]{krestaniavmeste.sk/rozpis}.

\autor {Beata Bogárová}


\clanok{Verš na zapamätanie}
Tento mesiac máme nový veršík, ktorý sa chceme spoločne učiť. Veríme, že poznanie Písma prospeje našej duši i našej mysli:

{\it „Zavolal k~sebe zástup aj s~učeníkmi a povedal im: „Ak ma niekto chce nasledovať, nech zaprie sám seba, vezme svoj kríž a nasleduje ma. Lebo kto by si chcel zachrániť život, stratí ho. Kto však stratí svoj život pre mňa a pre evanjelium, zachráni si ho.“}

\autor{Mk~8,34-35}


\clanok{Zbierky za uplynulé obdobie}
Milí bratia a sestry,

v máji ste prispeli:

\vskip-1ex\begitems
* Misia: 230,00~€
* Investície: 230,00~€
\enditems

V osobitnej zbierke pre rodinu Sochorovcov, ktorej v~apríli čiastočne vyhorel dom, sa vyzbieralo 3000,--~€. Táto suma bola prevedená na ich účet a slúži na zmiernenie následkov po požiari.

Ďakujeme vám, že napriek okolnostiam a neistým ekonomickým vyhliadkam do budúcnosti, ste mnohí prispeli na činnosť a službu zboru. Aj naďalej máte možnosť prispieť do „nedeľnej zbierky“, a to prevodom na účet zboru. Do poznámky pre prijímateľa, prosím, uveďte „zbierka“.

Bankové spojenie: SK36 0900 0000 0000 1147 1836, SWIFT: GIBASKBX

Ďakujeme!


\n Miriam	KEŠJAROVÁ;
\n Pavel	KOHÚT;
\n Samuel	PLETT;
\n Ján	LAURENČÍK;
\n Ľubica	HOVORKOVÁ;
\n Peter	LICHANEC;
\n Trey	ATKINS;
\n Juraj	KVAČKA;
\n Anna	ŠANDOROVÁ;
\n Kristína	KEŠJAROVÁ;
\n Peter	ŽEMBERY;
\n Marica	ŠČEVLÍKOVÁ;
\n Roman	ŽIARAN;
\n Sylvia	PRIBULOVÁ;
\n Jana	PERKNOVSKÁ;
\narodeniny


\program{
\p  1 ; ut ; 15.15 ; Stretnutie pri Biblii (P. Pivka, Zrínskeho 2) ;.;;
\p  2 ; st ;  6.00 ; Modlitby -- muži (Zoom) ;.;;
\p  3 ; št ; 19.00 ; Biblická hodina (J. Szőllős) ;.;;
\p  4 ; pi ; 17.30 ; Dorast (Súľovská 2) ;.;;
\p  5 ; so ; 18.00 ; Mládež (Súľovská 2) ;.;;
\p  6 ; ne ;  9.30 ; Bohoslužby (J. Szőllős) ;.;;
\p  7 ; po ; 17.30 ; Modlitby -- ženy (Zrínskeho 2) ;.;;
\p  8 ; ut ; 15.15 ; Stretnutie pri Biblii (P. Pivka, Zrínskeho 2) ;.;;
\p  9 ; st ;  6.00 ; Modlitby -- muži (Zoom) ;.;;
\p 10 ; št ; 19.00 ; Biblická hodina (J. Szőllős) ;.;;
\p 11 ; pi ; 17.30 ; Dorast (Súľovská 2) ;.;;
\p 12 ; so ; 18.00 ; Mládež (Súľovská 2);.;;
\p 13 ; ne ;  9.30 ; Bohoslužby (D. Plett) ;.;;
\p 14 ; po ; 17.30 ; Modlitby -- ženy (Zrínskeho 2) ;.;;
\p 15 ; ut ; 15.15 ; Stretnutie pri Biblii (P. Pivka, Zrínskeho 2) ;.;;
\p 16 ; st ;  6.00 ; Modlitby -- muži (Zoom) ;.;;
\p 17 ; št ; 19.00 ; Biblická hodina (J. Szőllős) ;.;;
\p 18 ; pi ; 17.30 ; Dorast (Súľovská 2) ;.;;
\p 19 ; so ; 18.00 ; Mládež (Súľovská 2) ;.;;
\p 20 ; ne ;  9.30 ; Bohoslužby (R. Krupa) ; 15.00 ; Rodinný koncert ;
\p 21 ; po ; 17.30 ; Modlitby -- ženy (Zrínskeho 2) ;.;;
\p 22 ; ut ; 15.15 ; Stretnutie pri Biblii (P. Pivka, Zrínskeho 2) ;.;;
\p 23 ; st ;  6.00 ; Modlitby -- muži (Zoom) ;.;;
\p 24 ; št ; 19.00 ; Biblická hodina (J. Szőllős) ;.;;
\p 25 ; pi ; 17.30 ; Dorast (Súľovská 2) ;.;;
\p 26 ; so ; 18.00 ; Mládež (Súľovská 2) ;.;;
\p 27 ; ne ;  9.30 ; Bohoslužby (P. Kolárovský) ;.;;
\p 28 ; po ; 17.30 ; Modlitby -- ženy (Zrínskeho 2) ;.;;
\p 29 ; ut ; 15.15 ; Stretnutie pri Biblii (P. Pivka, Zrínskeho 2) ;.;;
\p 30 ; st ;  6.00 ; Modlitby -- muži (Zoom) ;.;;
}


\tiraz
\bye
