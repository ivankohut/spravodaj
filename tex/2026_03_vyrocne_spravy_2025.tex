\def\velkostpisma{9}
\def\velkostriadku{12}
\input makra.tex % nacitanie Ivanom pripravenych nastaveni a prikazov
\hyphenation{star-šov-stvo} % rozdelenie slov na konci riadku, treba tu uviest slova, ktore sam nepozna

\vyrocnespravy{2025}


\clanok{Zbor}

\cast{Úvod}

Keď som rozmýšľal nad tým, akým slovom by som vystihol rok~2025 v~našom zbore, tak si myslím, že by to mohlo byť slovo „oživenie“. Vnímal som vo viacerých sférach života nášho zboru po určitom útlme, hľadaní a trápení sa v~predchádzajúcom období obnovenú chuť slúžiť a pracovať na Božom diele aj v~prostredí a skrze prácu nášho zboru. Dôležité je, že zároveň vnímam aj to, že môžeme mať radosť z~práce, že to nie je vynútené trápenie, ale radostná, hoci vo viacerých prípadoch aj veľmi náročná, dobrovoľná služba každého tým darom, ktorý mu Pán zveril.
Vnímal som aj to, k~čomu nás vyzýval aj verš pre náš zbor na rok~2025: „Vyuč ma, Hospodin, svojej ceste a budem žiť podľa Tvojej pravdy. Upriam moju myseľ na bázeň pred Tvojím menom.“ (Ž~86,11), že nás Pán vyučoval svojim cestám, že dôležité bolo a je nie to, aby sme my svojou službou zažiarili, ale aby sme tým ako žijeme boli Jeho svedkami a prinášali slávu Jemu.
Len služba orientovaná na Pána a konaná pre Neho a kvôli Nemu nám v~konečnom dôsledku prináša požehnanie, uspokojenie aj radosť. Božie slovo nás vyzýva, aby sme sa v~takejto službe rozhojňovali a potvrdzuje, že takáto služba nie je márna. „A~tak bratia moji milovaní, buďte pevní, neklátiví, rozhojňujte sa v~diele Pánovom, vediac, že vaša námaha nie je márna v~Pánovi.“ (1K~15,58)

\cast{Najdôležitejšie udalosti v~živote zboru}

Na začiatku uplynulého roku sme stáli pred viacerými dôležitými, najmä personálnymi výzvami. Bol som vtedy jediný kazateľ v~zbore a ešte k~tomu na polovičný úväzok. Boli pred nami voľby správcu zboru, staršovstva a proces hľadania ďalšieho kazateľa. Považujem za veľkú Božiu milosť, ako nás cez tieto dôležité rozhodnutia viedol. Základným a najdôležitejším krokom pre život zboru, bola voľba staršovstva. Voľby do staršovstva, ktoré sa opakujú každé tri roky nám možno pripadajú ako bežná, samozrejmá rutina. Nie je to samozrejmé, že sa podarilo zvoliť piatich bratov, ktorí boli ochotní prijať túto službu. Je to Božia milosť, že daroval do nášho zboru bratov so srdcom pastiera, zakotvených
v~Písme, ochotných slúžiť ako starší zboru. Viacerí z~našich bratov starších slúžia už niekoľko období, kým niektorí mali za sebou jedno funkčné obdobie. Veľmi dobrým rozhodnutím bola aj voľba dvoch bratov za starších v~príprave, čo je nový spôsob, ako pripravovať mladších bratov do náročnej služby staršieho. Voľby nespôsobili prestávku, ale dali nový impulz do práce staršovstva. Som veľmi vďačný Pánovi za bratov starších v~našom zbore, s~ktorými tvoríme nielen dobrý pracovný tím, ale aj skupinku bratov a priateľov, kde nielen riešime veci a rozhodujeme, ale sa spolu modlíme a nesieme bremená.

Určite významným pozitívnym impulzom bolo rozhodnutie vytvoriť pozíciu kazateľského asistenta a zvoliť do nej brata Filipa Barkócziho na plný úväzok od marca. Brat Filip slúžil aj dovtedy ako dobrovoľník v~rôznych zložkách nášho zboru. Jeho služba sa mohla ďalej rozvinúť a skôr je potrebné ho brzdiť, aby neprepínal svoje kapacity a sily. Je nám v~zbore veľkou pomocou a požehnaním a vnímam jeho obdarovanie v~pastoračnej službe. Slúži rôznym vekovým kategóriám od detí, cez dorast a mládež, strednú generáciu až po seniorov.

Ďalším dôležitým personálnym impulzom bolo rozhodnutie prijať do zboru na ročnú kazateľskú prax od 1.~8. brata Dávida Mária Chuchúta. Brat Dávid veľmi rýchlo „zapadol“ do života a služby v~našom zbore a rozhodol sa stať hneď na začiatku praxe aj členom nášho zboru. Vnímam jeho obdarovanie najmä v~oblasti vyučovania, či už na biblických hodinách a v~nedeľných kázňach, ale ochotne slúži aj v~mládeži, doraste a iných oblastiach. Obaja bratia priniesli nové impulzy aj do vzdelávania na biblických hodinách a novým inšpiratívnym prvkom boli aj kontemplatívne bohoslužby v~advente, ktoré obaja spolu s~tímom pripravili. Som vďačný Bohu, že sme vytvorili spolu kazateľský tím, ktorý sa dopĺňa, čo sa prejavilo aj vo vnímaní členov zboru, keď v~modlitbách zaznievala vďačnosť za troch kazateľov zboru. Rád by som v~tejto našej spolupráci pokračoval aj v~ďalších rokoch, ale to, či to bude možné závisí od rozhodnutí, ktoré ako zbor urobíme už v~najbližších týždňoch.

Popri personálnych otázkach považujem za ďalšie „highlighty“ v~živote nášho zboru v~roku~2025 to, že sme mohli uskutočniť dva krsty, z~toho jeden netradične na zborovom tábore, a že sa prehĺbila, najmä vďaka Filipovi a Dávidovi, naša spolupráca so zborovou stanicou Connect. Výborným impulzom pre náš zbor bola aj naša spoločná služba pri príprave a organizácii česko-slovenskej konferencie sestier z~BJB začiatkom mája. Jej úspešné zvládnutie a požehnanie z~toho prijaté prispelo aj k~tomu, že sme sa v~zariadení SÚZA (kde nás vlastne pozvali, aby sme tam urobili ďalšie podujatie) podujali zorganizovať po viac ako desiatich rokoch oblastnú slávnosť vďakyvzdania. Bola to výborná príležitosť byť spolu s~bratmi a sestrami z~ďalších zborov našej Západnej oblasti BJB. Za dôležitú súčasť oživenia života nášho zboru považujem aj rozvoj skupiniek v~našom zbore a to, že po prestávke sa vďaka spontánnej iniciatíve sestier obnovila „zdola“ modlitebná skupinka, kde sa v~pondelky večer modlíme za deti z~našich rodín, ale aj ďalšie modlitebné predmety, ktoré si zdieľame aj cez WhatsApp skupinu.

Po viac ako desiatich rokoch sa pod názvom Rozhovory na Palisádach obnovili aj diskusné stretnutia a v~decembri sa uskutočnilo prvé stretnutie na tému „Umelá inteligencia: Nádej alebo hrozba?“ Chceme v~tomto cykle diskutovať raz mesačne v~soboty dopoludnia, na zaujímavé témy týkajúce sa nášho života s~ľuďmi, odborníkmi, ktorí sú v~danej téme „doma“ a ponúknuť kresťanský pohľad na danú problematiku. Som vďačný za tím zapálených ľudí, ktorí sa zapojili do prípravy týchto stretnutí. Verím, že to bude ďalší spôsob a nástroj nielen na vzdelávanie nás, kresťanov, ale aj na oslovenie a zasiahnutie ľudí okolo nás evanjeliom.

\cast{Štatistika}

Počet členov nášho zboru zaznamenal výraznejšie zmeny ako v~predchádzajúcich rokoch najmä z~dôvodu, že sme mali dva krsty a prijímali sme nových členov zboru z~radov pokrstených a tiež preto, že niektorým roky neaktívnym členom, ktorí nenavštevujú už dlhodobo náš zbor, staršovstvo zboru v~súlade so zborovým poriadkom pozastavilo členstvo. Ku koncu roka~2025 sme mali 153~registrovaných členov, z~toho 11~členov v~zborovej stanici Connect.

Z~Božej milosti sa do rodiny v~našom zbore narodilo v~uplynulom roku jedno dieťa, v~marci sa narodila Vilma Paulen. V~júni (1.~6.) priniesli rodičia predstaviť a požehnať Joshuu Edwarda Smolku, ktorý sa narodil ešte v~roku~2024 a v~auguste (3.~8.) sme prosili o~požehnanie pre Eliota Mizeru, ktorý sa narodil vo februári.

V~roku~2025 sme mali len jeden sobáš v~našom zbore. Na spoločnú cestu životom vykročili 23.~8. Tamara Syčová a Jozef Smutný, ktorých sobášil brat kazateľ Timotej Hanes.

Ako som uviedol predtým, mali sme z~Božej milosti minulý rok dva krsty a celkovo sa krstilo 10~bratov a sestier. Na zborovom tábore v~Častej-Papierničke sme 21.~8. mohli pokrstiť na vyznanie viery úplným ponorením Ninu Bednárikovú, Tomáša Čatlóša, Naomi Dzuriak, Erika Vašinu, Slávku Volentičovú a Lenku Vráblovú. V~modlitebni CASD na Cablkovej v~Bratislave sme 30.~11. krstili Ladislava Baláža, Evu Molnárovú, Michala Vlačuhu a Silviu Markotánovú (zo zboru BJB P.~Biskupice).

Do členstva zboru sme prijali 7~členov, Ninu Bednárikovú, Naomi Dzuriak a Erika Vašinu spomedzi krstencov, Tatianu Havrosh (z~Ukrajiny), Dávida Mária Chuchúta
(z~BJB Banská Bystrica), Jozefa Smutného (z~BJB Revúcka Lehota) a Zuzanu Pařízkovú (z~AC Bratislava). O~ukončenie členstva v~našom zbore požiadali brat Dávid Valchář a sestra Milluška Bažalová (Škulcová), ktorí už roky nežijú v~Bratislave a preto sa nezúčastňujú života zboru.

Staršovstvo zboru pozastavilo členstvo 15~členom, ktorí sa dlhodobo nezapájajú do zborového života a nenavštevujú náš zbor (Peter Buzáš ml., Jana Čahojová, Zdenko Filip, Richard Halamiček, Hana Halamičková, Judita Kobzová, Peter Lichanec, Milica Malá, Matej Matušek, Kristína Matušeková, Alena Svobodová, Marica Ščevlíková, Ladislav Taliga, Elena Taligová, Kamila Zajíčková).
Do nebeskej vlasti si Pán v~uplynulom roku povolal 4~členov nášho zboru.
S~poďakovaním Pánovi za jej život a požehnanie, ktoré sme skrze ňu prijali sme sa v~septembri (23.~9.) rozlúčili so sestrou Alžbetou Betkovou (92), v~Košiciach, kde posledné roky žila, sa v~septembri rozlúčili so sestrou Margitou Elischerovou (95), na Chvojnici vo februári so sestrou Zuzanou Sukupčákovou (70) a na Vianoce zomrela aj najstaršia členka nášho zboru sestra Anna Pavlíková (103), s~ktorou nám však posledné roky nebol umožnený kontakt.

Účasť na hlavných nedeľných bohoslužbách sa v~priebehu uplynulého roku postupne zvyšovala, ale aj kolísala a dosahovala v~priemere 120~--~130 fyzicky prítomných na bohoslužbách a v~nedeľných besiedkach. Zároveň prebieha aj priame online vysielanie, ktoré sledujú najmä členovia, ktorí sa zo zdravotných alebo iných dôvodov nemôžu zúčastňovať našich bohoslužieb, a ľudia z~iných miest na Slovensku aj v~zahraničí.
Evidujeme 33~členov ako vzdialených, ktorí sa dlhodobo z~dôvodu zdravotného stavu, veku, vzdialeného bydliska, alebo z~pastoračných dôvodov dlhodobo nezúčastňujú na bohoslužbách a neparticipujú na živote zboru. Je dôležité, aby všetci vzdialení členovia ostávali predmetom našich modlitieb spolu s~tými členmi našich rodín, ktorí ešte neprijali Krista za svojho Spasiteľa, alebo sa vzdialili od Pána.
Našich nedeľných bohoslužieb aj rôznych iných zborových aktivít sa viac-menej pravidelne zúčastňujú aj viacerí priatelia, ktorí nie sú, alebo sa nechcú stať z~rôznych dôvodov členmi nášho zboru. Nemáme ich presne spočítaných, ale môže ich byť okolo 35~--~40, z~ktorých asi pätina sa aj aktívne zapája do služby v~našom zbore.
Za veľmi dôležité považujem, aby sme prehlbovali svoj záujem o~ľudí, ktorí prichádzajú do nášho spoločenstva ako nepravidelní alebo noví návštevníci a majú záujem o~naše spoločenstvo. Nie je to len úloha uvítacej služby pri dverách, ktorej služba sa, žiaľ, ešte úplne neobnovila v~organizovanej forme, ďakujem bratovi S.~Máťušovi, ktorý ju „ťahá“ s~občasnou pomocou niektorých členov staršovstva. Je to však úlohou v~podstate každého z~nás „domácich“, aby sme prejavovali záujem o~ľudí, ktorí k~nám prídu, aby si mohli u~nás nájsť priateľov, vybudovať kontakty a nájsť v~našom spoločenstve svoju duchovnú rodinu.

\cast{Modlitby}

Veľkým povzbudením pre mňa bolo, že sa v~polovici uplynulého roka obnovili z~iniciatívy sestier pravidelné spoločné modlitby v~pondelok večer. Pôvodná iniciatíva smerovala k~modlitbám za veľké aj malé deti, za mladých ľudí, ktorí vyrastali v~rodinách v~našom zbore a vzdialili sa od Boha aj z~nášho spoločenstva, alebo doteraz nespoznali Pána Ježiša ako svojho Spasiteľa. Postupne sa modlitebné predmety rozširovali aj na ďalšie oblasti nášho zborového života a aj za osobné potreby, najmä za zdravotné problémy a rôzne životné okolnosti nás a našich blízkych.  Spočiatku bola účasť vyššia a naplnili sme kapacitu miestnosti na Zrínskeho a postupne sa vytvoril užší okruh tých, ktorí viac menej prichádzajú fyzicky na stretnutia a širší okruh tých, ktorí sú zapojení do modlitebnej skupiny v~aplikácii WhatsApp. Začiatkom minulého roka sme obnovili aj tlačený modlitebný kalendár nášho zboru, ktorý je pomôckou spájajúcou nás pri našich individuálnych modlitbách. Príležitosťou na spoločné modlitby sú aj spoločné zhromaždenia v~nedeľu, kde sú nepravidelne zaraďované aj špeciálne modlitebné chvíle s~témami, alebo je priestor na reakciu v~modlitbách na počuté Božie Slovo. Som vďačný Bohu, že priestor na modlitbu je aj na rôznych stretnutiach zložiek a skupiniek.

Celý uplynulý rok pokračovali v~našich priestoroch na Zrínskeho ulici aj modlitby (najmä počas schôdzí NR SR) kresťanov z~rôznych spoločenstiev z~Bratislavy a okolia s~poslancami NR SR. Je to reakcia na výzvu Božieho Slova (1Tim~2,2), aby sme sa modlili za tých v~moci postavených a konkrétny spôsob jej realizácie už od roku~2012. Je to jedna z~príležitostí, ako sa konkrétne zapojiť do modlitebného zápasu za našu krajinu. V~súčasnosti bývajú tieto modlitby v~dohodnuté stredy od~7.45 do~8.45~hod.

Prehĺbenie modlitebného života nášho zboru, modlitby v~rodinách a rôznych zborových skupinách je cestou k~duchovnému prebudeniu a oživeniu. Je to veľká Božia milosť a zároveň príležitosť, že môžeme predkladať naše vďaky, chvály aj prosby všemocnému Bohu a môžeme mať skúsenosť, že On viditeľne odpovedá.

\cast{Nedeľné bohoslužby}

Naše spoločné nedeľné dopoludňajšie zhromaždenia sú najviditeľnejšou časťou našej práce a zároveň časom, kde sa môžeme všetci spolu bez rozdielu veku, pohlavia, národnosti, postavenia stretnúť ako zborová rodina pri oslave nášho Pána, načerpať povzbudenie a napomenutie z~Božieho Slova, zo svedectiev, piesní, modlitieb do ďalšej služby a mať požehnanie zo spoločenstva s~ostatnými bratmi a sestrami. Hoci je dôležité, aby sme neboli „nedeľnými kresťanmi“ a žili svoj život ako svedkovia Kristovi aj počas týždňa, nedeľné zhromaždenia majú svoje nezastupiteľné miesto. Snažili sme sa počas týchto zhromaždení v~prvom rade osláviť Pána a naplniť rôznorodé potreby ľudí, ktorí prichádzajú na naše bohoslužby a dať priestor na rôznorodé prejavy zbožnosti. Som vďačný, že niektoré nedeľné rána v~prvú nedeľu v~mesiaci sa opäť obnovila služba spevokolu hoci je to náročné, nakoľko viacerí speváci už prichádzajú špeciálne kvôli službe z~iných spoločenstiev. Služba chválospevových skupín, ktorá nebola na začiatku roka pravidelná z~personálnych a kapacitných dôvodov sa stala pravidelnou vďaka obetavosti tých, ktorí sú do nej zapojení a nasadeniu sestry Diany Dzuriakovej, ktorá prebrala starostlivosť o~vedenie tejto služby.

Nevieme si predstaviť nedeľné bohoslužby bez služby br. Sláva Kráľa pri hudobnom doprovode piesní. Celý rok, až na výnimky, keď nebol prítomný alebo bol chorý, doprevádzal piesne on. Aj pri tejto službe však je potrebné hľadať ďalších obdarovaných ľudí, ktorí by mohli spolu niesť bremeno tejto dôležitej služby.

Veľmi dôležitou je aj služba moderátorov, ktorí svojou službou nielen koordinujú a pripravujú naše stretnutia, ale aj nás privádzajú do Božej prítomnosti ako spoločenstvo, prepájajú jednotlivé časti zhromaždenia a dávajú konkrétnym bohoslužbám špecifický charakter a „vôňu“.
Je veľkou Božou milosťou aj to, že máme viacerých bratov a sestry, ktorí majú obdarovanie vyučovať dospelých aj deti a slúžiť kázaním Božieho Slova. Kázanie a výklad Božieho Slova je ústrednou časťou našich nedeľných bohoslužieb.
V~nedeľu dopoludnia sa na ňom podieľajú kazatelia s~teologickým vzdelaním a ochotní slúžiť týmto spôsobom sú aj bratia starší. Záujem o~službu Slovom v~našom zbore je aj zo strany hostí, čo nás teší.

Božím Slovom som slúžil minulý rok podľa dohody väčšinou prvú nedeľu v~mesiaci, keď bola väčšinou (až na mesiac máj, keď bola prvú nedeľu sesterská konferencia) aj slávnosť pamiatky Večere Pánovej. Jedine v~novembri ma pri tejto službe zastúpil brat kazateľ Viktor Potockij z~ukrajinského zboru Nádej, ktorý slúžil Slovom aj vo februári aj v~máji.

Od februára slúžil pravidelne raz v~mesiaci v~nedeľu aj kazateľský asistent brat Filip Barkóczi a od augusta aj kazateľský praktikant brat Dávid Mário Chuchút. Zvyšné nedele slúžili bratia zo zboru a pozvaní hostia. Z~ordinovaných kazateľov, členov a priateľov nášho zboru nám tri nedele v~januári, v~júni a na Vianoce poslúžil brat kazateľ Stanislav Baláž. Rovnako tri nedele slúžil aj brat kazateľ T.~Valchář. Dve nedele kázal Božie slovo aj br.~D.~Uhrin.
Z~ďalších kazateľov BJB slúžil v~marci brat M.~Mišinec zo zboru BJB Viera a v~septembri nám bol poslúžiť a predstaviť víziu svojej služby v~rámci možnej kandidatúry za ďalšieho kazateľa nášho zboru brat Timotej Hanes zo zboru BJB Revúcka Lehota. Špeciálnu službu mal v~prvú októbrovú nedeľu tím zo vznikajúceho zboru BJB v~Revúcej na čele s~bratom kazateľom R.~Nagypálom, ktorí nám predstavili svoju službu a perspektívu kúpy objektu bývalej synagógy pre svoje stretávanie. Sme radi, že sme mohli pomôcť modlitbami aj finančným príspevkom na úspešnú realizáciu kúpy modlitebne v~Revúcej.

Z~členov zboru slúžili viackrát bratia Peter Pribula, Rado Nemec a Peter Kolárovský. Teším sa, že sa do tejto služby zapojili aj ďalší bratia: Mirek Ira, Ľubo Syč a Slávo Kráľ. Z~medzinárodných návštev sa uskutočnila v~novembri návšteva a služba brata kazateľa Kálmána Mészárosa z~Budapešti, profesora cirkevnej histórie a rektora Baptistickej teologickej akadémie, ktorý mal druhú časť svojej prednášky o~histórii baptistov.

Veľkým prínosom je, že naše bohoslužby sú vysielané priamo cez internet a sú tam aj archivované. Patrí za to veľká vďaka našim technikom, ktorí prenosy a archiváciu zabezpečujú. Naše bohoslužby sledujú týmto spôsobom nielen naši členovia, ktorí nemôžu byť prítomní na bohoslužbách, ale ohlasy, že ich sledujú a prijímajú požehnanie z~tejto služby, máme aj od bratov a sestier z~iných miest našej krajiny a aj zo zahraničia.

\cast{Vzťahy so zbormi v~západnej oblasti BJB a ostatnými cirkvami a medzinárodná spolupráca}

Hneď začiatkom roku sme sa už tradične ako zbor zapojili do Aliančného modlitebného týždňa. Na stretnutí vo štvrtok u~nás som slúžil úvodom k~modlitbám ja.

Tradičná Záhradná slávnosť v~zbore BJB Podunajské Biskupice v~júni mala tohto roku ešte slávnostnejší charakter ako po iné roky, pretože sme spoločne s~bratmi a sestrami zo zborov BJB v~našej oblasti ďakovali za požehnanie, ktoré sme mohli prijať za 70~rokov existencie biskupického zboru.

Ako som uviedol v~úvode, podarilo sa nám po vyše desiatich rokoch zorganizovať v~októbri spoločnú slávnosť vďakyvzdania zborov Západnej oblasti BJB v~zariadení SÚZA v~Bratislave. Do vďakyvdania sa zapojili zbory BJB~Palisády, P.~Biskupice, Miloslavov a Nádej spôsobom, že zrušili svoje zhromaždenia a zbory Bernolákovo a Viera sa pripojili účasťou jednotlivcov a službou svojich kazateľov. Pri chválach nás viedli speváci zo zborov Palisády a Viera. Pripravený bol aj špeciálny program pre deti (divadielko) v~malej sále. Kapacitu sály 300~miest sme zaplnili a využili sme aj priestranný vestibul na spoločenstvo pri rozhovoroch. Ohlasy viacerých zúčastnených boli veľmi priaznivé a v~spoločných oblastných stretnutiach by sme chceli pokračovať.

V~uplynulom roku som sa zúčastnil ako zástupca BJB v~januári Ekumenických bohoslužieb v~rámci Týždňa modlitieb za jednotu kresťanov, ktoré sa konali v~zbore ECAV Bratislava-Dúbravka. Bol som pozvaný slúžiť Slovom v~marci do zboru AC Nesvady. V~prvú novembrovú nedeľu som mohol slúžiť Slovom aj predstavením histórie a práce nášho časopisu Rozsievač v~zbore BJB Praha-Vinohrady. Počas veľkonočných a vianočných sviatkov som slúžil na misijnej stanici Čučma zboru BJB Revúcka Lehota a na Veľkonočnú nedeľu v~centre zboru v~Jelšave.

Tak ako v~predchádzajúcich rokoch pokračovala v~roku~2025 spolupráca so zbormi v~rámci platformy Kresťania v~meste najmä pri varení polievky bezdomovcom. Dobrovoľníci z~nášho zboru pokrývajú značnú časť tejto služby, ktorá sa koná 2~--~3-krát do týždňa a varia niekoľkokrát do mesiaca, za čo patrí vďaka Slávke Volentičovej za organizáciu varenia a všetkým, čo sa zapájajú a aj financujú túto veľmi potrebnú službu. Začiatkom júna (6.~6.) sme sa niekoľkí jednotlivci z~nášho zboru zapojili spolu s~ostatnými kresťanmi z~nášho mesta do Pochodu pre Ježiša. Napriek nepriaznivému počasiu sa pochodu ulicami centra mesta z~Petržalky zo Sadu Janka Kráľa na Hviezdoslavovo námestie zúčastnili stovky ľudí a pochod bol zavŕšený koncertom a evanjelizáciou.

\cast{Služba zborových zložiek a život zboru}

Som veľmi vďačný Pánovi za obetavú službu viacerých bratov a sestier v~rôznych zložkách nášho zboru počnúc od besiedok, dorastu, mládeže, cez diakoniu, hospodársky výbor až po spevokol, či skupinky, ktoré dotvárajú obraz o~živote nášho zboru. O~ich službe nájdete informáciu v~správach za jednotlivé zložky, tu uvediem len niektoré vybrané udalosti. Takou vlajkovou loďou nášho zboru, ktorou je známy najmä v~kresťanských kruhoch v~našom meste je náš spevokol a vianočné a veľkonočné koncerty, ktorými prispieva k~oslavám týchto sviatkov v~našom zbore aj meste. Aj minulý rok mohli byť opäť dva veľkonočné a dva vianočné koncerty. Teší ma aj to, že sa zapojili noví speváci. Dôležité je aj to, že pokračovala spolupráca začatá v~roku~2024 so zborom ECAV v~Petržalke, keď nedeľný koncert z~dvoch na oba sviatky sa môže konať v~chráme ECAV v~Petržalke, čo pomáha riešiť nedostatočnú kapacitu nášho kostola a pokryť záujem o~tieto koncerty. Súčasťou našej spolupráce bola aj v~roku~2025 služba nášho spevokolu na Novoročnom koncerte.

Rodinný koncert sa minulý rok nepodarilo z~kapacitných dôvodov organizátorov uskutočniť, čo je mi ľúto.

K~utuženiu nášho zborového života prispievajú aj spoločné obedy, ktoré organizujú diakoni a uskutočnili sa v~septembri v~hoteli Plus, keď predstavil víziu svojej služby v~prípade zvolenia za kazateľa nášho zboru brat kazateľ Timotej Hanes a tiež obed seniorov začiatkom decembra.

Nepodarilo sa nám z~dôvodu nedostatku vhodných termínov zorganizovať minulý rok ani zborovú víkendovku, ktorá už tiež našla v~uplynulých rokoch svoje miesto v~živote nášho zboru. Svojím spôsobom zborovou víkendovkou bol víkend na Letnice na chalupe na Chvojnici, ktorý zorganizovali rodiny a jednotlivci zo zboru a v~rámci neho bolo po dlhom čase aj zhromaždenie na Chvojnici. Možno to bude nový spôsob ako pokračovať v~tradícii výletov zboru na Letnice na Chvojnicu.

Aj v~minulom roku sme organizovali v~zbore počas jarných a letných prázdnin viaceré tábory a pobyty. Počas jarných prázdnin bola vďaka organizačnému nasadeniu manželov Petra a Barbi Antalíkovcom opäť zborová lyžovačka v~Račkovej doline. Je to výborný neformálny čas na rozhovory, oddych a vzájomné spoločenstvo. Tím obetavých spolupracovníkov a vedúcich pripravil opäť aj dorastenecko-mládežnícky tábor v~júli v~Novej Lehote, ktorého sa zúčastnili aj mladí ľudia nielen z~rodín z~nášho zboru. Jeho atmosféra a požehnanie bolo také intenzívne, že niektorí vedúci (R.~Nemec, M.~Hovorková, F.~Barkóczi) sa rozhodli ponúknuť záujemcom začiatkom augusta ešte pár dní neformálneho pobytu (bez vopred organizovaného programu). Zúčastnilo sa ho asi 15~účastníkov a som vďačný, že jeden deň som mohol aj ja, spolu s~D.~Chuchútom zažiť takýto nový spôsob (ne)organizovaného pobytu. Som veľmi rád, že so staro-novým organizačným tímom a s~novým elánom sa po ročnej prestávke obnovila (pokračovala) tradícia zborových rodinných táborov. Špecifikom bola okrem výborne zvládnutej organizácii vďaka manželom Dzuriakovcom a Kešjarovcom aj vysoká účasť, keď kapacita strediska v~Častej-Píle bola naplnená. Prvýkrát sme v~rámci tábora uskutočnili aj krst, čo bolo veľkým požehnaním, a možno by sme mohli založiť aj novú tradíciu a robiť krsty na tábore.
Som veľmi vďačný tým menovaným aj nemenovaným, ktorí už roky organizujú tieto tábory, či pobyty, ale aj tým, ktorí sa zapojili len nedávno.

V~minulom roku sa udiali aj jednorazové podujatia, ktoré však boli návratom v~inovovanej forme po rokoch k~niečomu, čo sme v~našom zbore robili. Spomeniem jedno z~nich, a to Noc kostolov, do ktorej sme sa opäť zapojili. Nerobili sme špeciálny program, len sme otvorili náš kostol a rozprávali sme sa s~ľuďmi, ktorí sa k~nám prišli pozrieť.

Celý minulý rok sme mali možnosť parkovať v~areáli Strednej elektrotechnickej školy a som vďačný, že v~podstate bez väčších problémov fungovala aj služba pri otváraní a zatváraní parkoviska. Postupne sa aj tento priestor stal miestom, kde prebiehali, či pokračovali vzájomné rozhovory a utužovalo sa naše spoločenstvo. Sme vďační škole, že nám tieto priestory zadarmo poskytovali celé roky až do konca roku~2025. Nakoľko sa hneď začiatkom januára~2026 začala prestavba parkoviska na športoviská, prišli sme o~možnosť využívať tento priestor. Modlím sa, aby táto komplikácia s~parkovaním nespôsobila zníženie účasti na našich bohoslužbách a stretnutiach.

Z~kapacitných dôvodov firmy, ktorú sme vybrali na realizáciu, sa minulý rok nezačala plánovaná rekonštrukcia fasády nášho kostola, ale v~závere roka sa podarilo podpísať s~firmou zmluvu a rekonštrukcia by mala začať 1.~4.~2026.

V~lete sme presťahovali aj zborovú kanceláriu z~priestorov na prízemí, ktoré potrebujú rekonštrukciu, do pôvodného priestoru na poschodí. Ďakujem aj tým bratom a sestrám a dorastencom a mládežníkom, ktorí sa zapojili aj do brigády pri tomto sťahovaní.

Vytvorili sme a dobre sa osvedčil administratívno-správcovský tím v~zložení F.~Barkóczi, D.~M.~Chuchút, K.~Kerekréty a ja. Stretnutia v~perióde raz mesačne a komunikácia cez elektronické prostriedky sa ukázali ako dostatočné pre operatívne riešenie otázok spojených s~prevádzkou zboru.

\cast{Záver}

Možno si spomínate, že na začiatku tohto mesiaca sme sa zamýšľali nad tým, nakoľko je pri nás viditeľné Božie dielo (Ž~90,16-17), či by sme niekomu v~tomto meste chýbali, keby sme prestali ako zbor existovať. Považujem za dôležité nezabúdať na to, že všetko, čo konáme, je Božie dielo, ktoré On koná skrze nás, a je to na Jeho slávu a nie na našu. Ak je na nás zrejmé Božie dielo, tak to nepôsobíme my sami, ale Boh sám zo svojej milosti. Sme služobníci Boží, ktorí z~Jeho milosti boli povolaní a vyvolení, aby sme sa pripojili a pracovali na Jeho diele. Tieto naše výročné správy sú pokusom o~to, aby sme uvideli a uvedomili si, čo Boh vykonal. Jedine Jemu patrí za to sláva a naša vďaka.

Čakajú nás dôležité rozhodnutia. Už bezprostredne na výročnom zborovom členskom zhromaždení sa začnú voľby ďalšieho kazateľa. Je to Božia milosť, že máme až troch výborných kandidátov: Filipa Barkócziho, Timoteja Hanesa a Dávida Mária Chuchúta. Je ťažké sa rozhodnúť koho si zvoliť, ale teraz môžeme zvoliť len jedného. Toto rozhodnutie výrazne ovplyvní život nášho zboru. Chceme sa rozhodovať podľa Božej vôle. Tiež nás čaká rekonštrukcia fasády našej modlitebne, ktorú je tiež potrebné na modlitbách pripraviť a sprevádzať. Je dôležité predkladať tieto veci Pánovi na modlitbách a pravidelne sa modliť za rast Pánovho diela v~našom zbore.

Na tento rok~2026 máme ako zbor verš z~Božieho Slova z~Iz~43,10-11: „Vy ste moji svedkovia, znie výrok Hospodinov a moji sluhovia, ktorých som si vyvolil, aby ste poznali a verili mi, aby ste pochopili, že som to ja. Predo mnou nebol utvorený Boh a ani po mne nebude. Ja, ja som Hospodin a okrem mňa nieto Spasiteľa.“ Utvrdzuje nás v~tom, že sme ako spoločenstvo vyvolenými svedkami živého Boha, Pána Ježiša Krista v~našom meste. Je to výzva, aby sme aj v~tomto roku spoznávali suverénneho Boha novými spôsobmi, dôverovali Mu, že On bude aj v~tomto roku pokračovať vo svojom diele záchrany ľudí. Všetko, čo konáme, má byť svedectvom o~tom, že jedine On je Záchranca-Spasiteľ.

Ak v~našom osobnom živote, ak v~zbore, v~cirkvi nahradíme osobu Pána Ježiša niekým alebo niečím iným pri motívoch aj hodnotení toho, čo robíme, ak postavíme do centra niekoho alebo niečo iné, tak to nevedie k~želateľnému výsledku, teda k~oslave Boha a požehnaniu pre ľudí. Také dielo, ktoré konáme z~lásky k~Pánovi a s~láskou k~ľuďom, Boh potvrdzuje a priznáva sa k~tomu a upevňuje a utvrdzuje dielo našich rúk. Spoločenstvo, ktoré koná takéto dielo, by určite chýbalo, keby prestalo existovať a okolie by si to určite všimlo.

Mojou túžbou a prosbou k~Pánovi je, aby sme boli takými jednotlivcami a takým spoločenstvom, kde je zrejmá Božia láska v~našich vzťahoch, v~našej službe, v~našom živote aj v~našich slovách.
\autor{Ján Szőllős}
\vfill\break


\clanok{Staršovstvo}
Staršovstvo dostalo pre rok~2025 slovo zo Žalmu~91: „Kto býva v~úkryte Najvyššieho, odpočíva v~tôni Všemohúceho, nech povie Hospodinovi: ‚Moje útočisko a moja pevnosť je môj Boh, v~ktorého dúfam.‘“

Rok~2025 vnímam ako obdobie, kedy sme sa mohli cítiť ako v~Božej pevnosti alebo Jeho útočisku. Je to milosť, ktorú sme od Neho dostali, že sme mohli žiť a pracovať v~Jeho tôni.

V~roku~2025 sme pracovali v~zložení: kazateľ zboru Ján Szőllős, asistent kazateľa Filip Barkóczi, členovia staršovstva Peter Antalík, Marcel Maďar, Radislav Nemec, Peter Pribula, Ľubomír Syč a starší v~zácviku Ján Kováčik a Martin Simon.

Minulý rok sme začali učenícky projekt „Starší v~zácviku“. Tento model nám na základe Nového Zákona predstavil br.~kazateľ Timotej Hanes. Jedná sa o~zácvik mladších bratov do služby v~staršovstve. Okrem ich aktívnej účasti na stretnutiach staršovstva sme všetci spolu študovali biblický pohľad na službu starších. Spätný pohľad na rok, ktorý je za nami v~tomto projekte so staršími
v~zácviku, ukazuje, že to bolo dobré rozhodnutie. Obaja bratia sa stali aktívnymi a užitočnými členmi tímu.

V~minulom roku sme organizovali konferenciu sestier. Aj keď celý zbor musel vynaložiť veľa práce a úsilia, to, čo vo mne zostalo, je prijaté požehnanie z~celej akcie. Pán Boh nám znova ukázal, že zjednotenie sa v~službe pre Neho a pre bratov a sestry nás spája a prináša nám všetkým Jeho požehnanie.

Po skúsenosti s~konferenciou sestier sme sa rozhodli pozvať bratov a sestry z~okolitých zborov na oblastné stretnutie pri príležitosti vďakyvzdania.

Ktosi prišiel s~nápadom, aby sme si v~kruhu zborovej rodiny pripomenuli niekoľko 50.~výročí manželstva. Áno, tešili sme sa spolu s~našimi oslávencami, ale tá najdôležitejšia správa, ktorá z~toho vzišla, je vďaka nášmu Nebeskému Otcovi za milosť, ktorú naši oslávenci prežívali v~ich manželstvách. Napriek tomu, že aj oni prežívajú turbulencie vo vzťahoch, vedia si odpúšťať a vždy znova sa navzájom prijímať.

V~rámci procesu hľadania ďalšieho kazateľa sa ako kandidát prišiel v~septembri predstaviť br.~kazateľ Timotej Hanes. Strávili sme spolu nedeľu nielen na bohoslužbách, ale aj pri spoločnom obede a následnej diskusii.

Robili sme niekoľko finančných zbierok pre potreby ľudí postihnutých konfliktami a prírodnými katastrofami v~rôznych častiach sveta. Je to veľká Božia milosť, že nám dáva ochotné srdcia deliť sa s~tým, čo nám požehnal. A~Bohu patrí vďaka aj za to, že mení našu situáciu z~prijímania darov a podpory na dávanie darov a podporu iných. Vďaka mu za to.

V~roku~2025 sme plánovali opravu fasády kostola. Pán Boh to zmenil a toto dielo nám ostalo do ďalšieho obdobia. Ďakujem bratom v~hospodárskom výbore za ich snahu a prácu. Vám, nám všetkým chcem poďakovať za to, že prísľuby z~roku~2024 sme zmenili na finančné príspevky na jej opravu. V~téme financovania opravy fasády nám Pán Boh ukázal jedny dvere. Netušíme, či nám ich otvorí, ale „zaklopali“ sme na ne. Ukázal nám možnosť získať financie z~fondu „Obnovme si svoj dom~2026“. Modlíme sa za to, aby Pán Boh otvoril tieto dvere, ak je to Jeho vôľa.

Naďalej zostáva jednou z~našich hlavných tém starostlivosť o~zverený zbor. Modlíme sa za vás a aj takýmto spôsobom chceme vyjadriť vám aj Bohu vďaku za vašu prácu a podporu pri nesení evanjelia v~našom meste.
\autor{Peter Pribula}


\clanok{Diakonia}

V~uplynulom roku mala diakonia nášho zboru dve veľké pracovné stretnutia všetkých členov, a to 24.~3.~2025 a 14.~10.~2025. Medzi týmito stretnutiami boli aj malé pracovné stretnutia, a to 27.~3.~2025, 23.~4.~2025, 17.~5.~2025, 22.~9.~2025 a 18.~11.~2025.
Predmetom stretnutí bola príprava a následne aj zabezpečovanie a vykonávanie nasledovných činnosti:
\begitems
* materiálne a personálne zabezpečovanie pravidelného mesačného vysluhovania Večere Pánovej;
* organizovanie a zabezpečenie spoločných zborových obedov;
* pravidelná návšteva seniorov, chorých a imobilných členov zboru v~ich domácnostiach, a zariadeniach pre seniorov s~cieľom zistenia ich potrieb a budovania a udržiavania prirodzených bratsko-sesterských vzťahov v~niektorých prípadoch spojená aj vysluhovaním Večere Pánovej;
* kontaktovanie členov zboru, ktorí už dlhší čas zo zdravotných, ale aj neznámych dôvodov nenavštevujú pravidelné zhromaždenia s~cieľom zistenia príčin a následne zaradenia do predmetov modlitieb.
\enditems

V~roku~2025 sa vykonávali pravidelne utorkové biblické hodiny pre seniorov v~našom cirkevnom zborovom dome na Zrínskeho~2 pod vedením brata kazateľa Pavla Pivku a v~spolupráci s~kazateľským asistentom bratom Filipom Barkóczim. Preberalo sa Evanjelium podľa Jána.

Pokračovala pravidelná návštevná služba v~domovoch pre seniorov (Stredisko evanjelickej diakonie, Partizánska~2; Betánia, Partizánska~6; Hestia, Bošániho~2; Dúbravská oáza pokoja a~oddychu, Plachého~1D). Návštevu spojenú s~výkladom Božieho Slova, niekedy spevom a vysluhovaním Večere Pánovej vykonával brat kazateľ Pavel Pivka za účasti členov diakonie (Elenka Gubová, Vlaďka Laurenčíková, Filip Barkóczi, Ján Štefko) najmä pre členov nášho cirkevného zboru, ale na požiadanie aj pre iných klientov týchto zariadení na ich izbách.

V~súvislosti s~vyjadrovaním úprimnej vďaky a úcty seniorom boli pri príležitosti ich narodenín vykonávané gratulácie, a to počas nedeľných bohoslužieb a návštevou na adrese, kde v~súčasnosti bývajú. Takto sme boli zablahoželať aj teraz najstaršej členke nášho cirkevného zboru sestre Alžbete Dudášovej, ktorá sa dňa 12.~1.~2025 dožila deväťdesiat rokov.

Pre budovanie spoločenstva a vzťahov medzi seniormi cirkevných zborov BJB v~Bratislave, Bernolákove a Miloslavove, vrátane vytvorenia priestoru pre rozhovory, bol dňa 7.~12.~2025 zorganizovaný  pravidelný vianočný obed v~reštaurácii HOTELY Plus, a.s., Bulharská~72. Organizačne a materiálne (darčeky) verne túto službu zabezpečili najmä Katarína Kráľová, Vlasta Šalingová, Vlaďka Laurenčíková a Ján Štefko.

Pre zabezpečenie diskusie s~možným kandidátom za kazateľa nášho cirkevného zboru, bratom kazateľom Timotejom Hanesom, bol dňa 21.~9.~2025 zorganizovaný obed pre členov nášho cirkevného zboru.

Neoddeliteľnou službou diakonie bolo organizačné zabezpečovanie mesačného vysluhovania Večere Pánovej. Jej materiálne zabezpečenie (chlieb, víno) a prípravu stolovania verne vykonávali manželia Miroslav a Štefánia Antalikovci a manželia Dávid a Barbora Pribulovci. Za túto službu im patrí veľká vďaka.

Koncom roka bola organizačne zabezpečená finančná pomoc jednotlivcom a rodinám členov nášho cirkevného zboru s~cieľom zmierniť ich sociálnu alebo finančnú situáciu počas vianočného obdobia a spríjemniť im sviatky.

Touto cestou diakonia nášho cirkevného zboru ďakuje všetkým členom diakonie za aktívnu a obetavú prácu v~roku~2025 a zároveň aj staršovstvu cirkevného zboru za spoluprácu a podporu.
\autor{Ján Štefko}


\clanok{Hospodársky výbor}

V~zložení br. Šrankota, Kešjar, Syč, Štefko, Maďar, Mikletič a po vyprosení požehnania od nášho Pána veršom z~Písma (J~15,16), ktorý nás viedol rokom~2025, sme započali prácu hospodárskeho výboru nášho zboru.

Po rekapitulácii uplynulého obdobia sme presunuli nezrealizované práce a projekty do roku~2025. Za prioritu sme považovali rekonštrukciu fasády kostola. Brat Ľ.~Syč počas celého roku priebežne informoval o~výbere firmy a o~jednaní s~pamiatkovým úradom, od ktorého bolo potrebné získať súhlas k~realizácii opravy fasády. Po celoročnom snažení a prekonaní mnohých prekážok sa br. Syčovi podarilo dopracovať k~podpisu Zmluvy o~dielo. Patrí mu od nás veľká vďaka za odvedenú prácu. Termín uskutočnenia prác bol dohodnutý od~1.~4.~2026 do~31.~7.~2026. Uvedomujeme si, že ak Pán dá, tak sa dopracujeme k~zdarnému koncu. Nesme toto dielo na modlitbách.

V~priestoroch na Zrínskeho boli vykonané potrebné práce v~byte kazateľa po odchode Šrankotovcov. Byt bol pripravený k~ubytovaniu kazateľského praktikanta br. Chuchúta. Kancelária zboru bola presťahovaná zo suterénu na prvé poschodie. Priestory v~suteréne čaká rekonštrukcia.

Po odchode br.~Šrankotu prejavil záujem pracovať v~hospodárskom výbore br.~J.~Kerekréty. Radi sme ho prijali medzi seba. Každá ochotná ruka je potrebná.

V~kostole boli obnovené nátery znečistených soklov. Chcel by som poďakovať za prácu zvukárov pod vedením br. Kešjara. Vďaku si zaslúži aj upratovací servis priestorov na Zrínskeho a v~kostole.

Chalupa a kostolík na Chvojnici sú udržiavané v~prevádzky-schopnom stave. Na zborovej chalupe bolo počas uplynulého roku 23~rekreačných pobytov, 10~zborových brigád, zborový výlet na svätodušné sviatky. Dary od rekreantov pokrývajú prevádzku chalupy. V~budúcnosti chceme vybudovať čištičku odpadovej vody.

Pre rok~2026 nám bol daný veršík z~Písma Ž~112,5-7. S~bázňou pred naším Pánom pristupujeme k~realizácii projektov, ktoré sú pred nami. Ak Pán požehná, tak naša práca bude zdarná.
\autor{Daniel Mikletič}


\clanok{Biblické a iné vzdelávanie}
Správa o~biblickom a inom vzdelávaní obsahuje len vzdelávanie formou stretnutí na biblických hodinách organizovaných vo štvrtok večer na Zrínskeho (Palisádach). Ostatné formy biblického vzdelávania organizované jednotlivými zložkami zboru môžu byť zahrnuté v~správach za zložky, alebo v~správe kazateľa zboru. V~tejto správe nie sú zahrnuté ani vzdelávacie aktivity na nadzborovej úrovni, ktorých sa zúčastnili členovia nášho zboru a ani vzdelávanie v~skupinkách.

Spoločné štúdium Svätého Písma prebiehalo viac-menej pravidelne v~týždennej periodicite, okrem prázdninovej prestávky, ktorá trvala od začiatku júla až do konca septembra. Pravidelnosť bola občas narušená, keď som kvôli pracovným povinnostiam bol nútený stretnutie zrušiť. Online vzdelávanie vzhľadom na charakter nášho vzdelávania na biblických hodinách a zloženie účastníkov a aj moje kapacitné možnosti sme nezrealizovali.

Počas prvej polovice uplynulého roku od januára až do začiatku apríla sme dokončili štúdium knihy Ozeáš od 7.~kapitoly až po jej koniec (14.~kapitola). Od apríla až po jún sme prebrali celú knihu proroka Jóela. Vzdelávanie sa konalo vo štvrtky v~priestoroch na Zrínskeho~2. Po prázdninách prevzali vedenie biblických hodín kazateľský asistent Filip Barkóczi a kazateľský praktikant Dávid Chuchút. Od začiatku októbra až po koniec novembra preberali knihu Kazateľ (Filip) a list Efezským (Dávid). V~preberaní týchto kníh pokračujú aj po vianočných prázdninách. Cez adventný čas pripravovali Filip s~Dávidom kontemplatívne bohoslužby na Palisádach, ktoré sa stretli s~priaznivým ohlasom, a preto budú aj v~pôstnom období pred Veľkou nocou. Myslím si, že štúdium tak kníh prorokov Ozeáš a Jóel do prázdnin ako aj knihy Kazateľ a listu Efezským prinieslo a prináša nielen nové vedomosti a objavy, ale prežívame pri tom aj živý dotyk Božieho Slova, ktoré hovorí do našej situácie a životov. Výmena pri vedení biblických hodín zvýšila účasť na biblickom vzdelávaní, ktorá v~niektorých prípadoch dosahovala kapacitu miestnosti na Zrínskeho (20 osôb).
\autor{Ján Szőllős}


\clanok{Sestry}
Sme Kristovou vôňou

Rok~2025 sa líšil od ostatných rokov najmä tým, že sme prijali úlohu usporiadať Konferenciu sestier BJB v~Českej a Slovenskej republike. Hoci konferencia sa mala konať v~prvý májový víkend, stretli sme sa v~užšom kruhu s~Petrom Pribulom už v~januári, aby sme si ozrejmili, čo všetko s~organizáciou súvisí. Boli sme si vedomé, že potrebujeme nielen aktívnu účasť sestier, ale aj pomoc bratov, a predovšetkým Božie vedenie a múdrosť pri výbere miesta, ubytovania, výzdoby, občerstvenia, darčekov, či uvítacej služby. Verím, že všetci, ktorí sme sa za konferenciu modlili a mali sme možnosť sa jej zúčastniť, sme boli za vynaloženú námahu bohato odmenení radosťou z~Božej prítomnosti a požehnania. Témou konferencie bolo: „Sme Kristovou vôňou“ (2K~2,15-16).

Počas roka sme sa stretávali raz mesačne na Zrínskeho. V~januári medzi nás zavítala sestra farárka Anička Činčuráková, ktorej vyučovanie o~zdržanlivosti sa nás hlboko dotklo. Sústredila sa najmä na vyjadrovanie sa o~ľuďoch v~ich neprítomnosti, čo je jeden z~hriechov, ktorého sa často dopúšťame a neuvedomujeme si všetky jeho dôsledky. Nielen že ničí vzťahy, ale môže byť aj dôvodom, prečo nerastieme a prečo medzi nás neprichádzajú noví ľudia. Ako nás upozornila sestra Anička, keď poukazujeme na zlyhanie svojej sestry (či brata) a nie na to, že je milovaná, vraciame ju späť do špiny.

Vo februári sme sa stretli v~krásnych priestoroch diakonie na Partizánskej, kde si pre nás Beátka Dobová pripravila skvelú tému: „Kufor plný pokladov: Čo si zbaliť na cestu do staroby?“ Bolo to pre nás nielen zaujímavé, ale hlavne mimoriadne užitočné.

Na stretnutí v~marci sme boli veľmi povzbudené Dáškou Leeder, ktorá sa s~nami zamýšľala o~„Milosti a nádeji v~pravý čas“ na základe príbehu Naomi a Rút.

V~marci sa viaceré sestry z~nášho zboru zúčastnili aj ekumenického Svetového dňa modlitieb v~hradnej kaplnke. Spoločne sme sa modlili za Cookove ostrovy, a stali sme sa tak súčasťou modlitebnej reťaze obopínajúcej celú našu Zem. Bohoslužbu pripravili kresťanské ženy z~Cookových ostrovov na základe 139.~žalmu. Pán Boh každého z~nás nádherne stvoril -- aj my sa máme preto správať k~druhým tak, aby cítili, že sú úžasne stvorení, dôležití a vzácni pre Boha.

V~apríli sa s~nami sestra Ester Jankovičová podelila so svojím neľahkým životným príbehom a s~hľadaním odpovede na otázku, kde je Boh, keď to bolí. Tému nazvala: „Zaujíma Boha dlhá verzia môjho príbehu?“ Jej inšpirujúce slová sa veľmi hlboko dotýkali našich sŕdc.

Po májovej sesterskej konferencii sme sa stretli až v~júni. Tešili sme sa zo vzájomného zdieľania o~prežitých skúškach a skúsenostiach, z~duchovných víťazstiev a rastu v~poznávaní nášho Pána.

V~lete sme sa mimoriadne stretli, aby sme nadviazali na našu novú tradíciu modlitieb za sestru vstupujúcu do manželstva -- za Tamarku Syčovú a jej nastávajúceho, Jožka Smutného.

September bol veľmi bohatý na rozličné akcie a programy, takže sme pokračovali v~stretávaní až v~októbri. Boli sme veľmi vďačné našej drahej Ester Jankovičovej, že sa s~nami podelila o~to, čo prežila na jubilejnej Konferencii EBF a EBWU -- Únie baptistických žien Európy v~Ammáne.

V~novembri sme mali príležitosť tešiť sa z~toho, že sme súčasťou veľkej rodiny baptistických žien a že si môžeme pripomínať 75.~výročie Svetového dňa modlitieb. BWAW -- Odbor žien Svetovej baptistickej aliancie -- združuje ženy zo 151~krajín sveta. Je veľmi podnetné dozvedieť sa, čím žijú ženy v~krajinách vzdialených od nás tisíce kilometrov, s~akými problémami zápasia, či už sú to vojny, ekonomické problémy, živelné katastrofy, náboženské prenasledovanie alebo domáce násilie, a na základe toho sa môžeme modliť za konkrétne veci a tak našim sestrám pomáhať niesť ich bremená. Často zisťujeme, že máme s~nimi mnoho spoločného, ale zároveň si uvedomujeme, že v~mnohých ohľadoch je náš život oveľa ľahší. Svojimi finančnými príspevkami sme mohli podporiť službu sestier Svetovej baptistickej aliancie a rôzne projekty.

Naše sesterské stretnutia v~roku~2025 sme uzavreli inšpirujúcou témou „Žiť ako Božia dcéra“, ktorou nám poslúžila sestra farárka Anička Činčuráková.

Sme veľmi vďačné všetkým vzácnym sestrám, ktoré nám boli ochotné poslúžiť. Ďakujem za spoluprácu aj sestrám v~tíme, Mirke Hovorkovej, Bake Pribulovej a Barbi Antalíkovej.

Chválime z~hĺbky srdca nášho Pána za to, že nám prejavoval dobrotu a milosť a celý rok~2025 nás sprevádzal svojím bohatým požehnaním.
\autor{Jarmila Cihová}


\clanok{Mládež}
Naše mládežnícke stretnutia už tradične prebiehajú každú sobotu na Súľovskej. Schádzame sa v~počte 15~--~30 mládežníkov. Začíname vždy nejakou aktivitou, ktorú si dopredu pripravia jeden/dvaja mládežníci. Milou súčasťou stretnutí je občerstvenie, na ktorom sa učíme podieľať všetci, každý podľa svojich schopností a možností. Hlavnou náplňou stretnutí je téma, ktorú máva jeden
z~vedúcich. Po téme sa zvyčajne delíme na menšie skupinky, kde sa prebratú tému snažíme aplikovať do našich životov, a zároveň to je aj priestor na zdieľanie sa o~tom, čo prežívame. Tento rok študujeme Ev.~sv.~Jána. Naším cieľom je spoločne spoznávať nášho Boha z~Jeho slova. Okrem pravidelného zamýšľania sa nad Ev.sv. Jána, máme jedenkrát do mesiaca voľnú tému, napr. nebo, manželstvo a vzťahy, čistota, svätosť, posvätenie, umelá inteligencia...

Súčasťou mládeží sú aj akcie. Na začiatku roka vo februári to bola konferencia v~Banskej Bystrici. Najočakávanejšou akciou je vždy letný tábor, ktorý bol opäť v~Novej Lehote, kde sme preberali list Galatským. Bol to vzácny čas na vyučovanie z~Božieho slova, na modlitby, rozhovory a skvelé hry a zábavu. Milým prekvapením bol aj ďalší tábor na Chvojnici, kde bol priestor na osobné duchovné otázky, na spoločné varenie, výlety a oddych.

Nový školský rok sme naštartovali super grilovačkou u~Dzuriakovcov. V~októbri sme sa zúčastnili volejbalového turnaja v~Poprade s~dvomi tímami. A~aj keď ani jeden náš tím nevyhral, aj tak sa chceme tento rok zúčastniť znova, a preto usilovne trénujeme volejbal 2-krát do mesiaca. Je to ďalšia super príležitosť byť spolu.

Je pre nás obrovskou radosťou kráčať časť cesty s~vašimi/našimi deťmi za Pánom Ježišom. Prosíme, aby ste nás niesli na modlitbách, aby naši mladí mohli odovzdať svoje životy do najlepších a najláskavejších rúk aké poznáme, do rúk Pána Ježiša.
\autor{Mirka a Martin Hovorkovci}


\clanok{Dorast}
Na doraste sme v~roku~2025 pokračovali v~tradičnom stretávaní každý piatok medzi 17.30 a 19.00~hod. na Súľovskej~2. Po generačnej obmene sa zúčastňovalo stretnutí priemerne 5~--~6 dorastencov a 3~vedúci zo šesťčlenného tímu. Počas celého roku sme pokračovali hlavne v~štúdiu evanjelia Matúša.

V~lete sme mali už tradičný dorastenecko-mládežnícky tábor v~Novej Lehote. Stíšenia v~skupinách, hlavné témy aj diskusné skupinky boli zamerané na štúdium listu Galatským. Tieto témy boli doplnené množstvom hier, súťaží a rozhovorov. Výzvou bolo popasovať sa s~čoraz väčším vekovým rozpätím účastníkov (11~--~19 rokov). Vďaka Pánu Bohu sme zažili požehnaný týždeň.

Počas prázdnin pracovali dorastenci usilovne na letnej výzve. Úlohou bolo naučiť sa naspamäť Žalm~19 a prečítať Skutky apoštolov. Odmenou nám bola spoločná večera a kino.

Do leta fungoval dorastový tím v~zložení Radislav Nemec, Filip Barkóczi, Ján Kováčik, Tamara Syčová, Angela a Michal Vráblovci. Viacerí členovia tímu po dôkladnom zvažovaní od septembra presunuli ťažisko služby v~zbore na iné miesta. V~súčasnosti fungujeme v~zložení Filip Barkóczi a Michal Vrábel. Rado, Angie a Janko nám aktívne vypomáhajú s~témami a ostávajú tak v~občasnom kontakte s~dorastencami.

Tento rok sa snažíme vytvárať viac priestoru na stretnutia aj mimo bežné piatky (kino, prespávačky, spoločenské hry, šport). Uvedomujeme si, aké je dôležité mať priestor a čas budovať s~dorastencami otvorené a hodnotné vzťahy. Sme veľmi vďační, že môžeme aktívne sprevádzať dorastencov na ich ceste životom, že sa spolu môžeme učiť čítať Bibliu a v~evanjeliu Matúša spoznávať, kým je Ježiš. Radi sa spolu hráme aj rozprávame. Už druhý rok po sebe sme si pre dorast vytiahli rovnaký verš: „Lásku sme poznali podľa toho, že On položil svoj život za nás; aj my máme klásť život za bratov.“ (1J~3,16) V~zmysle tohto verša Vás prosíme o~modlitby, aby sme vedeli dorastencom zrozumiteľne slovom aj skutkom zvestovať Ježiša, aby v~Ňom mohli spoznávať Božiu lásku a učiť sa ju aj prakticky žiť.
\autor{Michal Vrábel}


\clanok{Besiedka}
Dianie v~besiedke je úzko prepojené so školským rokom, preto správa za kalendárny rok je iná ako pri ďalších zborových zložkách. Stav v~besiedke sa mení s~nástupom nového školského roka. Tak tomu bolo aj v~uplynulom roku~2025. Po letných prázdninách nám z~malej do veľkej besiedky prechádzalo 5~detí, čo znížilo počet detí v~malej besiedke. V~1.~polroku sme ich mali evidovaných~10, po prázdninách počet malých besiedkárov klesol na~6. Pribudli nám detičky, ktoré dosiahli alebo čoskoro dosiahnu 3~roky, niektoré deti z~nášho spoločenstva odišli. Predpokladali sme, že reálne bude chodiť do malej besiedky menej detí, preto sme si na Zrínskeho vymenili miestnosti. Deti od~7 do~11 rokov sa stretávajú vo väčšej miestnosti a deti od~3 do~7 rokov v~menšej miestnosti. Koncom roka pribudli do besiedky súrodenci Pribulovci, ktorí sa presťahovali z~Popradu do Bratislavy. Do malej besiedky prichádzajú tiež návštevníci a doprovod najmenších detí, takže nás je niekedy v~malej miestnosti celkom dosť (10~--~14 ľudí). Veľkých besiedkárov máme evidovaných~13 (od~2.~polroku~2025), pravidelne ich chodieva do besiedky okolo~10.

Čo sa týka vyučovania, v~malej besiedke sme v~1.~polroku~2025 preberali tému modlitby a koncom školského roka sme sa zamerali na niektoré udalosti zo Skutkov apoštolov. Nový školský rok sme začali stvorením a kalendárny rok sme zakončili vianočnými udalosťami. Pri vyučovaní kombinujeme materiály z~Detskej misie, zo stránky \ulink[https://www.timdvadva.cz/]{timdvadva.cz} a \ulink[https://truewaykids.com]{truewaykids.com}.

Vo veľkej besiedke preberali v~1.~polroku materiál z~Detskej misie Boh sa stará a pokračovali témou stvorenia a patriarchami (Abrahám, Izák). V~sledovaní života patriarchov pokračujú aj v~tomto školskom roku. Ako hovoria učitelia veľkej besiedky, máme veľmi zlaté a šikovné deti, ktoré sa radi zapájajú do programu. Radi hrajú v~scénkach (čo ste mohli vidieť na Vianoce i na rodinnom tábore). Stále ochotne pokračujú v~zbierke pre chlapca z~Etiópie (Bereket Daniel Tadese). Cez nadáciu Integra sme tak zapojení do projektu podpory detí z~chudobných rodín v~meste Yabelo.

Príjemným občerstvením pre malých i veľkých bol rodinný tábor v~stredisku Detskej misie v~Častej. Témou tábora bolo 10~Božích prikázaní. O~jednotlivých prikázaniach sa rozprávali dospelí aj deti. Je veľmi dobré, keď sa rodiny stretávajú aj mimo zboru a deti si tak môžu budovať zmysluplné vzťahy s~rovesníkmi.

Deti ste mohli v~priebehu roka už tradične vidieť a zažiť pri vítaní ľudí na Kvetnú nedeľu, na Deň matiek i počas vianočného programu, ktorý deti nielen bravúrne zvládli, ale sa podieľali aj na tvorbe scenára.

Vyučovanie detí je dynamický proces a vyžaduje si, aby sa učitelia a pracovníci s~deťmi neustále vzdelávali a flexibilne prispôsobovali novým potrebám. Začiatkom roka sme v~spolupráci s~Detskou misiou pripravili jednodňové školenie pre učiteľov besiedky. Na jar sa uskutočnilo regionálne stretnutie pre učiteľov detí a dorastu organizované Odborom služby deťom BJB. V~priebehu roka sa ako tím besiedky stretávame podľa potreby a riešime aktuálnu situáciu v~besiedke.

Tím učiteľov v~besiedke sa výrazne nezmenil. Učiteľky v~malej besiedke: Kika Horvátiková, Mirka Hovorková, Miriam Kešjarová; pomocníci: Katka Kerekréty, Martin Hovorka, Tamarka Syčová. Učitelia vo veľkej besiedke: Kvetka Maďarová, Barborka Pribulová, Slávka Volentičová, Filip Barkóczi.

Ďakujeme všetkým, ktorí sa za deti na Palisádach modlia! Prosíme, vytrvajte v~modlitbách aj naďalej. Túžime, aby naše deti čo najskôr smeli prijať Pána Ježiša do svojho srdca a stali sa pre svoje okolie svetlom a soľou. Veľmi by sme si tiež priali, aby detí v~našom spoločenstve bolo viac, pretože práve deti a mladí ľudia sú našou budúcnosťou…
\autor{Miriam Kešjarová}


\clanok{Spevokol}
Aj v~roku~2025 sme zamerali našu službu hlavne na oslovenie ľudí počas hlavných sviatkov v~roku, na Vianoce a Veľkú noc. Vtedy sú ľudia citlivejší a hľadajú duchovné povzbudenie viac ako po ostatné dni v~roku. Aj tento rok sme začali na „Troch kráľov“ Novoročným koncertom v~kostole ECAV v~Petržalke, kde domáci zbor už pravidelne pozýva rôzne spevokoly, aby sa nielen viac navzájom spoznávali, ale aby sa uisťovali o~dôležitosti aj tejto služby. Na Novoročný koncert je pozývaný už niekoľko rokov aj náš spevokol. Na konci tohtoročného koncertu prišli za mnou organizátori podujatia s~otázkou, či by sme náš veľkonočný koncert nemohli výnimočne urobiť aj v~ich kostole. Vedeli totiž, že pravidelne robievame na domácej pôde dva rovnaké koncerty pred oboma sviatkami v~sobotu
i v~nedeľu a tak prišli s~návrhom, že by sme jeden koncert urobili doma na Palisádach, ale druhý v~ich kostole v~Petržalke. Radi sme prisľúbili a tak sme mohli prežiť nádherné a požehnané obecenstvo s~novými bratmi a sestrami. Náš Koncert veľkonočných piesní mal takú nádhernú odozvu, že sme hneď dostali aj pozvanie na druhý koncert s~piesňami vianočnými.

Okrem týchto koncertov sme občas zaspievali aj na nedeľných bohoslužbách v~našom zbore. Aj keď by sme radi spievali doma aj častejšie, mnohí členovia nášho spevokolu nie sú z~nášho zboru a v~nedeľu dopoludnia sa tejto služby u~nás nemôžu zúčastniť, lebo sú aktívni vo svojich vlastných zboroch.

Začiatkom mája sme mali možnosť poslúžiť aj na sesterskej konferencii v~Bratislave.

Spievaním piesne pre verejnosť naša práca nekončí. Všetky koncerty nahrávame, hotové videá ukladáme na sociálne siete, a tak sa piesne dostávajú do celého sveta a podľa odozvy, oslovujú naozaj mnohých.

Ďakujeme Pánu Bohu, že môžeme konať túto prácu, do ktorej nám dáva silu, radosť, požehnanie, a takto byť užitoční na Božej vinici.
\autor{Slávo Kráľ}


\clanok{Služba chválospevy}
„Pieseň chvály spievať túžim, obeť chvály pred Tvoj trón nesiem.“

Služba chvál je miestom, kde môžeme Bohu slúžiť celou svojou silou a spoločne Ho vyvyšovať. Ak túžiš spievať Pánovi novú pieseň, rásť v~službe a byť súčasťou tímu, pozývame ťa zapojiť sa. Hľadáme najmä hudobníkov -- basgitaristu, gitaristu a klaviristu, no všetky nástroje sú vítané.

„Môj Boh, Tebe spievam, môj Pán, Tebe slúžim -- len z~lásky.“

Ak cítiš pozvanie, ozvi sa Diane Dzuriakovej na +43 676 3755 189. Budeme vďační za každé otvorené srdce a ochotu slúžiť.

Chválospevy v~našom zbore vnímame ako duchovnú výživu pre všetkých -- pre tých, ktorí sa zúčastňujú bohoslužieb osobne, aj pre tých, ktorí nás sledujú prostredníctvom online prenosu. Každú nedeľu prinášajú niečo trochu iné. Niekedy sa podobajú trojchodovému sviatočnému menu, inokedy zas jednoduchému a výživnému pokrmu na cestu vlakom. Tak ako v~živote, aj v~chválach sa striedajú obdobia, no cieľ zostáva rovnaký: obrátiť naše srdcia k~Bohu.

V~priebehu roka~2025 sme sa spolu ako tím snažili zo všetkých síl, aby žiadne bohoslužby nezostali bez spevu a spoločnej chvály. Či už počas bežných nedieľ, letných prázdnin alebo v~čase výnimočných sviatkov, túžili sme byť verní v~službe a prinášať Bohu chválu v~každom období. Naším cieľom je, aby každú nedeľu zaznievali chvály a vyznania na Božiu slávu -- či už sú letné prázdniny alebo vianočné sviatky.

„Celú noc, celý deň Bohu chválu vzdávajme.“

Vždy je čas chváliť Pána.

Prvú nedeľu v~mesiaci pravidelne ponúkame priestor spevokolu, za ich službu sme veľmi vďační. Zároveň sa tešíme aj výnimočným príležitostiam, ako sú svadby a krsty -- nech ich je čím viac, tým lepšie, na radosť a oslavu Boha.

Za každými chválami však vždy stojí tímová spolupráca. Príprava, služba aj samotné prevedenie sú výsledkom ochoty, flexibility a vzájomnej podpory všetkých členov tímu. Každý z~nich prináša svoje dary, čas a energiu, aby mohla táto služba fungovať.

Do služby chvál sa v~uplynulom období zapojili (v~abecednom poradí):
Adam Alexaj, Boba Šalingová, Daniel Plett, David Potocký, Dávid Pribula, Diana Dzuriaková, Janko Kováčik, Jožko Smutný, Kvetka Maďarová, Lenka Pribulová, Ľubka Kováčiková, Martin Hovorka, Naomi Dzuriak, Peter Kolárovský, Štefan Synovec, Tamarka Smutná, Vierina Kolárovská.

Ďakujeme každému jednému z~vás za ochotu slúžiť, za otvorené srdce a za vernosť, s~akou sa podieľate na tejto službe. Veríme, že chválospevy nie sú len hudobnou súčasťou bohoslužieb, ale priestorom, kde môže každý -- bez ohľadu na okolnosti -- načerpať, stíšiť sa a priblížiť sa k~Bohu.

Budeme vďační za ďalšiu podporu tejto služby, či už aktívnym zapojením, povzbudením alebo modlitbami.
\autor{Diana Dzuriaková}


\clanok{Služba zakladania nového zboru -- Connect}
Rok~2025 bol naplnený vzájomnou a misijnou službou.

Sme vďační za nadviazanie spolupráce s~Detskou misiou, osobitne so Zuzkou Magdoškovou, ktorej pomáhame v~misijnej službe v~Podunajských Biskupiciach.

Osvedčili sa nám tri spoločné akcie s~misijným dosahom, ktoré plánujeme realizovať aj v~nasledujúcich rokoch:
\begitems
* posledná aprílová sobota -- Deň otvorených dverí na Súľovskej;
* predposledná júnová nedeľa -- poobedie pre rodiny;
* posledná septembrová sobota -- Šarkaniáda.
\enditems

Pokračovali sme s~Junior Connectom, ktorý je súčasťou našich nedeľných Connectov.

Sestry, nielen z~Connectu, sa začali raz za mesiac stretávať k~štúdiu Biblie.

Sme vďační za výbornú spoluprácu s~kazateľmi zboru a praktikantom.

Modlíme sa za záchranu ľudí a aby Pán Boh požehnával svoju cirkev a v~rámci nej aj Connect novými ľuďmi.
\autor{Tomáš Valchář}
\vfill\break


\clanok{Revízia hospodárenia}

Revízna komisia v~zložení Miroslav Antalík, Barbora Antalíková, Barbora Pribulová za spolupráce účtovníčky zboru Ľubomíry Kohútovej vykonala revíziu hospodárenia za rok~2025.
Boli prekontrolované nasledovné doklady:
\begitems \style -
* výpisy z~bežného účtu vedeného v~Slovenskej sporiteľni za mesiace 1, 4, 5, 7, 9, 11
* výdavkové a príjmové pokladničné doklady za mesiace 1~--~12
\enditems

Revízna komisia konštatuje, že uvedené doklady sú vedené prehľadne v~súlade
s~účtovnými predpismi. Pokladničná kniha je vedená mesačne a založená priamo pri pokladničných dokladoch.

Neboli zistené žiadne nedostatky.

Stav finančnej hotovosti ku dňu 31.~12.~2025 bol:

\vskip1em\hskip1cm\table{lr}{
pokladňa     &   3~875,17~€ \cr
bankový účet & 208~869,70~€ \crl
spolu        & 212~744,87~€ \cr
}\vskip1em

Tento stav súhlasí so stavom v~účtovnej evidencii k~uvedenému dátumu.



\tiraz
\bye
