\def\velkostpisma{9}
\def\velkostriadku{12}
\input makra.tex % nacitanie Ivanom pripravenych nastaveni a prikazov
\hyphenation{star-šov-stvo} % rozdelenie slov na konci riadku, treba tu uviest slova, ktore sam nepozna

\spravodaj{1}{2025}


\clanok {Požehnaný rok 2025}

V~týchto dňoch častejšie ako počas roka vyslovujeme, či píšeme našim blízkym, priateľom a známym a možno aj neznámym prianie Božieho požehnania v~novom roku 2025. Je to dobrá príležitosť žehnať ľuďom okolo nás. Pre nás dlhoročných veriacich je to možno už aj také automatické a už ani neuvažujeme nad tým, čo to vlastne ľuďom a určite aj sebe želáme, či o~čo prosíme. Čo je to „Božie požehnanie“? Možno niekto má aj pochybnosť, či ho my môžeme priať ľuďom, máme tú kompetenciu?

Požehnanie sa najčastejšie zamieňa, najmä pri prianí do nového roka, s~prianím šťastia. Požehnanie je však viac ako šťastie aspoň v~tom význame, že požehnanie nie je náhodný úspech, ale Bohom daný dar. Stručne povedané Božie požehnanie zahŕňa všetko dobré pre človeka, pre život -- duchovné, duševné a hmotné. Požehnanie vyjadruje dar, ktorý zasahuje do hĺbky života. Požehnanie je zároveň slovom aj darom. Vyjadrujú to aj latinské (bene - dictio) a grécke (eu - logia) slová. Požehnanie sa netýka vlastníctva, ale bytia. Nezávisí na ľudskom, ale na Božom tvorčom čine. Žehnať znamená poukázať na dar, ktorý tvorí a oživuje, či už pred jeho príchodom formou modlitby, prosby alebo dodatočne formou vďakyvzdania.

Môžeme a máme si žehnať aj my ľudia navzájom. Vyslovovať slová dobropriania, požehnania druhým ľuďom, želať im Božie požehnanie, alebo prosiť, aby ich Pán Boh požehnal je podľa Božej vôle. V~Božom slove nájdeme mnohé príklady žehnania v~Starej aj Novej zmluve. Nemusia vždy obsahovať slová „žehnať“, „požehnať“, ale stále to je dobroprianie, prianie blaha -- „blahoslovenie“. Vo všetkých formách je to vlastne prosba, aby Boh dal svoje požehnanie, aby On požehnal. Ľudia, napríklad izraelskí kňazi, alebo dnes kazatelia vyslovujú aj liturgicky požehnanie nad celým zhromaždením Božieho ľudu podľa Božieho príkazu (Nm 6, 22-27), ale ten, kto požehnáva, kto dáva požehnanie je jedine Boh. „Tak nech kladú moje meno na Izraelcov a ja ich požehnám“ (v.27).

Možno si neuvedomujeme, že keď želáme požehnanie, tak želáme sebe aj iným aj ťažké veci, skúšky, ktoré prídu do života človeka a až neskôr, niekedy až po rokoch, ich spätne rozpoznáme ako Božie požehnanie.

Najväčšie požehnanie, ktoré nám Pán Boh dáva je odpustenie hriechov, záchrana, večný život v~našom Pánovi a Spasiteľovi Ježišovi Kristovi. Ak niekomu zo svojich neveriacich priateľov prajeme Božie požehnanie, tak mu v~prvom rade želáme, aby spoznal a uveril v~Pána Ježiša, ako vo svojho Záchrancu.

Boh nepochybne dával, dáva každý deň a chce dávať aj v~roku 2025 svoje požehnanie na naše životy v~rôznej forme, pretože nás miluje. Nemusíme sa báť, že by ho bolo málo pre nás či pre všetkých ľudí. On dáva svoje požehnanie v~hojnosti. Aj pre každý deň roku 2025 ho má pripravené pre každého z~nás dostatok. Problém s~„nedostatkom“ Božieho požehnania nie je na strane poskytovateľa, ale na našej strane ako prijímateľa. My jednoducho nie sme schopní prijať to všetko Božie požehnanie, ktoré On je schopný a ochotný dať. To požehnanie je totiž nekonečné a neobsiahnuteľné, zdroj Božieho požehnania nemôže byť nikdy vyčerpaný. My obmedzujeme tok Božieho požehnania do našich životov, do života našich rodín, nášho zboru, nášho národa svojimi hriechmi, svojou nedôverou Bohu, svojou neochotou podriadiť sa Božiemu vedeniu, svojou neschopnosťou rozoznať Božie požehnanie. Želať Božie požehnanie teda znamená aj priať, aby sme nekládli prekážky Božiemu požehnaniu a boli schopní ho v~čo najväčšej miere prijať.

Nech nás žehná Hospodin aj v~roku 2025.

\autor {Ján Szőllős}


\clanok {Správy zo staršovstva}

V~decembri sme sa stretli na jednom pravidelnom stretnutí. Programovo sme ho mali zamerané na budovanie vzťahov pri kapustnici, ktorú nám pripravil br.~kaz. Janko Szőllős a hodnotenie uplynulého roku. Udialo sa toho dosť a Bohu patrí vďaka za to, že nás viedol, dával nám silu, múdrosť a odvahu ku všetkému, čo sme robili.

Popri tom sme pripravovali služby na obdobie Vianoc a koniec roku 2024. Výsledok sme už všetci spolu zažívali.

Ľubo Syč, za hospodársky výbor, predložil výsledok cenového prieskumu na rekonštrukciu fasády našej modlitebne. Vzali sme na vedomie informácie a odporúčania HV pre realizáciu rekonštrukcie aj s~finančným krytím. Sme vďační nášmu Bohu za Jeho vedenie aj v~tejto oblasti života zboru.

Pripravili sme poďakovanie Šrankotovcom za ich službu v~zbore. Aj touto cestou im vyjadrujeme vďaku za všetko, čo robili v~zbore, a prajeme im Božie vedenie a poznanie cesty, po ktorej ich bude viesť. Poďakovanie sme mali na bohoslužbách počas 4. adventnej nedele.

Na začiatku nového roku 2025 sme vyberali biblické slovo pre zložky nášho zboru.

Pre zbor sme dostali slovo zo Žalmu 86, verš 11: „Vyuč ma, Hospodine, svojej ceste: nech chodím v~Tvojej pravde; na to mi sústreď myseľ, aby som sa bál Tvojho mena.“

Pre staršovstvo sme dostali slovo zo Žalmu 91, verše 1 a 2: „Kto býva v~úkryte Najvyššieho, odpočíva v~tôni Všemohúceho, nech povie Hospodinovi: ‚Moje útočisko a moja pevnosť je môj Boh, v~ktorého dúfam.‘“

\autor {za staršovstvo Peter Pribula}


\clanok {Verš na mesiac}

„Vyuč ma, Hospodine, svojej ceste a budem žiť podľa Tvojej pravdy. Upriam moju myseľ na bázeň pred Tvojím menom.“ (Ž 86,11 -- Slovenský ekumenický preklad)


\clanok {Informácie pre členov zboru}

Návrhy do rozpočtu posielajte do utorka 7.~1.~2025 na \email{starsovstvo@bjbpalisady.sk}.

Zborové členské zhromaždenie je naplánované na nedeľu 26.~1.~2025 o~16.30~hod. v~našej modlitebni na Palisádach.


\clanok {Aliančný modlitebný týždeň -- „Bojujte dobrý boj“}

\table{lcll}{
\mspan4[l]{\bf Bohoslužby} \crl
Ne & 12. 1. & 17.00 & Reformovaná kresťanská cirkev, Nám. SNP 4 \cr
Po & 13. 1. & 18.00 & Apoštolská cirkev -- Otcov dom, Račianska 72 \cr
Ut & 14. 1. & 18.00 & Evanjelická cirkev a.v., Schn. Trnavského 2 \cr
St & 15. 1. & 18.00 & Cirkev bratská, Cukrová 4 \cr
Št & 16. 1. & 18.00 & Bratská jednota baptistov, Palisády 27/A \cr
Pi & 17. 1. & 18.00 & Kresťanské zbory, Pri Šajbách 1 \cr
So & 18. 1. & 18.00 & Cirkev adventistov siedmeho dňa, Cablkova 3 \cr
Ne & 19. 1. & 18.00 & Evanjelická cirkev a.v., Legionárska 2 \cr
}
\vskip0.5em
\table{lcll}{
\mspan4[l]{\bf Témy} \crl
Ne & 1. & BOJUJTE s láskou a za lásku. & Júda 21; 1K 13,6 \cr
Po & 2. & BOJUJTE, pretože sme povolaní, milovaní a chránení. & Júda 1-2 \cr
Ut & 3. & BOJUJTE o vieru, ktorá bola raz navždy odovzdaná svätým. & Júda 3-4 \cr
St & 4. & BOJUJTE a nezabúdajte na predošlé boje. & Júda 5-11 \cr
Št & 5. & BOJUJ tak, že sa budeš vyhýbať nevhodnému správaniu. & Júda 12-16 \cr
Pi & 6. & BOJUJ tak, že budeš trpezlivý voči pochybovačom. & Júda 17-23 \cr
So & 7. & BOJUJ tak, že budeš oslavovať Trojjediného Boha. & Júda 24-25 \cr
Ne & 8. & SKONČITE S ROZPORMI: Spojenci pod jedným Pánom & \cr
}


\clanok {Celkové plnenie rozpočtu k~31.~12.~2024}

\table{lrrr}{
Príjem				& Plán		& Skutočnosť	& podiel z~ročného plánu \crli
Nedeľné zbierky		& 28 000 €	& 26 277 €	 	&  93,85 \% \cr
Dary a desiatky		& 34 000 €	& 36 802 €	 	& 108,24 \% \cr
Misijný fond 		&  5 500 €	&  5 875 €	 	& 106,82 \% \cr
Investičný fond		&  4 000 €	&  1 696 €	 	&  42,40 \% \cr
Záväzky na fasádu	& 92 700 €	& 94 940 €	 	& 102,42 \% \cr
Plat J. Szőllősa	&  7 493 €	&  5 723 €	 	&  76,38 \% \cr}
\vskip1ex

\autor{Ľubomíra Kohútová}


\n 2.	1.	Ivan	KOHÚT;
\n 3.	1.	Ľubomír	KEŠJAR;
\n 11.	1.	Nataša	HOVORKOVÁ;
\n 16.	1.	Blahoslava	BETKOVÁ;
\n 19.	1.	Andrej	CIHO;
\n 21.	1.	Zdeněk	FILIP;
\n 22.	1.	Jana	LAURENČÍKOVÁ;
\n 22.	1.	Jana	ČAHOJOVÁ;
\n 23.	1.	Elena	GUBOVÁ;
\narodeniny


\program{
\p  5 ; ne ;  9.30 ; Bohoslužby (J. Szőllős + VP);.;;
\p  6 ; po ;.;;.;;
\p  7 ; ut ; 14.00 ; Biblická hodina pre seniorov (P. Pivka);.;;
\p  8 ; st ;.;;.;;
\p  9 ; št ; 18.00 ; Biblická hodina (J. Szőllős);.;;
\p 10 ; pi ; 17.30 ; Dorast;.;;
\p 11 ; so ; 18.00 ; Mládež;.;;
\p 12 ; ne ;  9.30 ; Bohoslužby (T. Valchář);.;;
\p 13 ; po ; 18.00 ; Skupinka „Základy viery“;.;;
\p 14 ; ut ; 14.00 ; Biblická hodina pre seniorov (P. Pivka);.;;
\p 15 ; st ; 17.30 ; Stretnutie sestier;.;;
\p 16 ; št ; 18.00 ; Aliančný modlitebný týždeň;.;;
\p 17 ; pi ; 17.30 ; Dorast;.;;
\p 18 ; so ; 18.00 ; Mládež;.;;
\p 19 ; ne ;  9.30 ; Bohoslužby (V. Potockij);.;;
\p 20 ; po ;.;;.;;
\p 21 ; ut ; 14.00 ; Biblická hodina pre seniorov (P. Pivka);.;;
\p 22 ; st ;.;;.;;
\p 23 ; št ; 18.00 ; Biblická hodina (J. Szőllős);.;;
\p 24 ; pi ; 17.30 ; Dorast;.;;
\p 25 ; so ; 18.00 ; Mládež;.;;
\p 26 ; ne ;  9.30 ; Bohoslužby (J. Szőllős);.;;
\p 27 ; po ; 18.00 ; Skupinka „Základy viery“;.;;
\p 28 ; ut ; 14.00 ; Biblická hodina pre seniorov (P. Pivka);.;;
\p 29 ; st ;.;;.;;
\p 30 ; št ; 18.00 ; Biblická hodina (J. Szőllős);.;;
\p 31 ; pi ; 17.30 ; Dorast;.;;
}

\tiraz
\bye
