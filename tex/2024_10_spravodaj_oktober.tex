\def\velkostpisma{9}
\def\velkostriadku{12}
\input makra.tex % nacitanie Ivanom pripravenych nastaveni a prikazov
\hyphenation{star-šov-stvo} % rozdelenie slov na konci riadku, treba tu uviest slova, ktore sam nepozna

\spravodaj{10}{2024}

\def\sekcia#1{\vskip0.5em\noindent #1}


\clanok {JAHVE RAFFA JE BOH, KTORÝ UZDRAVUJE}

Hebrejské slovo {\em raffa} znamená vyliečiť, uzdraviť, obnoviť alebo urobiť celým, kompletným. Krátko nato, ako Izraeliti odišli z~Egypta do zasľúbenej zeme, sa im Boh zjavil ako Jahve Raffa -- Pán, ktorý uzdravuje. Starý zákon naznačuje, že Boh je zdrojom každého uzdravenia. Keď sa modlíš k~Jahve Raffa, pros ho, aby skúmal tvoje srdce. Nechaj ho, nech ti ukáže, čo je v~ňom. Ak odkryje nejaký hriech, pros o~odpustenie a potom sa modli, aby ťa uzdravil. Nový zákon Ježiša ukazuje ako lekára, ako toho, ktorý uzdravuje telo aj dušu a jeho zázraky zjavujú Božie kráľovstvo.

\sekcia{KĽÚČOVÝ VERŠ}

„Povedal: ‚Ak budeš naozaj poslúchať hlas Hospodina, svojho Boha, a budeš robiť to, čo je v~jeho očiach správne, ak poslúchneš jeho príkazy a zachováš všetky jeho ustanovenia, nedopustím na teba nijakú chorobu, akú som dopustil na Egypťanov, lebo ja, Hospodin, som tvoj lekár.‘“ (Ex 15,26)

\sekcia{ZAMYSLI SA}

Boh skúšal Izraelitov nepriaznivými okolnosťami, aby odkryl to, čo bolo skutočne v~ich srdciach. Ako Boh skúša teba? Ako vieš v~skúške obstáť?

{\em Jahve Raffa, skláňam sa pred Tebou a~uznávam, že nie si len môj Stvoriteľ, ale si aj Boh, ktorý ma uzdravuje. Že Ty si mojím uzdravením. Prosím, uzdrav moju dušu aj telo a~uzdrav aj mojich blížnych. Modlím sa, aby si uzdravil všetko bolestivé v~našich životoch. Prosím, konaj svoje dielo v~nás tak, aby to hovorilo o~Tvojej sláve. Amen.}

\sekcia{JAHVE RAFFA AKO BOH, KTORÝ UZDRAVUJE}

„Ako šiel, videl človeka, ktorý bol od narodenia slepý. Jeho učeníci sa ho spýtali: ‚Rabbi, kto zhrešil -- on, alebo jeho rodičia, že sa narodil slepý?‘ Ježiš odpovedal: ‚Nezhrešil ani on, ani jeho rodičia, ale majú sa na ňom zjaviť Božie skutky. Musíme konať skutky toho, ktorý ma poslal, dokiaľ je deň. Ide noc, keď nik nebude môcť pracovať. Kým som na svete, som svetlo sveta.‘ Keď to povedal, napľul na zem, urobil zo sliny blato, a blatom mu potrel oči a povedal mu: ‚Choď, umy sa v~rybníku Siloe‘, čo v~preklade znamená: Poslaný. On šiel umyl sa a vrátil sa vidiaci.“ (J 9,1-7)

\vskip-1ex\begitems
* {\it Chváľ ho}: Pretože Ježiš uzdravuje.
* {\it Ďakuj mu}: Za to, ako prejavuje svoje konanie v~tvojom živote v~oblasti duševného či telesného uzdravenia.
* {\it Vyznaj mu}: Všetky snahy spoliehať sa na svoj úsudok namiesto toho, aby si dôveroval Bohu a jeho pomoci skrze pôsobiacu moc Jeho Ducha.
* {\it Pros ho}: Aby odkryl oblasti duchovnej slepoty v~tvojom živote a v~živote Cirkvi.
\enditems

{\em JAHVE RAFFA -- BOH, KTORÝ UZDRAVUJE}

\sekcia{ZASĽÚBENIA SPOJENÉ S~MENOM JAHVE RAFFA}

Uzdravenie, ktorým Boh odpovie na základe viery na našu modlitbu, je milosť a požehnanie. Fyzické uzdravenie je dočasné, nakoľko sa prejavuje na našom smrteľnom tele, ktoré neustále starne. Nakoniec Božie uzdravovanie na tejto zemi nám poukazuje na večné uzdravenie, ktoré prijmeme v~nebi. Tam už nebude bolesť, ani vrodené poruchy, rakovina, či choroby srdca, depresia, cukrovka, leukémia, astma, ani obyčajné prechladnutie. V~tomto svete, kde je bolesť prirodzenou súčasťou nášho života, potrebujeme žiť v~blízkosti Boha ako nášho Jahve Raffa, aby sa vo svojej milosti na nás prejavoval ako Boh, ktorý uzdravuje.

\sekcia{ZASĽÚBENIA V~PÍSME}

„Toto hovorí Pán, Boh tvojho otca Dávida: Počul som tvoju modlitbu. Videl som tvoje slzy: hľa, uzdravím ťa.“ (2.Kron 20,5)

„Je niekto z~vás chorý? Nech si zavolá starších Cirkvi; a nech sa nad ním modlia a mažú ho olejom v~Pánovom mene. Modlitba s~vierou uzdraví chorého a Pán mu uľaví; a ak sa dopustil hriechov, odpustia sa mu.“ (Jk 5,14-15)

\autor{na základe knihy Božie mená, Peter Šrankota}


\clanok {Správy zo staršovstva za september 2024}

Staršovstvo zboru sa stretlo v~mesiaci september na dvoch riadnych stretnutiach a to 3.~9. a 17.~9. V~prvom rade sa venovalo otázke ukončenia služby nášho kazateľa Petra Šrankotu, konkrétne dopadom na život jeho rodiny a tiež dopadom na zbor. Pripravovalo podklady pre plánovanú diskusiu na túto tému na jesennej víkendovke a pripravovanom ZČZ.
Venovalo sa zhodnoteniu letných aktivít v~zbore. Hovorilo tiež o~dôvodoch neuskutočnenia zborového letného tábora. Zároveň rámcovo diskutovalo, kto bude poverený prípravou tábora budúci rok. V~nadväznosti na to hovorilo o~rozbehu práce v~jednotlivých zložkách.
Ďalším okruhom bolo plánovanie zborových akcií na jeseň tohto roka. Konkrétne sa jednalo o~krst, ktorý sa konal 29.~9. na Zlatých pieskoch a krstili sa tri sestry. Ďalej to bolo oblastné vďakyvzdanie, ktoré sa žiaľ neuskutoční, nakoľko sa nám nepodarilo nájsť kapacitne vyhovujúci priestor. Samozrejme, že sem patrila aj inštalácia nášho ďalšieho kazateľa J.~Szőllősa, ktorá sa konala 8.~9. V~neposlednom rade to bola aj príprava zborovej víkendovky, ktorá sa konala 20.~--~22.~9.~2024 v~Modre.
Okrem toho sa venovalo tiež otázke doplňujúcich volieb do Rady BJB (jeden podpredseda, dvaja členovia KRK-u), rekonštrukcie fasády modlitebne a kancelárie zboru, prezentácii projektu Dobrodina v~našom zbore atď.

Naďalej budeme vďační za vaše modlitby vysielané za zbor a za nás starších. V~prípade, ak máte nejaké námety alebo otázky, kontaktujte, prosím, kazateľov alebo členov staršovstva.

\autor {P. Antalík}


\clanok {Jesenná víkendovka v~Berei}

Tretí septembrový víkend sa konala jesenná víkendovka nášho zboru. V~poradí už tretia. V~komornom zložení sme sa stretli v~piatok večer pri ohni a opekaní špekáčiek. V~sobotu sa pripojili rodiny s~deťmi a začali sme s~nabitým programom. Náš plán bolo naplno využiť deň aj pekné počasie. Najprv nás Vladko Boško previedol témou biblické správcovstvo. Mali sme možnosť sa aj rozhýbať pri hre a to všetci spoločne aj s~deťmi. Nápaditá hra v~okolí chaty nám pripomenula chvíle na zborových táboroch. Poobede sme pokračovali v~téme a neskôr aj v~diskusii o~vyhlásení staršovstva 1.~9.~2024. Diskusia bola potrebnou reflexiou vyhlásenia. Nielen naša rozprava, ale aj pohľady zvonku nášho zboru prinášajú staronovú tému, s~ktorou budeme ešte pracovať. Tak ako každý na osobnej úrovni, tak aj ako spoločenstvo.

Pre mňa ako účastníka bola víkendovka splneným očakávaním. Bola takým malým „víkendovým táborom“, na ktorom sme sa tento rok nestretli. Priniesla mi príjemné aj pracovné chvíle s~cirkevnými súrodencami. Nové poznania, rozuzlenia a také spoločné odhodlanie -- ako ísť ďalej a v~živote zboru.

\autor {K. Kerekréty}


\clanok {Verš na mesiac}

„Práve toto mienil Hospodin, ked povedal: ‚Na tých, čo prichádzajú ku mne, ukážem svoju svätosť, pred všetkým ľudom budem zvelebený.‘“ (Lev. 10,3)


\clanok {Zborové členské zhromaždenie}

Staršovstvo zboru zvoláva zborové členské zhromaždenie na nedeľu 13.~10.~2024 o~16.30~hod. v~modlitebni nášho zboru. Program pozostáva z~prijímania nových členov, informácií o~ukončení služby kazateľa a informácií k~voľbe správcu zboru.


\clanok {Biblické hodiny}

Biblické hodiny pre seniorov s~br.~kaz.~P.~Pivkom sa venujú knihe proroka Daniela. Témou biblických hodín s~br. kaz. J.~Szőllosom je kniha Ozeáš.


\clanok {Stretnutie sestier}

Sestry sa stretnú v~stredu 16.~10.~2024 o~17.30~hod. na~Zrínskeho~2. Hosťkou bude sestra Táňa Trúsiková zo zboru Viera.

\vskip2ex
\n 2.	10.	Peter	ANTALÍK;
\n 6.	10.	Daniel	BALÁŽ;
\n 12.	10.	Barbora	PRIBULOVÁ;
\n 14.	10.	Martin	SIMON;
\n 20.	10.	Ida	PUČEKOVÁ;
\n 22.	10.	Hana	HALAMIČKOVÁ;
\n 25.	10.	Vladimír	IRA;
\n 26.	10.	Martin	HOVORKA;
\n 27.	10.	Miriam	KRÁĽOVÁ;
\n 28.	10.	František	VRABČEK;
\n 28.	10.	Ľubomír	SYČ;
\narodeniny


\program{
\p  1 ; ut ; 15.15 ; Biblická hodina pre seniorov (P. Pivka);.;;
\p  2 ; st ;.;;.;;
\p  3 ; št ; 18.00 ; Biblická hodina (J. Szőllős);.;;
\p  4 ; pi ; 17.30 ; Dorast;.;;
\p  5 ; so ; 18.00 ; Mládež;.;;
\p  6 ; ne ;  9.30 ; Bohoslužby (J. Szőllős + VP);.;;
\p  7 ; po ; 18.00 ; Mládež ;.;;
\p  8 ; ut ; 15.15 ; Biblická hodina pre seniorov (P. Pivka);.;;
\p  9 ; st ;.;;.;;
\p 10 ; št ; 18.00 ; Biblická hodina (J. Szőllős);.;;
\p 11 ; pi ; 17.30 ; Dorast;.;;
\p 12 ; so ; 18.00 ; Mládež;.;;
\p 13 ; ne ;  9.30 ; Bohoslužby (P. Šrankota) ; 16.30 ; Zborové členské zhromaždenie ;
\p 14 ; po ;.;;.;;
\p 15 ; ut ; 15.15 ; Biblická hodina pre seniorov (P. Pivka);.;;
\p 16 ; st ; 17.30 ; Stretnutie sestier;.;;
\p 17 ; št ; 18.00 ; Biblická hodina (J. Szőllős);.;;
\p 18 ; pi ; 17.30 ; Dorast ;.;;
\p 19 ; so ; 18.00 ; Mládež ;.;;
\p 20 ; ne ;  9.30 ; Bohoslužby (V. Pototski);.;;
\p 21 ; po ;.;;.;;
\p 22 ; ut ; 15.15 ; Biblická hodina pre seniorov (P. Pivka);.;;
\p 23 ; st ;.;;.;;
\p 24 ; št ; 18.00 ; Biblická hodina (J. Szőllős);.;;
\p 25 ; pi ; 17.30 ; Dorast;.;;
\p 26 ; so ; 18.00 ; Mládež;.;;
\p 27 ; ne ;  9.30 ; Bohoslužby (R. Nemec);.;;
\p 28 ; po ;.;;.;;
\p 29 ; ut ; 15.15 ; Biblická hodina pre seniorov (P. Pivka);.;;
\p 30 ; st ;.;;.;;
\p 31 ; št ;.;;.;;
}


\tiraz
\bye
