\def\velkostpisma{10}
\def\velkostriadku{12.5}
\input makra.tex % nacitanie Ivanom pripravenych nastaveni a prikazov
\hyphenation{star-šov-stvo} % rozdelenie slov na konci riadku, treba tu uviest slova, ktore sam nepozna

\spravodaj{6}{2024}

\def\sekcia#1{\vskip0.5em\noindent #1}

\clanok {JAHVEH JIRE -- BOH, KTORÝ SA POSTARÁ}

\sekcia{MENO}

Hebrejské slovo {\it raah}, od ktorého je odvodené {\it Jire}, znamená „vidieť“. Pretože Boh pozná budúcnosť tak dobre ako prítomnosť a minulosť, dokáže sa dopredu postarať o~to, čo je potrebné. Preto sa v~tomto význame meno prekladá ako „postarať sa“.

Zaujímavé je tiež poznamenať, že anglické slovo {\it provision} (zaobstaranie, zaopatrenie) vo svojom pôvode pochádza z~dvoch latinských slov, ktoré znamenajú „vidieť vopred“. Keď sa modlíš k~{\it Jahve Jire}, modlíš sa k~Bohu, ktorý pozná tvoju situáciu vopred a je schopný sa postarať o~všetky tvoje potreby.

\sekcia{KĽÚČOVÝ VERŠ}

„Tu zdvihol Abrahám oči a uzrel barana, ktorý bol rohami zachytený v~kroví. Abrahám podišiel, barana vzal a obetoval ho ako zápalnú obetu namiesto svojho syna. A~Abrahám nazval toto miesto ‚Pán vidí‘. Tak sa ešte do dnes na tých miestach hovorí: ‚Na vŕšku Pán uvidí‘.“ (Gn 22,13-14)

\sekcia{ZAMYSLI SA}

\vskip-1ex\begitems
* Aká bola najväčšia obeta, o~ktorú ťa Boh prosil? Ako si ty na túto žiadosť odpovedal?
* Spomeň si na oblasti svojho života, v~ktorých si už zažil Božiu starostlivosť.
\enditems

{\em Jahve Jire, Bože, ktorý sa o~mňa staráš, ďakujem Ti za všetko Tvoje požehnanie, za odpustenie aj prijatie, za zmysel a~nádej, za pokoj a~istotu, za vieru a odvahu, za lásku a~výdrž, za radosť aj smiech, za silu aj múdrosť, za silu odpustiť a~ochotu pomôcť, za odpočinok a~prácu, ale aj za jedlo a~strechu nad hlavou, za rodinu či priateľov, a~Tvoje svetlo do každej situácie.

Verím, že Tvoje požehnanie nikdy neprestane, pretože si Bohom nekonečnej milosti. Amen.}

\sekcia{JAHVE JIRE JE BOH, KTORÝ SA POSTARÁ}

„Keď došli na miesto, ktoré mu označil Boh, Abrahám tam postavil oltár, naukladal drevo, poviazal svojho syna Izáka a položil ho na oltár na drevo. Potom Abrahám siahol rukou a vzal nôž, aby zabil svojho syna. Vtedy naň zavolal Pánov anjel z~neba: ‚Abrahám, Abrahám!‘ On odpovedal: ‚Tu som.‘ On mu povedal: ‚Nevystieraj ruku na chlapca a neubližuj mu! Teraz som totiž poznal, že sa bojíš Boha a neušetril si svojho jediného syna kvôli mne.‘“ (Gn 22,9-12)

\vskip-1ex\begitems
* {\it Chváľ ho}: Pretože je nesmierne vyvýšený, ďaleko za naším chápaním aj naším myslením.
* {\it Ďakuj mu}: Za to, že Boh ešte stále túži mať s~tebou vzťah hlbší vzťah ako doteraz.
* {\it Vyznaj mu}: Akékoľvek sklony kapitulovať, keď od teba Boh žiadal niečo nad tvoje sily. No nedôveroval si mu, že On ťa posilní, uschopní aj povedie.
* {\it Pros ho}: Za milosť a silu sa skláňať milosťou k~iným.
\enditems

{\em Jahve Jire je pripomienkou toho, že On je ten, ktorý sa postará.}

\sekcia{PRISĽÚBENIA SPOJENÉ S~MENOM JAHVE JIRE}

Biblia je plná Božích prisľúbení. Keď ostávame verní, nič nemôže zabrániť naplneniu týchto prisľúbení bez ohľadu na okolnosti. Zamysli sa nad okolnosťami v~Abrahámovom prípade. Boh prisľúbil, že jeho potomkov bude toľko ako hviezd na nebi, toľko ako piesku na morskom brehu. Ale ak by jeho jediný syn Izák zomrel, ako by sa toto mohlo naplniť? Ak by Abrahám odmietol poslúchnuť Boha, pravdepodobne by prišiel o~obrovské požehnanie, ktoré nasledovalo.

Tento istý princíp platí aj pre nás. Keď ide o~to, čo Boh prisľúbil, naša poslušnosť je kľúčová.
Bez ohľadu na okolnosti, neexistuje nič, čo by mohlo Bohu zabrániť splniť to, čo sľúbil - ak mu dôverujeme a prejavíme to v~poslušnosti, či podriadenosti sa Jemu.

\sekcia{PRISĽÚBENIA V~PÍSME}

„A vôbec nech niet medzi vami chudobného ani žobráka, lebo Pán, tvoj Boh, ťa požehná v~krajine, ktorú ti dá do vlastníctva. Ale iba ak budeš poslúchať Pána, svojho Boha, a zachovávať všetko, čo ti prikázal a čo ti ja nariaďujem dnes, bude ťa požehnávať, ako sľúbil.“ (Dt 15,4-5)

\autor{na základe knihy Božie mená, Peter Šrankota}


\clanok {Správy zo staršovstva za máj}

V~máji 2024 sme mali dve pravidelné stretnutia a výjazdové stretnutie na Chvojnici. Venovali sme sa témam súvisiacim s~voľbou kazateľa, situáciou v~zbore, NCD, ZČZ, KDZ, ale aj aktivitám počas leta.

Situáciu v~zbore vnímame ako značne dôležitú. Preto sme sa rozhodli zvolať diskusné stretnutie. Pôvodne sme ho plánovali na 16.~6.~2024. Nakoľko je na tento termín plánovaný aj krst, stretnutie sa uskutoční 9.~6.~2024 o~18.00 hod. v~modlitebni na Palisádach.

Rovnako dôležité je pre nás vnímanie Božej vôle aj v~otázke ďalšieho kazateľa. Preto sa chceme stretnúť a spoločne hľadať Jeho vôľu na modlitbách. Modlitby sme zvolali na štvrtok 23. 5. 2024. Modlitby chceme sústrediť aj za situáciu v~zbore a prosiť aj o~upokojenie napätia v~spoločnosti.

Z~výsledkov NCD, ktoré sme prezentovali na ZČZ, pripravujeme konkrétne aktivity zamerané na rozvoj spoločného života zboru.

O~výsledkoch KDZ, ku ktorému sme mali aj ZČZ, sme informovali e-mailom aj osobne na bohoslužbách.

Rovnako ako v~minulých rokoch aj teraz organizujeme letný zborový tábor. Pre plynulý priebeh príprav aj realizácie potrebujeme dobrovoľníkov, ktorí sa zapoja do jeho priebehu.

Na výjazdovom stretnutí, ktoré bolo na zborovej chalupe na Chvojnici, sme sa okrem zborových tém venovali aj vzájomnému poznávaniu sa.

\autor {za staršovstvo v~láske Peter Pribula}


\clanok{Voľby ďalšieho kazateľa}

Od nedele 26.~5.~2024 sa začínajú voľby ďalšieho kazateľa zboru. Voľby sa budú konať celkovo po 4 nedele, t.j. posledná volebná nedeľa bude 16.~6.~2024 do~12.00 hod.
Voľby budú prebiehať nasledovne: po skončení nedeľnej bohoslužby sa vpredu modlitebne zhromaždia členovia, ktorí chcú v~danú nedeľu voliť. Bude potrebné podpísať svoju účasť na krátkej zborovej hodine a rovnako aj účasť na voľbách. Priebeh volieb zabezpečuje volebná komisia, ktorá vás vo všetkom usmerní.


\clanok{Večer modlitieb a chvál}

Prvý júnový podvečer v~sobotu 1.~6.~2024 sa o~17.00~hod. stretneme v~našej modlitebni k~modlitbám a chválam. Hosťami budú Samuel a Dávid Cekov. Dávid Cekov povedie aj seminár o~modlitbe a uctievaní a poslúži v~chválospevovej skupine. Slúžiť v~chválach nám bude aj chválospevová skupina nášho zboru.


\clanok{Stretnutie sestier}

Sestry sa stretnú v~júni, a to v~stredu 5.~6.~2024 o~17.30~hod. na Zrínskeho. O~téme stretnutia vás budeme informovať v~najbližších dňoch.


\clanok{Diskusné stretnutie}

V~nedeľu 9.~6.~2024 sa v~našej modlitebni na Palisádach uskutoční diskusné stretnutie k~situácii v~našom zbore.


\clanok{Krst}

V~nedeľu 16.~6.~2024 popoludní o~16.00~hod. sa zídeme pri jazere Kuchajda, kde budeme mať slávnosť krstu. Konkrétne miesto bude upresnené.


\clanok{Letné tábory}

Blíži sa leto a znovu pripomíname dátumy táborov, ktoré organizuje náš zbor.

Letný tábor dorastu a mládeže sa bude konať v~termíne 21.~--~27.~7.~2024 v~Novej Lehote. Prihlasovanie už bolo spustené. Bližšie informácie u~A.~Vrábelovej.

Zborový tábor sa uskutoční v~auguste 18.~--~24.~8.~2024 v~stredisku Prameň, Častá.


\clanok{Rekonštrukcia exteriéru našej modlitebne}

\table{(\hskip-1.5mm)lr}{
Predpoklad realizácie prác r.~2025 & \crli
Predpokladaná medziročná inflácia +4\% & 5 700 € \cr
{\bf Predpokladané celkové náklady} & {\bf 148 200 €} \cr
Z invest. fondu (tvoreného z~ned. zbierok a darov) chceme použiť & 60 000 € \cr
Zaokrúhlene treba ešte vyzbierať & 90 000 € \cr
Počet aktívnych členov & cca 100 ľudí \cr
{\bf Odporúčaná priemerná výška daru na jedného člena} & {\bf 900 €} \cr}
\vskip1em

Realizácia daru je prevodom na zborový účet SK36 0900 0000 0000 1147 1836, variabilný symbol 777.

Naďalej budú pokračovať aj nedeľné zbierky počas štvrtých nedieľ v~mesiaci (zbierka na investičný fond), určené hlavne pre priateľov zboru.
Je potrebné si tiež uvedomiť, že financie na tento účel sú nad rámec nášho bežného dávania do zborovej pokladne, teda je potrebné zachovať aj dary (desiatky....), ktoré sme dávali doteraz.

26.~4.~2024 sa vykonal odber vzoriek z~povrchových vrstiev fasády našej modlitebne odborným reštaurátorom za účelom vypracovania reštaurátorského výskumu a dokumentácie s~návrhom obnovy, v~ktorej budú zapracované výsledky realizovaného reštaurátorského výskumu, bude sumarizovaný návrh technologických postupov obnovy jednotlivých častí pamiatky vrátane uvedenia navrhovaných materiálov. Podľa zmluvy má reštaurátor na dodanie spomínanej dokumentácie 2 mesiace. Celková cena za to je 5 304~€.

\vskip1em
\table{lrrr}{
Príjmy						 & Plán		& Skutočnosť & podiel z~ročného plánu \crli
Investičný fondu (do 31. 3.) &  4 000 €	&  1 696~€	 & 42,40 \% \cr
Záväzky na fasádu	   		 & 92 700 €	& 16 916~€	 & 18,25 \% \cr}
\vskip1em

\autor {za hospodársky výbor Ľubomír Syč}


\clanok{Celkové plnenie rozpočtu k~22.~5.~2024}

\table{lrrr}{
Príjmy	         & Plán	  	& Skutočnosť 	& podiel z~ročného plánu \crli
Nedeľné zbierky	 & 28 000 € & 10 762 €		& 38,44 \% \cr
Dary a desiatky	 & 34 000 € & 13 019 €		& 38,29 \% \cr
Misijný fond	 &  5 500 € &  3 366 €		& 61,20 \% \cr}


\clanok{Zbierky za máj}
Milí bratia a sestry, ďakujeme za vašu obetavosť. V~mesiaci máj ste prispeli:
\vskip-1ex\begitems
* misia: 437 €
* zbierka na investície(28.~4.): 75 €
\enditems


\n  2. 6. Miriam	KEŠJAROVÁ;
\n  7. 6. Pavel	KOHÚT;
\n  9. 6. Samuel	PLETT;
\n 10. 6. Ján	LAURENČÍK;
\n 15. 6. Ľubica	HOVORKOVÁ;
\n 15. 6. Peter	LICHANEC;
\n 15. 6. Trey	ATKINS;
\n 17. 6. Juraj	KVAČKA;
\n 19. 6. Anna	ŠANDOROVÁ;
\n 19. 6. Oľga	VALCHÁŘOVÁ;
\n 22. 6. Kristína	HORVÁTIKOVÁ;
\n 25. 6. Peter	ŽEMBERY;
\n 25. 6. Marica	ŠČEVLÍKOVÁ;
\n 27. 6. Sylvia	PRIBULOVÁ;
\n 28. 6. Jana	PERKNOVSKÁ;
\narodeniny


\program{
\p  1 ; so ; 17.00 ; Večer modlitieb a chvál ;.;;
\p  2 ; ne ;  9.30 ; Bohoslužby (P. Šrankota + VP) ;.;;
\p  3 ; po ;.;;.;;
\p  4 ; ut ; 15.15 ; Biblická hodina pre seniorov (P. Pivka) ;.;;
\p  5 ; st ; 17.30 ; Stretnutie sestier ;.;;
\p  6 ; št ; 18.00 ; Biblická hodina (J. Szőllős) ;.;;
\p  7 ; pi ; 17.30 ; Dorast ;.;;
\p  8 ; so ; 18.00 ; Mládež ;.;;
\p  9 ; ne ;  9.30 ; Bohoslužby (Andrew Hayes) ;.;;
\p    ;    ; 18.00 ; Diskusné stretnutie ;.;;
\p 10 ; po ;.;;.;;
\p 11 ; ut ; 15.15 ; Biblická hodina pre seniorov (P. Pivka) ;.;;
\p 12 ; st ;.;;.;;
\p 13 ; št ; 18.00 ; Biblická hodina (J. Szőllős) ;.;;
\p 14 ; pi ; 17.30 ; Dorast ;.;;
\p 15 ; so ; 18.00 ; Mládež ;.;;
\p 16 ; ne ;  9.30 ; Bohoslužby (P. Šrankota) ;.;;
\p    ;    ; 16.00 ; Krst (Kuchajda) ;.;;
\p 17 ; po ;.;;.;;
\p 18 ; ut ; 15.15 ; Biblická hodina pre seniorov (P. Pivka);.;;
\p 19 ; st ;.;;.;;
\p 20 ; št ; 18.00 ; Biblická hodina (J. Szőllős) ;.;;
\p 21 ; pi ; 17.30 ; Dorast ;.;;
\p 22 ; so ; 18.00 ; Mládež ;.;;
\p 23 ; ne ;  9.30 ; Bohoslužby (D. Plett) ;.;;
\p 24 ; po ;.;;.;;
\p 25 ; ut ; 15.15 ; Biblická hodina pre seniorov (P. Pivka);.;;
\p 26 ; st ;.;;.;;
\p 27 ; št ; 18.00 ; Biblická hodina (J. Szőllős) ;.;;
\p 28 ; pi ; 17.30 ; Dorast ;.;;
\p 29 ; so ; 18.00 ; Mládež ;.;;
\p 30 ; ne ;  9.30 ; Bohoslužby (J. Szőllős) ;.;;
}


\tiraz
\bye
