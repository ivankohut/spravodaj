\def\velkostpisma{10}
\def\velkostriadku{12.5}
\input makra.tex % nacitanie Ivanom pripravenych nastaveni a prikazov
\hyphenation{star-šov-stvo} % rozdelenie slov na konci riadku, treba tu uviest slova, ktore sam nepozna

\spravodaj{9}{2025}


\clanok {Správy zo staršovstva}
Staršovstvo zboru sa počas leta stretlo na svojom plánovanom stretnutí 1.~8. Prvým bodom programu bolo privítanie D.~Chuchúta na kazateľskej praxi v~zbore a nastavenie a odsúhlasenie štruktúry a harmonogramu jeho činnosti v~našom zbore. Na tento bod bol okrem samotného D.~Chuchúta prizvaný ako hosť aj br.~kazateľ T.~Valchář, ktorý bude spolu s~kazateľom nášho zboru J.~Szőllősom spoluvedúcim praxe. Brat D.~Chuchút bude mať v~rámci praxe službu Slovom, bude sa zapájať do práce v~jednotlivých zložkách nášho zboru aj v~Connecte a bude sa zúčastňovať aj stretnutí a práce staršovstva zboru.

V ďalšom bode staršovstvo revidovalo zoznam členov zboru a konštatovalo v~zmysle ustanovení zborového poriadku skončenie členstva 15 členom, ktorým bolo členstvo pozastavené ešte v~roku 2016, a pozastavilo členstvo 16 členom, ktorí sa dlhodobo nezúčastňujú života zboru. Členovia staršovstva diskutovali o~možnostiach kontaktu a pastoračnej starostlivosti o~týchto členov.

Staršovstvo odsúhlasilo aj uskutočnenie sobášu v~rámci zboru BJB Viera v~našej modlitebni a vypočulo informáciu o~prípravách oblastného vďakyvzdania. Na stretnutí sa riešili aj otázky prevádzky zboru (presťahovanie kancelárie späť do bytu na poschodí), prepis odberu plynu na zbor a zámenu a predaj malej časti pozemkov na Chvojnici.

Prvé stretnutie v~rámci nového školského roku sa uskutočnilo 2.~9. Navrhnutý program bol vzhľadom na začiatok pracovného (školského) roka bohatý. Staršovstvo zhodnotilo letné aktivity (najmä tábory) v~zbore, ocenilo výbornú organizáciu a duchovný prínos a požehnanie, ktoré aj cez tábory prišlo od Pána, a vyjadrilo vďačnosť všetkým, ktorí sa na uskutočnení táborov podieľali. Zvlášť vyzdvihlo krst, ktorý sa udial na zborovom tábore. Bratia starší naplánovali mimoriadne podujatia, ktoré by sa popri pravidelných podujatiach a stretnutiach mohli do Vianoc konať, a tiež aktualizovali termíny a rozdelenie služieb Slovom v~zbore. Opäť sa venovali aj členstvu v~zbore, prijali žiadosti dvoch členov o~ukončenie členstva a konštatovali ukončenie jedného členstva. Ján Szőllős predstavil aktualizovanú matriku zboru od roku 2001.

Ďalšia časť stretnutia bola venovaná príprave návštevy kazateľa Tima Hanesa v~zbore 21.~9. a stretnutia s~ním ako možným kandidátom na kazateľa zboru a príprave oblastného vďakyvzdania.

\autor {za staršovstvo J.~Szőllős}


\clanok {Z prázdninového života zboru alebo „Ako bolo v~lete na táboroch?“}
V lete sa v~našom zbore uskutočnili 2 tábory -- dorastenecko-mládežnícky v~júli a zborový/rodinný v~auguste. Mladí, starší a deti prežili krásny a Pánom Bohom požehnaný čas. Pre tých, ktorí nemali možnosť sa zúčastniť, prinášame „ochutnávku“ vo forme malého interview táborníkov, odpovedali na tieto otázky:
\vskip-1ex\begitems \style n
* Čím bol tábor iný než predošlý?
* Čo pekné si prežil?
* Priniesol/la si si z~tábora nejaké odhodlanie/rozhodnutie/alebo niečo príjemné na srdci?
\enditems

\def\tabornik#1{\noindent #1: \vskip-1ex\begitems \style n}

\cast{Zborový/rodinný tábor}

\tabornik{Slávo K.}
* S~akým zámerom si išiel na tábor? -- Prehovorili ma členovia našej rodiny a som vďačný, že sa im to podarilo.
* Obyčajne na táboroch treba riešiť nejaké konflikty, urážky, nedorozumenia. Tu som nezažil ani jedno. Vynikajúca atmosféra.
* Chcem čo najskôr spracovať video z~toho pobytu, aby sme si mohli s~vďakou pripomínať tie požehnané chvíľky.
\enditems

\tabornik{Mirka S.}
* Boli iné jedlá :), zapojila som sa aj ja.
* Mala som príjemné rozhovory, ťahalo ma to vrátiť sa po práci (dopoludnia som chodila do práce) na tábor a nie domov.
* Pocit rodiny a spolupatričnosti.
\enditems

\tabornik{Naomi D.}
* Tento tábor bol iný tým, že som mala tú príležitosť sa dať pokrstiť v~Častej, na mieste, ktoré pre mňa veľmi veľa znamená.
* Každé ráno o~6.30 sme s~Mirkou Hovorkovou mali stíšenia a čítali sme si spolu Bibliu. Potom sme sa rozprávali o~všeličom, dokonca ona je žena s~dokonalým humorom.
* Tento rok som mala na Teenzone svedectvo a bola som v~úplnom úžase z~toho, ako veľmi som tých mladých oslovila. Túžim naďalej slúžiť Bohu tým, že sa budem venovať mladým. Záleží mi na nich veľmi a som odhodlaná všetko pre to spraviť, keď mi to dá Boh najavo.
\enditems

\cast{Dorastenecko-mládežnícky tábor}

\tabornik{Gabriel K.}
* Tábor bol iný tým, že sme mali trochu iné aktivity, ale moc sa nelíšil.
* Prežil som pekný čas s~kamarátmi a s~Bohom.
* Na tábore som zistil, že by som mal zamerať svoj život viac k~Bohu. A~ešte som si našiel veľa dobrých kamarátov a mal som veľmi dobrý čas s~nimi.
\enditems

\tabornik{Klára K.}
* Mali sme menej lesných hier.
* Všetko, rozhovory, večerné hry, táborovú poštu.
* Čítať Písmo viac.
\enditems

\tabornik{Martin H.}
* Deti v~tímoch boli natoľko zažraté do hier, že nám ako vedúcim inak nedalo, ako sa tiež zapojiť do hier.
* Lásku od všetkých naokolo, lebo sa máme čím ďalej viac radi.
* Odnášam si kopu receptov od Hanky Šandorovej, ktorá nám skvelo varila.
\enditems

\autor {táborníci}


\clanok {Verš na mesiac}
V~septembri sa budeme učiť verš, ktorý dostala pre rok 2025 veľká besiedka: „Lebo som presvedčený, že ani smrť ani život, ani anjeli, ani kniežatstvá, ani prítomnosť, ani budúcnosť, ani mocnosti, ani vysokosť, ani hlbokosť, ani nijaké iné stvorenstvá nemôžu nás odlúčiť od lásky Božej, ktorá je v~Ježišovi Kristovi, našom Pánovi.“ (R~8,38-39)


\clanok {Zborové členské zhromaždenie 14.~9.~2025}
Krátke zborové členské zhromaždenie sa uskutoční v~nedeľu 14.~9.~2025 po bohoslužbe (cca o~11.00~hod.) na Palisádach. Programom bude prijímanie nového člena -- Naomi Dzuriak.


\clanok {Spoločný obed -- Hotel Plus na Trnávke}
V nedeľu 21.~9.~2025 sa uskutoční spoločný obed. Po obede bude priestor na diskusiu s~br.~Timotejom Hanesom, ktorý bude mať dopoludnia u~nás kázeň. Prihláste sa na obed zapísaním sa do online tabuľky, ktorá bola zaslaná e-mailom, a to do 15.~9.~2025. Cena obeda je 8~€ a je už zverejnená aj v~tabuľke.
\vfill\break


\clanok {Šarkaniáda}
V sobotu 27.~9.~2025 organizuje Connect „šarkaniádu“. Stretnutie je od~10.00~hod. v~Jelke na ihrisku spolu s~gulášom a opekačkou. Pridajte sa aj s~deťmi, budete vítaní!


\clanok {Vďakyvzdanie zborov BJB z~Bratislavy a okolia}
Po dlhšej prestávke opäť plánujeme stretnutie zborov BJB z~Bratislavy a jej okolia. Spoločná bohoslužba je v~nedeľu 19.~10.~2025 od~9.30~hod. v~SUZA. Bude možné sa prihlásiť aj na obed. Cena obeda je 8~€, káva 1,60~€, detské porcie možné nie sú.


\clanok {Zborové členské zhromaždenie -- 26.~10.~2025}
Zborové členské zhromaždenie sa uskutoční v~nedeľu 26.~10.~2025 na Palisádach. Jedným z~bodov programu bude prijímanie nových členov. Celý program a čas bude oznámený neskôr.


\clanok {Krst}
Náš zbor pripravuje krst v~termíne 30.~11.~2025, v~nedeľu poobede (miesto a čas ešte určíme). Kto túži vyznať svoju vieru v~krste, nech sa prihlási u~br.~kazateľa Janka Szőllősa alebo u~členov staršovstva.


\clanok {Obedy s~bratmi}
Bratia Janko Szőllős, Filip Barkóczi a Dávid Chuchút prijímajú naše pozvania na obed do našich domácností. Prosíme, zapíšte sa do online tabuľky.
\vfill\break


\clanok {Zbierky v~auguste}
\table{lr}{
Na misiu			&   338~€ \cr
Na investičný fond 	&   171~€ \cr}
\vskip1em

Aj naďalej máte možnosť prispieť do „nedeľnej zbierky“, a to prevodom na účet zboru. Do poznámky pre prijímateľa, prosím, uveďte „zbierka“.

Bankové spojenie: SK36 0900 0000 0000 1147 1836, SWIFT: GIBASKBX


\n 2.	9.	Radislav	NEMEC;
\n 5.	9.	Dušan	UHRIN;
\n 14.	9.	Štefan	SYNOVEC;
\n 16.	9.	Daniel	PLETT;
\n 19.	9.	Richard	HALAMÍČEK;
\n 21.	9.	Kvetoslava	MAĎAROVÁ;
\n 21.	9.	Miroslava	SIMONOVÁ;
\n 22.	9.	Viera	KOLÁROVSKÁ;
\n 25.	9.	Stanislav KRÁĽ;

\narodeniny


\program{
\p  1 ; po ; 18.00 ; Skupinka Základy viery;.;;
\p  2 ; ut ; 15.00 ; Biblická hodina pre seniorov (P.~Pivka);.;;
\p  3 ; st ;.;;.;;
\p  4 ; št ;.;;.;;
\p  5 ; pi ; 17.30 ; Dorast ;.;;
\p  6 ; so ; 15.00 ; Mládež ;.;;
\p  7 ; ne ;  9.30 ; Bohoslužby (J.~Szőllős + VP);.;;
\p  8 ; po ;.;;.;;
\p  9 ; ut ; 15.00 ; Biblická hodina pre seniorov (P.~Pivka);.;;
\p 10 ; st ;.;;.;;
\p 11 ; št ;.;;.;;
\p 12 ; pi ; 17.30 ; Dorast ;.;;
\p 13 ; so ; 18.00 ; Mládež ;.;;
\p 14 ; ne ;  9.30 ; Bohoslužby (D.~Chuchút);.;;
\p 15 ; po ; 18.00 ; Skupinka Základy viery;.;;
\p 16 ; ut ; 15.00 ; Biblická hodina pre seniorov (P.~Pivka);.;;
\p 17 ; st ;.;;.;;
\p 18 ; št ; 18.00 ; Biblická hodina (D.~Chuchút);.;;
\p 19 ; pi ; 17.30 ; Dorast ;.;;
\p 20 ; so ; 18.00 ; Mládež ;.;;
\p 21 ; ne ;  9.30 ; Bohoslužby (T.~Hanes);.;;
\p 22 ; po ;.;;.;;
\p 23 ; ut ; 15.00 ; Biblická hodina pre seniorov (P.~Pivka);.;;
\p 24 ; st ;.;;.;;
\p 25 ; št ; 18.00 ; Biblická hodina (F.~Barkóczi);.;;
\p 26 ; pi ; 17.30 ; Dorast ;.;;
\p 27 ; so ; 18.00 ; Mládež ;.;;
\p 28 ; ne ;  9.30 ; Bohoslužby (F.~Barkóczi);.;;
\p 29 ; po ; 18.00 ; Skupinka Základy viery;.;;
\p 30 ; ut ; 15.00 ; Biblická hodina pre seniorov (P.~Pivka);.;;
}


\tiraz
\bye
