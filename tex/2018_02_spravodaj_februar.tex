% DOKUMENTACIA:

% Prazdny riadok za textom znamena ukoncenie odstavca.
% Cierne obldzniky na konci riadku (v PDF) - to nechaj na mna (moze to o.i. znamenat, ze treba pridat nejake slovo do \hyphenation, lebo ho sam nevie rozdelit na konci riadku)

% Prikazy pre casti spravodaja:
% \spravodaj{<mesiac>}{<rok>}
% \clanok{<nazov clanku>}
% \autor{<autor clanku>}
% \n<den.mesiac.meno> - zadefinovanie oslavenca
% \narodeniny - vytvorenie tabulky s narodeninami vsetkych zadefinovanych oslavencov
% \tiraz - ukoncenie spravodaja tirazou

% Styl fontu:
% \bf - bold, plati do konca aktualne skupiny, napr. ak mas {aaa \bf bbb} ccc, tak aaa bude normalne, bbb bude bold, ccc bude normalne
% \it - italic (pouzit rovnakym sposobom ako \bf)
% \bi - bold italic (pouzit rovnakym sposobom ako \bf)
% \rm - normalne (pouzit rovnakym sposobom ako \bf)

% Dalsie prikazy a znaky:
% \begitems - zoznam (odrazky), informacie najdes na stranke http://petr.olsak.net/ftp/olsak/opmac/opmac-u.pdf#toc%3A.5
% \ulink[<cielova adresa]{<zobrazena adresa>} - klikatelny odkaz na webstranku
% \email{<adresa>} - klikatelny odkaz na e-mailovu adresu
% ~ - nedelitelna medzera, napr. v~dome, 21.~6.~2018
% -- - pomlcka (dvakrát -)
% „ - zaciatocna uvodzovka
% “ - koncova uvodzovka
% \noindent - najblizsi odstavec nebude odsadeny
% \vskip<velkost> - vertikalna medzera, napr. \vskip3pt alebo \vskip-3ex (zaporna medzera, t.j. posun smerom hore)


\input makra.tex % nacitanie Ivanom pripravenych nastaveni a prikazov
\hyphenation{star-šov-stvo} % rozdelenie slov na konci riadku, treba tu uviest slova, ktore sam nepozna

\spravodaj{2}{2018}

\clanok{S úctou a bázňou oslavujme nášho Boha!}
Text: Žalm 95,1-7a

Najprv si stručne zadefinujme, čo znamená chvála a uctievanie podľa Božieho slova: Znamená to nesebecké zvelebovanie väčšej a významnejšej osoby, ako sme my sami. Ako máme uctievať nášho Boha Hospodina? Prvá vec je, ktorá mi príde na myseľ ako odpoveď, sú slová Pána Ježiša žene Samaritánke: „Boh je Duch a tí, ktorí ho vzývajú, musia Ho vzývať v Duchu a v pravde.“ (Ján 4,24) Ak máme uctievať Boha prijateľným spôsobom, musíme byť znovuzrodení z Ducha Svätého. Naše uctievanie Boha má byť vedené a inšpirované Duchom Svätým, pretože iba On pozná, aké uctievanie sa Bohu páči (viď. 1K 2,11b). Nikto a nič nemôže chváliť Boha tak ako človek -- koruna stvorenia, lebo len človek z celého stvorenia má ducha. Jedno z hlavných poslaní človeka, pre ktoré ho Boh stvoril a postavil do záhrady Eden, bolo a stále je práve to, aby sme uctievali a chválili Boha. Naše uctievanie musí prameniť z reality života, ktorý žijeme v~spoločenstve s Bohom Otcom cez Ježiša Krista, Jeho Syna. Musíme uctievať Boha nie len našimi ústami, ale celým svojím srdcom, inak to nie je pravé uctievanie Boha.

{\bf Pamätajme na deň sviatočného odpočinku.}

Na veľké židovské sviatky prichádzali do Jeruzalema tisíce pútnikov. Z chrámu, ktorý bol centrom náboženského života Izraela, zaznievala mohutná výzva pre celý zástup -- oslavovať Najvyššieho: „Poďte, plesajme pred Hospodinom, zvučne ospevujme skalu našej spásy!“

{\bf Pred Boha môžeme predstupovať len s úctou a bázňou.}

Hospodin je veľký Kráľ nad všetkými božstvami. Len s veľkou bázňou môže človek predstúpiť pred tvár vznešeného Kráľa. Človek je poctený, že môže spievať na Jeho slávu a česť, v~spoločenstve s Ním, aby sme Ho svojím životom uctievali.

{\bf Veď On je náš Boh, my sme ľud Jeho pastvy, ovce, ktoré vedie dobrý Pastier!}

Za vysokú cenu sme boli vykúpení krvou, a nie hocijakou, ale svätou krvou Božieho Syna. Pán Ježiš nás posvätil, a to nie len v zmysle, že si nás oddelil pre seba, ale taktiež nás uspôsobil pravej úcte a službe Bohu. To sa stalo na základe toho, že nám odpustil previnenia a očistil naše svedomie od zlých skutkov (viď Žd 9,14).

{\bf Máme sa radovať v Pánovi.}

Preto sa ľud Boží môže a má radovať aj uprostred búrok a bojov, lebo my môžeme vidieť viac ako neveriaci človek, my vieme, že náš domov je v nebesiach, tu sme len hostia a cudzinci, preto sa môžeme za každých okolností radovať a chváliť Pána za všetky dobrodenia, ktoré nám denne preukazuje, k tomu nás povzbudzuje aj apoštol Pavel: „Radujte sa v Pánu, vždycky opakujem radujte sa!“ (F 4,4)

Neviem ukončiť túto úvahu lepšími slovami vďaky, ako to vyjadril Dávid v jednom zo svojich žalmov: „Dobroreč duša moja Hospodinu a celé moje vnútro Jeho svätému menu! Dobroreč duša moja Hospodinu a nezabúdaj na žiadne Jeho dobrodenia.“ (Žalm 103,1-2)

\autor{Pavel Pivka}

\clanok{Správy zo staršovstva}
% Neviem urobit tab pri prvom odseku... (platí to pre tento i nasledujúce články)
V januári 2018 sme sa stretli dva razy. Jednou dôležitou témou, ktorej sme sa venovali, boli pripravované voľby do Rady BJB a komisií. Uvažovali sme o možných kandidátoch. Po  oslovení a získaní ich súhlasu sme nominačnej komisii navrhli nasledujúcich kandidátov:

br. B. Uhrin – predseda Rady BJB

br. Z. Kakaš – podpredseda Rady BJB

br. J. Szőllős – podpredseda Rady BJB

br. D. Kraljik – člen Rady BJB za Západnú oblasť

br. D. Smolník – člen Kontrolno-rozhodcovskej komisie

br. D. Uhrin – člen Kontrolno-rozhodcovskej komisie

br. Z. Podobný – člen Revíznej komisie

s. J. Lukáčová – člen Revíznej komisie

s. Z. Dóczeová – člen Revíznej komisie

Ďalšou témou, o ktorej sme hovorili, bol prebiehajúci projekt duchovnej starostlivosti o seniorov. Brat Pali Pivka nám podal správu za predchádzajúce obdobie a zároveň návrh pokračovania projektu v ďalšom období.

Na obdobie ďalších troch mesiacov sme plánovali služby v našom zbore a na Chvojnici.
Diskutovali sme o niekoľkých návrhoch členov zboru.

Na stretnutí 23. januára sme sa stretli s tromi záujemcami o členstvo v zbore. Rozprávali sme o~podmienkach členstva, predstavách možného zapojenia do života zboru a predstave, ako môže zbor pomôcť im.

Najobsiahlejšou témou, ktorej sme sa venovali, je návšteva Dannyho a Clary Jonesovcov a ich trvalý príchod na Slovensko. Tešíme sa na ich návštevu začiatkom februára. Natrvalo prídu na Slovensko už koncom mája.

\autor{za staršovstvo zboru Peter Pribula}

\clanok{Stretnutie sestier}
% tento článok nemusí byť - je uverejnený v týždňových oznamoch
Milé sestry, pozývame Vás na {\bf stretnutie sestier} dňa {\bf 6. 2. 2018 o 17.00 hod.} na Zrínskeho.
Naším hosťom bude sestra Clara, manželka nášho brata kazateľa. Porozpráva nám niečo o sebe a predstaví nám svoje plány pre prácu OS v našom zbore. Je to naše prvé stretnutie s ňou, tak sa všetci veľmi tešíme.
Prosím, nezabudnite sa modliť za požehnaný čas, ktorý tu brat kazateľ s~rodinou budú tráviť v kruhu našej zborovej rodiny.

\autor{Vlasta Šalingová}

\clanok{Spoločný obed}

Milí bratia a sestry, drahí priatelia,
pozývame vás na {\bf spoločný obed}, ktorý sa uskutoční {\bf 11.~2.~2018 o~12.00 hod.} v~Hoteli Tatra na námestí 1. mája v Bratislave. Možnosť záväzného prihlásenia sa záujemcov o spoločný obed bude k dispozícii v nedeľu 4. 2. 2018 vo foyeri našej modlitebne, kde budú pripravené hárky a proces bude prebiehať rovnakým spôsobom ako pri predchádzajúcich spoločných obedoch. Všetci ste srdečne pozvaní a v prípade súvisiacich otázok je vám k dispozícii br. M. Kolářik.

\clanok{Národný týždeň manželstva 2018}

Národný týždeň manželstva (NTM) je týždenná iniciatíva na podporu manželstva. Cieľom NTM je propagovať a podporiť manželstvo ako vzácnu hodnotu pre moderného človeka. NTM vytvára priestor, kde si môžeme pripomenúť, že spokojné manželstvo nie je samozrejmosť, ale vzťah, ktorý by sme mali rozvíjať.

NTM sa uskutoční v týždni od 12. do 18. februára 2018. Tento rok heslo NTM znie: Manželstvo ako umenie lásky. Viac informácií o tejto kampani, ako aj o akciách spojených s NTM, nájdete na \ulink[http://www.ntm.sk]{www.ntm.sk}.

\clanok{Zbierka pre Ukrajinu}

Zbierka na pomoc Ukrajine sa uskutoční 18. 2. 2018 počas dopoludňajšieho zhromaždenia. Podrobnosti budú uvedené v ďalších oznamoch.

\clanok{2 percentá z dane -- Chata v Račkovej doline}

Milí priatelia, v roku 2004 sme prevzali do plnej správy Chatu Komenského v Račkovej doline. Odvtedy ju prevádzkujeme a zároveň aj rekonštruujeme. Na jar 2015 sme zrealizovali rozsiahlu rekonštrukciu spoločenských miestností. Z milosti Božej sú teda tri a pol zo štyroch poschodí už hotové, pričom sme preinvestovali vyše 373 000 €.

Tento rok by sme radi v prípade dostatku financií dokončili celé prvé poschodie (spoločné toalety, prestavba turistických izieb…). Je to veľká výzva, a preto sa modlíme a hľadáme jednotlivcov aj zbory, ktoré by toto dielo finančne podporili:
\begitems
* darovaním 2 percent z daní
* dobrovoľnými darmi
\enditems
Z plánovaného rozpočtu 85 000 €, sa nám doteraz podarilo zabezpečiť polovicu. Chcel by som Vás poprosiť, či by ste mohli informovať ľudí okolo seba, známych, priateľov o možnosti podporiť toto dielo. Vopred ďakujem aj za Vaše 2 percentá, ak sa rozhodnete.

\autor{Palino Šinko}

\clanok{Senior klub}

Senior klub v mesiaci február bude posledný štvrtok v mesiaci, t. j. dňa {\bf 22.~2.~2018 od~10.00 do~14.00~hod.} na Súľovskej ul.

„Spievajte na slávu Jeho mena, Jeho slávu šírte chválospevom!“ Žalm 66, 2.
Témou februárového Senior klubu bude oslava nášho drahého Pána a Spasiteľa z červených spevníkov. Preto prosíme všetkých seniorov, aby si ho nezabudli doniesť so sebou.
Ďakujem!

\autor{Jana Makovíniová}

\clanok{Výročné celozborové zhromaždenie}

Staršovstvo zboru BJB Bratislava – Palisády zvoláva podľa platného Zborového poriadku

Výročné celozborové zhromaždenie členov zboru

Dátum: 11. 3. 2018

Miesto konania: modlitebňa zboru Palisády 27/A

\clanok{Pomoc ľuďom bez domova}

Milé sestry a milí bratia, rada by som vám dala do pozornosti termíny varenia polievok ľuďom v núdzi na rok 2018.

17.~marec (sobota); 17.~apríl (utorok); 15.~máj (utorok); 19.~jún (utorok); 17.~júl (utorok); 21.~august (utorok); 18.~september (utorok); 20.~október (sobota); 17.~november (sobota); 15.~december (sobota)

Na rezervované termíny sa, prosím, nahláste u mňa.

\autor{Beata Bogárová}

\clanok{Zbierky za január}
Milí bratia a sestry, ďakujeme za vašu obetavosť. V mesiaci január ste prispeli:
\begitems
* misia: 237,70 €
* podpora Samaritán Tekovské Lužany: 300 €
* investičný fond: 506,70 €
\enditems

\clanok{Klub D.E.P.O. na Súľovskej}
D.E.P.O. junior je v piatok od 16.00 do 18.00 hod. a je určený hlavne dorastu. D.E.P.O. je každý piatok od 18.00 do 22.00 hod. a je určený mládeži. Náplňou klubov sú moderné spoločenské hry, stolný futbal, pingpong, posedenie, rozhovory, hudba, občerstvenie, tvorivé dielne.

\clanok{Skupina anonymných alkoholikov Maják}
AA-liečebný program anonymných alkoholikov má svoje pravidelné stretnutia vo štvrtok od 17.30 do 18.30 hod. v našich zborových priestoroch na Zrínskeho 2.

\autor{Kontakt: Želka 0903 294 927}


\n 3.	2.	Vlasta	BALÁŽOVÁ;
\n 3.	2.	Margita 	KRÁĽOVÁ;
\n 3.	2.	Miroslav 	ANTALÍK;
\n 4.	2.	Veronika	VEČEREKOVÁ;
\n 5.	2.	Štefánia 	ANTALÍKOVÁ;
\n 5.	2.	Barbora 	ANTALÍKOVÁ;
\n 11.	2.	Juraj 	BALÁŽ;
\n 11.	2.	Oľga 	KOVÁČOVÁ;
\n 11.	2.	Beata 	BOGÁROVÁ;
\n 12. 2.	Martin	PRIBULA;
\n 13. 2.	Zlatica 	VYSKOČILOVÁ;
\n 15.	2.	Ingrida 	JANČULOVÁ;
\n 16. 2.	Lenka	PRIBULOVÁ;
\n 23.	2.	Anna	PLETT;
\narodeniny

\program{
\p 1  ; št ; 19.00 ; Biblická hodina (J. Szőllős, Zrínskeho 2);.;;
\p 2  ; pi ; 16.00 ; D.E.P.O. (Súľovská 2);.;;
\p 3  ; so ; 18.00 ; Mládež (Súľovská 2);.;;
\p 4  ; ne ; 9.30 ; Bohoslužby (D. Jones);.;;
\p 5  ; po ; 18.00 ; Modlitby (Zrínskeho 2);.;;
\p 6  ; ut ; 15.00 ; Popoludnie pri Biblii (P. Pivka, Zrínskeho~2); 17.00 ; Stretnutie sestier (Zrínskeho 2);
\p 7  ; st ;.;;.;;
\p 8  ; št ; 19.00 ; Biblická hodina (J. Szőllős, Zrínskeho 2);.;;
\p 9  ; pi ; 16.00 ; D.E.P.O. (Súľovská 2);.;;
\p 10  ; so ; 18.00 ; Mládež (Súľovská 2);.;;
\p 11  ; ne ; 9.30 ; Bohoslužby (D. Jones); 10.00; Chvojnica (P. Škulec);
\p 12  ; po ; 18.00 ; Modlitby (Zrínskeho 2);.;;
\p 13  ; ut ; 15.00 ; Popoludnie pri Biblii (P. Pivka, Zrínskeho 2);.;;
\p 14  ; st ;.;;.;;
\p 15  ; št ; 19.00 ; Biblická hodina (J. Szőllős, Zrínskeho 2);.;;
\p 16  ; pi ; 16.00 ; D.E.P.O. (Súľovská 2);.;;
\p 17  ; so ; 18.00 ; Mládež (Súľovská 2);.;;
\p 18  ; ne ; 9.30 ; Bohoslužby (P. Pribula); 10.00; Chvojnica (M. Ira);
\p 19  ; po ; 18.00 ; Modlitby (Zrínskeho 2);.;;
\p 20  ; ut ; 15.00 ; Popoludnie pri Biblii (P. Pivka, Zrínskeho 2);.;;
\p 21  ; st ;.;;.;;
\p 22  ; št ; 19.00 ; Biblická hodina (J. Szőllős, Zrínskeho 2);.;;
\p 23  ; pi ; 16.00 ; D.E.P.O. (Súľovská 2);.;;
\p 24  ; so ; 18.00 ; Mládež (Súľovská 2);.;;
\p 25  ; ne ; 9.30 ; Bohoslužby (J. Kohút); 10.00; Chvojnica (P. Pribula);
\p 26  ; po ; 18.00 ; Modlitby (Zrínskeho 2);.;;
\p 27  ; ut ; 15.00 ; Popoludnie pri Biblii (P. Pivka, Zrínskeho 2);.;;
\p 28  ; st ;.;;.;;
}

\tiraz
\bye
