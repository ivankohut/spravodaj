% DOKUMENTACIA:

% Prazdny riadok za textom znamena ukoncenie odstavca.
% Cierne obldzniky na konci riadku (v PDF) - to nechaj na mna (moze to o.i. znamenat, ze treba pridat nejake slovo do \hyphenation, lebo ho sam nevie rozdelit na konci riadku)

% Prikazy pre casti spravodaja:
% \spravodaj{<mesiac>}{<rok>}
% \clanok{<nazov clanku>}
% \autor{<autor clanku>}
% \n<den.mesiac.meno> - zadefinovanie oslavenca
% \narodeniny - vytvorenie tabulky s~narodeninami vsetkych zadefinovanych oslavencov
% \tiraz - ukoncenie spravodaja tirazou

% Styl fontu:
% \bf - bold, plati do konca aktualne skupiny, napr. ak mas {aaa \bf bbb} ccc, tak aaa bude normalne, bbb bude bold, ccc bude normalne
% \it - italic (pouzit rovnakym sposobom ako \bf)
% \bi - bold italic (pouzit rovnakym sposobom ako \bf)
% \rm - normalne (pouzit rovnakym sposobom ako \bf)

% Dalsie prikazy a znaky:
% \begitems - zoznam (odrazky), informacie najdes na stranke http://petr.olsak.net/ftp/olsak/opmac/opmac-u.pdf#toc%3A.5
% \ulink[<cielova adresa]{<zobrazena adresa>} - klikatelny odkaz na webstranku
% \email{<adresa>} - klikatelny odkaz na e-mailovu adresu
% ~ - nedelitelna medzera, napr. v~dome, 21.~6.~2018
% -- - pomlcka (dvakrát -)
% „ - zaciatocna uvodzovka
% “ - koncova uvodzovka
% \noindent - najblizsi odstavec nebude odsadeny
% \vskip<velkost> - vertikalna medzera, napr. \vskip3pt alebo \vskip-3ex (zaporna medzera, t.j. posun smerom hore)

%\typosize[10/12.5]% - pouzita velkost pisma/riadku (trochu vacsie)
\input makra.tex % nacitanie Ivanom pripravenych nastaveni a prikazov
\hyphenation{star-šov-stvo} % rozdelenie slov na konci riadku, treba tu uviest slova, ktore sam nepozna

\spravodaj{7-8}{2019}


\clanok {Čas na oddych}
Tešil sa Pán Ježiš na leto a prázdniny? Rád rozmýšľam nad takými otázkami. Prečo sa na leto a prázdniny tešíme my? Väčšinou sa tešíme, lebo sme unavení a potrebujeme spomaliť tempo. Život beží príliš rýchlo a potrebujeme si oddýchnuť. Potrebujeme trochu viac spánku a pokoja, a to sa nám v~dnešnom svete ťažko hľadá. Dokonca máme výčitky z~toho, keď spomalíme, lebo potom menej stíhame. Výkonnosť je kráľom tejto doby. Vždy môžeme spraviť viac a pomocou technologických vymožeností nás vždy nájdu. Preto si kladiem otázku: „Čo by robil Ježiš?“

Neustále bol obklopený davom ľudí, ktorých sa ešte nedotkol a neuzdravil. Stále bol obťažený životom. Ale vidíme, že aj On občas dospel do bodu, kedy si povedal: „Dosť! Poďme na nejaké pokojné miesto a oddýchnime si.“ Vidíme, že niekedy bol tak unavený, že zaspal aj v~nepohodlnej loďke. Poznal svoje obmedzenia a slabosti ako človek. Zopakujem to: poznal svoje obmedzenia ako človek. V~tom máme často problém. Nechceme si svoje obmedzenia priznať a ťaháme to stále viac a viac. Preto je pre nás leto dôležité. Treba spomaliť a nadýchnuť sa.

V zbore máme počas leta menej organizované aktivity. Máme tábory, aby sme mohli vyjsť z~rutiny nášho života. Plánujeme dovolenky, lebo potrebujeme byť mimo všetkého. Rovnako ako Pán Ježiš potrebujeme byť s~našimi blízkymi osamote. Pán Ježiš sa často so svojimi učeníkmi utiahol do ústrania u~priateľov v~Betánii alebo doma. Viem si predstaviť, ako oddychoval u~Lazara v~tôni stromov. Možno práve na to narážala Marta, keď si Mária pri Ňom odpočinula.

Chcem nás vyzvať k~tomu, aby sme počas tohto leta spomalili. Vyzývam nás k~tomu, aby sme sa odpojili od toho, čo nás zaneprázdňuje a zanechali všetko, čo nás vytláča zo života. Vyzývam nás, aby sme v~pokoji a v~Kristovej prítomnosti znovu načerpali z~Jeho milosti. Vyzývam nás, aby sme spolu s~Bohom pozreli na to, čo sme už vykonali, a povedali si: „Stačí. Je čas na oddych.“

Prajem vám úspešný letný oddych!

\autor{Danny Jones}


\clanok {Správy zo staršovstva}
Staršovstvo sa stretlo v~júni na dvoch stretnutiach.

Rozprávali sme so záujemcami o~členstvo v~zbore. Sme radi, že priatelia v~našom zbore majú záujem o~aktívny život v~zbore.

Pripravovali sme:
\vskip-1ex\begitems
* službu na Chvojnici počas sviatku zoslania Ducha Svätého
* zborové členské zhromaždenie
* koncert detí a hudobníkov v~zbore
* letné zborové dni počas letných prázdnin
\enditems

Venujeme sa aj téme práce našej mládeže a s~našou mládežou. Sme vďační za to, že Pán pracuje aj v~ich životoch a používa ich pre nesenie evanjelia.

Pustili sme sa do vysporiadania majetkových pomerov v~dome na Zrínskeho ulici. Máme záujem o~to, aby boli aj naše majetkové vzťahy jasné a čisté.

Pripravili sme ponuku pre záujemcov o~cestu do Izraela. Termín, program a cenu sme oznámili e-mailom aj prostredníctvom letáku.

Stretli sme sa s~Viktorom Potockým. Viktor hľadá možnosť, ako po ukončení štúdia zostať na Slovensku. Chcel by dostať na Slovensko aj svoju rodinu. Modlite sa, prosím, za to, aby Pán Boh ukázal svoju vôľu a aby Viktor porozumel, kam ho náš Pán posiela.

\autor {za staršovstvo Peter Pribula st.}


\clanok {Letné zborové dni}
Počas letných prázdnin plánujeme predbežne dva letné zborové dni, kde by sme sa mohli spolu najesť a stráviť čas pri spoločných aktivitách. Predbežné termíny týchto popoludňajších stretnutí sú 28. júl (v Pod. Biskupiciach) a 11. august 2019 (miesto ešte určíme). Bližšie informácie poskytneme po detailnejšom naplánovaní plánovaných stretnutí.


\clanok{Dorastenecký tábor}
Tábor našich dorastencov sa plánuje počas posledného prázdninového týždňa v~termíne 24. -- 31.~augusta (predbežne na Chvojnici).

Viac informácií môžete získať u~br. Riša Halamíčka alebo Martina Simona.


\clanok {Kemp mládeže}
Aj tento rok sa bude konať už 14.~ročník celoslovenského kempu mládeže BJB, a to v~termíne 27. -- 31.~augusta v~Novej Lehote, v~tábore Royal Rangers. Kemp je určený pre účastníkov vo veku od~14 do~26 rokov.

Viac informácií ako aj prihlášku nájdete na stránke \ulink[https://mladez.baptist.sk/kemp]{mladez.baptist.sk/kemp}.


\clanok {Kurz {\it Kairos}}
Radi by sme vám dali do pozornosti kurz {\it Kairos}, ktorý slúži na vystrojenie jednotlivcov aj celých zborov, aby si našli svoje miesto vo Veľkom poverení \hbox{(Mt 28,19-20)}. Počas 5-dňového intenzívneho kurzu účastník prejde biblickými, historickými, strategickými a kultúrnymi základmi, ktoré zavŕši praktickou implementáciou. Kapitoly sú poprepájané interaktívnymi vstupmi zážitkovou formou.

Dátum: 18.~8.~--~23.~8.~2019

Miesto: Stredisko Berea, Modra-Harmónia

Cena: 140~€

Viac informácií môžete získať na emailovej adrese \email{kurzkairos@gmail.com}, prípadne na telefónnom čísle 0905~680~085.

Prihlásiť sa môžete do 17.~júla prostredníctvom elektronického formulára \ulink[https://bit.ly/2LoxkQA]{bit.ly/2LoxkQA}.


\clanok {Campfest}
Dynamické chvály, hlboké uctievanie, inšpiratívne slovo, semináre, svedectvá, koncerty, tanec, netradičné športy a najmä prežívanie Božej prítomnosti spolu so 6,5~tisíc mladými ľuďmi!

Pozývame vás na 21. ročník festivalu Campfest, ktorý sa bude konať 1.~--~4. augusta v~Kráľovej Lehote na tému {\it Krok vpred}. Chceme urobiť ten nadprirodzený Boží krok vpred. Chceme prekročiť čiary, na ktoré sme si zvykli. Pravdepodobne budeme musieť prekročiť samých seba, svoje pravdy, svoje schopnosti, vzťahy, názory. Vykročíme na cestu, ktorá je úzka, ale ktorú pred nami už prešliapal sám Kristus. Tieto kroky dokážu robiť tí, ktorí sú už príliš unavení, alebo ešte príliš mladí a nezrelí.

Viac informácií a možnosť zakúpenia vstupeniek nájdete na webovej stránke \ulink[www.campfest.sk]{www.campfest.sk}. Ak máte chuť pomôcť pri príprave festivalu, stále sú voľné dobrovoľnícke miesta.


\clanok {Zájazd do Izraela}
Zbor BJB Palisády pripravuje pre záujemcov zájazd do Izraela na jar 2020. Ide o~zájazd zvlášť plánovaný pre náš palisádsky zbor v~spolupráci so sprievodcom Eli Bar David, mesiánskym židom, vedúcim v~kibuci Yad HaShmobna. Počas zájazdu bude poskytnutý aj biblický výklad, ktorý pripravujú br. kaz. Danny Jones a Ľuboš Dzuriak.

Máme k~dispozícii približne 20 miest v~termíne 2.~--~8.~marca~2020.

Podrobnejšiu informáciu nájdete v~modlitebni vo foyer alebo priamo od br. kaz. Dannyho Jonesa (\email{djones@bjbpalisady.sk}).


\clanok {Kemp Detskej misie pre mladých}
Srdečne vás pozývame na kemp pre mladých, ktorý organizuje Detská misia. Kemp sa uskutoční na konci letných prázdnin v~termíne 27.~8.~--~1.~9.~2019 v~Častej, v~stredisku Prameň.

Viac informácií môžete získať u~sestry Mirky Kešjarovej.


\clanok {Poďakovanie a pozdrav}
Hovorí sa, že zdravý človek má tisíce prianí, chorý len jedno -- byť zdravý. Moje  dlhoročné zdravotné problémy nakoniec skončili operáciou chrbtice. Mnohí z~vás o~tom vedeli a mysleli na mňa počas týchto týždňov.

Chcem sa poďakovať za prejavy vašej lásky a starostlivosti rôznymi spôsobmi modlitbami, sms-kami, telefonátmi aj osobnými návštevami. Všetko to sú vzácne prejavy lásky Božej rodiny. Viem, že všetky sú v~Božej evidencii a nebudú zabudnuté.

Mimoriadne som vďačná za možnosť sledovať živé vysielanie Bohoslužieb v~nedeľu. Aspoň na chvíľu som mohla byť s~vami.

Prichádza čas prázdnin a dovoleniek. Prajem vám požehnaný čas s~rodinou a priateľmi a Božiu ochranu na cestách. Prajem Božie požehnanie pre celý zbor.

S láskou a vďačnosťou

\autor {Vlasta Šalingová}


\clanok {Ubytovanie pre mladého Ukrajinca}
Mladý brat z~Ukrajiny (18 rokov), člen baptistického zboru, hľadá v~Bratislave izbu na prenájom od septembra aspoň na 10 mesiacov. Účel jeho pobytu je štúdium jazykových kurzov na UK.

Ak máte možnosť poskytnúť ubytovanie alebo o~takej možnosti viete, kontaktujte, prosím, br. Viktora Potockého na tel. 0919~285~654.
\vfill\break


\clanok{Verš na zapamätanie}
Na mesiace júl a august máme nový veršík, ktorý sa chceme spoločne učiť. Veríme, že poznanie Písma prospeje našej duši i našej mysli:

{\it „Hospodin upevňuje kroky človeka, v~jeho ceste má záľubu. Ak spadne, nezostane ležať, lebo mu Hospodin podoprie ruku.“}

\autor{Ž~37,~23~--~24}


\clanok{Zbierky za uplynulé obdobie}
Milí bratia a sestry, ďakujeme za vašu obetavosť. V~uplynulom období ste prispeli:
\vskip-1ex\begitems
* investičný fond: 316,50~€ (máj)
* misia: 0,--~€ (jún)
\enditems


\n 1.	7.	Ľudovít	BETKO;
\n 4.	7.	Margita	ELISCHEROVÁ;
\n 4.	7.	Ľubomíra	KOHÚTOVÁ;
\n 10.	7.	Slavomír	MÁŤUŠ;
\n 10.	7.	Katarína	KEREKRÉTY;
\n 11.	7.	Milada	KREJČOVÁ;
\n 16.	7.	Rút	BEDNÁRIKOVÁ;
\n 20.	7.	Mária	KOHÚTOVÁ;
\n 27.	7.	Lenka	KOHÚTOVÁ;
\n 28.	7.	Elena	ŠALINGOVÁ;
\n 28.	7.	Pavlína	SYNOVCOVÁ;
\n 31.	7.	Marína	CIHOVÁ;
\n 1.	8.	Dana	KEŠJAROVÁ;
\n 1.	8.	Zuzana	HRAŠKOVÁ;
\n 1.	8.	Vlasta	ŠALINGOVÁ;
\n 2.	8.	Marta	RAČIČOVÁ;
\n 7.	8.	Anna	KOPČOKOVÁ;
\n 11.	8.	Šimon	HOVORKA;
\n 16.	8.	Radovan	JANČULA;
\n 18.	8.	Anna	LIPTÁKOVÁ;
\n 23.	8.	Danica	PAULENOVÁ;
\n 25.	8.	Ivan	PAULEN;
\n 31.	8.	Miroslava	HOVORKOVÁ;
\narodeniny


\programna{7}{
\p 2  ; ut ; 15.15 ; Stretnutie pri Biblii (P. Pivka, Zrínskeho 2) ;.;;
\p 7  ; ne ; 9.30 ; Bohoslužby (R. Petrilák); 10.00 ; Chvojnica (Komárno) ;
\p 9  ; ut ; 15.15 ; Stretnutie pri Biblii (P. Pivka, Zrínskeho 2) ;.;;
\p 14 ; ne ; 9.30 ; Bohoslužby (Ľ. Dzuriak) ; 10.00 ; Chvojnica (P. Kohút)  ;
\p 21 ; ne ; 9.30 ; Bohoslužby (D. Jones) ; 10.00 ; Chvojnica (P. Pribula) ;
\p 28 ; ne ; 9.30 ; Bohoslužby (P. Pribula) ; 10.00 ; Chvojnica (P. Škulec) ;
}
\vskip3ex
\programna{8}{
\p 4  ; ne ; 9.30 ; Bohoslužby (D. Jones); 10.00 ; Chvojnica ;
\p 11 ; ne ; 9.30 ; Bohoslužby (D. Jones) ; 10.00 ; Chvojnica  ;
\p 18 ; ne ; 9.30 ; Bohoslužby (V. Ira) ; 10.00 ; Chvojnica (J. Szőllős) ;
\p 25 ; ne ; 9.30 ; Bohoslužby (P. Antalík) ; 10.00 ; Chvojnica ;
}


\tiraz
\bye
