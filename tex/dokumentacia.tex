\shyph % use csplain
\input opmac
\parindent 1em % odstavcova zarazka
\emergencystretch=20em % roztiahnutie medzislovnych medzier v riadku
\hyperlinks{\Blue}{\Blue} % farba hyperlinkov - vnutorne aj URL modre
\input cs-pagella % font Palatino
\activettchar|


\tit Dokumentácia k najčastejšie používaným príkazom pri tvorbe spravodaja a výročných správ

Prázdny riadok za textom znamená ukončenie odstavca.

Čierne obdĺžniky na konci riadku (v PDF) -- to nechaj na mňa (môže to o.i. znamenať, že treba pridať nejaké slovo do |\hyphenation|, lebo ho sám nevie rozdeliť na konci riadku)

Príkazy pre časti spravodaja:
\begitems
* |\spravodaj{|{\it mesiac}|}{|{\it rok}|}| -- titulka spravodaja, napr. |\spravodaj{2018}{4}|
* |\vyrocnespravy{|{\it rok}|}| -- titulka výročných správ za daný rok, napr. |\vyrocnespravy{2017}|
* |\clanok{|{\it názov článku}|}| -- začiatok článku, napr. |\clanok{Výlet na Chvojnicu}|
* |\cast{|{\it názov časti článku}|}| -- začiatok časti článku, napr. |\cast{Harmonogram}|, |\cast{Prihlasovanie}|
* |\autor{|{\it autor článku}|}| -- autor článku, napr. |\autor{Ján HRACH}|
* |\n |{\it den}|.|{\it mesiac}|.|{\it meno} -- zadefinovanie oslávenca, napr. |\n 19. 11. Ján HRACH|
* |\narodeniny| -- vytvorenie tabuľky s~narodeninami všetkých zadefinovaných oslávencov
* |\tiraz| -- ukončenie spravodaja tirážou
\enditems

Štýl fontu:
\begitems
* |\bf| -- bold, platí do konca aktuálnej skupiny, napr. |{aaa \bf bbb} ccc| bude vyzerať takto: \hbox{{aaa \bf bbb} ccc}
* |\it| -- italic (použiť rovnakým spôsobom ako |\bf|)
* |\bi| -- bold italic (použiť rovnakým spôsobom ako |\bf|)
* |\rm| -- normálne (použiť rovnakým spôsobom ako |\bf|)
\enditems

Ďalšie príkazy a znaky:
\begitems
* |\begitems| -- zoznam (odrážky), informácie nájdeš v dokumentácii k makrám \ulink[http://petr.olsak.net/ftp/olsak/opmac/opmac-u.pdf\#toc\%3A.5]{\OPmac}
* |\ulink[|{\it cieľová adresa}|]{|{\it zobrazená adresa}|}| -- klikateľný odkaz na webovú stránku, napr. |\ulink[https://www.sme.sk/]{sme.sk}|
* |\email{|{\it adresa}|}| -- klikateľný odkaz na e-mailovú adresu
* |~| -- nedeliteľná medzera, napr. |v~dome|, |21.~6.~2018|
* |--| -- pomlčka (dvakrát |-|)
* „ -- začiatočná úvodzovka
* “ -- koncová úvodzovka
* |\noindent| -- najbližší odstavec nebude odsadený
* |\vskip|{\it veľkosť} -- vertikálna medzera, napr. |\vskip3pt| alebo |\vskip-3ex| (záporná medzera, t.~j. posun smerom hore)
* |\vfill\break| -- zalomenie strany
\enditems

\bye
