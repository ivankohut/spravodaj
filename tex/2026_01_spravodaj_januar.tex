\def\velkostpisma{9}
\def\velkostriadku{12}
\input makra.tex % nacitanie Ivanom pripravenych nastaveni a prikazov
\hyphenation{star-šov-stvo} % rozdelenie slov na konci riadku, treba tu uviest slova, ktore sam nepozna

\spravodaj{1}{2026}


\clanok{Modlitba za nový rok}
Niektorí z~nás o~týchto otázkach možno premýšľajú už od konca januára minulého roka, lebo už vtedy začínali naše nádherné nové predsavzatia práchnivieť a zapadať prachom. Čím to je, že sa naše dobré predsavzatia pokazia rýchlejšie ako kartón mlieka alebo vajec?

Mnoho predsavzatí sa nenaplní jednoducho preto, lebo sa nemodlíme. Vykročili sme odvážne, ambiciózne a aj s~istým nadšením. Možno, že v~prvý januárový deň sa za svoje predsavzatia aj modlíme tak, ako sa človek modlí v~aute pred tým, než vyrazí na dlhú cestu. No ešte skôr, než sa vôbec vydáme na diaľnicu nového roka, sme už stratili niekde za sebou modlitbu a spolu s~ňou aj silu, ktorú potrebujeme, aby sme v~akomkoľvek novom zvyku, či spôsobe vytrvali. Bez modlitby za pomoc od Boha sa aj naše najzmysluplnejšie predsavzatia rozplynú a celkom stratia. Keby aj zdanlivo uspeli, bolo by to bez toho, že by akýmkoľvek spôsobom prinášali zmysluplné svedectvo o~Bohu. Ešte skôr, než si dáte nejaké predsavzatia, rozhodnite sa pre modlitbu. Ak si žiadne predsavzatie nedáte, rozhodnite sa pre zmenu a rast vďaka modlitbe a nie vďaka svojej vlastnej vôli.

Moja nová modlitba, dôležitejšia pre budúci rok než akákoľvek iná, je táto:
„Pane, nauč ma o~sebe viac než to, čo už viem, pokor ma znova tým všetkým, čo neviem a učiň, aby to, čo viem, v~mojom živote viacej ožilo a bolo sa pravdivejšie. Pane, pomôž mi, aby som Ťa videl viac ako kedykoľvek predtým.“

Každý nový deň a každý nový rok sa začína rovnakou modlitbou: „Otvor mi oči, aby som videl zázraky Tvojho zákona.“
(Ž~119,18)  Vďaka Duchu, ktorý v~nás prebýva a neuveriteľnou nádherou slov samotného Boha pred očami, nemáme nikdy dôvod uspokojiť sa len s~tým, čo už poznáme. Jednoznačne musíme očakávať, že tento rok uvidíme a porozumieme veciam o~Bohu, ktoré sme nikdy predtým nevideli.

Nikdy sa neprestaňme modliť, aby nám Boh dal „Ducha múdrosti a zjavenia v~poznávaní Jeho a osvietil oči… srdca“, aby sme Ho spoznali lepšie, aby sme spoznávali Jeho nádej, bohatstvo a Jeho moc (Ef~1,17–18). Poproste Boha, aby sa vám tento rok zjavil skrze svoje slovo ešte viac, než kedykoľvek predtým.

Pane, odhaľ mi, ako málo Ťa ešte poznám.
Satan je natoľko prefíkaný, že aj naše poznanie Boha dokáže prekrútiť na pokušenie k~hriechu. Nepoznať Boha vždy vedie k~zlu, ale aj poznať Boha môže byť bezbožné. O~Bohu môžeme vedieť dosť na to, aby sme boli spasení, ale mnohí máme aj poznanie, vďaka ktorému sme pyšní.
Apoštol Pavol vystríha: „Ale poznanie vedie k~namyslenosti, láska však buduje. Ak si niekto myslí, že poznal niečo, ešte vždy nepoznal tak, ako treba poznať. Ale keď niekto miluje Boha, toho pozná Boh.“ (1K~8,1–3).
Je tragédiou, keď teológia, ktorá by nás mala voviesť do úplnej pokory, nám dá zvláštnym obratom pocit, že sme lepší, než naozaj sme (R 12,3).

Pravá teológia, nech už je akokoľvek sofistikovaná, dôsledná a precízna, má znieť ako chvála: „Keď hľadím na Tvoje nebesá, na dielo Tvojich prstov, na mesiac a hviezdy, ktoré si upevnil: Čo je človek, že naň pamätáš?“ (Ž~8,4–5) Ako sa vám bude Boh viac zjavovať, poproste Ho, aby vám ukázal aj to, ako málo Ho ešte poznáte a ako málo si zaslúžite poznanie, ktorého sa vám dostalo. Poproste Ho, aby ste sa pokorili.
Pane, sprav, aby to, čo o~Tebe viem, sa v~mojom srdci viac zakorenilo.

Svet nás od útleho detstva učí, aby sme ohodnotili napredovanie všetkými možnými, ale nie správnymi mierami. Dvadsať aj viac rokov sme sa učili matematiku, dejepis alebo inú vedu a rok čo rok sme samých seba hodnotili výsledkami testov a známkami na vysvedčení. No kresťanský život, to jednoducho nie je len nejaký kurz systematickej teológie. Zrelosť sa meria duchovným monitorom srdca, a nie nejakým globálnym teologickým informačným systémom. Meria sa charakterom, nie vedomosťami.

Ako teda zmeníme to, čo vieme, na to, aby sme naozaj rástli ako kresťania? Skrze modlitbu. Modlitba je iskrou, ktorá zapaľuje plameň poznania, ktoré sme časom nadobudli. Tim Keller píše, že modlitba pretvára teológiu na skúsenosť. Skrze modlitbu prežívame Jeho prítomnosť a prijímame Jeho radosť, lásku, pokoj a Jeho uistenie. Mení sa náš postoj, správanie a charakter… Modlitba je cesta, skrze ktorú sa všetko, čomu veríme a čo Kristus pre nás vybojoval, mení na silu, ktorou disponujeme. Modlitba je cesta, ktorou je pravda zapracovávaná do srdca a začínajú sa vytvárať nové inštinkty, reflexy a sklony.(Modlitba, Keller)
Príliš často milujeme viac to, čo sme sa o~Bohu naučili ako Boha samotného a
v~takom prípade sa náš život nijako podstatne nemení. Získavame viac a viac poznania, ale pritom sa vôbec nemeníme. No ak sa naozaj nikdy nemeníme, je pravda, že sme vôbec poznali Boha? Keller, opierajúc sa o~Kalvína, pokračuje: „Možno o~Bohu viete veľa, ale Boha naozaj nepoznáte dovtedy, kým poznanie toho, čo pre vás v~Ježišovi Kristovi vykonal, nezmení štruktúru základov vášho srdca.“

Viac Boha, menej pýchy a byť viac podobný Kristovi. Tak, ako slnko zapadlo nad rokom, ktorý uplynul, nech Syn povstane tak ako ešte nikdy predtým na horizonte nášho srdca.

\autor{Marshall Segal}


\clanok{Správy zo staršovstva}
Bratia a sestry, v~minulom mesiaci sa na pôde nášho zboru uskutočnilo niekoľko podujatí. Sme vďační, že náš zbor môže byť miestom, kde sa môžeme stretať pri Božom slove, Jeho oslave, ale aj pri diskusii k~otázkam dotýkajúcim sa života v~dnešnom svete. Stretli sme sa pri akcii „Rozhovory na Palisádach“, tentokrát zamerané na umelú inteligenciu. Pri príležitosti adventu sme mali štvrtkové kontemplatívne bohoslužby, ktoré boli inšpiratívne a zároveň veľmi osobné. Oslavu narodenia Pána Ježiša sme si pripomenuli detskou slávnosťou, koncertom spevokolu, ale aj bohoslužbami na Štedrý večer a Prvý sviatok vianočný. Na konci roka sme si pripomenuli, za čo všetko môžeme byť vďační, ale aj to, ako Pán Boh vypočúval naše modlitby.

Sme vďační za to, že mnohé iniciatívy v~zbore prichádzajú od členov zboru. Je to dôkaz zdravého záujmu o~život a smerovanie nášho zboru.

Naďalej prosíme o~modlitby a podporu v~našej službe.
\autor{Peter Pribula}


\clanok{Aliančný modlitebný týždeň -- Božia vernosť voči svojmu ľudu (Žalm 78)}
\table{lcll}{
\mspan4[l]{\bf Bohoslužby} \crl
Ne & 11. 1. & 17.00 & Reformovaná kresťanská cirkev, Nám. SNP 4 \cr
Po & 12. 1. & 18.00 & Cirkev bratská, Cukrová 4 \cr
Ut & 13. 1. & 18.00 & Evanjelická cirkev a.v., Schn. Trnavského 2 \cr
St & 14. 1. & 18.00 & Apoštolská cirkev Otcov dom, Račianska 72 (suterén)\cr
Št & 15. 1. & 18.00 & Bratská jednota baptistov, Palisády 27/A\cr
Pi & 16. 1. & 18.00 & Kresťanské zbory, Pri Šajbách 1 \cr
So & 17. 1. & 18.00 & Cirkev adventistov siedmeho dňa, Cablkova 3 \cr
Ne & 18. 1. & 18.00 & Evanjelická cirkev a.v., Legionárska 2 \cr
}


\clanok{Obedy s~bratmi}
Bratia Janko Szőllős, Filip Barkóczi a Dávid Chuchút prijímajú naše pozvania na obed do našich domácností. Pokračujeme aj v~novom roku 2026. Prosíme, zapíšte sa do online tabuľky.


\clanok{Zbierky v~decembri}
\table{lr}{
Na misiu					& 1~028~€ \cr
Na investičný fond		&    93~€ \cr
„Nádej hreje Ukrajinu“	& 1~500~€ \cr
}
\vskip1em

Pripomíname, že zbierky sa u~nás konajú pravidelne takto:
\vskip-1ex\begitems
* každú 2. nedeľu v~mesiaci je zbierka venovaná misii a
* každú 4. nedeľu je zbierka na investície.
\enditems

Aj naďalej máte možnosť prispieť do „nedeľnej zbierky“, a to prevodom na účet zboru. Do poznámky pre prijímateľa, prosím, uveďte „zbierka“.

Bankové spojenie: SK36 0900 0000 0000 1147 1836, SWIFT: GIBASKBX


\n 2.	1.	Ivan	KOHÚT;
\n 3.	1.	Ľubomír	KEŠJAR;
\n 11.	1.	Nataša	HOVORKOVÁ;
\n 16.	1.	Blahoslava	BETKOVÁ;
\n 19.	1.	Andrej	CIHO;
\n 22.	1.	Jana	LAURENČÍKOVÁ;
\n 22.	1.	Jana	ČAHOJOVÁ;
\n 23.	1.	Elena	GUBOVÁ;
\narodeniny


\program{
\p  1 ; št ; 10.00 ; Novoročné bohoslužby (J.~Szőllős) ;.;;
\p  2 ; pi ;.;;.;;
\p  3 ; so ;.;;.;;
\p  4 ; ne ;  9.30 ; Bohoslužby (J.~Szőllős + VP) ;.;;
\p  5 ; po ; 18.00 ; Modlitebná skupinka ;.;;
\p  6 ; ut ;.;;.;;
\p  7 ; st ;.;;.;;
\p  8 ; št ;.;;.;;
\p  9 ; pi ; 17.30 ; Dorast ;.;;
\p 10 ; so ; 18.00 ; Mládež ;.;;
\p 11 ; ne ;  9.30 ; Bohoslužby (D.~M.~Chuchút) ; 15.30 ; ZČZ ;
\p 12 ; po ; 18.00 ; Modlitebná skupinka / Skupinka „Rastieme v~Kristovi“ ;.;;
\p 13 ; ut ; 15.00 ; Biblická hodina pre seniorov (P.~Pivka) ;.;;
\p 14 ; st ;.;;.;;
\p 15 ; št ; 18.00 ; AMOT na Palisádach ;.;;
\p 16 ; pi ; 17.30 ; Dorast ;.;;
\p 17 ; so ; 18.00 ; Mládež ;.;;
\p 18 ; ne ;  9.30 ; Bohoslužby (F.~Barkóczi) / začína besiedka ;.;;
\p 19 ; po ; 18.00 ; Modlitebná skupinka ;.;;
\p 20 ; ut ; 15.00 ; Biblická hodina pre seniorov (P.~Pivka) ;.;;
\p 21 ; st ; 17.30 ; Stretnutie sestier ;.;;
\p 22 ; št ; 18.00 ; Biblická hodina ;.;;
\p 23 ; pi ; 17.30 ; Dorast ;.;;
\p 24 ; so ; 18.00 ; Mládež ;.;;
\p 25 ; ne ;  9.30 ; Bohoslužby (V.~Pototskyi) ;.;;
\p 26 ; po ; 18.00 ; Modlitebná skupinka / Skupinka „Rastieme v~Kristovi“ ;.;;
\p 27 ; ut ; 15.00 ; Biblická hodina pre seniorov (P.~Pivka) ;.;;
\p 28 ; st ;.;;.;;
\p 29 ; št ; 18.00 ; Biblická hodina ;.;;
\p 30 ; pi ; 17.30 ; Dorast ;.;;
\p 31 ; so ; 18.00 ; Mládež ;.;;
}


\tiraz
\bye
