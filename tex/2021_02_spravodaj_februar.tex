%\typosize[10/12.5]% - pouzita velkost pisma/riadku - trochu vacsie
\input makra.tex % nacitanie Ivanom pripravenych nastaveni a prikazov
\hyphenation{star-šov-stvo} % rozdelenie slov na konci riadku, treba tu uviest slova, ktore sam nepozna

\spravodaj{2}{2021}


\clanok {Zachovávajme jednotu ducha vo zväzku pokoja}
Satan je majstrom rozbrojov a použije všetko, aby ľudí rozdeľoval. Podarilo sa mu rozdeliť nás lockdownom a bezpečnostnými opatreniami. Dokonca sa mu podarilo spôsobiť hanbu a pocit viny, aby nás rozdeľoval, keď sa niekto, koho poznáme, nakazil vírusom, a my posudzovali, čo mal alebo nemal robiť. Obávam sa, že v~najbližších dňoch znovu budeme vidieť, ako sa satan bude pokúšať nás rozdeliť do dvoch skupín: tých, ktorí sa zaočkovali, a tých, ktorí nie. Z~rôznych dôvodov sa ľudia rozhodnú pre jedno alebo druhé a náš zbor sa ocitne v~dvoch skupinách. To je realita.

Nemusíme byť však rozdelení. Nemôžeme si dovoliť, aby satan zničil našu jednotu. Musíme si navzájom dôverovať, modliť sa jeden za druhého a zjednotiť sa ako rodina, aby sme prešli touto skúškou a utrpením v~našom zbore. Vyzývam každého z~nás, aby sme sa nestali nástrojom rozdelenia v~satanových rukách.

V inej súvislosti, ktorá sa týkala iného problému, nás Pavol varuje pred takýmto rozdelením na základe osobných rozhodnutí. V~liste Rimanom 14,4-7 píše: {\it „Kto si ty, že súdiš cudzieho sluhu? On s~vlastným pánom stojí alebo padá. Bude však stáť, lebo Pán ho môže postaviť.  Niekto pokladá určitý deň za dôležitejší ako iný, druhý hodnotí každý deň rovnako. Každý nech je úplne presvedčený o~svojom presvedčení. Kto zachováva určitý deň, zachováva ho pre Pána. Kto je, robí to na česť Pána, lebo ďakuje Bohu, a kto neje, robí to tiež na česť Pánovi, lebo aj on ďakuje Bohu. Lebo nikto z~nás nežije sebe samému a nikto sebe samému neumiera.“} Musíme si navzájom dopriať slobodu nechať sa viesť Duchom dôverovať rozhodnutiam, ku ktorým pristúpime, a to bez kritiky a posudzovania. Môže sa to stať veľmi ľahko, najmä v~cirkvi, že čelíme satanovým útokom, aby nás rozdelil. Preto Pavol píše Efezanom: {\it „Usilujte sa zachovávať jednotu ducha vo zväzku pokoja. Jedno je telo a jeden Duch, ako ste aj boli povolaní k~jednej nádeji svojho povolania“} (Ef 4,3-4).

Keď to urobíme, budeme sa chrániť pred útokmi diabla a budeme pokračovať v~našom úsilí budovať ten najláskavejší zbor v~Bratislave. Budem sa naďalej modliť za vás a za Božiu milosť, aby sa nám podarilo dosiahnuť tento cieľ.

\autor{Danny Jones}


\clanok {Správy zo staršovstva}
Aj naďalej je pri stretnutiach staršovstva prioritnou témou zabezpečenie života zboru.

Prvý mesiac roku 2021 sa niesol v~znamení obmedzení spôsobených šíriacim sa vírusom.

Napriek obmedzeniam sme začali pripravovať výročné celozborové členské zhromaždenie. Spolu s~vedúcimi zborových zložiek pripravujeme výročné správy za jednotlivé zložky a vyzvali sme členov zboru k~návrhom do zborového rozpočtu na rok 2021. Termín konania zborového členského zhromaždenia plánujeme koncom marca.

V januári bola dokončená nová webová stránka zboru, ktorú sme začali používať.

Aj naďalej nás všetkých voláme k~tomu, aby sme udržiavali, alebo skôr rozširovali naše vzájomné povzbudzovanie sa prostredníctvom dostupných technológií. Potrebujeme sa navzájom povzbudzovať v~živote viery a vernosti k~nášmu nebeskému Otcovi.

Chceme vyjadriť vďaku všetkým vám, ktorí aj v~tomto neľahkom období venujete svoj čas, prácu a energiu pre život nášho zboru.

\autor {za staršovstvo Peter Pribula st.}


\clanok {Výzva k~modlitbám}
Keď sa dvaja alebo traja spolu modlia, stane sa niečo špeciálne a mocné. Samozrejme, Boh počuje modlitbu jednotlivca, ale keď sa modlíme spolu, Ježiš sám povedal, že sa deje niečo iné: „Amen, opäť vám hovorím: Ak dvaja z~vás budú na zemi jednomyseľne prosiť o~čokoľvek, dostanú to od môjho Otca, ktorý je v~nebesiach“ (Mt 18,19). Spoločná modlitba slúži aj nám na povzbudenie a budovanie. Už mesiace sme izolovaní a možností na spoločné modlitby je čoraz menej. Popritom sledujeme zhoršovanie zdravotnej situácie na Slovensku.

Z toho dôvodu staršovstvo vyzýva celý zbor k~pravidelným modlitbám. Každý týždeň v~pondelok medzi 18.00~hod. a 20.00~hod. zavolaj niekomu a modlite sa spolu za Slovensko, za zdravotnú situáciu a stav nemocníc, za vládu a múdrosť pri rozhodovaní, za duševný a duchovný stav ľudí okolo nás. Je celkom na Tebe, komu zavoláš a za čo sa budete modliť. Netreba sa modliť celý čas ani dlho. Nič oficiálne nebudeme organizovať. Je to na Tebe. Bude však dobre, keď sa celý zbor zjednotíme pred Bohom v~modlitbe a spoločne Ho budeme prosiť o~pomoc. Každý týždeň to môže byť s~niekým iným, alebo aj vždy s~tým istým modlitebným partnerom. Modlitbou dokážeme ovplyvniť veľa. Nech je to na Jeho slávu!
\vfill\break


\clanok {Rozpočet zboru na rok 2021}
V týchto dňoch staršovstvo spolu s~účtovníčkou zboru pripravujú rozpočet na tento rok. Do 14.~februára môžete posielať svoje podnety a návrhy do rozpočtu na e-mailovú adresu \email {starsovstvo@bjbpalisady.sk}. Členovia zboru budú mať takisto možnosť zapojiť sa do online diskusie k~príprave rozpočtu 2.~a~16.~marca o~19.00~hod. Záujemcovia o~túto diskusiu sa môžu prihlásiť na uvedenej e-mailovej adrese.


\clanok {Kairos}
Ako bolo v~januárovom spravodaji oznámené, tento rok máme možnosť zúčastniť sa misijného kurzu {\it Kairos} online. Je to kurz, ktorý nám pomôže lepšie porozumieť tomu, čo v~tomto svete Boh robí a prečo to robí. Kairos nám pomôže žiť každodenný život s~jasným zámerom. Pomocou kurzu Kairos môžeme takisto vidieť, čo Boh koná v~národoch sveta a ako sa môžeme k~tomuto dielu pripojiť.

Kurz Kairos bude prebiehať počas dvoch víkendov, a to v~termínoch 11.~--~13. a 18.~--~20.~februára. Viac informácií nájdete na \ulink [https://www.facebook.com/KurzKairos/]{facebook.com/KurzKairos}.


\clanok {Seminár pre pracovníkov s~deťmi a dorastom}
V dňoch 11.~--~14.~marca sa uskutoční seminár pre tých, ktorí sa venujú práci s~deťmi a dorastencami. Toto školenie sa uskutoční prostredníctvom štúdia biblickej knihy Jonáš. Viac informácií o~tejto službe i o~tomto tréningu nájdete na webovej stránke: \ulink [https://preceptslovakia.estranky.sk]{preceptslovakia.estranky.sk} (vzdelávací program PLA).

Koordinátorom tejto služby na Slovensku je br. kazateľ Darko Kraljik.
\vfill\break


\clanok {Národný týždeň manželstva}
V týždni od 7. do 14. februára máme počas Národného týždňa manželstva možnosť venovať sa rôznym aktivitám na upevnenie našich manželstiev, ktoré pripravila Služba manželským párom BJB spolu s~národným tímom NTM. Viac informácií nájdete na webovej stránke \ulink [https://www.ntm.sk]{ntm.sk}.


\clanok {Zbierka pre Etiópiu}
V dôsledku nepokojov a násilia v~oblasti Tigray na severe Etiópie utiekli zo svojich domovov desaťtisíce ľudí. Nemajú takmer nič a utekajú v~strachu o~svoj život. Ako zbor by sme sa chceli zapojiť do zbierky spolu s~Nadáciou Integra a pomôcť ľuďom na úteku. Viac informácií nájdete na našej zborovej stránke: \ulink [https://www.bjbpalisady.sk/etiopia]{\hbox{bjbpalisady.sk/etiopia}}.

Ďakujeme.


\clanok {Plánované letné tábory}
Zborový tábor plánujeme v~termíne 15.~--~21.~augusta~2021 v~stredisku Detskej misie v~Častej-Papierničke.

Dorastenecký tábor je predbežne plánovaný v~termíne 29.~7.~--~1.~8.~2021 na Chvojnici.


\clanok {Ak potrebujete pomoc, napíšte nám!}
V našom zbore sme zriadili emailovú adresu \email{pomoc@bjbpalisady.sk}, na ktorú môžete napísať, ak ste sa dostali do zlej situácie alebo potrebujete nejakú pomoc. Takisto sa môžete ozvať, ak ste ochotní s~niečím pomôcť.
\vfill\break


\clanok{Verš na zapamätanie}
Tento mesiac máme nový veršík, ktorý sa chceme spoločne učiť. Veríme, že poznanie Písma prospeje našej duši i našej mysli:

{\it „Počúvajte, poviem vám tajomstvo: Nie všetci zomrieme, ale všetci budeme premenení, naraz, v~tom okamihu, na posledný zvuk poľnice. Keď zatrúbi, mŕtvi budú vzkriesení neporušiteľní a my budeme premenení.“}

\autor{1Kor~15,~51~--~52}


\clanok{Zbierky za uplynulé obdobie}
Milí bratia a sestry,

v januári ste prispeli:

\vskip-1ex\begitems
* Misia: 378,00 €
* Investície: 378,00 €

\enditems

Ďakujeme vám, že napriek okolnostiam a neistým ekonomickým vyhliadkam do budúcnosti, ste mnohí prispeli na činnosť a službu zboru. Aj naďalej máte možnosť prispieť do „nedeľnej zbierky“, a to prevodom na účet zboru. Do poznámky pre prijímateľa, prosím, uveďte „zbierka“.

Bankové spojenie: SK36 0900 0000 0000 1147 1836, SWIFT: GIBASKBX

Ďakujeme!


\n 3.	2.	Vlasta	BALÁŽOVÁ;
\n 3.	2.	Margita	KRÁĽOVÁ;
\n 3.	2.	Miroslav	ANTALÍK;
\n 5.	2.	Štefánia	ANTALÍKOVÁ;
\n 5.	2.	Barbora	ANTALÍKOVÁ;
\n 11.	2.	Juraj	BALÁŽ;
\n 11.	2.	Oľga	KOVÁČOVÁ;
\n 11.	2.	Beáta	BOGÁROVÁ;
\n 12.	2.	Martin	PRIBULA;
\n 13.	2.	Zlatica	VYSKOČILOVÁ;
\n 15.	2.	Ingrid	JANČULOVÁ;
\n 16.	2.	Lenka	PRIBULOVÁ;
\n 23.	2.	Anna	PLETT;
\narodeniny


\program{
\p  1 ; po ;.;;.;;
\p  2 ; ut ;.;;.;;
\p  3 ; st ;.;;.;;
\p  4 ; št ;.;;.;;
\p  5 ; pi ; 17.30 ; Dorast (Zoom) ;.;;
\p  6 ; so ;.;;.;;
\p  7 ; ne ; 10.30 ; Bohoslužby (J. Szőllős, online) ; 14.00 ; Veľká besiedka (FB Messenger) ;
\p  8 ; po ; 18.00 ; Modlitby za Slovensko (individuálne, telefonicky) ;.;;
\p  9 ; ut ;.;;.;;
\p 10 ; st ;.;;.;;
\p 11 ; št ;.;;.;;
\p 12 ; pi ; 17.30 ; Dorast (Zoom) ;.;;
\p 13 ; so ;.;;.;;
\p 14 ; ne ; 10.30 ; Bohoslužby (D. Jones, online) ;.;;
\p 15 ; po ; 18.00 ; Modlitby za Slovensko (individuálne, telefonicky) ;.;;
\p 16 ; ut ;.;;.;;
\p 17 ; st ;.;;.;;
\p 18 ; št ;.;;.;;
\p 19 ; pi ; 17.30 ; Dorast (Zoom) ;.;;
\p 20 ; so ;.;;.;;
\p 21 ; ne ; 10.30 ; Bohoslužby (D. Jones, online) ; 14.00 ; Veľká besiedka (FB Messenger) ;
\p 22 ; po ; 18.00 ; Modlitby za Slovensko (individuálne, telefonicky) ;.;;
\p 23 ; ut ;.;;.;;
\p 24 ; st ;.;;.;;
\p 25 ; št ;.;;.;;
\p 26 ; pi ; 17.30 ; Dorast (Zoom) ;.;;
\p 27 ; so ;.;;.;;
\p 28 ; ne ; 10.30 ; Bohoslužby (D. Jones, online) ;.;;
}


\tiraz
\bye
