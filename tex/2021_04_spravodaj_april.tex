%\typosize[9/12]% - pouzita velkost pisma/riadku
\input makra.tex % nacitanie Ivanom pripravenych nastaveni a prikazov
\hyphenation{star-šov-stvo} % rozdelenie slov na konci riadku, treba tu uviest slova, ktore sam nepozna

\spravodaj{4}{2021}


\clanok {Nevidíme veľa vecí, ony však existujú}
Veľkonočný kríž má v~sebe niečo desivé. Prizerať sa najhroznejšiemu spôsobu popravy vo vtedajšom antickom svete a ešte k~tomu Toho najčistejšieho z~ľudského rodu v~nás môže vyvolať aj strach z~nás samých. Potrebujeme sa báť aj samých seba a najmä svojej nepredvídateľnosti, nespoľahlivosti a ľahostajnosti. Človek sa musí najprv zhroziť sám zo seba, aby sa mohol dostať na správnu cestu, alebo späť na správnu cestu v~prípade jej opustenia.

Nie je to však len des, ktorý prežívame pri pohľade na kríž. Z~kríža sa na nás totiž nepozerá nešťastný  zúfalec. Pozerá sa na nás dobrota samého Boha, ktorý sa vydal do ľudských rúk, aby mohol niesť všetky krízy dejín, ale aj ťarchu budúcich udalostí. To Boh sám je na ňom ukrižovaný a tým nám hovorí, že tento zdanlivo premožený Boh je nepochopiteľne odpúšťajúci, milujúci, dobrotivý a vo svojej zdanlivej neprítomnosti je obrovsky silný.

Je veľmi mnoho vecí, ktoré nevidíme, ale ony predsa existujú a sú ešte k~tomu mimoriadne dôležité. Elektrický prúd nevidíme a pohľad na blankytnú oblohu nám takisto nezastiera hustá sieť práve prebiehajúcich mobilných hovorov a zemskú príťažlivosť takisto nevídať, a predsa to všetko existuje veľmi blízko nás. Netreba ani len domýšľať možné devastačné následky niekoľkominútového výpadku účinkov gravitácie našej planéty na nás, jej obyvateľov. Vecou, ktorá sa nás osobne viac týka, je náš um, o~ktorom evidentne nepochybujeme. Nevidíme síce svoje myšlienky, a predsa ich máme. Nevidíme a dokonca ani iní nemôžu vidieť alebo „namerať“ ani naše duševné schopnosti, a predsa existujú; pozorujeme totiž dôkazy ich pôsobenia, pretože môžeme hovoriť, myslieť, rozhodovať sa. Takže práve tie najhlbšie veci, ktoré sú základom tohto nášho sveta a bytia nevidíme, ale môžeme pozorovať a vnímať dôkazy ich pôsobenia.

Každý z~nás však môže byť pokúšaný podobným spôsobom, ako bol učeník Pánov Tomáš. Bolesť, zlo, nespravodlivosť, smrť, toto všetko vystavuje našu vieru skúškam. A~predsa je pre nás, paradoxne, práve v~týchto prípadoch Tomášova nedôverčivosť užitočná a cenná, pretože nám pomáha zbaviť sa  všetkých falošných predstáv o~Bohu a vedie nás k~objaveniu Jeho autentickej tváre, tváre Boha, ktorý na seba v~Kristovi vzal všetky rany boľavého ľudstva. Tomáš prijal nakoniec od Pána dar viery preverenej Ježišovým utrpením a smrťou, potvrdenej stretnutím s~Ním ako Vzkrieseným Pánom, a odovzdal tento dar ďalej aj nám. Dar viery, ktorá u~Tomáša skoro zomrela, sa teraz kontaktom s~Kristovými ranami oživuje.

Aj nám, Jeho učeníkom tejto doby, Pán Ježiš Kristus ukazuje svoje rany v~každej našej núdzi a utrpení. Nech Mu je za to vzdaná chvála a česť.

\autor{Miroslav Kolářik}

\clanok {Správy zo staršovstva}
Staršovstvo zboru sa stretlo prostredníctvom aplikácie Zoom v~mesiaci marec trikrát, a to 2., 16. a 30.

Pravidelnou súčasťou našich stretnutí býva diskusia na tému kontaktu a starostlivosti o~členov a priateľov nášho zboru počas lockdownu nielen po duchovnej, ale aj po duševnej, materiálnej a organizačnej stránke. Taktiež je naďalej prioritnou témou zabezpečenie života zboru v~súčasnej pandemickej situácii.

Nemálo času venovalo staršovstvo riešeniu rôznych nových i starších pastoračných otázok. Okrem iného sme spolu diskutovali o~tom, ako komunikovať deťom zložité životné situácie, ktorých sú svedkami, či už priamo v~rodine alebo v~jej širšom okruhu.

Neľahkú situáciu v~rodine prežívajú aj Jonesovci. Nevyhnutným sa stalo, aby Clara odišla pomôcť deťom do USA a koncom apríla ju bude nasledovať aj Danny so súhlasom a odporúčaním staršovstva. Túto pomoc spoja s~plánovanou dovolenkou a už začiatkom leta by mali byť opäť medzi nami. Neznamená to, že sa náš kazateľ odmlčí. Naďalej bude pracovať tak ako doposiaľ v~tejto situácii, cez internet, Zoom, e-maily a pod.

Okrem iných veci sme zostavovali návrh zborového rozpočtu, pripravovali sme podklady na zmenu zborového poriadku, ďalšiu modernizáciu prenosovej techniky, pripravovali sme sa na DKDZ a opätovne otvorili otázku hľadania ďalšieho kazateľa do nášho zboru.

Ďakujeme za vašu podporu a modlitby. Prosíme, modlite sa nielen za nás, ale aj za situáciu, v~ktorej sa ako ľudstvo nachádzame, a tiež za oblasti, ktoré som zmienil vyššie.

Ďakujem.

\autor {za staršovstvo Peter Antalík}


\clanok {Bohoslužby počas veľkonočných sviatkov}
Na Veľký piatok 2. apríla o~17.00~hod. bude spoločná veľkonočná bohoslužba BJB v~SR. Slovom bude slúžiť náš br.~kazateľ Danny Jones.

V pripravovanej spoločnej bohoslužbe budeme môcť aspoň cez obrazovky navštíviť Banskú Bystricu, Bratislavu a Košice.

Prenos bohoslužby bude na YouTube: \ulink[https://www.youtube.com/watch?v=3wP_gsuTCHA]{youtube.com/watch?v=3wP\_gsuTCHA}. Jej súčasťou bude aj Večera Pánova, a preto si, prosíme, pripravte chlieb a pohár (prípadne poháriky) s~vínom alebo hroznovou šťavou.

Na veľkonočnú nedeľu 4.~apríla o~10.30~hod. nám zvesťou Božieho Slova bude slúžiť br. kazateľ Danny Jones. Takisto bude vysluhovaná Večera Pánova.

Prenos týchto bohoslužieb môžete sledovať online len tu: \ulink[https://bit.ly/3cgSMBG]{bit.ly/3cgSMBG}.
\vfill\break


\clanok {Pomoc ľuďom v~núdzi}
Kresťania v~meste hľadajú dobrovoľníkov na výdaj potravín na Ambroseho~6.

\vskip-1ex\begitems
* Pondelok 9.00 -- 11.00~hod.
* Streda 17.00 -- 19.00~hod.

\enditems

Kontaktná osoba: Vojtech Sirkovský (0951 524 561)

Viac informácií o~projekte môžete nájsť na webovej stránke \ulink[https://www.krestaniavmeste.sk/vydajna/]{krestaniavmeste.sk/vydajna}.


\clanok {Ak potrebujete pomoc, napíšte nám!}
V našom zbore sme zriadili e-mailovú adresu \email{pomoc@bjbpalisady.sk}, na ktorú môžete napísať, ak ste sa dostali do zlej situácie alebo potrebujete nejakú pomoc. Takisto sa môžete ozvať, ak ste ochotní s~niečím pomôcť.

Ak ste boli pozitívne testovaní na koronavírus a potrebujete pomoc so zabezpečením nákupov potravín či liekov, dajte nám vedieť.

V zbore sme zakúpili niekoľko kusov oximetrov na meranie saturácie kyslíku v~krvi. V~prípade potreby je ich možné zapožičať.


\clanok{Verš na zapamätanie}
Tento mesiac máme nový veršík, ktorý sa chceme spoločne učiť. Veríme, že poznanie Písma prospeje našej duši i našej mysli:

{\it „Kto chce byť medzi vami prvý, bude sluha všetkých. Lebo ani Syn človeka neprišiel, aby sa dal obsluhovať, ale aby sám slúžil a dal svoj život ako výkupné za mnohých.“}

\autor{Mk~10,~44 -- 45}


\clanok{Zbierky za uplynulé obdobie}
Milí bratia a sestry, v marci ste prispeli:

\vskip-1ex\begitems
* Misia: 448,00 €
* Investície: 398,00 €

\enditems

Ďakujeme vám, že napriek okolnostiam a neistým ekonomickým vyhliadkam do budúcnosti, ste mnohí prispeli na činnosť a službu zboru. Aj naďalej máte možnosť prispieť do „nedeľnej zbierky“, a to prevodom na účet zboru. Do poznámky pre prijímateľa, prosím, uveďte „zbierka“.

Bankové spojenie: SK36 0900 0000 0000 1147 1836, SWIFT: GIBASKBX

Ďakujeme!


\n 1.	4.	Miroslav	KOLÁŘIK;
\n 4.	4.	Vierka	ŠKODÁK;
\n 6.	4.	Jana	ZAJACOVÁ;
\n 6.	4.	Jarmila	CIHOVÁ;
\n 10.	4.	Anna	PAVLÍKOVÁ;
\n 11.	4.	Daniel	MIKLETIČ;
\n 16.	4.	Clara	JONES;
\n 19.	4.	Marta	PRIBULOVÁ;
\n 22.	4.	Alexander	Koloman	ERDÉLYI;
\n 25.	4.	Elena	TALIGOVÁ;
\n 30.	4.	Jaroslav	VOLENTIČ;
\n 30.	4.	Ľuboš	DZURIAK;
\narodeniny


\program{
\p  1 ; št ;.;;.;;
\p  2 ; pi ; 17.00 ; Bohoslužby -- Veľký piatok (D. Jones, online) ;.;;
\p  3 ; so ;.;;.;;
\p  4 ; ne ; 10.30 ; Bohoslužby -- Veľkonočná nedeľa (D. Jones, online) ;.;;
\p  5 ; po ; 18.00 ; Modlitby za Slovensko (individuálne, telefonicky) ;.;;
\p  6 ; ut ;.;;.;;
\p  7 ; st ;.;;.;;
\p  8 ; št ;.;;.;;
\p  9 ; pi ; 17.30 ; Dorast (Zoom) ;.;;
\p 10 ; so ;.;;.;;
\p 11 ; ne ; 10.30 ; Bohoslužby (P. Kolárovský, online) ; 14.00 ; Veľká besiedka (FB Messenger) ;
\p 12 ; po ; 18.00 ; Modlitby za Slovensko (individuálne, telefonicky) ;.;;
\p 13 ; ut ;.;;.;;
\p 14 ; st ;.;;.;;
\p 15 ; št ;.;;.;;
\p 16 ; pi ; 17.30 ; Dorast (Zoom) ;.;;
\p 17 ; so ;.;;.;;
\p 18 ; ne ; 10.30 ; Bohoslužby (D. Jones, online) ;.;;
\p 19 ; po ; 18.00 ; Modlitby za Slovensko (individuálne, telefonicky) ;.;;
\p 20 ; ut ;.;;.;;
\p 21 ; st ;.;;.;;
\p 22 ; št ;.;;.;;
\p 23 ; pi ; 17.30 ; Dorast (Zoom) ;.;;
\p 24 ; so ;.;;.;;
\p 25 ; ne ; 10.30 ; Bohoslužby (R. Krupa, online) ; 14.00 ; Veľká besiedka (FB Messenger) ;
\p 26 ; po ; 18.00 ; Modlitby za Slovensko (individuálne, telefonicky) ;.;;
\p 27 ; ut ;.;;.;;
\p 28 ; st ;.;;.;;
\p 29 ; št ;.;;.;;
\p 30 ; pi ; 17.30 ; Dorast (Zoom) ;.;;
}


\tiraz
\bye
