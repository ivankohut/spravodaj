%\typosize[9/12]% - pouzita velkost pisma/riadku - standard
\input makra.tex % nacitanie Ivanom pripravenych nastaveni a prikazov
\hyphenation{star-šov-stvo} % rozdelenie slov na konci riadku, treba tu uviest slova, ktore sam nepozna

\spravodaj{5}{2020}


\clanok {\it Comunitas}
Jeden z~významov latinského slova {\it comunitas} označuje skupinu ľudí, ktorá spolu zažíva zmenu na prahu niečoho nového. Zdá sa mi, že {\it comunitas} sa teraz u~nás vytvára. Cítim sa možno podobne ako Jozua s~Izraelitmi tesne pred príchodom do zasľúbenej krajiny. Sme v~situácii, z~ktorej by sme sa najradšej čo najskôr dostali. Akoby ten koronavírus tu bol už takmer štyridsať rokov. Veríme, že nás čaká požehnanie a dobro, ale nevieme, aké to presne bude, kedy sa tam dostaneme a po akej ceste pôjdeme. Z~toho neznámeho pociťujeme neistotu a strach. V~tejto súvislosti sú preto nesmierne relevantné slová, ktorými Pán Boh povzbudil Jozuu a Izraelitov: {\it „Buď silný a odvážny, nestrachuj sa a nezúfaj!“} Keby to bolo všetko, čo Pán povedal, stačilo by to; ale povedal mu ešte niečo iné, niečo dôležitejšie. Možno mu to povedal preto, že Jozua bol svedkom Mojžišovho sporu s~Bohom. Pán Boh sa vtedy nahneval na Izraelitov. Mojžiš vtedy stratil istotu, či Boh pôjde s~nimi, a preto Mu povedal: {\it „Ak nepôjdeš s~nami, ani nás odtiaľto nevyvádzaj! Ako inak poznám, že som spolu s~Tvojím ľudom získal Tvoju priazeň, ak nie podľa toho, že pôjdeš s~nami? Tým sa budem spolu s~Tvojím ľudom líšiť od ostatných národov na zemi.“}

Jozua vtedy spoznal, že nie je nič dôležitejšie ako Božia prítomnosť. Keď ideme spolu s~Pánom, ani veľmi nevadí, keď nevieme, kam, ako a kedy pôjdeme. A~tak Boh Jozuovi a Izraelitom povedal niečo, čo bolo ešte dôležitejšie: {\it „Budem s~tebou, ako som bol s~Mojžišom. Nenechám ťa ani ťa neopustím. Buď silný a odvážny, nestrachuj sa a nezúfaj! Veď Hospodin, tvoj Boh, bude s~tebou, kamkoľvek pôjdeš.“}

Nič viac nepotrebujeme, zvlášť teraz, keď na Slovensku odpovede nemáme. Čakajú nás nové boje, ale Hospodin, náš Boh, bude s~nami, kamkoľvek pôjdeme. Možno narastie počet nakazených a všetko sa znovu zastaví, ale Hospodin, náš Boh, bude s~nami, kamkoľvek pôjdeme. Ekonomika je obrovská záhada budúcnosti, ale Hospodin, náš Boh, bude s~nami, kamkoľvek pôjdeme. Je možné, že sa už nikdy nevrátime do tej bývalej normy, ktorú poznáme, ale Hospodin, náš Boh, bude s~nami, kamkoľvek pôjdeme. Našou jedinou konštantou je Boh sám a Jeho prítomnosť. {\it „Nenechám ťa ani ťa neopustím. Preto buď silný a odvážny, nestrachuj sa a nezúfaj!“}

{\it Comunitas} -- to je to, kam sme sa teraz dostali. To staré pominulo, ale nové ešte nenastalo. Veľa nevieme, ale krok po kroku nám náš verný Pán ukáže, čo ďalej. Verím tomu! Preto sa teším, že sme v~tom spolu.

\autor{Danny Jones}


\clanok {Správy zo staršovstva}
Je to už niekoľko týždňov, čo to okolo nás naoko vyzerá, ako keby sa svet spomalil, ba až skoro zastavil. Počúvame o~tom, aké je to ťažké a v~niektorých prípadoch priam tragické. No napriek tomu prežívame nepochopiteľnú ochranu nášho Nebeského Otca. Uvedomujeme si to, a preto sme za to vďační.

Veľa našich tém, o~ktorých sme v~minulosti hovorili, sme museli odložiť a začať s~novými témami. Tie nové boli pre nás nepoznané. Postupne sa učíme ako sa komunikuje bez osobného kontaktu. Toto mi pripomína situáciu po vzkriesení Pána Ježiša a Jeho návrate do neba k~Otcovi. Aby sme nezostali sami, poslal Svätého Ducha. On je s~nami, potešuje nás, ukazuje nám, čo nie je dobré a aj to, u~koho môžeme hľadať pomoc a riešenie. Sme s~Ním spojení skrze Svätého Ducha.

Nás dnes navzájom spája technika. Tú tiež máme iba vďaka Nemu. Niekomu dal toľko múdrosti a nápadov, aby sme mohli dnes maximálne využívať tieto prostriedky. Uvedomujeme si, že napriek technike, ktorú máme, to najlepšie je, že sa môžeme jeden za druhého modliť.

S týmto posolstvom sme začínali posledné staršovstvo v~apríli. Starostlivosť o~zverené stádo. Technika a modlitby.

Máme vás všetkých radi.

\autor {za staršovstvo zboru Peter Pribula st.}


\clanok {Zborový život počas koronavírusu}
V máji budeme pokračovať v~našich online stretnutiach na Webex-e, a to každú stredu a nedeľu o~20.00~hod. s~br. kazateľom Dannym Jonesom a každý pondelok a štvrtok o~20.00~hod. s~br. Danielom Plettom. V~priebehu dňa vždy dostanete prostredníctvom e-mailu bližšie informácie o~stretnutí a možnosti prihlásenia sa.

Počas najbližších dní či týždňov očakávame isté zmeny v~opatreniach, ktoré sa týkajú stretávania sa na bohoslužbách a iných zborových aktivítách. O~všetkých zmenách budeme informovať prostredníctvom e-mailu ako aj počas prenosu nedeľných bohoslužieb.

Prenosy bohoslužieb z~nášho zboru môžete sledovať ako obyčajne prostredníctvom živého prenosu nášho zboru: \ulink[https://bit.ly/3aL34co]{https://bit.ly/3aL34co}.


\clanok {Lúčime sa s~Kristínkou Jones}
V utorok 5.~mája sa Kristínka Jones vráti späť do Spojených štátov a bude bývať v~meste Conway v~štáte Arkansas, kde bývajú aj Hudson, Hanička a Emilka. Počas leta bude bývať spolu so štyrmi kamarátkami zo strednej školy a bude sa snažiť si nájsť prácu na leto. V~auguste nastúpi na štúdium na University of Central Arkansas v~Conwayi. Bude bývať na internáte a študovať anglický jazyk a literatúru. V~budúcnosti by rada učila angličtinu. Teší sa, že bude súčasťou zboru v~Summite, kde Hudson slúži ako pastor pre chválospevy a uctievanie. Plánuje sa zapojiť do služby spolu s~Hudsonom vo vedení chvál.


\clanok{Verš na zapamätanie}
Na mesiac máj máme nový veršík, ktorý sa chceme spoločne učiť. Veríme, že poznanie Písma prospeje našej duši i našej mysli:

{\it „Kroky muža vedie Hospodin, ako môže človek svojej ceste rozumieť?“}

\autor{Príslovia~20,~24}


\clanok{Zbierky za uplynulé obdobie}
Milí bratia a sestry,

v súčasnom období náš zbor funguje v~tzv. rozpočtovom provizóriu. To znamená, že pokým rozpočet na rok 2020 nie je schválený zborovým členských zhromaždením, vychádzame pri našich výdavkoch z~minuloročného rozpočtu. Zároveň sme sa ako staršovstvo uzniesli, že naše výdavky v~danom mesiaci neprekročia príjmy zo zbierok a darov z~uplynulého mesiaca.

V apríli ste prispeli:

\vskip-1ex\begitems
* Misia: 318,75~€
* Investície: 318,75~€

\enditems

Ďakujeme vám, že napriek okolnostiam a neistým ekonomickým vyhliadkam do budúcnosti ste mnohí prispeli na činnosť a službu zboru. Aj naďalej máte možnosť prispieť do „nedeľnej zbierky“, a to prevodom na účet zboru. Do poznámky pre prijímateľa, prosím, uveďte „zbierka“.

Bankové spojenie: SK36 0900 0000 0000 1147 1836,  SWIFT: GIBASKBX

Ďakujeme!


\n 1.	5.	Milica	MALÁ;
\n 1.	5.	Andrea	ČURILLOVÁ;
\n 3.	5.	Dárius	KRÁĽ;
\n 4.	5.	Peter	BUZÁŠ ml.;
\n 8.	5.	Vladimír	KRAJČÍ;
\n 8.	5.	Jana	ŠEBO;
\n 11.	5.	Želmíra	PRAŽENICOVÁ;
\n 16.	5.	Ján	SZŐLLŐS;
\n 17.	5.	Lenka	KOVÁČOVÁ;
\n 17.	5.	Lívia	KOLÁŘIKOVÁ;
\n 18.	5.	Anna	DANTEROVÁ;
\n 19.	5.	Oľga	VALCHÁŘOVÁ;
\n 20.	5.	Rastislav	PAULEN;
\n 30.	5.	Miluška	BALÁŽOVÁ;
\narodeniny


\program{
\p  1 ; pi ; 14.00 ; Sobáš Ľubky Kráľovej a Janka Kováčika (online) ;.;;
\p  2 ; so ;.;;.;;
\p  3 ; ne ;  9.30 ; Bohoslužby (D. Jones, online); 20.00 ; Online stretnutie (D. Jones, Webex) ;
\p  4 ; po ; 20.00 ; Online stretnutie (D. Plett, Webex) ;.;;
\p  5 ; ut ;.;;.;;
\p  6 ; st ; 20.00 ; Online stretnutie (D. Jones, Webex) ;.;;
\p  7 ; št ; 20.00 ; Online stretnutie (D. Plett, Webex) ;.;;
\p  8 ; pi ;.;;.;;
\p  9 ; so ;.;;.;;
\p 10 ; ne ;  9.30 ; Bohoslužby (D. Jones, online) ; 20.00; Online stretnutie (D. Jones, Webex) ;
\p 11 ; po ; 20.00 ; Online stretnutie (D. Plett, Webex) ;.;;
\p 12 ; ut ;.;;.;;
\p 13 ; st ; 20.00 ; Online stretnutie (D. Jones, Webex) ;.;;
\p 14 ; št ; 20.00 ; Online stretnutie (D. Plett, Webex) ;.;;
\p 15 ; pi ;.;;.;;
\p 16 ; so ;.;;.;;
\p 17 ; ne ;  9.30 ; Bohoslužby (T. Papp, online); 20.00 ; Online stretnutie (D. Jones, Webex) ;
\p 18 ; po ; 20.00 ; Online stretnutie (D. Plett, Webex) ;.;;
\p 19 ; ut ;.;;.;;
\p 20 ; st ; 20.00 ; Online stretnutie (D. Jones, Webex) ;.;;
\p 21 ; št ; 20.00 ; Online stretnutie (D. Plett, Webex) ;.;;
\p 22 ; pi ;.;;.;;
\p 23 ; so ;.;;.;;
\p 24 ; ne ;  9.30 ; Bohoslužby (T. Valchář, online); 20.00 ; Online stretnutie (D. Jones, Webex) ;
\p 25 ; po ; 20.00 ; Online stretnutie (D. Plett, Webex) ;.;;
\p 26 ; ut ;.;;.;;
\p 27 ; st ; 20.00 ; Online stretnutie (D. Jones, Webex) ;.;;
\p 28 ; št ; 20.00 ; Online stretnutie (D. Plett, Webex) ;.;;
\p 29 ; pi ;.;;.;;
\p 30 ; so ;.;;.;;
\p 31 ; ne ;  9.30 ; Bohoslužby (D. Plett, online);.;;
}

\tiraz
\bye
