%v programe je vela poloziek, musel som znizit vysku riadku: \vrule height2.4ex% -> \vrule height2.1ex%
\def\velkostpisma{10}
\def\velkostriadku{12.5}
\input makra.tex % nacitanie Ivanom pripravenych nastaveni a prikazov
\hyphenation{star-šov-stvo} % rozdelenie slov na konci riadku, treba tu uviest slova, ktore sam nepozna

\spravodaj{10}{2021}


\clanok {Trpezlivosť v~neľahkých situáciách}
Keď sa nám všetko darí, nie je nič jednoduchšie než vyžarovať trpezlivosť. Skutočná skúška trpezlivosti prichádza vtedy, keď sú obmedzované naše práva; keď si myslíme, že naše okolie sa k~nám správa nespravodlivo; keď sa nezlepšuje situácia v~spoločnosti a napr. pretrvávajú nepopulárne hygienické opatrenia.

Trpezlivosť je veľký problém. Ale podobne ako láska, radosť a pokoj, poukazujú na prácu Ducha Svätého v~našich životoch. Tu sú tri dôvody, prečo je trpezlivosť skutočne veľkým problémom:

\cast{Trpezlivosť je aktom pokory}

Keď sú naše denné plány (alebo naše životné plány) narušené a musíme ísť inou cestou, naša miera trpezlivosti ukazuje, ako sme schopní pokorne prijímať Božie plány. Keď sú naše plány narušené a my sa trpezlivo prispôsobujeme Božiemu plánu a konáme podľa toho, dávame najavo, že pokorne rozpoznávame, ako sme aj krátkozrakí a nemáme dostatok múdrosti.

\cast{Trpezlivosť je skutok služby}

Pri riešení rôznych problémov v~každodennom živote máme na výber, či trpezlivo pomôžeme alebo netrpezlivo odpovieme. Zakaždým, keď sa rozhodneme pre trpezlivosť, rozhodneme sa odložiť svoje vlastné túžby a prijať nepohodlie služby.

\cast{Trpezlivosť je prejavom viery}

Jednou zo základných otázok, na ktorú si musíme každý deň odpovedať, je, či skutočne veríme, že Boh je zvrchovaný, alebo nie. Ak to urobíme, musíme tiež uznať, že On je v~konečnom dôsledku Ten, kto riadi naše kroky. Tieto kroky mnohokrát nie sú kroky, ktoré by sme si vybrali sami; ale sú to tie, ktoré pre nás vybral. Keď sme teda trpezliví, dávame najavo svoje pevné presvedčenie, že Boh vládne nad okolnosťami sveta a jeho cesta je správna.

Trpezlivosť je veľký problém -- pravdepodobne väčší, ako si myslíme. Ale našou úlohou je pracovať na spasení s~bázňou a chvením (Flp 2,12). To zahŕňa precvičenie trpezlivosti. Robme to ako prejav pokory, služby a viery.

Ak budeme nabudúce prežívať zložitú situáciu, ako budeme reagovať? Prirodzenou reakciou je netrpezlivosť, ktorá vedie k~stresu, hnevu a frustrácii. Ako kresťania nie sme otrokmi „prirodzených reakcií“. Máme možnosť reagovať s~trpezlivosťou a úplnou dôverou v~Pána. Pán Boh sa odpláca každému podľa jeho skutkov: {\it „večným životom tým, ktorí sa vytrvalosťou v~dobrom konaní usilujú o~slávu, česť a nepominuteľnosť“ (Rim 2,7).}

\autor{Vladimír Ira}


\clanok {Správy zo staršovstva}
{\it „Ukážeš mi cestu života. U~teba je plnosť radosti, po tvojej pravici večná slasť.“ (Ž 16,11).}

Ako zbor, ako staršovstvo, ako jednotlivci, hľadáme a kráčame po Jeho cestách. Keď chodíme po Jeho cestách, sme blízko Neho. Keď sme blízko Neho, môžeme poznať Jeho radosť.

V septembri začal nový školský rok a staršovstvo znova začalo svoje stretnutia. Vždy začíname s~Božím Slovom a modlitbou, pretože potrebujeme Jeho lásku, milosť a múdrosť. Prehodnotili sme rôzne tábory a stretnutia, ktoré sa udiali v~lete, duchovné a duševné ovocie z~nich, a ako sme videli Pána Boha konať v~individuálnom a v~zborovom živote.

Veľmi podstatná časť našich stretnutí sú pastoračné otázky. Ako jednotlivci alebo spolu staršovstvo chce prioritne pomôcť a povzbudiť tých, ktorí to potrebujú.

Rozlúčili sme sa so Zuzkou Filipovou, ktorá nás náhle opustila. Pripravujeme krst a oživujeme zborové aktivity, ktoré boli zastavené kvôli covidu. Tešíme sa z~praktických vecí, napríklad z~parkovania na školskom dvore Strednej školy elektrotechnickej. Naši milí Bánovci sa presťahovali do Košíc. Jaro založil službu s~video prenosmi, vďaka čomu môžeme sledovať naše bohoslužby online. Sme za zo veľmi vďační.

{\it„Je niekto chorý medzi vami? Nech si zavolá starších zboru a nech sa modlia nad ním, keď ho v~Pánovom mene pomazali olejom“ (Jk 5,14).}

{\it „Musí sa pridŕžať pravého slova podľa učenia, aby bol schopný aj napomínať podľa zdravého učenia a podvracať tých, ktorí odporujú“ (Tít 1,9).}

{\it „Bedlite teda o~seba a o~celé stádo, v~ktorom vás Duch Svätý ustanovil za biskupov, aby ste pásli cirkev Božiu, ktorú si vydobyl svojou krvou“ (Sk 20,28).}

\autor {Daniel Plett}
\vfill\break


\clanok {Nedeľné bohoslužby v~najbližšom období}
Bratislava sa v~súčasnosti nachádza v~oranžovej fáze ostražitosti, čo znamená, že sa musíme vysporiadať s~istými obmedzeniami. Podľa smerníc Úradu verejného zdravotníctva je počet účastníkov na podujatí (vrátane bohoslužieb) obmedzený na 50~osôb, ak sa ho zúčastnia očkovaní i neočkovaní. V~prípade, že sa podujatia zúčastnia iba očkovaní, je počet prítomných neobmedzený.

Vzhľadom na túto skutočnosť budeme mať v~najbližšom období dve zhromaždenia, o~9.00~hod. a 10.30~hod., pričom prvé zhromaždenie bude pre všetkých (očkovaných i neočkovaných proti COVID-19) a druhé zhromaždenie výlučne pre očkovaných s~neobmedzeným počtom zúčastnených.

Online prenos bude zabezpečný z~bohoslužieb o~9.00~hod.: \ulink[https://bit.ly/3cgSMBG]{bit.ly/3cgSMBG}.

Prosíme vás, aby ste nechodili do zhromaždenia, pokiaľ máte akékoľvek príznaky.


\clanok {Sesterské stretnutia namiesto zrušenej víkendovky v~Častej-Papierničke}
Namiesto zrušenej víkendovky v~stredisku Detskej misie Prameň pripravujeme ako alternatívu prednášky v~modlitebni na Palisádach. Raz za mesiac od októbra do decembra budeme mať stretnutie v~sobotu doobeda namiesto stredajšieho stretnutia v~daný týždeň. Veľmi sa tešíme, že Danka Paštrnáková, ktorú sme pozvali byť našou rečníčkou na víkendovke, je ochotná prísť aj na tieto stretnutia v~sobotu. Budú pre všetky sestry v~zbore spolu s~ich dcérami, nevestami, mamami a svokrami, aj ak nie sú z~nášho zboru.

Prvé stretnutie bude v~sobotu 9.~10. a Danka spolu so svojou dcérou budú hovoriť o~vzťahu medzi matkou a dcérou. Stretnutie bude v~čase 9.00~--~11.30~hod.

Ostatným témam sa plánujeme venovať nasledovne:
\begitems
* 6.~11.~2021 Danka s~nevestou Miškou budú spolu hovoriť na tému: Vzťah medzi svokrou a nevestou.
* 4.~12.~2021 Danka so synom Oďom budú spolu hovoriť na tému: Ako mať vzťah s~dieťaťom, keď je „iné“.
\enditems

\clanok {Krst v~zbore}
Krst na vyznanie viery plánujeme v~našom zbore v~nedeľu 24.~októbra o~16.00~hod. v~modlitebni na Cablkovej.
\vfill\break


\clanok {Vianoce spolu}
V spolupráci s~o. z. Detská misia sme sa ako cirkevný zbor zapojili do projektu Vianoce spolu. Cieľom projektu je pred Vianocami priniesť čo najväčšiemu počtu detí Dobrú správu o~narodení Spasiteľa. Je to tiež príležitosť ako rozvíjať misijnú aktivitu zboru za múrmi kostola.

Spoločne sa modlíme za ZŠ Milana Hodžu, s~ktorou už máme isté kontakty. Prosíme za deti, učiteľov i rodičov, aby ich srdcia boli pripravené pre počutie evanjelia. Prosíme za priaznivú epidemiologickú situáciu, aby sme mohli ísť do škôl. Prosíme aj sami za seba, aby nás Pán Boh vystrojil múdrosťou, vytrvalosťou a odhodlaním byť tomuto svetu svetlom a soľou.

Do projektu ideme s~vierou, že Pán Boh má situáciu pod kontrolou. Ak by sa nám nepodarilo ísť priamo na školu, ponúkneme im vianočné video, ktoré pripraví Detská misia. Taktiež škole poskytneme vianočné letáčiky, ktoré môžu rozdať deťom. A~modliť sa môžeme vždy a za každých okolností. Pán Boh určite nenechá naše modlitby bez odozvy!

Ak budete mať k~pripravovanej akcii pripomienky, nápady alebo sa budete chcieť priamo zapojiť do vianočného programu pre deti, obráťte sa na Miriam Kešjarovú (\email{kesjarova@detskamisia.sk}).

Viac info na \ulink[https://www.detskamisia.sk/vianoce-spolu.html]{detskamisia.sk/vianoce-spolu.html}.

Ďakujeme Vám!


\clanok {Pomoc ľuďom v~núdzi}
Milí súrodenci v~Pánovi Ježišovi,

opäť sa hlásim s~ponukou varenia pre ľudí bez domova. Ak by ste mali záujem, tak voľné termíny na varenie sú ešte tieto: 23.~10., 28.~10., 30.~10.

Dôležité informácie:
\begitems
* Potravinová výdajňa je otvorená každý pondelok a stredu od~17.00 do~18.30~h. Ak viete o~niekom, komu by potraviny pomohli, lebo je vo finančnej tiesni, prosím povedzte mu o~výdajni. Ďakujeme!
* Zbierka šatstva stále prebieha. Momentálne potrebujeme najmä pánske tričká, spodnú bielizeň, ponožky, deky, spacáky. Zbierka prebieha každý štvrtok od~16.00 do~19.00~hod. Prosím, kontaktujte vopred Sylviu Vaniherovú na t.~č. 0905~484~675. Ďakujeme!
\enditems

\autor {Beata Bogárová}
\vfill\break


\clanok{Verš na zapamätanie}
Tento mesiac máme nový veršík, ktorý sa chceme spoločne učiť. Veríme, že poznanie Písma prospeje našej duši i našej mysli:

{\it „Veď ja sa nehanbím za evanjelium, lebo ono je Božou mocou na spásu pre každého veriaceho, najprv Žida, potom Gréka. Pretože v~ňom sa zjavuje Božia spravodlivosť z~viery pre vieru, ako je napísané: Spravodlivý z~viery bude žiť.“}

\autor{Rim~1,~16--17}


\clanok{Zbierky za uplynulé obdobie}
Milí bratia a sestry,

v septembri ste prispeli:

\vskip-1ex\begitems
* Misia: 259,00 €
* Investície: 259,00 €
\enditems

Aj naďalej máte možnosť prispieť do „nedeľnej zbierky“, a to prevodom na účet zboru. Do poznámky pre prijímateľa, prosím, uveďte „zbierka“.

Bankové spojenie: SK36 0900 0000 0000 1147 1836, SWIFT: GIBASKBX

Ďakujeme!


\n 2.	10.	Peter	ANTALÍK;
\n 6.	10.	Daniel	BALÁŽ;
\n 12.	10.	Barbora	PRIBULOVÁ;
\n 14.	10.	Martin	SIMON;
\n 20.	10.	Ida	PUČEKOVÁ;
\n 22.	10.	Hana	HALAMIČKOVÁ;
\n 25.	10.	Vladimír	IRA;
\n 26.	10.	Viktor	POTOTSKYI;
\n 26.	10.	Martin	HOVORKA;
\n 27.	10.	Miriam	KRÁĽOVÁ;
\n 28.	10.	František	VRABČEK;
\n 28.	10.	Ľubomír	SYČ;
\narodeniny


\program{
\p  1 ; pi ; 17.30 ; Dorast (Súľovská 2) ;.;;
\p  2 ; so ; 18.00 ; Mládež (Zrínskeho 2) ;.;;
\p  3 ; ne ;  9.00 ; Bohoslužby (D. Jones) ; 9.00 ; Besiedka (Zrínskeho 2) ;
\p    ;    ; 10.30 ; Bohoslužby (D. Jones) ;.;;
\p  4 ; po ;.;;.;;
\p  5 ; ut ;  9.30 ; Klubík (C. Jones, Zrínskeho 2) ; 15.15 ; Stret. pri Biblii (P. Pivka, Zrín. 2) ;
\p  6 ; st ;  6.00 ; Modlitby -- muži (Zoom) ;.;;
\p  7 ; št ; 19.00 ; Biblická hodina (J. Szőllős) ;.;;
\p  8 ; pi ; 17.30 ; Dorast (Súľovská 2) ;.;;
\p  9 ; so ;  9.00 ; Stretnutie sestier; 18.00; Mládež (Zrínskeho 2) ;
\p 10 ; ne ;  9.00 ; Bohoslužby (J. Szőllős) ; 9.00 ; Besiedka (Zrínskeho 2) ;
\p    ;    ; 10.30 ; Bohoslužby (J. Szőllős) ;.;;
\p 11 ; po ;.;;.;;
\p 12 ; ut ;  9.30 ; Klubík (C. Jones, Zrínskeho 2) ; 15.15 ; Stret. pri Biblii (P. Pivka, Zrín. 2) ;
\p 13 ; st ;  6.00 ; Modlitby -- muži (Zoom) ;.;;
\p 14 ; št ; 19.00 ; Biblická hodina (J. Szőllős) ;.;;
\p 15 ; pi ; 17.30 ; Dorast (Súľovská 2) ;.;;
\p 16 ; so ; 18.00 ; Mládež (Zrínskeho 2) ;.;;
\p 17 ; ne ;  9.00 ; Bohoslužby (R. Nagypal) ; 9.00 ; Besiedka (Zrínskeho 2) ;
\p    ;    ; 10.30 ; Bohoslužby (R. Nagypal) ;.;;
\p 18 ; po ;.;;.;;
\p 19 ; ut ;  9.30 ; Klubík (C. Jones, Zrínskeho 2) ; 15.15 ; Stret. pri Biblii (P. Pivka, Zrín. 2) ;
\p 20 ; st ;  6.00 ; Modlitby -- muži (Zoom) ; 17.30 ; Stretnutie sestier (C. Jones) ;
\p 21 ; št ; 19.00 ; Biblická hodina (J. Szőllős) ;.;;
\p 22 ; pi ; 17.30 ; Dorast (Súľovská 2) ;.;;
\p 23 ; so ; 18.00 ; Mládež (Zrínskeho 2) ;.;;
\p 24 ; ne ;  9.00 ; Bohoslužby (D. Jones) ;  9.00 ; Besiedka (Zrínskeho 2) ;
\p    ;    ; 10.30 ; Bohoslužby (D. Jones) ; 16.00 ; Krst (Cablkova 3) ;
\p 25 ; po ;.;;.;;
\p 26 ; ut ;  9.30 ; Klubík (C. Jones, Zrínskeho 2) ; 15.15 ; Stret. pri Biblii (P. Pivka, Zrín. 2) ;
\p 27 ; st ;  6.00 ; Modlitby -- muži (Zoom) ;.;;
\p 28 ; št ; 19.00 ; Biblická hodina (J. Szőllős) ;.;;
\p 29 ; pi ; 17.30 ; Dorast (Súľovská 2) ;.;;
\p 30 ; so ; 18.00 ; Mládež (Zrínskeho 2) ;.;;
\p 31 ; ne ;  9.00 ; Bohoslužby (D. Jones) ; 9.00 ; Besiedka (Zrínskeho 2) ;
\p    ;    ; 10.30 ; Bohoslužby (D. Jones) ;.;;
}


\tiraz
\bye
