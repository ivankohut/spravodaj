%\typosize[9.5/12.2]% - pouzita velkost pisma/riadku
\input makra.tex % nacitanie Ivanom pripravenych nastaveni a prikazov
\hyphenation{star-šov-stvo} % rozdelenie slov na konci riadku, treba tu uviest slova, ktore sam nepozna

\vyrocnespravy{2018}

\clanok{Zbor}
\cast{Úvod zo srdca}

Prvýkrát mám za úlohu napísať správu za zbor. Až dodnes sa mi to zdá až neuveriteľné, že z~Božej milosti ste nás s~Clarou  prijali do spoločenstva a~zverili nám vedenie tohto zboru. Je nám cťou a~tú zodpovednosť berieme vážne a~s~radosťou.  Ďakujeme vám za dôveru.

Priznám sa, že veľmi nemám možnosť  porovnávať tento rok s~predchádzajúcimi rokmi a~hodnotiť ho podľa nich. Pre mňa je stále všetko krásne a~smeruje len k~dobrému.  Som nesmierne vďačný za atmosféru očakávania, ktorú zažívam od príchodu do zboru. Verím, že je to vždy tak, keď sa za niečo modlíme a~hľadáme Božiu vôľu. Odpoveď na zborové a~osobné modlitby je cítiť a~som presvedčený, že je tam aj nadšenie, aj očakávanie toho, čo nám Pán pripravuje.

Prvé obrovské prekvapenie bol zborový byt. Keď sme doň prišli, sme sa skutočne rozplakali. Taký obrovský prejav lásky a~prijatia! Nemali sme a~ani dodnes nemáme slov. Nebolo to len v~tom, že všetko bolo perfektne zrekonštruované, ale skôr, že ste tam strávili hodiny usilovnej práce a~robili ste to s~láskou. Ďakujeme aj za ochotu premiestniť zborovú kanceláriu, a~tým uvoľniť priestory pre obývačku. Vždy keď je plná mládežníkov alebo sestier, alebo keď tam sedíme s~návštevami, sme Bohu a~vám veľmi vďační. Bol to pre nás krásny prejav prijatia a~sme vám vďační z~celého srdca.

Na mojom prvom stretnutí so staršovstvom ako aj s~členmi zboru na stretnutí 12.~2.~2018 mi bolo jasne komunikované, že sa odo mňa očakáva vedenie. Ako vizionár, ktorý stále skúma, čo nie je a~sníva o~tom, čo by mohlo byť, som si dával pozor od svojho príchodu, aby som predčasne nezačal príliš veľa zmien. Vedel som, že treba zbor spoznávať, nielen tradície a~zvyky, ale hlavne ľudí. S~Clarou sme boli veľmi potešení pozvaniami na návštevy k~vám. Počas tých hodín, ktoré sme s~vami strávili, sme zažili veľa radosti a~boli sme aj veľmi požehnaní. Rýchlejšie sme spoznávali zbor a~dostali sme lepší prehľad o~zborových veciach. Ďakujem vám srdečne za pohostinnosť a~lásku, čo ste nám prejavili.  Aj za to, že sa to ešte stále deje. Tešíme sa na každého z~vás. Veľmi rýchlo sme sa do zboru zamilovali, a~preto sme veľmi radi s~každým z~vás.

Rodinný tábor bol pre nás požehnaním a~pomohol nám ešte viac sa zoznámiť s~členmi zboru. Teším sa, že roky dodržiavame túto tradíciu. Pobyty takéhoto typu sú pre zbor vzácne a~dôležité. Bol by som rád, keby sme v~budúcnosti zorganizovali viac takýchto pobytov.

\cast{Za číslami sú ľudia}

Našou motiváciou nikdy nemôžu byť čísla, aby sme sa cítili úspešne alebo dôležito. Je jedno, aké sú tie čísla veľké, za každým je človek. Ale je zaujímavé, že v~Biblii sú často uvedené čísla. Dôvodom je podľa mňa to, že čísla sú jednotlivci a~každý človek je pre nás dôležitý.

Hneď na začiatku mojej služby v~zbore som sa zúčastnil na slávnosti krstu, na ktorom sa dali pokrstiť Kristína (Lászlóová) Matušeková a~Lukáš Máťuš. Potom v~septembri sa dali pokrstiť ďalší štyria: Martin Hovorka, Radovan Nemec, Dara Plett a~Angela Vráblová. Verím, že Pán Boh by chcel, aby sa tieto čísla krstencov stále zdvojnásobovali. Preto sa za to modlím. Zapojte sa vy do týchto modlitieb a~verme Bohu aj v~roku~2019.

Členstvo v~zbore je dôležité hlavne preto, že prijatím členstva robíme vzájomný záväzok -- medzi členom a~zborom alebo zborom a~členom. Autorita je daná Bohom a~s~tou autoritou súvisí aj zastrešenie a~starostlivosť o~členov. Avšak mimo spoločenstva je človek bez ochrany v~ohrození, a to v~duchovnej ako aj duševnej oblasti. Preto je pre nás členstvo dôležité. Tešíme sa, že v~roku 2018 sme prijali za členov zboru Zoru Fedákovú, Mateja Matušeka, Kristínu Matušekovú a~Martina Pelíška. Žiadosť o~zrušenie členstva sme dostali 7.~mája 2018 od Kamila Šalinga.

So smútkom v~srdci sme sa v~roku 2018 s~niektorými rozlúčili. Do nebeského domova odišli 24.~7. Marta Kešjarová, 8.~10. Juraj Hovorka, 18.~10. Hector Blanco, 4.~11. Helena Žiaranová, 18.~9. Vladislav Rízek a 5.~12. Zuzana Štefeková.

\cast{Zbor ako rodina rastieme}

Od začiatku hovorím, že sme rodinou. Verím, že to pre nás nie je len frázou, ale že tým žijeme v~každej oblasti zborového života. Kvôli tomu sme spolu oslavovali aj vytvorenie, aj rozrastanie rodín. V~roku 2018 sa zosobášili Jaro a~Radka Bánovci, Dávid a~Barborka Pribulovci a~Matej a~Kristína Matušekovci.  Do našej zborovej rodiny nám Pán daroval tieto deti: Melánia Kolářiková, Nela Kráľová, Izabela Pelíšková, Matúš Vrábel a~Danica Maksimović.

\cast{Modlitby}

Nedávno som čítal, že keď sa miestny zbor v~Jeruzaleme modlil za Petra, bol z~väzenia vyslobodený napriek tomu, že mali málo viery. Stačí sa len modliť a~Boh odpovie, aj keď je naša viera slabá. Preto sa modlíme:  ženy v~pondelok, seniori v~utorok, muži v~stredu. Tieto stretnutia sú oficiálne organizované, ale verím, že aj inokedy, po dvoch alebo v~kaviarňach sa spolu modlíme. Veríme, že modlitba je mocná a~môže zmeniť celé mesto. Tak sa stalo r.~1857 v~New Yorku, kedy sa dvaja členovia Reformovanej Cirkvi stretávali pravidelne v~kostole o~12.00 a~modlili sa. Po 6~mesiacoch sa viac ako 50~000 obyvateľov New York City pravidelne o~12.00 modlilo za prebudenie v~meste, a~stalo sa to. Čo keby sa nám to stalo aj v~Bratislave? Začneme tento rok s~týmito verejnými modlitbami o~12.00 znovu a~verím, že uvidíme, ako mocne na to Pán odpovie. V~stredu sa pravidelne na Zrínskeho stretávame na modlitbách za vládu a~poslancov. Nikdy sa nemôžeme modliť príliš veľa. Preto sa tento rok modlime ešte viac.

\cast{Vyučovanie}

Som vďačný Bohu za tím kazateľov/učiteľov, ktorí u~nás slúžia. Spolu s~Ľubošom Dzuriakom, Petrom Kolárovským a~Tomášom Valchářom sa pravidelne stretávame na plánovanie, prípravu, a~zhodnotenie sérií kázní. V~roku~2018 sme sa v~sériách kázní zamerali na rodinné vzťahy, listy zborom zo Zjavenia Jána a~Božie kráľovstvo. Verím, že takéto zameranie a~spracovanie určitých tém nám pomôže spoznávať Písmo a~byť Ním menení.

Biblické hodiny pre seniorov prebiehajú v~utorok s~Palim Pivkom. Ján Szőllős vedie štúdium Nehemiáša vo štvrtok. Som veľmi vďačný Bohu za týchto bratov a~ich službu. Som povzbudený tým, koľko ľudí sa pravidelne počas týždňa stretáva k~štúdiu Písma.

\cast{Hudba, spev a chvály}

Verím, že tradičné piesne a~súčasné chvály sú rovnako dôležité. Takisto verím, že spevokol môže byť krásnym nástrojom na to, aby sme Bohu slúžili a~povzbudili iných ľudí. Preto sme sa so Slávom Kráľom rozhodli vytvoriť 5~zborových hudobných skupín, jednu na každý týždeň v~mesiaci. Prvý týždeň slúži spevokol. Som vďačný, že je rastúci počet spevákov a~znovu pravidelne chválime Pána. Ostatné týždne v~mesiaci slúžili chválospevové skupiny pod vedením Diany Dzuriakovej, Petra Kolárovského, Lívie Kolářikovej a~Martina Pribulu. Ďakujem aj Matúšovi Rajskému za koordináciu tejto služby.  Tým, že slúžili iba raz v~mesiaci, mali viac času na to, aby sa dobre pripravili a~viedli. Stretávame sa a~hodnotíme aj túto službu. Verím, že smeruje stále k~lepšiemu a~z~toho sa teším.

\cast{Spoločenstvo}

Som veľmi vďačný za Sama Koriťáka a~Líviu Kolářikovú a~za ich pravidelnú nedeľnú službu s~raňajkami. Táto káva a~občerstvenie pred aj po zhromaždení nám vytvárajú priestor pre dôležité spoločenstvo a~rozhovory. Tešíme sa, že niekedy až hodinu sa spolu rozprávame a~zoznamujeme sa. Je to aj biblické a~kľúčová časť zborového života. Viem si predstaviť, že sme všetci z~toho povzbudení.

\cast{Záver k~srdcu}

Sme s~Clarou naplnení radosťou, že nás Pán povolal do zboru na Palisádach, a~za to, že ste nás s~láskou prijali. Cítili sme sa hneď ako doma, a~stále sme viac a~viac Bohu vďační za Jeho milosť voči nám. Verím, že Pán s~nami koná unikátnym spôsobom a~má pre nás prichystané dobré veci do budúcnosti. Tešíme sa na nové skupinky, misijné cesty, ktoré plánujeme, návštevy do Rakúska a~Maďarska a~rôzne tábory a~výlety. Verím, že každým z~tých plánov budeme viac pripravení na naplnenie Božej vôle pre náš zbor. Nebojte sa! Na všetko je dosť milosti a~Boh sám nás vedie k~tomu, čo je Jemu na slávu a~česť.  Nebuď prekvapený, keď Ťa Boh tento rok niečím prekvapí. Neviem ako, ale buď pripravený! Modli sa za to. Nechaj  Pána, aby cez Tvoj život niečo robil, čo sa nedá vysvetliť inak ako modlitbou a~zázrakom. Predpokladám, že sa Ti ozve a~ukáže Ti svoju moc.

S~vďačnosťou a~na modlitbe,

\autor{Danny Jones, kazateľ zboru}


\clanok{Staršovstvo}
Staršovstvo dostalo na rok 2018 verš z~1Pt 2,~9: {\it „Vy však ste vyvolený rod, kráľovské kňazstvo, svätý národ, ľud určený na vlastníctvo, aby ste oznámili veľké skutky toho, čo vás povolal z~temnoty do svojho predivného svetla.“}  Pán Ježiš nám skrze apoštola Petra pripomenul, kto sme. Možno niekedy zabúdame na svoju identitu, ale náš Pán nám to vždy vo vhodnej chvíli pripomenie. A~On to urobil aj týmto veršom. Pripomenul nám, že sme vyvolení. To znamená, že On sám sa rozhodol pre nás. Rozhodol sa, že chce, aby sme patrili Jemu. Náš ľudský pohľad by mohol zablúdiť a~myslieť si, že nás potrebuje ako otrokov, pracovnú silu, alebo panáčikov bez ducha. On nám ale pripomína, že si nás vyvolil za kňazov Kráľa. Toto spojenie s~Kristom ako údy Jeho tela nás zaväzuje k~dielu zmierenia medzi človekom a~Bohom. Volá nás k~tomu, aby sme oznamovali skutky Boha, ktorý nás vyvádza z~temna hriechu, odporu voči Nemu, do svetla poznania, spasenia a~života zameraného na Nebeské kráľovstvo.

V~roku 2018 pracovalo staršovstvo v~zložení Peter Antalík, Radovan Hovorka, Vladimír Ira, Pavel Kohút, Peter Kolárovský, Miroslav Kolářik, Peter Pribula. Od júna sa aktívnym členom staršovstva stal aj brat kazateľ Danny C.~Jones.

Rok 2018 bol rovnako bohatý na udalosti v~našom zbore, tak ako to bolo aj v~minulosti. Niektoré zborové aktivity sa pravidelne opakujú a~napriek tomu sú vždy iné a~jedinečné. Iné bývajú neopakovateľné a~preto aj jedinečné svojou podstatou. Všetky tieto aktivity sú nasmerované na jedno jediné -- osláviť spasiteľa Pána Ježiša a~oznamovať Jeho skutky.

Uvedomujeme si, že zameranie našej práce nemôže byť úzko špecializované na niektorú vekovú či inak definovanú skupinu. Snažíme sa preto citlivo vnímať potreby všetkých členov zboru, ale aj priateľov, ktorí navštevujú naše stretnutia.

Počas minulého roku sme sa venovali potrebám našich najmenších, týkajúcich sa priestorov na stretávanie, vplyvu prítomnosti rodičov na sústredenosť pri ich stretnutiach a~aj potrebe prítomnosti detí a~dorastencov na spoločných bohoslužbách. Túto potrebu vnímame ako obojstranne potrebnú, nakoľko buduje nás starších aj samotné deti. Záujem o~deti sa prejavil aj tým, že sa našli členovia zborovej rodiny ochotní pripravovať pre najmenších pravidelnú publikáciu {\it Relaxík}.

Spolu s~bratom Pavlom Pivkom pracujeme na napĺňaní potrieb najstarších členov zboru. Majú svoje pravidelné spoločné stretnutia, ale aj príležitostné stretnutia, či výjazdy mimo Bratislavu.

Do portfólia nášho záujmu v~minulom roku pribudla skupina tých, ktorí nám rozumejú trochu menej. Podľa ich vyjadrenia, miera porozumenia je iba asi 40 percent. Preto sme vďační nášmu Pánovi, že už dopredu pripravoval niektoré veci pre to, aby sme mohli poslúžiť bratom a~sestrám z~Ukrajiny. Hľadajú v~našom regióne prácu a~hmotné zabezpečenie svojich rodín. Sme radi, že nezabúdajú na svoj duchovný život, a~pre posilnenie tejto snahy radi využívame služby brata Viktora Potockého. V~rámci svojho teologického štúdia na VŠ v~Banskej Bystrici navštevuje náš zbor a~slúži ukrajinsky a~rusky hovoriacim bratom a~sestrám. V~súčasnej dobe hľadáme spôsob systematickej podpory jeho práce.

Mládež patrí asi medzi najľahšie ovplyvniteľnú vekovú skupinu. Práci s~mládežou a~pochopeniu jej potrieb venujeme čas na našich stretnutiach. Chceme, aby našli svoj domov medzi nami a~boli zapojení do života našej zborovej rodiny.

Sestry našli novú inšpiráciu príchodom Clary Jones do zboru. Po rokoch, kedy náš zbor nemal kazateľov s~manželkou, alebo neboli aktívne zapojené do zborovej práce, sme vďační za Claru a~jej zápal pre sestry a~ženy hľadajúce cestu k~spaseniu.
Ja osobne si nepamätám, že by v~niektorom zbore, do ktorého som patril, existovala aktívna práca s~mužmi. Zameranie nášho kazateľa Dannyho nevynechalo zo svojho hľadáčika ani nás. Uvedomujeme si, že bez zdravých, duchovne zdravých, mužov, nie je možné mať zdravé rodiny ani zdravý zbor.

Modlíme sa za to, aby sa všetci v~našom zbore cítili prijatí, cítili sa ako doma, vedeli, že ak nie sú medzi nami, tak nám chýbajú, a~zároveň chceme plniť výzvu milovať sa navzájom.

Udalosť minulého roku, ktorou sme asi najviac žili, bol príchod kazateľa s~rodinou. Privítali sme ich v~nedeľu 27.~mája. Príprava na ich príchod prebiehala v~niekoľkých rovinách. Rekonštrukcia bytu a~kancelárie na Zrínskeho zabrala čas, peniaze, prácu a~aj iniciatívu mnohých členov a~priateľov zboru. Povolenie na pobyt a~s~tým súvisiace vybavovanie si vyžiadalo zapojenie ľudí, ktorí sa tejto téme venujú profesionálne. Pomoc pri vykladaní osobných vecí z~prepravného kontajneru zas vyžiadala množstvo ochotných a~silných rúk. Zosúladenie všetkých týchto aktivít si vyžiadalo množstvo našich modlitieb. Všetkým, ktorí sa akokoľvek zapojili do tejto práce, chceme znova vyjadriť vďaku.  24.~jún bol deň s~veľkým D. Bolo to zavŕšenie niekoľkomesačnej snahy o~poznanie Božej vôle a~hľadanie kazateľa pre náš zbor. V~tento deň bol Danny Jones inštalovaný za kazateľa zboru.

Iné témy, ktorým sme sa venovali, spomeniem iba v~bodoch. Neznamená to, že by boli menej dôležité. Chcem, aby sme na ne nezabudli a~zároveň, aby sme boli vďační Bohu za to, že nám v~tom pomáhal, dával múdrosť, ale aj zabránil urobiť to, čo nebolo v~súlade s~Jeho vôľou:
\begitems
* voľby predsedníctva a členov Rady BJB a členov komisií
* zabezpečenie služieb v~zbore a na Chvojnici
* smerovanie práce na Chvojnici
* národný týždeň manželstva
* letné tábory a~práca s~deťmi, dorastencami a~mládežou v~zbore
* pastoračné otázky a~rozhovory s~ľuďmi, ktorí potrebujú pastoráciu
* záujemcovia o~členstvo v~zbore
* krst na vyznanie svojej viery
* spoločné obedy a~občerstvenie pred a~po bohoslužbách
* služba spevokolu na Vianoce a~Veľkú noc
* služba JASu v~zbore
* idea Domov pre seniorov v~Bernolákove
\enditems

Sme vďační nášmu nebeskému Otcovi za Jeho vedenie, múdrosť a~požehnanie pri všetkom, čomu sme sa počas roku 2018 venovali. Zároveň prosíme o~Jeho vedenie, múdrosť, požehnanie, ale aj o~pokoru, ochotu a~silu k~práci.

\autor{Peter Pribula}


\clanok{Diakonia}
Verš na rok 2018: {\it „Cti Hospodina darmi svojho imania a~prvotinami z~každého svojho výnosu. Tak sa ti naplnia stodoly hojnosťou a~muštom pretečú tvoje kade.“}

\autor{Príslovia 3,~9 -- 10}

\cast{Pracovné stretnutia}

Tím diakonov sa v~roku 2018 pravidelne stretával na pracovných stretnutiach každé dva mesiace na Zrínskeho ulici v~zborových priestoroch. (Zápisnice z~pracovných stretnutí boli pravidelne zaslané všetkým členom e-mailom, prípadne osobne odovzdané). Pozvánky na pracovné stretnutia pre členov tímu diakonov boli zasielané e-mailom spravidla tri dni vopred. Taktiež boli vyhlasované v~oznamoch v~rámci nedeľného zhromaždenia.

\def\aktivita#1{{\it #1\par}\firstnoindent}
\cast{I. Vnútrozborové aktivity}

\begitems \style n
* \aktivita{Návštevná služba}
Pravidelne pokračovala návšteva našich imobilných členov v~domácnostiach, ktorú vykonávajú jednotlivé sestry a~bratia, ktorí majú s~navštevovanými bratmi a~sestrami vytvorený niekoľkoročný blízky vzťah. Chcel by som aj menovite spomenúť aspoň niektorých členov tímu diakonov, ktorí pravidelne navštevovali našich imobilných seniorov. Sú to sestry Lenka Gubová, Vladka Laurenčíková, Juditka Kolářiková, Gitka Kráľová, brat Jurko Kvačka a~manželia Valentovci. Taktiež sa návštev zúčastňoval aj brat kazateľ D.~Jones a~brat diakon P.~Pivka.

Okrem našich seniorov boli navštevovaní aj naši chorí bratia a~sestry či už v~domácnostiach alebo v~nemocniciach. Sestry navštevovali aj mladé mamičky s~bábätkami z~nášho zboru.

* \aktivita{Služba núdznym}
Bratovi Romanovi Žiaranovi sa venoval verne brat kazateľ D.~Uhrin spolu s~bratmi zo zboru, ktorí majú s~ním blízky vzťah. Brat Roman je momentálne na Kráľovej Lehote, kde je zapojený do práce, ktorá ho napĺňa. Je tam o~neho dobre postarané, sme za to vďační Pánu Bohu.

Taktiež ako každý rok sme na decembrovom stretnutí rozdelili sociálny fond na „vianočnú výpomoc“ sociálne slabším členom nášho zboru.

* \aktivita{Vianočný obed pre seniorov}
 2.~12.~2018 sme mali už tradičný {\it Vianočný obed pre seniorov}. Zákusky zabezpečili sestry. V~rámci obeda boli odovzdané aj vianočné darčeky našim seniorom.

* \aktivita{Svätodušné sviatky na Chvojnici}
Aj v~roku 2018 sme na Letnice boli poslúžiť bratom a~sestrám na Chvojnici. Bol to pre mnohých z~nás celodenný zborový výlet, na ktorý nás odviezol už tradične „Barnibus“.

Chcem sa poďakovať aj bratovi Paľkovi Škulecovi za vernú službu, ktorú koná na Chvojnici.

* \aktivita{Vysluhovanie Večere Pánovej}
Večera Pánova sa vysluhovala pravidelne každú prvú nedeľu v~mesiaci (podľa rozpisu). Okrem toho sa Večera Pánova vysluhovala aj v~domácnostiach. Brat V.~Krajčí nás informoval, že potrebujeme nových mladých bratov -- služobníkov k~vysluhovaniu Večere Pánovej.

(Poznámka: V~súčasnosti máme prihlásených nových bratov do služby v~diakonskom tíme, s~ktorými budeme mať na našom najbližšom stretnutí rozhovor).
\enditems

\cast{II. Aktivity zboru smerom von}

\begitems \style n
* \aktivita{Služba v~domovoch sociálnej starostlivosti}
Okrem služby v~našom zbore sa venujeme aj službe mimo zboru v~domovoch dôchodcov pod vedením brata P.~Pivku za vernej pomoci sestier L.~Gubovej a V.~Laurenčíkovej. Pravidelne navštevujeme „Domovy sociálnej starostlivosti“ v~Starom Meste a~v~Dúbravke.

* \aktivita{Služba ľuďom bez domova}
Aj tento rok sme podporovali službu varenia pre bezdomovcov v~rámci spoločenstva {\it Kresťania v~meste}. Túto službu koordinujú sestry t.~č. pod vedením sestry B.~Bogárovej.
\enditems

\autor{Pavel Pivka}


\clanok{Hospodársky výbor}
V~mesiaci január sme po podrobnom rozkalkulovaní položiek materiálu a~práce z~ponuky od firmy br.~Kovaľa zo zboru Viera (vodoinštalácia a~kúrenie) pristúpili k~plánovaniu rekonštrukcie bytu kazateľa zboru a~kancelárie zboru na Zrínskeho. Stavebné a~obkladačské práce vykonala firma rod.~Dzuriakovej. Po uvoľnení priestorov bytu 5.~2.~2018 sme pristúpili s~bázňou a~očakávaním na požehnanie a~pomoc od nášho Pána. Keďže o~postupe prác spojených s~fotodokumentáciou sme priebežne informovali zborové spoločenstvo (rodinu), spomeniem len pracovné úkony, ktoré bolo potrebné vykonať pri rekonštrukcii: natieračské, elektoinštalačné, murárske, maliarske, upratovacie, sťahovacie práce, odvoz stavebného odpadu~atď. Sponzorská skupinka uhradila náklady za kompletne novú kúpelňu, ale aj mnohí jednotlivci, ktorí zaplatili napr. kompletné kovanie na všetky dvere, domový telefón, výrobu nových vonkajších okenných krídel do kuchyne, zasklievanie okien~atď. Mená bratov a~sestier neuvádzam, aby som na niekoho nezabudol, ale Pán Boh to vidí.  Vieme, že vďaka patrí v~prvom rade nášmu Pánovi za všetko, čo sme mohli z~Jeho milosti vykonať. V~druhom rade ďakujeme Dannymu a~Clare, že nás v~týždenných listoch pre Božiu rodinu povzbudzovali a~boli hnacím motorom pre našu prácu. Nakoniec patrí vďaka všetkým, ktorí sa za úspešné ukončenie diela a~ochranu pri práci modlili, priložili ruky k~dielu, alebo sa podielali finančne.
Chceme poďakovať za prácu všetkých zborových zložiek, ktorých služba nás všetkých obohacuje na ceste za naším Pánom. Takisto chceme osloviť všetkých, ktorí si ešte nenašli službu v~zbore, aby im Pán ukázal, kde môžu pracovať pre povzbudenie všetkých nás, lebo práce je veľa, ale robotníkov málo.

Zborový dom (chalupa) na Chvojnici potrebuje svoju údržbu, aby mohla slúžiť všetkým nám i~ľuďom, ktorí si zamilovali tento krásny kopaničiarsky kraj. Plánujeme tento rok 1.~--~15.~4. vyspraviť a~vymaľovať celý objekt. Na túto prácu sa prisľúbili br.~P.~Kohút a~D.~Mikletič, ktorí sa tešia na pomoc brigádnikov (dúfame, že sa prihlásia). V~ďalších mesiacoch: príprava dreva, kosenie, drobná údržba, upratovanie. Pri kosení pred prázdninami ukázal svoj potenciál aj br.~kazateľ Danny Jones. Povedal: „Milujem kosenie“. V~r.~2018 bol vymenený bojler v~starej kúpeli (po cca.~23 rokoch) a~prívodné inštalácie~+~ventily. Okolie kostolíka pravidelne udržujeme vykášaním trávnatých porastov. Upratovanie vykonáva s.~Mirka Rusňáková, za čo jej patrí naša vďaka. Práce na kostolíku, ktoré sme nestihli v~r.~2018, ak Pán Boh dá, budú realizované v~nasledujúcom období: vonkajšie odvodnenie, oprava fasády, náter fasády, podbitie podhľadov, výmena bočných dverí~atď.

Prácu hospodárskeho výboru sme zahájili 4.~2.~2019 povzbudení veršíkom z~Príslovia 3,~25~--~26 v~zložení: Ľ.~Kohútová, D.~Jones, M.~Kolářik, J.~Szőllős, Ľ.~Kešjar, Ľ.~Syč, M.~Maďar, J.~Štefko a~D.~Mikletič vyprosením si požehnania pre prácu na Božej vinici. Tešíme sa na vašu spoluúčasť pri našej spoločnej rodinnej práci.

\autor{Daniel Mikletič}


\clanok{Biblické a iné vzdelávanie}
Spoločné štúdium Svätého Písma prebiehalo vo viac-menej pravidelnom rytme a~nezmeneným spôsobom. Takmer každý utorok popoludní preberal kazateľ Pavel Pivka so skupinou záujemcov, najmä z~radov seniorov, texty z~Božieho Slova. V~prvom polroku do mája študovali spolu List Židom (Hebrejom) a~v~júni List Júdov. Po letných prázdninách preberali od septembra až do decembra List Jakuba.

Počas štvrtkových večerných biblických hodín pokračoval kazateľ Ján Szőllős do začiatku februára vo výklade Evanjelia podľa Jána a~od februára až do konca júna preberal so záujemcami List Galaťanom. Po letnej prestávke v~mesiacoch júl~--~september sa od polovice októbra začali venovať knihe Ezdráš.

Na každom z~uvedených vzdelávaní, ktoré prebiehajú na~Zrínskeho ulici, sa zúčastňovalo v~priemere len okolo 10 bratov a~sestier, čo je veľmi malý počet. Okrem uvedených celozborových príležitostí bolo zamyslenie nad Písmom a~diskusia aj súčasťou stretávania viacerých skupiniek.

Napriek existencii širokej škály ponúk rôznych typov vzdelávania v~rámci našej cirkvi aj na naddenominačnej úrovni, a~napriek ponúkanej podpory zo strany zboru, sú tieto možnosti málo využívané našimi členmi, ale nemám o~tom podrobný prehľad. Na medzidenominačnej úrovni sa tím pracujúci s~mládežou zúčastnil Konferencie pre pracovníkov s~mládežou (KPM) v~Žiline. Do kurzov Detskej misie sa mohli zapojiť niektorí jednotlivci spomedzi vedúcich besiedky.

Pokračovali aj stretnutia mamičiek malých detí v~našich priestoroch približne raz týždenne na Klubíku, čo je veľmi cenené zo strany účastníčok.

\autor{Ján Szőllős}


\clanok{Sestry}
Rok 2018 priniesol do života zboru veľa zmien. Sestra Andrejka Balážová odišla slúžiť so~svojím manželom do~Čiech, a~preto som bola poverená vedením sestier na jeden rok. Stretávali sme sa pravidelne na našich spoločných stretnutiach, tiež ekumenických modlitbách v~marci, zúčastnili sme sa konferencie sestier v~Košiciach a~každoročne sa zapájame aj do modlitieb  baptistických žien na celom svete v~novembri. Stretli sme sa aj s~našimi sestrami z~ostatných zborov v~Bratislave pred Vianocami pri zhotovovaní rôznych vianočných výzdob. Služba žien v~zbore je úzko prepojená s~prácou diakonov a~ostatných zložiek.

Sme zapojené vo všetkých aktivitách zboru (besiedka, dorast, chvály, vedenie zhromaždení, obedy pre prichádzajúcich a~seniorov), neviem ani všetky aktivity vymenovať. Som veľmi vďačná Pánu Bohu za všetky moje sestry v~tomto zbore, ktoré sú ochotné slúžiť.
Stane sa, že nejaká sestra by aj rada slúžila, ale nepozná svoje obdarovanie a~nevie, kde by mohla slúžiť. A~práve na túto tému sme mali veľmi dobré vyučovanie so sestrou Elise Atkins niekoľko mesiacov na jeseň. Viedla nás k~poznaniu, že každá jedna z~nás je rovnako vzácna v~Božích očiach, hoci navonok sme odlišné ako vločky snehu. Previedla nás testom o~spoznávaní svojich duchovných darov a~následnom prijatí a~stotožnení sa s~ním, aby sme ho využívali v~správnej službe a~vtedy nám bude pôsobiť radosť a~požehnanie. Ďakujeme Elise za jej vzácnu službu!

{\it „Toto je moje prikázanie: aby ste sa milovali navzájom ako som ja miloval vás.“} Ján 15,~12. Tento text sme dostali na Silvestra pre sestry na rok 2018. Pán Boh nám preukázal svoju lásku aj tým, že nám poslal nového kazateľa s~manželkou, ktorý nás stále povzbudzuje, aby sme sa milovali navzájom, a~pripomína nám, aké by to bolo dobré, keby sme boli najmilší zbor v~meste. Je to výzva aj pre nás ženy, aby sme boli spolutvorkyňami lásky, porozumenia a~pokoja v~našich rodinách a~potom aj v~zbore. Ako som už spomenula, sestry sú zapojené v~skoro všetkých aktivitách zboru, mnohé z~nás chodia do práce, alebo majú viacero povinností. Možno by nebolo na škodu nájsť si viac času na oddych, pohodu a~hľadať svoj vnútorný pokoj. Toto sú témy, nad ktorými sa zamýšľala Clara Jones vo svojej knihe a~chcela by sa podeliť s~nami o~svoje skúsenosti v~novom roku. Býva zvykom, že sestra, ktorá odchádza z~vedenia, osloví ďalšiu sestru a~po určitom čase modlitieb a~rozhovoroch odovzdá prácu ďalej. Som rada, že Clara túto službu prijala a~verím, že jej vyučovanie a~životné skúsenosti nám budú na úžitok a~požehnanie.

\autor{Judita Kolářiková}


\clanok{Mládež}
V~roku 2018 sa naša mládež početne nerozrástla, ani pravidelných návštevníkov nie je viac. Bohužiaľ, potýkali sme sa s~tým, že mládežníci, ktorí predtým chodievali pravidelne na mládeže, postupne strácali záujem. V~dnešnej dobe už nie je samozrejmosť, že si mladý človek spraví v~sobotu poobede čas, aby prišiel na mládež, a~to platí aj o~mladých z~kresťanských kruhov z~nášho zboru a~našich rodín. Mladí ľudia majú dnes nespočetné množstvo možností, majú mnoho záujmov a~záľub. Nad tým sme sa aj ako výbor častokrát zamýšľali a~modlili sme sa za to, aby sme vedeli spraviť naše mládeže zaujímavejšie, príťažlivejšie a~lákavejšie. Avšak si uvedomujeme, že nedokážeme konkurovať v~atraktívnosti svetských aktivít, preto o~to viac sa snažíme vytvoriť si čo najpevnejšie a~najlepšie puto a~vytvoriť prostredie, kde je Boh v~centre celého diania.

Od septembra tohto roku sme zaviedli určité zmeny chodu mládeže. Raz do mesiaca sme usporiadali „DEPO” mládež, t.~j. stretnutie bez nejakého organizovaného programu, kedy mládežníci pri spoločenských hrách a~rozhovoroch spolu trávia čas. Bola to hlavne veľmi dobrá príležitosť pozvať si svojich priateľov, ktorí nemali žiadnu skúsenosť s~naším spoločenstvom. Raz do mesiaca nás ako mládež pozvala rodina Jonesovcov k~nim domov, kde nám slúžila slovom. Danny vytvoril sériu tém s~názvom {\it Život so zámerom}, kde na príklade niektorých biblických postáv nám ukazuje, čo to znamená žiť s~Bohom. Zároveň si ako mládežnícky výbor uvedomujeme, že v~našom zbore sú ľudia, ktorí svojou životnou skúsenosťou sú zaujímaví pre mládežníkov. Preto od septembra dávame v~prvom rade priestor ľuďom, ktorí prezentujú svoj duchovný zážitok. Taktiež sme vytvorili priestor pre samotných mládežníkov (nazvali sme to {\it Päť minút slávy}), ktorých vedie Pán k~tomu, aby nám povedali, čo im leží na srdci. Sme skutočne vďační za to, že sme znovu obnovili chválospevovú skupinku, ktorá nás na samom začiatku vedie ku chválam.

Stále sme vďační za priestory nášho zboru na Chvojnici, kde sme spolu prežili víkend, a~napriek veľkej zime a~nie až tak vysokej účasti sme spolu strávili požehnaný čas. Počas roka 2018 sme viackrát mali možnosť byť s~mládežníkmi z~iných krajov. Ako každý rok bola aj v~r.~2018 česko-slovenská mládežnícka konferencia, ktorej sme sa zúčastnili aj my. Okrem konferencie sme boli aj na letnom tábore na Muránskej Zdychave spolu s~mládežou z~okolia Revúcej, ale boli tam aj účastníci z~viacerých kútov Slovenska. Tábor na Muránskej Zdychave bol špeciálnym práve z~toho dôvodu, že bol takým pokračovaním misijných aktivít Richarda Nagypala a~novozakladajúceho sa zboru v~Revúcej. Na konci leta sme sa zúčastnili Celoslovenského kempu mládeže, ktorý je v~menšom merítku považovaný za pokračovanie mládežníckej konferencie. Neskôr začiatkom školského roka sme dostali pozvanie na oblastné stretnutie do Lučenca, ktoré sa konalo v~októbri 2018, kde sa stretávajú mladí z~Lučenca a~okolia, ale aj oblasti južného Slovenska. Začiatkom decembra sme sa ako mládežnícky výbor spolu s~Dannym stretli v~Banskej Štiavnici spolu s~ďalšími ľuďmi, ktorí vedú mládeže z~rôznych kútov Slovenska. Spolu sme tam diskutovali o~smerovaní súčasných mladých ľudí a~hľadaní rôznych spôsobov k~približovaniu mladých k~Bohu. Na všetkých týchto akciách sme mali možnosť spoznať nových ľudí, nadviazať kontakt s~ľuďmi z~iných zborov a~miest. Sme vďační za prijatie, ktorého sa nám dostalo vo všetkých zboroch a~spoločenstvách, ktoré sme navštívili. Verím a~teším sa na to, že aj v~budúcnosti táto spolupráca bude pokračovať, a~chcem, aby aj naša mládež bola schopná pripraviť program a~prijať ľudí či už na tábore alebo pripraviť oblastné stretnutie pre západnú oblasť.

V~radoch nášho mládežníckeho výboru sme pocítili veľké zmeny. Od septembra sa do výkonnej funkcie mládežníckeho výboru pridal Radovan Paulen spolu s~Lukášom Máťušom, ktorý od nového roku odišiel na pol roka do Nemecka. Čo sa týka starej zostavy, Dávid Pribula a~Anička Plett naďalej slúžia v~mládeži a~zabezpečujú jej chod. Peter Antalík, ktorý je hlavou mládežníckeho výboru, nás so svojimi dlhoročnými skúsenosťami usmerňuje a~vedie.  Našou túžbou je aj to, aby sa aj rady nášho výboru rozrástli, aby súčasní mládežníci cítili zodpovednosť a~potrebu pracovať s~mládežou.  Prosím, myslite na nás, aby aj naše životy mohli poukazovať na Krista, viesť mládež bližšie k~Bohu a~aby sa tak naplnilo aj Božie slovo, ale aj naše motto: {\it Mládež je živé spoločenstvo v~Ježišovi Kristovi, ktorého cieľom je uctievanie Boha, budovanie sa navzájom a~početný rast.}

\autor{Radovan Paulen}


\clanok{Dorast}
So skupinou dorastencov vo veku od 11 do 15 rokov sme sa stretávali každú nedeľu počas bohoslužieb. Uvažovali sme aj o~návrate k~piatkovému termínu, ale obávame sa nižšej účasti dorastencov kvôli krúžkom, ktoré deti navštevujú, ale aj logistickým problémom v~piatkové popoludnia.

Na našich stretnutiach sme preberali témy z~materiálu Detskej misie {\it Božia cesta} a od začiatku školského roka {\it Nasleduj Ježiša}. Tu sme sa cez texty z~Biblie učili o~Bohu na príbehoch ľudí, čo žili pred nami a~Pán Boh rôznymi spôsobmi vstupoval do ich životov. Chceli by sme viesť dorastencov k~pravidelnému čítaniu Biblie aj mimo stretnutí dorastu, preto si každý mesiac rozdáme rozpis čítania z~Biblie a~snažíme sa ho aj dodržiavať. Okrem toho sa zvykneme porozprávať o~tom, čo prežívame a~občas sa zahráme nejakú hru.

Počas letných prázdnin sme spolu strávili pekný týždeň na tábore na Chvojnici, ktorá ponúka okrem prekrásnej prírody aj obrovské možnosti na rôzne hry a~športy, posedenia pri ohníku, výlety do blízkeho aj vzdialenejšieho okolia (hrad Branč), či retro kúpalisko v~Senici (keďže Kunovská priehrada bola tento rok ešte vypustená). V~letných mesiacoch sme organizovali aj s~pomocou ďalších dobrovoľníkov program pre dorastencov na rodinnom tábore v~Častej.

Tento rok sme stretnutia dorastu viedli v~zložení L.~Kamocsai, M.~Simon, manželia Hovorkovci a~manželia Halamičkovci. Čas strávený s~dorastencami nás všetkých veľmi obohacuje. Vyslovené otázky nás často nútia zamýšľať sa hlbšie nad tým, čo hovoríme, a~nad vecami, ktoré považujeme za samozrejmé. Ale občas to vôbec nie je tak. Takto spoločne hľadáme cestu k~Bohu a~budujeme vzájomné vzťahy.

\autor{Rišo Halamiček}


\clanok{Besiedka}
V~roku 2018 sme mali 35 detí vo veku od 3 do 15 rokov. Detí od 0 do 2 rokov máme 15. Okrem týchto detí na besiedku prichádzajú návštevníci, ktorí zvyšujú naše počty. Je potešujúce, že v~našom zbore sa rozrastá mladšia generácia a~verím, že tento trend bude pokračovať.

V~priestoroch besiedky sa už tretí rok delíme na tri skupiny: malá besiedka (3--7 rokov), veľká besiedka (8--10 rokov) a~mladší dorast (11--14 rokov). Nakoľko detí pribúda, uvažujeme o~rozdelení veľkej besiedky. To by si však vyžadovalo ďalších pracovníkov a~priestory. Prosím, modlite sa za to, aby sme našli optimálne riešenia.

V~malej besiedke vyučujú Mirka Hovorková, Janka Máťušová, Mirka Kešjarová, Rada Bánová a~Kika Kešjarová. S~deťmi nám príležitostne pomáha Katka Kerekréty. Vo veľkej besiedke slúžia Vierka Kolárovská, Baka Pribulová, Ľubka Kráľová a~Janko Kováčik. Neoddeliteľnou súčasťou každej besiedky je aj spev, ktorý už roky verne vedie Diana Dzuriaková.

V~besiedke sa s~deťmi hráme, spievame, vyrábame si rôzne drobnosti. Hlavnou časťou programu však vždy bolo a~je vyučovanie. Deti sa potrebujú oboznamovať s~Božím slovom primeraným spôsobom. Na tento účel nám slúžia hlavne materiály z~Detskej misie a~zo stránky \ulink[https://www.tim22.cz/]{www.tim22.cz}. Učitelia sú kreatívni, a~preto si mnohé aktivity vytvárajú sami a~prispôsobujú ich aktuálnym potrebám detí. Mnohé zaujímavé podnety deti dostávajú aj od moderátorov zhromaždení.

Na záver by som chcela ešte podotknúť, že deti sú naším zrkadlom. Sú zrkadlom rodiny, v~ktorej vyrastajú. Prosím, uvedomujme si, že deti nás neustále pozorujú a~napodobňujú. Ak niečo iné hovoríme a~niečo iné žijeme, veľmi ľahko si to všimnú. Myslime na to a~modlime sa, aby nám Pán Boh pomáhal byť každý deň dobrým príkladom pre naše deti. Nie je to ľahké! Je to celoživotná drina… Ale stojí to za to.

\autor{Miriam Kešjarová}


\clanok{Matuzalem}
Skupina Matuzalem v~roku 2018 ukončila po 35-ročnej aktívnej a~požehnanej službe svoju činnosť.  Dôvody ukončenia boli nedostatok pokory a~lásky. Nedostatok práve toho, o~čom sme najviac spievali.
Nech je nám Pán Boh milostivý.

\autor{Slávo Kráľ}


\clanok{Spevokol}
V~roku 2018 sme do nášho spevokolu zas mohli prijať niekoľko mladých hlasov, čo povzbudzuje nás všetkých do novej aktivity. Sme vďační Pánu Bohu za možnosť spevom oslavovať nášho Stvoriteľa.

Podobne ako po minulé roky sme začali službou na Novoročnom koncerte v~evanjelickom kostole v~Petržalke. Pravidelne sa tam stretávame s~viacerými spevokolmi i~skupinami z~Bratislavy a~blízkeho okolia. Tentoraz sme poslúžili nielen za sprievodu klavíra, ale aj nášho komorného orchestra.

V~tomto roku sme sa snažili viac ako v~minulosti slúžiť na domácej pôde počas nedeľných bohoslužieb. Ide hlavne o~spievanie počas prvej nedele v~mesiaci pri slávnosti Večere Pánovej. Počas uplynulého roka sme aj trikrát slúžili v~našom zbore na pohrebných spomienkových zhromaždeniach. V~októbri bratom Jurajovi Hovorkovi a~Hectorovi Blancovi a~v~novembri sestre Elenke Žiaranovej.

Hlavný dôraz našej služby cielene prikladáme na naše koncerty pre širokú verejnosť. Vianoce a~Veľká noc sú vynikajúcou príležitosťou osloviť aj takých ľudí, ktorí by k~nám inokedy neprišli. Účasť cudzích ľudí je taká vysoká, že našich členov medzi nimi treba hľadať.  Pre veľký záujem verejnosti už pravidelne pripravujeme dva rovnocenné koncerty v~sobotu a v~nedeľu. Nechodíme evanjelizovať do mesta, preto vítame aj túto možnosť evanjelizácie nášmu mestu.

Aby sme sa na koncerty čo najlepšie pripravili, snažíme sa robiť generálne skúšky niekde mimo nášho zboru. Tento rok na generálku vianočného koncertu sme išli dva týždne pred našimi termínmi na Palisádach do nášho zboru v~Komárne.

Uvedomujeme si dôležitosť tejto služby, ktorá má dosah nielen na poslucháčov, ale aj na nás samotných, a~preto prosíme o~silu a~požehnanie pre túto prácu.

\autor{Slávo Kráľ}


\clanok{Služba ľuďom v~núdzi}
V~spolupráci s~občianskym združením {\it Kresťania v~meste} sa náš zbor aj v~roku 2018 zapojil do pomoci ľuďom v~núdzi, či už varením polievok, službou vo výdajových tímoch alebo dobrovoľnými finančnými darmi na nákup surovín na polievku. Jedlo sa vydávalo pod mostom Lafranconi počas chladných mesiacov (október až marec) hromadne každý utorok a~štvrtok od 19.30~hod. a~každú sobotu od 17.00~hod. Výdaj počas teplejších mesiacov (apríl až september) bol dvakrát do týždňa v~utorky a~štvrtky od 19.30~hod. Vo výdajovom tíme minulý rok pomohol Peter Žembery.

Uplynulý rok sme dokopy varili 26x30~litrov polievky, t.~j.~780~litrov spolu -- okrem dohodnutých mesačných termínov sa varilo aj navyše. Pre zaujímavosť uvádzam mesiace v~roku s~počtom služieb: január~4x, február~1x, marec~1x, apríl~1x, máj~1x, jún~1x, júl~4x, august~2x, september~2x, október~2x, november~4x a~v~decembri~3x.

Do prípravy a~varenia polievok sa tento rok zapojili bratia a~sestry: Slávka Volentičová so známymi~(13x), Lacko a~Elenka Taligovci~(2x), Rút Bednáriková, Marta Račičová~(2x), Danka Kešjarová~(1x), Danka Kotmanová~(1x), Vladina a~Janko Laurenčíkovci~(2x), Beata Bogárová s~mamou~(7x).

Dobrovoľné finančné dary od darcov z~nášho zboru na nákup surovín na polievky boli v~roku 2018 celkovo vo~výške~280~€. Zostatok z~roku 2017 bol~145,48~€. V~r.~2018 sa na varenie polievok použilo~325~€. Zostatok do ďalšieho roku je~100,48~€.

Zo srdca ďakujem všetkým vám, ktorí ste ochotne a~s~láskou darovali svoj čas, financie, schopnosti, poskytli ste svoje autá do tejto užitočnej služby núdznym,  spolupodieľali ste sa na prípravách polievok, či už nákupom surovín, krájaním zeleniny a~mäsa, či samotným varením. Vďaka Pánu Bohu za vás! Verím, že sa ešte mnohí z~nášho zboru zapoja, či už do varenia polievky alebo inak.

\autor{Beata Bogárová}


\clanok{Revízia hospodárenia}

Revízna komisia v~zložení Miroslav Antalík, Helena Mikletičová, Katarína Valentová za prítomnosti účtovníčky zboru Ľubomíry Kohútovej vykonali revíziu hospodárenia za rok~2018. Boli prekontrolované nasledovné doklady:
\vskip-1ex\begitems \style -
* výpisy z~bežného účtu vedeného v~Slov. sporiteľni za mesiace 3,~5,~6,~9,~10~a~12
* výdavkové pokladničné doklady za mesiace 1,~4,~6,~8,~9~a~11
* príjmové pokladničné doklady za mesiace 1,~4,~6,~8,~9~a~11
\enditems
Revízna komisia konštatuje, že uvedené doklady sú vedené prehľadne v~súlade s~účtovnými predpismi. Pokladničná kniha je vedená mesačne a~založená priamo pri pokladničných dokladoch.

Neboli zistené žiadne nedostatky.

Stav finančnej hotovosti ku dňu 31.~12.~2018 bol:

\vskip1em\hskip1cm\table{lr}{
pokladňa & 4~875,12~€ \cr
bankový účet & 34~564,12~€ \crl
spolu & 39~439,24~€ \cr
}\vskip1em

Tento stav súhlasí so stavom v~účtovnej evidencii k~uvedenému dátumu.

Dňa 2.~5.~2018 prebehla kontrola zo~Sociálnej poisťovne. Kontrolovaný bol odvod poistného na sociálne poistenie a~príspevku na starobné dôchodkové sporenie za obdobie 1.~1.~2017~--~28.~2.~2018. Výsledok kontroly je uvedený v~zápisnici a~je bez pripomienok.

Revízna komisia konštatuje zvýšenie obetavosti zboru, ktoré sa prejavilo vyššími príjmami zo zbierok a~darov.

\autor{Katarína Valentová}

\tiraz
\bye
