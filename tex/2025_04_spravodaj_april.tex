\def\velkostpisma{9}
\def\velkostriadku{12}
\input makra.tex % nacitanie Ivanom pripravenych nastaveni a prikazov
\hyphenation{star-šov-stvo} % rozdelenie slov na konci riadku, treba tu uviest slova, ktore sam nepozna

\spravodaj{4}{2025}


\clanok {Úvaha na 1. Jána 4,12}

„Nikto nikdy nevidel Boha. Ak sa milujeme navzájom, Boh v~nás zostáva a Jeho láska je v~nás dokonalá.“ Tento verš nás vyzýva na hlboké zamyslenie nad tým, akým spôsobom prežívame a vyjadrujeme lásku. V~dnešnom svete, kde je často ťažké nájsť pravú lásku a porozumenie, nám Ján pripomína, že láska je kľúčom k~poznaniu Boha. Aj keď Boha nikto nikdy nevidel, Jeho prítomnosť sa prejavuje cez naše vzťahy a vzájomnú lásku.

\cast {Láska ako odraz Boha}
Láska je jedným z~najdôležitejších atribútov Boha. Keď milujeme druhých, odrážame Jeho charakter a prítomnosť v~našich životoch. Je to, ako by sme boli zrkadlom, ktoré ukazuje Božiu lásku svetu okolo nás. Každý akt lásky, či už je to malý alebo veľký, je príležitosťou ukázať, kto Boh je. Môžeme sa pýtať: Ako môžem dnes prejavovať lásku? Ako môžem byť nástrojom Božej lásky v~živote druhých?

\cast {Dokonalosť lásky}

Ján tiež hovorí o~dokonalosti lásky. Čo to znamená mať dokonalú lásku? Nie je to o~bezchybnosti, ale o~ochote milovať aj v~ťažkých chvíľach, o~odpustení a o~trpezlivosti. Dokonalá láska je láska, ktorá sa nebojí obetovať sa pre druhých. Je to láska, ktorá sa nezameriava na vlastný prospech, ale na blaho druhých. Ako sa môžeme usilovať o~túto dokonalú lásku vo svojich vzťahoch?

\cast {Praktické kroky}

Zamyslime sa nad tým, ako môžeme túto lásku žiť v~každodennom živote. Môžeme začať malými krokmi -- úsmevom na niekoho, kto to potrebuje, pomocou susedovi, alebo jednoducho načúvaním priateľovi, ktorý prechádza ťažkým obdobím. Každý z~týchto činov je príležitosťou, ako ukázať Božiu lásku. Vedomie, že Boh v~nás zostáva, keď milujeme, nás môže povzbudiť k~tomu, aby sme sa snažili byť lepšími ľuďmi. Keď sa snažíme milovať, nielenže odrážame Boha, ale tiež sa stávame jeho nástrojmi v~tomto svete.

\autor {NA}


\clanok {Správy zo staršovstva}
Staršovstvo zboru sa stretlo v~mesiaci marec na dvoch riadnych stretnutiach a to 4. a 18.~3.~2025.
Na prvom stretnutí sa staršovstvo venovalo príprave výročného zborového zhromaždenia, ktoré sa konalo 9.~3. Program VZČZ bol bohatý, okrem prijímania nových členov zboru to bola voľba nových členov staršovstva, voľba revíznej komisie a voľba kazateľského asistenta. Ďalej to boli správy kazateľa, zborových zložiek a revíznej komisie, návrh rozpočtu na rok 2025.

Starší zboru tiež hovorili o~ďalšom termíne spoločnej zborovej víkendovky, návšteve Tima Hanesa, návrhoch na oslovenie kandidátov na ďalšieho kazateľa zboru, ktoré staršovstvo dostalo. Diskutovalo aj o~formáte možných stretnutí kazateľa zboru a starších zboru so záujemcami z~radov členov a návštevníkov zboru. Pozvalo na stretnutie sestru Tanjou Havoš, ktorá prejavila záujem stať sa členkou nášho zboru.

Na svojom druhom marcovom stretnutí sa staršovstvo venovalo zhodnoteniu VZČZ. Skonštatovalo, že voľby prebehli úspešne. Bolo zvolené nové staršovstvo v~zložení Ľubomír Syč, Peter Pribula, Radislav Nemec, Marcel Maďar a Peter Antalík. Taktiež boli zvolení tzv. starší v~zácviku -- Ján Kováčik a Martin Simon. Ďalej bola zvolená revízna komisia v~zložení Miroslav Antalík, Barbora Antalíková a Barbora Pribulová. A~napokon bol zvolený aj kazateľský asistent Filip Barkóczi.

Staršovstvo sa venovalo nastaveniu svojho organizačného fungovania, určilo spomedzi seba predsedu staršovstva (Peter Pribula), hovorilo o~prioritách práce ap. Tiež začalo rozprávať o~náplni práce kazateľského asistenta a o~tom ako v~najbližšom období riešiť otázku správcu zboru.

Ďalším nemenej dôležitým bodom stretnutia bol začiatok diskusie k~reflexii na pôsobenie a ukončenie služby brata kazateľa Petra Šrankotu, ktorú pripravil súčasný kazateľ Ján Szőllős. Potom to bolo plánovanie programu služieb na najbližšie týždne.

Staršovstvo uvíta vaše podnety a návrhy, ktoré, prosím, smerujte na členov staršovstva zboru, prípadne kazateľa zboru. Zároveň vyjadruje vďaku za modlitby a akýkoľvek záujem o~zbor.


\autor {za staršovstvo P. Antalík}


\clanok {Verš na mesiac}
„Boha nikdy nikto nevidel; ak sa milujeme, Boh zostáva v~nás a Jeho láska je v~nás dokonalá.“ (1. Jána 4,12)

\clanok {Stretnutie sestier}
Sesterské stretnutie bude v~stredu, dňa 9.~4. o~17.30 hod. na Zrínskeho 2. Pozvanou hosťkou je s.~Ester Jankovičová s~témou: Zaujíma Boha dlhá verzia môjho príbehu?

\clanok {Koncert veľkonočných piesní}
Veľký spevokol a komorný orchester CZ BJB Palisády bude mať koncert veľkonočných piesní v~sobotu, dňa 12.~4.~2025 o~17.00 hod. v~modlitebni na Palisádach.
Všetci ste srdečne vítaní.

\clanok {Sesterská konferencia v~máji}
V~nedeľu 4.~5.~2025 sa bude konať spoločné zhromaždenie v~čase konania sesterskej konferencie v~priestoroch SUZA na Drotárskej ceste. Je možné sa tu prihlásiť aj na spoločný obed. Cena pre jednu osobu je 12~€.

Dôležité upozornenie: Tí, ktorí ste sa už prihlasovali na konferenciu a aj na nedeľný obed, sa už druhýkrát nezapisujte, aby ste nemali dupľovaný obed. Je možnosť voľby aj vegetariánske jedlo. Detské porcie možné nie sú, ale ak si objednáte jednu štandardnú porciu, personál Vám ju ochotne rozdelí na dve menšie. Pre prihlásenie na tento obed sa prosím, zapíšte sa do online tabuľky čím skôr.
Ak nemáte možnosť internetu, prosím poproste svojho spolu-brata/sestru, aby Vám s~prihlásením pomohli. O~spôsobe a dátume úhrady, Vas ešte budeme informovať.


\clanok {Zbierky v~marci}
\table{lr}{
Všeobecne			& 1 879,00~€ \cr
Na misiu			&   572,00~€ \cr
Na investičný fond 	&   454,00~€ \cr}
\vskip1em

Aj naďalej máte možnosť prispieť do „nedeľnej zbierky“, a to prevodom na účet zboru. Do poznámky pre prijímateľa, prosím, uveďte „zbierka“.

Bankové spojenie: SK36 0900 0000 0000 1147 1836, SWIFT: GIBASKBX


\n  4.   4.	Viera   ŠKODÁK;
\n  6.   4.	Filip   BARKÓCZI;
\n  6.	 4.	Jarmila CIHOVÁ;
\n  6.	 4.	Jana    ZAJACOVÁ;
\n 10.	 4.	Anna	PAVLÍKOVÁ;
\n 11.	 4.	Daniel	MIKLETIČ;
\n 19.	 4.	Marta	PRIBULOVÁ;
\n 21.	 4. Ladislav TALIGA;
\n 22.	 4.	Alexander Koloman ERDÉLYI;
\n 25.	 4.	Elena   TALIGOVÁ;
\n 30.	 4. Ľuboš   DZURIAK;
\n 30.   4. Jaroslav VOLENTIČ;
\narodeniny


\program{
\p  1 ; ut ; 13.30 ; Biblická hodina pre seniorov (P. Pivka);.;;
\p  2 ; st ;.;;.;;
\p  3 ; št ; 18.00 ; Biblická hodina (J. Szőllős);.;;
\p  4 ; pi ; 17.30 ; Dorast;.;;
\p  5 ; so ; 18.00 ; Mládež;.;;
\p  6 ; ne ;  9.30 ; Bohoslužby (J. Szőllős + VP);.;;
\p  7 ; po ;.;;.;;
\p  8 ; ut ; 13.30 ; Biblická hodina pre seniorov (P. Pivka);.;;
\p  9 ; st ; 17.30 ; Stretnutie sestier (E. Jankovičová);.;;
\p 10 ; št ; 18.00 ; Biblická hodina (J. Szőllős);.;;
\p 11 ; pi ; 17.30 ; Dorast;.;;
\p 12 ; so ; 18.00 ; Mládež;.;;
\p 13 ; ne ;  9.30 ; Bohoslužby (P. Pribula);.;;
\p 14 ; po ; 18.00 ; Skupinka „Základy viery“;.;;
\p 15 ; ut ; 13.30 ; Biblická hodina pre seniorov (P. Pivka);.;;
\p 16 ; st ;.;;.;;
\p 17 ; št ; 18.00 ; Bohoslužba na Zelený štvrtok (F. Barkóczi);.;;
\p 18 ; pi ; 10.00 ; Bohoslužba na Veľký piatok (J. Szőllős);.;;
\p 19 ; so ; 18.00 ; Mládež;.;;
\p 20 ; ne ;  9.30 ; Bohoslužby (D. Uhrin);.;;
\p 21 ; po ;.;;.;;
\p 22 ; ut ; 13.30 ; Biblická hodina pre seniorov (P. Pivka);.;;
\p 23 ; st ;.;;.;;
\p 24 ; št ; 18.00 ; Biblická hodina (J. Szőllős);.;;
\p 25 ; pi ; 17.30 ; Dorast;.;;
\p 26 ; so ; 18.00 ; Mládež;.;;
\p 27 ; ne ;  9.30 ; Bohoslužby (D. Chuchút);.;;
\p 28 ; po ; 18.00 ; Skupinka „Základy viery“;.;;
\p 29 ; ut ; 13.30 ; Biblická hodina pre seniorov (P. Pivka);.;;
\p 30 ; st ;.;;.;;
}

\tiraz
\bye
