\def\velkostpisma{10}
\def\velkostriadku{12.5}
\input makra.tex % nacitanie Ivanom pripravenych nastaveni a prikazov
\hyphenation{star-šov-stvo} % rozdelenie slov na konci riadku, treba tu uviest slova, ktore sam nepozna

\spravodaj{2}{2024}


\clanok {Na počiatku Boh stvoril nebo a zem. (1M~1,1)}

„Na začiatku si stvoril zem, aj nebesia sú dielo tvojich rúk. Ony sa pominú, ale ty ostaneš, rozpadnú sa sťa odev, vymeníš ich ako rúcho a zmenia sa. Ale ty ostávaš vždy ten istý a tvoje roky sú bez konca.“ (Ž~102,26-28)

{\it Elohím / El.} Elohím je hebrejský výraz pre „Boha stvoriteľa“, ktorý sa vyskytuje hneď v~prvej vete v~Biblii. Pri mene Elohím máme na pamäti, že je to ten, ktorý stál na začiatku všetkého, stvoril nebo a zem, oddelil svetlo od tmy, vodu od súše, noc od dňa. Toto Božie meno má hĺbku významu Božej autority a zvrchovanosti, lebo On je skutočným počiatkom všetkého. Elohím ako tvar množného čísla od El sa nepoužíva preto, aby viedlo k~viere vo viacerých bohov, ale preto, aby obsiahlo veľkosť a majestát jediného Boha.

Keď hovoríme o~Božom stvoriteľskom diele, zamysli sa v~tomto kontexte, čo podľa teba znamená byť stvorený na Boží obraz. Uvažuj nad tým, ako by sa tvoj život zmenil, keby si mal stále na pamäti, že si bol stvorený, aby si bol Božím obrazom na zemi.

Vyzerá to tak, že Boh bol na konci dňa vždy nadšený z~toho, čo stvoril, keď vyhlasuje, že je to dobré, dokonca veľmi dobré. Zmení toto Božie nadšenie tvoj pohľad, to, ako sa pozeráš na svet aj na seba? Pravda je, že neskôr vstúpil hriech, ktorý narušil Bohom stvorené dielo. Ako vnímaš svoju úlohu pri tom, aby si svojim životom zjavoval Boží obraz? Vstup hriechu nemení podstatu Božieho plánu, v~ktorom túži po obecenstve s~človekom. Žime tak, aby Boží stvoriteľský podpis bol jasne viditeľný v~našom živote.

Tu je niekoľko podnetov na oslavu nášho Boha:
\vskip-1ex\begitems
* {\it Chváľ ho}: Za jeho nemennú podstatu -- aj keď sa my alebo svet okolo nás meníme. Boh ostáva taký istý.
* {\it Ďakuj mu}: Za to, že náš Stvoriteľ je dokonalý! Nepotrebuje nič pridávať alebo uberať, aby bol lepším.
* {\it Vyznaj}: Ak si chcel Pána Boha znížiť na svoju úroveň a mal si pocit, že podlieha tým istým zákonom a obmedzeniam ako my.
* {\it Popros ho}: Aby ti pomohol viac vnímať svoju veľkosť, a tak prestaneš prenášať na neho vlastné pocity a pohľady.
\enditems

Bože, ďakujem, že aj ja som Tvojím stvorením. Ty si vstúpil do chaosu a priniesol si poriadok, prosím, prines ho aj do môjho života. Ďakujem, že aj na mňa sa pozeráš ako na krásu Tvojho diela a nazývaš ma synom / dcérou, a to len pre to, že som v~Kristovi. Pomôž mi poznať Ťa hlbšie a viac ako len povrchne, že si jediný skutočný Boh. Chcem Ťa poznať viac, aj osobne. Chcem Ťa poznať ako toho, ktorý urobil všetko pre moju záchranu. Ako toho, ktorý chcel, aby som žil na zemi s~jasným zámerom oslavy Tvojho mena, Bože.
Amen.

\autor{Peter Šrankota, inšpirované knihou {\it Božie mená}}


\clanok {Správy zo staršovstva}

Staršovstvo nášho zboru sa v~januári tohto roku stretlo trikrát. Na svojom prvom stretnutí v~roku 2024, ktoré sa konalo 9.~1., bratia starší okrem pravidelnej kontroly plnenia uznesení z~predchádzajúcich stretnutí pripravovali zborové členské zhromaždenie (ZČZ) na termín 21.~1. Hlavnou témou ZČZ bola voľba ďalšieho kazateľa, a preto sa otázke pokračovania v~procese voľby venovalo aj staršovstvo.

Ďalšou dôležitou témou stretnutia bola aj diskusia o~vzdialených členoch zboru a udržiavania kontaktov s~nimi ako aj možnostiach pastorálnej starostlivosti o~nich. Témou stretnutia bolo aj obnovenie vydávania zborového spravodaja v~tlačenej forme. V~„rôznom“ sa vyskytlo najmä stanovenie, zmena či potvrdenie rôznych pripravovaných podujatí a stretnutí v~zbore a ich plánovanie.

Mimoriadne stretnutie staršovstva bolo celodenné stretnutie v~sobotu 20.~1. s~bratom Michalom Kevickým, ktoré bolo venované zhodnoteniu výsledkov prieskumu Prirodzeného rastu cirkvi (NCD), ktorý sme robili v~našom zbore už štvrtýkrát v~júni 2023. Bratia starší v~diskusii nad výsledkami hovorili o~silných oblastiach života a služby nášho zboru ako aj o~slabých oblastiach, kde je potrebné urobiť zmeny v~našej práci. Stanovili aj konkrétne kroky, ktoré by sme chceli urobiť. Základný výsledok prieskumu bol prezentovaný aj na ZČZ. Kompletné výsledky prieskumu môžu záujemcovia získať od členov staršovstva. Prieskum je len zrkadlo, ktoré sme si nastavili sami sebe. Vieme, že skutočnú zmenu pôsobí Duch Svätý.

Druhé riadne stretnutie staršovstva sa konalo 23.~1. a na programe bolo zhodnotenie víkendových stretnutí staršovstva a zboru. Hovorili sme o~spôsobe pokračovania v~práci s~výsledkami NCD a konkrétnych krokoch, ktoré staršovstvo urobí. Bratia starší zhodnotili aj priebeh ZČZ a výsledok hlasovania o~predbežnej kandidátke na voľbu ďalšieho kazateľa. Dohodli sa na pokračovaní v~procese voľby oslovením navrhnutých kandidátov, či chcú kandidovať za kazateľa nášho zboru a za akých podmienok. Staršovstvo diskutovalo aj o~možných spôsoboch financovania platu ďalšieho kazateľa.

Dôležitým bodom rokovania bola aj príprava výročného ZČZ, ktorého termín bol stanovený na 17.~marca. Dôležitou súčasťou prípravy bude vypracovanie návrhu rozpočtu. Podnety do rozpočtu môžete dávať priamo sestre Ľ.~Kohútovej. Vedúci jednotlivých zložiek majú pripraviť správu za rok~2023. Staršovstvo schválilo návrh hospodárskeho výboru o~ďalšom postupe v~príprave rekonštrukcie fasády na modlitebni nášho zboru.

V~bode „rôzne“ bratia starší rozhodovali a riešili rôzne otázky každodenného života a služby nášho zboru.

Svoje podnety, otázky, pripomienky či návrhy týkajúce sa nášho zboru môžete povedať písomne, alebo poslať písomne členom staršovstva, alebo na adresu \email{starsovstvo@bjbpalisady.sk}.

\autor {za staršovstvo Ján Szőllős}


\clanok {Dámske raňajky}

Všetky sestry pozývame na Dámske raňajky, ktoré sa budú konať v~sobotu ráno o~9.00~hod. v~stredisku ev.˚~diakonie na Partizánskej~2. Hosťom bude s.~Marta Tóthová s~témou: „Je možné mať v~tomto období dobrý / víťazný život?“ Prihlasovanie prebieha elektronicky cez tabuľku \ulink[https://bit.ly/dam_ranajky]{bit.ly/dam\_ranajky} do 5.~2.~2024. Sestry, ktoré chcú pomôcť s~ prípravami, kontaktujte s.~Angie Vráblovú.
\vfill\break


\clanok {Národný týždeň manželstva}


Národný týžden manželstva (NTM) bude prebiehať v~termíne od 11.~--~17.~2.~2024.

V~zime sa viac ako inokedy zamýšľame nad významom vitamínov pre naše telo, keďže prirodzených spôsobov, ako ich môžeme prirodzene prijať, je menej a naopak naše telo je vystavené väčšej záťaži. Podobne je to s~manželstvom -- nie v~každom jeho období máme (bez vedomého úsilia) dostatočný a pravidelný prísun vitamínov dôležitých pre zdravý vzťah. A~nikto predsa nechce v~manželstve len živoriť, ale chce zažívať vitálny vzťah, ktorý je odolný aj voči ohrozeniam a záťaži. A~k~tomu potrebuje každé manželstvo malé, no pravidelné dávky vitamínov. Ak tieto nedostáva prirodzene vďaka tomu, že je v~počiatočnej fáze zamilovanosti a starostlivosť o~vzťah ide akoby bez námahy, potrebuje ich dopĺňať cielene.

Čo môže byť vitamínom pre vitálny manželský vzťah? Napr. drobné, no každodenné vzájomné ocenenie, chvíle spoločnej zábavy alebo odhodlanie robiť to, čo druhému prináša potešenie a komunikuje lásku... Ak sa o~vitamínoch pre zdravie manželského vzťahu chcete dozvedieť viac, NTM~2024 prinesie užitočné informácie o~tom, čo pre zdravý manželský vzťah potrebujeme a tiež inšpiratívne nápady ako potrebné vitamíny našim vzťahom dopĺňať. Viac čítajte na \ulink[https://www.ntm.sk/]{ntm.sk}.


\clanok {Stretnutie sestier}

V~mesiaci február sa sestry stretnú v~stredu 21.~2.~2024 o~17.30 hod. na Zrínskeho~2. Hosťkou stretnutia bude s.~Natalija Elijas s~témou Galaťanom 6,1-2. Všetky sestry sú srdečne vítané.


\clanok {Mládežnícka konferencia}

Mládežnícka konferencia BJB sa bude konať 23. -- 25. februára 2024 v~Banskej Bystrici. „Ak hľadáš náboženstvo, ktoré je pohodlné, určite neodporúčam kresťanstvo.“ Tieto slová C.~S.~Lewisa budú mladých sprevádzať počas troch dní mládežníckej konferencie 2024. Hlavnými rečníkmi sú Pavel Hanes, Richard Nagypál a Ben Uhrin. Viac o~programe a taktiež registráciu nájdete na \ulink[https://mk.baptist.sk]{mk.baptist.sk}.
\vfill\break


\clanok{Zborová lyžovačka / zimný pobyt v~Račkovej}

Aj tento rok počas jarných prázdnin bratislavského kraja máme možnosť pobytu v~chate Račkova dolina. Ubytovanie je možné počas oboch víkendov -- pred týždňom prázdnin aj po ňom. Prihlásiť sa môžete, aj keď plánujete prísť len na pár dní. Cenník ubytovania nájdete na stránke \ulink[http://www.rackova.sk/cennik/]{www.rackova.sk/cennik}, avšak máme dohodnuté zvýhodnené ceny. Prihlasovanie a akékoľvek otázky ohľadom pobytu u~br.~Petra Antalíka \email{antalikp@yahoo.com}.

\clanok{Zbierky}
Milí bratia a sestry, ďakujeme za vašu obetavosť. V~mesiaci január 2024 ste prispeli:
\vskip-1ex\begitems
* misia: 731 €
\enditems

\n 3.	2.	Miroslav	ANTALÍK;
\n 3.	2.	Vlasta	BALÁŽOVÁ;
\n 5.	2.	Štefánia	ANTALÍKOVÁ;
\n 5.	2.	Barbora	ANTALÍKOVÁ;
\n 11.	2.	Juraj	BALÁŽ;
\n 11.	2.	Beata	BOGÁROVÁ;
\n 11.	2.	Oľga	KOVÁČOVÁ;
\n 12.	2.	Martin	PRIBULA;
\n 15.	2.	Ingrid	JANČULOVÁ;
\n 16.	2.	Lenka	PRIBULOVÁ;
\n 18.	2.	Ľudmila	VIDA;
\n 23.	2.	Anna	RUCIN;

\narodeniny


\program{
\p  1 ; št ; 18.00 ; Biblická hodina (J. Szőllős) ;.;;
\p  2 ; pi ; 17.30 ; Dorast ;.;;
\p  3 ; so ; 18.00 ; Mládež ;.;;
\p  4 ; ne ;  9.30 ; Bohoslužby (P. Šrankota + VP) ;.;;
\p  5 ; po ;.;;.;;
\p  6 ; ut ; 15.15 ; Biblická hodina pre seniorov (P. Pivka) ;.;;
\p  7 ; st ;.;;.;;
\p  8 ; št ; 18.00 ; Biblická hodina (J. Szőllős) ;.;;
\p  9 ; pi ; 17.30 ; Dorast ;.;;
\p 10 ; so ;  9.00 ; Dámske raňajky ;.;;
\p 11 ; ne ;  9.30 ; Bohoslužby (J. Szőllős) ;.;;
\p 12 ; po ;.;;.;;
\p 13 ; ut ; 15.15 ; Biblická hodina pre seniorov (P. Pivka) ;.;;
\p 14 ; st ;.;;.;;
\p 15 ; št ; 18.00 ; Biblická hodina (J. Szőllős) ;.;;
\p 16 ; pi ; 17.30 ; Dorast ;.;;
\p 17 ; so ; 18.00 ; Mládež ;.;;
\p 18 ; ne ;  9.30 ; Bohoslužby (I. Staroň) ;.;;
\p 19 ; po ;.;;.;;
\p 20 ; ut ; 15.15 ; Biblická hodina pre seniorov (P. Pivka) ;.;;
\p 21 ; st ; 17.30 ; Stretnutie sestier ;.;;
\p 22 ; št ; 18.00 ; Biblická hodina (J. Szőllős) ;.;;
\p 23 ; pi ; 17.30 ; Dorast ;.;;
\p 24 ; so ; 18.00 ; Mládež ;.;;
\p 25 ; ne ;  9.30 ; Bohoslužby (P. Šrankota) ;.;;
\p 26 ; po ;.;;.;;
\p 27 ; ut ; 15.15 ; Biblická hodina pre seniorov (P. Pivka) ;.;;
\p 28 ; st ;.;;.;;
\p 29 ; št ; 18.00 ; Biblická hodina (J. Szőllős) ;.;;
}
\vskip1ex
\riadokkoncaprogramu{Z bohoslužieb je zabezpečený online prenos.}


\tiraz
\bye
