\shyph % use csplain
\input opmac
%\input lmfonts
%\input cs-pagella
\margins/1 a5 (1.2, 1.2, 1.2, 1.7)cm
\parindent 1em % odstavcova zarazka
\emergencystretch=2em % roztiahnutie medzislovnych medzier v riadku

\hyperlinks{\Blue}{\Blue} % vnitřní odkazy modré, URL zelené
% vseobecne fonty
%\font\tenrm="Lucida Sans Unicode:mapping=tex-text"
%\font\tenit="Lucida Sans Unicode:mapping=tex-text,slant=0.2"
%\font\tenbf="Lucida Sans Unicode:mapping=tex-text,embolden=1.5"
%\font\tenbi="Lucida Sans Unicode:mapping=tex-text,slant=0.2,embolden=1.5"
%\font\tenrm="Linux Libertine O:mapping=tex-text"
%\font\tenit="Linux Libertine O/I:mapping=tex-text"
%\font\tenbf="Linux Libertine O/B:mapping=tex-text"
%\font\tenbi="Linux Libertine O/BI:mapping=tex-text"
% Poznamka k "Butler" fontom: Fonty stiahnute z webstranky tvorcu nie su priamo pouzitelne, je potrebne ich otvorit v programe FontForge a vyexportovat (File/Generate Fonts...).
% Dovod: niektore pismena s diakritikou nie su vo fonte ulozene korektne (neviem co to presne znamena) a XeTeX ich nevie pouzit, FontForge zobrazuje pre nich nejaky warning.
% Reexport fontu odstranuje problem (nevie preco), warning zmizne a XeTeX je tiez spokojny. Tento postup som nasiel na internete (v suvislosti s inym fontom).
\font\tenrm="[Butler_Regular-reexported]:mapping=tex-text"
\font\tenit="[Butler_Regular-reexported]:mapping=tex-text,slant=0.2"
\font\tenbf="[Butler_Bold-reexported]:mapping=tex-text"
\font\tenbi="[Butler_Bold-reexported]:mapping=tex-text,slant=0.2"
%\typosize[10/13] - standard
\typosize[10/12.5]% zvacsuje pismo, pretoze to inak vychadza na 5 stran

% fonty pre specificke pouzitie
\font\ftitle="[Butler_Bold-reexported]" at42pt
\font\ffootertitle="[Butler_Regular-reexported]" at9.5pt
\def\fissuemonth{\ftitle}
\font\fissueyear="[Butler_Bold-reexported]" at20pt

\def\monthname#1{%
\ifcase#1
\or január% 1
\or február% 2
\or marec% 3
\or apríl% 4
\or máj% 5
\or jún% 6
\or júl% 7
\or august% 8
\or september% 9
\or október% 10
\or november% 11
\or december% 12
\fi}
\def\nazovmesiaca{\monthname\cislomesiaca}

% Titulka
\newcount\issueyear
\def\rom#1{\uppercase\expandafter{\romannumeral#1}}
\def\spravodaj#1#2{\gdef\cislomesiaca{#1}\gdef\rokvydania{#2}
%v oboch nasledujucich riadkoch musi byt ciselny rozmer rovnaky ale s opacnym znamienkom (napr. 4cm a -4cm)
\picheight=4cm\hfill\inspic kostol.png
\vskip-4cm
{\noindent\ftitle Spravodaj\hfill#1{\fissueyear\ #2}}\par
\issueyear = #2
\advance \issueyear by -1997
\vskip-0.3cm
\noindent\hfill ročník \rom{\the\issueyear}\par
\noindent {\typosize[11/15]\bf cirkevného zboru BJB Palisády}\par
\noindent{\typosize[9/12]\ulink[https://www.bjbpalisady.sk/spravodaj]{www.bjbpalisady.sk/spravodaj}}\par
\vskip 1.6cm
\noindent{\it ...hovorím vám, že aj ten, kto verí vo mňa, bude konať skutky, aké ja konám, ba bude konať ešte väčšie... \hfill Ján 14, 12}%
\footline={\ifnum\pageno=1 \hfil \else \ifodd\pageno {\ffootertitle spravodaj}\kern0.8em #1/#2\hfil \the\pageno \else \the\pageno \hfil {\ffootertitle spravodaj}\kern0.8em #1/#2\fi\fi}%
}

% Tiraz
\long\def\tiraz{
\footline={\hfil} % na poslednej strane nechceme paticku
\vfill % spolu s \vfil\break na konci makra to sposobi umiestnenie textu na koniec stranky
\centerline{\typoscale[900/]\table{|c|}{\crl
e-mail: \email{zbor@bjbpalisady.sk} webstránka: \ulink[http://www.bjbpalisady.sk/]{www.bjbpalisady.sk} telefón: 207 00 307 \cr
vydal Cirkevný zbor Bratskej jednoty baptistov Bratislava I Palisády (Zrínskeho 2, 81103)\cr
bankové spojenie (IBAN): SK36 0900 0000 0000 1147 1836\cr
redakcia: D. Uhrin, M. Kešjarová, grafická úprava: M. Kohút, I. Kohút\crl
}}
\vfil\break
}

\def\clanok#1{\nonum\secc{#1}}
\def\autor#1{{\hfill\tenit #1}\par}

% Narodeniny

\newcount\pocetoslavencov \pocetoslavencov=0
\def\oslavenciden[#1]{\csname oslavenciden:#1\endcsname}
\def\oslavencimesiac[#1]{\csname oslavencimesiac:#1\endcsname}
\def\oslavencimeno[#1]{\csname oslavencimeno:#1\endcsname}
\def\pridajoslavenca#1#2#3{\advance\pocetoslavencov by 1
\expandafter\def\csname oslavenciden:\the\pocetoslavencov\endcsname{#1}
\expandafter\def\csname oslavencimesiac:\the\pocetoslavencov\endcsname{#2}
\expandafter\def\csname oslavencimeno:\the\pocetoslavencov\endcsname{#3}
}
\def\n #1.#2.#3;{\pridajoslavenca{#1}{#2}{#3}}

\newdimen\halfhsize
\halfhsize=\hsize
\divide\halfhsize by 2

\def\printnarodeniny#1#2#3{\hbox to0.5cm{\hfil #1}\hbox to0.5cm{#2\hfil}\vrule\hskip0.5em #3\hfil\vrule}
\def\printoslavenec#1{\printnarodeniny{\oslavenciden[#1].}{\oslavencimesiac[#1].}{\oslavencimeno[#1]}}
\def\printprazdnyoslavenec{\printnarodeniny{}{}{}}
\def\printriadoknarodenin#1#2{\hbox to\hsize{\hbox to\halfhsize{\vrule height2.5ex depth1ex #1}#2}\par\hrule}

\newcount\oslavenciindex \oslavenciindex=0
\newcount\oslavenciindexleft
\newcount\oslavenciindexright
\newcount\polovicapoctuoslavencov

\def\narodeniny{
% hlavicka
\vfill
\vbox{
\centerline{\bf Výročie príchodu na tento svet oslavujú}\nobreak
\frame{\vbox{
\hrule\nobreak
% oslavenci
\polovicapoctuoslavencov=\pocetoslavencov
\divide\polovicapoctuoslavencov by 2
\ifodd \pocetoslavencov \advance\polovicapoctuoslavencov by 1 \fi
\loop\ifnum \polovicapoctuoslavencov>\oslavenciindex{
  \oslavenciindexleft=\oslavenciindex
  \advance\oslavenciindexleft by 1
  \oslavenciindexright=\oslavenciindexleft
  \advance\oslavenciindexright by \polovicapoctuoslavencov
  \parindent=0pt
  \printriadoknarodenin{\printoslavenec{\the\oslavenciindexleft}}{\ifnum\oslavenciindexright>\pocetoslavencov\printprazdnyoslavenec\else\printoslavenec{\the\oslavenciindexright}\fi}
}
\advance\oslavenciindex by 1
\repeat
}}}
\vfil\break
}

% Program

\newcount\hasmultieventrowtoday % 0 - default, netlacime zvislu ciaru za prvym eventom v danom riadku, 1 - tlacime medzeru, lebo v dany den uz boli dva eventy v jednom riadku
\hasmultieventrowtoday=0
\newcount\issunday
\def\textcell#1#2{\hskip2pt\vtop{\noindent\hsize#2cm\baselineskip=1pt#1{\emergencystretch=2em\par}\vskip1pt}\hskip2pt}
\def\programevent#1#2#3#4{\vrule\hbox to.4cm{\hfil #1}%
\hbox to.4cm{\kern.5pt\lower.21ex\hbox{$^{#2}$}\hfil}%
\textcell{#3}{#4}%
}
\def\p#1;#2;#3.#4;#5;#6.#7;#8;{{\typoscale[950/]\if^#1^\hrule height0pt\else\hrule\global\hasmultieventrowtoday=0\gdef\dayofweek{#2}\fi\hbox to\hsize{%
\ifnum\strcmp{ ne }{\dayofweek}=0\issunday=1\else\issunday=0\fi%
\vrule height2.5ex% depth1ex%
{%
%ciselny rozmer width a kern musi byt rovnaky, ale s opacnym znamienkom (napr. 1.5cm a -1.5cm)
\ifnum 1=\issunday{\localcolor \LightGrey \vrule width1.52cm}\kern-1.52cm\bf\fi%
\hbox to.4cm{\hfil #1\unskip}%
\hbox to.5cm{\if^#1^\ \else .~\cislomesiaca .\fi\hfil}%
\vrule\hbox to.6cm{#2\hfil}%
}%
\if^#6^\ifnum 0=\hasmultieventrowtoday\programevent{#3}{#4}{#5}{9.5}\else\programevent{#3}{#4}{#5}{4.45}\programevent{}{}{}{4.45}\fi\else%
\programevent{#3}{#4}{#5}{4.45}%
\programevent{#6}{#7}{#8}{4.45}%
\global\hasmultieventrowtoday=1
\fi%
\hfil\vrule\par}}}
\def\program#1{\centerline{\bf Program na \nazovmesiaca}\nobreak\frame{\vbox{#1\hrule}}
\noindent\centerline{\it Ak nie je uvedené inak, zhromaždenia a bohoslužby sa konajú v modlitebni na Palisádach.}
}

% Linky
\def\email#1{\ulink[mailto:#1]{#1}}

\rm
