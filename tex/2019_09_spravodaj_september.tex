% DOKUMENTACIA:

% Prazdny riadok za textom znamena ukoncenie odstavca.
% Cierne obldzniky na konci riadku (v PDF) - to nechaj na mna (moze to o.i. znamenat, ze treba pridat nejake slovo do \hyphenation, lebo ho sam nevie rozdelit na konci riadku)

% Prikazy pre casti spravodaja:
% \spravodaj{<mesiac>}{<rok>}
% \clanok{<nazov clanku>}
% \autor{<autor clanku>}
% \n<den.mesiac.meno> - zadefinovanie oslavenca
% \narodeniny - vytvorenie tabulky s~narodeninami vsetkych zadefinovanych oslavencov
% \tiraz - ukoncenie spravodaja tirazou

% Styl fontu:
% \bf - bold, plati do konca aktualne skupiny, napr. ak mas {aaa \bf bbb} ccc, tak aaa bude normalne, bbb bude bold, ccc bude normalne
% \it - italic (pouzit rovnakym sposobom ako \bf)
% \bi - bold italic (pouzit rovnakym sposobom ako \bf)
% \rm - normalne (pouzit rovnakym sposobom ako \bf)

% Dalsie prikazy a znaky:
% \begitems - zoznam (odrazky), informacie najdes na stranke http://petr.olsak.net/ftp/olsak/opmac/opmac-u.pdf#toc%3A.5
% \ulink[<cielova adresa]{<zobrazena adresa>} - klikatelny odkaz na webstranku
% \email{<adresa>} - klikatelny odkaz na e-mailovu adresu
% ~ - nedelitelna medzera, napr. v~dome, 21.~6.~2018
% -- - pomlcka (dvakrát -)
% „ - zaciatocna uvodzovka
% “ - koncova uvodzovka
% \noindent - najblizsi odstavec nebude odsadeny
% \vskip<velkost> - vertikalna medzera, napr. \vskip3pt alebo \vskip-3ex (zaporna medzera, t.j. posun smerom hore)

%\typosize[9/12]% - pouzita velkost pisma/riadku (standard)
\input makra.tex % nacitanie Ivanom pripravenych nastaveni a prikazov
\hyphenation{star-šov-stvo} % rozdelenie slov na konci riadku, treba tu uviest slova, ktore sam nepozna

\spravodaj{9}{2019}


\clanok {Odbočka do neznámeho}
Toto leto som cestou z~Washingtonu do Arkansasu spolu s~Elliotom videl niečo, čo sa mi ani nechcelo veriť. Elliot mi povedal, že sa od kamaráta dozvedel, že v~Kansase sa nachádza skalisko s~názvom Monument Rocks. Vraj dosahuje výšku 20 metrov, čo sa mi zdalo nemožné, pretože Kansas je jedno obrovské pole. Nechcelo sa mi tomu veriť. V~Kansase nie sú žiadne hory ani kopce, len samá rovina.

Odbočili sme z~hlavnej cesty, potom sme znovu odbočili a znovu a pokračovali sme po jednej veľmi zanedbanej ceste pomedzi polia asi ďalších desať kilometrov. A~predstavte si -- tam sa pred nami týčilo skalisko Monument Rocks. Skutočne sme ho našli.

Pred nami sa rozprestieralo niekoľko obrovských skál, asi sto metrov širokých a dvadsať metrov vysokých, ako stena z~ničoho v~strede poľa. S~Elliotom sme boli úplne nadšení, že sme odbočili z~tej známej cesty a vydržali sme po tých bočných cestách až k~cieľu. Oplatilo sa to.

Som presvedčený o~tom, že aj s~Bohom je to často podobne. Počuli sme o~Ňom niečo, čo sa nám zdá neuveriteľné, ale nechceme vybočiť z~toho známeho, aby sme to osobne preskúmali a zažili. Vyžaduje si to trochu viac času a energie a veľmi sa nám do toho nechce. Sme spokojní a pohodlní, a preto premeškáme úžasné veci. Povieme síce, že Boh je obrovský a všemohúci, ale naša osobná skúsenosť je stále to nudné známe. Čo keby sme odbočili z~toho známeho a inak hľadali obrovského Boha? V~liste Rimanom 11,33 sa píše: „Ó, hĺbka Božieho bohatstva, múdrosti a poznania! Aké nepochopiteľné sú Jeho súdy a aké nevyspytateľné Jeho cesty.“

Za čo sa teraz modlíš? Čo je tou vecou, ktorú dokáže len Boh? Na čo tento mesiac očakávaš, čo sa bez Boha nedá? Veľmi ľahko sa uspokojíme. Odboč k~niečomu neznámemu, naber vieru a nájdi všemohúceho Boha tak, ako si doteraz len počul.

\autor{Danny Jones}


\clanok {Milujem svoje mesto}
Pod záštitou platformy {\it Kresťania v~meste} sa ako zbor chceme zapojiť do spoločnej služby mestu. Predmetom našej zborovej služby bude Základná škola Milana Hodžu a oblasť v~jej okolí. Stretneme sa v~sobotu 14.~9. o~8.00~hod. pred modlitebňou na Palisádach. Táto praktická pomoc mestu je výbornou príležitosťou na evanjelizáciu. Preto vás chcem vyzvať k~tomu, aby ste na tento dobrovoľnícky deň pozvali aj svojich priateľov. Verím, že viacerí sa zapoja, zoznámia sa s~ľuďmi z nášho zboru a budú môcť popri práci počuť aj o~Ježišovi.

Modlite sa za to spolu so mnou!

\autor {Danny Jones}


\clanok {Konferencia seniorov BJB v~SR}
Srdečne pozývame staršiu generáciu bratov a sestier BJB zo Slovenska a Česka na konferenciu seniorov v~táborovom duchu, ktorá sa bude konať v~Račkovej doline v~dňoch 25.~--~29.~septembra 2019.

Téma konferencie je {\it Čas počúvať}.

Cena za ubytovanie hotelového typu s~plnou penziou je 80~€.

Prihlásiť sa je možné u~Anny Trnavskej, správkyne chaty, na tel. č.: 044/5293~292, 0903~501~852 alebo e-mailom: \email{info@rackova.sk}.


\clanok {Motivačný víkend pre manželov}
V dňoch 3.~--~6.~októbra sa v~Račkovej doline bude konať motivačný víkend pre manželov, ktorého program bude obsahovať nasledovné témy:
\vskip-1ex\begitems \style n
* Intimita -- dôverní priatelia
* Vieš, čo potrebujem?
* Keď pohár pretečie
* Aby manželstvo nebolelo
\enditems

Cena kurzu je 98~€ na osobu, spolu s~ubytovaním a plnou penziou.
Prihlášku nájdete na webovej stránke \ulink [http://www.nasemanzelstvo.sk/]{www.nasemanzelstvo.sk}.


\clanok {Konferencia {\it 100 let spolu}}
Spoločná česko-slovenská konferencia BJB s~názvom {\it 100 let spolu} sa uskutoční v~termíne 25.~--~27.~októbra v~Litoměřiciach v~Českej republike.
Ešte stále je možnosť prihlásiť sa, a to prostredníctvom webovej stránky \ulink [https://www.stoletspolu.cz/registrace/]{stoletspolu.cz/registrace} (pre samotné prihlásenie je po otvorení stránky potrebné kliknúť na „odkaz“ v~treťom odstavci).


\clanok{Spoločné modlitby}
\vskip-1ex\begitems
* Muži -- streda {\bf od 6.00~hod. do 7.00~hod.}, kostol na Palisádach
* Ženy -- pondelok {\bf od 17.00~hod.}, Zrínskeho 2
\enditems

Priveďte na spoločné modlitby aj svojich priateľov a známych, ktorým leží na srdci naše mesto a ľudia v~ňom.


\clanok {Stretnutia detskej besiedky a dorastu}
Pravidelné stretnutia detskej besiedky a dorastu začnú 15.~septembra.


\clanok {Nácvik spevokolu}
Prvý nácvik veľkého spevokolu v~novom školskom roku bude 15.~septembra o~18.00 v~modlitebni na Palisádach.


\clanok {Služba núdznym}
Dňa 23.~septembra sa bude konať stretnutie dobrovoľníkov slúžiacich ľuďom bez domova. Stretnutie bude o~18.00 na Cablkovej~3.

Budeme veľmi radi, ak si tento termín rezervujete. Tešíme sa na osobné stretnutie s~vami, duchovné povzbudenie a modlitby.

\autor {Lenka Antalíková}


\clanok {Kniha {\it Malé veľké postavy Biblie} od J. Pribulu}
Radi by sme vám dali do pozornosti novú knihu od zosnulého brata kazateľa Juraja Pribulu pod názvom {\it Malé veľké postavy Biblie}. Kniha približuje čitateľovi mužské a ženské postavy Písma, o~ktorých nemáme veľa informácií, no napriek tomu sú ich príbehy dôležité.

Knihu si môžete kúpiť v~nedeľu po zhromaždení, prípadne po dohode u~sestry Elenky Pribulovej. Cena je 5~€.


\clanok{Verš na zapamätanie}
Na mesiac september máme nový veršík, ktorý sa chceme spoločne učiť. Veríme, že poznanie Písma prospeje našej duši i našej mysli:

{\it „On nás vytrhol z~moci tmy a preniesol do kráľovstva svojho milovaného Syna, v~ktorom máme vykúpenie a odpustenie hriechov.“}

\autor{Kol~1,~13~--~14}


\clanok{Zbierky za uplynulé obdobie}
Milí bratia a sestry, ďakujeme za vašu obetavosť. V~uplynulom období ste prispeli:
\vskip-1ex\begitems
* investičný fond: 368,40~€ (júl -- august)
* misia: 408,50~€ (júl -- august)
\enditems


\n 5.	9.	Dušan	UHRIN;
\n 9.	9.	Daniel	VALENTA;
\n 14.	9.	Štefan	SYNOVEC;
\n 16.	9.	Daniel	PLETT;
\n 19.	9.	Richard	HALAMÍČEK;
\n 21.	9.	Kvetoslava	MAĎAROVÁ;
\n 21.	9.	Miroslava	SIMONOVÁ;
\n 22.	9.	Viera	KOLÁROVSKÁ;
\narodeniny


\program{
\p 1  ; ne ;  9.30 ; Bohoslužby (D. Jones); 10.00 ; Chvojnica (J. Szőllős) ;
\p 2  ; po ; 17.00 ; Modlitby -- ženy (Zrínskeho 2) ;.;;
\p 3  ; ut ; 15.15 ; Stretnutie pri Biblii (P. Pivka, Zrínskeho 2) ;.;;
\p 4  ; st ;  6.00 ; Modlitby -- muži (kostol Palisády) ;.;;
\p 5  ; št ;.;;.;;
\p 6  ; pi ;.;;.;;
\p 7  ; so ;.;;.;;
\p 8  ; ne ;  9.30 ; Bohoslužby (D. Jones) ; 10.00 ; Chvojnica (J. Štefko) ;
\p 9  ; po ; 17.00 ; Modlitby -- ženy (Zrínskeho 2) ;.;;
\p 10 ; ut ; 15.15 ; Stretnutie pri Biblii (P. Pivka, Zrínskeho 2) ;.;;
\p 11 ; st ;  6.00 ; Modlitby -- muži (kostol Palisády) ;.;;
\p 12 ; št ;.;;.;;
\p 13 ; pi ;.;;.;;
\p 14 ; so ;  8.00 ; Milujem svoje mesto ; 18.00 ; Mládež (Súľovská 2) ;
\p 15 ; ne ;  9.30 ; Bohoslužby (P. Kolárovský) ; 10.00 ; Chvojnica (P. Škulec) ;
\p 16 ; po ; 17.00 ; Modlitby -- ženy (Zrínskeho 2) ;.;;
\p 17 ; ut ; 15.15 ; Stretnutie pri Biblii (P. Pivka, Zrínskeho 2) ;.;;
\p 18 ; st ;  6.00 ; Modlitby -- muži (kostol Palisády) ;.;;
\p 19 ; št ;.;;.;;
\p 20 ; pi ;.;;.;;
\p 21 ; so ; 18.00 ; Mládež (Súľovská) ;.;;
\p 22 ; ne ;  9.30 ; Bohoslužby (D. Jones); 10.00 ; Chvojnica (M. Antalík) ;
\p 23 ; po ; 17.00 ; Modlitby -- ženy (Zrínskeho 2) ;.;;
\p 24 ; ut ; 15.15 ; Stretnutie pri Biblii (P. Pivka, Zrínskeho 2) ;.;;
\p 25 ; st ;  6.00 ; Modlitby -- muži (kostol Palisády) ;.;;
\p 26 ; št ;.;;.;;
\p 27 ; pi ;.;;.;;
\p 28 ; so ; 18.00 ; Mládež (Súľovská) ;.;;
\p 29 ; ne ;  9.30 ; Bohoslužby (D. Jones); 10.00 ; Chvojnica (Uherské Hradiště) ;
\p 30 ; po ; 17.00 ; Modlitby -- ženy (Zrínskeho 2) ;.;;
}


\tiraz
\bye
