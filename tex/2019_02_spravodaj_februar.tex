\input makra.tex % nacitanie Ivanom pripravenych nastaveni a prikazov
\hyphenation{star-šov-stvo} % rozdelenie slov na konci riadku, treba tu uviest slova, ktore sam nepozna

\spravodaj{2}{2019}


\clanok {Atlét dodržiavajúci pravidlá}
Asi najčastejšou témou a otázkou, s~ktorou sa posledné dni stretávam, je téma a otázka rýchlosti. Ani neviem, koľkokrát som za posledné obdobie počul povzdych: „Ale ten čas letí, však?“

No realita naozaj je tá, že už skončil január. A ako som sa dočítal v~úvodníku jedného bežeckého časopisu: „...už pomaly prestávame plniť prvé predsavzatia na rok 2019...“ Dnes sa skutočne veľmi veľa hovorí o~zdravom životnom štýle, o~výžive, stravovaní, športe...

Každý z~nás potrebuje vo svojom živote počuť slová povzbudenia a vidieť, že smer, ktorým v~živote ide, je správny. No povzbudenie potrebujeme aj v~živote viery. Keď som o~tom všetkom premýšľal, priviedlo ma to k~textu z~2.~Timoteovi~2,~5. Ak aj niekto preteká, nebude ovenčený, ak nepreteká podľa pravidla.

A skutočne, každý šport mal a má svoje pravidlá. Každé podujatie malo a má svoje ceny. Víťazi gréckych hier dostávali vavrínové vence. Neboli to zlaté či strieborné medaily, ako je to dnes. No pravda je, že nijaký atlét, nech bol akokoľvek dobrý, nebol ovenčený, ak nesúťažil podľa pravidiel.

Božie slovo život kresťana na viacerých miestach prirovnáva práve k~pretekom. Nehovorí však o~tom, že máme súťažiť navzájom proti sebe, ale v~zmysle tvrdej sebadisciplíny pri tréningu a v~tom, že máme zložiť všetko, čo je nám na ťarchu.

Pavol už vo svojom prvom liste Timoteovi nabádal: „Vyhýbaj sa bezbožným a babským bájkam. Radšej sa cvič v~zbožnosti. Telesné cvičenie je totiž málo užitočné, ale zbožnosť je užitočná na všetko, lebo má prisľúbenie pre terajší aj budúci život.“

Rovnako sám apoš. Pavol vyznáva v~liste Filipským 3,~12 -- 14: „Nie že by som to všetko bol už dosiahol a bol už dokonalý, ale bežím, aby som to získal, pretože si ma získal aj Kristus Ježiš. Bratia, ja si nenamýšľam, že som to už získal, ale len jedno robím: zabúdam na to, čo je za mnou, a usilujem sa o~to, čo je predo mnou. Bežím k~cieľu za nebeskou cenou, totiž za Božím povolaním v~Kristovi Ježišovi.“

Teraz ho upozorňuje, aby sa riadil pravidlami. Ten, kto sa prihlási do športovej súťaže, musí rešpektovať jej pravidlá. To znamená žiadne skratky.

Z ľudského pohľadu Pavol iba strácal. Nikto nebol na tribúne, aby mu fandil a povzbudzoval ho. Naopak, bol vo väzení a trpel ako zločinec. Napriek tomu bol víťazom! Dodržal pravidlá dané Božím slovom.

Pavol hovoril mladému Timoteovi: „Je dôležité, aby si sa riadil Božím slovom bez ohľadu na to, čo tomu budú hovoriť ľudia. Nebežíš kvôli ľuďom, alebo aby si sa preslávil. Bežíš k~cieľu za nebeskou cenou.“

A tak nech je aj pre teba povzbudením a stane sa i tvojím osobným vyznaním vyznanie apoš. Pavla v~2. Timoteovi 4, 7 -- 8: „Dobrý boj som dobojoval, beh som dokončil, vieru som zachoval. Už mám pripravený veniec spravodlivosti, ktorý mi dá v~onen deň Pán, spravodlivý sudca, no nie iba mne, ale aj všetkým, čo s~láskou očakávajú jeho zjavenie.“

\autor{Miroslav Tóth}


\clanok{Správy zo staršovstva}
Staršovstvo dostalo na rok 2019 verš z~listu Kolosenským 1, 10: „...aby ste žili hodní Pána a páčili sa mu vždy, keď budete prinášať ovocie v~každom dobrom skutku a hlbšie poznávať Boha.“ Sú tam štyri veci:
\vskip-1ex\begitems \style n
* Žiť život, ktorý bude svedectvom o~Pánovi Ježišovi Kristovi. Je to život tých, ktorí sú občanmi nebeského kráľovstva.
* Páčiť sa Pánovi Ježišovi Kristovi. Našou ľudskou snahou je urobiť všetko pre to, aby sme vyhoveli predstavám ľudí okolo nás. Predstavám v~práci, v~rodine, v~škole, v~zbore, medzi kamarátmi, kolegami, spolužiakmi či manželmi. Ale podľa tohto slova sa máme páčiť niekomu inému. Má to byť náš Pán, ktorému sa máme páčiť. A máme pre to urobiť všetko. Ísť aj za hranice vlastného JA.
* Prinášanie ovocia. Žalmista to vystihuje asi najlepšie. Ovocie prináša ten, kto je „...zasadený nad potokmi vôd, ktorého list nevädne...“ Potoky vôd je prítomnosť Božia v~mojom aj vašich životoch. Iba tak dokážeme žiť život, ktorý sa páči Pánovi, prináša ovocie a stále viac Ho poznáva.
* A to je to štvrté. Poznávanie Boha. Bez toho, aby sme mali túžbu po Jeho prítomnosti v~našom živote, Ho nemôžeme poznávať. Poznáme iba toho, s~kým sme v~kontakte. Pre poznanie Pána Ježiša potrebujeme byť v~neustálom kontakte s~Ním.
Je to výzva pre súčasné aj budúce staršovstvo, ale aj pre každého z~nás.
\enditems

V ostatnom období sa zaoberáme voľbami a doplnením do niektorých zložiek zboru. Túto časť práce vnímame ako veľkú zodpovednosť pred Pánom. Nájsť a vybrať pracovníkov do zborových zložiek je zodpovedná a náročná práca. Vieme, že túto prácu je potrebné urobiť, ale rovnako vnímame, že to nemôže byť iba naše ľudské vnímanie, ale že to musí byť Božie povolanie do služby pre každého, kto sa rozhodne povedať ÁNO.

Dlhodobo vnímame potrebu posilniť tím bratov, ktorí pomáhajú pri vysluhovaní Večere Pánovej. Oslovili sme viacerých bratov z~nášho zboru, ktorých vnímame ako mužov, ktorí svoj vzťah s~Bohom myslia vážne. Mužov, ktorí nie sú „nováčikovia“ vo viere. Mužov, ktorí možno aj prešli mnohými skúškami, ale na „rázcestí života“ sa rozhodli ísť cestou,  ktorá je v~Biblii označená ako úzka. Kandidátov za diakonov, pomocníkov pri vysluhovaní Večere Pánovej nájdete v~tomto spravodaji. Prosíme vás všetkých o~modlitby predtým, ako budeme na VCZČZ rozhodovať o~ich zaradení do služby. Chceme, aby toto naše rozhodovanie bolo vedené Božím Svätým Duchom.

Súčasnému staršovstvu končí volebné obdobie. Keď sa náš brat kazateľ dozvedel, že majú byť nové voľby a do staršovstva môžu byť zvolení noví členovia zboru, nebolo mu to celkom jedno. Jeho túžbou bolo, aby mohol spolupracovať so súčasným staršovstvom. Pre nás ľudí to môže byť lichotivé. Ale dôležité je to, čo si o~tom myslí náš Pán. Prieskum medzi oslovenými ukázal, že  záujem o~tento druh služby v~zbore nie je veľmi veľký. Vyzerá to, ako keby nesenie zodpovednosti za celý zbor, bolo príliš veľkým bremenom. Napriek tomu potrebujeme vybrať spomedzi seba tých, ktorí budú toto bremeno služby niesť. Zoznam kandidátov, tých za ktorých sa potrebujeme modliť, nájdete tiež v~spravodaji.

Spolu so staršovstvom končí volebné obdobie aj revíznej komisii. Môže sa zdať, že táto služba je jednoduchšia, alebo menej zodpovedná. Opak je však pravdou. Pán Ježiš hovorí „...Dobre, sluha, dobrý a verný, nad málom si bol verný, nad mnohom ťa ustanovím...“ Akákoľvek drobná služba, ktorú verne vykonávame, nás stavia do pozície tých, ktorí sú v~nebeskom kráľovstve ustanovení do služby. Okrem tohto, čistota a presnosť vo financiách je veľmi dôležitým svedectvom pred ľuďmi z~tohto sveta. Aj do služby Revíznej komisie potrebujeme vybrať tých, ktorí budú slovami Písma „dobrí a verní“. Modlime sa aj za túto voľbu a tých, ktorí sú ochotní zobrať na seba túto prácu a zodpovednosť.

Zároveň sa venujem otázkam služieb v~našom zbore.

Službu sestrám prevzala Clara Jones. Sme vďační za to, že okrem kazateľa zboru, máme aj sestru, manželku kazateľa, ktorej Pán Ježiš položil na srdce službu zboru a konkrétne ženám. Túžim po tom, aby táto služba priniesla ovocie nie iba sestrám samotným, ale aj manželstvám, rodinám a celému zboru.

Nedeľné kázne sú momentálne asi hlavným vyučujúcim prvkom v~zbore. Vytvorili sme tím bratov, ktorí majú na starosti tematické zameranie kázní. Chceme, aby toto tematické zameranie budovalo náš zbor, viedlo a vyučovalo nás k~obrazu najpriateľskejšieho, najláskavejšieho zboru v~Bratislave. Prvých kresťanov poznali podľa lásky. Poznali ich podľa toho, ako sa milovali. V~dnešnej dobe môže toto vyjadrenie v~neveriacich vyvolať svetské predstavy. Potrebujeme sa preto modliť o~to, aby naša láska prezentovala len a len lásku Pána Ježiša. Lásku, ktorá sa obetovala za nás a za naše hriechy. Lásku, ktorá sa obetovala za tých, ktorí Ho dnes ešte nepoznajú alebo odmietajú.

Informovali sme už o~snahe založiť nový zbor. Informácia o~tom je samostatnou časťou spravodaja. Modlíme a modlime sa za to, aby táto snaha spôsobila radosť v~nebeskom kráľovstve. Tam je veľká radosť vtedy, keď hriešnik činí pokánie. Chceme, aby tí, ktorí majú povolanie robiť misiu, ju mohli robiť tak, aby sa nebeské kráľovstvo šírilo medzi tými, ktorí nepoznajú alebo hľadajú Pána Ježiša Krista. Tento spôsob misie zatiaľ nie je medzi nami „doma“. Nezahadzujme ho do koša. Modlime sa za to, aby Pán Ježiš požehnal prácu medzi tými, ktorí nie sú znovuzrodení alebo tými, ktorí nepoznajú Pána Ježiša. Modlime sa aj za tím pracovníkov. Pán Ježiš nás poslal do sveta. Buďme poslušní a choďme tam, kam nás On posiela.

\autor{za staršovstvo Peter Pribula}


\clanok{Kandidáti do staršovstva zboru}
Členovia zboru navrhli do staršovstva nasledujúcich bratov (menovaní prijali kandidatúru):
\vskip-1ex\begitems
* Peter Antalík
* Radovan Hovorka
* Vladimír Ira
* Pavel Kohút
* Peter Kolárovský
* Miroslav Kolářik
* Ján Szőllős
\enditems


\clanok{Kandidáti do revíznej komisie zboru}
Členovia zboru navrhli do revíznej komisie nasledujúcich bratov a sestry (menovaní prijali kandidatúru):
\vskip-1ex\begitems
* Miroslav Antalík
* Barbora Antalíková
* Slavomír Máťuš
* Helena Mikletičová
* Ladislav Taliga
* Katarína Valentová
\enditems


\clanok{Kandidát do diakonského tímu}
Tím diakonov by mal doplniť brat Slavomír Máťuš.


\clanok{List kazateľa a staršovstva zboru}
Milá rodina,

teším sa, že Pán Ježiš nás povoláva do toho, čo koná okolo nás Jeho Otec. On už dávno povedal: „Môj Otec pracuje doteraz, aj ja pracujem.“ (Ján 5, 17) Verím, že toto slovo platí aj dnes, lebo náš Boh stále koná. Verím tiež, že koná aj v~našom zbore a chystá nám dobrodružstvo služby. Veľmi sa na to spolu s~vami teším.

V nedeľu som na zborovej hodine oznámil, že časťou tohto dobrodružstva bude v~tomto roku zakladanie nového zboru. Hovoril som, že podobne ako sa rozrastá ľudská rodina a deti zakladajú svoje vlastné rodiny, tak isto je to prirodzené aj v~cirkvi. Je to normálny vývoj zdravého miestneho zboru. Je veľa dôvodov, prečo zakladanie nových zborov prináša ovocie všetkým.
Cirkev bratská v~Českej republike je dobrým príkladom toho, ako zakladanie zborov prináša ovocie. V~roku 1995 mala Cirkev bratská v~Českej republike 39 zborov. Devätnásť z~nich sa rozhodlo stať sa materskými zbormi, ktoré v~období 1995 – 2018 založili spolu 35 nových zborov. Počas tohto obdobia celkovo 54 zborov získalo 2272 nových členov, pričom prírastok v~dvadsiatich zboroch, ktoré sa nezapojili do zakladania zborov, bol len 238 členov. Z~toho vyplýva, že 90,5 percent novoobrátených ľudí a nárast členstva v~CB pochádza z~materských zborov, ktoré sa rozhodli založiť nové zbory.
To je sila! Počas jedenástich rokov služby v~Little Rocku náš baptistický zbor Summit založil 5 zborov. S~každým novým zborom sme zažili Božie požehnanie, či už obrátením a pokrstením nových ľudí, prijatím nových členov, či vznikom nových služieb zboru. Je to pre všetkých požehnanie, lebo je to biblické a Pán Boh to požehnáva. Postupne budeme o~tom hovoriť viac. V~nasledujúcich riadkoch sa dozviete, aká je aktuálna situácia v~súvislosti so~zakladaním nových zborov u~nás.

Tento rok začneme takýto zdravý proces pod vedením Tomáša Valchářa a tímu zakladateľov aj s~podporou staršovstva na Palisádach. Nový zbor vznikne oficiálne pod BJB  a jeho materským zborom bude BJB Palisády. Som veľmi vďačný Pánu Bohu za Tomáša a jeho túžbu podriadiť sa vo všetkom vedeniu zboru. Už niekoľko mesiacov spolu so mnou zdieľal toto povolanie, ale od začiatku jasne komunikoval, že chce, aby nový zbor založil palisádsky zbor a bol pod autoritou a vedením staršovstva a mňa – ako kazateľa. Veľmi si jeho postoj vážim. Spolu so svojím tímom nechcú založiť nový zbor rozdelením pôvodného a odlákaním členov z~Palisád do nového zboru, ale členovia nového zboru majú byť najmä noví ľudia zo sveta. Verím, že nám bude cťou, ak do tímu zakladateľov Pán Boh povolá ľudí aj z~Palisád. V~decembri bol Tomáš pozvaný na staršovstvo a vypočuli sme si, čo mu Pán kladie na srdce. Všetci starší mu žehnali a jednohlasne mu dali súhlas pokračovať. Znovu sme sa stretli v~januári. Rozhodli sme sa, že oznámime túto udalosť zboru. Chceme spoločne pod vedením Ducha Svätého pokračovať v~začatom diele.

V súčasnosti sa formuje zakladajúci tím. Zatiaľ sa párkrát stretli a spoločne sa modlili. Chceli by sa stretávať pravidelne niekoľkokrát mesačne v~domácnostiach. Tím pozostáva z~nasledujúcich párov:
\begitems
* Elise a Trey Atkins
* Miriam a Lester Peters
* Palina a Matúš Šutkovci
* Oľga a Tomáš Valchářovci
* Andrew a Kristine Hayes
\enditems

Na výročnom zborovom zhromaždení v~marci nám Tomáš  spolu s~tímom zakladateľov predloží svoju víziu a podrobne nás oboznámi so svojím plánom. Predbežne predpokladáme postupný rozvoj od diaspóry cez zborovú stanicu až po samostatný zbor. Zatiaľ sa nebudú konať verejné bohoslužby, ale občas v~priebehu mesiaca sa stretnú na modlitbách a prípravných stretnutiach. S~Tomášom sme sa dohodli, že sa bude pravidelne stretávať so staršovstvom a informovať nás o~pokrokoch. Pod vedením staršovstva sa rozhodneme, kedy a kde sa uskutočnia bohoslužby tohto formujúceho sa baptistického zboru.

Chcem Vám však úprimne povedať, že vstupujeme do duchovného boja. Musíme byť jednotní na modlitbách. Drvivá väčšina Bratislavčanov je duchovne stratená a bez Pána do večnosti nevstúpia. Satan rád rozdeľuje a môžeme s~tým rátať, že sa bude veľmi snažiť. Ale my vieme, že je len klamár a zlodej a na kríži bol úplne porazený. Nedovoľme mu nič! Chceme, aby sa všetko dialo v~úprimnosti a otvorenosti. Ak budete mať akékoľvek otázky, príďte za mnou, za Tomášom alebo za členmi staršovstva a pýtajte sa. Sme prichystaní vám odpovedať. Len treba, aby sme vedeli, čo vás zaujíma.

S Božou pomocou a milosťou prinesieme viac svetla do tohto tmavého sveta! Je to naše povolanie. Poďme smelo do toho!

Posilňujte sa milosťou v~Ježišovi Kristovi,

\autor{Danny Jones a vaše staršovstvo}
\vfill\break


\clanok{Oznam staršovstva zboru}
Staršovstvo oznamuje, že členovia zboru môžu prísť prezentovať svoje pripomienky a návrhy k~rozpočtu na stretnutia staršovstva, ktoré sa uskutočnia {\bf 12.~februára~2019 a~5.~marca~2019 o~18.00~hod.} v~kancelárii zboru na Zrínskeho~2.


\clanok{Spoločné modlitby}
\vskip-1ex\begitems
* Muži -- streda {\bf od 6.00~hod. do 7.00~hod.}, kostol na Palisádach
* Ženy -- pondelok {\bf od 17.00~hod.}, Zrínskeho 2
\enditems

Priveďte na spoločné modlitby aj vašich priateľov a známych, ktorým leží na srdci naše mesto a ľudia v~ňom.


\clanok{Verše na zapamätanie}
V nasledujúcich mesiacoch sa budeme spoločne učiť naspamäť niektoré pasáže z~Božieho slova. Prospeje to našej duši i našej mysli.

Verš na február: {\it „A všetko mu podrobil pod jeho nohy. Jeho však, ako hlavu nad všetkým, dal Cirkvi, ktorá je jeho telom, plnosťou toho, ktorý napĺňa všetko vo všetkom.“} Efezským~1,~22~--~23


\clanok{Zborové skupinky}
Milí priatelia,

od marca by sme radi rozbehli päť zborových skupiniek. Tri z~nich by boli zamerané na manželstvá. Ich vedúcimi by mali byť manželia Hovorkovci, Jonesovci a Kešjarovci. Ďalšie dve skupinky povedú manželia Antalíkovci (Miro a Štefka) a manželia Kohútovci (Pavel a Ľubka). Podrobnejšie informácie o~stretávaní skupiniek dostanete v~nasledujúcich týždňoch.


\clanok{Národný týždeň manželstva}
Milí priatelia,

v rámci Národného týždňa manželstva vás srdečne pozývame na stretnutie venované povzbudeniu a posilneniu manželstva. Uskutoční sa {\bf v~pondelok 11. februára 2019 o~18.00~hod.} v~priestoroch kresťanského spoločenstva Slovo života na Tomášikovej 30 v~Bratislave. Téma je Nekonečná láska. Hosťami stretnutia sú manželia Ewa a Robo Šipoš, ktorí nám rozpovedia svoj príbeh.

Na stretnutí sú vítané všetky manželské páry, ale aj tí, ktorých partneri sa nemôžu zúčastniť.

\break


\clanok{Národný týždeň manželstva v~Bernolákove}
\vskip-1ex\begitems
* Streda 13.~2.~2019 o~19.00~hod.: Ester Jankovičová, porozumenie v~manželstve
* Piatok 15.~2.~2019 o~19.00~hod.: Budovanie pevných základov
* Nedeľa 17.~2.~2019 o~17.00~hod.: Láska v~akcii
\enditems

Podujatia sa uskutočnia v~cirkevnom zbore BJB Bernolákovo.

Vstup voľný!


\clanok{Stretnutie sestier}
Milé sestry,

najbližšie stretnutie sa uskutoční {\bf 20.~februára o~17.00~hod.} na Palisádach. Budeme spoločne študovať knihu Milovaná (ako nájsť odpočinok), ktorej autorkou je sestra Clara Jones. Knihu si môžete zakúpiť u~s.~Heleny Mikletičovej. Stojí 10~€. Do stretnutia si treba naštudovať prvú lekciu po stranu č. 46.

Ženy všetkých vekových kategórií sú srdečne vítané!


\clanok{Výlet počas jarných prázdnin}
Pre všetkých záujemcov, ktorí by radi počas jarných prázdnin strávili spoločné chvíle na prechádzkach, pri lyžovaní, snowboardovaní, šantení na snehu, spoločných hrách, rozhovoroch, či zamysleniach, máme túto ponuku:

Poďme spolu v čase od 22.~2. do 1.~3.~2019 do nášho rekreačného zariadenia v Račkovej doline. Dĺžku pobytu si môžete prispôsobiť vašim finančným a časovým možnostiam. Pre viac informácií môžete kontaktovať Barbi Antalíkovú (\email{bantalikova@yahoo.com}, 0903~255~010) alebo Petra Antalíka (\email{antalikp@yahoo.com}, 0903~109~988). Tešíme sa na vás!


\clanok{Mládežnícka konferencia 2019}
\vskip-1ex\begitems
* Termín: 22. -- 24. február 2019
* Miesto: Gymnázium Andreja Sládkoviča (GAS) v~Banskej Bystrici
* Téma: Buď vzorom!
* Rečníci: \vskip-1ex\begitems
* Marek Macák
* Dodo Michalec
* Darina Malá
* Timotej Hanes
* Dawson Jones
* a iní
\enditems
\enditems


\clanok{Zbierky za január}
Milí bratia a sestry, ďakujeme za vašu obetavosť. V~mesiaci január ste prispeli:
\vskip-1ex\begitems
* misia: 588 €
* investičný fond: 552 €
\enditems


\clanok{Senior klub vo februári}
Ak dá Pán zdravia a života, v~mesiaci február sa opäť stretneme {\bf posledný štvrtok, t.~j.~dňa 28.~februára~2019 na Súľovskej ul. od 10.00~hod. do 14.00~hod.}
Téma stretnutia budú biblické veršíky na rok 2019, ktoré sme si vytiahli na Silvestra.

Všetci sú srdečne vítaní!

V láske Kristovej

\autor{Jana Makovíni}


\clanok{Služba ľuďom bez domova}
Zbierka šatstva:

Aktuálne stále zbierame zimné pánske šatstvo, najmä bundy, čiapky, rukavice, spodné prádlo a topánky. Šatstvo je možné priniesť na Ambroseho v~pondelky od~17.00~hod. do~19.00~hod. po telefonickom dohovore so Sylviou Vaniherovou (0905 484 675). Ďakujeme.

\autor {Lenka Antalíková}


\n 3.	2.	Vlasta	BALÁŽOVÁ;
\n 3.	2.	Margita	KRÁĽOVÁ;
\n 3.	2.	Miroslav	ANTALÍK;
\n 4.	2.	Veronika	VEČEREKOVÁ;
\n 5.	2.	Štefánia	ANTALÍKOVÁ;
\n 5.	2.	Barbora	ANTALÍKOVÁ;
\n 11.	2.	Juraj	BALÁŽ;
\n 11.	2.	Oľga	KOVÁČOVÁ;
\n 12.	2.	Martin	PRIBULA;
\n 13.	2.	Zlatica	VYSKOČILOVÁ;
\n 15.	2.	Ingrid	JANČULOVÁ;
\n 16.	2.	Lenka	PRIBULOVÁ;
\n 23.	2.	Anna	Plett;
\narodeniny


\program{
\p 1  ; pi ;.;;.;;
\p 2  ; so ; 18.00 ; Mládež (Súľovská 2);.;;
\p 3  ; ne ;  9.30 ; Bohoslužby (D. Jones) ; 10.00 ; Chvojnica (O. Škodák);
\p 4  ; po ; 17.00 ; Modlitby -- ženy (Zrínskeho 2);.;;
\p 5  ; ut ;.;;.;;
\p 6  ; st ;  6.00 ; Modlitby -- muži (kostol Palisády);.;;
\p 7  ; št ; 19.00 ; Biblická hodina (J. Szőllős, Zrínskeho 2);.;;
\p 8  ; pi ;.;;.;;
\p 9  ; so ; 18.00 ; Mládež (Súľovská 2);.;;
\p 10 ; ne ;  9.30 ; Bohoslužby (T. Valchář); 10.00 ; Chvojnica (J. Szőllős);
\p 11 ; po ; 17.00 ; Modlitby -- ženy (Zrínskeho 2);18.00;Manželský večer (Tomášikova 30, priestory Slova života);
\p 12 ; ut ; 15.15 ; Stretnutie pri Biblii (P. Pivka, Zrínskeho 2);.;;
\p 13 ; st ;  6.00 ; Modlitby -- muži (kostol Palisády);.;;
\p 14 ; št ; 19.00 ; Biblická hodina (J. Szőllős, Zrínskeho 2);.;;
\p 15 ; pi ;.;;.;;
\p 16 ; so ; 18.00 ; Mládež (Súľovská 2);.;;
\p 17 ; ne ;  9.30 ; Bohoslužby (Ľ. Dzuriak); 10.00 ; Chvojnica (J. Laurenčík);
\p 18 ; po ; 17.00 ; Modlitby -- ženy (Zrínskeho 2);.;;
\p 19 ; ut ; 15.15 ; Stretnutie pri Biblii (P. Pivka, Zrínskeho 2);.;;
\p 20 ; st ;  6.00 ; Modlitby -- muži (kostol Palisády) ; 17.00 ; Stretnutie sestier (kostol Palisády);
\p 21 ; št ; 19.00 ; Biblická hodina (J. Szőllős, Zrínskeho 2);.;;
\p 22 ; pi ;.;;.;;
\p 23 ; so ;.;Mládežnícka konferencia (Banská Bystrica);.;;
\p 24 ; ne ;  9.30 ; Bohoslužby (P. Pribula);.;;
\p 25 ; po ; 17.00 ; Modlitby -- ženy (Zrínskeho 2);.;;
\p 26 ; ut ; 15.15 ; Stretnutie pri Biblii (P. Pivka, Zrínskeho 2);.;;
\p 27 ; st ;  6.00 ; Modlitby -- muži (kostol Palisády);.;;
\p 28 ; št ; 19.00 ; Biblická hodina (J. Szőllős, Zrínskeho 2);.;;
}

\tiraz
\bye
