\def\velkostpisma{10}
\def\velkostriadku{12.5}
\input makra.tex % nacitanie Ivanom pripravenych nastaveni a prikazov
\hyphenation{star-šov-stvo} % rozdelenie slov na konci riadku, treba tu uviest slova, ktore sam nepozna

\spravodaj{2}{2025}


\clanok {Čo znamená hľadať Pána?}

Hľadať Pána znamená hľadať Jeho prítomnosť. „Prítomnosť“ je častým prekladom hebrejského slova „tvár“. Doslovne to znamená, že máme hľadať Jeho „tvár“. Je to však hebrejský spôsob, ako sa dostať k~Bohu. Byť pred Jeho tvárou znamená byť v~Jeho prítomnosti.

Nie sú ale Jeho deti v~Jeho prítomnosti stále? Áno aj nie. Áno dvoma spôsobmi: Po prvé, že Boh je všadeprítomný, a teda je stále blízko všetkého a každého. Všetko bytie je v~Ňom. Svojou všadeprítomnou mocou udržiava a riadi všetky veci.

Po druhé je stále prítomný pri svojich deťoch, lebo nám zasľúbil, že pri nás bude vždy stáť. Sľúbil, že bude pre nás pracovať a obráti všetko na naše dobro. „Ajhľa, ja som s~vami po všetky dni, až do konca sveta.“ (Mt 28,20)

\cast{Kedy s~nami nie je?}

Určitým spôsobom s~nami Božia prítomnosť nie je neustále. Z~toho dôvodu nás Biblia opakovane povoláva, aby sme „hľadali Pána... nepretržite hľadali Jeho prítomnosť.“ Božie zjavenie, vedomie, dôveryhodnú prítomnosť nezažívame neustále. Mávame obdobia, keď na Boha nedbáme, nemyslíme na Neho a neodovzdávame Mu našu dôveru. Zdá sa nám „nezjavený“. Inými slovami, nevidíme Ho naším srdcom ako veľkého, krásneho a cenného Boha.

Jeho tvár -- jas Jeho osobnosti -- je skrytá za závesom našich telesných túžob. Tie sú neustále pripravené na to, aby nás pohltili. Z~tohto dôvodu máme „hľadať Jeho prítomnosť nepretržite.“ Boh nás povoláva, aby sme si užívali poznanie Jeho najvyššej veľkosti, krásy a hodnoty.

\cast{Čo znamená hľadať?}

To sa deje „hľadaním“. Nepretržitým hľadaním. Čo to ale znamená v~praxi? Starý aj Nový zákon vraví, že je to „zameranie mysle a srdca“ na Boha. Znamená to nepretržite zameriavať a sústrediť na Boha pozornosť našej mysle a túžby nášho srdca.

„Zamerajte teda svoju myseľ i srdce na hľadanie Hospodina, svojho Boha!“ (1.~Kron 22,19)
(pozn. náš verš na február).

„Ak ste teda boli vzkriesení s~Kristom, hľadajte to, čo je hore, kde Kristus sedí na pravici Božej. Myslite na to, čo je hore, a nie na to, čo je na zemi.“ (Kol 3,1-2)

\cast{Vedomá voľba}

Toto zameranie mysle je opakom duševného plachtenia. Je to vedomé rozhodnutie, že nasmerujeme srdce na Boha. Za toto sa Pavol modlí pre cirkev: „Nech vám Pán vedie srdce k~Božej láske a ku Kristovej trpezlivosti.“ (2. Tes 3,5) Je to vedomá snaha z~našej strany. Snaha hľadať Boha je však zároveň darom od Boha.

„Hľadanie Boha je vedomá snaha dostať sa prirodzenými spôsobmi k~samotnému Bohu.“

Túto duševnú a emočnú snahu hľadať Boha nevynakladáme kvôli tomu, že je Boh stratený. Preto by sme hľadali mincu alebo ovcu. Boh však nie je stratený. Napriek tomu je tu niečo, cez čo alebo okolo čoho musíme prejsť, aby sme Ho vedome stretli. To znamená hľadať Boha. Je často skrytý. Zahalený. Musíme prejsť cez prekážky.

Nebesá hovoria o~Božej sláve. Môžeme Ho hľadať cez ne. On sa nám dáva poznať v~Písme. Môžeme Ho hľadať cez neho. Dáva sa nám poznať cez milosť iných ľudí. Môžeme Ho hľadať cez nich. Hľadanie je vedomá snaha dostať sa prirodzenými spôsobmi k~samotnému Bohu. Je to nepretržité zameranie mysle na Boha vo všetkých skúsenostiach, zameranie našej mysle a srdca na Neho cez rôzne spôsoby Jeho zjavovania. Toto znamená hľadať Boha.

\cast{Prekážky, ktoré musíme prekonať}

Aby sme Ho videli jasne, a teda mohli byť vo svetle Jeho prítomnosti, musíme prekonávať nekonečné množstvo prekážok. Musíme sa vyhnúť aktivitám, ktoré otupujú naše zmysly. Musíme pred nimi utekať, dostať sa cez ne. Stoja nám v~ceste.

Vieme, čo spôsobuje, že sme citliví na Božie zjavenia vo svete a v~Písme. A~vieme, čo nás otupuje a oslepuje a dokonca čo spôsobuje, že Ho ani nechceme vidieť. Týmto veciam sa musíme vyhýbať a dostať sa cez ne, aby sme Ho videli. Aj toto je súčasťou hľadania Boha.

„Tí, ktorí hľadajú Pána, majú veľké zasľúbenie, že Ho nájdu.“

Keď zameriame svoju myseľ a srdce na Pána vo všetkom, zvoláme na Neho. Aj to znamená hľadať Ho.

„Hľadajte Hospodina, kým sa dá nájsť, vzývajte Ho, kým je blízko.“ (Iz 55,6)

„Ale ak budeš hľadať Boha a Všemocného budeš prosiť o~milosť… “ (Job 8,5)

Hľadanie zahŕňa vzývanie a prosenie. „Ó, Pane, odtiahni záves mojej vlastnej slepoty. Pane, maj milosť a zjav sa. Túžim vidieť Tvoju tvár.“

\cast{Pokora ako základ}

Veľkou prekážkou pri hľadaní Pána je pýcha. „Bezbožník vo svojej namyslenosti Ho nehľadá.“ (Ž 10,4) Z~toho dôvodu je pokora základom hľadania Pána.

Tí, ktorí hľadajú Pána, majú veľké zasľúbenie, že Ho nájdu. „Ak Ho budeš hľadať, dá sa ti nájsť… “ (1. Kron 28,9) Ak Ho nájdeme, budeme odmenení. „Lebo ten, kto pristupuje k~Bohu, musí veriť, že Boh je a odplatí sa tým, ktorí Ho hľadajú“ (Žid 11,6). Našou najväčšou odmenou je samotný Boh. Ak Ho máme, máme všetko. Preto „hľadajme Pána a Jeho silu, hľadajme Jeho prítomnosť nepretržite!“

\autor {John Piper, prevzaté z \ulink[https://chcemviac.com/clanky/co-znamena-hladat-pana/]{chcemviac.com}}


\clanok {Správy zo staršovstva}
V~januári sme sa stretli dvakrát. Na prvom stretnutí sme urobili vyhodnotenie prieskumu pre staršovstvo a pre službu správcu zboru a následne sme oslovovali navrhnutých bratov. Novinkou je pozícia staršieho v~príprave. Inšpirovali sme sa Revúckym zborom, kde už tento model prebieha viac rokov a plní svoj účel.
Venovali sme sa konkrétnym návrhom do zborového rozpočtu na tento rok. Pripravovali sme zborové členské zhromaždenie, zaoberali sme sa podnetom od br.~A.~Erdélyiho ohľadom vydania knihy „Tí, ktorí mnohých priviedli k~pravde“.

V~období medzi stretnutiami staršovstva sme doladili návrh znenia zmeny do zborového poriadku, ktorý praktickejšie rieši voľbu správcu zboru (v~situácii, ak ním nie je kazateľ) a tiež dopĺňa pozíciu kazateľského asistenta (praktikanta) a jeho voľbu. Obe úpravy boli na ZČZ schválené.

Na druhom stretnutí staršovstva sme sa rozprávali s~bratom Filipom Barkóczim ohľadom prípadného zapojenia sa do zborovej práce v~pozícii kazateľského asistenta. S~br.~Petrom Žemberym sme prešli kandidátky do staršovstva (kandidujú len súčasní piati členovia staršovstva, čo je požadované minimum, ale tešíme sa, že dvaja bratia prijali kandidatúru na staršieho v~príprave) a tiež prebehla príprava na voľby do revíznej komisie. Taktiež sa už dlhšie rieši zabezpečenie ubytovania pre účastníčky sesterskej konferencii, ktorá sa bude konať v~máji v~Bratislave. Ďalšími témami bolo aj financovanie platu kazateľa, víkendovka (žiaľ tento polrok sa neuskutoční -- nenašiel sa vhodný termín, ktorý by vyhovoval nám aj Berei), zborový rodinný tábor a iné.

Aj touto cestou vás prosíme o~modlitby za vedenie a tiež budeme radi, aby ste nás v~prípade otázok oslovili, alebo ak máte návrhy na zlepšenie a rozvoj chodu zborového života, ozvite sa nám, prosím. Aby sa potenciál, ktorý do každého z~nás vkladá náš Boh, naplnil a nevyšiel nazmar.

\autor {za staršovstvo Ľubomír Syč}
\vfill\break


\clanok {Verš na mesiac}
„Zamerajte teda svoju myseľ i srdce, na hľadanie Hospodina, svojho Boha!“ (1.~Kron~22,19a)


\clanok {Národný týždeň manželstva}
Národný týždeň manželstva (NMT 2025) sa koná v~týždni od~10.~2. do~16.~2.~2025 a má názov: Beh na dlhú trať. Podrobnosti o~celej akcii nájdete na webe \ulink[https://www.ntm.sk]{www.ntm.sk}.


\clanok {Stretnutie sestier}
Sestry sa stretnú v~stredu 19.~2.~2025 o~17.30~hod. na Zrínskeho~2. Hosťom bude s.~Beatka Dobová, téma je: „Kufor plný pokladov: Čo si zbaliť na ceste do staroby?“


\clanok {Lyžovačka v~Račkovej doline}
Počas jarných prázdnin, v~termíne od~21.~2. do~28.~2.~2025, máme rezervovaný pobyt na chate v~Račkovej doline. Záujemcovia, hláste sa u~br.~P.~Antalíka: \email{antalikp@yahoo.com}.
\vfill\break


\clanok {Zbierky}
\vskip-1ex\begitems
* Zbierka 01/25 -- pre Ukrajinu cez Integru: 1774,00~€
* Zbierka 01/25 -- na investičný fond:        224,00~€
\enditems

Aj naďalej máte možnosť prispieť do „nedeľnej zbierky“, a to prevodom na účet zboru. Do poznámky pre prijímateľa, prosím, uveďte „zbierka“.

Bankové spojenie: SK36 0900 0000 0000 1147 1836, SWIFT: GIBASKBX


\n  3.	2.	Vlasta	BALÁŽOVÁ;
\n  3.	2.	Miroslav	ANTALÍK;
\n  5.	2.	Štefánia	ANTALÍKOVÁ;
\n  5.	2.	Barbora	ANTALÍKOVÁ;
\n 11.	2.	Oľga	KOVÁČOVÁ;
\n 11.	2.	Beáta	BOGÁROVÁ;
\n 12.	2.	Martin	PRIBULA;
\n 13.	2.	Zlatica	VYSKOČILOVÁ;
\n 15.	2.	Ingrid	JANČULOVÁ;
\n 15.	2.	Dávid VALCHÁŘ;
\n 23.	2.	Anna RUCIN (PLETT);

\narodeniny


\program{
\p  1 ; so ; 18.00 ; Mládež;.;;
\p  2 ; ne ;  9.30 ; Bohoslužby (J. Szőllős + VP);.;;
\p  3 ; po ;.;;.;;
\p  4 ; ut ; 14.00 ; Biblická hodina pre seniorov (P. Pivka);.;;
\p  5 ; st ;.;;.;;
\p  6 ; št ; 18.00 ; Biblická hodina (J. Szőllős);.;;
\p  7 ; pi ; 17.30 ; Dorast;.;;
\p  8 ; so ; 18.00 ; Mládež;.;;
\p  9 ; ne ;  9.30 ; Bohoslužby (F. Barkóczi);.;;
\p 10 ; po ; 18.00 ; Skupinka „Základy viery“;.;;
\p 11 ; ut ; 14.00 ; Biblická hodina pre seniorov (P. Pivka);.;;
\p 12 ; st ;.;;.;;
\p 13 ; št ; 18.00 ; Biblická hodina (J. Szőllős);.;;
\p 14 ; pi ; 17.30 ; Dorast;.;;
\p 15 ; so ; 18.00 ; Mládež;.;;
\p 16 ; ne ;  9.30 ; Bohoslužby (Ľ. Syč);.;;
\p 17 ; po ;.;;.;;
\p 18 ; ut ; 14.00 ; Biblická hodina pre seniorov (P. Pivka);.;;
\p 19 ; st ; 17.30 ; Stretnutie sestier (B. Dobová);.;;
\p 20 ; št ; 18.00 ; Biblická hodina (J. Szőllős);.;;
\p 21 ; pi ; 17.30 ; Dorast;.;;
\p 22 ; so ; 18.00 ; Mládež;.;;
\p 23 ; ne ;  9.30 ; Bohoslužby (V. Potockij);.;;
\p 24 ; po ; 18.00 ; Skupinka „Základy viery“;.;;
\p 25 ; ut ; 14.00 ; Biblická hodina pre seniorov (P. Pivka);.;;
\p 26 ; st ;.;;.;;
\p 27 ; št ; 18.00 ; Biblická hodina (J. Szőllős);.;;
\p 28 ; pi ; 17.30 ; Dorast;.;;
}

\tiraz
\bye
