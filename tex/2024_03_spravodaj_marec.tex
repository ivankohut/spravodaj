\def\velkostpisma{9}
\def\velkostriadku{12}
\input makra.tex % nacitanie Ivanom pripravenych nastaveni a prikazov
\hyphenation{star-šov-stvo} % rozdelenie slov na konci riadku, treba tu uviest slova, ktore sam nepozna

\spravodaj{3}{2024}

\def\sekcia#1{\vskip0.2em\noindent #1}

\clanok {Boh, ktorý ma vidí}

\sekcia{EL ROI}

Egyptská otrokyňa Agar, ktorá Boha stretla v~púšti, o~Ňom hovorila ako o~El Roi -- o~Bohu, ktorý ma vidí. Je zvláštne, že toto je jediné miesto v~Biblii, kde sa Boh označuje ako El Roi. Boh, ktorý sa zjavuje Agar, je Bohom, ktorý pozná počet vlasov na našej hlave, ktorý pozná všetky okolnosti, minulosť, prítomnosť aj budúcnosť. Keď sa modlíš k~Bohu, ktorého meno je El Roi, modlíš sa k~Tomu, ktorý vie o~tebe všetko.

\sekcia{KĽÚČOVÉ VERŠE}

„Potom nazvala Pána, ktorý k~nej hovoril: ‚Aha, ty si Boh, ktorý ma videl, lebo -- povedala -- či som azda nevidela Toho, ktorý sa na mňa díval?‘ Preto studňu nazvala Ber-Lachaj-roí (studňa Živého, ktorý ma vidí).“ (Gn 16,13-14)

\sekcia{ZAMYSLIME SA}

Čo ti zíde na um, keď počuješ meno El Roi -- Boh, ktorý ťa vidí? Zažil si už Božiu dôslednú starostlivosť?

BOH MA VIDÍ ako Pán a ako EL ROI hľadí z~neba a vidí všetkých ľudí. Pozerá a zasahuje z~miesta, kde sám prebýva, na všetkých obyvateľov zeme, On, čo každému osve utvoril srdce a chápe všetky ich skutky a pozná všetky ich úmysly. Hľa, oko Pánovo bdie nad tými, čo sa ho boja, nad tými, čo v~Jeho milosrdenstvo dúfajú, aby ich zachránil pred smrťou a v~čase hladu nakŕmil. (Ž~33,13-15;18-19)

Ako Boh skrze Zachariáša v~2,12 hovorí: „Lebo takto hovorí Hospodin zástupov… Kto sa vás dotkne, dotýka sa zrenice môjho oka.“

\vskip-1ex\begitems
* {\it Chváľ ho}: Pretože ťa miluje a túžobne pomáha aj tebe, keď si ubiedený alebo utláčaný.
* {\it Ďakuj mu}: Za to, akými rôznymi spôsobmi ťa ochraňuje, uvedomujúc si, že o~mnohých veciach, od ktorých nás Boh ochránil, ani nemáme vedomie.
* {\it Vyznaj}: Boh hľadí na teba a tvoje srdce, preto vyznaj akýkoľvek svoj egocentrizmus a nedostatok pozornosti voči Nemu.
* {\it Popros ho}: Aby si vnímal Božiu vôľu a aby aj tvoje srdce bolo zlomené pre rovnaké veci, pre ktoré je zlomené aj Jeho srdce.
\enditems
\hskip2cm LEBO ON JE BOH, KTORÝ MA VIDÍ -- EL ROI
\vskip1em

\sekcia{AKÉ MÁME PRISĽÚBENIA SPOJENÉ S~MENOM EL ROI?}

Božie Slovo je veľmi jasné v~tom, že Boh nikdy nespí, ani nedrieme. Ak my smerujeme jedným smerom, On sa nikdy nedíva iným. Neunikne mu ani milisekunda toho, čo sa deje na zemi, ani čo sa deje v~našom živote. Božie Slovo nás takisto uisťuje o~tom, že Boh je rozhľadňou pre ľudí, ktorí sú mu úplne oddaní, pretože keď mu slúžia, dodáva im silu. El Roi sa raduje, keď môže nad tebou bdieť.

\sekcia{PRISĽÚBENIA V~PÍSME}

„Lebo Pánove oči pozerajú po celej zemi, aby posilnili tých, ktorí sú mu zo srdca oddaní.“ (2~Krn 16,9)

„Nedovolí, aby sa ti noha zachvela, nedrieme ten, čo ťa stráži. Nedrieme veru, ani nespí ten, čo stráži Izrael. Pán ťa stráži. Pán je tvoja záštita po tvojej pravici. Za dňa ťa slnko nezraní, ani mesiac za noci. Pán ťa bude chrániť od všetkého zlého: Pán ti bude chrániť život. Pán bude chrániť tvoj odchod i príchod odteraz až naveky.“ (Ž~121,3-8)

Akoby ovplyvnilo tvoj život vedomie toho, že Boh na Teba neustále hľadí? No nie však ako Ten, ktorý ťa kontroluje, aby ťa potrestal, ale ako Ten, ktorý má o~teba záujem a záleží Mu na tebe. Nechajme sa ovplyvniť vedomím prítomnosti Toho, ktorý nás stráži dňom i nocou, aby sme mu mohli v~každom čase prinášať úctu a vzdávať chválu. Boh má záujem o~človeka komplexne a stará sa o~Jeho telo, dušu i ducha. Túto starostlivosť môžeme najviac zažívať a pociťovať práve vtedy, keď prehlbujeme osobný vzťah s~Tým, ktorý ma vidí.

{\it „Pane, chválim Ťa za to, že poznáš celý môj príbeh. Od začiatku do konca, Ty vieš všetko. Daj mi pokorné srdce, aby som uznal svoje limity, pretože často nevidím úplne správne minulosť, mám zahmlený pohľad na prítomnosť a pokiaľ ide o~budúcnosť, som takmer úplne slepý. Pomôž mi upriamiť svoj pohľad na to, že dbáš o~mňa, ako aj na Teba samého a zároveň dôverovať Tvojmu pohľadu na môj život.“} Amen.

\autor{inšpirované knihou Božie mená, Peter Šrankota}
%\vfill\break


\clanok {Správy zo staršovstva za február}

Staršovstvo zboru sa stretlo v~mesiaci február dva razy, a to v~utorky 6.~a 20.~2.~2024.
Ťažiskom stretnutí bola príprava výročného zborového členského zhromaždenia, ktoré je plánované na 17.~3.~2024, a otázka voľby ďalšieho kazateľa zboru. Príprava VZČZ zahŕňala oboznámenie sa s~hospodárením zboru za rok 2023.

Ďalej to bolo vypracovanie návrhu rozpočtu na rok 2024 v~spolupráci s~Ľ.~Kohútovou. Súčasťou toho bola aj diskusia so záujemcami z~radov členov zboru a zodpovedanie ich otázok k~predloženému návrhu. Taktiež to bola diskusia k~zamýšľanej rekonštrukcii fasády našej modlitebne. Konkrétne zosumarizovanie možností financovania a príprava konečného návrhu pre zbor.

V~otázke voľby ďalšieho kazateľa zboru staršovstvo zboru v~spolupráci s~volebnou komisiou oslovilo oboch navrhnutých kandidátov a to P. Kolárovského a J. Szőllősa. P. Kolárovský odmietol kandidovať za kazateľa zboru, naopak J. Szőllős je rozhodnutý kandidovať. V~tejto chvíli prebieha diskusia s~J. Szőllősom o~vzájomných predstavách a očakávaniach týkajúcich sa služby kazateľa.

Okrem týchto dvoch ťažiskových bodov sa staršovstvo zaoberalo aj požiadavkou F. Barkócziho. Uchádza sa o~pozíciu v~kaplanskom zbore v~Ozbrojených silách SR. Staršovstvo po rozhovore s~F. Barkóczim odporúča Rade BJB jeho kandidatúru do kaplanskej služby.

Staršovstvo sa taktiež začalo zaoberať požiadavkou Odboru sestier BJB na zorganizovanie konferencie sestier v~roku 2025, ktorá nám bola adresovaná. Nakoľko pôjde o~akciu, kde je predpokladaná účasť cca 300 účastníčok z~Čiech a Slovenska, nejde len o~akciu našich sestier, ale bremeno organizácie bude musieť prevziať celý zbor.

Okrem toho to bolo tiež riešenie otázky užívania priestorov na Súľovskej, ukončenie členstva niektorých členov nášho zboru z~dôvodu odsťahovania sa a iných otázok každodenného života zboru.
Ďakujeme za vašu podporu a modlitby.

\autor {za staršovstvo Peter Antalík}


\clanok{Verš na mesiac marec}

Na tento mesiac máme nový veršík, ktorý sa chceme spoločne učiť. Veríme, že poznanie Písma prospeje našej duši i našej mysli:

{\it „Neboj sa, veď som s~tebou, neobzeraj sa, veď Ja som tvoj Boh, posilním Ťa a určite ti pomôžem, veď ťa podopieram spravodlivou pravicou.“}

\autor{Izaiáš~41,~10}


\clanok{Zborová lyžovačka / zimný pobyt v~Račkovej}

Aj tento rok počas jarných prázdnin bratislavského kraja máme možnosť pobytu v~chate Račkova dolina. Ubytovanie je možné počas oboch víkendov -- pred týždňom prázdnin aj po ňom. Prihlásiť sa môžete aj keď sa plánujete prísť len na pár dní. Cenník ubytovania nájdete na stránke \ulink[http://www.rackova.sk/cennik/]{www.rackova.sk/cennik}, avšak máme dohodnuté zvýhodnené ceny. Prihlasovanie a akékoľvek otázky ohľadom pobytu u~br.~Petra Antalíka \email{antalikp@yahoo.com}.


\clanok{Veľkonočný program}

Veľkonočné obdobie otvorí veľkonočný koncert spevokolu, a to v~sobotu 23.~3.~2024 o~17.00~hod. na Palisádach.

Bohoslužby podvečer na Veľký piatok 29.~3.~2024 sa budú konať u~nás na Palisádach (pp. čas 17.00 hod.), slúžiť bude br.~J.~Szőllős.

Rovnako ako minulý rok, máme možnosť pripojiť sa k~spoločným nad-denominačným bohoslužbám, ktoré organizujú Kresťania v~meste v~UPeCe. Uskutočnia sa v~piatok 29.~3.~2024 o~10.00~hod., kázať bude br.~kazateľ Pavol Zsolnay zo zboru BCC.

Na veľkonočnú nedeľu 31.~marca nám od 9.30~hod. poslúži náš br.~kazateľ P.~Šrankota.


\n 10.	3.	Rada	BÁNOVÁ;
\n 12.	3.	Alžbeta	SMOLKOVÁ;
\n 17.	3.	Tamara	SYČOVÁ;
\n 20.	3.	Jana	MÁŤUŠOVÁ;
\n 21.	3.	Ladislav	TALIGA;
\n 25.	3.	Filip	KOVÁČ;
\n 26.	3.	Matej	KOLÁŘIK;
\n 27.	3.	Marta	MAJEROVÁ;
\n 28.	3.	Marta	BARKÓCI;
\n 29.	3.	Marcel	MAĎAR;
\n 29.	3.	Peter	PRIBULA ml.;
\n 30.	3.	Marta	GULDANOVÁ;
\n 31.	3.	Judit	KOBZOVÁ;
\narodeniny


\program{
\p  1 ; pi ; 17.30 ; Dorast ;.;;
\p  2 ; so ; 18.00 ; Mládež ;.;;
\p  3 ; ne ;  9.30 ; Bohoslužby (P. Šrankota + VP) ;.;;
\p  4 ; po ;.;;.;;
\p  5 ; ut ; 15.15 ; Biblická hodina pre seniorov (P. Pivka) ;.;;
\p  6 ; st ;.;;.;;
\p  7 ; št ; 18.00 ; Biblická hodina (J. Szőllős) ;.;;
\p  8 ; pi ;.;;.;;
\p  9 ; so ; 18.00 ; Mládež ;.;;
\p 10 ; ne ;  9.30 ; Bohoslužby (P. Šrankota ) ;.;;
\p 11 ; po ;.;;.;;
\p 12 ; ut ; 15.15 ; Biblická hodina pre seniorov (P. Pivka) ;.;;
\p 13 ; st ; 17.30 ; Stretnutie sestier ;.;;
\p 14 ; št ; 18.00 ; Biblická hodina (J. Szőllős) ;.;;
\p 15 ; pi ; 17.30 ; Dorast ;.;;
\p 16 ; so ; 18.00 ; Mládež ;.;;
\p 17 ; ne ;  9.30 ; Bohoslužby (J. Szőllős) ; . ; VZČZ ;
\p 18 ; po ;.;;.;;
\p 19 ; ut ; 15.15 ; Biblická hodina pre seniorov (P. Pivka) ;.;;
\p 20 ; st ;.;;.;;
\p 21 ; št ; 18.00 ; Biblická hodina (J. Szőllős) ;.;;
\p 22 ; pi ; 17.30 ; Dorast ;.;;
\p 23 ; so ; 17.00 ; Veľkonočný koncert spevokolu ;.;;
\p 24 ; ne ;  9.30 ; Bohoslužby (D. Cekov) ;.;;
\p 25 ; po ;.;;.;;
\p 26 ; ut ; 15.15 ; Biblická hodina pre seniorov (P. Pivka) ;.;;
\p 27 ; st ;.;;.;;
\p 28 ; št ; 18.00 ; Biblická hodina (J. Szőllős) ;.;;
\p 29 ; pi ; 17.00 ; Bohoslužby (J. Szőllős) ;.;;
\p 30 ; so ; 18.00 ; Mládež ;.;;
\p 31 ; ne ;  9.30 ; Bohoslužby (P. Šrankota) ;.;;
}


\tiraz
\bye
