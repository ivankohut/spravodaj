% DOKUMENTACIA:

% Prazdny riadok za textom znamena ukoncenie odstavca.
% Cierne obldzniky na konci riadku (v PDF) - to nechaj na mna (moze to o.i. znamenat, ze treba pridat nejake slovo do \hyphenation, lebo ho sam nevie rozdelit na konci riadku)

% Prikazy pre casti spravodaja:
% \spravodaj{<mesiac>}{<rok>}
% \clanok{<nazov clanku>}
% \autor{<autor clanku>}
% \n<den.mesiac.meno> - zadefinovanie oslavenca
% \narodeniny - vytvorenie tabulky s~narodeninami vsetkych zadefinovanych oslavencov
% \tiraz - ukoncenie spravodaja tirazou

% Styl fontu:
% \bf - bold, plati do konca aktualne skupiny, napr. ak mas {aaa \bf bbb} ccc, tak aaa bude normalne, bbb bude bold, ccc bude normalne
% \it - italic (pouzit rovnakym sposobom ako \bf)
% \bi - bold italic (pouzit rovnakym sposobom ako \bf)
% \rm - normalne (pouzit rovnakym sposobom ako \bf)

% Dalsie prikazy a znaky:
% \begitems - zoznam (odrazky), informacie najdes na stranke http://petr.olsak.net/ftp/olsak/opmac/opmac-u.pdf#toc%3A.5
% \ulink[<cielova adresa]{<zobrazena adresa>} - klikatelny odkaz na webstranku
% \email{<adresa>} - klikatelny odkaz na e-mailovu adresu
% ~ - nedelitelna medzera, napr. v~dome, 21.~6.~2018
% -- - pomlcka (dvakrát -)
% „ - zaciatocna uvodzovka
% “ - koncova uvodzovka
% \noindent - najblizsi odstavec nebude odsadeny
% \vskip<velkost> - vertikalna medzera, napr. \vskip3pt alebo \vskip-3ex (zaporna medzera, t.j. posun smerom hore)


\input makra.tex % nacitanie Ivanom pripravenych nastaveni a prikazov
\hyphenation{star-šov-stvo} % rozdelenie slov na konci riadku, treba tu uviest slova, ktore sam nepozna

\spravodaj{10}{2018}


\clanok{Vôňa najmilšieho zboru}

{\it Lebo sme Kristovou vôňou, príjemnou Bohu uprostred tých, čo získavajú spásu aj medzi tými, čo hynú. \hfill 2. Korintským 2, 15}

\vskip1ex Čím dlhšie som v~zbore na Palisádach, tým viac som povzbudený vecami, ktoré vidím a cítim a o~ktorých počujem. Každý týždeň vidím, ako vítate v~zhromaždení tých, ktorí prichádzajú prvýkrát do zboru.

Počul som tento týždeň svedectvo mladého muža. Je pomerne nový v~zbore a oceňuje, že našiel u~nás domov a rodinu, ktorú veľmi hľadal a potreboval. Každý týždeň počúvam od jednej z~našich drahých starších sestier, akú radosť má v~nedeľu, keď môže prísť do zhromaždenia a ako miluje náš zbor a jeho zborovú rodinu. Tieto veci ma veľmi povzbudzujú.

Za posledných pár týždňov som ochutnal dobrú polievku, ktorú varia niektorí z~vás, skôr ako bola odnesená bezdomovcom. Počul som tak isto, ako sa staráte o~ľudí v~núdzi a chorých. Či už návštevou v~nemocnici alebo doma, donáškou jedla tým, ktorí to potrebujú, alebo pravidelným dovozom menej mobilných bratov a sestier. Teším sa, že sme na ceste k~vytýčenému cieľu -- byť najmilším zborom v~Bratislave.

Musíme byť vždy a všade vnímaví na ľudí okolo nás. Ježiš v~tom bol majster. Neodmietal deti a privinul ich k~sebe, všimol si nenápadnú ženu, ktorá 12 rokov tajne trpela. Ježiš sa dokázal zastaviť a s~láskou sa ľudí dotýkal. Ľudia okolo nás hľadajú lásku. Preto prichádzajú aj k~nám do zboru. Tí, ktorí nemôžu prísť pre chorobu alebo vek, tiež hľadajú lásku. Musíme im ju priniesť. Ide to nám. Buďme v~tom vytrvalí a dôslední aj naďalej.

\autor{Danny Jones}


\clanok{Biblické vzdelávanie}
Milí bratia a sestry, drahí priatelia, pozývame vás na stretnutia pri Biblii. V~utorok o~15.15~hod. na Zrínskeho 2 sa môžete zamýšľať nad Božím slovom spolu s~bratom kazateľom P.~Pivkom. Vo štvrtok o~19.00~hod. vás na Zrínskeho 2 pozýva brat kazateľ J.~Szőllős. Spolu s~ním budete študovať starozákonnú knihu Ezdráša.


\clanok{Nezabúdajme na spoločné modlitby}
\vskip-1ex\begitems
* Muži -- streda od 6.00 hod. do 7.00 hod., kostol na Palisádach
* Muži -- štvrtok od 17.30 hod. do 18.30 hod., Zrínskeho 2
* Ženy -- pondelok od 18.00 hod., Zrínskeho 2
* Verejné modlitby za mesto a za Slovensko -- streda od 12.00 hod., kostol na Palisádach
\enditems

Priveďte na spoločné modlitby aj vašich priateľov a známych, ktorým leží na srdci naše mesto a ľudia v~ňom.


\clanok{Skupinky}
Proces formovania zborových skupiniek stále prebieha. Musíme sa stretávať aj cez týždeň, nielen v~nedeľu. Vo svete plnom nepokoja a stresu potrebujeme spoločenstvo a povzbudenie. Naším cieľom je, aby v~každej mestskej časti vznikla skupinka. Ak už nejakú skupinku vedieš alebo by si bol ochotný/ochotná viesť, kontaktuj, prosím, brata kazateľa Dannyho Jonesa. Je dobré, keď máme prehľad o~tom, čo funguje a čo je potrebné zorganizovať.


\clanok{Stretnutie sestier}
Najbližšie stretnutie sestier sa uskutoční {\bf 17. októbra o~17.00 hod. na Palisádach}. Budeme pokračovať v~téme: Identita a úloha ženy v~zborovom živote. Ženy všetkých vekových kategórií sú srdečne vítané!


\clanok{Krst v~našom zbore}
Ohlasovaný krst ponorením na vyznanie viery sa uskutoční {\bf 28. októbra 2018}. Ak máte záujem verejne vyznať svoju vieru v~Pána Ježiša, čo najskôr kontaktujte kazateľa zboru Dannyho Jonesa. Krst sa uskutoční o~15.00 hod. v~modlitebni BJB v~Miloslavove.


\clanok{Konferencie pre pracovníkov s~deťmi a mládežou}
\vskip-1ex\begitems
* Konferencia pre kresťanských učiteľov 12. -- 14. októbra 2018 -- Ranč v~Kráľovej Lehote, viac info na \ulink[https://www.konferencie.mpks.sk/kku]{www.konferencie.mpks.sk/kku}
* Konferencia Viva Network 19. -- 21. októbra 2018 -- Hotel Biela Medvedica v~Bystrej, viac info na \ulink[http://www.vivanet.sk]{www.vivanet.sk}
* Konferencia Detskej misie 26. -- 28. októbra 2018 -- Banská Bystrica, viac info na \ulink[https://www.detskamisia.sk/event/301/konferencia-2018]{www.detskamisia.sk/event/301/konferencia-2018}
\enditems


\clanok{Služba JASu v~novembri}
{\bf 24. -- 25.~novembra 2018} plánujeme službu speváckeho zboru JAS v~našom kostole. V~sobotu o~17.00 hod. sa uskutoční koncert, ktorý bude spojený so svedectvami a sprievodným evanjelizačným slovom. V~nedeľu 25.~novembra dopoludnia poslúži brat kazateľ Timotej Hanes. Využite túto mimoriadnu príležitosť a pozvite svojich známych, susedov i príbuzných. Ak ich nezavoláte, neprídu! Ak ich pozvete, Pán Boh môže urobiť veľké veci. Preto neváhajte a povedzte im o~koncerte čo najskôr.


\clanok{Vstup do Národnej rady SR}
Milované sestry a bratia!

{\bf 23.~októbra~2018} od 17.30~hod. do 18.30~hod. bude Aglow-tím viesť modlitby a chvály v~Národnej rade SR. Chvály bude viesť Marina Wiesnerová.
Predtým, od 16.00~hod. do 17.00~hod. nám Marek Krajčí vybavil exkurziu a vstup do rokovacej sály počas zasadania.
Je tam vynikajúca sprievodkyňa a je zaujímavé počuť históriu a výklad o~našej krajine. Môžete pozvať ďalších.
Do 15.~októbra~2018 potrebujem menoslov a číslo OP.

\autor{Rút Krajčíová, \email{rutkrajci@gmail.com}, 0910 943 994}


\clanok{Zbierky za september}
Milí bratia a sestry, ďakujeme za vašu obetavosť. V~mesiaci september ste prispeli:
\vskip-1ex\begitems
* misia: 474 €
* investičný fond: 797 €
\enditems


\clanok{Služba ľuďom bez domova}
Milí bratia a sestry,

do pozornosti vám dávam zatiaľ neobsadené termíny pre náš zbor na varenie polievok ľuďom v~núdzi na najbližšie mesiace. Polievku je v~sobotu potrebné mať uvarenú už o~16.30~hod.

{\bf 17. november; 15. december}

Nahláste sa, prosím, u~mňa. Ďakujem za ochotu poslúžiť.

\autor{Beata Bogárová}


\clanok{Senior klub}
V mesiaci október, ak dá Pán zdravia a života, by sme sa radi stretli v~našom Senior klube ako obvykle
posledný štvrtok v~mesiaci t.~j. {\bf 25.~októbra~2018 o~10.00~hod. na Súľovskej ul}.

Vzhľadom k~tomu, že sa blíži Pamiatka zosnulých, radi by sme si pripomenuli našich vodcov, príp. tých ľudí, ktorí v~našom živote zanechali hlbokú brázdu a ovplyvnili náš duchovný život.
Prosíme, keby ste si pripravili krátke spomienky. Všetci sú srdečne vítaní.

Na spoločné stretnutie sa teší

\autor{Jana Makovíni}


\n 2.	10.	Peter	ANTALÍK;
\n 6.	10.	Daniel	BALÁŽ;
\n 6.	10.	Michal	KAJAN;
\n 12.	10.	Barbora	PRIBULOVÁ;
\n 14.	10.	Martin	SIMON;
\n 20.	10.	Ida	PUČEKOVÁ;
\n 22.	10.	Hana	HALAMIČKOVÁ;
\n 25.	10.	Vladimír	IRA;
\n 27.	10.	Miriam	KRÁĽOVÁ;
\n 28.	10.	František	VRABČEK;
\n 28.	10.	Ľubomír	SYČ;
\n 31.	10.	Samuel	KORIŤÁK;
\narodeniny

\program{
\p 1  ; po ; 18.00 ; Modlitby -- ženy (Zrínskeho 2);.;;
\p 2  ; ut ; 15.15 ; Popoludnie pri Biblii (P. Pivka, Zrínskeho 2);.;;
\p 3  ; st ; 6.00 ; Modlitby -- muži (kostol Palisády);12.00;Modlitby -- verejné (kostol Palisády);
\p 4  ; št ; 17.30 ; Modlitby -- muži (Zrínskeho 2);.;;
\p 5  ; pi ;.;;.;;
\p 6  ; so ; 18.00 ; Mládež (Súľovská 2);.;;
\p 7  ; ne ; 9.30 ; Bohoslužby (D. Jones); 10.00; Chvojnica (V. Ira);
\p 8  ; po ; 18.00 ; Modlitby -- ženy (Zrínskeho 2);.;;
\p 9  ; ut ; 15.15 ; Popoludnie pri Biblii (P. Pivka, Zrínskeho 2);.;;
\p 10  ; st ; 6.00 ; Modlitby -- muži (kostol Palisády);12.00;Modlitby -- verejné (kostol Palisády);
\p 11  ; št ; 17.30 ; Modlitby -- muži (Zrínskeho 2); 19.00; Biblická hodina (J. Szőllős, Zrínskeho 2);
\p 12  ; pi ;.;;.;;
\p 13  ; so ; 18.00 ; Mládež (Súľovská 2);.;;
\p 14  ; ne ; 9.30 ; Bohoslužby (D. Jones); 10.00; Chvojnica (P. Škulec);
\p 15  ; po ; 18.00 ; Modlitby -- ženy (Zrínskeho 2);.;;
\p 16  ; ut ; 15.15 ; Popoludnie pri Biblii (P. Pivka, Zrínskeho 2);.;;
\p 17  ; st ; 6.00 ; Modlitby -- muži (kostol Palisády);12.00;Modlitby -- verejné (kostol Palisády);
\p 18  ; št ; 17.30 ; Modlitby -- muži (Zrínskeho 2);.;;
\p 19  ; pi ;.;;.;;
\p 20  ; so ; 18.00 ; Mládež (Súľovská 2);.;;
\p 21  ; ne ; 9.30 ; Bohoslužby (T. Valchář);.;;
\p 22  ; po ; 18.00 ; Modlitby -- ženy (Zrínskeho 2);.;;
\p 23  ; ut ; 15.15 ; Popoludnie pri Biblii (P. Pivka, Zrínskeho 2);.;;
\p 24  ; st ; 6.00 ; Modlitby -- muži (kostol Palisády);12.00;Modlitby -- verejné (kostol Palisády);
\p 25  ; št ; 17.30 ; Modlitby -- muži (Zrínskeho 2);.;;
\p 26  ; pi ;.;;.;;
\p 27  ; so ; 18.00 ; Mládež (Súľovská 2);.;;
\p 28  ; ne ; 9.30 ; Bohoslužby (Ľ. Dzuriak); 10.00; Chvojnica (J. Štefko);
\p 29  ; po ; 18.00 ; Modlitby -- ženy (Zrínskeho 2);.;;
\p 30  ; ut ; 15.15 ; Popoludnie pri Biblii (P. Pivka, Zrínskeho 2);.;;
\p 31  ; st ; 6.00 ; Modlitby -- muži (kostol Palisády);12.00;Modlitby -- verejné (kostol Palisády);
}

\tiraz
\bye
