%\typosize[9/12]% - pouzita velkost pisma/riadku
\input makra.tex % nacitanie Ivanom pripravenych nastaveni a prikazov
\hyphenation{star-šov-stvo} % rozdelenie slov na konci riadku, treba tu uviest slova, ktore sam nepozna

\vyrocnespravy{2020}

\clanok{Zbor}
Písať správu a vyhodnotiť rok 2020 je veľká výzva. Ešte stále veľa z~tohto roku spracovávam. Verím, že to bol rok pravdy a odkrývania reality, či už v~rámci života jednotlivcov, rodín alebo zboru. Bol to rok zastavenia, keď sme spomalili, zastavili sa a rozmýšľali, skúmali a hodnotili. Stálo to za to. Súženia nám veľa ukážu a verím, že do roku 2021 vstúpime vnímavejší, čo sa týka nášho osobného a zborového stavu. Napriek všetkému, čo sa vo svete aj okolo nás udialo, zažili sme vzácny Boží dotyk.

Do príchodu covidu a potom aj trochu cez leto sme zažili do nejakej miery zborový život. Väčšinou sme sa stretávali, hoci v~obmedzenom počte. Besiedky prebiehali zriedkavo, keďže väčšinou boli zrušené. Ak zborové skupinky fungovali, bolo to cez Webex alebo Zoom. Mužské modlitby boli aj na Palisádach, ale najviac cez Zoom. Avšak kvôli zmenám v~domácnostiach, domácom vyučovaní detí a home office, bolo pre väčšinu mužov ťažké sa zapojiť.  Ženy pod vedením Clary Jones preberali ďalšie štúdium pod názvom {\it Trvalá sloboda}. Po vypuknutí covidu a prijatí opatrení ženy doštudovali tento kurz samoštúdiom.  Clara im posielala materiály a pomôcky k~štúdiu.

Po dlhých rokoch života a vernej služby sa zbor Palisády 21.~1.~2020 rozlúčil na smútočnom zhromaždení s~bratom Vojtechom (Bélom) Paulenom. Brat Paulen viedol ako dirigent „Veľký“ zborový spevokol viac ako 6 desaťročí. Spevokol pod jeho vedením slúžil každú nedeľu pri bohoslužbách v~zbore.  Slúžil aj takmer vo všetkých zboroch bývalého Československa. Mimo územia našej republiky spieval o~Božej láske v~Maďarsku, Rakúsku a Srbsku. Pod jeho vedením spevokol naštudoval aj náročné skladby slávnych svetových skladateľov -- Bacha, Haydna, Händla, Mozarta, Beethovena až po menej známych skladateľov.

Skupina dvadsiatich ľudí z~nášho zboru sa vo februári zúčastnila na poznávacom zájazde v~Izraeli. K~našej skupine sa pripojilo ešte 25 ďalších ľudí z~celého Slovenska. Izrael sme spoznávali pod vedením a učením Eli bar Davida. Zažili sme spolu požehnanie, radostné spoločenstvo, spev, modlitby a veľa spoznávania Písma. Celý čas sme sledovali nárast covidu. Domov sme sa stihli vrátiť deň pred zatvorením izraelskej hranice a zavedením opatrení na Slovensku.  Veríme, že sa oplatí znovu zopakovať tú cestu, možno v~roku 2022.

Ešte pred príchodom covidu sme 2.~februára stihli poslednú zborovú hodinu. Počuli sme svedectvá dvanástich bratov a sestier, ktorí vstúpili do zboru: Eva Rudy Dorová, Marta Brnová, Dara Plett, Samuel Plett, Tamara Syčová, Zdenko a Zuzka Filipovci, Ľudmila Drietomská, Rado Nemec, Martin Hovorka, Janko a Ľubka Kováčikovci. Šiesti z~nich boli pokrstení v~roku 2019, traja sa začlenili prestúpením z~iných zborov a dvaja znovuprijatím. Veľmi sa tešíme, že sú súčasťou našej zborovej rodiny. Väčšina nich už dlhšie slúži a je zapojená do zborových skupín.

Počas roka sme sa tešili z~ďalších prírastkov novorodencov v~ piatich rodinách. Privítali sme Árona Pribulu (Silvia a Maco), Lukáša Laurenčíka (Hanka a Tomáš), Dalilu Pelíškovú (Dana a Martin), Jonatána Kráľa (Miriam a Dárius), Jaroslava Jakova Bána (Radka a Jaro) a Kristánu Pototskyi (Liliia a Viktor). Blahoželáme vám!

Na diaľku, prostredníctvom  YouTubu, sme oslavovali štyri sobáše. Kvôli obmedzeniam sme museli všetko prispôsobiť a väčšina ľudí bola „prítomná“ cez živý prenos. Manželský sľub zložili Ľubka (Kráľová) a Janko Kováčikovci, Marta (Pribulová) a Tomáš Barkóczi, Betka (Brnová) a Roman Smolkovci, a Eva (Brokešová) a Tomáš Mizera. Eva a Tomáš začali k~nám do zboru chodiť niekoľko mesiacov pre príchodom covidu. Bola to pre mňa veľká radosť bližšie ich spoznávať počas príprav na manželstvo. Teším sa spolu s~nimi na to, čo má Pán pre nich v~rámci týchto nových manželstiev prichystané. V~súčasnosti bývajú Kováčikovci a Mizerovci v~Bratislave, Barkóciovci v~Dánsku a Smolkovci v~Zázrivej.

Počas prvej vlny covidu sme sa veľa učili a prišli na to, aké je nám spoločenstvo vzácne. Celý svet sa postupne zmenil, vrátane nás. Väčšina pracovala z~domu, deti sa učili doma a v~modlitbách a s~nádejou na milosť sme sledovali nárast nakazených na Slovensku. Naši technici sa rýchlo zmobilizovali a začali sme zhromaždenia organizovať iba cez živý prenos. Snažili sme zostať spojení cez Webex, ale pre väčšinu z~nás to bolo náročné a málo ľudí v~tom dlho vydržalo.  Veľa nových vecí sme nepochopili, a preto bola prvá vlna vlnou obáv a strachu. Keď sa to trochu uvoľnilo, organizovali sme dvoje zhromaždení, aby sa staršia generácia mohla bezpečne schádzať. Boli sme vďační za tú možnosť, ale bolo nám aj smutno, lebo nám chýbalo spoločenstvo, podávanie rúk a rozhovory, na čo sme boli pri káve a koláčoch zvyknutí.

Chcem sa veľmi poďakovať všetkým, ktorí až dodnes robia ešte stále viac a viac, aby mohli prebiehať  online bohoslužby. Bez nich by sa nič nedalo robiť: Jaro Bán, Ľuboš Kešjar, Mišo Kešjar, Ivan Kohút, Maco Pribula, Kika Kešjarová, Mirka Kešjarová a Slávo Kráľ. Táto skupina služobníkov je veľmi vzácna. Som vďačný aj za tých, ktorí viedli chvály, či zo zboru alebo z~iných zborov. Bolo ich veľa, aj za tú službu sme vďační.

Celý rok sa staršovstvo pravidelne stretávalo, či osobne na Zrínskeho alebo cez internet (väčšinou cez Zoom). Pán Boh stále medzi týmito bratmi zachováva jednotu, čo v~časoch súženia nie je samozrejmé. Jedným srdcom a zámerom sme navigovali cez veľa ťažkých mesiacov a komplikované situácie. Naše staršovstvo je pre zbor darom od Boha.

Ešte na konci školského roka sme zorganizovali dve víkendovky pre mužov s~ich synmi, počas ktorých sme slúžili a zažili spoločenstvo. Pán majster Chvojnice, Daniel Mikletič, zabezpečil všetko možné, aby sme dobre pracovali, a šéfkuchár Lacko Taliga zabezpečil, aby sme sa bohato najedli, a brat Danny, aby sme sa zamýšľali nad Božím Slovom. Počas oboch víkendov nás bolo 25 mužov a chlapcov.

Tešili sme sa, že príchod leta priniesol aj viac slobody. Zrealizoval sa trojdňový dorastenecký denný tábor. Počas tých dní skupina 25 dorastencov s~vedúcimi (a jedným starším kazateľom) bicyklovali do Čunova (prvý deň), hľadali poklad vo Svätom Juri v~ruinách hradu Biely Kameň (druhý deň), a kvôli dažďu si oddýchli pri spoločenských hrách na Zrínskeho (tretí deň).

Z Božej milosti sa zrealizoval zborový tábor. Mali sme plný počet ľudí (44 dospelých a 55 detí). Téma tábora bola Obyčajní hrdinovia: štúdium knihy Sudcov. Boh nás milostivo ochránil od covidu a celý týždeň sme bez problémov zažili Božie požehnanie. Prišla kontrola z~hygieny, ale keď pani zistila, že je to rodinný tábor, potešila sa a s~požehnaním nás nechala pokračovať.

V októbri sme uskutočnili pracovný projekt v~okolí nášho kostola. Zapojilo sa okolo 30 dobrovoľníkov, ktorí čistili chodníky a záhradky okolo Strednej priemyselnej školy elektrotechnickej a Domova sociálnych služieb ECAV. Pri Základnej škole Milana Hodžu sme dokončili náter plotu a brány, čo sme začali pred rokom. Všetci riaditelia nám vyjadrili veľkú vďaku.

Žiaľ, počas jesene prišla druhá vlna covidu a situácia sa postupne zhoršila.  Zborové členské zhromaždenie sme presunuli na neurčito. Máme ešte niekoľko ľudí, ktorí čakajú na vstup do zboru. Takisto sme na neurčito preložili krst v~zbore. S~veľkou ľútosťou sme museli zrušiť všetky vianočné zhromaždenia a vianočný koncert, Silvester a zhromaždenie na Nový rok. Oslavné silvestrovské zhromaždenie plánujeme uskutočniť až vtedy, keď sa uvoľnia opatrenia.

\autor{Danny Jones, kazateľ zboru}


\clanok{Staršovstvo}
Staršovstvo dostalo pre rok 2020 slovo zo Žalmu 25, 8 -- 9: {\it „Hospodin je dobrý a úprimný, preto učí hriešnikov svojej ceste. Vedie pokorných podľa práva, učí pokorných svojej ceste.“}

V roku 2020 sme zažívali, že náš nebeský Otec je naozaj dobrý. Napriek zložitej situácii, v~ktorej žijeme, nás neopustil. Práve naopak, zažívame Jeho láskavé vedenie a vyučovanie. V~minulom roku nás učil, ako inými spôsobmi môžeme zažívať spoločenstvo. Je to niečo, s~čím sme sa doteraz nemuseli zaoberať. Mali sme svoje dobre vyšliapané chodníčky a nič nám na nich nechýbalo. Ale v~minulom roku sme museli začať uvažovať ako iným, novým spôsobom mať a budovať spoločenstvo. Sme Mu vďační za Jeho vedenie. Nik iný by nás neviedol lepšie ako On.

V roku 2020 sme pracovali v~zložení kazateľ zboru Danny Jones a členovia Peter Antalík, Vladimír Ira, Peter Kolárovský, Miroslav Kolářik, Daniel Plett, Peter Pribula a Ján Szőllős.

Pre pandémiu sme mali počas veľkej časti roku obmedzené možnosti stretávania. Viditeľne sa to  prejavilo pri organizovaní bohoslužieb, biblických vyučovaní, stretnutí zborových zložiek, služieb organizovaných mimo nášho zboru, stretnutí skupiniek, stretnutí zborových členských zhromaždení a schvaľovania rozhodnutí nutných pre náš spoločný zborový život na tejto zemi. Tak ako civilný aj zborový život bol a zatiaľ aj je poznačený tým, že nevieme, čo bude zajtra. Učíme sa tak, ako to je napísané v~epištole Jakuba: {\it „Ak bude Pán chcieť, budeme žiť a vykonáme to alebo ono!“}

Stretávali sme sa v~pravidelných dvojtýždňových intervaloch. Niekedy to bolo osobne, inokedy zas online prostredníctvom webových služieb. Na mnohých našich stretnutiach sme mali aj pozvaných hostí. Dôvodom bol rozhovor so záujemcami o~členstvo v~zbore, príprava výročného zborového členského zhromaždenia, ale aj rozhovory pri pastoračných otázkach.

Porozumeli sme tomu, že v~aktuálnej situácii musíme hľadať spôsoby ako sa priblížiť k~ľuďom. Preto sme zmenili štruktúru vysielaných bohoslužieb prostredníctvom webu, zaviedli online stretnutia cez Zoom alebo podobné programy, kompletne sme prepracovali internetovú stránku zboru, intenzívnejšie sme začali využívať komunikáciu prostredníctvom sociálnych sietí. V~staršovstve nemáme kapacitu na všetky tieto veci, preto chcem aj touto cestou poďakovať všetkým dobrovoľníkom, ktorí sa podieľali a podieľajú na službe nám všetkým týmto spôsobom.

Príprava bohoslužieb, či už plán služieb alebo priamo jednotlivé bohoslužby, vyžadujú oveľa viac flexibility, osobného nasadenia a ochoty, ako tomu bolo v~minulosti. Aj za toto nasadenie chceme vyjadriť vďaku vám, ktorí sa podieľate na tejto službe.

Niekedy mám pocit, akoby sme zažívali slovo, ktoré povedal Pán Ježiš Petrovi:{\it „… satan si vás vyžiadal, aby vás preosial ako pšenicu. Ale ja som za teba (za vás) prosil, aby tvoja (vaša)  viera neochabla.“} Preto som presvedčený, že viac ako inokedy potrebujeme plniť výzvu: {\it„Jedni druhých bremená neste a tak naplňte zákon Kristov.“}

Nesieme vás na svojich modlitbách a prosíme vás, aby ste aj vy niesli nás a našu prácu na svojich modlitbách.

Vďaka nášmu Spasiteľovi a Pánovi Ježišovi Kristovi za Jeho požehnanie, podporu, vedenie, vyučovanie, múdrosť a dobrotu, ktoré nám daroval v~uplynulom roku 2020.

\autor{Peter Pribula}


\clanok{Diakonia}
Verš na rok 2020: {\it „Boh nám nedal ducha bojazlivosti, ale Ducha sily, lásky a rozvahy.“}

\autor{2Tim 1,7}

\cast{Pracovné stretnutia}
Rok 2020 bol pre nás všetkých zvláštnym rokom, rokom celosvetovej pandémie. To ovplyvnilo zborový život a samozrejme aj službu nášho tímu diakonov. Vidno to aj z~toho, že sme sa počas celého roku pracovne mohli fyzicky stretnúť len dvakrát (20.~1. a 8.~6.). Ďalšie naplánované stretnutia sa už nemohli uskutočniť z~vyššie uvedených dôvodov (zápisnice z~pracovných stretnutí boli pravidelne zaslané všetkým členom e-mailom, prípadne osobne odovzdané). Pozvánky na pracovné stretnutia pre členov tímu diakonov boli zasielané e-mailom spravidla tri dni vopred.

\def\aktivita#1{{\it #1\par}\firstnoindent}
\cast{I. Vnútrozborové aktivity}

\begitems \style n
* \aktivita{Návštevná služba}
Pravidelnú návštevnú službu našich imobilných členov v~domácnostiach vykonávali sestry a bratia, ktorí majú s~navštevovanými členmi zboru vytvorený prirodzený, blízky vzťah. Samozrejme, v~minulom roku aj tieto návštevy boli obmedzené a prispôsobené daným podmienkam. Tiež bola vyslúžená aj pamiatka Večere Pánovej v~domácnostiach na požiadanie bratmi zo skupiny diakonov. Okrem našich seniorov boli navštevovaní aj naši chorí bratia a sestry, či už v~domácnostiach alebo v~nemocniciach.

* \aktivita{Dopravná služba}
Okrem návštevnej služby pokračovala aj dopravná služba pre málo mobilných bratov a sestry do nedeľného zhromaždenia. V~tejto službe dlhodobo verne slúžia bratia Ladislav Taliga a Vladimír Krajčí.

* \aktivita{Zborové obedy}
V minulom roku sa nekonali.

* \aktivita{Svätodušné sviatky na Chvojnici}
Tohto roku bolo na Chvojnici na Letnice miesto tradičného programu nášho spevokolu len menšie zhromaždenie v~kostolíku, kde kázal brat kazateľ Danny Jones. V~okolí „našej chalúpky“ sa konala menšia brigáda, pod vedením brata Daniela Mikletiča.

* \aktivita{Vysluhovanie Večere Pánovej}
Večera Pánova sa vysluhovala pravidelne každú prvú nedeľu v~mesiaci (podľa rozpisu). Okrem toho sa Večera Pánova vysluhovala na požiadanie aj v~domácnostiach.
\enditems

\cast{II. Aktivity zboru smerom von}

\begitems \style n
* \aktivita{Služba v~domovoch sociálnej starostlivosti}
Okrem služby v~našom zbore sa venujeme aj službe mimo zboru v~domovoch dôchodcov pod vedením brata P.~Pivku za vernej pomoci sestier Lenky Gubovej a Vladky Laurenčíkovej. V~prípade potreby brata Pivku zastúpil „v Betánii“ brat kazateľ D.~Jones. Pravidelne navštevujeme „Domovy sociálnej starostlivosti“ v~Starom Meste a v~Dúbravke.

* \aktivita{Biblické vyučovanie}
V rámci služby seniorom máme aj duchovnú časť služby v~podobe pravidelných biblických hodín s~názvom „Popoludnie pri Biblii“, kde preberáme postupne biblické knihy. V~tomto roku sme študovali Prvý list apoštola Pavla Korintským. Našich stretnutí sa zúčastňuje cca 10 -- 12 bratov a sestier aj z~iných spoločenstiev (KZ Rača, Kresťanské spoločenstvo Christiana). Tohto roku aj tieto naše stretnutia boli podmienené daným možnostiam stretávania podľa pandemickej situácie v~SR.

* \aktivita{Služba bezdomovcom}
    Aj tento rok sme podporovali službu varenia pre bezdomovcov v~rámci spoločenstva {\it Kresťania v~meste}.
\enditems

\autor{Pavel Pivka}


\clanok{Hospodársky výbor}
Stíšením a vyprosením si požehnania podľa textu písma Matúš 6,34, ktorý sme dostali od Pána, sme pristúpili k~plánovaniu úloh a potrieb zboru.

Rozpracované úlohy, ktoré neboli splnené, sme preniesli do plánu roku 2021.

Na prelome mesiacov apríl a máj sme pristúpili k~nedokončeným prácam na chvojnickom kostolíku. S~firmou Mantap zo Sobotišťa sme vybudovali odvodňovací systém, chrániaci pred vlhkosťou obvodové múry kostolíka a upravili jeho okolie. Na zborovej chalupe sme zastabilizovali padajúci plot a jeho odvodnenie v~dĺžke cca 30~m, odvedenie spádovej dažďovej vody potrubím v~dĺžke 40~m od chalupy a výmenu poškodených zemných elektrických káblov v~dĺžke 50~m.

Brigádnicky sme vykonali generálne upratanie chalupy -- povalu, kompletnú rekonštrukciu latkového plotu (očistenie, napustenie a výmena poškodených lát), odstránili sme poruchy z~elektrickej revízie, montáž nového elektrického rozvádzača, montáž svietidiel v~stodole, oprava obloku na stodole atď. Pod vedením br. kazateľa Dannyho Jonesa sa uskutočnili brigády na Palisádach (okolie kostola, SPŠE, ZŠ a v~zariadení DSS). Chceli by sme poďakovať všetkým účastníkom spomínaných akcií, že pridali ruku k~dielu a mohli sme byť svetlom nášmu okoliu aj v~tejto dobe. Vďaka patrí tímu zvukárov, projekcii, upratovaciemu servisu a všetkým, ktorí sa podieľali na živom vysielaní bohoslužieb. Ak Pán dá, chceme pokračovať v~tomto roku na prácach potrebných pre chod nášho zboru.

Prevádzka zborovej chalupy v~tomto roku bude možná až po vybudovaní nového septiku. Čakáme na cenovú ponuku a v~jarných mesiacoch pristúpime k~realizácii. Uvedomujeme si, že všetko je v~Božích rukách a ak situácia dovolí, naplnia sa aj naše plány.

\autor{Daniel Mikletič}


\clanok{Biblické a iné vzdelávanie}
Spoločné štúdium Svätého Písma bolo uplynulý rok, tak ako celý náš život, ovplyvnené nástupom koronakrízy začiatkom marca 2020, keď naše stretnutia na biblických hodinách museli byť viackrát prerušené a online vzdelávanie vzhľadom na charakter nášho vzdelávania na biblických hodinách a zloženie účastníkov a aj moje kapacitné možnosti sme nezrealizovali.

Po vianočných a novoročných prázdninách sme od druhej polovice januára (23.~1.) 2020 pokračovali v~preberaní Apoštolského vyznania viery druhým článkom o~našej viere v~Pána Ježiša Krista. Pokračovali sme až do začiatku marca (5.~3.), kým neprišiel prvý lockdown a zrušenie našich stretnutí.  Po uvoľnení opatrení sme od 7.~5. pokračovali až do konca školského roka a dokončili sme preberanie Apoštolského vyznania.

Po prázdninách sme začali naše biblické vzdelávanie opäť začiatkom októbra (8.~10.) prebraním ostatných starokresťanských vyznaní viery (Rímske krstné symbolum, Nicejské, Nicejsko-Carihradské, Chalcedónske a Atanáziovo). Touto témou sme preberanie rôznych vyznaní viery ukončili a rozhodli sme sa nepokračovať v~štúdiu ďalších, historicky neskorších a denominačne špecifických vyznaní.

Po ďalšej prestávke spôsobenej lockdownom a inými, najmä mojimi obmedzenými časovými možnosťami, sme začiatkom decembra (3.~12.) začali preberať Kázeň na hore a do vianočných prázdnin a opätovného zavedenia lockdownu sme stihli prebrať úvody k~tejto téme a prvé blahoslavenstvo. Od nového roku 2021 sa biblické vzdelávanie na biblických hodinách vo štvrtky kvôli zákazu prezenčného stretávania neobnovilo.

Paralelne pokračovalo v~utorky podľa možností aj v~roku 2020 biblické štúdium (najmä) seniorov pod vedením brata Pavla Pivku.

O ďalších prebiehajúcich formách biblického vzdelávania v~zbore v~rámci jednotlivých zložiek a skupiniek nemám prehľad a správa o~nich môže byť obsiahnutá v~správach za jednotlivé zložky. Viaceré prebiehajúce formy vzdelávania v~rámci našej BJB ako aj naddenominačné sa v~roku 2020 zrušili, diali v~redukovanej forme, alebo prešli na online formu.

\autor{Ján Szőllős}


\clanok{Sestry}
Veľmi som vďačná za sestry, ktoré v~roku 2020 so mnou pokračovali vo vedení služby sestrám: Gitka Kráľová, Jarka Cihová, Mirka Hovorková, Barbi Antalíková, Angie Vráblová a Radka Bánová. Po Radkinom pôrode jej miesto v~tíme s~radosťou zaujala Barborka Pribulová.

Keď sa spätne pozriem na službu sestrám v~roku 2020, napadne mi verš z~Prísloví 16: „Človek vo svojom srdci plánuje, odpoveď však prichádza od Hospodina.“ Naše plány pre sestry na tento rok sa veľmi zmenili z~dôvodu obmedzení kladených na zbor v~dôsledku koronavírusu. Rok sme začali spoločným štúdiom biblického štúdia {\it Trvalá Sloboda} aj s~niekoľkými sestrami zo zboru Viera. Aj keď ide o~šesťtýždňové štúdium, podarilo sa nám stretnúť iba dvakrát, aby sme sa mu venovali, a ešte raz so sestrami z~výboru OS BJB skôr, ako boli všetky stretnutia zrušené kvôli pandemickým obmedzeniam. Na týchto prvých stretnutiach roka pribudli nové ženy a viaceré sa pripojili aj spomedzi starších. Naše sestry si kúpili 60 kópií biblického štúdia {\it Trvalá sloboda}, ale viaceré to zrejme mohli študovať len doma, lebo na stretnutia prišlo v~priemere 30 -- 35 sestier všetkých vekových skupín. Keď nás navštívili sestry z~výboru BJB, všetky naše prítomné sestry povedali niečo o~sebe. Bola som vďačná sestrám z~výboru, že nás o~to požiadali, lebo sme tým boli veľmi požehnané.

Počas prvej vlny koronavírusu ma Pán usmernil, aby som denne posielala sestrám aspoň jeden verš so stručným komentárom. Týmto spôsobom som sa snažila naďalej povzbudzovať sestry Božím Slovom a zostať s~nimi v~spojení v~čase, keď sme sa nemohli stretať.

Po uvoľnení pandemických obmedzení sme mali možnosť stretnúť sa ešte dvakrát pred letnými prázdninami. Na týchto stretnutiach niekoľko sestier hovorilo silné svedectvá o~tom, čo zažili s~Bohom počas pandémie. Takisto sa zdieľali o~tom, čo sa naučili zo štúdia a ako im Pán dal v~ich živote slobodu; potom sme sa spolu modlili. Tieto dve stretnutia boli pre nás veľkým povzbudením.

Tento rok bola konferencia sestier BJB zrušená kvôli koronavírusu.

Na jeseň sme sa pokúsili pokračovať v~štúdiu {\it Trvalej slobody}, ale podarilo sa nám stretnúť iba trikrát, kým znovu neboli stretnutia zakázané. Keďže ubehlo tak veľa času, odkedy sme preberali tie prvé týždne štúdia, zopakovali sme si ich, a potom sa nám podarilo prebrať a prediskutovať len po tretí týždeň štúdia. No aj na tých stretnutiach sme boli povzbudené spolu stráveným časom pri Božom Slove ako aj tým, že sme počuli mocné svedectvá.

Okrem stretnutí sestier sme pred prvou vlnou pandémie, ako aj po nej, mali každý týždeň v~utorok o~9.30~hod. stretnutie Klubíku pre mamičky na materskej so svojimi deťmi. Mojím cieľom na týchto stretnutiach bolo budovať vzájomné vzťahy, počúvať mamičky a povzbudzovať ich vlastnými skúsenosťami a Božím Slovom. Každý týždeň si mamičky vybrali nejakú tému, spoločne sme si prečítali stíšenie a diskutovali o~ňom. Týchto stretnutí sa zúčastňovalo asi desať mladých mamičiek so svojimi deťmi vrátane jednej zo zboru Viera.

Na sesterské stretnutia ako aj Klubík som dostala pozitívnu spätnú väzbu. Sestry sa tešili, že môžu spolu študovať Božie Slovo a v~skupinkách diskutovať o~tom, čo sme sa naučili.

Pre všetky sestry a ich dcéry sme mali v~októbri dlho pripravovanú a do posledných detailov naplánovanú víkendovku, ktorú sme však nakoniec museli zrušiť pre pandémiu.

Svetový deň modlitieb baptistických žien nám ponúka príležitosť uvedomiť si, že našu rodinu tvoria aj sestry v~iných krajinách a na iných kontinentoch. Pretože sme sa k~tejto príležitosti nemohli spolu stretnúť v~modlitebni v~novembri, spojili sme sa na modlitbách vo dvojiciach alebo trojiciach telefonicky alebo cez internet. V~každej zo skupiniek, ktoré sme si vytvorili, sme sa mohli sústrediť na aspoň jeden kontinent a spoločne sme sa modlili za modlitebné potreby všetkých žien Únie baptistických žien.

12. decembra sme prijali požehnanie návštevou sestier z~výboru Odboru sestier.

Aj keď bol rok 2020 pre mnohých z~nás skúškou, verím, že Boh pôsobil medzi nami a skrze nás. Veľmi ma povzbudil počet sestier, ktoré boli ochotné udržiavať vzájomný kontakt telefonicky alebo písomne cez internet.

\autor{Clara Jones}


\clanok{Mládež}
Rok 2020 pre mládež začal ako každý iný, mládež sa stretávala pravidelne a dokonca sme zažili aj spoločnú mládežnícku konferenciu v~Banskej Bystrici, ktorá sa konala vo februári. Na konferencii sme zažili požehnané chvíle a boli sme pripravení na ďalší rok, ktorý môžeme prežiť v~spoločenstve. Avšak tak isto ako každého aj mládež prekvapila pandémia. Po konferencii sme sa niekoľkokrát videli, mali sme spoločnú mládež s~Teen-z-one (mládež Viera), kde sme sa pridali k~sérii tém br. kazateľa Dannyho o~vzťahoch. Mali sme aj mládeže na Súľovskej, kde sme trávili spoločný čas najmä vonku, relatívne bezpečne, keď to opatrenia a situácia dovoľovali. Neskôr sme stretnutia mládeže prerušili a stretávali sme sa už len občasne vonku, keď to bolo možné, alebo na niektorých stretnutiach Teen-z-one, ktoré organizuje mládež zboru Viera.

Sme radi aj za to, že napriek obmedzeniam sme spolu nestratili kontakt úplne a máme technológie a možnosti, ktoré nám umožňujú stretávanie.

\autor{Dávid Pribula}


\clanok{Dorast}
Rok 2020 bol v~doraste plný rôznych zmien. Od januára do začiatku marca sme sa stretávali počas nedeľných popoludní na Zrínskeho. Pokračovali sme v~predtínedžerskom programe, ktorý mal za cieľ pripraviť dorastencov na obdobie dospievania po teoretickej aj praktickej stránke. Počas ďalších mesiacov naše stretávanie prerušila korona. V~júli sme absolvovali trojdňový denný tábor, kde sme v~spolupráci s~Dannym a niektorými mládežníkmi zažili mestskú hru vo Sv. Jure, bicyklovačku do Čunova a spoločenské hry s~prechádzkou Starým Mestom. Každým dňom sa niesla téma, ktorá nám približovala cestu k~osobnému nasledovaniu Ježiša v~našich životoch.

Od septembra sme znovu obnovili osobné stretnutia dorastu v~novom čase, v~piatok o~17.30~hod. Osobne sa nám podarilo stretnúť len štyrikrát, ale boli to pre nás po dlhej pauze vzácne stretnutia so vzácnymi ľuďmi. Od októbra sme stretnutia presunuli do online priestoru. V~novembri sme ukončili predtínedžerský program a počas adventu sme sa venovali  stretnutiam Ježiša s~rôznymi ľuďmi (samaritánka, Nikodém, Zacheus, Mária a Marta). Je pre nás zložité viesť dorasty online formou. Je to zložité aj pre dorastencov, keď majú po týždni v~škole pred monitorom stráviť ďalšie online stretnutie. Sme ale veľmi vďační, že dorastenci si zvykli na tento formát a nevynechávajú skoro žiadne stretnutia. Niekedy si kladieme otázku, či majú online stretnutia pre túto vekovú skupinu význam. Ale pokračujeme s~nádejou, že každé zasiate Božie Slovo prinesie úžitok v~príhodný čas.

Minulý rok sme stretnutia dorastu viedli v~zložení Martin Simon, Rado Nemec, manželia Halamičkovci a manželia Vráblovci. Počas celého roka aktívne spolupracujeme s~Dannym Jonesom. Na dorastoch sa zúčastňovali Dara Plett, Tamarka a Marek Syčovci, Matej a Benko Maďarovci, Damián a Diana Mikolášovci, Daniel a Lenka Vráblovci, Tomáš a Šimon Halamičkovci, Oskar Kolárovský, Radko Nemec, Tobi a Adelka Dzuriakovci.

Sme Pánu Bohu vďační za každé osobné stretnutie v~tomto roku, ale aj za to, že môžeme pokračovať v~online stretávaní pri Božom Slove. Sme vďační za každého menovaného dorastenca a za to, že môžeme byť súčasťou jeho cesty k~Bohu. Prosíme vás o~modlitby za dorastencov zo zboru, ktorí sa nezúčastňujú na našich stretnutiach. Prosíme o~modlitby za Davyda a Illiu Pototských, ktorí sa k~nám koncom roka začali pripájať. Aby porozumeli slovenčine a cítili sa medzi nami prijatí. Prosíme o~modlitby za to, aby sme vedeli verne žiť a komunikovať Božie Slovo zrozumiteľným spôsobom pre dorastencov. Ďakujeme, že aj touto formou sa zapojíte do sprevádzania dorastencov na ich ceste k~Pánu Bohu.

\autor{Michal a Angela Vráblovci}


\clanok{Besiedka}
Tak ako všetky oblasti života zboru aj besiedku poznačila v~roku 2020 pandémia. Stretli sme sa len niekoľkokrát, a to nám je ľúto. Žiaľ, aj takéto obdobia prichádzajú a my ich musíme prekonať.

V našom zbore sme mali v~minulom roku 16 detí v~malej besiedke (od 3 do 7 rokov). Na začiatku roka sme s~deťmi preberali tému „modlitba“. Hovorili sme o~tom, ako sa modlil Pán Ježiš, učeník Peter alebo Pavol a Sílas. Niektoré deti sa radi modlia aj nahlas vlastnými slovami, iné radšej modlitbu opakujú po učiteľovi. Dôležité je, že sa deti modlia. Povzbudzujme ich k~tomu a modlime sa spolu s~nimi. V~modlitbe sa často dozvieme, čo dieťa trápi alebo z~čoho má radosť.

V malej besiedke pracovali Miriam Kešjarová, Kristína Kešjarová, Katka Kerekréty, Janka Máťušová, v~druhom polroku sa k~nám pridala Mirka Hovorková. Sme radi, že párkrát sme sa stretli aj po letných prázdninách a pomáhala nám Tamarka Syčová a Marta Brnová.

Vo veľkej besiedke sa stretávalo okolo 10 detí vo veku od 7 do 10 rokov. Na začiatku roka sa deti zoznamovali so životnými osudmi misionárov a po letných prázdninách sme s~nimi začali preberať modlitbu.

Veľkú besiedku viedla Vierka Kolárovská spolu s~Barborkou Pribulovou a Ľubkou Kováčikovou.

V čase, keď sme sa s~deťmi nemohli stretávať „naživo“, komunikovali sme cez sociálne siete, e-maily a videá. Uvedomujeme si, že deti a rodičia prežívajú náročné obdobie, preto sa snažíme myslieť na nich na modlitbách a povzbudzovať ich aspoň vo virtuálnom priestore.

Príjemným osviežením pre všetky vekové kategórie bol tradičný rodinný tábor v~Častej Papierničke, ktorého sa zúčastnilo množstvo ľudí nielen z~nášho zboru. Postarané bolo o~všetky vekové kategórie a vďaka Bohu tábor prebehol bez závažnejších komplikácií. Modlíme sa, aby sme zborový tábor mohli zorganizovať aj v~tomto roku.

Náročné obdobie spôsobuje tlak na rodiny. Sme nervóznejší, netrpezlivejší a možno prežívame strach a neistotu. To všetko dolieha na naše deti -- malé i veľké. Prosím, nezabúdajme na najmladšiu generáciu -- modlime sa za nich -- to je najdôležitejšie, povzbudzujme ich a upriamujme ich pohľad na Pána Ježiša -- slovom, ale predovšetkým životom.

\autor{Miriam Kešjarová}


\clanok {Ukrajinská služba}
V roku 2020 bola ukrajinská služba do veľkej miery ovplyvnená pandémiou ochorenia COVID-19. Začiatkom roka sme mali pravidelné stretnutia každú nedeľu a stredu na Zrínskeho 2. Každú druhú nedeľu prebiehali stretnutia ukrajinskej skupinky aj v~Nitre v~priestoroch zboru Cirkvi bratskej. V~niektoré nedele prekročil počet ľudí na našich stretnutiach 20. Na Veľkú noc sme už začali plánovať aj prvé ukrajinské bohoslužby.

Od začiatku pandémie a v~dôsledku štátom prijatých opatrení sme mali raz do týždňa v~nedeľu stretnutia cez Zoom. V~čase uvoľnení pandemických opatrení sme mali stretnutia každú nedeľu na Zrínskeho alebo v~Medickej záhrade. Podarilo sa nám v~tomto období zorganizovať spoločný výlet na Železnú studničku.

V roku 2020 som začal prípravu na krst s~dvoma ľuďmi – s~jedným bratom v~Bratislave a jednou sestrou v~Nitre. Viacerí spomedzi Ukrajincov prejavili záujem o~členstvo v~zbore, k~čomu prebehlo aj stretnutie so staršovstvom, avšak zborové členské zhromaždenie sa kvôli pandemickým opatreniam nakoniec neuskutočnilo.

17. októbra sa na Slovensko presťahovala moja rodina -- manželka Liliia a tri deti: Davyd, Illia a Solomia.

\autor {Viktor Pototskyi}


\clanok{Spevokol}
Na rok 2020 len tak skoro nezabudneme. Ešte nikdy v~mojom živote nenastala situácia, keď náš spevokol, ako celkom slušne sa rozvíjajúce teleso, prestal zrazu existovať. Myslím tým nielen na to, že po celý rok sme neslúžili, ale nemohli sme sa ani stretávať na nácvikoch.

Nebolo to však zapríčinené vlastnou pohodlnosťou či dokonca nechuťou k~tejto službe, ale pôsobením nebezpečného vírusu, ktorý zasiahol celý svet.

Ešte v~prvých mesiacoch roka sme sa usilovne začali pripravovať na Veľkú noc a dokonca nás pozvali koncertovať do Báčskeho Petrovca. Naše účinkovanie  sme naplánovali tak, aby tesne pred veľkonočnými sviatkami sme mali koncert v~Bratislave a o~dva týždne by sme už slúžili na Dolnej zemi, keď tieto sviatky oslavujú v~pravoslávnych kostoloch. Už sme aj s~dolnozemskými bratmi virtuálne  rozdeľovali našich spevákov na ubytovanie v~príbytkoch našich bratov, keď zákaz verejných akcií všetko zablokoval.

Tento stav trval prakticky do konca roka a my môžeme len dúfať, že v~nasledujúcom roku sa to všetko zmení.

Vieme, že bez Božieho súhlasu sa nám ani vlas na hlave neskriví, tak všetko prijímame s~pokorou, vďakou a v~snahe rozpoznať dôvod, prečo k~nám Stvoriteľ takto prehovoril.

Do nového roka sme prešli na modlitbách s~nádejou opätovného rozbehnutia tejto služby na oslavu nášho Spasiteľa. Ak to bude Jeho vôľa.

\autor{Slávo Kráľ}


\clanok{Služba ľuďom v~núdzi}
V roku 2020 sme sa opäť v~spolupráci s~občianskym združením {\it Kresťania v~meste} (ďalej ako „KvM“) zapojili do pomoci ľuďom v~núdzi, a to hlavne varením polievok a ich výdajom.  Mobilné výdaje stravy,  nápojov a šatstva prebiehali pod mostom Lafranconi napriek pandémii COVID-19 v~priebehu celého roka 2020. Dobrovoľníci KvM varia 30 litrov výdatnej polievky a pripravujú 30 litrov nápojov, ktoré v~letnom období vydávajú dvakrát týždenne a v~zimnom období trikrát týždenne. Na jeden výdaj chodí cca 50 -- 100 ľudí bez domova. Do tejto služby je zapojených cca 150 dobrovoľníkov z~rôznych kresťanských spoločenstiev, z~ktorých niektorí varia polievku, iní zabezpečujú prevoz a výdaj stravy, ďalší zabezpečujú zbierky šatstva atď. Ročne navaria dobrovoľníci, ku ktorým patrí a aktívne sa zapája aj náš zbor, približne 4~000 litrov výdatnej polievky a vydá sa viac ako 10~000 porcií teplej stravy.

Vo výdajovom tíme pri výdaji polievky a nápojov sa minulý rok aktívne zapájala Zuzka Pařízková. Do služby varenia sa zapojili: Kohútovci st., Laurenčíkovci, Zuzka Pařízková, Taligovci, Slávka Volentičová, Rút Bednáriková, Marta Račičová, Danka Kotmanová, Pribulovci st. a Beata Bogárová s~mamou.

Finančné dary na nákup surovín na polievky boli preposlané na účet zboru a od januára 2020 ich spravuje zborová účtovníčka Ľubka Kohútová.

Rada by som poďakovala všetkým vám, ktorí ste napriek zložitej situácii spojenej s~pandémiou priložili ruku k~dielu a poslúžili ste núdznym či už uvarením polievky, jej výdajom, financiami či modlitbami. Vďaka Pánu Bohu za Vás!

\autor{Beata Bogárová}


\clanok{Revízia hospodárenia}
Revízna komisia v~zložení Miroslav Antalík, Helena Mikletičová, Barbora Antalíková za spolupráce účtovníčky zboru Ľubomíry Kohútovej vykonali revíziu hospodárenia za rok 2020.

Boli prekontrolované nasledovné doklady:
\begitems \style -
* výpisy z~bežného účtu vedeného v~Slovenskej sporiteľni za mesiace 1, 3, 4, 6, 8, 10 a 12
* výdavkové pokladničné doklady za mesiace 1 -- 12
* príjmové pokladničné doklady za mesiace 1 -- 12
\enditems
Revízna komisia konštatuje, že uvedené doklady sú vedené prehľadne v~súlade s~účtovnými predpismi. Pokladničná kniha je vedená mesačne a založená priamo pri pokladničných dokladoch.

Neboli zistené žiadne nedostatky.

Stav finančnej hotovosti ku dňu 31.~12.~2020 bol:

\vskip1em\hskip1cm\table{lr}{
pokladňa & 1~863,86~€ \cr
bankový účet & 71~467,97~€ \crl
spolu & 73~331,83~€ \cr
}\vskip1em

Tento stav súhlasí so stavom v~účtovnej evidencii k~uvedenému dátumu.

\autor{Miroslav Antalík}

\tiraz
\bye
