\def\velkostpisma{10}
\def\velkostriadku{12.5}
\input makra.tex % nacitanie Ivanom pripravenych nastaveni a prikazov
\hyphenation{star-šov-stvo} % rozdelenie slov na konci riadku, treba tu uviest slova, ktore sam nepozna

\spravodaj{4}{2024}

\def\sekcia#1{\vskip0.2em\noindent #1}

\clanok {Všemohúci Boh -- Él Šaddaj}

Toto Božie meno hovorí o~tom, že Boh nie je obmedzený vo svojom konaní. Ani časom, ani vzdialenosťou, ani okolnosťami, ani osobou, ani žiadnou duchovnou či telesnou bytosťou. Všemohúci znamená, že môže všetko. Ako to vnímaš vo svetle svojej skúsenosti s~Bohom alebo vo svetle okolností svojho života?

Boh sa Abramovi zjavil ako Él-Šaddaj, všemohúci Boh a uzavrel s~ním večnú zmluvu, ktorú ustanovil medzi sebou a Abramom a jeho potomstvom. Až do čias Mojžiša, keď bolo zjavené ďalšie Božie meno, považovali patriarchovia Él-Šaddaj za zmluvné meno Boha. Keď sa modlíš k~Bohu, vnímaš to, že Boh ako Él-Šaddaj je ten, ktorému nič nie je nemožné? Uctievaš Ho ako toho, ktorý je všemocný a má posledné slovo aj v~tvojom živote? Modlíš sa k~Bohu s~obmedzením Božích schopností, ktoré si schopný prijať vierou? Alebo Ho vnímaš ako Boha, ktorý môže len niečo…? Ak áno, môžeme Ho potom nazvať slovom „Boh“? Na druhej strane Jeho všemocnosť potrebujeme vnímať vo svetle zasľúbení, že dodrží to, čo zasľúbil, a nepôjde proti sebe.

\sekcia{KĽÚČOVÉ VERŠE}

„Keď mal Abrám deväťdesiatdeväť rokov, zjavil sa mu Hospodin a povedal: Ja som Él-Šaddaj (všemohúci Boh), choď stále predo mnou, buď bezúhonný (dokonalý). Uzavriem (uzavrel som) zmluvu medzi mnou a tebou a veľmi ťa rozmnožím.“ (Gn 17,1-2)

\sekcia{ZAMYSLIME SA}

V~liste Galaťanom sa píše, že tí, ktorí veria, sú Abrahámovými synmi. Čo pre teba znamená byť synom všemohúceho Boha? Él-Šaddaj je meno Boha, ktoré sa viaže so zmluvou, ktorú On uzavrel. Zažívaš naplnenie Božej zmluvy s~Abrahámom vo svojom živote? Zažil si moc všemohúceho Boha konajúceho v~tvojom záujme?

Modlitba: {\em Pane, pomôž mi poznať Ťa ako všemocného Boha, ako toho, ktorý ma dokáže zachovať a~požehnať, ako toho, ktorý splní aj v~mojom živote každý prísľub, ktorý sa na mňa vzťahuje. Daj mi, prosím, bázeň voči Tebe a~Tvojej moci, aby som Ti mohol byť verný nasledujúc príklad Abraháma a~vždy Ti veriť, že Ty si všetko, čo potrebujem.}

„Kto pod ochranou Najvyššieho prebýva a v~tôni Všemohúceho sa zdržiava, povie Pánovi: ‚Ty si moje útočište a pevnosť moja; v~Tebe mám dôveru Bože môj.‘ Pretože sa ku mne pritúlil, hovorí Pán, vyslobodím ho, ujmem sa ho, lebo pozná moje meno.“ (Ž 91,1-2,14)

\vskip-1ex\begitems
* {\it Chváľ ho}: Pretože On je tvojim úkrytom pred každou búrkou.
* {\it Ďakuj mu}: Za Jeho ochraňujúcu starostlivosť o~teba a tvoj život.
* {\it Vyznaj mu}: Ak si sa strachoval namiesto toho, aby si sa spoľahol a dôveroval Mu ako Všemohúcemu.
* {\it Pros ho}: Aby posilnil tvoju vieru a dôveru v~Jeho zasľúbenia.
\enditems

\sekcia{VŠEMOHÚCI BOH = ÉL-ŠADDAJ}

\sekcia{PRISĽÚBENIA SPOJENÉ S~MENOM ÉL-ŠADDAJ}

Abrahám a jeho rodina žili uprostred ľudí, ktorí verili v~iných bohov -- krehkých mlčanlivých bôžikov z~kameňa, či z~dreva…. Ako protiklad -- náš Boh je živý a mocný, schopný splniť každý Jeho prísľub. Boh požehnal Abraháma špeciálnymi prisľúbeniami pre jeho život a nám tieto prisľúbenia patria tiež, pretože sme skrze vieru Abrahámovými potomkami.

\sekcia{PRISĽÚBENIA V~PÍSME}

„Pánovo meno je pevnou vežou, spravodlivý do nej utečie a je bezpečný.“ (Pr.~18,10)

Nech sú nám slová nasledujúcich veršov z~Genezis 12,2-3 na povzbudenie: „Urobím z~teba veľký národ, požehnám ťa a preslávim tvoje meno a ty budeš požehnaním. Požehnám tých, čo ťa budú žehnať, a prekľajem tých, čo ťa budú preklínať! V~tebe budú požehnané všetky pokolenia zeme!“

\autor{inšpirované knihou Božie mená, Peter Šrankota}


\clanok {Správy zo staršovstva za marec}

Staršovstvo zboru BJB Palisády sa v~marci stretlo dva razy a to 12.~3. a 26.~3.~2024. Na prvom stretnutí boli pozvaní záujemcovia o~členstvo v~zbore, kde sa zdieľali s~nami so svojimi osobnými svedectvami. Boli prerokované prípravy na jarnú víkendovku v~zariadení Berea v~Harmónii. Ďalej sme sa pripravovali na konanie VZČZ. V~rôznom sa riešilo plánovanie krstu na 16.~6.~2024, výsledok stretnutia ohľadom Súľovskej a informácia o~konferencii Cesta obnovy. Ďalej sa venovalo našej mládeži a odchodu Radka Paulena.

Na druhom stretnutí sme najprv vyhodnocovali právnu analýzu možností uplatnenia reštitučných nárokov zboru. Opäť sme sa zaoberali prípravou jarnej víkendovky Berea Harmónii. Ďalej sme plánovali organizovanie letného zborového tabora v~DM Častej. Vyhodnotili sme priebeh VZČZ aj s~podnetmi členov zboru. Venovali sme sa samostatne podnetu členky zboru. V~rôznom sme plánovali organizovanie konferencie odboru sestier v~roku 2025, potreby ukrajinského zboru a plánovanie termínu rodinného koncertu na Palisádach. V~závere sme si dohodli ďalšie kroky pri vyhodnocovaní NCD dotazníku.

Nasledujúce stretnutie staršovstva sa uskutoční 9.~4.~2024.

\autor {za staršovstvo Marcel Maďar}


\clanok{Rekonštrukcia exteriéru našej modlitebne}

Objekt našej modlitebne bol postavený v~r.~1868 podľa projektu významného bratislavského architekta a staviteľa Ignáca Feiglera ako pohrebná kaplnka. Do užívania pre účely nášho zboru sme ju prebrali v~roku cca 1975, kedy prebehla aj rekonštrukcia.

Odvtedy prebehlo viacej rekonštrukčných zásahov za účelom skvalitnenia týchto priestorov. Spomeniem aspoň niektoré:

\vskip1em
\table{c(-)p{11cm}}{
rok & rekonštrukčné zásahy \crli
2004 & výmena okien a rekonštrukcia kúrenia \cr
2008 & rekonštrukcia krovu a výmena krytiny \cr
2010 & zateplenie stropu a nové elektrorozvody na strope \cr
2013 & odizolovanie obvodových múrov podrezaním (obitie omietky sokla na fasáde do 1 m výšky) \cr
 & kompletná rekonštrukcia interiéru (sanácia omietok, nové elektrorozvody, výmena podlahy, maľovanie stien a stropu, maľovanie lavíc, reštaurovanie bočných aj vstupných dverí) \cr
 & rekonštrukcia vykurovacieho systému (nové rozvody pod omietkou, pridanie radiátorov) \cr}
\vskip1em

Fasáda nášho objektu má na mnohých miestach praskliny, niekde sa šúpe farba, kríž nad strechou je poškodený, chýba omietka sokla po celom obvode, ostatné maľovanie bolo robené v~r.~1975.

Naša modlitebňa je pamiatkovým objektom -- {\em Pohrebná kaplnka – solitér} -- evidovaná v Ústrednom zozname pamiatkového fondu SR a tvorí súčasť nehnuteľnej národnej kultúrnej pamiatky Cintorín Kozia Brána (súbor hrobov a náhrobníkov) na Šulekovej ulici v~Bratislave, zapísanej v~Ústrednom zozname pamiatkového fondu SR.

Z~tohto dôvodu sme oslovili Krajský pamiatkový úrad Bratislava so zámerom obnovy fasády. Z~KPÚ sme dostali predbežný súhlas aj so stanovenými podmienkami. Jednou z~nich je zabezpečiť vypracovanie reštaurátorského výskumu a spracovanie dokumentácie s~návrhom obnovy prostredníctvom zodpovedných osôb.

\cast{Financie}

Dve cenové ponuky na kompletnú rekonštrukciu fasády máme z~roku 2021 a cena bola v~rozmedzí 95~000 € až 111~000~€ zaokrúhlene. Ďalšia cenová ponuka je z~jesene 2023 a je vo výške 137~000~€ zaokrúhlene. Z~tejto CP vychádzame, nakoľko v~nej už boli zohľadnené skokové nárasty cien v~stavebníctve z~rokov 2022 a 2023. Taktiež sme oslovili reštaurátorov na dodanie cenovej ponuky na výskum a vypracovanie dokumentácie s~návrhom obnovy. Najvýhodnejšia ponuka bola v~hodnote 5500~€ zaokrúhlene.

\vskip1em
\table{(\hskip-1.5mm)lr}{
Reštaurátorský výskum + dokumentácia návrhu & 5 500 € \cr
Rekonštrukcia fasády (materiál + práca) & 137 000 € \cr
Spolu & 142 500 € \cr
Predpoklad realizácie prác r.~2025 & \cr
Predpokladaná medziročná inflácia +4\% & 5 700 € \cr
{\bf Predpokladané celkové náklady} & {\bf 148 200 €} \cr
Z invest. fondu (tvoreného z~ned. zbierok a darov) chceme použiť & 60 000 € \cr
Zaokrúhlene treba ešte vyzbierať & 90 000 € \cr
Počet aktívnych členov & cca 100 ľudí \cr
{\bf Odporúčaná priemerná výška daru na jedného člena} & {\bf 900 €} \cr}
\vskip1em

Prosíme členov zboru, aby rozmýšľali o~tejto veci pred Pánom aj v~zmysle biblických textov Mal. 3,10 a 2.Kor 8,12. Je potrebné nahlásiť výšku svojho záväzku ekonómke zboru s.~Kohútovej do konca apríla 2024. Realizácia daru je prevodom na zborový účet SK36 0900 0000 0000 1147 1836, variabilný symbol: 777.

Naďalej budú pokračovať aj nedeľné zbierky počas štvrtých nedieľ v~mesiaci (zbierka na investičný fond), určené hlavne pre priateľov zboru.

Je potrebné si tiež uvedomiť, že financie na tento účel sú nad rámec nášho bežného dávania do zborovej pokladne, teda je potrebné zachovať aj dary (desiatky....), ktoré sme dávali doteraz.

\autor {za hospodársky výbor Ľubomír Syč}
\vfill\break

\clanok{Stretnutie sestier}

Sestry sa stretnú v~apríli na svojom pravidelnom stretnutí v~stredu dňa 24.~4.~2024 o~17.30~hod. na Zrínskeho 2. Hosťkou stretnutia bude s.~Dáša Leeder, dcéra Danky Hanesovej. Všetky sestry sú srdečne vítané.


\clanok{Senior klub}

Posledný štvrtok v~mesiaci dňa 25.~4.~2024 budú mať seniori svoje stretnutie od~10.00 do~14.00~hod. na Súľovskej~2. Program a iné bližšie informácie budú oznámené neskôr.


\n 4.	4.	Viera	ŠKODÁK;
\n 6.	4.	Filip	BARKÓCZI;
\n 6.	4.	Jarmila	CIHOVÁ;
\n 6.	4.	Jana	ZAJACOVÁ;
\n 10.	4.	Anna	PAVLÍKOVÁ;
\n 11.	4.	Daniel	MIKLETIČ;
\n 19.	4.	Marta	PRIBULOVÁ;
\n 22.	4.	Alexander	Koloman	ERDÉLYI;
\n 25.	4.	Elena	TALIGOVÁ;
\n 30.	4.	Ľuboš	DZURIAK;
\n 30.	4.	Jaroslav	VOLENTIČ;
\narodeniny


\program{
\p  1 ; po ;.;;.;;
\p  2 ; ut ; 15.15 ; Biblická hodina pre seniorov (P. Pivka) ;.;;
\p  3 ; st ;.;;.;;
\p  4 ; št ; 18.00 ; Biblická hodina (J. Szőllős) ;.;;
\p  5 ; pi ; 17.30 ; Dorast ;.;;
\p  6 ; so ; 18.00 ; Mládež ;.;;
\p  7 ; ne ;  9.30 ; Bohoslužby (P. Šrankota + VP) ;.;;
\p  8 ; po ;.;;.;;
\p  9 ; ut ; 15.15 ; Biblická hodina pre seniorov (P. Pivka) ;.;;
\p 10 ; st ;.;;.;;
\p 11 ; št ; 18.00 ; Biblická hodina (J. Szőllős) ;.;;
\p 12 ; pi ; 17.30 ; Dorast ;.;;
\p 13 ; so ; 18.00 ; Mládež ;.;;
\p 14 ; ne ;  9.30 ; Bohoslužby (Blitz návšteva) ;.;;
\p 15 ; po ;.;;.;;
\p 16 ; ut ; 15.15 ; Biblická hodina pre seniorov (P. Pivka) ;.;;
\p 17 ; st ; 17.30 ; Stretnutie sestier (D. Leeder) ;.;;
\p 18 ; št ; 18.00 ; Biblická hodina (J. Szőllős) ;.;;
\p 19 ; pi ; 17.30 ; Dorast ;.;;
\p 20 ; so ; 18.00 ; Mládež ;.;;
\p 21 ; ne ;  9.30 ; Bohoslužby (P. Šrankota) ;.;;
\p 22 ; po ;.;;.;;
\p 23 ; ut ; 15.15 ; Biblická hodina pre seniorov (P. Pivka) ;.;;
\p 24 ; st ;.;;.;;
\p 25 ; št ; 18.00 ; Biblická hodina (J. Szőllős) ;.;;
\p 26 ; pi ; 17.30 ; Dorast ;.;;
\p 27 ; so ; 18.00 ; Mládež ;.;;
\p 28 ; ne ;  9.30 ; Bohoslužby (M. Ira) ;.;;
\p 29 ; po ;.;;.;;
\p 30 ; ut ; 15.15 ; Biblická hodina pre seniorov (P. Pivka) ;.;;
}


\tiraz
\bye
