\def\velkostpisma{9}
\def\velkostriadku{12}
\input makra.tex % nacitanie Ivanom pripravenych nastaveni a prikazov
\hyphenation{star-šov-stvo} % rozdelenie slov na konci riadku, treba tu uviest slova, ktore sam nepozna

\spravodaj{7-8}{2025}


\clanok {Varovanie: Biblia ohrozí vaše sústredenie sa na smartfón}
Sú aplikácie hrozbou pre sústredenie sa na Boha? Áno. Ale funguje to obojsmerne. Bojujte proti ohňu ohňom. Ak čítate Bibliu na počítači alebo smartfóne či iPade, prítomnosť e-mailovej aplikácie, aplikácie na správy a facebookovej aplikácie v~každej chvíli hrozia, že odtiahnu vašu pozornosť od Slova Božieho. Pravda. Bojujte proti tomu. Ak vás prst zvádza, odrežte ho. Alebo použite akékoľvek iné cnostné násilie (Mt~11,12), ktoré vás uvoľňuje, aby ste mohli svoju myseľ zamerať na Boha. Ale nezaujímajte iba obranný postoj. Bojujte proti ohňu ohňom.
Prečo by sme mali rozmýšľať o~aplikácii Facebook, ktorá ohrozuje aplikáciu Biblia? Prečo nie aplikácia Biblia, ktorá ohrozuje Facebook, e-mailovú aplikáciu, RSS aplikáciu a správy?
Rozhodnite sa, že dnes otvoríte aplikáciu Biblia tri razy za deň. Nie, päť ráz. Desať ráz! Možno stratíte kontrolu a stanete sa závislými od Biblie!
Znova a znova získavajte svoju dvojminútovú dávku život dávajúceho Jedla. Nie samým Facebookom bude človek žiť. Myslím to vážne. Boží hlas nebol nikdy tak ľahko dostupný. A dovoľte Biblii ohrozovať vaše sústredenie. Alebo ešte lepšie: nechajte Bibliu vniesť vás späť do reality znova a znova počas dňa...

\autor {John Piper © Desiring God}


\clanok {Správy zo staršovstva}

V~júni sme sa stretli na dvoch pravidelných stretnutiach.

8.~6. strávila časť zboru na Chvojnici. Pripomínali sme si tam sviatok zoslania Ducha Svätého. Časť účastníkov strávila na Chvojnici celý víkend a druhá časť prišla na nedeľné bohoslužby, spoločný obed a poobedné rozhovory. Slovom nám slúžil brat Rado Nemec na tému „Pošlem vám iného Tešiteľa“. Následne sme pri stretnutí staršovstva hovorili o~tejto zborovej akcii. Sme potešení tým, že Pán Boh dáva do sŕdc členov nášho zboru túžbu po spoločne strávenom čase. Máme radosť z~toho, že sme výlet na Chvojnicu nemuseli organizovať, ale že snaha a iniciatíva vyšla z~radov členov zboru.

Ďalšou témou, o~ktorej sme hovorili je dorast a mládež. Na dorast chodili stabilne štyria dorastenci. Vedúci dorastu nastavovali zloženie tímu na budúci školský rok. Cieľom práce s~mládežou a v~mládeži musí byť duchovný rast a zameranie sa na silný vertikálny vzťah s~Bohom. Inak životné krízy odvedú mladých ľudí preč od Boha. Z~našej strany sú dôležité modlitby, aby sa nestratili vo svete, ostali pri Pánovi a snáď aj v~našom zbore.

Na stretnutie 24.~6. prišiel br. kazateľ Miloš Masarik z~Banskej Bystrice a „odovzdal“ nám štafetu praxe brata Dávida Chuchúta. Hovorili sme o~ceste, ktorou už Dávid prešiel počas praxe v~Banskej Bystrici, jeho silných stránkach aj o~tom, v~čom potrebuje porásť a v~čom mu má prax v~našom zbore pomôcť. Dávidova prax má prebiehať aj v~zborovej stanici Connect. Preto bol s~nami aj br.~Tomáš Valchář.

Po ukončení témy praxe Dávida Chuchúta sme hovorili o~Connecte, rozvoji ich práce a o~tom, ako sa môžeme pripojiť k~ich letnej misijnej aktivite. O~tejto téme budeme informovať začiatkom júla na pravidelných bohoslužbách.

Od br.~kazateľa Tima Hanesa sme dostali pozitívne informácie ohľadom liečenia a rehabilitácie jeho zranenia. Sme vďační Pánu Bohu, že ho postupne stavia na nohy a po rehabilitácii ho budeme môcť privítať aj v~našom zbore. V~Bratislave ho stretneme koncom augusta, ale jeho službu slovom na Palisádach plánujeme na 21.~9. Modlime sa aj za túto príležitosť.

Pre podporu finančných zbierok sme pripravili QR kódy.
Môžete ich nájsť v~laviciach. Našou snahou je:
\begitems
* Umožniť prispieť finančným darom aj tým, ktorí nenosia hotovosť so sebou,
* Reagovať na pravidelné účelové zbierky, ale aj mimoriadne zbierky, ktoré vopred ohlasujeme:
\begitems
* misia -- pravidelne 2.~nedeľu,
* investičné účely zboru -- pravidelne 4.~nedeľu,
* vopred oznámené mimoriadne zbierky.
\enditems
\enditems

S~modlitbou, láskou a prianím Božieho požehnania vo vašich, v~našich životoch

\autor {za staršovstvo P.~Pribula}


\clanok {Leto}
Počas leta sa naše pravidelné stretnutia počas týždňa neuskutočnia. V~nedeľu sa budeme stretávať na bohoslužbách. Dorastenci a mládežníci sa v~júli zúčastnia tábora v~Novej Lehote, širšia zborová rodina na zborovom tábore v~auguste. Modlime sa za prípravné tímy týchto táborov ako aj ich samotný priebeh. Nech  popri oddychu, nás -- účastníkov, inšpiruje Svätý Duch ešte k~väčšej svätosti a Božej blízkosti.


\clanok {Krst}
Počas zborového tábora je naplánovaný aj krst ponorením (priamo v~areáli Detskej misie, alebo v~blízkosti -- podľa podmienok).
Máme niekoľko záujemcov. Ak by mal ešte niekto záujem z~našich radov (nemusí byť účastník tábora), prosíme nahláste sa u~br.~kazateľa Janka Szőllősa. Presný dátum a čas zverejníme neskôr v~ústnych a e-mailových oznamoch.


\clanok {Verš na mesiac}
V~júli sa budeme učiť verš, ktorý dostal pre rok 2025 hospodársky výbor: „Nie vy ste si mňa vyvolili, ale ja som si vyvolil vás a ustanovil som vás, aby ste išli a prinášali ovocie a vaše ovocie, aby zostávalo, tak aby vám Otec dal, čokolvek by ste v~mojom mene prosili od Neho.“ (J~15,16)

V~auguste máme verš, ktorý dostala chválospevová skupinka: „Neboj sa, lebo ja som s~tebou, nepozeraj sa ustrašene vôkol seba, lebo ja som Boh tvoj. Posilním ťa a pomôžem ti, i podopriem ťa svojou spásonosnou pravicou.“ (Iz~41,10)


\clanok {Naši predkovia viery}
4.~--~5.~7. sa vo Veľkých Levároch v~kultúrnom dome uskutoční sympózium Spoznávanie dedičstva histórie Hutteritov a Habánov. Vstup je voľný. Bude to jedinečná príležitosť vypočuť si odborníkov s~trvalým odkazom habánskej tradície v~kultúrnom dedičstve.


\clanok {Campfest}
V~auguste sa tradične na Ranči v~Kráľovej Lehote uskutoční už 27.~ročník festivalu chvál -- Campfest. Termín je 7.~--~9.~8.~2025. Viac sa dozviete na webovej stránke \ulink[https://www.campfest.sk/]{campfest.sk}.
\vskip1cm
\noindent Prajeme všetkým pekné, oddychové a požehnané leto.


\n 4.	7.	Margita	ELISHEROVÁ;
\n 4.	7.	Ľubomíra	KOHÚTOVÁ;
\n 10.	7.	Slavomír	MÁŤUŠ;
\n 10.	7.	Katarína	KEREKRÉTY;
\n 16.	7.	Rút	BEDNÁRIKOVÁ;
\n 20.	7.	Mária	KOHÚTOVÁ;
\n 27.	7.	Lenka	KOHÚTOVÁ;
\n 28.	7.	Elena	ŠALINGOVÁ;
\n 28.	7.	Pavlína	SYNOVCOVÁ;
\n 31.	7.	Marína	CIHOVÁ;
\n 1.	8.	Dana	KEŠJAROVÁ;
\n 1.	8.	Zuzana	HRAŠKOVÁ;
\n 1.	8.	Vlasta	ŠALINGOVÁ;
\n 2.	8.	Marta	RAČIČOVÁ;
\n 8.	8.	Ksenia	IONOVÁ;
\n 11.	8.	Šimon	HOVORKA;
\n 11.	8.	Ľuboslava	KOVÁČIKOVÁ;
\n 16.	8.	Radovan	JANČULA;
\n 18.	8.	Anna	LIPTÁKOVÁ;
\n 23.	8.	Danica	PAULENOVÁ;
\n 25.	8.	Ivan	PAULEN;
\n 31.	8.	Miroslava	HOVORKOVÁ;
\narodeniny


\programna{7}{
\p  6 ; ne ;  9.30 ; Bohoslužby (Ján Szőllős + VP);.;;
\p 13 ; ne ;  9.30 ; Bohoslužby (Filip Barkóczi);.;;
\p 20 ; ne ;  9.30 ; Bohoslužby (Peter Pribula);.;;
\p 27 ; ne ;  9.30 ; Bohoslužby (Ľubomír Syč);.;;
}
\vskip3ex
\programna{8}{
\p  3 ; ne ;  9.30 ; Bohoslužby (Ján Szőllős + VP);.;;
\p 10 ; ne ;  9.30 ; Bohoslužby (Filip Barkóczi);.;;
\p 17 ; ne ;  9.30 ; Bohoslužby (Slávo Kráľ);.;;
\p 24 ; ne ;  9.30 ; Bohoslužby (Radislav Nemec);.;;
\p 31 ; ne ;  9.30 ; Bohoslužby (Dávid Chuchút);.;;
}

\tiraz
\bye
