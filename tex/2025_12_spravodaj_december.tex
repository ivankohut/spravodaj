\def\velkostpisma{10}
\def\velkostriadku{12.5}
\input makra.tex % nacitanie Ivanom pripravenych nastaveni a prikazov
\hyphenation{star-šov-stvo} % rozdelenie slov na konci riadku, treba tu uviest slova, ktore sam nepozna

\spravodaj{12}{2025}


\clanok{Budú tvoje Vianoce pravé alebo falošné?}

Vianoce môžeš prežiť bez Santa Klausa a jeho škriatkov, bez soba Rudolfa s~červeným nosom, či sobov, ktoré ťahajú jeho sane, či bez Deda Mráza. Môžeš ich prežiť bez Grincha či Ebenezera Scrooga a Malého Tima z~Vianočnej koledy. Pravé Vianoce však nemôžeš prežiť bez Ježiša. Inak to bude len ich napodobenina.

Necháp ma zle. Vianoce mám rád. Problém však spočíva v~tom, že Vianoce sme si až príliš skrášlili kočmi, ktoré ťahajú kone, snehom, praskajúcim ohňom v~kozube a horúcim kakaom. Príbeh narodenia sme utopili vo farebných svetlách a nepretržitom hudobnom podmaze. To všetko je krásne, ale niekedy si myslím, že sa nám stratila surovosť toho príbehu tým, že sme ho urobili takmer sentimentálnym.

Vianoce sme zromantizovali, či dokonca ich urobili bezobsažnými, a otupili sme príbeh, ktorý hovorí, že Všemocný Boh zostúpil z~neba, aby sa narodil v~stajni na podlahe. Pomyslenie, že Boh toto všetko urobil pre nás, je silnejšie než akákoľvek zromantizovaná verzia Vianoc.

Ježiš prišiel v~určenom čase, v~dokonalom čase. Biblia nám hovorí: „Ale keď nadišiel určený čas, poslal Boh svojho Syna. Narodil sa ako každý iný človek a ako taký sa musel podrobovať zákonu, aby nás, ktorí sme boli otrokmi zákona, mohol oslobodiť, a tak aby nás Boh mohol znova prijať za svoje deti“ (G~4,4–5).

Pochop, okolnosti v~Izraeli boli v~danej dobe veľmi temné. Ľudia žili ako otroci pod nadvládou Rímskej ríše, ktorá každého nútila podriadiť sa prostredníctvom Pax Romana vynúteného mieru. Buď si robil, čo diktoval Rím, alebo si draho platil.

Pridaj k~tomu fakt, že nebo už celých 400~rokov mrazivo mlčalo. Nebol tam žiaden prorok, ktorý by sprostredkoval Božie Slová a nezjavovali sa ani žiadni anjeli, prinášajúci správy od Boha. Nebol tam nikto, kto by konal zázraky. Nič sa nedialo. Bol tam iba zmätok, násilie a bieda.

Izaiáš 9,1 opisuje túto dobu slovami: „Ľud, ktorý chodí vo tme, uzrie veľké svetlo; nad tými, ktorí bývajú v~temnej krajine, zažiari svetlo.“ Výraz „tma“ hovorí o~zlobe a nevedomosti.

Ľudia žili v~dobe plnej zloby a nevýslovného utrpenia, v~dobe plnej násilia, útlaku a nespravodlivosti. Dosť to pripomína dnešok. Čakali, že sa niečo stane. A~niečo sa aj stalo. V~správnom okamihu Boh poslal svojho Syna.

Predstav si, čo pre nás urobil. Predstav si, čo všetko nám dal. Biblia hovorí: „Hoci sám bol Boh, nedomáhal sa božských nárokov, zriekol sa slávy a moci a ochotne prijal úlohu najobyčajnejšieho človeka“ (F~2,6–7).

Ježiš sa nezriekol svojej božskej podstaty; zriekol sa božských nárokov, ktoré z~nej vyplývali. Kráčal medzi nami ako človek -- bezhriešny človek. Žil naším životom a zomrel našou smrťou. Áno, On prišiel na túto zem. Práve toto oslavujeme na Vianoce. Nadišiel čas, aby sme vzdali úctu Tomu, ktorý sa narodil uprostred Božieho ľudu.

Biblia nám hovorí aj to, že jedného dňa sa Ježiš opäť vráti. No tentoraz nepríde, aby Ho uložili do jasieľ; vráti sa v~sláve.

Pri jeho prvom príchode Ho zavinuli do plienok. Pri jeho druhom príchode bude oblečený ako kráľ a bude mať rúcho namočené do krvi.

Pri jeho prvom príchode bol obklopený zvieratami a pastiermi. Pri jeho druhom príchode Ho budú sprevádzať svätí a anjeli.

Pri jeho prvom príchode nebolo pre neho miesta v~hostinci. Pri jeho druhom príchode budú preňho otvorené brány nebies.

Pri svojom prvom príchode bol Baránkom Božím, ktorý zomrie za hriechy sveta. Pri svojom druhom príchode bude zúrivým Levom z~Júdovho kmeňa a prinesie súd. Toto je deň, o~ktorom Biblia svedčí, keď hovorí: „V Ježišovom mene pokľaklo každé koleno tých, čo sú na nebi aj na zemi, aj pod zemou“ (F~2,10). Ježiš sa opäť vráti.

Ježiš môže práve teraz prísť aj do tvojho života. Môže dnes vstúpiť do tvojho sveta, ak Ho ochotne pozveš do svojho života. Ježiš povedal: „Poďte ku mne všetci, ktorí sa namáhate a ste unavení, nájdete u~mňa pokoj a úľavu.“ (Mt~11,28) Tieto Ježišove slová sú určené každému vystresovanému človeku.

Si práve teraz vystresovaný? Aj počas vianočnej oslavy úplne zabúdame na Ježiša, lebo sme v~takom zhone, sme tu a o~chvíľu už inde. Počas oslavy môžeme byť zaneprázdnení.

Ježiš v~podstate hovorí: „Príď ku mne so svojimi problémami. Príď ku mne so svojimi starosťami. Poď sem a nájdi pri mne odpočinok.“

Tento rok máš na výber: Môžeš zažiť pravé Vianoce, alebo môžeš zažiť falošné Vianoce. Môžeš mať ich falošnú verziu, alebo môžeš mať ich pravú verziu so samotným Kristom žijúcim v~tvojom vnútri.

Boh ponúka, že odpustí všetky tvoje hriechy. Boh ponúka, že ti dá v~živote druhú šancu. Boh ti ponúka lístok do neba, garantovanú rezerváciu miesta. Hovorí, že to všetko môže byť tvoje, ak natiahneš ruku a prijmeš jeho dar.

Dar večného života je jediným, ktorý sa stále znova dáva. Nie sú potrebné žiadne batérie. Nie je potrebné žiadne zhromaždenie. Úplne stačí, keď natiahneš ruku, prijmeš ho a jednoducho poďakuješ.

\autor{Greg Laurie}


\clanok{Správy zo staršovstva}
Starší zboru sa stretli v~novembri na dvoch riadnych stretnutiach, konkrétne 11.~11. a 25.~11.~2025. Venovali sa príprave zhromaždení a podujatí. Od tých na najbližšie obdobie (krst), cez predvianočné stretnutia (kontemplatívne bohoslužby, vianočný koncert), vianočné stretnutia až po tie povianočné (Silvester, Nový rok).

Ďalšou oblasťou bola príprava návrhu rozpočtu na rok 2026. Sestra Ľ.~Kohútová po dohode so stašovstvom zboru pripravila jeho novú, zjednodušenú verziu. Cieľom je urobiť ho prehľadnejším, zlúčiť podobné položky do rovnakých kategórií a pod. Pokračovať v~tomto bode budeme začiatkom budúceho roka.

Taktiež sa staršovstvo venovalo záverom KDZ, s~ktorými bolo oboznámené zúčastnenými delegátmi zboru. Rozpočet bol schválený spolu so súvisiacimi dokumentmi. Bola tiež schválená nominačná komisia, nakoľko nás na jarnej KDZ čakajú voľby do orgánov BJB v~SR. Myslite, prosím, na to vo svojich modlitbách. Tiež vás prosíme, aby ste sa obrátili na členov staršovstva so svojimi návrhmi kandidátov.

Staršovstvo zboru sa tiež venovalo otázke nájmu za priestory na Súľovskej. Zo strany Rady bolo avizované razantné navýšenie platieb za nájom od budúceho roka. Nájdené riešenie (správa Súľovskej pod patronátom Connect-u) umožní zachovať nájom v~jeho pôvodnej výške.

Okrem toho to bola štandardná agenda týkajúca sa chodu zboru. Ďakujeme za vaše modlitby a prosíme, aby ste v~nich vytrvali. Nech sa veci dejú na Jeho slávu.
\autor{Peter Antalík}
\vfill\break


\clanok{Advent}
V tomto adventom období sa máme možnosť stíšiť počas štvrtkových večerov na adventných kontemplatívnych bohoslužbách. Pripravujú to pre nás bratia Dávid a Filip. Ak ste sa ešte nezúčastnili, máme pred sebou ešte 2 štvrtky -- 11.~12. a 18.~12. o~18.00 v~našej modlitebni na Palisádach.

Milým vianočným „pozdravom“ nás potešia naše deti z~besiedok a dorastu a to v~nedeľu 14.~12. vrámci nedeľných bohoslužieb.

V sobotu 20.~12. o~17.00 sa svetlá rozsvietia, hudba a spev zazneje na oslavu príchodu nášho Spasiteľa. Veľký spevokol a komorný orchester nášho zboru majú opäť tradične pripravený vianočný koncert na Palisádach. Pozvite svojich známych, priateľov, susedov. Druhý koncert bude v~nedeľu 21.~12. o~17.00 v~evanjelickom kostole na Strečnianskej 15. Všetci ste srdečne vítaní.

Termíny bohoslužieb počas vianočných sviatkov si môžete prezrieť na zadnej strane tohto spravodaja.


\clanok{Krst}
V nedeľu 30.~11.~2025 sa v~modlitebni CASD uskutočnil krst. Svoju vieru krstom vyznali dvaja bratia Michal a Ladislav a dve sestry Eva a Silvia. V~malom kruhu bratov a sestier bolo veľmi zaujímavé počuť osobné svedectvo každého z~nich. Radosť a modlitby požehnania ich vyprevádzali do ďalších dní.


\clanok{Rozhovory na Palisádach}
Minulý mesiac sme zverejnili rozbehnutie projektu „Rozhovory na Palisádach“.
Prvé stretnutie sa uskutoční 6.~12.~2025 na tému Umelá inteligencia. V~novom roku 2026 sú naplánované aj ďalšie témy. O~nich aj presných dátumoch budete čoskoro informovaní.
Takisto vám budeme vďační za modlitby aj za tento projekt a sú vítané aj vaše podnety, ktoré môžete posielať na emailovú adresu \email{vzdelavanie@bjbpalisady.sk}
\autor{organizačný tím}


\clanok{Obedy s~bratmi}
Bratia Janko Szőllős, Filip Barkóczi a Dávid Chuchút prijímajú naše pozvania na obed do našich domácností. Prosíme, zapíšte sa do online tabuľky.


\clanok{Modlitby}
Myslime na modlitbách na: opravu fasády našej modlitebne, na rodiny, ktoré podporujeme v~misiách, na teplo pre ukrajinské zbory, na chorých, osamelých a v~trápeniach duše...
...a ďakujme za dar Ježiša.


\clanok{Zbierky v~novembri}
\table{lr}{
Účelová zbierka na budovu v~Revúcej (2.~11.)	& 1~823,00~€ \cr
Na misiu (9.~11.)								&   380,00~€ \cr
Na investičný fond (23.~11.) 					&   222,50~€ \cr}
\vskip1em

Pripomíname, že zbierky sa u~nás konajú pravidelne takto:
\vskip-1ex\begitems
* každú 2. nedeľu v~mesiaci je zbierka venovaná misii a
* každú 4. nedeľu je zbierka na investície.
\enditems

Aj naďalej máte možnosť prispieť do „nedeľnej zbierky“, a to prevodom na účet zboru. Do poznámky pre prijímateľa, prosím, uveďte „zbierka“.

Bankové spojenie: SK36 0900 0000 0000 1147 1836, SWIFT: GIBASKBX

\n 2.	12.	Helena	MIKLETIČOVÁ;
\n 3.	12.	Ľubica	IROVÁ;
\n 5.	12.	Tomáš	LAURENČÍK;
\n 6.	12.	Elise	ATKINS;
\n 9.	12.	Ondrej	ŠKODÁK;
\n 9.	12.	Kamila	ZAJÍČKOVÁ;
\n 11.	12.	Vladimíra	LAURENČÍKOVÁ;
\n 11.	12.	Maroš	KOHÚT;
\n 13.	12.	Peter	KOLÁROVSKÝ;
\n 16.	12.	Pavel	KONDAČ ml.;
\n 16.	12.	Martin	PELÍŠEK;
\n 23.	12.	Diana	DZURIAKOVÁ;
\n 23.	12.	Eva	Rudy	DOROVÁ;
\n 24.	12.	Slávka	VOLENTIČOVÁ;
\n 25.	12.	Dana	PELÍŠKOVÁ;
\n 26.	12.	Jana	KRÁĽOVÁ;
\n 28.	12.	Dara	PLETT;
\n 29.	12.	Ján	KOVÁČIK;
\narodeniny


\program{
\p  1 ; po ; 18.00 ; Modlitebná skupinka ;.;;
\p  2 ; ut ; 15.00 ; Biblická hodina pre seniorov (P.~Pivka) ;.;;
\p  3 ; st ;.;;.;;
\p  4 ; št ; 18.00 ; Adventné kontemplatívne bohoslužby ;.;;
\p  5 ; pi ; 17.30 ; Dorast ;.;;
\p  6 ; so ; 18.00 ; Mládež ;.;;
\p  7 ; ne ;  9.30 ; Bohoslužby (J.~Szőllős + VP) ;.;;
\p  8 ; po ; 18.00 ; Modlitebná skupina / skupinka „Základy viery“ ;.;;
\p  9 ; ut ; 15.00 ; Biblická hodina pre seniorov (P.~Pivka) ;.;;
\p 10 ; st ;.;;.;;
\p 11 ; št ; 18.00 ; Adventné kontemplatívne bohoslužby ;.;;
\p 12 ; pi ; 17.30 ; Dorast ;.;;
\p 13 ; so ; 18.00 ; Mládež ;.;;
\p 14 ; ne ;  9.30 ; Bohoslužby (F.~Barkóczi + Program besiedky a dorastu) ;.;;
\p 15 ; po ; 18.00 ; Modlitebná skupinka ;.;;
\p 16 ; ut ; 15.00 ; Biblická hodina pre seniorov (P.~Pivka) ;.;;
\p 17 ; st ;.;;.;;
\p 18 ; št ; 18.00 ; Adventné kontemplatívne bohoslužby ;.;;
\p 19 ; pi ; 17.30 ; Dorast ;.;;
\p 20 ; so ; 17.00 ; Vianočný koncert spevokolu ; 18.00 ; Mládež ;
\p 21 ; ne ;  9.30 ; Bohoslužby (D.~M.~Chuchút) ;.;;
\p 22 ; po ;.;;.;;
\p 23 ; ut ;.;;.;;
\p 24 ; st ; 15.00 ; Bohoslužby (P.~Kolárovský) ;.;;
\p 25 ; št ; 10.00 ; Bohoslužby (S.~Baláž) ;.;;
\p 26 ; pi ;.;;.;;
\p 27 ; so ;.;;.;;
\p 28 ; ne ;  9.30 ; Bohoslužby (D.~M.~Chuchút) ;.;;
\p 29 ; po ;.;;.;;
\p 30 ; ut ;.;;.;;
\p 31 ; st ; 18.00 ; Silvestrovské bohoslužby (F.~Barkóczi) ;.;;
\p  1 ; št ; 10.00 ; Novoročné bohoslužby 2026 (J.~Szőllős) ;.;;
}


\tiraz
\bye
