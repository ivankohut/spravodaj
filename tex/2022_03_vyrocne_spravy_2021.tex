\def\velkostpisma{10}
\def\velkostriadku{12.5}
\input makra.tex % nacitanie Ivanom pripravenych nastaveni a prikazov
\hyphenation{star-šov-stvo} % rozdelenie slov na konci riadku, treba tu uviest slova, ktore sam nepozna

\vyrocnespravy{2021}

\clanok{Zbor}
Iz. 43, 1 {\it A~teraz,takto hovorí Hospodin, tvoj stvoriteľ, Jákob, tvoj tvorca, Izrael: „Neboj sa, veď som ťa vykúpil. Zavolal som ťa po mene, ty si môj. Keď budeš prechádzať cez vody, budem s~tebou, a keď cez rieky, nezaplavia ťa. Keď pôjdeš cez oheň, nepopáliš sa a plameň ťa nespáli.Veď ja som Hospodin, tvoj Boh, Svätý Izraela, tvoj Spasiteľ.}


Z~roku 2021 máme všetci pocit, že sme prechádzali cez oheň a vodu. Z~každého pohľadu to bol rok výziev. Ale Boh, verný svojmu Slovu, nás doviedol do nového roku a milosťou nás premenil. Som za to vďačný. Už sme žiaľ boli zvyknutí na nečakané zmeny, či už spôsobenými covidom, karanténami, zmenami opatrení, alebo životnými výzvami v~rámci práce a rodiny. Som Bohu vďačný za každého, kto robil extra viac, aby zbor mohol naďalej fungovať.
Veľká vďaku patrí bratom zo staršovstva, ktorí počas celého roku hodiny riešili tieto komplikované okolnosti a problémy. Čelili ťažkým rozhodnutiam, ktoré mali dopad na zhromaždenia, besiedky, skupiny, ale aj jednotlivcov. Niekedy sme s~bratmi skutočne nevedeli čo ďalej, ale vedení modlitbami, Boh sám nás viedol. Bratia sú obetaví služobníci a patrí im vďaka a česť. 

Diakoni obetavo slúžili starým a chorým. Kvôli opatreniam ich kontakt bol najmä telefonický, volali, aby  povzbudili, čo bolo pre mnohých veľké požehnanie. Riskovali občas aj svoje zdravie, aby navštívili nedeľné zhromaždenie a vysluhovali Večeru Pánovu. Je to skupinka vzácnych služobníkov. Vďaka patrí Palimu Pivkovi za roky vo vedení tejto služby. 
Tento rok sme vedenie diakonskej služby odovzdali bratovi Jankovi Štefkovi. Aj Jankovi vďaka za ochotu slúžiť. 

Hudobníkom sme vďační za hodiny príprav a služby, či prezenčne alebo cez internet. Bolo to náročné aj tým, že niektorí bývajú až v~Rakúsku. Veľká vďaka Diane Dzuriakovej za vedenie tejto služby. Vďaka patrí aj Slávkovi Kráľovi, ktorý hodiny za počítačom pripravoval piesne a chvály na ne-prezenčné zhromaždenia.

Ani skupina technikov nesedela nečinne. Počas celého roku ešte viac hodín ako bežne, obetovali svoj čas a zlepšovali zborové technické zariadenie. Pracovali na zlepšení živého prenosu. Kúpili sme dve nové kamery, a to veľmi zmenilo kvalitu služby k~lepšiemu. Robili x vecí, ktorým by sme my ostatní ani nerozumeli, len aby všetko fungovalo najlepšie, ako sa dá. Veľká vďaka patrí Ľubošovi Kešjarovi a jeho tímu. 

Vedúci našich 7 skupiniek sa snažili slúžiť za každú cenu, či prezenčne, cez Zoom alebo osobným kontaktom. Bolo to skutočne náročné, a preto som každému vďačný. Niektoré skupinky dokonca narastali, pridávali sa noví členovia.

Ďakujem ženám, ktoré slúžili sestrám. Covid opatrenia, ktoré nás obmedzovali, sa často menili a tým aj ich plány. Ale vidíme, že ženy zboru sa nevzdali a počas roka bolo veľa žien dotknutých Božou prácou. Vďaka Clare a jej celému tímu vedúcich. 

Tak isto pokračovala naša služba budúcej generácii, hoci to bolo tiež občas náročné. Vďaka každému vedúcemu besiedky, dorastu aj mládeži. Asi ste boli najviac obmedzení a museli ste sa stále prispôsobovať. Stretnutie dorastu cez Zoom nie je jednoduché, ale zvládli ste to.

Vďaka všetkým, čo pripravovali tábory, pracovali vo volebnej a revíznej komisii, starali sa o~majetky zboru v~Bratislave alebo na Chvojnici.
Vodu a oheň sme zvládli len vďaka Božej milosti. Chvála Jemu!

2021 bol celoročný zápas s~covidom a lockdownmi. Veľa veci sme síce plánovali, ale aj sme museli zrušiť. Popri tom všetkom sme zažili Božiu milosť:
\begitems
* Počas lockdownu Danny s~Clarou boli v~máji a júni v~USA, z~rodinných dôvodov.
* Dňa 1.~8.~2021 sme do zborovej rodiny privítali Paulínku Kolářikovú. 
* Mládežnícky/Dorastenecký tábor sa uskutočnil v~termíne 26.~8. --- 31.~8.~2021 na Chvojnici. 
* Na zborovom tábore 15.~8. -- 21~.8.~2021 sme v~stredisku DM v~Častej mali plnú kapacitu. Preberali sme List Efežanom a zamerali sa na duchovnú výzbroj. 
* Na Zrínskeho sa začala stretávať skupina Anonymných Alkoholikov pod vedením Zuzky Tomanovej. Za tieto priestory sú veľmi vďační. 
* Na vyznanie viery sme 23.~10.~2021 krstili Adama Antalíka, Dávida Jančulu, Oskara Kolárovského, Saru Škodákovú a Rustama Zardieva (z~ukrajinského zboru).
* Chvála Bohu za to, že sa nám podarilo zorganizovať stretnutie členov zboru 21.~11.~2021, počas ktorého sme do zborovej rodiny privítali Dávida Jančulu, Adama Antalíka, Oskara Kolárovského, Saru Škodákovú, Betku Smolkovú, Rustama Zardieva, Dariu Lobodu, Mychaila Vaska, a Mariiu Vaska. 
* Danny s~Clarou museli odísť do USA z~rodinných dôvodov. Ich pobyt tam trval od 24.~11.~2021 do 14.~1.~2022. Z~Arkansasu slúžil Danny slovom cez nahrávky. Počas týchto dní Pán potvrdil návrat Jonesovcov natrvalo do USA, čo oficiálne oznámili zboru 23.~1.~2022.
\enditems

\autor{Danny Jones}


\clanok{Staršovstvo}
Staršovstvo zboru pracovalo v~roku 2021 v~zložení kazateľ zboru Danny Jones a členovia Peter Antalík, Vladimír Ira, Peter Kolárovský, Miroslav Kolářik, Daniel Plett, Peter Pribula a Ján Szőllős. Od polovice roku pribudol do tímu Viktor Potocky ako kazateľský praktikant po ukončení štúdia.

Celý rok 2021 bol poznačený koronavírusom a zabezpečovaním služieb v~zbore. Záver roku bol poznačený sledovaním situácie u~Jonesovcov. 

Témy, ktorým sme sa v~priebehu roku venovali, boli:
\begitems
* Zabezpečenie bohoslužieb počas pandémie; 
* Príprava a realizácia Zborových členských zhromaždení;
* Stretnutia so záujemcami o~krst a členstvo v~zbore;
* Pastoračné otázky;
* Práca medzi ukrajinsky hovoriacimi návštevníkmi zboru;
* Príprava krstu a vďakyvzdania;
* Ukončeniu služby na Chvojnici;
* Rozhodovanie Dannyho a Clary o~návrate do USA.
\enditems

Sme vďační nášmu nebeskému Otcovi za Jeho vedenie a požehnanie pri všetkom, čomu sme sa počas minulého roku venovali. Zároveň sa tešíme na všetko, čo má pre nás pripravené v~roku 2022. Na ten sme dostali verš zo Žalmu 50,23 - {\it Kto vďaku obetuje, ten ma ctí, tomu, kto správnou cestou kráča, ukážem Božiu pomoc.} Rozumieme, že máme byť vďační a snažiť sa chodiť po Božej ceste. Prosím o~to, aby sme to dokázali najmä v~týchto ťažkých časoch.

\autor{Peter Pribula}


\clanok{Diakonia}

V roku 2021 sa menil vedúci zborovej diakonie na Palisádach. Brata Pavla Pivku po 11 rokoch v~tejto službe vystriedal brat Ján Štefko. Oficiálne k~odovzdávaniu tejto funkcie došlo 14.~11.~2021 v~rámci nedeľného zhromaždenia. Bratovi Pivkovi sa zbor poďakoval aj hodnotným vecným darom, za čo brat Pivka aj touto cestou ešte raz ďakuje zboru BJB na Palisádach.

\cast{Pracovné stretnutia}
Rok 2021 bol pre nás všetkých zvláštnym rokom. Bol už druhým rokom celosvetovej pandémie. To ovplyvnilo zborový život a samozrejme aj službu nášho tímu diakonov. Vidno to aj z~toho, že sme sa počas celého roku pracovne mohli fyzicky stretnúť len raz (20.~9.). Bolo to posledné stretnutie tímu diakonov pod vedením brata Pavla Pivku. (Zápisnice z~pracovných stretnutí boli pravidelne zaslané všetkým členom e-mailom, prípadne osobne odovzdané). Pozvánky na pracovné stretnutia pre členov tímu diakonov boli zasielané e-mailom spravidla tri dni vopred. Taktiež boli vyhlasované v~oznamoch v~rámci nedeľného zhromaždenia.

\def\aktivita#1{{\it #1\par}\firstnoindent}
\cast{I. Vnútrozborové aktivity}

\begitems \style n
* \aktivita{Návštevná služba}
Pravidelnú návštevnú službu našich imobilných členov v~domácnostiach, vykonávali sestry a bratia, ktorí majú s~navštevovanými členmi zboru vytvorený prirodzený, blízky vzťah. Samozrejme v~minulom roku aj tieto návštevy boli obmedzené a prispôsobené daným podmienkam. Okrem našich seniorov, boli navštevovaní aj naši nemocní bratia a sestry, či už v~domácnostiach, alebo nemocniciach vrámci daných možností.

* \aktivita{Dopravná služba}
O dopravnú službu do zhromaždenia v~minulom roku nebol záujem.

* \aktivita{Zborové obedy sa v~minulom roku neorganizovali}

* \aktivita{Svätodušné sviatky na Chvojnici}
Náš zbor v~minulom roku neslúžil na Svätodušné sviatky na Chvojnici, ako po iné roky, kvôli pandémii.

* \aktivita{Vysluhovanie Večere Pánovej}
Večera Pánova sa vysluhovala pravidelne každú prvú nedeľu v~mesiaci (podľa rozpisu).
\enditems

\cast{II. Aktivity zboru smerom von}

\begitems \style n
* \aktivita{Služba v~domovoch sociálnej starostlivosti}
Okrem služby v~našom zbore sa venujeme aj službe mimo zboru v~domovoch dôchodcov pod vedením brata P. Pivku za vernej pomoci sestier Lenky Gubovej a Vladky Laurenčíkovej. V~prípade potreby, brata Pivku zastúpil „v~Betánii“ brat kazateľ D. Jones. Pravidelne navštevujeme „Domovy sociálnej starostlivosti“ v~Starom meste, v~Dúbravke.

* \aktivita{Biblické vyučovanie}
V rámci služby seniorom máme aj duchovnú časť služby v~podobe pravidelných biblických hodín -- „popoludnie pri Biblii“ kde preberáme postupne biblické knihy. V~tomto roku sme študovali (2.list apoštola Pavla Korintským a list Židom). Našich stretnutí sa zúčastňuje cca 8-10 bratov a sestier aj z~iných spoločenstiev (napr. KZ Rača). Tohto roku aj tieto naše stretnutia boli podmienené daným možnostiam stretávania podľa pandemickej situácie v~SR.

* \aktivita{Služba bezdomovcom}
Ako zbor sme aj tento rok podporovali službu varenia pre bezdomovcov v~rámci spoločenstva {\it Kresťania v~meste}.
\enditems

\autor{Pavel Pivka}


\clanok{Hospodársky výbor}
Plní očakávania z~ukončenia pandémie, pod vedením nášho Pána sme boli odhodlaní postaviť sa čelom k~povinnostiam a plánovaniu projektov r. 2021.

Cenovú ponuku od firmy Mantap p. Krčeka na vybudovanie nového septiku, ktorý bol podmienkou na prevádzkovanie zborovej chalupy na Chvojnici sme obdržali začiatkom mája. K~realizácii sme pristúpili 31.~5. -- 7.~6.~2021. S~firmou Mantap spolupracovali a boli nápomocní pri práci J. Perička, P. Kovaľ, D. Mikletič. Starý septik bude využívaný na odvodnenie objektov a stiahnutie dažďovej vody. Zachytená voda bude využitá ako užitková pre stavebnú činnosť. V~tomto roku bolo štrnásť rekreačných pobytov, jedna svadba a sedem brigád. Vďaka patrí predovšetkým nášmu Pánovi, že i počas pandémie bolo možné cez uvoľnené opatrenia využívať toto naše zborové zariadenie.
V~našom kostole na Palisádach v~predsieni bol zhotovený mobiliár firmou Šramo, potrebný ako úložný priestor. 

Chcel by som poďakovať všetkým sestrám aj bratom, ktorí priložili ruku k~dielu, prípadne podporili finančne zborovú prácu. Zvlášť by som chcel poďakovať s. Anke Šandorovej za prácu kostolníčky, ktorú vykonávala s~ prestávkami celé desaťročia. Vďaka patrí aj tímu pod vedením br. Ľ. Kešjara, zvukárom, projekcii a všetkým, ktorí zabezpečovali online vysielanie.

Veršíkom na povzbudenie pre r. 2022 zo Žalmu 37,39 sme dostali od Hospodina uistenie, že v~Ňom máme útočište v~čase súženia.


\autor{Daniel Mikletič}

\clanok{Technika}
Technika v~dnešných covid časoch je dôležitou súčasťou v~našich životoch, pomáha nám preklenúť nemožnosť stretávania sa. Týka sa to aj zborového života. Kedysi doplnková služba online prenosu našich bohoslužieb sa stala dôležitejšou. V~minulom roku prešla búrlivým vývojom, mnohými zmenami a hľadaním optimálneho riešenia rôznych technických výziev. Podstatným bodom bolo aj vytvorenie novovznikajúceho tímu, ktorý zabezpečuje online prenosy.

Jedným zveľkých problémov bolo zabezpečenie spoľahlivosti prenosu, ktorý spočiatku bol riešený prostredníctvom mobilných sietí. Od mája sa nám podarilo zriadiť prípojku optického internetu. Vďaka nej máme spoľahlivé internetové pripojenie s~dostatočným dátovým tokom a pevnou IP adresou. 
Internetová prípojka nám umožnila premiestnenie zborového serveru na nové miesto. Server bol nanovo nakonfigurovaný, vrátane výmeny HDD. Bol zabezpečený záložným UPS zdrojom, z~ktorého je napájaná aj celá jeho infraštruktúra. Výsledkom zmien je jeho dostupnosť 24/365. Momentálne je server využívaný ako centrálne zariadenie na ukladanie všetkej elektronickej dokumentácie zboru. Najviac ho využíva administratíva zboru a technický tím.

Najväčšou investíciou minulého roku bolo zakúpenie dvoch PTZ kamier, ktoré priniesli kvalitatívne zlepšenie obrazu a umožnili nám umiestniť kamery na pozične výhodnejšie miesta, bez okolitého rušenia prostredím. Prínosom je aj výrazné zníženie kabeláže a eliminovanie potreby rôznych prevodníkov, čo zvýšilo spoľahlivosť celého systému. Celé riešenie je postavené na NDI protokole a ethernetovej komunikácii vrátane obrazovej zložky, napájania a ovládania. 
 
V poslednom štvrťroku sa podarilo stabilizovať štandardy pre video systém a mohli sme rozšíriť tím o~službu v~tejto novej oblasti. Náš tím sa rozrástol o~nových mladých bratov, a to Martina Hovorku, Mateja Maďara a Benjamína Maďara. So zostrihom zhromaždení nám pomáha brat Slávo Kráľ. Pri projekcii sa striedajú bratia Ivan Kohút, Martin Pribula a Dávid Pribula, pri audiu Daniel Plett, Marcel Maďar, Štefan Synovec a Peter Pribula ml. Umožnili sme našim bratom z~Ukrajiny využívať všetky naše zariadenia, kde s~nimi pracujú Rustam, Andrej a David.

Sme vďační Bohu, že sme moli v~minulom roku urobiť tieto technologické zmeny, ktoré nám otvárajú nové možnosti, ako prenášať Božie slovo aj na iné miesta ako je naše zhromaždenie. Budeme veľmi radi, keď budete s~nami zdieľať akékoľvek pripomienky, ale tiež návrhy a nápady. Môžete tak urobiť aj prostredníctvom e-mailu \email{technika@bjbpalisady.sk}, ktorý je našim komunikačným kanálom.

Záverom možno ešte pohľad do budúcna. Radi by sme dokončiť chýbajúcu kabeláž, pridali do sály ruchové mikrofóny, rozšírili audio linky vybudovaním nového reku na pódiu, vymenili streamovacieho PC, zriadili bezpečnostnú kameru vo vstupnom vestibule... Potrebovali by sme si nájsť čas aj na vzdelávanie sa, aby sme sa technické možnosti naučili využívať naplno, ku všeobecnej spokojnosti nás všetkých.

\autor{Ľubomír Kešjar}


\clanok{Biblické a iné vzdelávanie}
Správa o~biblickom a inom vzdelávaní obsahuje len vzdelávanie formou stretnutí na biblických hodinách organizovaných v~utorky popoludní na Zrínskeho ulici, najmä pre seniorov pod vedením br. kazateľa Pavla Pivku a vo štvrtok večer na Palisádach pod mojím vedením. Ostatné formy biblického vzdelávania organizované jednotlivými zložkami zboru môžu byť zahrnuté v~správach za zložky, alebo v~správe kazateľa zboru. V~tejto správe nie sú zahrnuté ani vzdelávacie aktivity na nadzborovej úrovni, ktorých sa zúčastnili členovia nášho zboru a ani vzdelávanie v~skupinkách.

Spoločné štúdium Svätého Písma bolo celý uplynulý rok, tak ako väčšinu predchádzajúceho roku, ovplyvnené možnosťami stretávania sa v~rámci pandemických opatrení a dvakrát kvôli tomu muselo byť prerušené. Online vzdelávanie sme vzhľadom na charakter nášho vzdelávania na biblických hodinách a zloženie účastníkov a aj kapacitné možnosti vedúcich biblického vzdelávania nezrealizovali.

Vo štvrtky sme po ukončení lock downu a uvoľnení opatrení od~22.~4. do~24.~6., pokračovali v~preberaní Kázne na hore. Svoje stretnutia sme presunuli zo Zrínskeho ulice do modlitebne na Palisádach, aby sme splnili kritériá týkajúce sa veľkosti priestoru pre počet účastníkov. Po prázdninách sme od~7.~10. do~18.~11., keď začalo ďalšie uzatvorenie, pokračovali v~preberaní Kázne na hore a dokončili sme preberanie blahoslavenstiev a začali s~preberaním ďalších častí. Vzdelávania vo štvrtky sa zúčastňovalo v~priemere 10 až 12 členov a priateľov nášho zboru.

Paralelne pokračovalo v~utorky podľa možností aj v~roku 2021 biblické štúdium (najmä) seniorov pod vedením brata Pavla Pivku. Na týchto stretnutiach na Zrínskeho ulici sa zúčastňovalo v~priemere 8 ľudí. Bratia a sestry preberali od júna do septembra Druhý list Korintským a od septembra do novembra list Hebrejom.

\autor{Ján Szőllős}


\clanok{Sestry}
Veľmi som vďačná za to, že tieto sestry v~roku 2021 pokračovali so mnou pri vedení služby sestrám: Gitka Kráľová, Jarka Cihová, Mirka Hovorková, Barbi Antalíková, Angie Vráblová a Barborka Pribulová. V~auguste 2021 odišla Angie z~nášho tímu, nakoľko sa chcela venovať primárne dorastu, kde už pracovala.
Na jar 2021, sa kvôli pandémii sestry nemohli stretávať. Napriek tomu som robila, čo som mohla. Každý týždeň som posielala sestrám veršíky aj spolu s~krátkym komentárom, ktorý im pomáhal ich pochopiť a aplikovať. Týmto spôsobom som sa snažila naďalej povzbudzovať sestry Božím slovom a zostať v~spojení s~nimi v~čase, keď sme sa nemohli stretnúť.

Každý utorok, od septembra do júna, sme mladým mamičkám ponúkali stretnutia: Klubík na Zrínskeho 2 o~9.30~hod. Kvôli obmedzeniam sme sa v~roku 2021 mohli stretnúť len asi polovicu času. Keď sa dalo, stretávala som sa spolu s~mladými matkami na materskej dovolenke, aby sme prediskutovali úlohu manželky a rodičovstva. Okolo 10 mamičiek sa zúčastnilo v~školskom roku 2020/2021 a okolo 7 na jeseň 2021. Využívali sme knihu 365 Otázok v~duši ženy od Katherine J. Butler ako zdroj na podnietenie diskusie a rozprávali sme sa o~čomkoľvek, čo si matky vybrali z~knihy na diskusiu. Som veľmi rada, že som mala túto možnosť lepšie spoznať naše mladé mamičky zo zboru. Videla som ich hlad po poznaní Boha a jeho ciest aj ich túžbu rásť v~láske k~svojim manželom a deťom. Aj keď bolo niekedy ťažké diskutovať v~hlučnej prítomnosti detí, verím, že tieto stretnutia boli pre naše mladé mamičky užitočné a povzbudzujúce.

Tento rok konferencia Odboru sestier BJB v~ČR a SR bola online a uskutočnila sa 19.~6.~2021. Danka a Pavel Hanesovci boli rečníci a téma konferencie bola „Jedno potrebné v~21.storočí?“.

Po letných prázdninách sme v~tíme so ženami dúfali, že sa uskutoční dlho plánovaná víkendovka pre sestry v~Častej. Plán bol na víkend 8.~--~10.~október~2021. Vzhľadom na zhoršujúcu sa pandemickú situáciu sme sa rozhodli víkendovku v~Častej zrušiť, ale pripravili sme alternatívu plánovaných prednášok v~modlitebni na Palisádach. Danka Paštrnáková, ktorá bola našou hlavnou prednášajúcou, súhlasila, že bude prichádzať na Palisády a urobí 3 rôzne stretnutia pre naše sestry.

Naplánovali sme aj stretnutie sestier na Palisádach v~stredu každý druhý týždeň o~17.30 hod., kde sme sa chceli venovať štúdiu knihy „Úvahy o~kázni na hore“ od známeho anglického kazateľa Martyna Lloyd Jonesa.

Nakoniec sme sa so sestrami stretli 4x od začiatku septembra do októbra 2021, napriek covidovej situácii. Na jeseň sme sesterské stretnutia začali návštevou sestry misionárky Qamar Titus z~Pakistanu. Boli sme povzbudené jej osobným svedectvom a rozprávaním o~jej službe ženám v~Pakistane.

V sobotu 9.~10. Danka Paštrnáková spolu so svojou dcérou Maruškou hovorili o~vzťahu medzi matkou a dcérou. Stretnutie sa uskutočnilo od~9.00 do~11.30~hod. Bolo to nádherné stretnutie. Všetky sestry, ktoré prišli, boli témou veľmi požehnané. Počuli sme praktické a inšpiratívne spôsoby ako prehĺbiť vzťahy so svojimi matkami a so svojimi dcérami, s~Božou láskou k~nám a Jeho prijatím nás ako zdrojom posilnenia. Na toto stretnutie boli pozvané všetky sestry v~zbore spolu s~ich dcérami, nevestami, mamami a svokrami, aj ak nie sú z~nášho zboru.

V~stredu 20.~10. sme sa na stretnutí sestier začali spolu venovať knihe Úvahy o~kázni na hore od Martyna Lloyd - Jonesa. Ide o~kázne autora na 5. kapitolu Matúšovho evanjelia, v~ktorých sú podrobne vysvetlené blahoslavenstvá. Hlavný dôraz je byť ponorený v~Kristovi, a tak zažívať pravé šťastie. Na tomto stretnutí sme boli aj veľmi požehnané svedectvom Hely Hlubockej. Podelila sa o~tom, čo ju Boh naučil ohľadom prvého blahoslavenstva (blahoslavení chudobní duchom) počas jej pôsobenia ako asistentky Mareka Krajčího. Hovorili sme aj o~3. kapitole z~knihy Úvahy o~kázni na Hore, a myslím si, že sme dostali nový pohľad na Blahoslavenstvá a pochopili sme, prečo „pri štúdiu Písma by sme sa mali vždy riadiť pravidlom, že skôr ako začneme venovať pozornosť jednotlivostiam, treba sa zamerať na celok,“ ako autor Martin Lloyd-Jones napísal na strane 32. Potom sme sa rozdelili do skupiniek a diskutovali o~kapitole 4: „Blahoslavení chudobní duchom.“
Na stretnutí v~stredu 3.~11.~2021 sme sa venovali blahoslavenstvám aj Svetovému dňu modlitieb baptistických žien. Venovali sme sa úvodu k~blahoslavenstvám (3.~kapitola) aj kapitole 5: „Blahoslavení žalostiaci“. Prednášala som o~kapitolách 3-5 a v~skupinkách sme diskutovali o~tretej kapitole. Oslávili sme Svetový deň modlitieb, radovali sme sa z~našej jednoty i rozmanitosti a hovorili sme o~tom, ako žiť odvážny život. V~modlitbách sme sa spojili so sestrami zo 147 krajín, ktoré sú združené v~239 národných organizáciách a siedmych kontinentálnych úniách. Zbierka v~tento deň bol zdrojom príjmov Odboru sestier Svetovej baptistickej aliancie a naše dary podporili prácu sestier po celom svete.

Ostatné témy, ktoré sme mali v~pláne preberať na víkendovke s~Dankou Paštrnákovou, sme museli presunúť až na budúci rok, až do vtedy, keď sa pandemická situácia upokojí.
Som vďačná, že sme na jeseň mali znovu aspoň tieto príležitosti spolu študovať Božie Slovo na stretnutiach, keď sme sa po prednáške rozdelili do diskusných skupiniek. Bola som povzbudená tým, keď som videla, že Boh ďalej pomáha našim sestrám, aby sa otvárali a zdieľali o~sebe, o~svojich zápasoch a o~tom, čo ich Boh učí, a že v~skupinkách sa povzbudzovali navzájom vo viere.

\autor{Clara Jones}


\clanok{Mládež}
Rok 2021 bol pre našu mládež veľmi zvláštny. Svojim charakterom sa nepodobal na žiaden iný rok, aký sme doposiaľ prežili. Po približne štyroch mesiacoch tvrdého lockdownu a po necelých šiestich mesiacoch od poslednej mládeže, sme sa 1.~mája~2021 prvýkrát stretli na Palisádach. Bolo to veľmi zvláštne stretnutie hlavne v~tom, že pre mnohých z~nás to bolo po prvýkrát, čo sme boli v~spoločnosti viacerých ľudí. Preto bola stále medzi nami cítiť určitá opatrnosť a obozretnosť. Zároveň sme nedokázali odhadnúť, ako mladí zvládli lockdown, online vyučovanie a obmedzený kontakt s~priateľmi, nedokázali sme odhadnúť, či mladí stále túžia po duchovnom spoločenstve svojich rovesníkov. Preto ako výbor sme sa museli pripraviť na rôzne scenáre, napríklad aj na taký, že na prvú mládež prídu iba ľudia z~výboru. Na naše prekvapenie na prvú mládež prišiel veľký počet mladých, a to požehnanie, ktoré sme mohli prijať z~osobných rozhovorov, nás motivovalo a zároveň utvrdilo, že aj v~dnešnej dobe mladí ľudia túžia po živom spoločenstve. Preto sme ďalej pokračovali v~organizovaní mládeží, snažili sme sa rozprávať o~témach, ktoré nás v~našich životoch hlboko zasiahli a snažili sme sa budovať hlboké a úprimné priateľstvá, pretože sme verili, že jedine vďaka nim dokážeme budovať „novú mládež“.

Počas leta sa nám vyskytla príležitosť spolupodieľať sa na organizovaní dorasteneckého tábora. Bol to veľmi zaujímavý čas, ktorý sme strávili či už s~dorastencami alebo s~vedúcimi dorastu. Z~mládeže sme tam boli štyria „chalani“, ktorí sme pomáhali či už pri organizačnej stránke tábora alebo potom ako vedúci jednotlivých tímov. Dorastenecký tábor bol veľmi zaujímavý aj v~tom, že ako mládežníci sme prišli dva dni pred začiatkom tábora a spolu s~Dannym sme trávili čas pri vode, kde sme sa učili jazdiť na paddleboarde, alebo sme na vlastnej koži zažili, čo obnáša eticky usmrtiť, spracovať a nakoniec opiecť králika a kohúta. Hlavne pre nás, bratislavských chlapcov, to bol nezabudnuteľný zážitok. Napriek tomu, že sme do celej služby išli s~množstvom otázok, na ktoré sme nemali odpovede, tak na konci tábora, keď sme mali možnosť ho zhodnotiť, tak sme boli zaskočení zvláštnym pocitom, že Boh zasial v~našom zbore niečo veľké, niečo, čo túžime zažívať každé leto.

Začal nový školský rok a prichádzala tretia vlna - delta variant. Nedokázali sme predpokladať, ako budú vyzerať nasledujúce mesiace, či nám hrozí lockdown, alebo nie a vlastne, koľko mladých bude ochotných chodievať na mládeže. Avšak znova sme boli zaskočení tou túžbou mladých byť súčasťou spoločenstva. Rozhodli sme sa nemrhať našimi obmedzenými zdrojmi a určili sme si priority, ktoré definovali, ako budú vyzerať ďalšie mládeže. Verili sme a naďalej sme presvedčení o~tom, že je dobré, aby na mládeži zaznelo Božie slovo, preto jedna z~našich priorít bola, aby na každej mládeži zaznela aspoň krátka a výstižná myšlienka, ktorá reflektovala na to, kto je Boh a ako ho prakticky nájsť v~našom každodennom živote. Zároveň ako výbor už roky túžime po mládeži, ktorá bude plná úprimných priateľstiev a rozhovorov, kde mladí bez akýchkoľvek pochybností môžu úprimne zdieľať svoje každodenné prežívanie života a viery. Snažili sme sa to docieliť rôznymi spôsobmi, ktoré boli viac či menej účinné. Avšak sme usúdili, že jediný spôsob, ako to docieliť je, že my musíme vytvárať ten priestor, my musíme byť prví, kto bude úprimní a zraniteľní pred mládežníkmi, skrátka stať sa tým „chodiacim evanjeliom“, dôkazom toho, že život nikdy nebol ani nebude ľahký, že život je plný sklamaní a pádov a práve preto sme prijali tú „Dobrú Správu“ (lat. evanjelium), ktorá nás každý deň premieňa v~dokonalý Ježišov obraz. 
Bolo pre nás dôležité, aby sa mládežníci cítili prijatí takí, akí sú bezo zvyšku, bez akéhokoľvek tlaku, že sa musia aktívne podieľať na fungovaní mládeže, čo prakticky znamenalo, že prvé mesiace bola jediným oficiálnym bodom mládeže téma. Po nejakom čase nás proaktívne začali oslovovať viacerí mladí s~tým, že sa natoľko cítia na mládeži „ako doma“, že túžia byť jej aktívnou súčasťou. Aj toto bola jedna z~mnoha maličkostí, čím nás Boh absolútne zaskočil. 

Čo sa týka našich mladých, Boh nás tento rok obdaroval novými ľuďmi, ktorí sa stali našou súčasťou, či už skrze dorast, alebo vďaka našim mladým, ktorí aktívne svedčia evanjelium medzi svojimi spolužiakmi, alebo vďaka Dannymu, ktorý pozýva na mládež mladých, ktorí sa objavia prvýkrát na Palisádach. Sú pre nás obrovským prínosom a požehnaním. Pre výbor je bytostne dôležité, aby sa cítili prijatí a túžime, aby mládež bola pre nich rovnakým domovom, ako pre nás ostatných. 
Zároveň koncom októbra sa v~našom zbore konal krst a veľmi milo nás prekvapilo, že práve mladí tvorili najväčší podiel z~tých, ktorí sa verejne rozhodli vyznať Ježiša ako svojho osobného Spasiteľa. Ako výbor nás to veľmi teší a túžime, aby sa mládež stala pre nich miestom, kde môžu duchovne načerpať.

Prosíme vás, modlite za našich mladých, majú za sebou náročný rok plný neistoty a izolácie. Majú pochopiteľné obavy, že im necelé dva roky pretiekli pomedzi prsty a mnohí z~ich sú poznačení toxickým prostredím sociálnych sietí. Myslite na nich, nech napriek všetkým vonkajším tlakom a nereálnym očakávaniam, ktoré sú na nich kladené, dokázali upriamovať pozornosť na to, čo je v~živote podstatné. Zároveň vás prosíme o~modlitby za nových ľudí, ktorým srdce horí pre službu s~mládežou.

\autor{Dávid Pribula / Radovan Paulen}


\clanok{Dorast}
Som vďačný Bohu, že napriek všetkým opatreniam a obmedzeniam sme sa mohli celý rok 2021 pravidelne stretávať či online alebo osobne.

Od začiatku roka až do apríla sme pokračovali v~online stretávaní vzhľadom na pretrvávajúce protiepidemické opatrenia. V~tomto období sme spolu študovali list Jakuba. Osobné stretnutia sme obnovili koncom apríla, najprv sme sa stretávali v~modlitebni na Palisádach a po mesiaci sme sa presunuli na Súľovskú 2. Začali sme študovať životy judských kráľov z~druhej knihy Kronickej (od Rechabeáma a do leta sme stihli prísť po Joráma).

V~poslednom júlovom týždni sme sa zúčastnili na dorasteneckom tábore na Chvojnici. Ráno sme si čítali príbehy ľudí z~evanjelií, ktorí sa stretli s~Pánom Ježišom, a rozprávali sme sa na o~tom, ako zareagovali na toto stretnutie. Doobeda po rannej téme a poobede sme sa hrali hry. Večer pri táboráku sme sa venovali vzťahovým témam (budovanie postojov -- dôležitosť charakteru, rôzne druhy lásky, muž a žena -- odlišnosti, čo sú ich úlohy). Som vďačný Bohu za túto príležitosť viac sa vzájomne spoznať a budovať spoločenstvo.

Od začiatku školského roka sme obnovili stretávanie dorastu na Súľovskej ulici. Pokračovali sme v~štúdiu životov judských kráľov (až po Acháza). V~druhej polovici novembra sme boli opäť nútení presunúť sa do online priestoru. Vzhľadom na čas adventu sme sa rozprávali o~tom, čo hovorí Božie Slovo o~nádeji, pokoji, radosti a láske.

Minulý rok sme stretnutia dorastu viedli v~zložení manželia Vráblovci, Martin Simon, Janko Kováčik a Rado Nemec. Počas celého roka sme spolupracovali s~Dannym Jonesom. Na dorastoch sa pravidelne zúčastňovali Tamarka a Marek Syčovci, Matej a Benko Maďarovci, Daniel a Lenka Vráblovci, Davyd a Iľja Potockí, Emmka Čonková, Radko Nemec, Dara Plett, Damián a Diana Mikolášovci.

Našim spoločným cieľom je viesť dorastencov k~poznaniu, že potrebujú Pána Ježiša, len On je našou jedinou nádejou, a modlíme sa, aby On urobil to, čo žiaden človek nedokáže, premenil ich životy. Prostredníctvom Božieho slova sa im snažíme komunikovať základné pravdy a tiež preberáme témy, ktoré sú pre ich momentálny vek aktuálne. Prostredníctvom hier, výletov a táborov chceme vytvoriť priestor pre vznik priateľských vzťahov, lebo si uvedomujeme dôležitosť spoločenstva.

Prosím, modlite sa za našich dorastencov, za zmenu ich života, za rast v~poznaní Pána Ježiša, za priateľstvo medzi dorastencami -- aby neboli len také osamelé „ostrovy“, ktoré sa raz do týždňa stretnú a tiež za tých dorastencov, ktorí sa z~rôznych dôvodov na našich stretnutiach a akciách nezúčastňujú. Ďakujem za vaše modlitby a verím, že Pán Boh má svoj dobrý a múdry plán pre životy našich dorastencov.

\autor{Radislav Nemec}


\clanok{Besiedka}
Rok 2021 bol 2. pandemickým rokom. Nakoľko situácia bola počas celého roka nestabilná, stretávali sme sa sporadicky. Na začiatku roka to bolo niekoľko virtuálnych stretnutí cez videá a záznamy bohoslužieb. Ľubka Kováčiková sa pokúsila aj o~online besiedky, ale pripájalo sa na ne len málo detí. V~júni sa niekoľkokrát stretla veľká besiedka (deti od 7 do 11 rokov). Po prázdninách sme rozbehli stretávanie v~malej (deti od~3 do~7 rokov) i veľkej besiedke. Detí nechodilo veľa, nakoľko bohoslužby prebiehali v~súlade s~platnými nariadeniami, takže sa ich nemohli zúčastňovať všetky rodiny. Besiedky pokračovali do novembra, kedy bol vyhlásený lockdown a prezenčné bohoslužby boli zrušené. Ani tento rok sa nám nepodarilo pripraviť vianočný program, na ktorý sa tešia všetky vekové kategórie. Veríme, že rok 2022 bude priaznivejší.
 
V~malej besiedke sme od septembra preberali Starý zákon. Začali sme prvým hriechom a dostali sme sa k~Dávidovi. Vo veľkej besiedke využívali párty balíčky z~Detskej misie, ktoré sú pripravené ako jednolekciovky na rôzne témy. Učiteľky sa s~deťmi veľa rozprávali o~modlitbe a nádeji, lebo cítili, že deti to potrebujú.

Žiaľ, pandémia prináša so sebou rôzne negatívne dôsledky, ktorých dopad si možno uvedomíme až neskôr. Jedným z~takýchto dôsledkov je skutočnosť, že viaceré deti hoci už dosiahli „besiedkársky“ vek, na besiedke ešte neboli alebo sa jej zúčastnili len minimálne. Nevedia, čo je to besiedka, a nechcú ísť do prostredia, ktoré nepoznajú. Verím, že rodičia nájdu vhodný spôsob ako deti povzbudiť a besiedka pre ne nebude „strašiakom“. Uvedomujem si však, že návrat do zabehaných koľají, aké sme poznali pred koronou, bude trvať dlho… Modlím sa, aby sme sa čoskoro mohli slobodne stretávať a aby sa návšteva zhromaždenia i besiedky stala pre malých i veľkých neoddeliteľnou súčasťou nedele.
\begitems
* Malá besiedka v~r. 2021 -- 13 detí 
* Veľká besiedka v~r. 2021 -- 8 detí 
* Učiteľky v~malej besiedke: Miriam Kešjarová, Kika Kešjarová, Mirka Hovorková, pomocníci: Katka Kerekréty, Tamarka Syčová, Martin Hovorka
* Učiteľky vo veľkej besiedke: Ľubka Kováčiková, Baka Pribulová, Kvetka Maďarová, Slávka Volentičová
\enditems

\autor{Miriam Kešjarová}


\clanok{Spevokol}
Kým v~predchádzajúcom roku 2020 nás prekvapilo zamorenie nebezpečným koronavírusom a dve tretiny roka sme ostali bez možnosti sa stretávať s~našim spevokolom a spievať na Božiu slávu, minulý rok to bolo ešte horšie. Celý rok sme boli odsúdení na totálny deficit zborového spevu.     
Počas rokov hojnosti sme si natočili videá z~našich koncertov, a tak sme mohli niektoré piesne postupne vyťahovať z~archívu a pripomínať si tie sväté chvíle, ktoré sme mohli prežívať z~Božej milosti. 

Niekoľkokrát som sa pristihol, ako pri sledovaní týchto videí mi tiekli po lícach slzy. Boli to slzy vďaky a súčasne bezmocnosti.
Keď som mal možnosť hovoriť s~našimi spevákmi, mnoho z~nich prežívalo podobné chvíle. Dokonca som sa stretol s~jednotlivcami, ktorí robili pokánie, lebo nedostatočne a nevďačne využívali tieto možnosti. 

Podľa optimistických predpovedí by tento rok už covid nemal byť pre nás tak nebezpečný a zas by sme sa mohli začať stretávať a začať cvičiť hlasivky. Zaujímavé je, že kým väčšina spevákov sa už nevie tejto služby dočkať, nájdu sa aj takí, ktorí sú presvedčení, že už takéto teleso dokopy nedáme. 

Sme však veriaci a keďže vieme, že sme v~Božích rukách a u~Boha nič nie je nemožné, tak chceme veriť a dúfať, že zborový spev ešte bude znieť nielen v~našom zhromaždení, ale aj na verejnosti.  
Vieme, že bez Božieho súhlasu sa nám ani vlas na hlave neskriví, tak všetko prijímame s~pokorou, vďakou a v~snahe rozpoznať dôvod, prečo k~nám Stvoriteľ takto prehovoril. 
Preto do nového roka sme prešli na modlitbách s~nádejou opätovného rozbehnutia tejto služby na oslavu nášho Spasiteľa. Ak to bude Jeho vôľa.

\autor{Slávo Kráľ}


\clanok{Služba ľuďom v~núdzi}
V~roku 2021, počas pandémie COVID-19, sme sa, čo do počtu síce v~menšej miere, ale predsa zapojili do služby ľuďom v~núdzi a to hlavne varením polievok a ich výdajom. Služba prebiehala pod mostom Lafranconi bez prestávky tak ako po minulé roky, v~letnom období 2x do týždňa -- utorky a štvrtky a v~zimnom období 3x do týždňa -- k~utorokom a štvrtkom sa pridali aj soboty. Spolu s~polievkou sa naďalej vydávali aj nápoje a šatstvo a poskytovalo sa ošetrenie.

Vo výdajovom tíme pri výdaji polievky a nápojov sa minulý rok opäť aktívne zapájala Zuzka Pařízková. Do služby varenia sa zapojili: Slávka Volentičová (9x), Laurenčíkovci (1x), Kohútovci (1x) a Hanka Šandorová (1x). 

Vďaka patrí všetkým Vám, ktorí ste napriek pandemickej situácii boli ochotní poslúžiť núdznym uvarením polievky, jej výdajom, financiami alebo modlitbami. Vďaka Pánu Bohu za Vás! 

\autor{Beata Bogárová}


\clanok{Connect}
V~roku 2021 služba zakladania nového zboru Connect prechádzala podobnými úskaliami, akým čelili aj ostatné  podobné služby. Najvážnejšou limitáciou našej misie boli veľmi obmedzené možnosti na trávenie času s~ľuďmi, čo je jeden z~kľúčových pilierov evanjelizácie. 

Všetky spoločné stretnutia (Connect nedele, diskusie, stretnutia tímu) sme v~tom čase preniesli do online priestoru. Kontakty s~ľuďmi sme udržiavali cez sociálne siete a telefonovanie. Praktickú pomoc sme riešili iba v~naliehavých prípadoch. Ale aj v~tomto čase sme sa snažili poskytovať si navzájom vedomie vzájomného prijímania a podpory.

Keď sa situácia uvoľnila a boli sme zase spolu, potešilo nás, že na Connect nedele začali prichádzať najmä rodiny s~deťmi. Preto sme priebeh Connect nedieľ prispôsobili rodinám. Koncom roka náš tím posilnil praktikant Filip, ktorý časť svojej praxe bude vykonávať v~Connecte.

Stále sa modlíme o~spásu ľudí, tiež aby Boh vysielal pracovníkov na svoju žatvu a aktuálne za muzikantov a spevákov, ktorí by posilnili službu hudbou a spevom na Connect nedeliach. 

\autor{Tomáš Valchář}


\clanok{Revízia hospodárenia}
Revízna komisia v~zložení Miroslav Antalík, Helena Mikletičová, Barbora Antalíková za spolupráce účtovníčky zboru Ľubomíry Kohútovej vykonali revíziu hospodárenia za rok 2021.

Boli prekontrolované nasledovné doklady:
\begitems \style -
* výpisy z~bežného účtu vedeného v~Slovenskej sporiteľni za mesiace 1, 4, 5, 7, 9 a 11
* výdavkové pokladničné doklady za mesiace 1 -- 12
* príjmové pokladničné doklady za mesiace 1 -- 12
\enditems
Revízna komisia konštatuje, že uvedené doklady sú vedené prehľadne v~súlade s~účtovnými predpismi. Pokladničná kniha je vedená mesačne a založená priamo pri pokladničných dokladoch.

Neboli zistené žiadne nedostatky.

Stav finančnej hotovosti ku dňu 31.~12.~2021 bol:

\vskip1em\hskip1cm\table{lr}{
pokladňa & 2~987,71~€ \cr
bankový účet & 66~273,16~€ \crl
spolu & 69~260,87~€ \cr
}\vskip1em

Tento stav súhlasí so stavom v~účtovnej evidencii k~uvedenému dátumu.


Revízna komisia konštatuje zvýšenie obetavosti zboru, ktoré sa prejavilo vyššími príjmami zo zbierok a darov.

\autor{Miroslav Antalík}

\tiraz
\bye
