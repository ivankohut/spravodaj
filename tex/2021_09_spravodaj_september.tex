%\typosize[10/12.5]% - pouzita velkost pisma/riadku - trochu vacsie
\input makra.tex % nacitanie Ivanom pripravenych nastaveni a prikazov
\hyphenation{star-šov-stvo} % rozdelenie slov na konci riadku, treba tu uviest slova, ktore sam nepozna

\spravodaj{9}{2021}


\clanok {Ten bol s~Ježišom!}
{\it „Lebo ich poznali, že boli s~Ježišom“ (Sk 4, 13b).}

Kresťan by sa mal nápadne podobať na Ježiša Krista. Čítame krásne a podrobne opísané záznamy o~Kristovom živote, ale najlepším opisom Ježišovho života je Jeho živý životopis, napísaný v~slovách a skutkoch Jeho ľudí.

Ak by sme boli takí, ako o~sebe vyhlasujeme, a akí by sme mali byť, boli by sme zobrazením Krista. Áno, tak nápadne by sme sa naňho mali podobať, že ľudia zo sveta by nás s~Ním nemuseli každú chvíľu porovnávať a posúdiť: „Našlo by sa čosi podobného.“ Keď nás raz zbadajú, mali by zvolať: „Ten bol s~Ježišom! Učil sa od Neho, podobá sa naňho, pochopil základný postoj muža z~Nazareta, žije podľa neho a vidno to i na jeho každodenných skutkoch.“

Kresťan by sa mal na Krista podobať v~smelosti. Nikdy sa nehanbi za svoju vieru! Povolanie pre teba nikdy nebude zahanbením. Daj pozor, aby si ho ty nezahanbil. Buď ako Ježiš, konaj odvážne pre svojho Boha! Napodobňuj Jeho milujúci postoj. Premýšľaj láskavo, hovor láskavo a konaj láskavo, aby ľudia mohli o~tebe povedať: „Ten bol s~Ježišom!“ Napodobňuj Ježiša v~svätosti. Horlil za svojho Pána? Rob tak i ty! Buď vždy pripravený konať dobré. Nemárni čas, lebo je príliš drahý!

Zapieral Ježiš sám seba a nikdy nehľadel na vlastný prospech? Buď tiež taký. Trávil veľa času v~modlitbách? I~ty sa zapálene modli. Mal v~úcte Otcovu vôľu? Podvoľ sa jej aj ty. Bol trpezlivý? Tak sa uč vytrvalosti. A~nakoniec, ako dokonalý obraz Ježiša, snaž sa odpúšťať svojim nepriateľom. Nech ti v~ušiach neustále znejú nádherné slová: „Otče, odpusť im, lebo nevedia, čo činia.“ Odpúšťaj v~nádeji, že i tebe je odpustené. Zhrň žeravé uhlie na hlavu svojich nepriateľov tým, že k~nim budeš láskavý. Spomeň si, že nám Boh hovorí, aby sme sa dobrým odplácali za zlé. Napodobňuj Ho teda vo všetkých ohľadoch a ži tak, aby o~tebe hovorili: „Ten bol s~Ježišom!“

\autor{Charles Spurgeon}


\clanok {Správy zo staršovstva}
Napriek dovolenkovému obdobiu sa staršovstvo zboru rozhodlo uskutočniť v~júli svoje mimoriadne stretnutie, a to po návrate brata kazateľa Dannyho Jonesa s~manželkou Clarou z~ich dovolenky v~USA. Na tomto stretnutí sme sa okrem bežnej rutinnej agendy venovali aj otázke multifunkčnosti priestorov vo vlastníctve zboru z~perspektívy budúcnosti, ďalej stavu zborových aktivít, takisto pastoračnými potrebami v~zbore, potom výsledkami a výstupmi KDZ, no primárne sme sa zaoberali dlho plánovaným výročným zborovým členským zhromaždením, ktoré z~dôvodu dovtedy platných Covid opatrení nebolo možné od začiatku roku zorganizovať skôr, ako práve v~letnom dovolenkovom období. Dôležitosť uskutočniť v~čo najskoršom možnom termíne VZČZ bola umocnená nielen rozpočtovým provizóriom nášho zboru, ale najmä našou obavou o~možný rýchly zvrat pandemickej situácie v~neprospech hromadných stretnutí. Sme preto vďační predovšetkým nášmu Pánovi, ale aj všetkým vám zúčastneným, že sa nakoniec VZČZ mohlo v~nedeľu odpoludnia 1.~8.~2021 napriek prázdninovému obdobiu aj uskutočniť, kde sme sa okrem prijímania nových členov do zboru mohli oboznámiť aj s~upraveným znením stanoviska staršovstva k~uskutočňovaniu sobášov v~našom zbore a takisto sme mohli odštartovať nemenej dôležité schvaľovanie novelizácie zborového a volebného poriadku, reflektujúce najmä uľahčenie fungovania zboru v~mimoriadnych časoch.

V mene staršovstva chcem aj týmto povzbudiť všetkých členov a priateľov nášho zboru k~modlitbám za upevnenie našej viery v~Pána Ježiša Krista v~týchto zložitých časoch, ako aj k~prosbám za posilnenie spolupatričnosti k~našej zborovej rodine.

\autor {Miroslav Kolářik}


\clanok {Zborový deň v~Bernolákove}
V nedeľu 12. septembra sa po zhromaždení na Palisádach celý zbor presunieme do Bernolákova, kde v~zborových priestoroch v~exteriéri strávime spoločný čas pri jedle a vzájomnom spoločenstve. Začiatok obeda plánujeme okolo 12.30~hod., aby bol čas zastaviť sa doma po jedlo. Bolo by dobré, keby každý priniesol aj porciu navyše a mohol sa podeliť s~ostatnými. Bude pripravený aj gril, keby si chcel niekto niečo ugrilovať.
\vfill\break


\clanok {Besiedka, dorast mládež}
Prvé stretnutie besiedky bude v~nedeľu 12.~septembra o~9.30~hod. na Zrínskeho~2 (súbežne s~dopoludňajším zhromaždením na Palisádach). Prosíme rodičov, aby svoje deti priviedli rovno na besiedku.

Dorastenci sa v~novom školskom roku budú stretávať každý piatok o~17.30~hod. na Súľovskej 2 (približne do 19.00~hod.). Prvé stretnutie sa uskutoční 10.~septembra. Prosíme dorastencov, aby si vždy nosili Biblie a rúška resp. respirátory.

Mládež sa prvýkrát stretne 11.~septembra o~18.00~hod. na Súľovskej~2.


\clanok {Biblické hodiny}
Biblické hodiny pre seniorov budú každý utorok o~15.15~hod. na Zrínskeho~2. Prvé stretnutie v~novom školskom roku bude už 7.~septembra.

Štvrtkové biblické hodiny začnú až v~októbri.


\clanok {Klubík}
Prvý stretnutie Klubíka pre mamičky bude 14.~septembra. Stretnutia budú prebiehať každý utorok od~9.30 do~11.00~hod. Clara Jones bude na nich viesť diskusiu na témy, ktoré sa týkajú matiek. Vítané sú všetky mamičky s~malými deťmi.


\clanok {Sesterské stretnutia}
Prvé stretnutie sa uskutoční vo štvrtok 23.~septembra o~17.30~hod. a plánujeme ho začať návštevou misionárky z~Pakistanu. Príďte, aby ste mali spoločenstvo so svojimi sestrami a vypočuli si úžasné príbehy o~tom, čo Boh robí v~životoch žien v~Pakistane.

Ďalšie plánované sesterské stretnutia na jeseň budú každý druhý týždeň o~17.30~hod. v~stredu v~týchto termínoch: 20.~10., 3.~11., 17.~11., 1.~12. a 15.~12.
\vfill\break


\clanok {Víkendovka pre sestry}
Víkendovka pre sestry bude v~Častej-Papierničke 8. -- 10.~októbra (začneme v~piatok večerou a skončíme v~nedeľu obedom). Je to víkendovka, ktorú sme plánovali ešte minulý rok, ale sme ju museli zrušiť kvôli pandémii. Určená je pre všetky sestry v~zbore spolu s~ich dcérami, nevestami, mamami a svokrami, aj ak nie sú z~nášho zboru. Téma bude rodinné vzťahy. Prihlasovanie sa spustí po 14.~septembri. Tešíme sa na Danku Paštrnákovú z~Liptovského Hrádku, ktorá bude našou rečníčkou. Danka je matkou piatich detí a babičkou ôsmich vnúčat. Venuje sa ženám na ženských skupinkách a spolu s~manželom aj manželským párom.


\clanok {Víkendovka na Chvojnici pre mužov so synmi}
V dňoch 1. -- 2.~októbra bude víkendovka na Chvojnici pre mužov a ich synov spojená s~brigádou. Okrem toho, že urobíme niečo užitočné, je to výborná príležitosť stráviť spolu dva dni a duchovne sa občerstviť. Záujemcovia sa môžu zapísať do tejto tabuľky: \ulink[https://bit.ly/3sLukBu]{bit.ly/3sLukBu}.


\clanok {Konferencia pre seniorov BJB Slovenska a Česka}
Konferencia pre seniorov BJB Slovenska a Česka sa bude konať v~Račkovej doline v~termíne 15. -- 18.~septembra. Všetky potrebné informácie nájdete na stránke: \ulink[https://baptist.sk/ts]{baptist.sk/ts}.
\vfill\break


\clanok {Krst v~zbore}
Chceme osloviť všetkých, ktorí by sa chceli dať pokrstiť na vyznanie viery, aby sa prihlásili u~br. kazateľa Dannyho Jonesa alebo ktoréhokoľvek staršieho zboru.


\clanok{Verš na zapamätanie}
Tento mesiac máme nový veršík, ktorý sa chceme spoločne učiť. Veríme, že poznanie Písma prospeje našej duši i našej mysli:

{\it „Jeho božská moc nám darovala všetko potrebné pre život a nábožnosť, keď sme poznali toho, ktorý nás povolal vlastnou slávou a účinnou mocou. Tým nám daroval vzácne a veľmi veľké prisľúbenia, aby ste prostredníctvom nich mali účasť na Božej prirodzenosti a unikli skaze, ktorú vo svete spôsobuje žiadostivosť.“}

\autor{2Pt~1,~3--4}


\clanok{Zbierky za uplynulé obdobie}
Milí bratia a sestry,

v letných mesiacoch júl a august ste prispeli:

\vskip-1ex\begitems
* Misia: 905,00 €
* Investície: 905,00 €
\enditems

Aj naďalej máte možnosť prispieť do „nedeľnej zbierky“, a to prevodom na účet zboru. Do poznámky pre prijímateľa, prosím, uveďte „zbierka“.

Bankové spojenie: SK36 0900 0000 0000 1147 1836, SWIFT: GIBASKBX

Ďakujeme!


\n 2.	9.	Radislav	NEMEC;
\n 5.	9.	Dušan	UHRIN;
\n 9.	9.	Daniel	VALENTA;
\n 14.	9.	Štefan	SYNOVEC;
\n 16.	9.	Daniel	PLETT;
\n 19.	9.	Richard	HALAMÍČEK;
\n 21.	9.	Kvetoslava	MAĎAROVÁ;
\n 21.	9.	Miroslava	SIMONOVÁ;
\n 22.	9.	Viera	KOLÁROVSKÁ;
\narodeniny


\program{
\p  1 ; st ;.;;.;;
\p  2 ; št ;.;;.;;
\p  3 ; pi ;.;;.;;
\p  4 ; so ;.;;.;;
\p  5 ; ne ;  9.30 ; Bohoslužby (D. Jones) ;.;;
\p  6 ; po ;.;;.;;
\p  7 ; ut ; 15.15 ; Stretnutie pri Biblii (P. Pivka, Zrínskeho 2) ;.;;
\p  8 ; st ;  6.00 ; Modlitby -- muži (Zrínskeho 2) ;.;;
\p  9 ; št ;.;;.;;
\p 10 ; pi ; 17.30 ; Dorast (Súľovská 2) ;.;;
\p 11 ; so ; 18.00 ; Mládež (Súľovská 2);.;;
\p 12 ; ne ;  9.30 ; Bohoslužby (P. Šrankota) ; 9.30 ; Besiedka (Zrínskeho 2) ;
\p    ;    ; 12.00 ; Zborový deň (Poštová 55, Bernolákovo) ;.;;
\p 13 ; po ;.;;.;;
\p 14 ; ut ;  9.30 ; Klubík (C. Jones, Zrínskeho 2) ; 15.15 ; Stret. pri Biblii (P. Pivka, Zrín. 2) ;
\p 15 ; st ;  6.00 ; Modlitby -- muži (Zrínskeho) ;.;;
\p 16 ; št ;.;;.;;
\p 17 ; pi ; 17.30 ; Dorast (Súľovská 2) ;.;;
\p 18 ; so ; 18.00 ; Mládež (Súľovská 2) ;.;;
\p 19 ; ne ;  9.30 ; Bohoslužby (D. Jones) ; 9.30 ; Besiedka (Zrínskeho 2) ;
\p 20 ; po ;.;;.;;
\p 21 ; ut ;  9.30 ; Klubík (C. Jones, Zrínskeho 2) ; 15.15 ; Stret. pri Biblii (P. Pivka, Zrín. 2) ;
\p 22 ; st ;  6.00 ; Modlitby -- muži (Zrínskeho 2) ;.;;
\p 23 ; št ; 17.30 ; Sesterské stretnutie ;.;;
\p 24 ; pi ; 17.30 ; Dorast (Súľovská 2) ;.;;
\p 25 ; so ; 18.00 ; Mládež (Súľovská 2) ;.;;
\p 26 ; ne ;  9.30 ; Bohoslužby (P. Lizardo) ; 9.30 ; Besiedka (Zrínskeho 2) ;
\p 27 ; po ;.;;.;;
\p 28 ; ut ;  9.30 ; Klubík (C. Jones, Zrínskeho 2) ; 15.15 ; Stret. pri Biblii (P. Pivka, Zrín. 2) ;
\p 29 ; st ;  6.00 ; Modlitby -- muži (Zrínskeho 2) ;.;;
\p 30 ; št ;.;;.;;
}


\tiraz
\bye
