\input makra.tex % nacitanie Ivanom pripravenych nastaveni a prikazov
\hyphenation{star-šov-stvo} % rozdelenie slov na konci riadku, treba tu uviest slova, ktore sam nepozna

\spravodaj{5}{2018}


\clanok{Mesiac máj a Svätodušné sviatky}
Drahí bratia a sestry!

Mesiac máj je tohto roku pre nás vzácny tým, že si ako Cirkev Kristova na Svätodušné sviatky budeme pripomínať príchod či zoslanie Svätého Ducha. Písmo nám zjavuje, že hoci bol Duch Svätý určitým spôsobom činný a prítomný už v~období pred Ježišom Kristom, hoci nechýbal v~SZ poriadku spasenia, hoci bol plne prítomný a činný v~živote, službe a diele spasenia Ježiša Krista, predsa sa v~Písme hovorí, že „Duch Svätý nebol ešte daný, lebo Ježiš nebol ešte oslávený“ J 7,39. Už Písmo Starej zmluvy spájalo prichádzajúci nový vek s~novým a slávnym príchodom Mesiáša a Ducha Svätého -- Iz 11,2. Nový vek Božieho kráľovstva sa začal realizovať v~Ježišovi Kristovi, v~Jeho smrti, vzkriesení a nanebovstúpení. Takže vyliatie či príchod Ducha na Letnice znamenal príchod Božieho kráľovstva do ľudskej histórie, ktoré sa začalo už príchodom Ježiša Krista, ktorý povedal: „Hľa kráľovstvo Božie medzi vami...“ Lk 17,21. Preto Pán Ježiš povedal, že pokiaľ On neodíde, Duch Svätý nepríde -- J 16,7. Ale keď Ježiš odíde k svojmu Otcovi, pošle nám svojho Ducha -- Tešiteľa. Príchod Ducha Svätého v~plnej sile a moci bol viazaný na víťazstvo a oslávenie Ježiša Krista Sk 1,9 -- 11; 2,1 -- 12; 32 -- 36. Duch Svätý mohol naplno „roztiahnuť svoj stan“ až keď bolo dokončené spasenie v~Kristu Ježišovi. Keď Pán Ježiš dielo spasenia dokonal a posadil sa po pravici Otca, mohol prísť Duch Svätý, aby toto, v~Kristu dokonané dielo spasenia, nám ľuďom zjavoval, aby nás viedol ku krížu Golgoty, ukazoval na Krista a privlastňoval nám Jeho dielo spásy skrze vieru v~Spasiteľa Ježiša Krista. Nech sú naše srdcia preto naplnené vďakou nášmu Pánovi, že poslal zasľúbenie svojho Otca (Lk 24,49) a~že je stále prítomný skrze svojho Ducha.

\autor{Darko Kraljik}


\clanok{Správy zo staršovstva}
Ako je už dlhšiu dobu zaužívaným zvykom, stretávame sa v~dvojtýždenných intervaloch. Otázky, ktorým sa venujeme, zaberajú široké spektrum. Sú to otázky praktického života zboru, vplyv vonkajšieho prostredia na naše fungovanie, ale aj duchovné otázky nášho zboru.

V apríli sme sa venovali týmto témam:
\begitems
* program a zabezpečenie služieb v~zbore,
* pomoc pri sťahovaní brata kazateľa Janka Szőllősa,
* rekonštrukcia bytu a kancelárie na Zrínskeho a organizačno-právne záležitosti týkajúce sa bytu a kancelárie,
* webová stránka nášho zboru,
* Dannyho a Clarin príchod na Slovensko a s~tým spojené vybavovanie povolení,
* prenájom priestorov na sobáše,
* program diskusnej konferencie delegátov zborov BJB na Slovensku,
* príprava programu na Chvojnici počas sviatku zoslania Ducha Svätého,
* záujem o~členstvo v~zbore,
* idea domova pre seniorov v~Bernolákove,
* pastoračné otázky.
\enditems
Ďakujeme za vašu podporu a modlitby

\autor{S prianím Božieho požehnania Peter Pribula}


\clanok{Koncert Sama Rotmana}
Pozývame vás na klavírny recitál Sama Rotmana (USA), ktorý sa bude konať vo štvrtok {\bf 10. mája 2018 o~19.00 hod.} v~Koncertnej sieni VŠMU Dvorana na Zochovej 1. Sam Rotman je medzinárodne uznávaný klavirista s~dokonale prepracovanou technikou hry. Jeho interpretácia skladieb je pôsobivá, dynamická a expresívna. Každé Rotmanovo vystúpenie odhaľuje jeho jedinečný hudobný talent a zároveň odkrýva osobný príbeh duchovnej cesty. Program koncertu: Beethoven – Sonáta č. 1 Malá Appassionata, Sonáta č. 8 Patetická, Sonáta č. 26 Lebewohl. Vstup je voľný.


\clanok{Letnice na Chvojnici}
V nedeľu 20. 5. 2018 plánujeme výlet na našu zborovú stanicu na Chvojnici. Stretneme sa tam pri slávnostných bohoslužbách, na ktorých poslúži brat kazateľ J. Szőllős, popoludní si zaspomíname na minulosť a pozrieme sa, čo nás čaká v~najbližšej budúcnosti. Obed bude zabezpečený na zborovej chalupe. O prihlasovaní na obed a do spoločného autobusu vás budeme informovať v~najbližších dňoch.


\clanok{Rodinný tábor 2018}
Milí priatelia, aj tento rok vás pozývame na rodinný tábor BJB Palisády, ktorý sa uskutoční v~táborovom stredisku o. z. Detská misia, neďaleko obce Častá pod hradom Červený Kameň. Tábor je možnosťou lepšie sa navzájom spoznať, načerpať veľa telesného i duchovného povzbudenia a prežiť príjemný týždeň s~rodinou a priateľmi. Tábor je tiež príležitosťou slúžiť si navzájom, takže príjmeme všetky dobré nápady, inšpirácie a pomoc každého druhu.
Bližšie informácie: \email{mkesjarova@bjbpalisady.sk}.


\clanok{Senior klub}
V mesiaci máj, ak dá Pán zdravia a života, budeme mať Senior klub, a to znova posledný štvrtok v~mesiaci, t.j. {\bf 31.~mája na Súľovskej~ul. od~10.00 do~14.00~hod.}

Téma: Letnice -- zrod cirkvi

\autor{Jana Makovíniová}


\clanok{Pomoc ľuďom bez domova}
Milé sestry a milí bratia, varenie polievok ľuďom v~núdzi pokračuje. Prosím, ak sa chcete zapojiť, vyberte si z~nasledujúcich termínov:

17.~júl (utorok); 21.~august (utorok); 18.~september (utorok); 20.~október (sobota); 17.~november (sobota); 15.~december (sobota)

Na rezervované termíny sa, prosím, nahláste u~mňa.

\autor{Beata Bogárová}


\clanok{Zbierky za apríl}
Milí bratia a sestry, ďakujeme za vašu obetavosť. V mesiaci apríl ste prispeli:
\vskip-1ex\begitems
* misia: 374 €
* rekonštrukcia v~BJB Podunajské Biskupice: 541,50 €
* investičný fond: 316 €
\enditems


\clanok{Klub D.E.P.O. na Súľovskej}
D.E.P.O. junior je v~piatok od 16.00 do 18.00 hod. a je určený hlavne dorastu. D.E.P.O. je každý piatok od~18.00 do~22.00 hod. a je určený mládeži. Náplňou klubov sú moderné spoločenské hry, stolný futbal, pingpong, posedenie, rozhovory, hudba, občerstvenie, tvorivé dielne.


\clanok{Skupina anonymných alkoholikov Maják}
AA-liečebný program anonymných alkoholikov má svoje pravidelné stretnutia vo štvrtok od 17.30 do 18.30 hod. v~našich zborových priestoroch na Zrínskeho 2.

\autor{Kontakt: Želka 0903 294 927}
\vskip0.5ex\null

\n 1.	5.	Elena	ŽIARANOVÁ;
\n 1.	5.	Milica	MALÁ;
\n 1.	5.	Andrea	ČURILLOVÁ;
\n 3.	5.	Dárius	KRÁĽ;
\n 4.	5.	Peter	BUZÁŠ ml.;
\n 8.	5.	Vladimír	KRAJČÍ;
\n 8.	5.	Jana	ŠEBOVÁ;
\n 11.	5.	Želmíra	PRAŽENICOVÁ;
\n 16.	5.	Mária	ŠEĎOVÁ;
\n 16. 5.	Ján	SZÖLLÖS;
\n 17. 5.	Juraj	HOVORKA;
\n 17. 5.	Lenka	KOVÁČOVÁ;
\n 17. 5.	Lívia	KOLÁŘIKOVÁ;
\n 18. 5.	Anna	DANTEROVÁ;
\n 19. 5.	Oľga	VALCHÁŘOVÁ;
\n 20. 5.	Rastislav	PAULEN;
\n 26. 5.	Radovan	HOVORKA;
\n 30. 5.	Miluška	BAŽALOVÁ;
\narodeniny


\program{
\p 1  ; ut ; 15.00 ; Popoludnie pri Biblii (P. Pivka, Zrínskeho 2);.;;
\p 2  ; st ;.;;.;;
\p 3  ; št ; 19.00 ; Biblická hodina (J. Szőllős, Zrínskeho 2);.;;
\p 4  ; pi ;.;;.;;
\p 5  ; so ;.;;.;;
\p 6  ; ne ; 9.30 ; Bohoslužby (J. Szőllős); 10.00; Chvojnica (D. Kráľ);
\p 7  ; po ; 18.00 ; Modlitby (Zrínskeho 2);.;;
\p 8  ; ut ; 15.00 ; Popoludnie pri Biblii (P. Pivka, Zrínskeho 2);.;;
\p 9  ; st ;.;;.;;
\p 10  ; št ; 19.00 ; Biblická hodina (J. Szőllős, Zrínskeho 2);.;;
\p 11  ; pi ; 16.00 ; D.E.P.O. (Súľovská 2);.;;
\p 12  ; so ; 18.00 ; Mládež (Súľovská 2);.;;
\p 13  ; ne ; 9.30 ; Bohoslužby (T. Valchář); 10.00; Chvojnica (P. Škulec);
\p 14  ; po ; 18.00 ; Modlitby (Zrínskeho 2);.;;
\p 15  ; ut ; 15.00 ; Popoludnie pri Biblii (P. Pivka, Zrínskeho 2);.;;
\p 16  ; st ;.;;.;;
\p 17  ; št ; 19.00 ; Biblická hodina (J. Szőllős, Zrínskeho 2);.;;
\p 18  ; pi ; 16.00 ; D.E.P.O. (Súľovská 2);.;;
\p 19  ; so ; 18.00 ; Mládež (Súľovská 2);.;;
\p 20  ; ne ; 9.30 ; Bohoslužby (P. Kolárovský); 10.00; Chvojnica (J. Szőllős);
\p 21  ; po ; 18.00 ; Modlitby (Zrínskeho 2);.;;
\p 22  ; ut ; 15.00 ; Popoludnie pri Biblii (P. Pivka, Zrínskeho 2);.;;
\p 23  ; st ;.;;.;;
\p 24  ; št ; 19.00 ; Biblická hodina (J. Szőllős, Zrínskeho 2);.;;
\p 25  ; pi ; 16.00 ; D.E.P.O. (Súľovská 2);.;;
\p 26  ; so ; 18.00 ; Mládež (Súľovská 2);.;;
\p 27  ; ne ; 9.30 ; Bohoslužby (D. Jones); 10.00; Chvojnica (M. Kolářik);
\p 28  ; po ; 18.00 ; Modlitby (Zrínskeho 2);.;;
\p 29  ; ut ; 15.00 ; Popoludnie pri Biblii (P. Pivka, Zrínskeho 2);.;;
\p 30  ; st ;.;;.;;
\p 31  ; št ; 19.00 ; Biblická hodina (J. Szőllős, Zrínskeho 2);.;;
}

\tiraz
\bye
