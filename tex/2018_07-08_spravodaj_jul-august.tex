\input makra.tex % nacitanie Ivanom pripravenych nastaveni a prikazov
\hyphenation{star-šov-stvo} % rozdelenie slov na konci riadku, treba tu uviest slova, ktore sam nepozna

\spravodaj{7-8}{2018}


\clanok{Hnojná brána}
V knihe Nehemiáš čítame strhujúci príbeh obnovy hradieb a brán zničeného Jeruzalema. Zároveň však môžeme prežívať podobnú obnovu v~našom živote. Čítame o~dôsledkoch hriechu a neposlušnosti Bohu (zničené hradby a spálené brány Jeruzalema), zároveň však vidíme potrebu obnovenia aj v~našich životoch dnes. Zborené hradby znamenajú absenciu jasných hraníc voči čomukoľvek, čo by mohlo ohroziť Boží život v~nás a naše povolanie Božích detí. Nuž a spálené brány hovoria o~neschopnosti nevpustiť nič, čo chce ničiť a zároveň o~slabej otvorenosti veciam, ktoré môžu byť pre nás a našich blížnych bohatým prameňom radosti, pokoja, spoločenstva a rastu.

Ďalšia, v~poradí piata, brána Jeruzalema, na ktorú sa v~tomto úvodníku pozrieme, sa volala Hnojná brána:

{\bf A Hnojnú bránu opravil Malkiáš, syn Rechabov, knieža okresu Bét-hakkérema. Ten ju vystavil a postavil jej vráta s~jej zámkami a s~jej závorami.} (Nehemiáš 3, 14)

Najpravdepodobnejšie ide o~Črepinovú bránu, ktorá je spomenutá v~knihe proroka Jeremiáša:

{\bf Takto riekol Hospodin: Choď, kúp si hlinený krčah od hrnčiara, vezmi so sebou niektorých zo starších ľudu i zo starších kňazstva, vyjdi do údolia Ben Hinnóm, ktoré je pri vchode do Črepinovej brány, a hlásaj tam slová, ktoré ti poviem.} (Jeremiáš 19, 1 -- 2)

Táto brána viedla na smetisko a teda bola dôležitá pre hygienu v~meste. Pre nás symbolizuje našu úroveň duchovnej hygieny, našu schopnosť rozoznávať čo je smetím a čoho sa potrebujeme zbaviť.

{\bf Ale čo mi bolo ziskom, uznal som pre Krista za stratu. A iste aj pokladám všetko za stratu pre nekonečne vzácne poznanie Ježiša Krista, svojho Pána, pre ktorého som všetko stratil a všetko pokladám za smeti, aby som Krista získal.} (Filipským 3, 7 -- 8)

„Všetko pokladám za smeti...“ -- v našom živote sú veci, ktoré sú len smeťami a špinia našu cestu s~Bohom. Pavol o~tom hovorí takto:
{\bf Milovaní, keďže teda máme také zasľúbenie, očisťme sa od všetkých poškvŕn tela i ducha a v~bázni Božej dokonajme svoje posvätenie.} (2. Korintským 7, 1)

Zaujímavé je, že túto bránu obnovoval muž, ktorého hebrejské meno Malkíjah znamená „Môj kráľ je Hospodin“ a bol kniežaťom okresu Bét-Hakkerem, čo v~preklade znamená Dom vinice. Bez toho, aby sme nášho Boha urobili skutočne naším kráľom, nebudeme môcť niesť ovocie. O tom hovorí 15. kapitola evanjelia Jána. Jednoznačné rozhodnutie zbaviť sa smetí a neplodných vecí z~nášho života prichádza vtedy, keď sa rozhodujeme úplne sa poddať Pánovi ako Kráľovi. Vtedy sa rozhodujeme zostávať v~Ňom a zbavujeme sa všetkého, čo nám v~tom akokoľvek bráni. A vtedy prichádza do nášho života aj orezávanie.

Práve táto Hnojná brána predstavuje našu ochotu a schopnosť vykročiť na smetisko a vyniesť tam všetko, čo patrí naozaj len a len na to miesto.

V našom živote je veľa vecí, ktoré by sme mali pri Hnojnej bráne odhodiť. Keď cez Slovo a Ducha vidíme veci, ktoré sú v~skutočnosti len smeťami a špinia našu vieru, odhoďme ich! Dovoľ Pánovi, aby ti ukázal, čo musí ísť preč z~tvojho života.

\autor{Michal Kevický}


\clanok{Správy zo staršovstva}
V júni, pred koncom školského roku, sme sa stretli dva razy. Boli to prvé stretnutia, na ktorých bol aj náš nový kazateľ Danny Jones. Okrem Dannyho bolo novinkou aj to, že sme sa stretli v~novozrekonštruovaných priestoroch kancelárie.

Pripravovali sme krst, ktorý bol 10. júna v~Miloslavove. Svoju vieru v~Pána Ježiša vyznali Lukáš Máťuš a Kristína Lászlóová. Obaja požiadali aj o~členstvo v~našom zbore. Zároveň chcem oznámiť, že na jeseň pripravujeme krst. Kto má záujem vyznať svoju vieru v~Pána Ježiša krstom ponorením, môže kontaktovať kazateľa zboru alebo členov staršovstva.

Pripravovali sme tiež slávnosť inštalácie kazateľa. Pri tejto slávnosti, ktorá bola 24. júna, nám slúžil novozvolený predseda Rady BJB, brat kazateľ Benjamín Uhrin.

Na konferencii delegátov zborov BJB na Slovensku bola prezentácia ideového zámeru postaviť v~Bernolákove domov pre seniorov. Na základe tejto prezentácie sme sa mali do konca júna vyjadriť, či podporujeme tento ideový zámer a s~ním súvisiacu požiadavku o~zmenu Územného plánu zóny. Sme presvedčení, že aj toto je príležitosťou naplniť skutkom našu vieru v~Pána Ježiša.

Dlhé roky do „portfólia“ nášho zboru patrila aj skupina AA. Sestra Želka Praženicová pracovala s~abstinujúcimi alkoholikmi. Po rokoch strávených v~tejto práci sa rozhodla zamerať svoju službu iným smerom. Chceme sa jej aj touto cestou poďakovať za vernosť v~práci so skupinou AA a modlíme sa za to, aby ju náš Pán viedol aj v~ďalšej práci, dával jej silu a múdrosť.

Počas prázdnin budú nedeľné dopoludňajšie zhromaždenia na Palisádach a na Chvojnici. Večerný program na Palisádach bude len príležitostne.

Biblické hodiny budú pokračovať po prázdninách.

Počas letných prázdnin sú v~našom zbore plánované tábory v~týchto termínoch:

\vskip-1ex\begitems
* 1. – 7. júla		rodinný tábor
* 28. júla – 4. augusta	dorastenecký tábor
* 5. – 12. augusta	mládežnícky tábor
\enditems

\autor{Za staršovstvo Peter Pribula}


\clanok{Krst}
Na jeseň pripravujeme ďalší krst v našom zbore. Kto má záujem vyznať svoju vieru v~Pána Ježiša krstom ponorením, môže kontaktovať kazateľa zboru D. Jonesa alebo členov staršovstva.


\clanok{Pozvanie na sobáš}
Kristína Lászlóová a Matej Matušek plánujú uzavrieť manželstvo 25. augusta 2018 v~našej modlitebni na Palisádach. Obaja snúbenci sú pokrstení a žiadajú o~členstvo v~našom cirkevnom zbore.


\clanok{Campfest 2018}
Dynamické chvály, hlboké uctievanie, inšpiratívne slovo, semináre, svedectvá, koncerty, tanec, netradičné športy a prežívanie Božej prítomnosti spolu s~tisíckami mladých ľudí! Pozývame Vás na 20. ročník festivalu Campfest, ktorý bude {\bf 2. -- 5. augusta v~Kráľovej Lehote} na tému „Premeniť“. Veľakrát sa zameriavame len na premenu samého seba. Veríme však, že Pán Boh sníva o~premene, ktorá má širší záber, než je naše srdce. O premene, ktorú môžeme reálne vnímať a vidieť na uliciach, v~školách a na pracoviskách. O premene, ktorú môžeme zažívať spolu a ktorá mení aj naše srdcia. Viac informácií a možnosť zakúpenia vstupeniek nájdete na stránke \ulink[https://www.campfest.sk]{www.campfest.sk}. Ak máte chuť pomôcť pri príprave festivalu, stále sú voľné dobrovoľnícke miesta.


\clanok{Mládežnícky kemp 2018}
Ahojte, aj tento rok sa bude konať už 13. ročník celoslovenského kempu mládeže BJB. Tak neváhaj a príď si oddýchnuť po náročných prázdninách strávených na brigádach, ale aj v~službe Pánovi na akciu, kde bude slúžené tebe. Viac info na \ulink[https://mladez.baptist.sk/kemp]{mladez.baptist.sk/kemp}.

Základné informácie o~kempe
\vskip-1ex\begitems
* Začiatok: 28. augusta 2018 o~17.00 hod.
* Koniec: 1. septembra 2018 o~11.00 hod.
* Miesto: Tábor Royal Rangers, Nová Lehota -- Dolina
* Max. počet účastníkov: 70
* Cena: 30 €
* Stravovanie: celodenné
* Ubytovanie: v~chate alebo v~stane
* Účastník kempu musí mať v~čase konania kempu vek min. 14 rokov a max. 26 rokov.
* Zober si so sebou spacák a aj karimatku!
* Maximálna kapacita chaty je 45 miest, po naplnení je možnosť spať iba v~stane -- stan je potrebné si priniesť.
\enditems

Uzávierka prihlášok je 24. augusta 2018.


\clanok{Konferencia seniorov}
V dňoch 11. -- 15. septembra 2018 sa uskutoční v~Chate J. A. Komenského v~Račkovej doline konferencia pre staršiu generáciu. Témou bude Bdelosť kresťana -- fyzická a duchovná prevencia (Efezským 2, 19 -- 22). Prihlášky posielajte na adresu: Chata J. A. Komenského, Račkova dolina, tel.: 0903 501 852.


\clanok{Pomoc ľuďom bez domova}
Ľuďom bez domova pomáhame počas celého roka. Jednou z~foriem pomoci sú zbierky šatstva, prikrývok, potravín a ďalších vecí. Šatstvo poskytujeme každý týždeň, preto potrebujeme neprestajne oblečenie a to najmä pre mužov: tenisky, športové sandále, ponožky, spodnú bielizeň atď.

Aktuálny zoznam vecí nájdete na stránke: \ulink[http://www.krestaniavmeste.sk/zbierka/]{www.krestaniavmeste.sk/zbierka}. Veci môžete doniesť do skladu pomoci na Ambroseho 6 každý pondelok od~17.00 do~19.00 hod. okrem mesiaca júl 2018, počas ktorého bude sklad zatvorený.

Ak chcete priniesť veci, odporúčame vám vopred kontaktovať Sylviu Vaniherovú, koordinátorku zbierok šatstva, mobil 0905~484~675. Ďakujeme!

\autor{Beata Bogárová}


\clanok{Zbierky za jún}
Milí bratia a sestry, ďakujeme za vašu obetavosť. V mesiaci jún ste prispeli:
\vskip-1ex\begitems
* misia: 495 €
* investičný fond: 393 €
\enditems


\clanok{Prázdninový režim}
Počas letných prázdnin neprebiehajú pravidelné zborové aktivity (nedeľná besiedka, dorast, mládež, biblické vzdelávanie, Klubík, DEPO, Senior klub). Stretneme sa v~septembri.


\clanok{Rádio 7}
Poslucháči Rádia 7, sme vďační Bohu za nové možnosti, ktoré nám dáva. Za to, že môžeme skvalitňovať a zlepšovať aj naše vysielanie. S týmto súvisí aj zmena v~našom archíve. Mohli sme zakúpiť nové technické vybavenie, ktoré nám umožňuje, aby relácie, ktoré vysielame, boli v~archíve uložené dlhšie a mohli tam byť zaradené vo väčšom množstve.

Takže od mesiaca jún 2018 dávame relácie do archívu postupne, vždy po odvysielaní premiér a budú v~archíve uložené počas celého štvrťroka. Každý kto bude chcieť počúvať relácie, ktoré nestihne zachytiť v~pravidelnom vysielacom čase, bude mať takto viac času na to, aby si z~archívu mohol vypočuť, alebo dopočúvať reláciu – tak ako ste nám aj mnohí písali alebo hovorili. Tešíme sa spolu s~vami z~tejto novej možnosti a veríme, že táto zmena obohatí váš archívny výber vo vašich chvíľach oddychu a inšpirácie pri Božom slove.


\clanok{Spoločné modlitby za aktuálne dianie na Slovensku}
Každú nedeľu od 16.00 do 17.00 hod. sa spoločne modlíme na Bratislavskom hrade.

Zraz je o~16.00 hod pri cisterne (studni) v~areáli hradu.
Modlitby vedie poverený koordinátor.

Termíny modlitieb na hrade počas júla a augusta: 1. 7., 8. 7., 15. 7., 22. 7., 29. 7., 5. 8., 12.~8., 19. 8., 26. 8. 2018.

K modlitbám sa môžete pridať aj zapojením do virtuálnej modlitebnej reťaze, ktorá prepája tých, ktorí sa chcú modliť aspoň 1 hodinu v~ľubovoľný deň.
\vskip-1ex\begitems
* Do modlitebnej reťaze sa pridáte tak, že zapíšete svoje meno a priezvisko do tabuľky modlitebnej reťaze vo vami vybraný deň, kedy sa budete modliť.
* Nezáleží, v~ktorú hodinu počas dňa sa budete modliť. Jediná podmienka však je, aby ste sa modlili aspoň 1 hodinu v~daný deň.
* Túto modlitebnú reťaz sme vytvorili, aby sme sa spoločne zjednotili na modlitbách, aj keď sa fyzicky nestretneme.
* Aktuálne je možné zapísať sa do modlitebnej reťaze počas júla a augusta 2018.
\enditems

Ak máte otázky ohľadne zapojenia sa do spoločných modlitieb, napíšte nám na adresu \email{kancelaria@krestaniavmeste.sk}


\n 1.	7.	Ľudovít	BETKO;
\n 4.	7.	Margita	ELISHEROVÁ;
\n 4.	7.	Ľubomíra	KOHÚTOVÁ;
\n 10.	7.	Slavomír	MÁŤUŠ;
\n 10.	7.	Katarína	KEREKRÉTY;
\n 11.	7.	Milada	KREJČOVÁ;
\n 15.	7.	Zlatica	BALÁŽOVÁ;
\n 16.	7.	Rút	BEDNÁRIKOVÁ;
\n 20.	7.	Mária	KOHÚTOVÁ;
\n 21.	7.	Lynn	PLETT;
\n 27.	7.	Lenka	KOHÚTOVÁ;
\n 28.	7.	Elena	ŠALINGOVÁ;
\n 28.	7.	Pavlína	SYNOVCOVÁ;
\n 31.	7.	Marína	CIHOVÁ;
\n 1.	8.	Dana	KEŠJAROVÁ;
\n 1.	8.	Zuzana	HRAŠKOVÁ;
\n 1.	8.	Vlasta	ŠALINGOVÁ;
\n 2.	8.	Marta	RAČIČOVÁ;
\n 4.	8.	Hector	BLANCO;
\n 7.	8.	Anna	KOPČOKOVÁ;
\n 11.	8.	Šimon	HOVORKA;
\n 16.	8.	Radovan	JANČULA;
\n 18.	8.	Anna	LIPTÁKOVÁ;
\n 23.	8.	Danica	PAULENOVÁ;
\n 25.	8.	Ivan	PAULEN;
\n 31.	8.	Miroslava	HOVORKOVÁ;
\narodeniny


\programna{7}{
\p 1  ; ne ; 9.30 ; Bohoslužby (D. Jones); 10.00; Chvojnica (P. Antalík);
\p 8  ; ne ; 9.30 ; Bohoslužby (T. Valchář); 10.00; Chvojnica (P. Škulec);
\p 15  ; ne ; 9.30 ; Bohoslužby (D. Jones); 10.00; Chvojnica (Z. Kakaš);
\p 22  ; ne ; 9.30 ; Bohoslužby (D. Jones); 10.00; Chvojnica (J. Štefko);
\p 29  ; ne ; 9.30 ; Bohoslužby (V. Krajčí); 10.00; Chvojnica (D. Jones);
}
\vskip3ex
\programna{8}{
\p 5  ; ne ; 9.30 ; Bohoslužby (D. Jones); 10.00; Chvojnica (M. Antalík);
\p 12  ; ne ; 9.30 ; Bohoslužby (T. Valchář); 10.00; Chvojnica (P. Škulec);
\p 19  ; ne ; 9.30 ; Bohoslužby (S. Kráľ); 10.00; Chvojnica (J. Štefko);
\p 26  ; ne ; 9.30 ; Bohoslužby (J. Szőllős); 10.00; Chvojnica (P. Kolárovský);
}
\koniecprogramu

\tiraz
\bye