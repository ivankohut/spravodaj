\def\velkostpisma{10}
\def\velkostriadku{12.5}
\input makra.tex % nacitanie Ivanom pripravenych nastaveni a prikazov
\hyphenation{star-šov-stvo} % rozdelenie slov na konci riadku, treba tu uviest slova, ktore sam nepozna

\spravodaj{2}{2026}


\clanok{Verš na mesiac február}
Tento mesiac sa budeme učiť nový verš, ktorý dostala pre rok 2026 naša chválospevová skupina: „Ústa spravodlivého vravia múdrosť a jeho jazyk vyslovuje právo. On má v~srdci zákon svojho Boha, jeho kroky sa nesklátia.“ (Ž 84,6-7)


\clanok{Správy zo staršovstva}
Staršovstvo sa počas januára stretlo na pravidelných plánovaných stretnutiach dvakrát. Prvé stretnutie sa konalo 13.~1. Bratia starší sa po sviatočnej prestávke zaoberali viacerými témami a problémami, ktoré sa nahromadili a bolo ich potrebné riešiť. Po úvodnom zamyslení br. R.~Nemca na text 1.~Petra 4,10-11 a modlitbe sa bratia zdieľali o~tom, ako sa im darí v~modlitbách a starostlivosti o~členov a priateľov nášho zboru.

Vyhodnotili priebeh ZČZ, ktoré sa konalo 11.~1. o~vyrovnaní majetkov medzi BJB a PD Bernolákovo. Ocenili, že sa podarilo dosiahnuť vysokú účasť potrebnú na schválenie a aj schváliť navrhovaný odpredaj časti pozemkov a uvoľnenie ďalšej časti na prenájom. Tiež určili delegátov, ktorí zastupovali náš zbor na on-line rokovaní mimoriadnej KDZ o~tomto rozhodnutí, ktorá sa konala 17.~1. KDZ potrebnou väčšinou hlasov odsúhlasila predaj malej časti pozemku v~Bernolákove PD Bernolákovo.

Súčasťou rokovania staršovstva boli aj ďalšie body týkajúce sa BJB, a to predpis pre platenie príspevkov pre náš zbor na rok 2026 v~súvislosti s~výškou nájmu za Súľovskú a tiež príprava volieb do orgánov BJB a navrhovanie kandidátov. V~tejto súvislosti bolo dohodnuté a aj sa uskutočnilo oblastné stretnutie staršovstiev západnej oblasti BJB 20.~1. na Súľovskej. Na tomto stretnutí sa navrhli potenciálni kandidáti a hovorilo sa o~procese prípravy volieb.

V~ďalších bodoch svojho programu stretnutia sa staršovstvo venovalo vnútrozborovým otázkam. Prerokovalo plán na hlavné zborové aktivity v~tomto roku 2026 a konkretizovalo najmä bohoslužby a podujatia v~rámci Veľkej noci. Pozornosť bola venovaná aj potrebe informovanosti členov zboru o~možnostiach zapojenia sa do rôznych zborových aktivít a služieb a oživenie akejsi „burzy služieb“.

Dôležitou súčasťou bolo stanovenie termínu výročného ZČZ na 8.~marca a odštartovanie jeho prípravy ako aj diskusia o~harmonograme volieb ďalšieho kazateľa zboru ako aj diskusia o~rámcoch prípravy rozpočtu zboru. Staršovstvo prerokovalo aj list od Viktora Potockého s~informáciou o~jeho službe v~ukrajinskom zbore a žiadosťou o~jeho podporu.
V~rôznom sa staršovstvo venovalo otázke možností parkovania počas podujatí a ďalším otázkam každodenného života zboru.

Na druhom stretnutí dňa 27.~1. sa bratia starší v~úvode zdieľali a diskutovali o~tom, čo to znamená viesť bez panovania, na základe kapitoly 5 z~knihy Starší zboru. Potom prebrali v~bode rôzne prevádzkovo -- technické a administratívne záležitosti a urobili potrebné rozhodnutia s~nimi spojené.

Hlavným bodom stretnutia bola diskusia a rozhodnutia týkajúce sa voľby ďalšieho kazateľa zboru a stanovenie časového harmonogramu volieb v~súlade s~ustanoveniami zborového volebného poriadku. Detaily sú v~samostatnom článku.

Svoje podnety a návrhy týkajúce sa života a služby zboru, či práce staršovstva môžete dávať ústne, alebo písomne členom staršovstva, alebo zaslať aj e-mailom na adresu \email{starsovstvo@bjbpalisady.sk}.


\clanok{Voľby ďalšieho kazateľa zboru}
Staršovstvo dostalo 19.~1. kladné vyjadrenie br.~kaz.~T.~Hanesa ku kandidatúre za kazateľa nášho zboru. Brat kazateľ Timo Hanes sa bol už predstaviť v~zbore a bol priestor na diskusiu s~ním o~jeho predstave služby v~našom zbore dňa 21.~9.~2025. Náš zbor je prvý v~poradí zborov (BJB Palisády, BJB Viera, BJB Komárno, BJB Košice), ktoré br.~kazateľa T.~Hanesa oslovili a preto by sme mali voľby uskutočniť čím skôr, aby v~prípade, že by nebol u~nás zvolený, bol priestor aj pre ďalší zbor (zbory) v~poradí uskutočniť voľby do ukončenia služby br.~kazateľa v~zbore BJB Revúcka Lehota v~lete tohto roku.

Staršovstvo zhodnotilo, že nie je potrebné v~zbore robiť ďalší prieskum na voľby, nakoľko prieskum bol robený na začiatku roku 2025 a prebiehal počas celého uplynulého roku aj prostredníctvom konzultácií s~ďalšími potenciálnymi kandidátmi a s~členmi nášho zboru. Výsledkom prieskumu je, že máme troch kandidátov, ktorí aj vyjadrili predbežný záujem slúžiť v~našom zbore v~pozícii kazateľa zboru.

Títo bratia, F.~Barkóczi, T.~Hanes a D. M.~Chuchút, sú zaradení podľa ustanovení volebného poriadku na predbežnú kandidátku, o~ktorej bude mať zbor príležitosť hlasovať.

\cast{Harmonogram}
\begitems
* 1. 2. -- Vysvetlenie súčasného stavu ohľadom procesu volieb a oznam o~konaní ZČZ, kde bude hlasovanie o~predbežnej kandidátke.
* 8. 2. -- ZČZ o~15.30~hod. s~jedným bodom -- odsúhlasenie kandidátky tajným hlasovaním, hlasovanie potrvá do 15.~2. Navrhnutí kandidáti, ktorí získajú nadpolovičnú väčšinu hlasov riadnych členov zúčastnených na hlasovaní, tvoria kandidátku zboru na voľby kazateľa.
* Od 15. 2.  (po zrátaní hlasov a zistení výsledku hlasovania) -- Získanie (potvrdenie) súhlasu s~kandidatúrou od kandidátov, ktorí získali potrebný počet hlasov.
* 22. 2. -- Oznámenie definitívnej kandidátky (tí, ktorí získali dostatok hlasov a súhlasili s~kandidatúrou) a vyhlásenie volieb.
* 22. 2. a 1. 3.  -- Priestor po dopoludňajších bohoslužbách pre predstavenie vízie svojej práce v~zbore pre kandidátov, ktorí ešte nemali priestor ju predstaviť.
* 8. 3. -- 22. 3. Voľby kazateľa zboru -- začnú na výročnom ZČZ.
* 29. 3. -- Oznámenie výsledku volieb.
\enditems

Ďalšie detaily a postupy pri voľbách budú spresnené volebnou komisiou a staršovstvom v~procese priebehu volieb. V~prípade otázok ohľadne volieb sa obráťte na bratov starších, alebo napíšte e-mail na adresu \email{starsovstvo@bjbpalisady.sk}
\autor{Ján Szőllős}


\clanok{Rekonštrukcia fasády našej modlitebne}

V~decembri 2025 sme podpísali zmluvu s~firmou Štukonz~s.r.o. Zazmluvnená cena je 181~673~€. Ku 31.~12.~2025 bolo na zborovom investičnom fonde 169~126~€. Rozdiel je cca 12~500~€. Rozhodli sme sa dosporiť ešte 15~000~€. Vychádzajúc z~našich finančných možností a skúseností navrhujeme príspevok, dar 150~€ za člena na tento účel.

Prispieť je možné formou zbierky počas 4.~nedieľ v~mesiaci, tieto zbierky sú určené na investičný fond. Alebo bankovým prevodom na zborový účet SK36 0900 0000 0000 1147 1836 a uvedením variabilného symbolu 777, prípadne aj osobne s.~Ľubke Kohútovej.

Práce na fasáde budú prebiehať v~mesiacoch apríl až júl 2026. Ďakujeme za doterajšiu ochotu a prosíme za modlitby, aby všetko prebehlo úspešne a pod Božou ochranou.
\autor{Ľubomír Syč}
\vfill\break


\clanok{Zborové členské zhromaždenie}
ZČZ sa bude konať v~nedeľu 8.~2.~2026 o~15.30~hod. v~modlitebni na Palisádach. Programom bude hlasovanie o~kandidátke na voľbu kazateľa. Hlasovanie bude prebiehať od~8.~2.~2026 do~15.~2.~2026 do~12.00~hod. Účasť členov je potrebná.


\clanok{Rozhovory na Palisádach}
Po dvoch prvých prednáškach v~decembri a januári, prinášajú Rozhovory na Palisádach tretiu tému, príznačnú pre február: Manželstvo -- priestor pre život. Stretnutie bude spojené s~diskusiou a uskutoční sa v~sobotu 14.~2.~2026 od 9.30~hod. v~našej modlitebni na Palisádach. Slúžiť nám budú Ivan a Ester Staroňovci, ktorí majú mnohoročnú skúsenosť so službou manželom.
Pozývame manželov, snúbencov, jednotlivcov všetkých generácií. Pozvite aj priateľov.


\clanok{Sestry}
Sestry sa v~tomto mesiaci stretnú v~stredu 18.~2.~2026 o~17.30~hod. na Zrínskeho~2. Budú vyprosovať požehnanie pre nevestu Martinku Javorníkovú, ktorá s~Adamom Alexajom uzavrie manželstvo v~našej modlitebni dňa 21.~2.~2026 o~14.00~hod.


\clanok{Lyžovačka v~Račkovej}
Počas jarných prázdnin Bratislavského kraja máme možnosť prihlásiť sa na pobyt / lyžovačku v~Chate Račkova dolina. Pobyt je možný v~termíne od 13.~2. do 20.~2., prihlasujte sa u~br. Petra Antalíka.
\vfill\break


\clanok{Zbierky v~januári}
\table{lr}{
Na misiu			& 444~€ \cr
Na investičný fond	& 557~€ \cr
}
\vskip1em

Pripomíname, že zbierky sa u~nás konajú pravidelne takto:
\vskip-1ex\begitems
* každú 2. nedeľu v~mesiaci je zbierka venovaná misii a
* každú 4. nedeľu je zbierka na investície.
\enditems

Aj naďalej máte možnosť prispieť do „nedeľnej zbierky“, a to prevodom na účet zboru. Do poznámky pre prijímateľa, prosím, uveďte „zbierka“.

Bankové spojenie: SK36 0900 0000 0000 1147 1836, SWIFT: GIBASKBX


\n 3.	2.	Vlasta	BALÁŽOVÁ;
\n 3.	2.	Miroslav	ANTALÍK;
\n 5.	2.	Štefánia	ANTALÍKOVÁ;
\n 5.	2.	Barbora	ANTALÍKOVÁ;
\n 11.	2.	Oľga	KOVÁČOVÁ;
\n 11.	2.	Beáta	BOGÁROVÁ;
\n 12.	2.	Martin	PRIBULA;
\n 13.	2.	Zlatica	VYSKOČILOVÁ;
\n 15.	2.	Ingrid	JANČULOVÁ;
\n 15.	2.	Dávid	VALCHÁŘ;
\n 23.	2.	Anna RUCIN (PLETT);
\narodeniny


\program{
\p  1 ; ne ;  9.30 ; Bohoslužby (J.~Szőllős + VP) ;.;;
\p  2 ; po ; 18.00 ; Modlitebná skupinka ;.;;
\p  3 ; ut ; 15.00 ; Biblická hodina pre seniorov (P.~Pivka) ;.;;
\p  4 ; st ;.;;.;;
\p  5 ; št ; 18.00 ; Biblická hodina (F.~Barkóczi) ;.;;
\p  6 ; pi ; 17.30 ; Dorast ;.;;
\p  7 ; so ; 18.00 ; Mládež ;.;;
\p  8 ; ne ;  9.30 ; Bohoslužby (D.~M.~Chuchút) ;.;;
\p  9 ; po ; 18.00 ; Modlitebná skupinka / skupinka „Rastieme v~Kristovi“ ;.;;
\p 10 ; ut ; 15.00 ; Biblická hodina pre seniorov (P.~Pivka) ;.;;
\p 11 ; st ;.;;.;;
\p 12 ; št ; 18.00 ; Biblická hodina (D.~M.~Chuchút) ;.;;
\p 13 ; pi ; 17.30 ; Dorast ;.;;
\p 14 ; so ; 18.00 ; Mládež ;.;;
\p 15 ; ne ;  9.30 ; Bohoslužby (S.~Baláž) ;.;;
\p 16 ; po ; 18.00 ; Modlitebná skupinka ;.;;
\p 17 ; ut ; 15.00 ; Biblická hodina pre seniorov (P.~Pivka) ;.;;
\p 18 ; st ; 17.30 ; Stretnutie sestier (J.~Cihová);.;;
\p 19 ; št ; 18.00 ; Biblická hodina (F.~Barkóczi) ;.;;
\p 20 ; pi ; 17.30 ; Dorast ;.;;
\p 21 ; so ; 18.00 ; Mládež ;.;;
\p 22 ; ne ;  9.30 ; Bohoslužby (P.~Pribula) ;.;;
\p 23 ; po ; 18.00 ; Modlitebná skupinka / skupinka „Rastieme v~Kristovi“ ;.;;
\p 24 ; ut ; 15.00 ; Biblická hodina pre seniorov (P.~Pivka) ;.;;
\p 25 ; st ;.;;.;;
\p 26 ; št ; 18.00 ; Biblická hodina (D.~M.~Chuchút) ;.;;
\p 27 ; pi ; 17.30 ; Dorast ;.;;
\p 28 ; so ; 18.00 ; Mládež ;.;;
}


\tiraz
\bye
