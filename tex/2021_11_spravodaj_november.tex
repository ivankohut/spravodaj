%v programe je vela poloziek, musel som znizit vysku riadku: \vrule height2.4ex% -> \vrule height2.2ex%
\def\velkostpisma{10}
\def\velkostriadku{12.5}
\input makra.tex % nacitanie Ivanom pripravenych nastaveni a prikazov
\hyphenation{star-šov-stvo} % rozdelenie slov na konci riadku, treba tu uviest slova, ktore sam nepozna

\spravodaj{11}{2021}


\clanok {Blahoslavení, ktorí hladujú a žíznia po spravodlivosti...}
Kto z~nás ľudí by netúžil po spravodlivosti? Boli sme tak stvorení, že „máme v~sebe zakódovanú túžbu po spravodlivosti“. Ak sa okolo nás deje nespravodlivosť, tak  veľmi ľahko dokážeme vzplanúť hnevom voči bezpráviu. A~to platí aj o~ľuďoch, ktorí nevyznávajú vieru v~jediného Boha vyjaveného tomuto svetu v~osobe Pána Ježiša Krista. Jednoducho je to v~nás ľuďoch, lebo sme stvorení Bohom, naším Kráľom, pre ktorého právo a spravodlivosť sú základmi Jeho trónu (Žalm 97,1--2). Pán Ježiš predpokladá, že jeho nasledovníci nielenže budú konať a žiť spravodlivo podľa Božích princípov, ale  budú v~tom príkladom svojmu okoliu. Nepoznám veriaceho človeka, ktorý by netúžil po spravodlivosti vo svojom živote, v~cirkvi a v~neposlednom rade aj v~spoločnosti. Aká je však holá realita, domôže sa bežný človek svojho práva? Môžeme si hovoriť, že sme kresťanská krajina, ale právny systém v~štáte hovorí jasnou rečou o~tom, kto tu vládne, právo alebo bezprávie?

Žiaľ, vidíme, že spravodlivosti je napriek všetkej ľudskej snahe o~jej nastolenie málo ako šafranu. Denne počujeme a čítame o~bezpráví, ktoré sa deje vo svete. Ale vidíme to aj v~našom širšom, či užšom okolí.

Nechcem ale hovoriť len o~spoločnosti, o~štáte, o~svete, ale predovšetkým si musíme nastaviť zrkadlo my sami, kresťania. Nazrime  do našej „kresťanskej kuchyne“, ako je to tam so spravodlivosťou, vládne tam? Položme si otázku každý pre seba osobne, ako je to konkrétne v~mojom živote, konám spravodlivo v~jednotlivých situáciách každodenného života? Ako je to s~nami kresťanmi v~našich spoločenstvách? Snažíme sa o~spravodlivosť v~našich bratsko-sesterských vzťahoch? A~nielen sa snažíme, ale žijeme ako spravodliví ľudia v~cirkvi i mimo nej? Aké svedectvo vydávame svojmu okoliu ohľadne života v~spravodlivosti a čistote Božej? Tieto otázky mi víria hlavou, keď sa zamýšľam nad Božou spravodlivosťou, ktorá je dokonalá ako všetko, čo pochádza od nášho nebeského Otca.

Musíme si odvyknúť vtesnať náš „kresťanský život“ do jedného dňa v~týždni, prípadne do cirkevných stretnutí uprostred týždňa. O~tom, akí sme kresťania, rozhodujú naše rozhodnutia v~bežnom týždni, v~tých najbežnejších situáciách života. Tá skutočnosť, že Hospodin je milostivý, ale zároveň aj Spravodlivý, je zaväzujúca pre nás, Jeho Cirkev.

Kresťan, ktorý hladuje a žízni po spravodlivosti, je šťastný, pretože vie, že je na tej správnej strane. Na strane Víťaza, na strane Ježiša Krista, a preto bude nasýtený tou Božou spravodlivosťou, ktorá prevyšuje tú ľudskú.

Lebo jedine ten človek, ktorý je vyzbrojený spravodlivosťou a svätosťou, ktorá má svoj pôvod u~Boha, môže priniesť druhému človeku tú pravú nádej, ktorá vidí za hranice tohto pominuteľného sveta a očakáva vierou na naplnenie Božieho zasľúbenia, {\it „na nové nebesá a novú zem, v~ktorých spravodlivosť prebýva“} (2Pt 3,13).

Túžme po tom, aby svet mohol spoznať cez naše Bohom ospravedlnené životy toho jediného Spravodlivého, ktorý kedy žil na tejto zemi, Ježiša Krista Nazaretského.

\autor{Pavel Pivka}



\clanok {Správy zo staršovstva}
Staršovstvo pokračovalo v~októbri vo svojich pravidelných stretnutiach v~dvojtýždňovom intervale. Stretnutie 7.~októbra sa mimoriadne konalo vo štvrtok miesto utorka. Druhé stretnutie sa konalo 19.~októbra. Bratia starší sa na oboch stretnutiach venovali dôležitej otázke voľby ďalšieho kazateľa zboru a rozhodli, že začneme proces hľadania ďalšieho kazateľa prieskumom, v~ktorom môžu členovia zboru navrhovať staršovstvu kandidátov do 14.~novembra. Dôležité je, aby súčasťou tohto procesu boli aj modlitby, aby Pán povolal ďalšieho kazateľa do nášho zboru a aby sme my vedeli prijať správne rozhodnutia. Staršovstvo rozhodlo aj o~konaní zborového členského zhromaždenia 21.~novembra. Dôležitou súčasťou stretnutí bola aj príprava krstu, ktorý sa konal 24.~októbra. V~rámci prípravy sa uskutočnilo aj stretnutie a rozhovor staršovstva s~krstencami. Témou na oboch stretnutiach bola aj výmena sedenia v~našej modlitebni, ktorá by umožnila flexibilnejšie usporiadanie a tým aj lepšie podmienky pre konanie rôznych akcií v~našej modlitebni. Staršovstvo poskytlo v~tejto veci podklady pre pracovnú skupinu, aby mohla pripraviť konkrétny návrh. Ďalšou dôležitou témou stretnutí boli aj pastoračné otázky. Staršovstvo riešilo aj ďalšie organizačné otázky súvisiace s~chodom zboru a uskutočňovaním zhromaždení. Prosím, nezabúdajte sa prihovárať za bratov starších a prácu staršovstva vo svojich modlitbách.

\autor {Ján Szőllős}
\vfill\break


\clanok {Nedeľné bohoslužby v~najbližšom období}
Bratislava sa od 8.~novembra bude nachádzať v~červenej fáze ostražitosti, čo znamená, že sa musíme vysporiadať s~istými obmedzeniami. Predbežne budeme naďalej pokračovať s~dvomi zhromaždeniami, pričom to prvé o~9.00~hod. bude v~režime OTP (účastníci sa budú musieť vedieť preukázať očkovaním, testom alebo prekonaním covidu) a to druhé o~10.30~hod. bude iba pre zaočkovaných.

Online prenos bude zabezpečný z bohoslužieb o~9.00~hod.: \ulink[https://bit.ly/3cgSMBG]{bit.ly/3cgSMBG}.

Prosíme vás, aby ste nechodili do zhromaždenia, pokiaľ máte akékoľvek príznaky.


\clanok {Zborové členské zhromaždenie}
Všetkým členom zboru dávame do pozornosti pripravované zborové členské zhromaždenie, ktoré sa bude konať v~nedeľu 21.~11.~2021 o~16.00~hod. na Palisádach.


\clanok {Prieskum na druhé kazateľské miesto}
Staršovstvo zboru sa spolu s~br. kazateľom  rozhodlo začať proces hľadania druhého (ďalšieho) kazateľa. V~tejto súvislosti vyhlasujeme prieskum na kandidátov na druhé kazateľské miesto. Návrhy môžu podávať členovia i nečlenovia zboru e-mailom na adresu \email{starsovstvo@bjbpalisady.sk}, alebo osobne kazateľovi, alebo ktorémukoľvek členovi staršovstva. Prieskum bude trvať do nedele 14.~11.~2021.


\clanok {Stretnutia sestier}
Chceli by sme vás upozorniť na zmenu termínov stretnutí s~Dankou Paštrnákovou. Stretnutia sa nebudú konať v~sobotu, ale v~stredu, a to nasledovne:
\vskip-1ex\begitems
* 17. 11. 2021 o~17.30 hod. bude Danka spolu so svojimi nevestami hovoriť na tému „Vzťah medzi svokrou a nevestou“
* 1. 12. 2021 o~17.30 hod. bude Danka so synom Oďom hovoriť na tému „Svedectvo matky a syna, ktorý mal problém so závislosťou“
\enditems

Nahrávka z~prvého stretnutia dňa 9.~10.~2021 na tému „Vzťah medzi matkou a dcérou“ je k~dispozícii na zborovej webstránke \ulink [https://www.bjbpalisady.sk/kazne]{bjbpalisady.sk/kazne} pod názvom „Matka a dcéra“.
\vfill\break


\clanok {Vianoce spolu}
V spolupráci s~o. z. Detská misia sme sa ako cirkevný zbor zapojili do projektu {\it Vianoce spolu}. Cieľom projektu je pred Vianocami priniesť čo najväčšiemu počtu detí Dobrú správu o~narodení Spasiteľa. Je to tiež príležitosť ako rozvíjať misijnú aktivitu zboru za múrmi kostola.

Spoločne sa modlíme za ZŠ Milana Hodžu, s~ktorou už máme isté kontakty. Prosíme za deti, učiteľov i rodičov, aby ich srdcia boli pripravené pre počutie evanjelia. Prosíme za priaznivú epidemiologickú situáciu, aby sme mohli ísť do škôl. Prosíme aj sami za seba, aby nás Pán Boh vystrojil múdrosťou, vytrvalosťou a odhodlaním byť tomuto svetu svetlom a soľou.

Do projektu ideme s~vierou, že Pán Boh má situáciu pod kontrolou. Ak by sa nám nepodarilo ísť priamo na školu, ponúkneme im vianočné video, ktoré pripraví Detská misia. Taktiež škole poskytneme vianočné letáčiky, ktoré môžu rozdať deťom. A~modliť sa môžeme vždy a za každých okolností. Pán Boh určite nenechá naše modlitby bez odozvy!

Ak budete mať k~pripravovanej akcii pripomienky, nápady alebo sa budete chcieť priamo zapojiť do vianočného programu pre deti, obráťte sa na Miriam Kešjarovú (\email{kesjarova@detskamisia.sk}).

Viac info na \ulink[https://www.detskamisia.sk/vianoce-spolu.html]{detskamisia.sk/vianoce-spolu.html}.

Ďakujeme Vám!


\clanok {Predvianočné zbierky pre ľudí bez domova a klientov výdajne potravín}
Obdobie Vianoc by sme radi spríjemnili ľuďom bez domova a aj klientom výdajne potravín a rozdali im taký tovar, ktorý máme obyčajne problém získať.

Preto Vás prosíme o~pomoc a darovanie nasledovného tovaru:

\cast{Zbierka do výdajne potravín:}
čaj, olej, rybie konzervy, mäsové konzervy, strukoviny, konzervované hotové jedlá, ochutené ovsené vločky, cereálie, Granko, káva, pracie a čistiace prostriedky.

\cast {Zbierka na vianočné balíčky pre ľudí bez domova:}
 čiapka, ponožky alebo rukavice, tekuté mydlá, antiperspiranty, zubné pasty a kefky, šampóny.

\cast {Zbierka oblečenia:}
viac info na webovej stránke \ulink [https://www.krestaniavmeste.sk/zbierka]{krestaniavmeste.sk/zbierka}.

\vskip1ex
Veci nám môžete doniesť každý štvrtok od 11.~novembra do 2.~decembra 2021 v~čase medzi 17.00 -- 19.00~hod. na Ambroseho~6 (vchod zboku).
\vfill\break


\clanok{Zbierky za október}
Milí bratia a sestry, ďakujeme za vašu obetavosť. V~mesiaci október ste prispeli:
\vskip-1ex\begitems
* misia: 382 €
* investičný fond: 382 €
\enditems


\clanok{Verš na zapamätanie}
Tento mesiac máme nový veršík, ktorý sa chceme spoločne učiť. Veríme, že poznanie Písma prospeje našej duši i našej mysli:

{\it „Ako ste teda prijali Ježiša Krista, Pána, tak v~Ňom žite zakorenení a v~Ňom budovaní, upevnení vo viere, ako ste sa naučili; rozhojňujte sa vo vzdávaní vďaky.“}

\autor{Kol~2,~6--7}


\n 2.	11.	Tomáš	VALCHÁŘ;
\n 5.	11.	Katarína	VALENTOVÁ;
\n 6.	11.	Elena	PRIBULOVÁ;
\n 6.	11.	Eva	SYČOVÁ;
\n 9.	11.	Ida	PUČEKOVÁ;
\n 9.	11.	Radovan	PAULEN;
\n 15.	11.	Bohumila	ŠALINGOVÁ;
\n 18.	11.	Jelka	NEVICKÁ;
\n 19.	11.	Dávid	PRIBULA;
\n 21.	11.	Ladislav	KAMOCSAI;
\n 22.	11.	Alena	SVOBODOVÁ;
\n 22.	11.	Peter	PRIBULA;
\n 23.	11.	Danny	JONES;
\n 25.	11.	Petra	ŠALINGOVÁ--FOSSETI;
\n 27.	11.	Judita	KOLÁŘIKOVÁ;
\n 29.	11.	Jaroslav	KRÁĽ;
\narodeniny


\program{
\p  1 ; po ;.;;.;;
\p  2 ; ut ;  9.30 ; Klubík (Zrínskeho 2) ; 15.15 ; Stret. pri Biblii (P. Pivka, Zrín. 2) ;
\p  3 ; st ; 17.30 ; Stretnutie sestier -- Svetový deň modlitieb ;.;;
\p  4 ; št ;  6.00 ; Modlitby -- muži (Zoom) ; 19.00 ; Biblická hodina (J. Szőllős) ;
\p  5 ; pi ; 17.30 ; Dorast (Súľovská 2) ;.;;
\p  6 ; so ;  9.00 ; Mládež (Súľovská 2) ;.;;
\p  7 ; ne ;  9.00 ; Bohoslužby (J. Szőllős) ; 9.00 ; Besiedka (Zrínskeho 2) ;
\p    ;    ; 10.30 ; Bohoslužby (J. Szőllős) ;.;;
\p  8 ; po ;.;;.;;
\p  9 ; ut ;  9.30 ; Klubík (Zrínskeho 2) ; 15.15 ; Stret. pri Biblii (P. Pivka, Zrín. 2) ;
\p 10 ; st ;.;;.;;
\p 11 ; št ;  6.00 ; Modlitby -- muži (Zoom) ; 19.00 ; Biblická hodina (J. Szőllős) ;
\p 12 ; pi ; 17.30 ; Dorast (Súľovská 2) ;.;;
\p 13 ; so ; 18.00 ; Mládež (Súľovská 2) ;.;;
\p 14 ; ne ;  9.00 ; Bohoslužby (P. Pivka) ; 9.00 ; Besiedka (Zrínskeho 2) ;
\p    ;    ; 10.30 ; Bohoslužby (P. Pivka) ;.;;
\p 15 ; po ;.;;.;;
\p 16 ; ut ;  9.30 ; Klubík (Zrínskeho 2) ; 15.15 ; Stret. pri Biblii (P. Pivka, Zrín. 2) ;
\p 17 ; st ; 17.30 ; Stretnutie sestier (D. Paštrnáková) ;.;;
\p 18 ; št ;  6.00 ; Modlitby -- muži (Zoom) ; 19.00 ; Biblická hodina (J. Szőllős) ;
\p 19 ; pi ; 17.30 ; Dorast (Súľovská 2) ;.;;
\p 20 ; so ; 18.00 ; Mládež (Súľovská 2) ;.;;
\p 21 ; ne ;  9.00 ; Bohoslužby (D. Jones) ; 9.00 ; Besiedka (Zrínskeho 2) ;
\p    ;    ; 10.30 ; Bohoslužby (D. Jones) ; 16.00 ; Zborové členské zhromaždenie ;
\p 22 ; po ;.;;.;;
\p 23 ; ut ;  9.30 ; Klubík (Zrínskeho 2) ; 15.15 ; Stret. pri Biblii (P. Pivka, Zrín. 2) ;
\p 24 ; st ;.;;.;;
\p 25 ; št ;  6.00 ; Modlitby -- muži (Zoom) ; 19.00 ; Biblická hodina (J. Szőllős) ;
\p 26 ; pi ; 17.30 ; Dorast (Súľovská 2) ;.;;
\p 27 ; so ; 18.00 ; Mládež (Súľovská 2) ;.;;
\p 28 ; ne ;  9.00 ; Bohoslužby (P. Kolárovský) ; 9.00 ; Besiedka (Zrínskeho 2) ;
\p    ;    ; 10.30 ; Bohoslužby (P. Kolárovský) ;.;;
\p 29 ; po ;.;;.;;
\p 30 ; ut ;  9.30 ; Klubík (Zrínskeho 2) ; 15.15 ; Stret. pri Biblii (P. Pivka, Zrín. 2) ;
}


\tiraz
\bye
