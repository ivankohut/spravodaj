%\typosize[10/12.5]% - pouzita velkost pisma/riadku - trochu vacsie
\input makra.tex % nacitanie Ivanom pripravenych nastaveni a prikazov
\hyphenation{star-šov-stvo} % rozdelenie slov na konci riadku, treba tu uviest slova, ktore sam nepozna

\spravodaj{12}{2020}


\clanok {Advent}
Naše vnúčatá v~USA sa už niekoľko týždňov tešili na príchod Clary. Vždy, keď sme si telefonovali, od radosti kričali, keď im Clara povedala, že už len toľko alebo toľko dní. Pre nich to bol vzácny advent – príchod Grammy (babky). Advent znamená príchod niekoho dôležitého. V~bežnom kontexte ho používame len v~súvislosti s~príchodom Ježiša na tento svet. Na advent a vianočné oslavy sa väčšinou tešíme. Ale tešil sa na advent aj Pán Ježiš? Nehovorím len o~tých štyroch decembrových týždňoch pred Vianocami, ale všeobecne o~Jeho príchode na tento svet. Aké boli tie dni pred začiatkom Jeho života v~lone Márie? O~čom sa s~Otcom porozprávali? Čakalo Ho veľa neznámeho. Aké to bude, keď bude uzavretý do tela zemskej bytosti, so všetkými obmedzeniami, chorobami a bolesťami? Dovtedy nezažil ani nádchu, ani nekašľal. Nikdy nezažil smrad, špinu a nefunkčnosť ničoho. Chudoba a nedostatok boli pre Neho cudzie.

Jedného dňa odložil ten svoj dokonalý raj a vzal na seba podobu služobníka a stal sa podobný ľuďom. Toto bol pre Neho advent. Deväť mesiacov čakal Boží Syn v~bruchu mladej ženy. Podľa mňa počas tých mesiacov nič nevnímal. Svoje právo vždy všetko vedieť odložil vo večnosti. A~potom prišla plnosť času; Boh poslal svojho Syna narodeného zo ženy, narodeného pod Zákonom, aby vykúpil tých, čo sú pod Zákonom, a aby sme dostali synovstvo.

Neprišiel do sveta, ktorý my poznáme. Prišiel do najchudobnejšej vrstvy obyvateľstva vtedajšej Galiley, do verejnej hanby týkajúcej sa príbehu jeho nevernej matky, do domu jednoduchého, nevzdelaného robotníka, do krajiny plnej chorých, posadnutých a zúfalých ľudí, ktorá bol pod útlakom Ríma. Nebolo to nič pekné, ale prišiel kvôli Tebe a kvôli mne. Možno sa nám zdá, že stav sveta je dnes lepší, vyspelejší a kultúrnejší. Avšak bieda tohto sveta sa časom veľmi nezmenila; hriech je len sofistikovanejší a zákernejší.

Advent 2020 nám pripomína, že stvorenstvo stále túžobne očakáva zjavenie Božích synov. Práve o~tom je advent. Prichádza Spasiteľ, Záchranca, naša Pomoc. Takého Spasiteľa v~roku 2020 naliehavo potrebujeme. Prežime tohtoročný advent ináč, s~očakávaním a radosťou. Určite bude iný, to už vieme, ale to neznamená, že bude horší. Očakávajme ten najlepší advent a Vianoce, aké sme kedy mali. Rozdiel je len vec perspektívy. Boh, ako dobrý Otec, nám pripravuje len to dobré, či s~Covidom 19 zápasíme, alebo nie. Ten Jeho dar sa okolnosťami nemení, lebo to najlepšie na Vianociach je dar Jeho Syna. Nech Ho tento rok spoznáme inak ako kedykoľvek predtým. Nech je advent 2020 „adventom Ježiša“ do našich domácností a sŕdc a nech nájdeme v~Jeho prítomnosti tú skutočnú radosť Vianoc.

\autor{Danny Jones}


\clanok {Správy zo staršovstva}
Bratia, sestry, milí priatelia,

pred rokom som písal o~tom, že sme koncom roku uskutočnili viacero podujatí. Rok~2020 je v~organizovaní zborových podujatí úplne iný. Sme vďační, ak môžeme mať spoločné bohoslužby v~našej modlitebni a o~aktivitách z~minulosti snívame a modlíme sa za to, aby sme mohli žiť plnohodnotný zborový život.

Prichádza advent a Vianoce. Chceli by sme slobodne oslavovať príchod nášho Spasiteľa Pána Ježiša Krista na tento svet. Ako to bude, zatiaľ nevieme. No bez ohľadu na situáciu, chceme s~radosťou v~srdci sláviť advent a Vianoce.

Pre vylepšenie elektronickej komunikácie nášho zboru smerom k~členom, ale aj ľuďom zo sveta, je pripravená nová webová stránka nášho zboru. Po doriešení posledných úprav bude spustená do prevádzky. Chceme sa poďakovať všetkým, ktorí sa podieľali na jej tvorbe, a veríme, že aj týmto spôsobom komunikácie budeme môcť zvestovať evanjelium ľuďom v~našom okolí.

18.~11.~2020 prebiehala konferencia delegátov zborov BJB prostredníctvom elektronickej komunikácie. Programom KDZ boli:
\vskip-1ex\begitems
* Schválenie výročných správ zložiek BJB za rok 2019
* Reštrukturalizácia Odboru misie
* Založenie Vzdelávacieho inštitútu BJB „EQUIP“
* Zriadenie Účelového zariadenia cirkvi Rodina
* Sociálno-misijný areál BJB v~Bernolákove
* Potvrdenie zrušenia nečinného účelového zariadenia BJB Domov J.~A.~Komenského
\enditems

Nakoľko sme nemohli zorganizovať ZČZ, na ktorom by členovia zboru vyjadrili svoj postoj k~jednotlivým bodom, staršovstvo zboru prijalo rozhodnutie a stanoviská k~jednotlivým bodom.

Myslime navzájom jeden na druhého. Modlime sa za tých, ktorí potrebujú naše modlitby. Nesme bremená jedni druhých.

Nech vás žehná Pán Ježiš Kristus a nech vás napĺňa svojím pokojom.

\autor {za staršovstvo Peter Pribula st.}


\clanok {Vianočný koncert na Palisádach}
Dňa 19.~decembra o~16.00 a 18.30~hod. sa v~našej modlitebni na Palisádach uskutočnia dva vianočné koncerty, na ktorých vystúpia:
\vskip-1ex\begitems
* Komorný zbor Bratislava Vocal Consort
* Sólisti: Alena Táborský a Karina Harmanová
* Klavírna spolupráca: Júlia Novosedlíková
* Zbormajsterka: Iveta Weis Viskupová
\enditems
a
\vskip-1ex\begitems
* The Hope Gospel Singers
* Klavírna spolupráca a dirigentka: Iveta Weis Viskupová
\enditems

Vstup na koncert je možný len so vstupenkou, ktorú môžete dostať po prihlásení sa cez online formulár: \ulink [https://bit.ly/33GXsi2]{bit.ly/33GXsi2}.


\clanok {Bohoslužby počas sviatkov}
\vskip-1ex\begitems
* 24. 12. 2020 -- 16.00 hod.
* 25. 12. 2020 -- 10.00 hod.
* 31. 12. 2020 -- 18.00 hod.
*  1.  1. 2021 -- 10.00 hod.
\enditems

Počas sviatkov bude účasť na zhromaždení regulovaná formou prihlasovania cez online formulár.
\vfill\break


\clanok {Služba ľuďom v~núdzi}
Od 23.~11. opäť zbierame zimné oblečenie. Čisté zimné oblečenie a obuv môžete priniesť každý pondelok v~čase od~17.00 do~19.00~hod. na Ambroseho~6 v~Petržalke (vchod za rohom).

Zbierame najmä zimné bundy, topánky, spodnú bielizeň, ponožky, rukavice, deky a spacáky. Prosím, nenoste nám v~tomto období letné oblečenie, nakoľko ho momentálne nemáme kde skladovať.

Prosím kontaktujte vopred Sylviu Vaniherovú, koordinátorku zbierky šatstva, mobil 0905~484~675. Ďakujeme!

Ak by ste aj počas karantény chceli aspoň na diaľku pomôcť a zapojiť sa do služby ľuďom bez domova, tu je spôsob ako: Hľadáme ochotných ľudí, ktorí by mali chuť ušiť a darovať rúška pre túto službu. Aj ľudia bez domova sa chcú chrániť a my by sme im ochranu v~podobe rúšok radi poskytli. Viac informácií o~možnosti zapojiť sa na t. č. 0948~115~515 alebo e-mail \email{antalikova@krestaniavmeste.sk}. Ďakujeme!

\autor {Kresťania v~meste}


\clanok {Ak potrebujete pomoc, napíšte nám!}
V našom zbore sme zriadili e-mailovú adresu \email{pomoc@bjbpalisady.sk}, na ktorú môžete napísať, ak ste sa dostali do zlej situácie alebo potrebujete nejakú pomoc. Takisto sa môžete ozvať, ak ste ochotní s~niečím pomôcť.
\vfill\break


\clanok{Verš na zapamätanie}
Tento mesiac máme nový veršík, ktorý sa chceme spoločne učiť. Veríme, že poznanie Písma prospeje našej duši i našej mysli:

{\it „Nemýľte sa, bratia moji milovaní. Každý dobrý údel a každý dokonalý dar pochádza zhora od Otca svetiel, v~ktorom niet premeny ani tieňa zmeny."}

\autor{Jk~1,~16--17}


\clanok{Zbierky za uplynulé obdobie}
Milí bratia a sestry,

v novembri ste prispeli:

\vskip-1ex\begitems
* Misia: 243,00 €
* Investície: 1000,00 €

\enditems

Ďakujeme vám, že napriek okolnostiam a neistým ekonomickým vyhliadkam do budúcnosti, ste mnohí prispeli na činnosť a službu zboru. Aj naďalej máte možnosť prispieť do „nedeľnej zbierky“, a to prevodom na účet zboru. Do poznámky pre prijímateľa, prosím, uveďte „zbierka“.

Bankové spojenie: SK36 0900 0000 0000 1147 1836, SWIFT: GIBASKBX

Ďakujeme!


\n 2.	12.	Helena	MIKLETIČOVÁ;
\n 3.	12.	Ľubica	IROVÁ;
\n 5.	12.	Tomáš	LAURENČÍK;
\n 6.	12.	Elise	ATKINS;
\n 9.	12.	Kamila	ZAJÍČKOVÁ;
\n 9.	12.	Ondrej	ŠKODÁK;
\n 11.	12.	Vladimíra	LAURENČÍKOVÁ;
\n 11.	12.	Maroš	KOHÚT;
\n 13.	12.	Peter	KOLÁROVSKÝ;
\n 16.	12.	Pavel	KONDAČ ml.;
\n 23.	12.	Diana	DZURIAKOVÁ;
\n 24.	12.	Slávka	VOLENTIČOVÁ;
\n 25.	12.	Dana	PELÍŠKOVÁ;
\n 29.	12.	Daniel	ŠALING;
\narodeniny


\program{
\p  1 ; ut ;.;;.;;
\p  2 ; st ;.;;.;;
\p  3 ; št ;.;;.;;
\p  4 ; pi ;.;;.;;
\p  5 ; so ;.;;.;;
\p  6 ; ne ;  9.00 ; Bohoslužby (D. Jones) ; 10.30 ; Bohoslužby (D. Jones) ;
\p  7 ; po ;.;;.;;
\p  8 ; ut ;.;;.;;
\p  9 ; st ;.;;.;;
\p 10 ; št ;.;;.;;
\p 11 ; pi ;.;;.;;
\p 12 ; so ;.;;.;;
\p 13 ; ne ;  9.00 ; Bohoslužby (J. Szőllős) ; 10.30 ; Bohoslužby (J. Szőllős) ;
\p 14 ; po ;.;;.;;
\p 15 ; ut ;.;;.;;
\p 16 ; st ;.;;.;;
\p 17 ; št ;.;;.;;
\p 18 ; pi ;.;;.;;
\p 19 ; so ; 16.00 ; Vianočný koncert ; 18.30 ; Vianočný koncert ;
\p 20 ; ne ;  9.00 ; Bohoslužby (D. Jones); 10.30 ; Bohoslužby (D. Jones) ;
\p 21 ; po ;.;;.;;
\p 22 ; ut ;.;;.;;
\p 23 ; st ;.;;.;;
\p 24 ; št ; 16.00 ; Bohoslužby -- Štedrý deň (P. Kolárovský) ;.;;
\p 25 ; pi ; 10.00 ; Bohoslužby -- Prvý sviatok vianočný (D. Jones) ;.;;
\p 26 ; so ;.;;.;;
\p 27 ; ne ;  9.00 ; Bohoslužby (Ľ. Dzuriak); 10.30 ; Bohoslužby (Ľ. Dzuriak) ;
\p 28 ; po ;.;;.;;
\p 29 ; ut ;.;;.;;
\p 30 ; st ;.;;.;;
\p 31 ; št ; 18.00 ; Bohoslužby -- Silvester ;.;;
}


\tiraz
\bye
