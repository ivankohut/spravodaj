% DOKUMENTACIA:

% Prazdny riadok za textom znamena ukoncenie odstavca.
% Cierne obldzniky na konci riadku (v PDF) - to nechaj na mna (moze to o.i. znamenat, ze treba pridat nejake slovo do \hyphenation, lebo ho sam nevie rozdelit na konci riadku)

% Prikazy pre casti spravodaja:
% \spravodaj{<mesiac>}{<rok>}
% \clanok{<nazov clanku>}
% \autor{<autor clanku>}
% \n<den.mesiac.meno> - zadefinovanie oslavenca
% \narodeniny - vytvorenie tabulky s~narodeninami vsetkych zadefinovanych oslavencov
% \tiraz - ukoncenie spravodaja tirazou

% Styl fontu:
% \bf - bold, plati do konca aktualne skupiny, napr. ak mas {aaa \bf bbb} ccc, tak aaa bude normalne, bbb bude bold, ccc bude normalne
% \it - italic (pouzit rovnakym sposobom ako \bf)
% \bi - bold italic (pouzit rovnakym sposobom ako \bf)
% \rm - normalne (pouzit rovnakym sposobom ako \bf)

% Dalsie prikazy a znaky:
% \begitems - zoznam (odrazky), informacie najdes na stranke http://petr.olsak.net/ftp/olsak/opmac/opmac-u.pdf#toc%3A.5
% \ulink[<cielova adresa]{<zobrazena adresa>} - klikatelny odkaz na webstranku
% \email{<adresa>} - klikatelny odkaz na e-mailovu adresu
% ~ - nedelitelna medzera, napr. v~dome, 21.~6.~2018
% -- - pomlcka (dvakrát -)
% „ - zaciatocna uvodzovka
% “ - koncova uvodzovka
% \noindent - najblizsi odstavec nebude odsadeny
% \vskip<velkost> - vertikalna medzera, napr. \vskip3pt alebo \vskip-3ex (zaporna medzera, t.j. posun smerom hore)


\input makra.tex % nacitanie Ivanom pripravenych nastaveni a prikazov
\hyphenation{star-šov-stvo} % rozdelenie slov na konci riadku, treba tu uviest slova, ktore sam nepozna

\spravodaj{3}{2018}


\clanok{Kristus vydal seba samého za nás}
{\it „Milosť vám a pokoj od Boha, nášho Otca, a od Pána Ježiša Krista, ktorý seba samého vydal za naše hriechy, aby nás podľa vôle Boha, Otca nášho, vytrhol z~tohto zlého veku. Jemu sláva naveky vekov. Amen.“} Galatským 1, 3 - 5

Predvčerom som dopálil adventné sviečky. Rozhodol som sa totiž, že ich budem ďalej používať a zapaľovať pri dennom čítaní Božieho slova. Možno sme ešte ani nestihli zabudnúť na Vianoce a už sa blíži Veľká noc a je tu obdobie pôstu. Vlastne ani by nebolo dobré, keby sme zabudli na Vianoce, na narodenie Božieho Syna ako človeka na túto zem. Veľká noc nám dáva zmysel Jeho príchodu a účinkovania na tejto zemi.

Apoštol Pavol to veľmi jasne a stručne píše hneď v~úvode svojho listu Galaťanom, ktorí sa od Ježiša obrátili k záslužným skutkom a obradom. Aj my máme stále tendenciu niečo si od Boha zaslúžiť vykonávaním nejakých obradov.

Preto je dôležité si neustále pripomínať, že v~základe našej záchrany nie sú naše náboženské výkony, ale milosť a pokoj, ktoré nám priniesol svojou obeťou Pán Ježiš Kristus. Priniesol nám pokoj alebo zmierenie s~Bohom a s~ľuďmi a dostávame tento pokoj od Neho z~milosti, bez akejkoľvek našej zásluhy.

Smrť Pána Ježiša Krista nebola len prejavom lásky alebo príkladom hrdinstva, ale predovšetkým obeťou za hriech. Dostávame záchranu vďaka tomu, že Pán Ježiš „vydal seba samého za naše hriechy“. Pán Ježiš Kristus nedal nič menšieho, než seba samého a to nie kvôli našej svätosti, nie preto, aby sme si mohli pochvaľovať, že sme svätí a spravodliví, ale dal seba za naše hriechy. Z toho jednoznačne vyplýva, že sme nenapraviteľní hriešnici, nemôžeme si zakladať na svojej spravodlivosti, na svojich skutkoch. Keď sme raz spoznali, že Kristus „dal seba samého za naše hriechy“, tak je nám jasné, že sme hriešnici, ktorí sa nemôžu sami spasiť.

Kristus zomrel, „aby nás vytrhol z~tohto zlého veku“, aby sme boli vytrhnutí, vyslobodení z~tohto terajšieho zlého veku, zo sveta, ktorý je ovládaný hriechom, sebastrednosťou a sebestačnosťou. Teda nie zo sveta (ako je to v~niektorých prekladoch). Božím cieľom nie je vziať nás z~tohto sveta,  musíme v~ňom zostať a byť „svetlom sveta“ a „soľou zeme“. Písmo rozdeľuje čas na dva veky: „tento vek“ a „vek, ktorý má prísť“. Hovorí nám, že tento druhý vek už prišiel, pretože Kristus ho vyhlásil, hoci prítomný vek ešte neskončil. Tak teda dva veky prebiehajú súčasne a vzájomne sa prelínajú. Naše obrátenie znamená vytrhnutie z~veku starého a prenesenie do nového, do veku, ktorý má prísť.  Pán Ježiš vydal seba samého za naše hriechy nielen preto, aby sme získali odpustenie, ale aby sme ty aj ja žili novým životom, životom veku budúceho.

Táto Kristova smrť nebola náhodná a tento spôsob nášho vyslobodenia nebol náhodný, ale bol výsledkom rozhodnutia „nášho Boha a Otca“, podľa Jeho zvrchovanej vôle. Rozhodne sa to neudialo podľa našej vôle, akoby sme sa zachránili sami. Ani sa to neudialo podľa vôle Kristovej, akoby Otec bol neochotný konať. V kríži sa vôľa Otca a Syna stretli v~dokonalej harmónii. Pavol píše, že Syn vydal seba samého (4a) a že jeho sebaobetovanie bolo podľa vôle nášho Boha a Otca (4b).
Je veľmi prirodzené, že za to všetko Pavol slávi Boha. Všetka sláva totiž patrí jedine Bohu a nikomu a ničomu inému. Oslavujme Boha svojím novým životom podľa Božej vôle nielen pred a počas sviatkov Veľkej noci. Oslavujme Ho preto, lebo Pán Ježiš dal seba samého za naše hriechy a vyslobodil nás pre život v~novom veku.

\autor{Ján Szőllős}


\clanok{Správy zo staršovstva}
Vo februári 2018 sme mali návštevu Dannyho a Claru Jonesovcov. Jeden celý týždeň sa niesol v~znamení väčšinou pracovných stretnutí, hlavne s~Dannym. Je nám ľúto, že Clara nemohla absolvovať všetky stretnutia, ktoré boli plánované aj pre ňu. Ale veríme tomu, že Pán Ježiš vie najlepšie, prečo to bolo tak.

Začiatkom mesiaca sme sa stretli na staršovstve aj s~Dannym. Rozprávali sme o~týchto témach:
\begitems
* hospodárenie zboru v~roku 2017 a návrh rozpočtu na rok 2018,
* skupinky v~zbore -- Dannyho zaujímalo, či je do skupiniek zapojený celý zbor alebo či sa skupiniek zúčastňujú iba tí, ktorí majú záujem,
* projekt pastoračnej starostlivosti o~seniorov. Hodnotili sme, že projekt v~minulom období bol užitočný a mal aj evanjelizačný rozmer. Zároveň sme odsúhlasili podanie návrhu na pokračovanie tohto projektu.
* plánovanie Veľkonočného koncertu spevokolu. V spolupráci so Slávom Kráľom sme sa dohodli, že koncert pri príležitosti Veľkej noci bude 25. marca.
* pastoračné otázky.
\enditems
Na druhom stretnutí sme dali priestor tým, ktorí mali záujem rozprávať o~hospodárení zboru a zároveň tým, ktorí chceli podať návrhy do rozpočtu na rok 2018. Spolu s~týmito témami sme rozprávali o~legislatívnych požiadavkách týkajúcich sa cirkvi a aj jednotlivých zborov. Časť diskusie sme venovali téme webovej stránke, jej výzoru a obsahu. Nakoľko sa rekonštruuje byt na Zrínskeho, dostali sme informácie o~stave rekonštrukčných prác. Pripravovali sme aj témy na Výročné celozborové členské zhromaždenie, ktoré bude 11.~marca. Ešte sme robili program služieb v~Bratislave a na Chvojnici.

\autor{Za staršovstvo zboru Peter Pribula}


\clanok{Výročné celozborové členské zhromaždenie}
Staršovstvo zboru BJB Bratislava – Palisády zvoláva podľa platného zborového poriadku
\vskip1ex
\centerline{\bf Výročné celozborové zhromaždenie členov zboru}
\centerline{Dátum: 11. 3. 2018 o~15.30 hod.}
\centerline{Miesto konania: modlitebňa zboru Palisády 27/A}

Program:
\begitems \style N
* Štatistiky zboru
* Správa o~hospodárení zboru za rok 2017
* Správa Revíznej komisie zboru
* Rozpočet zboru na rok 2018
* Diskusia k finančnej správe a návrhu rozpočtu
* Rôzne
\enditems
Účasť členov zboru je potrebná!


\clanok{Pro Christ 2018}
Aj tento rok sa pod heslom {\bf „NEUVERITEĽNÉ?“} uskutoční ďalší evanjelizačný týždeň PROCHRIST LIVE – od pondelka 12. do nedele 18. marca 2018.
Centrálne podujatie bude v~Lipsku – s~možnosťou prenosu programu jednotlivých večerov so slovenským dabingom. Hlavnou témou bude Apoštolské vierovyznanie, ktoré nás spája s~ostatnými kresťanskými konfesiami: Viera v~Boha, ktorý stvoril svet a ktorý každého človeka nekonečne miluje.

Tematický týždeň:
\begitems
* 12. 3. 2018	Pondelok	–   Boh ma miluje
* 13. 3. 2018	Utorok	–   Boh ma chce
* 14. 3. 2018	Streda	–   Boh ma hľadá
* 15. 3. 2018	Štvrtok	–   Boh ma zachraňuje
* 16. 3. 2018	Piatok	–   Boh ma obdaruje
* 17. 3. 2018	Sobota	–   Boh ma motivuje
* 18. 3. 2018	Nedeľa	–   Boh ma očakáva
\enditems
Priamych prenosov sa môžete zúčastniť v~Novom evanjelickom kostole na Legionárskej ul. v~Bratislave, každý deň od 18.00 hod.


\clanok{Veľkonočný koncert nášho spevokolu}
Srdečne vás pozývame na Veľkonočný koncert, ktorý sa uskutoční {\bf 25.~marca 2018 o~17.00~hod.} v~žltom kostolíku na Palisádach. Účinkuje veľký spevokol I. zboru BJB Bratislava a komorný orchester. Príďte zažiť jedinečnú atmosféru Veľkej noci a nezabudnite pozvať aj svojich priateľov a známych.


\clanok{Spoločné bohoslužby na Veľký piatok}
Koncom mesiaca budú Veľkonočné sviatky, počas ktorých si pripomíname najvýznamnejšie udalosti v~našich dejinách – ukrižovanie, smrť a zmŕtvychvstanie Pána Ježiša Krista. Udalosti, ktoré hlboko prenikli a zmenili životy veriacich v~Ježiša Krista, pretože spolu s~ním sme boli pochovaní v~smrti, aby sme tak, ako bol slávou Otca vzkriesený z~mŕtvych Kristus, aj my sme vstúpili na cestu nového života.

Tieto slávne Božie skutky si pripomíname na spoločných bohoslužbách viacerých bratislavských cirkevných spoločenstiev, ktoré sa uskutočnia na Veľký piatok {\bf 30.~3.~2018 od~10.00 do~12.00~hod.} vo Veľkej sále Istropolisu. Paralelne budú prebiehať aj bohoslužby pre deti vo veku 4 -- 12 rokov.


\clanok{Bohoslužby na Veľkú noc}
Na Veľký piatok sa stretneme na Palisádach o~17.00 hod. Poslúži nám brat Peter Kolárovský. Na Veľkonočnú nedeľu sa stretneme o~9.30 hod. a bude nám slúžiť brat kazateľ Ján Szőllős.


\clanok{Senior klub}
Srdečne vás pozdravujem a oznamujem vám, že Senior klub v~mesiaci marec nebude, nakoľko je Zelený štvrtok tesne pred Veľkou nocou.

Ak dá Pán zdravia a života, stretneme sa až posledný štvrtok v~mesiaci apríl, a to dňa {\bf 26.~4.~2018 od~10.00 do~14.00~hod.} na Súľovskej ul. Tému určíme dodatočne. Ďakujem!

\autor{Jana Makovíniová}


\clanok{Ponuka táborov Detskej misie}
Ak premýšľate, ako by ste vašim deťom vyplnili letné prázdniny, vyberte si niektorý z~turnusov, ktoré ponúka Detská misia, o. z. v~stredisku Prameň pod hradom Červený Kameň.
\begitems
* 7. -- 14. júla 2018 – deti (7 -- 11 rokov)
* 14. -- 21. júla 2018 – športový tábor (9 -- 12 rokov)
* 21. -- 28. júla 2018 – deti (7 -- 11 rokov)
* 28. júla -- 4. augusta 2018 – starší dorast (15 -- 18 rokov)
* 4. -- 11. augusta 2018 – mladší dorast (12 -- 15 rokov)
* 11. -- 18. augusta 2018 – deti (7 -- 11 rokov)
\enditems
Viac informácií a elektronickú prihlášku nájdete na \ulink[http://www.detskamisia.sk]{www.detskamisia.sk}.


\clanok Kalendár najbližších akcií
\begitems
* Československá mládežnícka konferencia, 16. -- 18. marca 2018, Banská Bystrica
* Spolu v~misii, 27. -- 28. apríla 2018, chata Račkova dolina
* Deň misie BJB, 30. apríla 2018
* Konferencia sestier, 4. -- 5. mája 2018, Košice
\enditems


\clanok{Pomoc ľuďom bez domova}
Milé sestry a milí bratia, varenie polievok ľuďom v~núdzi pokračuje. Prosím, ak sa chcete zapojiť, vyberte si z~nasledujúcich termínov:

17. apríl (utorok); 15. máj (utorok); 19. jún (utorok); 17. júl (utorok); 21. august (utorok); 18. september (utorok); 20. október (sobota); 17. november (sobota); 15. december (sobota)

Na rezervované termíny sa, prosím, nahláste u mňa.

\autor{Beata Bogárová}


\clanok{Zbierky za január}
Milí bratia a sestry, ďakujeme za vašu obetavosť. V mesiaci február ste prispeli:
\vskip-1ex\begitems
* misia: 197,40 €
* Ukrajina-Kidesh: 781,00 €
* investičný fond: 538,50 €
\enditems


\clanok{Klub D.E.P.O. na Súľovskej}
D.E.P.O. junior je v~piatok od 16.00 do 18.00 hod. a je určený hlavne dorastu. D.E.P.O. je každý piatok od~18.00 do~22.00 hod. a je určený mládeži. Náplňou klubov sú moderné spoločenské hry, stolný futbal, pingpong, posedenie, rozhovory, hudba, občerstvenie, tvorivé dielne.


\clanok{Skupina anonymných alkoholikov Maják}
AA-liečebný program anonymných alkoholikov má svoje pravidelné stretnutia vo štvrtok od 17.30 do 18.30 hod. v~našich zborových priestoroch na Zrínskeho 2.

\autor{Kontakt: Želka 0903 294 927}


\n 3.	3.	Elena	BUZÁŠOVÁ;
\n 10.	3.	Rada	VULIĆ;
\n 15.	3.	Jozef	DOBA;
\n 18.	3.	Kamil	ŠALING;
\n 20.	3.	Jana	MÁŤUŠOVÁ;
\n 25.	3.	Alžbeta	PAULENOVÁ;
\n 25.	3.	Pavol	ŠKULEC;
\n 25.	3.	Filip	KOVÁČ;
\n 26.	3.	Matej	KOLÁŘIK;
\n 28. 3.	Marta	PRIBULOVÁ ml.;
\n 29. 3.	Marcel	MAĎAR;
\n 29. 3.	Peter	PRIBULA ml.;
\n 30. 3.	Marta	GULDANOVÁ;
\n 31.	3.	Judit	KOBZOVÁ;
\narodeniny


\program{
\p 1  ; št ; 19.00 ; Biblická hodina (J. Szőllős, Zrínskeho 2);.;;
\p 2  ; pi ; 16.00 ; D.E.P.O. (Súľovská 2);.;;
\p 3  ; so ; 18.00 ; Mládež (Súľovská 2);.;;
\p 4  ; ne ; 9.30 ; Bohoslužby (J. Szőllős);.;;
\p 5  ; po ; 18.00 ; Modlitby (Zrínskeho 2);.;;
\p 6  ; ut ; 15.00 ; Popoludnie pri Biblii (P. Pivka, Zrínskeho 2);.;;
\p 7  ; st ;.;;.;;
\p 8  ; št ; 19.00 ; Biblická hodina (J. Szőllős, Zrínskeho 2);.;;
\p 9  ; pi ; 16.00 ; D.E.P.O. (Súľovská 2);.;;
\p 10  ; so ; 18.00 ; Mládež (Súľovská 2);.;;
\p 11  ; ne ; 9.30 ; Bohoslužby (T. Valchář); 10.00; Chvojnica (P. Škulec);
\p 12  ; po ; 18.00 ; Modlitby (Zrínskeho 2);.;;
\p 13  ; ut ; 15.00 ; Popoludnie pri Biblii (P. Pivka, Zrínskeho 2);.;;
\p 14  ; st ;.;;.;;
\p 15  ; št ; 19.00 ; Biblická hodina (J. Szőllős, Zrínskeho 2);.;;
\p 16  ; pi ; 16.00 ; D.E.P.O. (Súľovská 2);.;;
\p 17  ; so ;.;;.;;
\p 18  ; ne ; 9.30 ; Bohoslužby (T. Atkins); 10.00; Chvojnica (R. Hovorka);
\p 19  ; po ; 18.00 ; Modlitby (Zrínskeho 2);.;;
\p 20  ; ut ; 15.00 ; Popoludnie pri Biblii (P. Pivka, Zrínskeho 2);.;;
\p 21  ; st ;.;;.;;
\p 22  ; št ; 19.00 ; Biblická hodina (J. Szőllős, Zrínskeho 2);.;;
\p 23  ; pi ; 16.00 ; D.E.P.O. (Súľovská 2);.;;
\p 24  ; so ; 18.00 ; Mládež (Súľovská 2);.;;
\p 25  ; ne ; 9.30 ; Bohoslužby (P. Buzáš); 17.00; Veľkonočný koncert;
\p 26  ; po ; 18.00 ; Modlitby (Zrínskeho 2);.;;
\p 27  ; ut ; 15.00 ; Popoludnie pri Biblii (P. Pivka, Zrínskeho 2);.;;
\p 28  ; st ;.;;.;;
\p 29  ; št ; 19.00 ; Biblická hodina (J. Szőllős, Zrínskeho 2);.;;
\p 30  ; pi ; 17.00 ; Veľký piatok (P. Kolárovský);.;;
\p 31  ; so ;.;;.;;
}

\tiraz
\bye
