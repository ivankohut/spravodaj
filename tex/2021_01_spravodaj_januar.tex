%\typosize[10/12.5]% - pouzita velkost pisma/riadku - trochu vacsie
\input makra.tex % nacitanie Ivanom pripravenych nastaveni a prikazov
\hyphenation{star-šov-stvo} % rozdelenie slov na konci riadku, treba tu uviest slova, ktore sam nepozna

\spravodaj{1}{2021}


\clanok {Hospodin upevňuje kroky človeka}
Neviem, čo Ti  hovoria písmenká Y2K. Už pred 21 rokmi sme sa pripravovali na „koniec sveta“. Bol Silvester 1999 a nevedeli sme, či počítače zvládnu tú zmenu na číslo 2000. Bola obava, že zlyhaním počítačov by sa zrútili elektrárne a havarovali by lietadlá. Ľudia si kupovali veľké zásoby jedla a skladovali litre vody. Priznám sa, že niekoľko litrov vody sme v~Žiline mali pripravené aj my... len pre istotu. Bolo 00:01 a nič. Čakali sme a čakali a stále nič. Dlho sme mali vodu na polievanie kvetín.

Každý bežný nový rok nám nie je jasné, čo nás čaká. V~týchto dňoch sa nám však zdá, že vyhliadky sú ešte neistejšie. Samozrejme, je toho veľa, čo v~roku 2021 je ešte neznáme. Treba si dávať pozor, aby sme svoju dôveru nevkladali do niečoho neodskúšaného. Určite sa dám zaočkovať, ale dôveru musíme mať inde. Chvála Pánovi, máme dôvod pre istejšiu nádej ako Pfizer--BioNTech. Čítame o~tom v~Ž 37,23--24: {\it „Hospodin upevňuje kroky človeka, v~jeho ceste má záľubu. Ak spadne, nezostane ležať, lebo mu Hospodin podoprie ruku.“}

V týchto veršoch je napísaná základná pravda života. Boh sám upevňuje. Niekedy dávame dôraz na všetko možné, ale v~konečnom dôsledku sám Boh upevňuje kroky človeka. Prečo to robí? Lebo nás miluje. Kroky Tvojho života sú Bohu vzácne. S~očakávaním sleduje Tvoj každý krok, lebo Ťa stvoril a raduje sa z~Teba; nie z~toho, čo robíš, ale z~toho, že vôbec si. Jeho lásku si nemusíme zaslúžiť, podobne ako si Tvoje dieťa nemusí zaslúžiť Tvoju lásku. Ľúbiš ho napriek všetkému. Boh, ktorý je ešte láskavejší, Ťa takisto miluje; napriek všetkému. Preto má záľubu v~Tvojej ceste. S~touto istotou vstupujeme do roka 2021.

Aká je v~tomto všetkom naša úloha? Dôverovať. Dôverovať Mu, že je skutočne taký a že upevňuje naše budúce kroky. Prečo ich musí upevňovať? Lebo nás pozná, že sme slabí, krehkí a neistí. Pán nie je z~toho pohoršený, ale skôr ešte viac motivovaný, aby do toho spolu s~nami vstúpil.

To, čo nás čaká v~roku 2021, je dobré, lebo to, čo nás čaká, je od verného Pána, ktorý nám zo svojej lásky pripravuje to najlepšie. Tešme sa! Boh je dobrý a chce nám to v~roku~2021 ukázať. Budeme to vidieť.

\autor{Danny Jones}


\clanok {Správy zo staršovstva}
Pri posledných stretnutiach staršovstva je prioritnou témou zabezpečenie života zboru v~obmedzených podmienkach pre šíriaci sa vírus.

Advent, Vianoce, záver roku. Sviatky, na ktoré sa tešia aktívni aj pasívni veriaci. Sviatky, kedy viac ako inokedy sú ľudia ochotní počúvať evanjelium a dôvod príchodu Božieho Syna na svet. Narodenie nevinného dieťaťa, narodenie Božieho Syna. Jeho príchod pred 2000~rokmi prežívali ľudia v~panike, ale aj úplne ľahostajne nevnímajúc, že sa niečo deje.

Rozdelenie spoločnosti bolo vtedy také isté ako dnes. Niekto vníma dôsledky pandémie a potrebu urobiť niečo pre zastavenie šírenia vírusu veľmi intenzívne a iní zastávajú názor, že nie je potrebné robiť nič, alebo až bagatelizujú existujúci problém  a tvária sa, akoby neexistoval.

Realitou života je to, že až do konca januára~2021, sa nemôžeme schádzať na spoločné stretnutia. Čo bude po tomto termíne, vie iba náš nebeský Otec.

Vzájomné vzdialenie sa, strata osobných kontaktov, odlúčenie vedie k~ničeniu vzájomných vzťahov. Táto situácia ohrozuje naše vzájomné vzťahy, ale aj vzťah s~naším Spasiteľom Pánom Ježišom Kristom.

To, k~čomu vás chceme vyzvať, je, aby sme sa navzájom povzbudzovali. Tak ako je nám to vlastné, modlitbou jeden za druhého, telefonátom, e-mailom, … Majme na pamäti, že našou vzájomnou zodpovednosťou je {\it „niesť bremená jedni druhých“}. Božie Slovo nás v~knihe Zjavení Jána volá: {\it „Buď verný až do smrti a dám ti veniec života!“}

Žime preto aj v~tejto zvláštnej dobe tak, ako je hodné povolania, ktorým sme boli povolaní.

\autor {za staršovstvo Peter Pribula st.}
\vfill\break


\clanok {Kairos}
Je rok 2021 -- čas Kairosu.

Svet, do ktorého sa Pán Ježiš narodil, bol rovnako chaotický ako ten, v~ktorom sa my dnes nachádzame. Akoby ľudský život nemal takmer žiadnu hodnotu, politický odboj bol bežnou vecou, spravodlivosť určovali túžby tých, čo boli pri moci, a chudoba, choroby, vojny a hladomor postihovali veľké množstvo ľudských životov. Do takéhoto sveta sa Boh Otec rozhodol poslať svojho Syna v~podobe bezmocného dieťaťa. V~prvých mesiacoch jeho života sa zlo pokúša ho zabiť, takže sa stáva utečencom v~cudzej krajine. Možno Boh jednoducho takto od začiatku zaisťoval, aby posolstvo jeho Syna bolo vždy misijným posolstvom. Vianočný príbeh Božieho vykúpenia sa vo svojom počiatočnom zámere zavŕšil v~nasledujúcich 33~rokoch, keď sa z~dieťaťa v~jasliach stal Spasiteľ sveta. Spasenie bolo vykúpené. Odpustenie bolo udelené. Životy mužov a žien boli premenené láskou, milosťou a prebývajúcim Svätým Duchom. A~vďaka odovzdaniu sa tých, ktorí uverili, bol v~nasledujúcich 100 rokoch ovplyvnený pohanský svet od Španielska až po Čínu.

Ešte nikdy nebolo tak potrebné oživenie posolstva evanjelia, ako je tomu teraz. Vždy sa to deje len vďaka vernosti tých, ktorí chodia v~Božej moci. Boh vzbudzuje na tento účel poslov z~rôznych kútov sveta, odkiaľ by sme to nečakali. Brazílčania a Guatemalčania nesú evanjelium na východ. Číňania a Kórejčania nesú evanjelium na západ. A~Slovensko sa ocitá uprostred nového Božieho hnutia vo svetovej misii. Vidím na Slovensku počiatočný zápal pre šírenie evanjelia. Píšu sa prvé kapitoly slovenského misijného hnutia a verím, že aj náš zbor v~ňom môže zohrávať úlohu. Na to sa však musíme pripraviť. Pripravíme sa tak, že sa v~modlitbe spýtame: „Pane, ukáž mi, ako Ti môžeme slúžiť,“ a potom žijeme každodenný život s~misijným zámerom. Nie je nič veľkolepejšie, ako byť súčasťou Božej stratégie na šírenie Ježišovho mena, či v~Bratislave alebo v~iných národoch. Keby sme tak začali žiť, náš zbor a mesto Bratislava by sa zmenili.

Môžeme sa pripraviť aj prostredníctvom kurzu Kairos. Kairos je kurz, ktorý nám pomôže lepšie porozumieť tomu, čo v~tomto svete Boh robí a prečo to robí. Kairos nám pomôže žiť každodenný život s~jasným zámerom. Pomocou kurzu Kairos môžeme takisto vidieť, čo Boh koná v~národoch sveta a ako sa môžeme k~tomuto dielu pripojiť.

Kurz Kairos bude prebiehať počas dvoch víkendov. V~Bratislave je plánovaný v~termínoch 11.~--~13. a 18.~--~20.~februára. Viac informácií nájdete na \ulink [https://www.facebook.com/KurzKairos/]{facebook.com/KurzKairos}.
\vfill\break


\clanok {Doplňujúce voľby na miesto podpredsedu Rady BJB v~SR}
Na jesennej Konferencii delegátov zborov neprebehli doplňujúce voľby na uvoľnené miesto jedného podpredsedu Rady BJB v~SR, nakoľko sa nenašiel ani jeden kandidát, ktorý by kandidatúru prijal. Chceme vás preto požiadať, aby ste svoje návrhy na toto miesto komunikovali kazateľovi resp. ktorémukoľvek členovi staršovstva zboru, a to najneskôr do 20.~januára~2021.


\clanok {Ak potrebujete pomoc, napíšte nám!}
V našom zbore sme zriadili emailovú adresu \email{pomoc@bjbpalisady.sk}, na ktorú môžete napísať, ak ste sa dostali do zlej situácie alebo potrebujete nejakú pomoc. Takisto sa môžete ozvať, ak ste ochotní s~niečím pomôcť.


\clanok{Verš na zapamätanie}
Tento mesiac máme nový veršík, ktorý sa chceme spoločne učiť. Veríme, že poznanie Písma prospeje našej duši i našej mysli:
\vskip1ex
{\it Veď ma schová vo svojom úkryte v~deň pohromy.

Ukryje ma v~skrýši svojho stanu,

postaví ma vysoko na skalu.

Už teraz pozdvihne moju hlavu

nad nepriateľov vôkol mňa.

V jeho stane prinesiem obety s~plesaním.

Spievať a hrať budem Hospodinovi.}

\autor{Ž~27,~5 -- 6}
\vfill\break


\clanok{Zbierky za uplynulé obdobie}
Milí bratia a sestry,

v decembri ste prispeli:

\vskip-1ex\begitems
* Misia: 413,60 €
* Investície: 180,00 €

\enditems

Ďakujeme vám, že napriek okolnostiam a neistým ekonomickým vyhliadkam do budúcnosti, ste mnohí prispeli na činnosť a službu zboru. Aj naďalej máte možnosť prispieť do „nedeľnej zbierky“, a to prevodom na účet zboru. Do poznámky pre prijímateľa, prosím, uveďte „zbierka“.

Bankové spojenie: SK36 0900 0000 0000 1147 1836, SWIFT: GIBASKBX

Ďakujeme!


\n 2.	1.	Ivan	KOHÚT;
\n 3.	1.	Ľubomír	KEŠJAR;
\n 4.	1.	Pavel	VAJO;
\n 11.	1.	Nataša	HOVORKOVÁ;
\n 16.	1.	Blahoslava	BETKOVÁ;
\n 18.	1.	David	VALCHÁŘ;
\n 19.	1.	Andrej	CIHO;
\n 22.	1.	Jana	LAURENČÍKOVÁ;
\n 22.	1.	Jana	ČAHOJOVÁ;
\n 23.	1.	Elena	GUBOVÁ;
\narodeniny


\program{
\p  1 ; pi ; 10.30 ; Bohoslužby (D. Jones, online) ;.;;
\p  2 ; so ;.;;.;;
\p  3 ; ne ; 10.30 ; Bohoslužby (Robi a Ivka Petrilákovci, online) ;.;;
\p  4 ; po ;.;;.;;
\p  5 ; ut ;.;;.;;
\p  6 ; st ;.;;.;;
\p  7 ; št ;.;;.;;
\p  8 ; pi ;.;;.;;
\p  9 ; so ;.;;.;;
\p 10 ; ne ; 10.30 ; Bohoslužby (D. Jones, online) ;.;;
\p 11 ; po ;.;;.;;
\p 12 ; ut ;.;;.;;
\p 13 ; st ;.;;.;;
\p 14 ; št ;.;;.;;
\p 15 ; pi ;.;;.;;
\p 16 ; so ;.;;.;;
\p 17 ; ne ; 10.30 ; Bohoslužby (J. Szőllős, online) ;.;;
\p 18 ; po ;.;;.;;
\p 19 ; ut ;.;;.;;
\p 20 ; st ;.;;.;;
\p 21 ; št ;.;;.;;
\p 22 ; pi ;.;;.;;
\p 23 ; so ;.;;.;;
\p 24 ; ne ; 10.30 ; Bohoslužby (D. Jones, online) ;.;;
\p 25 ; po ;.;;.;;
\p 26 ; ut ;.;;.;;
\p 27 ; st ;.;;.;;
\p 28 ; št ;.;;.;;
\p 29 ; pi ;.;;.;;
\p 30 ; so ;.;;.;;
\p 31 ; ne ; 10.30 ; Bohoslužby (P. Kolárovský, online) ;.;;
}


\tiraz
\bye
