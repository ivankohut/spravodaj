%\typosize[10/12.5]% - pouzita velkost pisma/riadku
\input makra.tex % nacitanie Ivanom pripravenych nastaveni a prikazov
\hyphenation{star-šov-stvo} % rozdelenie slov na konci riadku, treba tu uviest slova, ktore sam nepozna

\spravodaj{4}{2019}


\clanok {Záleží na zmŕtvychvstaní?}
{\it „Ja som vzkriesenie a život. Kto verí vo mňa, bude žiť, aj keď umrie. Nik neumrie naveky, kto žije a verí vo mňa. Veríš tomu?“ Ján 11,~25}

Ľudia nevedia, čo si počať so zmŕtvychvstalým Kristom... a tak sa Ho snažia okrem iného vyhlásiť za dobrého učiteľa -- lenže dobrý učiteľ takéto vyhlásenia nerobí...
Pri Ježišovi sa totiž splnilo to, čo predpovedal...

Je v~tom teda nejaký rozdiel,  či Ježiš vstal alebo zostalo Jeho telo hniť v~hrobe? No určite je -- pretože pokiaľ sa v~Ježišovom živote naplnilo všetko to, čo o~sebe predpovedal,  potom to má nesmierny dopad aj práve na ten môj život. V~čom? Takže pokiaľ je Ježiš tým, za koho sa prehlasoval a pokiaľ sa pri Ňom naplnilo to, čo predpovedal, potom to zásadne ovplyvňuje moju minulosť, prítomnosť aj budúcnosť:

\noindent {\bf 1. Tvoja minulosť môže byť odpustená}

Ocitol si sa v~situácii, kedy si na niečom pracoval a keď si už bol za polovicou, tak si si povedal: „Keby som tak mohol začať odznova!“? Napr. pri maľovaní bytu, písaní diplomovky a pod. Poznám ľudí, ktorí si takto povzdychli nad životom samotným: „Keby som len mohol začať odznovu!“ Všetci máme v~živote veci, na ktoré nie sme hrdí a najradšej by sme boli, keby sa niektoré z~nich nikdy nestali.

Áno, každý z~nás už pocítil ľútosť nad niečím, čo v~živote spackal; a nikto z~nás nie je bez viny. Moje ženy rady sledujú seriál {\it Vraždy v~Midsomeri}. V~jednej z~epizód, keď bol odhalený vrah, povedal tieto slová: „Som rád, že už nemusím nič predstierať, pretože ten pocit viny ma už devastoval.“ Tragédiou je, že takýchto ľudí (ľudí, ktorí si nevedia rady s~prítomnosťou a nevedia vzhliadnuť k~budúcnosti, pretože zostali zaseknutí v~minulosti) osobne poznám. Vina, neurovnané vzťahy, ľútosť, minulé vzťahy, ktoré mixujú so súčasnými, spoločne s~výrokmi typu „toto si už ponesiem so sebou do konca svojho života“ ich zväzujú a ťahajú ku dnu. Snažia sa žiť život, v~ktorom vôbec nie sú šťastní.

Ale mám dobrú správu: Biblia hovorí, že Ježiš bol pribitý na kríž a zaplatil za mňa. Ja už teda nemusím platiť. On to vzal -- moju vinu -- na seba namiesto mňa, a mne mohlo byť odpustené. Či veríš tomu? Koľko času venuješ šeku, ktorý treba zaplatiť? Kým ho nezaplatíš... Koľko sa zvykneš zaoberať zaplateným ústrižkom??? JE ZAPLATENÉ a šlus, koniec -- už ma to nezaujíma. Biblia hovorí: „Nie je už viac odsúdenia pre tých, ktorí sú v~Kristovi.“

Pokiaľ sa teda pri Ježišovi naplnili slová, ktoré o~sebe predpovedal a ja sa môžem na Neho spoľahnúť, tak moja minulosť môže byť odpustená... ak o~to stojím.

\noindent {\bf 2. Tvoja prítomnosť môže byť zvládnuteľná}

Väčšina života je nezmanažovateľná. Múdry je ten, kto chápe, že život, ktorý nám bol daný, ja sám neukočírujem. Boh to však zvládne. Neviem mať všetko vo svojom živote pod kontrolou, ale Boh riadi každú bunku vo vesmíre. A~to je skvelá správa! Cítim sa bezmocný ohľadne tej a tej situácie, neviem sa nijakovsky zbaviť môjho zlozvyku či závislosti, nech sa snažím sebalepšie, nevládzem splácať úver, neviem si zmanažovať  čas. Len niekoľko výkrikov do tmy... Je tu však niekto, ktorý je väčší a schopnejší ako si ty sám.

{\it „Aká nesmierne veľká je Jeho moc pre nás, ktorí veríme tak, ako v~nás pôsobí Jeho mocná sila. Tú dokázal na Kristovi, keď Ho vzkriesil z~mŕtvych a posadil v~nebi po svojej pravici nad každé kniežatstvo, moc, silu a panstvo a nad každé meno, ktoré sa spomína nielen v~tomto, ale aj v~budúcom veku.“ Efezským 1,~20}

Ty a ani ja nevieme, čo nám prinesie zajtrajšok, ale nakoniec na tom nezáleží, pretože hoci aj zajtrajšok je mimo mojej kontroly, nie je mimo Božieho dozoru. A~Boh sám mi dáva silu čeliť dnešku aj všetkým zajtrajškom. Pre Boha nie je nič nemožné!

\noindent {\bf 3. Z~budúcnosti nemusíš mať obavy}

Pravdou je, že univerzálnym problémom, do ktorého je namočené celé ľudstvo ohľadne budúcnosti, je smrť! Každý raz zomrie -- ako ja, tak aj ty. A~len blázon sa prechádza životom s~úsmevom na tvári bez toho, aby sa pripravil na nevyhnutné.

Istý muž sa stal členom cisárskeho orchestra v~Číne napriek tomu, že nevedel hrať. Kedykoľvek bol nácvik tohto telesa, údajne len priložil svoju flautu k~perám a predstieral, že na nej usilovne hrá. (Nikto si to údajne nevšimol a tak si nejaký čas užíval benefity, ktoré mu táto práca ponúkala.) V~jeden deň však mal navštíviť skúšku orchestra sám cisár a požiadať každého muzikanta o~sólo. Náš muzikant zbledol ako stena. A~v deň svojho sóla požil jed, ktorým sa otrávil. Odvtedy sa medzi čínskymi hudobníkmi rozšírila fráza „Nevládal čeliť hudbe.“

Niektorí z~nás tu dnes len predstierajú, že sú členmi Božieho orchestra. Zamiešajú sa do davu a splynú s~ním. Nikto si to nevšimne, pretože rozprávajú správne veci -- používajú vhodnú terminológiu okorenenú kresťanským slovníkom, navštevujú miesta, ktoré sú odobrené a zdržiavajú sa okolo tých správnych ľudí. Raz však nastane čas, kedy bude potrebné „čeliť hudbe“. Niektorí sa v~ten deň budú radovať a iní budú zhrození, pretože gamblovali a hrali vabank so svojou budúcnosťou.

Ako sa vlastne dá prestať obávať smrti? Tým, že uzavrieš mier s~Bohom! Keď budeš mať pokoj s~Bohom, až vtedy porozumieš tajomstvu Veľkej noci. Vložiť svoju dôveru v~zmŕtvychvstalého Krista a rozhodnutie Ho nasledovať začína vytvárať podstatu nového vzťahu, ktorý je podmienkou k~vstupu do prítomnosti živého Boha -- Otca.

Nedávno som sa zúčastnil konferencie: {\it Učíme pre život}. V~sobotu ráno o~9.15 v~hlavnej konferenčnej sále zazneli tieto slová: „Vážené dámy, vážení páni, prosím povstaňte -- prichádza prezident Slovenskej republiky, Andrej Kiska...“ Načo sme povstali a nasledoval uvítací potlesk. A~teraz by som to povedal trochu inak: „Vážené dámy, vážení páni, otvorte svoje srdcia, prichádza Syn jediného živého Boha -- zmŕtvychvstalý Spasiteľ a Pán, Ježiš Kristus…“ A~naše životy nadobúdajú rovnováhu, pulzujú novou silou a je im prinavrátený pôvodný zmysel -- uctievať živého Boha!

Ježiš žije -- a tak môže byť moja minulosť odpustená, prítomnosť zvládnuteľná, a z~budúcnosti nemusím mať obavy! Navyše môžem byť skrze zmŕtvychvstalého Krista zrodený pre živú nádej. Na zmŕtvychvstaní záleží!

\autor{Miro Mišinec}


\clanok{Správy zo staršovstva}
Stretnutie staršovstva sme mali 5. a 19. marca 2019.

Na stretnutí 5. marca sme pripravovali podklady na VZČZ. Pracovali sme na definitívnej verzii rozpočtu, programu VZČZ a jednotlivých bodoch jednania. Zároveň sme s~volebnou komisiou pripravovali realizáciu volieb staršovstva, revíznej komisie a diakonov.
V súvislosti s~VZČZ chceme vyjadriť vďaku nášmu Pánovi a aj vám, za pokojný, láskavý a konštruktívny priebeh VZČZ. Vidíme, že Pán Ježiš odpovedá na modlitby aj tým, že dokážeme v~pokoji a láske rozprávať aj o~náročných témach, na ktoré máme rôzne názory. Zároveň vyjadrujeme vďaku Pánu Ježišovi za výsledky volieb, v~čom vnímame vyjadrenie Jeho vôle pre fungovanie nášho zboru.

Často sa na našich stretnutiach venujeme téme bratov a sestier z~Ukrajiny v~našom zbore. Ich porozumenie slovenčine, a teda aj bohoslužbám je na úrovni 40 percent. Preto sa snažíme o~dve riešenia:
\vskip-1ex\begitems
* technické zabezpečenie prekladu našich bohoslužieb do ukrajinčiny. Túto agendu má na starosti brat Ľuboš Kešjar. Realizáciu prekladu by mali na starosti dvaja Ukrajinci, ktorí rozumejú po slovensky.
* brat Viktor Potocki organizuje a vedie skupinku Ukrajincov v~nedeľu po našich bohoslužbách. Ich počet rastie a zloženie sa mení podľa ich pobytu na Slovensku. Ten je ovplyvnený prácou na tri mesiace bez potreby vybavovania povolení.
\enditems

Viacerí z~Ukrajincov hľadajú možnosť trvalého pobytu na Slovensku.  Tým by sa stabilizovala skupina tých, ktorí by tvorili základ skupinky vedenej Viktorom Potockim. V~dlhodobom horizonte je Viktorovým zámerom vytvoriť zbor pre rusky a ukrajinsky hovoriacich ľudí na Slovensku. Ponuka na stretnutie pri Božom Slove v~ukrajinčine v~nedeľu na obed je prezentovaná cez sociálne siete a pripravujeme informáciu aj na našej webovej stránke.

Raba BJB organizuje v~Račkovej doline v~termíne 26. -- 27.~4.~2019 konferenciu na tému „Spolu v~misii“. Miestom konania je chata J. A. Komenského v~Račkovej doline. Staršovstvo sa plánuje zúčastniť tejto konferencie v~spojení s~krátkym výjazdovým stretnutím v~Liptovskom Hrádku. Radi uvítame na tejto konferencii aj všetkých vás, ktorým leží na srdci misia nášho zboru.

V predveľkonočnom období pripravujeme sederovú večeru na troch miestach v~Bratislave. Cena pre jednu osobu je 7~€ a prihlásiť sa je možné prostredníctvom formuláru v~našej modlitebni resp. cez \ulink[https://docs.google.com/document/d/1tlDpV_HQPDJZxULZhCbjffxst5Jpks63CIbXPBRH7lk/edit?usp=sharing]{online formulár}.

Na VZČZ sme okrem iného hovorili aj o~službe hudobno-speváckej skupiny Matuzalem. Tejto téme sa venujeme tak, ako sme prezentovali na VZČZ.

Vnímame potrebu prezentovať náš zbor, baptizmus aj priestory, v~ktorých sa stretávame, pred turistami a ľuďmi prechádzajúcimi okolo našich modlitební na Palisádach aj na Chvojnici. Pre Chvojnicu máme pripravený materiál, ktorého definitívna verzia sa pripravuje pre tlač. Pre Palisády pripravujeme konečné znenie materiálu. Informácie chceme umiestniť tak, aby boli verejne prístupné pre tých, ktorí prechádzajú okolo našich modlitební.

Všetky témy, o~ktorých som písal, majú jedeného spoločného menovateľa. Potrebujeme a chceme, aby fungovali. Naša ľudská snaha je však veľmi málo. Túžime po tom, aby to bolo podľa Božej vôle. A~preto vás prosíme o~modlitby, aby sme poznávali Jeho vôľu a aby On požehnal naše snaženie.

\autor{za staršovstvo Peter Pribula}


\clanok {Rast zboru}
Verím, že sa spolu so mnou modlíte za záchranu ôsmich, ktorí sa krstom pridajú do našej rodiny v~roku 2019. V~skutočnosti ich už poznáme. Možno po mene ešte nie, ale som presvedčený, že už sú súčasťou okruhu našich priateľov a známych. K~takým máme najjednoduchší prístup a do nejakej miery už s~nimi máme vybudovanú dôveru. Ale nikto okrem Teba nemá k~nim prístup a bez Teba len ťažko budú počuť evanjelium. Možno si kladieš otázku: Čo mám s~tým robiť? Som rád, ak sa pýtaš, lebo mám zopár výziev.

Po prvé, budeš sa modliť za konkrétneho človeka alebo konkrétnych ľudí. Nos tento zoznam stále so sebou, aby si sa nezabudol modliť. Modli sa za nich aj osobne v~skupinkách. Budeme sa občas spolu modliť aj v~zhromaždení. Z~mojej skúsenosti viem, že viac ľudí príde k~Pánovi cez osobné vzťahy ako cez veľké evanjelizačné akcie. Treba začať s~modlitbou.

Po druhé, je potrebné, aby sme sa vystrojili. Bojíme sa s~tými ľuďmi rozprávať, lebo nevieme, ako začať. Minulý týždeň štyria z~nás absolvovali školenie {\bi Mosty k~lidem}, tréningový kurz zameraný na to, ako osloviť s~evanjeliom neveriacich priateľov. Bolo to jednoduché a praktické a verím, že keby sme ako celý zbor začali tým žiť, pokrstili by sme tento rok viac ako len ôsmich. Tento kurz bude pokračovať u~metodistov znova 11. -- 12. mája a potom 1. -- 2. júna. Modlím sa za to, aby od nás na tento tréning išlo 25 ľudí. Modli sa so mnou a buď sám odpoveďou na túto modlitbu. Čakajú nás zázraky! Teším sa na to!

\autor{Danny Jones}


\clanok{Spoločné modlitby}
\vskip-1ex\begitems
* Muži -- streda {\bf od 6.00~hod. do 7.00~hod.}, kostol na Palisádach
* Ženy -- pondelok {\bf od 17.00~hod.}, Zrínskeho 2
\enditems

Priveďte na spoločné modlitby aj svojich priateľov a známych, ktorým leží na srdci naše mesto a ľudia v~ňom.


\clanok{Verše na zapamätanie}
Na apríl máme nový veršík, ktorý sa chceme spoločne učiť. Veríme, že poznanie Písma prospeje našej duši i našej mysli:

{\it „Povzbudzujem vás teda, bratia, pre Božie milosrdenstvo, aby ste odovzdávali svoje telá ako živú, svätú, Bohu príjemnú obetu, ako vašu rozumnú službu Bohu. A~nepripodobňujte sa tomuto svetu, ale premeňte sa obnovením zmýšľania, aby ste vedeli rozoznať, čo je Božia vôľa, čo je dobré, čo mu je príjemné, čo je dokonalé.“}

\autor{R~12,~1~--~2}


\clanok{Stretnutia sestier}
Aprílové stretnutia sestier sa uskutočnia  {\bf 3.~a~17.~apríla o~17.30~hod.} v~modlitebni na Palisádach.

Ženy všetkých vekových kategórií sú srdečne vítané!
\vskip5pt


\clanok{Pozvánka na jubilejnú konferenciu sestier BJB SR a ČR}
V termíne 3. -- 5. mája 2019 sa bude v~Brne konať jubilejná sesterská konferencia, kedy si okrem iného spoločne pripomenieme 50. výročie obnovenia spoločnej práce sestier z~Čiech a Slovenska. Téma konferencie je {\it Obdarované milosťou}.  Prihlásiť sa je možné najneskôr do 10. apríla. Viac informácií ako aj pokynov pre prihlásenie nájdete na webovej stránke \ulink[https://www.baptist.sk/ks]{www.baptist.sk/ks}.
\vfill\break


\clanok{Evanjelizačné večery ProChrist 2019}
Pozývame vás na evanjelizačné večery ProChrist 2019, ktoré sa budú konať v~Novom ev. kostole na Legionárskej ulici v~Bratislave v~dňoch od 4.~4. do 8.~4.~2019 (štvrtok až pondelok), vysielané vždy o~18.00 h. Téma znie: Volanie bez poplatku -- „Otče náš“.

Pripravených je päť večerov tém, hudby, rozhovorov, životných príbehov známych osobností, v~ktorých živote Pán Boh spôsobil radikálnu zmenu. Zároveň päť príležitostí na spomalenie resp. zastavenie sa a „pripojenie sa na inú frekvenciu“, ktorá sa šíri v~hĺbke srdca, pokore mysle a spája s~domovom, ktorý je rovnako skutočný ako náš svet.

Program podujatia je zameraný na ľudí sekulárnych a hľadajúcich. Chce byť však aj novým duchovným impulzom povzbudenia pre stálych členov cirkvi.

Pozvite aj svojich priateľov a známych!


\clanok{Senior klub v~apríli}
Ak dá Pán zdravia a života, v~mesiaci apríl sa  stretneme {\bf posledný štvrtok, t.~j.~dňa 25.~apríla~2019 na Súľovskej ul. od 10.00~hod. do 14.00~hod.}

Téma stretnutia bude veľkonočná.

Všetci sú srdečne vítaní!

V láske Kristovej

\autor{Jana Makovíni}
\vfill\break


\clanok {Brigáda na Chvojnici}
V dňoch 2. -- 13.~apríla bude na Chvojnici prebiehať oprava vnútorných priestorov zborového domu a~kompletné vymaľovanie. Aby sa to mohlo uskutočniť, je potrebná pomoc brigádnikov 5.~apríla (jedno obsadené auto) a~13.~apríla (druhé vyťažené auto). Ide hlavne o~upratovacie práce. Ak by sa niekto našiel na týždňovky, bude určite vítaný!

\autor{Daniel Mikletič}


\clanok{Služba ľuďom bez domova}
Hľadáme dobrovoľníkov na varenie polievky pre ľudí v~núdzi na nasledovné termíny: 18.~júna, 16.~júla a 20.~augusta (všetky termíny sú utorokové). Po polievku príde výdajový tím o~19.00~hod.

Máme takisto nedostatok ľudí vo výdajových tímoch. Ak by ste boli ochotní pomôcť a zapojiť sa do výdaja teplej polievky, veľmi nám to pomôže.

Dobrovoľníci sa môžu hlásiť u~sestry Beaty Bogárovej.


\clanok{Zbierky za marec}
Milí bratia a sestry, ďakujeme za vašu obetavosť. V~mesiaci marec ste prispeli:
\vskip-1ex\begitems
* misia: 666 €
* investičný fond: informácia v~čase uzávierky ešte nebola k~dispozícii
\enditems


\n 1. 4.	Miroslav	KOLÁŘIK;
\n 4. 4.	Vierka	ŠKODÁKOVÁ;
\n 6. 4.	Jana	ZAJACOVÁ;
\n 6. 4.	Jarmila	CIHOVÁ;
\n 10. 4.	Anna	PAVLÍKOVÁ;
\n 11. 4.	Daniel	MIKLETIČ;
\n 16. 4.	Blažena	ŠKULECOVÁ;
\n 19. 4.	Marta	PRIBULOVÁ;
\n 22. 4.	Alexander Koloman	ERDÉLYI;
\n 25. 4.	Elena	TALIGOVÁ;
\n 30. 4.	Jaroslav	VOLENTIČ;
\n 30. 4.	Ľuboš	DZURIAK;
\narodeniny


\program{
\p 1  ; po ; 17.00 ; Modlitby -- ženy (Zrínskeho 2) ;.;;
\p 2  ; ut ; 15.15 ; Stretnutie pri Biblii (P. Pivka, Zrínskeho 2) ;.;;
\p 3  ; st ;  6.00 ; Modlitby -- muži (kostol Palisády) ; 17.30 ; Stretnutie sestier ;
\p 4  ; št ; 19.00 ; Biblická hodina (J. Szőllős, Zrínskeho 2) ;.;;
\p 5  ; pi ;.;;.;;
\p 6  ; so ; 18.00 ; Mládež (Súľovská 2) ;.;;
\p 7  ; ne ;  9.30 ; Bohoslužby (T. Valchář); 10.00 ; Chvojnica (P. Škulec) ;
\p 8  ; po ; 17.00 ; Modlitby -- ženy (Zrínskeho 2) ;.;;
\p 9  ; ut ; 15.15 ; Stretnutie pri Biblii (P. Pivka, Zrínskeho 2) ;.;;
\p 10 ; st ;  6.00 ; Modlitby -- muži (kostol Palisády) ;.;;
\p 11 ; št ; 19.00 ; Biblická hodina (J. Szőllős, Zrínskeho 2) ;.;;
\p 12 ; pi ;.;;.;;
\p 13 ; so ; 18.00 ; Mládež (Zrínskeho 2 -- byt kazeteľa) ;.;;
\p 14 ; ne ;  9.30 ; Bohoslužby (D. Jones) ; 10.00 ; Chvojnica (M. Antalík) ;
\p    ;    ; 17.00 ; Veľkonočný koncert Veľkého spevokolu ;.;;
\p 15 ; po ; 17.00 ; Modlitby -- ženy (Zrínskeho 2) ;.;;
\p 16 ; ut ; 15.15 ; Stretnutie pri Biblii (P. Pivka, Zrínskeho 2) ;.;;
\p 17 ; st ;  6.00 ; Modlitby -- muži (kostol Palisády) ; 17.30 ; Stretnutie sestier ;
\p 18 ; št ; 18.00 ; Zelený štvrtok -- Sederová večera (Súľovská 2 / Zrínskeho 2 / Partizánska 2) ;.;;
\p 19 ; pi ; 10.00 ; Ekumenické bohoslužby (Istropolis) ; 17.00 ; Veľký piatok -- bohoslužby (J.~Szől-lős) ;
\p 20 ; so ;   .   ; Mládež nebude -- veľkonočné prázdniny ;.;;
\p 21 ; ne ;  9.30 ; Bohoslužby (D. Jones) ; 10.00 ; Chvojnica ;
\p 22 ; po ; 17.00 ; Modlitby -- ženy (Zrínskeho 2) ;.;;
\p 23 ; ut ; 15.15 ; Stretnutie pri Biblii (P. Pivka, Zrínskeho 2) ;.;;
\p 24 ; st ;  6.00 ; Modlitby -- muži (kostol Palisády) ;.;;
\p 25 ; št ; 10.00 ; Senior klub (Súľovská 2) ; 19.00 ; Biblická hodina (J. Szőllős, Zrínskeho 2) ;
\p 26 ; pi ;.;;.;;
\p 27 ; so ; 18.00 ; Mládež (Súľovská 2) ; 18.00 ; Hudobný koncert (Ensemble Moscheles) ;
\p 28 ; ne ;  9.30 ; Bohoslužby (P. Kolárovský) ; 10.00 ; Chvojnica ;
\p 29 ; po ; 17.00 ; Modlitby -- ženy (Zrínskeho 2) ;.;;
\p 30 ; ut ; 15.15 ; Stretnutie pri Biblii (P. Pivka, Zrínskeho 2) ;.;;
}

\tiraz
\bye
