% DOKUMENTACIA:

% Prazdny riadok za textom znamena ukoncenie odstavca.
% Cierne obldzniky na konci riadku (v PDF) - to nechaj na mna (moze to o.i. znamenat, ze treba pridat nejake slovo do \hyphenation, lebo ho sam nevie rozdelit na konci riadku)

% Prikazy pre casti spravodaja:
% \spravodaj{<mesiac>}{<rok>}
% \clanok{<nazov clanku>}
% \autor{<autor clanku>}
% \n<den.mesiac.meno> - zadefinovanie oslavenca
% \narodeniny - vytvorenie tabulky s~narodeninami vsetkych zadefinovanych oslavencov
% \tiraz - ukoncenie spravodaja tirazou

% Styl fontu:
% \bf - bold, plati do konca aktualne skupiny, napr. ak mas {aaa \bf bbb} ccc, tak aaa bude normalne, bbb bude bold, ccc bude normalne
% \it - italic (pouzit rovnakym sposobom ako \bf)
% \bi - bold italic (pouzit rovnakym sposobom ako \bf)
% \rm - normalne (pouzit rovnakym sposobom ako \bf)

% Dalsie prikazy a znaky:
% \begitems - zoznam (odrazky), informacie najdes na stranke http://petr.olsak.net/ftp/olsak/opmac/opmac-u.pdf#toc%3A.5
% \ulink[<cielova adresa]{<zobrazena adresa>} - klikatelny odkaz na webstranku
% \email{<adresa>} - klikatelny odkaz na e-mailovu adresu
% ~ - nedelitelna medzera, napr. v~dome, 21.~6.~2018
% -- - pomlcka (dvakrát -)
% „ - zaciatocna uvodzovka
% “ - koncova uvodzovka
% \noindent - najblizsi odstavec nebude odsadeny
% \vskip<velkost> - vertikalna medzera, napr. \vskip3pt alebo \vskip-3ex (zaporna medzera, t.j. posun smerom hore)

%\typosize[9/12]% - standard% - pouzita velkost pisma/riadku
\input makra.tex % nacitanie Ivanom pripravenych nastaveni a prikazov
\hyphenation{star-šov-stvo} % rozdelenie slov na konci riadku, treba tu uviest slova, ktore sam nepozna

\spravodaj{6}{2019}


\clanok{Buďme nádejou vo svete bez nádeje!}
Drahí bratia a sestry,

jún je mesiac, kedy sa pravdepodobne, okrem iného, tešíme na blížiace sa leto, na obdobie dovoleniek a prázdnin, obdobie oddychu, ktorý tak veľmi potrebujeme. Ale tohto roku si v~júni pripomíname jednu dôležitú a pre nás ako cirkev Ježiša Krista slávnu udalosť: zoslanie Svätého Ducha, ktorého Ježiš zasľúbil svojim učeníkom. Pripomíname si, že cirkev Kristova bola Duchom Svätým zrodená; že týmto Božím Duchom potrebuje neustále žiť a byť Ním neustále naplňovaná, vedená a obnovovaná. Pripomíname si, že vierou v~Krista sme sa narodili pre život večný dielom a pôsobením Ducha, a preto z~moci Ducha potrebujeme neustále žiť život viery a byť Ním neustále obnovovaní a zmocňovaní na vnútornom človeku (Ef 3,16). Veď je to Duch Boží, ktorý nášmu životu prináša skutočný život, pokoj, radosť a nádej. Dnes by som chcel, slovami Rim 15,13 upriamiť našu pozornosť na tú úžasnú nádej, ktorú chce Duch Boží rozhojňovať v~našom živote.

Žijeme vo svete, ktorý postráda nádej. Veď aj my sme predtým žili bez Boha a bez nádeje (Ef 2,12b). Vplyv sveta, skúšky života, rôzne nepriaznivé okolnosti života neraz zakolíšu základmi našej viery a nádeje; neraz prežívame chvíle ako Pavol, keď sa zdá, že „bola vzatá nádej na vyslobodenie“ (Sk 27,20). Boh však nechce, aby Jeho deti žili ako tí, ktorí nemajú nádej (1Tes 4,13). Preto nám dal svojho Ducha, aby sme v~nádeji rástli mocou Ducha (Rim 15,13). Duch Svätý svojou mocou vštepuje novú nádej, ktorá nie je o~nejakom optimizme, ale stojí pevne v~Bohu, lebo náš Boh je Boh nádeje. V~tejto nádeji života večného sa môžeme radovať (Rim 12,12); nádej dostávame aj skrze potešenia Písma (Rim 15,4); nádej Ducha nám dáva odvahu nevzdať sa, ale pokračovať vo viere žiť a Pánovi slúžiť (2Kor 3,12); veď Duch nám zjavuje, čo a aká je to úžasná nádej Jeho povolania, ktorou nás povolal (Ef 1,18); je to nádej, ktorú máme odloženú v~nebesiach (Kol 1,5); tou nádejou je Kristus (Kol 1,27); Duch nám dáva schopnosť uchopiť vierou tú nádej, ktorá leží pred nami a ktorú máme sťa kotvu duše  bezpečnú a pevnú v~nebesiach (Žid 6,18-19). V~tejto nádeji Božej prítomnosti a Jeho vedenia a pomoci môžeme smelo vykročiť aj do nových dní a rokov života, lebo máme nádej, ktorá nezahanbuje (Rim 5,5a). Tak vám do nastávajúcich letných dní prajem, aby vás náš Boh, ktorý je Boh nádeje, naplnil svojou radosťou a pokojom a nádejou večnej slávy, ktorú máme v~Ježišovi Kristovi. Buďme ľuďmi nádeje v~tomto svete bez pravej nádeje.

\autor{Darko Kraljik}


\clanok{Správy zo staršovstva}
Od poslednej správy zo staršovstva sme mali päť stretnutí. Viaceré z~nich boli v~Bratislave v~zborovom dome, ale boli sme spoločne aj mimo Bratislavy. Stretli sme sa na výjazdovom stretnutí v~Liptovskom Hrádku s~následnou účasťou na konferencii v~Račkovej doline, zúčastnili sme sa diskusnej konferencie delegátov zborov (DKDZ) v~Banskej Bystrici, aj konferencie delegátov v~Nových Zámkoch.

V Račkovej doline sa v~apríli konala konferencia na tému „Spolu v~misii“. Staršovstvo sa zúčastnilo tejto konferencie v~spojení s~krátkym výjazdovým stretnutím v~Liptovskom Hrádku.

Počas veľkonočných sviatkov sme si mohli pripomínať dielo spasenia v~obeti Pána Ježiša aj pripomienkou poslednej „sederovej“ večere. Ďakujeme bratom a sestrám, ktorí túto príležitosť pripravili a viedli.

V máji sme mali zborové členské zhromaždenie, na ktorom sme diskutovali témy, ktoré boli programom na KDZ. Naši delegáti následne prezentovali postoj zboru na konferencii v~Nových Zámkoch.

Sme vďační nášmu Pánovi, že sme mohli osláviť Deň matiek a vyjadriť vďaku našim mamám za to, čo pre nás robili a robia a nášmu Pánovi za to, že sme Ho mohli poznávať skrze ich lásku, starostlivosť a výchovu.

Svätodušné sviatky plánujeme opäť stráviť spoločne na Chvojnici. Náplňou bude služba spevokolu, oslava Pána Ježiša a spoločenstvo pri aj po obede. Všetkých vás pozývame na toto stretnutie na Chvojnici.

Brat kazateľ navrhol, aby sme počas letných prázdnin zorganizovali popoludnia pre členov a priateľov zboru. Chceli by sme sa stretnúť po nedeľných bohoslužbách na mieste, kde by sme sa mohli spolu najesť a stráviť čas pri spoločných aktivitách. Predbežné termíny týchto popoludňajších stretnutí sú 28.~júl a 11.~august 2019. Bližšie informácie poskytneme po detailnejšom naplánovaní plánovaných stretnutí.

Nakoľko sa blíži leto a s~ním aj letné prázdniny, pripravovali sme program služieb na leto pre Bratislavu aj pre Chvojnicu.

Z našich priestorov bol odsťahovaný klavír. Zbor dostal účelový dar na jeho opravu a ladenie. Objednali sme preto generálnu opravu a ladenie u~špecialistu, ktorý sa venuje tejto práci. Oprava klavíra by mala byť hotová po letných prázdninách. Ďakujeme darcovi za jeho dar a veríme, že nám tento nástroj bude slúžiť dlhé roky na Božiu oslavu a pre naše potešenie.

\autor{za staršovstvo Peter Pribula st.}
\vfill\break


\clanok{Zborový výlet na Chvojnicu -- Letnice}
Aj tento rok chceme pokračovať v~dlhoročnej tradícii organizovania zborového výletu na sviatok zoslania Ducha Svätého a navštíviť 9.~júna~2019 bratov a sestry našej zborovej stanice Chvojnica, a tak sa navzájom povzbudiť pri budovaní spoločenstva.
Odporúčame sa zúčastniť výletu v~hojnom počte a využiť k~doprave objednaný autobus, ktorý bude pristavený na Palisádach v~blízkosti nášho kostola v~nedeľu ráno 9.~6.~2019 o~7.40 hod. a jeho odjazd bude presne o~8.00 hod.

Cena obojsmernej dopravy spolu s~obedom je pre dospelú osobu 10~€ a pre dieťa do 10~rokov 8~€. V~prípade individuálnej dopravy bude cena obeda pre dospelého 5~€ a pre dieťa do 10~rokov 3~€.

Prihlásiť sa na zborový výlet na Chvojnicu môžete dvomi spôsobmi, u~sestry Danky Paulenovej osobne alebo prostredníctvom \ulink[https://docs.google.com/spreadsheets/d/1f5-1DKmssXbyGS8oAjP15ImbinsuIYFwO7hl7no37KY/edit?usp=sharing]{elektronického formuláru}, najneskôr však do stredy 5.~6.~2019.


\clanok{Mládež na Chvojnici}
Ahojte mládežníci, rodičia a kamaráti!

Od 7.~do 9.~júna sa chystáme ako mládež na Chvojnicu, kde by sme chceli prežiť spolu víkend a takisto pripraviť miesto na zborový deň, ktorý sa tam bude konať v~nedeľu na~sviatok zoslania Ducha Svätého.

Ak máte záujem prísť, poslať svoje dieťa, prípadne sa inak zapojiť, prosím, prihláste sa prostredníctvom e-mailu \email{davidpribula1st@gmail.com} alebo telefonicky na čísle 0908~246~809.

Informácie o~odchode, cene a pod. budú poslané prihláseným.

\autor{Dávid Pribula}


\clanok{Zborové členské zhromaždenie}
Všetkých členov cirkevného zboru BJB Palisády pozývame na zborové členské zhromaždenie, ktoré sa bude konať v~nedeľu 16.~júna hneď po dopoludňajšom zhromaždení.

Program:
\vskip-1ex\begitems \style n
* Doplňujúce voľby delegáta zboru
* Pastoračné otázky
* Rôzne
\enditems


\clanok{Spoločné modlitby}
\vskip-1ex\begitems
* Muži -- streda {\bf od 6.00~hod. do 7.00~hod.}, kostol na Palisádach
* Ženy -- pondelok {\bf od 17.00~hod.}, Zrínskeho 2
\enditems

Priveďte na spoločné modlitby aj svojich priateľov a známych, ktorým leží na srdci naše mesto a ľudia v~ňom.


\clanok{Stretnutia sestier}
Júnové stretnutia sestier sa uskutočnia  {\bf 5.~a~19.~júna o~17.30~hod.} v~modlitebni na Palisádach.

Ženy všetkých vekových kategórií sú srdečne vítané!
\vskip5pt


\clanok{Zborový tábor v~Častej}
Zborový tábor (predtým rodinný tábor) sa bude konať na tradičnom mieste v~Častej v~termíne od 30.~júna do 6.~júla. Tešíme sa, že už veľa záujemcov sa prihlásilo.

Vzhľadom na to, že počet miest v~stredisku Detskej misie je obmedzený, chceme vás povzbudiť k~tomu, aby ste s~prihláškou neotáľali. Prihlásiť sa na zborový tábor môžete buď prostredníctvom \ulink[https://docs.google.com/forms/d/e/1FAIpQLScioR9Z-pYbkr4u_VKUTZ05ra0tOlUOdLo4pfiLQksVBKbVHw/viewform]{elektronického formulára} alebo osobne u~br. Petra Kolárovského, najneskôr do 21. júna 2019.


\clanok{Rodinný koncert na Palisádach}
Dňa 16.~júna 2019 o~17.00 hod. sa v~modlitebni na Palisádach bude konať rodinný koncert. Pozývame všetky deti, ktoré sa učia hrať na nejaký hudobný nástroj či spievať, aby sa prihlásili do programu. Zároveň pozývame aj dospelých, aby boli vzorom a pomocou našim deťom.

Viac informácií získate ako aj prihlásiť sa môžete u~Daniela Pletta, Samka Pletta, Petra Kolárovského alebo Diany Dzuriakovej.


\clanok{Záhradná slávnosť 2019}
Cirkevný zbor BJB v~Podunajských Biskupiciach nás všetkých pozýva na Záhradnú slávnosť, ktorá sa bude konať 16. júna 2019 o~16.00 hod. v~zborových priestoroch na Nákovnej~ul.~34.

Evanjelizačným slovom bude slúžiť br. Mirko Mišinec a takisto chválospevová skupina zo zboru BJB Viera.

Všetci sú srdečne vítaní!


\clanok{Senior klub v~júni}
Ak dá Pán zdravia a života, v~mesiaci jún sa stretneme {\bf posledný štvrtok, t.j.~dňa 27.~júna~2019 na Súľovskej ul. od 10.00~hod. do 14.00~hod.}

Nakoľko v~máji sme mali tému o~zoslaní Ducha Svätého, na stretnutí v~júni budeme pokračovať knižkou od Dereka Princa {\it Jazyk -- kormidlo tvojho života}.

Téma: Moc slova.

Všetci sú srdečne vítaní!

V láske Kristovej

\autor{Jana Makovíniová}
\vfill\break


\clanok{Pozvánka k~misii na „Habánsky hodový jarmok“}
Dňa 30.~júna sa vo Veľkých Levároch bude konať „Habánsky hodový jarmok“, kde svojou prítomnosťou a prezentáciou máme príležitosť priblížiť príbeh novokrstencov, anabaptistov, habánov, hutteritov z~Veľkých Levár. Program bude doplnený známymi remeselníckymi zručnosťami habánov.

Hľadáme aj dobrovoľníkov, ktorí by boli ochotní pomôcť so sprievodnými akciami (ideálne so znalosťou maďarského jazyka).

Viac informácií pre záujemcov ako aj ochotných dobrovoľníkov môžete získať u~br. Šaňa Erdélyiho (0903~864~435, \email{creativpress@creativpress.eu}).

\autor{menom Historickej komisie BJB v~SR -- Ján Szőllős a Šaňo Erdélyi}


\clanok{Verše na zapamätanie}
Na mesiac jún máme nový veršík, ktorý sa chceme spoločne učiť. Veríme, že poznanie Písma prospeje našej duši i našej mysli:

{\it „Veď ja sa nehanbím za evanjelium, lebo ono je Božou mocou na spásu pre každého veriaceho, najprv Žida, potom Gréka. Pretože v~ňom sa zjavuje Božia spravodlivosť z~viery pre vieru, ako je napísané: Spravodlivý z~viery bude žiť..“}

\autor{R~1,~16~--~17}


\clanok{Zbierky za uplynulé obdobie}
Milí bratia a sestry, ďakujeme za vašu obetavosť. V~uplynulom období ste prispeli:
\vskip-1ex\begitems
* investičný fond: 573,50 € (apríl + máj)
* misia: 291,50 € (máj)
\enditems


\n 2.	6.	Miriam	KEŠJAROVÁ;
\n 7.	6.	Pavel	KOHÚT;
\n 10.	6.	Ján	LAURENČÍK;
\n 15.	6.	Ľubica	HOVORKOVÁ;
\n 15.	6.	Peter	LICHANEC;
\n 16.	6.	Trey	ATKINS;
\n 17.	6.	Juraj	KVAČKA;
\n 19.	6.	Anna	ŠANDOROVÁ;
\n 22.	6.	Kristína	KEŠJAROVÁ;
\n 25.	6.	Peter	ŽEMBERY;
\n 25.	6.	Marica	ŠČEVLÍKOVÁ;
\n 26.	6.	Roman	ŽIARAN;
\n 27.	6.	Sylvia	PRIBULOVÁ;
\n 28.	6.	Jana	PERKNOVSKÁ;
\narodeniny


\program{
\p 1  ; so ; 18.00 ; DEPO mládež - aj pre mladšie ročníky (Súľovská 2) ;.;;
\p 2  ; ne ;  9.30 ; Bohoslužby (J. Szőllős); 10.00 ; Chvojnica (P. Škulec) ;
\p 3  ; po ; 17.00 ; Modlitby -- ženy (Zrínskeho 2) ;.;;
\p 4  ; ut ; 15.15 ; Stretnutie pri Biblii (P. Pivka, Zrínskeho 2) ;.;;
\p 5  ; st ;  6.00 ; Modlitby -- muži (kostol Palisády) ; 17.30 ; Stretnutie sestier ;
\p 6  ; št ; 19.00 ; Biblická hodina (J. Szőllős, Zrínskeho 2) ;.;;
\p 7  ; pi ;.;;.;;
\p 8  ; so ; 18.00 ; Mládež na Chvojnici ;.;;
\p 9  ; ne ;  9.30 ; Bohoslužby (D. Uhrin) ; 10.00 ; Chvojnica (P. Kolárovský + spevokol)  ;
\p 10 ; po ; 17.00 ; Modlitby -- ženy (Zrínskeho 2) ;.;;
\p 11 ; ut ; 15.15 ; Stretnutie pri Biblii (P. Pivka, Zrínskeho 2) ;.;;
\p 12 ; st ;  6.00 ; Modlitby -- muži (kostol Palisády) ;.;;
\p 13 ; št ; 19.00 ; Biblická hodina (J. Szőllős, Zrínskeho 2) ;.;;
\p 14 ; pi ;.;;.;;
\p 15 ; so ; 18.00 ; Mládež -- večer chvál s~evanjelickou mládežou (Súľovská 2) ;.;;
\p 16 ; ne ;  9.30 ; Bohoslužby (D. Jones) ; 10.00 ; Chvojnica (O. Škodák) ;
\p    ;    ; 11.00 ; Zborové členské zhromaždenie ; 17.00 ; Rodinný koncert ;
\p 17 ; po ; 17.00 ; Modlitby -- ženy (Zrínskeho 2) ;.;;
\p 18 ; ut ; 15.15 ; Stretnutie pri Biblii (P. Pivka, Zrínskeho 2) ;.;;
\p 19 ; st ;  6.00 ; Modlitby -- muži (kostol Palisády) ; 17.30 ; Stretnutie sestier ;
\p 20 ; št ; 19.00 ; Biblická hodina (J. Szőllős, Zrínskeho 2) ;.;;
\p 21 ; pi ;.;;.;;
\p 22 ; so ; 18.00 ; Mládež -- opekačka alebo grilovačka (miesto bude upresnené) ;.;;
\p 23 ; ne ;  9.30 ; Bohoslužby (P. Kolárovský) ; 10.00 ; Chvojnica (M. Kolářik) ;
\p 24 ; po ; 17.00 ; Modlitby -- ženy (Zrínskeho 2) ;.;;
\p 25 ; ut ; 15.15 ; Stretnutie pri Biblii (P. Pivka, Zrínskeho 2) ;.;;
\p 26 ; st ;  6.00 ; Modlitby -- muži (kostol Palisády) ;.;;
\p 27 ; št ; 10.00 ; Senior klub (Súľovská 2) ; 19.00 ; Biblická hodina (J. Szőllős, Zrínskeho 2) ;
\p 28 ; pi ;.;;.;;
\p 29 ; so ;.;;.;;
\p 30 ; ne ;  9.30 ; Bohoslužby (D. Jones); 10.00 ; Chvojnica (Komárno) ;
}

\tiraz
\bye
