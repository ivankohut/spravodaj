%\typosize[9/12]% - pouzita velkost pisma/riadku - standard
\input makra.tex % nacitanie Ivanom pripravenych nastaveni a prikazov
\hyphenation{star-šov-stvo} % rozdelenie slov na konci riadku, treba tu uviest slova, ktore sam nepozna

\spravodaj{1}{2020}


\clanok {Poučenie zo starých chýb}
Čas letí. Dni, roky a dekády letia.

Už je rok 2020 a rok 2019 je len spomienka z~minulosti. V~tomto novom roku nás čakajú nové skúsenosti a výzvy. Na niektoré z~nich sa vieme pripraviť, ale väčšina z~nich príde nečakane a prekvapí nás. Niektoré veci z~roku 2019 ešte stále ťaháme za sebou a pravdepodobne sa znovu zopakujú. Ak si nedáme pozor, vrátia sa tie isté chyby, ktoré sú nepríjemnou spomienkou z~minulého roku.

V Prísloviach 1,5 sa píše: „Nech múdry počúva a obohatí sa o~vedomosti, nech rozumný získa skúsenosť...“ Nič nie je náhodou alebo zbytočné, dokonca ani tie zlé veci, ak sa z~nich poučíme. Tento nový rok máme možnosť byť múdrejší ako kedykoľvek predtým, ak sa dobre poučíme. Ale poučenie sa nie je samozrejmosťou. Samozrejmosťou je to, že to isté zlé stále opakujeme. Poučiť sa je zámerné rozhodnutie, keď dovolíme Bohu, aby hovoril do nášho života, a počúvame, čo hovorí; keď sa necháme premeniť Jeho milosťou.

Z roku 2019 mám dlhý zoznam vecí, z~ktorých sa potrebujem poučiť. Nech je rok 2020 pre všetkých rokom múdrosti. Nech sa nielen poučíme, ale konečne aj vyliečime. Novoročné predsavzatia nestačia. Často ich len nadarmo skúšame. Potrebujeme Dobrého Pastiera. On sám nás poučí a vedie k~všetkému lepšiemu, čo nás v~roku 2020 čaká.

Vyzývam nás, aby sme tieto veci z~2019 pomenovali a predložili ich v~modlitbe pred Krista, aby nás konečne viedol k~víťazstvu. Majme veľké očakávania, lebo náš Pán je mocný a schopný.

Teším sa na tieto novoročné zázraky a svedectvá o~tom, čo v~našom zbore má Pán prichystané. „Nespomínajte predchádzajúce veci a o~dávnych nerozmýšľajte. Hľa, robím čosi nové, teraz to klíči, či to nebadáte? Áno, urobím cestu na púšti a rieky na pustatine“ (Iz~43,18-19).

Prajem vám požehnaný nový rok!

\autor{Danny Jones}


\clanok {Správy zo staršovstva}
Bratia a sestry,

\nobreak posledné stretnutie staršovstva sa nieslo v~duchu obzerania sa dozadu a pozerania dopredu. Hodnotili sme končiaci sa rok a bolo toho veľa, za čo sme vyjadrili vďaku nášmu Pánovi. Vyjadrili sme vďačnosť za:

\vskip-1ex\begitems
* pracovníkov v~zbore, ktorí sú dlhodobo zapojení do služby, ale aj za nových pracovníkov, ktorí sa rozhodli zapojiť akýmkoľvek spôsobom
* fungujúce skupinky a aj za to, že skupinky fungujú, aj keď ich zloženie je niekedy veľmi pestré
* novopokrstených členov Božej rodiny
* Božie vedenie v~ťažkých situáciách, jednaniach a rozhodovaniach
* silu, ktorú dáva Pán Boh, keď už nevieme ako ďalej
* množstvo detí v~zbore a aj za to, že pozývajú kamarátov medzi seba
* vzájomné spoločenstvo pri bohoslužbách ale aj mimo nich
\enditems

Na 2.~februára~2020 sme naplánovali zborové členské zhromaždenie. Na tomto stretnutí chceme rozprávať o~tých a s~tými, ktorí majú záujem o~členstve v~našom zbore.

Výročné zborové členské zhromaždenie sme naplánovali na 15.~marca~2020 o~15.30 hod.

Nakoniec sme rozprávali o~našich plánoch, túžbach a predstavách na rok~2020. Táto diskusia sa dotýkala najmä nasledujúcich tém:

\vskip-1ex\begitems
* biblické vyučovanie o~manželstve a práci s~manželskými pármi
* otázka ďalšieho kazateľa v~našom zbore
* ďalšia práca medzi Ukrajincami
* skupinky pre tých, ktorí nie sú zapojení do žiadnej skupinky
* práca s~mládežou; mužmi
* náš vzťah k~tomu, čo nám Pán Boh dáva
\enditems

O minulosti a budúcnosti sme diskutovali medzi sebou, ale hovorili sme o~nich aj s~naším nebeským Otcom.

\autor {za staršovstvo zboru Peter Pribula}


\clanok {Aliančný modlitebný týždeň}
Aliančný modlitebný týždeň sa bude konať od~12. do~19.~januára.

{\it Na ceste domov -- ako cestovať bezpečne} je témou aktuálnych aliančných modlitieb. Modlitebné témy pripravili bratia zo Španielska.


\clanok{Spoločné modlitby}
\vskip-1ex\begitems
* Muži -- streda {\bf od 6.00~hod. do 7.00~hod.}, kostol na Palisádach
* Ženy -- pondelok {\bf od 17.00~hod.}, Zrínskeho 2
\enditems

Priveďte na spoločné modlitby aj svojich priateľov a známych, ktorým leží na srdci naše mesto a ľudia v~ňom.


\clanok {Pozvania pre Dannyho a Claru Jonesovcov}
Br. kazateľ Danny Jones s~manželkou Clarou po návrate z~USA radi prijmú pozvania od členov a priateľov zboru. Ak by ste mali záujem ich niekedy pozvať, prosím vás, aby ste si s~nimi dohodli termín, najlepšie priamo s~Clarou: \email {clara.m.jones@gmail.com}, príp. 0948~288~879.


\clanok {Stretnutia sestier}
Milé sestry,

srdečne vás pozývam na ďalšie sesterské stretnutie, ktoré bude 15.~1. o~17.30~hod. v~modlitebni na Palisádach.

Zároveň sa teším na tie z~vás, ktoré ste manželky, na stretnutie na Zrínskeho v~stredu 29.~1. o~17.30 hod.

S láskou pre vás všetkých,

\autor {Clara Jones}


\clanok {Mládežnícka konferencia BJB}
Mládežnícka konferencia BJB sa bude konať v~termíne 21.~--~23.~februára v~Banskej Bystrici. Viac informácií nájdete na webovej stránke jednoty: \ulink [http://mk.baptist.sk/]{mk.baptist.sk}.

Takisto je možné prihlásiť sa ako dobrovoľník.


\clanok {Víkend pre manželské páry}
V rámci celonárodného podujatia {\it Národný týždeň manželstva} plánujeme v~našom zbore víkend pre manželské páry, ktorý bude 14.~--~15.~februára pod vedením Dannyho a Clary~Jones. Začiatok bude v~piatok o~18.00~hod. na Palisádach. Po krátkom príhovore pôjdu páry do mesta na rande. Zbor zabezpečí starostlivosť o~deti na Zrínskeho do~21.00~hod. pre tých, ktorí to budú potrebovať. V~sobotu budeme pokračovať v~programe v~čase od 9.00 do 15.00~hod. na Palisádach.

Počas celého dňa bude zabezpečený program pre deti na Zrínskeho pre tých, ktorí to budú potrebovať.


\clanok {Víkend pre mužov}
Chceli by sme vám dať do pozornosti víkend pre mužov, ktorý v~našom zbore plánujeme 24.~--~26.~apríla~2020 na Chvojnici.


\clanok {Služba núdznym}
Milé sestry a bratia,

aj v~tomto roku, ak dá Pán, budeme mať príležitosť poslúžiť varením polievok pre núdznych. Rada by som Vám dala do pozornosti {\bf termíny na rok 2020}.

Opäť som vybrala čo najviac sobotných termínov, ktoré Vám v~minulosti vyhovovali a takto sa možete do varenia zapojiť celé rodiny spolu aj s~vašimi deťmi:

{\bf 25.~január (sobota); 15.~február (sobota); 21.~marec (sobota); 21.~apríl (utorok); 19.~máj (utorok); 16.~jún (utorok); 21.~júl (utorok); 18.~august (utorok); 15.~september (utorok); 17.~október (sobota); 21.~november (sobota); 19.~december (sobota)}.

Na rezervované termíny sa, prosím, nahláste, u~mňa.

Objem je {\bf 30 litrov} hustej výživnej polievky buď s~mäsom, s~klobásou alebo s~párkami. Prosím Vás o~dodržanie času. Výdajový tím vo vestách {\it Pomoc ľuďom v~núdzi} príde po {\bf horúcu polievku o~19.00~hod. utorky a štvrtky a o~16.30~hod. soboty} na vami určené miesto odberu.

Variť môžeme v~kuchynke na Zrínskeho. Máme k~dispozícii 30-litrový hrniec, taktiež aj recepty na toto množstvo a aj finančné prostriedky, za ktoré štedrým darcom ďakujeme.

Naďalej {\bf hľadáme dobrovoľníkov do výdajových tímov} -- jeden tím vydáva teplý čaj a kávu a druhý tím vydáva polievku s~pečivom. Modlíme sa, aby Pán pridal pomocníkov do tejto služby.

Buďme takýmto spôsobm požehnaním pre druhých, ktorí túto pomoc potrebujú a ako spoločenstvo preukážme ľuďom v~núdzi lásku Pána Ježiša týmto praktickým spôsobom.

Ďakujem za spoluúčasť na tejto pomoci!

\autor {Beáta Bogárová}


\clanok {Senior klub}
Ak dá Pán zdravia a života, v~novom roku pripravujeme stretnutie senior klubu znova na posledný štvrtok v~mesiaci január, a to 30.~1.~2020 v~čase 10.00~--~14.00~hod. na Súľovskej~ul.

Téma: Osobné svedectvá s~čítaním novoročných veršíkov.

Všetci sú srdečne vítaní!

\autor {Jana Makovíniová}


\clanok{Verš na zapamätanie}
Na mesiac január máme nový veršík, ktorý sa chceme spoločne učiť. Veríme, že poznanie Písma prospeje našej duši i našej mysli:

{\it „... Ak vy zostanete v~mojom slove, budete naozaj mojimi učeníkmi, poznáte pravdu a pravda vás vyslobodí.“}

\autor{Ján~8,~31~--~32}


\clanok{Účelové zbierky za uplynulé obdobie}
Milí bratia a sestry, ďakujeme za vašu obetavosť. V~uplynulom období ste prispeli:

\vskip-1ex\begitems
* Misia: 250,00~€
* Ukrajina: 1971,00~€
* Investície: 0,00~€
\enditems
\vfill\break


\clanok {Verše pre zbor a zborové zložky na rok 2020}
Na rok 2020 sme pre zbor a zborové zložky vybrali nasledovné verše:

\begitems
* {\bf Zbor:} Hospodin stráži dych človeka, skúma všetky zákutia vnútra. (Prísl 20,27)
* {\bf Staršovstvo:} Hospodin je dobrý a úprimný, preto učí hriešnikov svojej ceste. Vedie pokorných podľa práva, učí pokorných svojej ceste. (Ž 25,8--9)
* {\bf Diakoni:} Chcú byť učiteľmi Zákona, ale nechápu ani čo hovoria, ani na čom tak neústupčivo trvajú. (2Tim 1,7)
* {\bf Hospodársky výbor:} Nikto nemôže slúžiť dvom pánom. Buď jedného bude nenávidieť a druhého milovať, alebo jedného sa bude pridŕžať a druhým pohrdne. Nemôžete slúžiť Bohu aj mamone. (Mt 6,24)
* {\bf Služba bezdomovcom:} Skúmaj ma, Bože, a poznaj moje srdce, skúšaj ma a poznaj moje zmýšľanie! Hľaď, či som na ceste trápenia a veď ma cestou večnosti! (Ž 139,23-24)
* {\bf Besiedka:} Boh je naše útočisko a sila, osvedčená pomoc v~súženiach. (Ž 46,2)
* {\bf Dorast:} Ak ma milujete, budete zachovávať moje prikázania. A~ja budem prosiť Otca a on vám dá iného Tešiteľa, aby bol s~vami až naveky. (Jn 14,15-16)
* {\bf Mládež:} Počiatok múdrosti je bázeň pred Hospodinom; rozumní sú všetci, čo takto žijú. Jeho chvála trvá navždy. (Ž 111,10)
* {\bf Sestry:} Hospodinove skutky lásky neprestávajú, lebo Jeho milosrdenstvo sa nekončí. (Náreky 3,22)
* {\bf Muži:} Len čo začali jasať a chválorečiť, nastražil Hospodin zálohy proti Ammónčanom, Moábčanom a obyvateľom seírskeho pohoria, ktorí napadli Judsko, takže utrpeli porážku. (2Krn 20,22)
* {\bf Seniori:} Lásku sme poznali podľa toho, že on položil svoj život za nás; aj my máme klásť život za bratov. (1Jn 3,16)
* {\bf Spevokol:} Ak Hospodin nestavia dom, márne sa namáhajú tí, čo ho stavajú; ak Hospodin nestráži mesto, márne bdie strážnik. Márne zavčasu vstávate, neskoro si líhate a jedávate chlieb tvrdej námahy; on zatiaľ svojmu milému dáva spánok. (Ž 127,1-2)
* {\bf Chválospevové skupinky:} Zlož svoje starosti na Hospodina, on sa o~teba postará. Nikdy nedopustí, aby sa spravodlivý sklátil. (Ž 55,23)
\enditems


\n 2.	1.	Ivan	KOHÚT;
\n 3.	1.	Ľubomír	KEŠJAR;
\n 4.	1.	Pavel	VAJO;
\n 11.	1.	Nataša	HOVORKOVÁ;
\n 16.	1.	Blahoslava	BETKOVÁ;
\n 18.	1.	Dávid	VALCHÁŘ;
\n 19.	1.	Andrej	CIHO;
\n 22.	1.	Jana	LAURENČÍKOVÁ;
\n 22.	1.	Jana	ČAHOJOVÁ;
\n 23.	1.	Elena	GUBOVÁ;
\narodeniny


\program{
\p  1 ; st ; 10.00 ; Bohoslužby (J. Szőllős) ;.;;
\p  2 ; št ;.;;.;;
\p  3 ; pi ;.;;.;;
\p  4 ; so ;.;;.;;
\p  5 ; ne ;  9.30 ; Bohoslužby (B. Uhrin) ; 10.00 ; Chvojnica ;
\p  6 ; po ;.;;.;;
\p  7 ; ut ; 15.15 ; Stretnutie pri Biblii (P. Pivka, Zrínskeho 2) ;.;;
\p  8 ; st ;.;;.;;
\p  9 ; št ;.;;.;;
\p 10 ; pi ;.;;.;;
\p 11 ; so ; 18.00 ; Mládež (Súľovská 2) ;.;;
\p 12 ; ne ;  9.30 ; Bohoslužby (D. Jones) ; 10.00 ; Chvojnica (J. Szőllős) ;
\p 13 ; po ; 17.00 ; Modlitby -- ženy (Zrínskeho 2) ;.;;
\p 14 ; ut ;.;;.;;
\p 15 ; st ;  6.00 ; Modlitby -- muži (kostol Palisády) ; 17.30 ; Stretnutie sestier (kostol Palisády) ;
\p 16 ; št ;.;;.;;
\p 17 ; pi ;.;;.;;
\p 18 ; so ; 18.00 ; Mládež (Súľovská 2) ;.;;
\p 19 ; ne ;  9.30 ; Bohoslužby (D. Jones); 10.00 ; Chvojnica (M. Antalík) ;
\p 20 ; po ; 17.00 ; Modlitby -- ženy (Zrínskeho 2) ;.;;
\p 21 ; ut ;.;;.;;
\p 22 ; st ;  6.00 ; Modlitby -- muži (kostol Palisády) ;.;;
\p 23 ; št ; 19.00 ; Biblická hodina (J. Szőllős, Zrínskeho 2) ;.;;
\p 24 ; pi ;.;;.;;
\p 25 ; so ; 18.00 ; Mládež (Súľovská 2) ;.;;
\p 26 ; ne ;  9.30 ; Bohoslužby (O. Kolárovský); 10.00 ; Chvojnica ;
\p 27 ; po ; 17.00 ; Modlitby -- ženy (Zrínskeho 2) ;.;;
\p 28 ; ut ;.;;.;;
\p 29 ; st ;  6.00 ; Modlitby -- muži (kostol Palisády) ; 17.30 ; Stretn. pre manželky (Zrínskeho~2) ;
\p 30 ; št ; 10.00 ; Senior klub (Súľovská 2) ; 19.00 ; Biblická hodina (J. Szőllős, Zrínskeho 2) ;
\p 31 ; pi ;.;;.;;
}


\tiraz
\bye
