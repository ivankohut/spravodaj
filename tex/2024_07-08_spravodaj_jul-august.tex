\def\velkostpisma{9}
\def\velkostriadku{12}
\input makra.tex % nacitanie Ivanom pripravenych nastaveni a prikazov
\hyphenation{star-šov-stvo} % rozdelenie slov na konci riadku, treba tu uviest slova, ktore sam nepozna

\spravodaj{7-8}{2024}

\def\sekcia#1{\vskip0.5em\noindent #1}

\clanok {MENO JAHVE je PÁN}

Iba veľkňazi v~jeruzalemskom chráme mohli meno Jahve vysloviť ako sväté a osobné meno Boha Izraela a to práve pri uctievaní Boha. Po tom, ako v~roku 70 nášho letopočtu bol chrám zničený, toto meno sa nevyslovovalo a v~Písme sa meno Jahve nahradilo označením Adonai. Anglické verzie Biblie zvyčajne Adonai prekladajú ako „Lord" (Pán) a Jahve ako „{\caps\rm Lord}". Jahve je meno, ktoré sa viaže práve s~Božími vykupiteľskými skutkami v~celých dejinách, ktoré urobil pre svoj vyvolený ľud.
Je to ten istý Boh, ktorý teba aj mňa vytrhol a zachránil z~otroctva hriechu a ktorý zachránil izraelský národ ako svoj ľud spod otroctva faraóna v~Egypte.

\sekcia{KĽÚČOVÉ VERŠE}

Boh, keď sa predstavil Mojžišovi, povedal: „Ja som, ktorý som!“ - a dodal: „Toto povieš Izraelitom: ‚Ja-som‘ ma poslal k~vám!“ A~Boh ešte povedal Mojžišovi: „povieš im toto: Pán a Boh vašich otcov, Boh Abraháma, Boh Izáka a Boh Jákoba, ma poslal k~vám. Toto je moje meno naveky a takto ma budú spomínať z~pokolenia na pokolenie.“ (Ex 3,14-15)

\sekcia{ZAMYSLI SA}

Jahve je stále blízko a koná v~záujme svojho ľudu. Aký Boží zásah by si potreboval dnes ty?

{\em Spievať budem Pánovi, lebo vznešený, slávne vznešený je: Koňa i~povoz zmietol do mora. Moja sila a~moja udatnosť je Pán a~On sa mi stal spásou. Toto je môj Boh, budem ho velebiť, Boh môjho otca, budem ho chváliť. Amen.}

\sekcia{JAHVE je PÁN}

„Dobroreč, duša moja, Pánovi (Jahvemu) a celé moje vnútro Jeho svätému menu. Dobroreč, duša moja Pánovi (Jahvemu) a nezabúdaj na Jeho dobrodenia. Ved On ti odpúšťa všetky neprávosti, On lieči všetky tvoje neduhy; On vykupuje tvoj život zo záhuby, On ťa venčí milosrdenstvom a milosťou; On naplňuje dobrodeniami tvoje roky, preto sa ti mladosť obnovuje ako orlovi. Milostivý a milosrdný je Pán (Jahve), zhovievavý a nesmierne dobrotivý. Nevyčíta nám ustavične naše previnenia, ani sa nehnevá naveky. Nezaobchodí s~nami podľa našich hriechov, ani nám neodpláca podľa našich neprávostí. Lebo ako vysoko je nebo od zeme, také veľké je Jeho zľutovanie voči tým, čo sa ho boja. Ako je vzdialený východ od západu, tak vzďaľuje od nás našu neprávosť.“ (Ž 103,1-5; 8-12)

\vskip-1ex\begitems
* {\it Chváľ ho}: Pretože Jeho dobrota trvá naveky.
* {\it Ďakuj mu}: Za to, že Božie odpustenie a milosrdenstvo je nekonečné.
* {\it Vyznaj mu}: Keď si ho neurobil Pánom v~každej oblasti svojho života.
* {\it Pros ho}: Aby svojou zvrchovanosťou ťa uschopnil poddať sa mu pod Jeho mocnú ruku.
\enditems

{\em JAHVE je môj PÁN}

\sekcia{PRISĽÚBENIA SPOJENÉ S~MENOM JAHVE}

Sľub je len taký dobrý ako osoba, ktorá ho urobila -- hovorí porekadlo.
Aký asi bude sľub „lásky až po hrob“, ak ho vysloví neverná žena, alebo aká bude „dvadsaťročná záruka“ od firmy, ktora práve krachuje, či dlžobná listina od podvodníka?
Našťastie, Boh nemusí zveličovať svoje schopnosti. Taktiež nikdy nesľúbi niečo, čo nemôže splniť -- nech by sľúbil čokoľvek. Žiješ tak, že plníš Božie príkazy? Povzbudením je, že na Boha menom Jahve sa môžeš vždy spoľahnúť, a to bez ohľadu na to, čomu čelíš.

\sekcia{PRISĽÚBENIA V~PÍSME}

„Lebo Pán bude tvojou dôverou a bude chrániť tvoju nohu pred pascou.“ (Pr~3,26)

„Ak budeš zachovávať príkazy Hospodina, svojho Boha, a kráčať po Jeho cestách, Hospodin ťa povýši na svätý ľud, ako ti prisahal.“ (Dt~28,9)

\autor {Peter Šrankota}


\clanok {Správy zo staršovstva za jún 2024}

Staršovstvo sa v~júni stretlo dvakrát. Na prvom stretnutí 4.~6. sme viac hovorili o~cirkevnom zbore z~USA (ktorého niektorí členovia tu boli už viackrát, naposledy počas jarných prázdnin -- stretnutia s~našou mládežou a dorastencami) a jeho ponuke na partnerstvo s~naším zborom. Venovali sme sa príprave diskusného stretnutia ohľadom situácie v~zbore (podnety na diskusiu vzišli aj z~jarnej víkendovky a tiež od členov zboru). Vyhodnocovali sme aktivity z~minulého obdobia:
\vskip-1ex\begitems
* Modlitebné stretnutie k~voľbe ďalšieho kazateľa a k~aktuálnym spoločenským udalostiam -- stretli sme sa v~slušnom počte v~modlitebni na Palisádach a vnímali sme, že naše naliehavé prosby, modlitby k~nášmu Bohu, sú tou správnou reakciou v~čase skúšok a problémov.
* Večer modlitieb a chvál -- prežili sme požehnané chvíle pri uctievaní spevom, ale aj pri výklade zmyslu uctievania a modlitby.
\enditems

Druhé stretnutie staršovstva sa venovalo primárne zhodnoteniu diskusného stretnutia členov a priateľov zboru. Stretnutie bolo otvorené, zaznievali veci, ktoré nás ťažia. Názory boli aj protichodné, ale všetko prebehlo v~láske, čo je vzácne. Staršovsto dostalo reakcie od niektorých členov aj po tomto stretnutí. Staršovstvo sa k~tejto téme stretne aj počas prázdnin. Hovorili sme aj o~krste, ktorý sa konal na Kuchajde. Vnímame, že pre niektorých starších členov je obtiažne sa zúčastniť slávnosti krstu na takom mieste, a v~budúcnosti by sme radi mali krst aj na tradičnejších miestach. Tiež sme zhodnocovali akciu {\em Noc otvorených kostolov} a potešili sme sa z~výsledku volieb ďalšieho kazateľa.

\autor {Ľ. Syč}


\clanok {Verš na mesiac júl 2024}

„V~tvojom strede je Hospodin, tvoj Boh, hrdina, čo zachraňuje, radostne nad tebou jasá, svojou láskou ťa tíši, zvučne nad tebou plesá.“ (Sof 3,17)


\clanok{Rekonštrukcia exteriéru našej modlitebne}

\table{(\hskip-1.5mm)lr}{
Predpoklad realizácie prác r.~2025 & \crli
Predpokladaná medziročná inflácia +4\% & 5 700 € \cr
{\bf Predpokladané celkové náklady} & {\bf 148 200 €} \cr
Z invest. fondu (tvoreného z~ned. zbierok a darov) chceme použiť & 60 000 € \cr
Zaokrúhlene treba ešte vyzbierať & 90 000 € \cr
Počet aktívnych členov & cca 100 ľudí \cr
{\bf Odporúčaná priemerná výška daru na jedného člena} & {\bf 900 €} \cr}
\vskip1em

Realizácia daru je prevodom na zborový účet SK36 0900 0000 0000 1147 1836, variabilný symbol 777.

Naďalej budú pokračovať aj nedeľné zbierky počas štvrtých nedieľ v~mesiaci (zbierka na investičný fond), určené hlavne pre priateľov zboru.
Je potrebné si tiež uvedomiť, že financie na tento účel sú nad rámec nášho bežného dávania do zborovej pokladne, teda je potrebné zachovať aj dary (desiatky....), ktoré sme dávali doteraz.


\clanok{Celkové plnenie rozpočtu k~22. 6. 2024}

\table{lrrr}{
Príjem				& Plán		& Skutočnosť	& podiel z~ročného plánu \crli
Nedeľné zbierky		& 28 000 €	& 12 705 €		& 45,38 \% \cr
Dary a desiatky		& 34 000 €	& 16 413 €		& 48,27 \% \cr
Misijný fond 		&  5 500 €	&  3 601 €		& 65,47 \% \cr
Investičný fond		&  4 000 €	&  1 696 €		& 42,40 \% \cr
Záväzky na fasádu	& 92 700 €	& 34 009 €		& 36,69 \% \cr}
\vskip1ex

\n  4.	7.	Margita	ELISHEROVÁ;
\n  4.	7.	Ľubomíra	KOHÚTOVÁ;
\n 10.	7.	Slavomír	MÁŤUŠ;
\n 10.	7.	Katarína	KEREKRÉTY;
\n 16.	7.	Rút	BEDNÁRIKOVÁ;
\n 20.	7.	Mária	KOHÚTOVÁ;
\n 27.	7.	Lenka	KOHÚTOVÁ;
\n 28.	7.	Elena	ŠALINGOVÁ;
\n 28.	7.	Pavlína	SYNOVCOVÁ;
\n 31.	7.	Marína	CIHOVÁ;
\n  1.	8.	Dana	KEŠJAROVÁ;
\n  1.	8.	Zuzana	HRAŠKOVÁ;
\n  1.	8.	Vlasta	ŠALINGOVÁ;
\n  2.	8.	Marta	RAČIČOVÁ;
\n 11.	8.	Šimon	HOVORKA;
\n 11.	8.	Ľuboslava	KOVÁČIKOVÁ;
\n 16.	8.	Radovan	JANČULA;
\n 18.	8.	Anna	LIPTÁKOVÁ;
\n 23.	8.	Danica	PAULENOVÁ;
\n 25.	8.	Ivan	PAULEN;
\n 31.	8.	Miroslava	HOVORKOVÁ;
\narodeniny

\programna{7}{
\p  2 ; ut ; 15.15 ; Biblická hodina pre seniorov (P. Pivka) ;.;;
\p  7 ; ne ;  9.30 ; Bohoslužby (J. Szőllős + VP) ;.;;
\p 14 ; ne ;  9.30 ; Bohoslužby (P. Pribula) ;.;;
\p 21 ; ne ;  9.30 ; Bohoslužby (P. Šrankota) ;.;;
\p 28 ; ne ;  9.30 ; Bohoslužby (Ľ. Syč) ;.;;
}
\vskip3ex
\programna{8}{
\p  4 ; ne ;  9.30 ; Bohoslužby (P. Šrankota + VP) ;.;;
\p 11 ; ne ;  9.30 ; Bohoslužby (J. Szőllős) ;.;;
\p 18 ; ne ;  9.30 ; Bohoslužby (S. Kráľ) ;.;;
\p 25 ; ne ;  9.30 ; Bohoslužby (P. Kolárovský) ;.;;
}

\tiraz
\bye
