%\typosize[10/12.5]% - pouzita velkost pisma/riadku - trochu vacsie
\input makra.tex % nacitanie Ivanom pripravenych nastaveni a prikazov
\hyphenation{star-šov-stvo} % rozdelenie slov na konci riadku, treba tu uviest slova, ktore sam nepozna

\spravodaj{7-8}{2021}


\clanok {Prázdniny a „plnédniny“}
Predpokladám, že vás zaujal novotvar „plnédniny“ v~názve. Kontrast ku prázdnym dňom sú plné dni. Slovo prázdniny v~nás vyvoláva pocit, že ide o~prázdne dni. Možnože aj takto vzniklo to slovo.

Tešíte sa na prázdniny? Predpokladám, že áno, hoci v~tejto medzicovidovej dobe sa možno nájdu ľudia, ktorí sa na tento čas netešia, možno sa niektorí aj obávajú týchto dní a toho, čo príde po nich. Prázdniny sa spájajú predovšetkým so školou, ale ovplyvňujú život celej spoločnosti. Dni sú skutočne „prázdnejšie“ od práce, od školy, od bežných aktivít, ktoré robíme cez školský alebo pracovný rok. Aj mestá sa cez prázdniny vyprázdňujú a napĺňa sa vidiek, hory a rekreačné strediská. Prázdniny teda nie sú všeobecné a všade, lebo na niektorých miestach sú to „plnédniny“.

Počas pandémie sme sa učili inak pozerať na čas. Zrazu sme mali viac času pre seba, svojich blízkych, viac času sme trávili doma, niektorí aj osamote. Naše možnosti cestovať a stretať iných ľudí boli obmedzované. Teraz si možno viac vážime možnosť stretávať sa s~inými ľuďmi, možnosť navštevovať sa, možnosť cestovať a vidieť iné krajiny, zažiť inú kultúru. Už nám všetky takéto samozrejmosti nepripadajú ako samozrejmé.

Toto obdobie dvoch mesiacov sa nazýva aj dovolenkové obdobie. Pri slove dovolenka mi napadá súvislosť s~„dovoliť si“. Tento čas je aj obdobím, keď si možno dovolíme aj veci, na ktoré počas bežných pracovných dní nemáme čas. Dovoľme si robiť nie čokoľvek, ale tie dobré veci, ktoré Pán Boh pre nás stvoril a pripravil.

Božie Slovo nás vyzýva, aby sme múdro využívali čas milosti (Kol~4,5), ktorý ešte máme k~dispozícii, a môžeme prosiť, aby nás Pán naučil múdro počítať dni života (Ž~90,12). Prázdniny a dni dovoleniek zvádzajú k~tomu, aby sme ich premárnili na nie dobré aktivity. Aj ku skutočnému oddychovaniu je potrebná múdrosť, aj dni oddychu, ktoré máme k~dispozícii, máme múdro počítať. V~pandemickom čase, ktorý žijeme, získava táto múdrosť ešte aj inú dimenziu.

Môžeme teda premeniť naše prázdniny na „plnédniny“, naplniť ich obsahom, ktorý nám prinesie oddych, nové zážitky a aktivity, stretnutia a spoločenstvo s~ľuďmi, prácu iného druhu a všetko iné, čo napĺňa náš život. Môžeme si byť istí, že Pán nám chce dopriať život v~jeho plnosti a bohatosti a v~hojnej miere (J~10,10b). Prijmime požehnanie, ktoré nám Pán chce dať aj cez tieto letné mesiace prázdnin a dovoleniek.

\autor{Ján Szőllős}


\clanok {Milá rodina!}
Viac ako kedykoľvek predtým je istota ilúziou. Samozrejme, stále máme istotu v~tom, že Boh je dobrý a verný, a Jeho Slovo zostáva pravdivé. Ale čo nás čaká zajtra alebo na jeseň, nevieme. Iba Boh vie, čo nám z~lásky pripravuje. Tešíme sa, že sme nútení ešte viac maximalizovať prítomnosť a dar dnešného dňa. Pavol píše v~Ef~5,15-16: {\it „Dbajte teda dôkladne na to, ako si počínate; nie ako nemúdri, ale ako múdri. Naplno využívajte čas, lebo dni sú zlé. Preto nebuďte nerozumní, ale pochopte, čo je Pánova vôľa.“}

Máme pred sebou leto a vďaka Bohu lepšiu zdravotnú situáciu a uvoľnené opatrenia. Práve preto si treba počínať múdro.  Aký máš plán na leto? Na koho sa chystáš a čo máš preňho prichystané? Nenechaj to na náhodu. Múdro si naplánuj svoj čas. Oddeľuj si dosť času pre Pána a rozjímanie nad Písmom. Si poznačený rokom covidu a potrebuješ uzdravenie, či zdravotné, duševné alebo duchovné. Uvoľnenie opatrení znamená zmenu rytmu, čo potrebujeme. Ale dávaj si pozor, aby ti zmena rytmu neukradla vzácny a potrebný čas s~Pánom.

Takisto treba čo najviac využiť spoločenské príležitosti. Nebuď medzi tými, ktorí majú vo zvyku opúšťať naše zhromaždenie. Boh nás k~tomu vyzýva v~liste Hebrejom, lebo vie, že to naliehavo potrebujeme, hlavne vtedy, keď si myslíme, že nepotrebujeme. Vtedy, keď mi je jedno, čí ísť do „zhromka“ alebo nie,  som najviac oklamaný a najviac ohrozený. Namiesto ľahostajnosti buďme radostne spolu. S~Clarou sa na Vás veľmi, veľmi tešíme.

Sme Bohu vďační aj za možnosť zborového tábora. Ak si sa ešte stále nezaregistroval, urob to čo najskôr. A~potom, počas celého leta, sa navštevujte. Organizujte výlety  do lesa alebo prechádzky do parku. Strávte spolu vzácne chvíle. Podeľte sa o~to, čo Boh počas covidu v~tvojom živote konal a modlite sa spolu. Potrebujeme byť spolu. Nedokážeme sami spoznať Boha dostatočne. Vnímam Ho len cez svoj pohľad, cez okuliare, cez ktoré sa na všetko pozerám. Preto je zbor a spoločenstvo dôležité. Potrebujem spoznávať Boha cez tvoje okuliare, tvoje vnímanie a tvoj pohľad. Ale dávaj si pozor. Nestačí len byť spolu. Treba sa vzdelávať o~tom, čo od Boha prežívame. Tvoj duchovný život nepatrí len tebe. V~rámci zboru patríme jeden druhému navzájom. Preto je pre mňa dôležité, kde sa duchovne nachádzaš, ako aj pre teba, kde sa nachádzam ja. Apoštol Peter o~tom krásne píše vo svojom prvom liste: {\it „Koniec všetkého sa priblížil. Buďte teraz rozvážni a triezvi, aby ste boli pohotoví modliť sa.  Nadovšetko majte vytrvalú lásku jedni k~druhým, lebo láska prikrýva množstvo hriechov. Buďte navzájom pohostinní, bez šomrania. Ako dobrí správcovia mnohorakej Božej milosti slúžte si navzájom každý tým duchovným darom, ktorý prijal. Keď niekto hovorí, nech hovorí slovami, ktoré mu dal Boh, keď niekto slúži, nech to robí silou, ktorú dáva Boh, aby bol vo všetkom oslavovaný Boh skrze Ježiša Krista! Jemu patrí sláva i moc na veky vekov. Amen“} (1Pt~4,7-11). Toto je opis zdravého zboru. Je to opis najláskavejšieho zboru v~meste. Verím, že postupne je to opis aj nášho zboru. Teším sa, že počas leta s~radosťou zažijeme tieto verše a budeme z~nich viac smerovať k~tomu nášmu zborovému cieľu.

Želám vám krásne leto. Ste milovaní večnou láskou.

\autor {Danny Jones}


\clanok {Správa zo staršovstva}
Staršovstvo zboru sa na svojom jedinom stretnutí v~júni venovalo téme života zboru „po covide“. Je zrejme naivné si myslieť, že sa vrátime k~tomu, čo sme poznali pred obdobím pandémie. Nová realita ovplyvnila to, ako ľudia vnímajú potrebu spoločenstva a zborového života. Vo svojom uvažovaní sa zameriavame predovšetkým na to, ako v~tejto novej realite budovať spoločenstvo a viesť ľudí k~nasledovaniu Pána Ježiša. Čoraz viac si uvedomujeme potrebu zapojiť čo najviac ľudí v~zbore do skupiniek, aby v~takýchto krízových situáciách nikto nevypadol zo zborového spoločenstva.

Tešíme sa, že sme už mohli začať s~prezenčnými bohoslužbami. Bolo vzácne sa stretnúť a vidieť niektoré tváre po dlhom čase. Modlíme sa, aby táto priaznivá situácia vydržala čo najdlhšie.

Tento mesiac sa na Chvojnici urobila oprava septiku, čo nám umožní využívať našu chalupu aj v~budúcnosti. Veľká vďaka za organizáciu a nasadenie pri tejto oprave patrí br. Danielovi Mikletičovi.

Po ukončení teologického štúdiu v~Banskej Bystrici nastúpil br. Viktor Pototski na kazateľskú prax v~našom zbore. Jeho služba bude naďalej zameraná predovšetkým na našich bratov a sestry z~Ukrajiny.

V júli sa staršovstvo plánuje stretnúť ešte raz kvôli príprave výročného zborového členského zhromaždenia, ktoré je plánované na 1.~august.

\autor {Peter Kolárovský}


\clanok {Biblická hodina pre seniorov počas leta}
Biblická hodina (predovšetkým) pre seniorov bude v~letných mesiacoch prebiehať pod vedením br. Pavla Pivku v~týchto termínoch: 6.~7., 20.~7., 3.~8. a 17.~8. vždy o~15.15~hod. na Zrínskeho.


\clanok {Dorastenecký a mládežnícky tábor}
Dorastenecký tábor sa uskutoční v~dňoch 28.~7.~--~1.~8. na Chvojnici. Cena za celý pobyt je 20~€. Prihlásiť sa je možné u~Vráblovcov resp. ktoréhokoľvek vedúceho dorastu.

Mládežnícky tábor bude takisto na Chvojnici (dorastenecký tábor bude naň nadväzovať). Bližšie informácie budú k~dispozícii neskôr.


\clanok {Zborový tábor}
Zborový tábor tento rok plánujeme v~čase od~15. do~21.~augusta v~stredisku Detskej misie v~Častej–Papierničke. Voľné sú už iba posledné miesta. Bližšie informácie nájdete v~elektronickej prihláške: \ulink[https://forms.gle/4kGK8pTsuQt5Sr7v7]{forms.gle/4kGK8pTsuQt5Sr7v7}.


\clanok {Krst v~zbore}
Chceme osloviť všetkých, ktorí by sa chceli dať pokrstiť na vyznanie viery, aby sa prihlásili u~br.~kazateľa Dannyho Jonesa alebo ktoréhokoľvek staršieho zboru.


\clanok {Pomoc ľuďom v~núdzi}
Ak by ste mali záujem zapojiť sa do služby varenia pre ľudí v~núdzi, v~mesiaci júl sú voľné nasledovné termíny: 6.~7., 8.~7., 13.~7., 15.~7., 20.~7. a 29.~7.

Prihlásiť sa môžete u~sestry Beaty Bogárovej.


\clanok {CampFest}
Tohtoročný hudobný festival CampFest sa bude konať na Ranči v~Kráľovej Lehote v~dvoch termínoch: 29.~7. -- 1.~8. a 5.~8. -- 8.~8. Je to festival, na ktorom vystúpia desiatky worshipových kapiel a taktiež množstvo rečníkov z~rôznych denominácií z~celého Slovenska. Víziou organizátorov je prebudenie, mobilizácia a spolupráca, s~cieľom zasiahnuť mladú generáciu a aj skrze túto platformu dať Bohu priestor konať a dotýkať sa ľudských sŕdc.

Na zrealizovanie tohto podujatia sa ešte stále hľadajú desiatky dobrovoľníkov (najmä chlapcov), a tak ak máte medzi sebou alebo v~okolí mladých ľudí, pre ktorých by to mohla byť skvelá skúsenosť a príležitosť, môžete ich pozvať! Registrácia dobrovoľníkov je už spustená na webovej stránke: \ulink[https://www.campfest.sk/dobrovolnik/]{campfest.sk/dobrovolnik}.

Predaj lístkov pre účastníkov je spustený už od 1.~7.~2021. Lístky je možné zakúpiť si iba v~predpredaji a ich počet je obmedzený.


\clanok{Verš na zapamätanie}
Na leto máme nový veršík, ktorý sa chceme spoločne učiť. Veríme, že poznanie Písma prospeje našej duši i našej mysli:

{\it „Nadovšetko majte vytrvalú lásku jedni k~druhým, lebo láska prikrýva množstvo hriechov. Buďte navzájom pohostinní, bez šomrania. Ako dobrí správcovia mnohorakej Božej milosti slúžte si navzájom každý tým duchovným darom, ktorý prijal.“}

\autor{1Pt~4,~8--10}


\clanok{Zbierky za uplynulé obdobie}
Milí bratia a sestry,

v júni ste prispeli:

\vskip-1ex\begitems
* Misia: 1081,00 €
* Investície: 1182,00 €
\enditems

Ďakujeme vám, že napriek okolnostiam a neistým ekonomickým vyhliadkam do budúcnosti, ste mnohí prispeli na činnosť a službu zboru. Aj naďalej máte možnosť prispieť do „nedeľnej zbierky“, a to prevodom na účet zboru. Do poznámky pre prijímateľa, prosím, uveďte „zbierka“.

Bankové spojenie: SK36 0900 0000 0000 1147 1836, SWIFT: GIBASKBX

Ďakujeme!


\n Ľudovít	BETKO;
\n Margita	ELISCHEROVÁ;
\n Ľubomíra	KOHÚTOVÁ;
\n Slavomír	MÁŤUŠ;
\n Katarína	KEREKRÉTY;
\n Milada	KREJČOVÁ;
\n Rút		BEDNÁRIKOVÁ;
\n Mária	KOHÚTOVÁ;
\n Lenka	KOHÚTOVÁ;
\n Elena	ŠALINGOVÁ;
\n Pavlína	SYNOVCOVÁ;
\n Marína	CIHOVÁ;
\n Danuška	KEŠJAROVÁ;
\n Zuzana	HRAŠKOVÁ;
\n Vlasta	ŠALINGOVÁ;
\n Marta	RAČIČOVÁ;
\n Anna	KOPČOKOVÁ;
\n Ľuboslava	KOVÁČIKOVÁ;
\n Šimon	HOVORKA;
\n Radovan	JANČULA;
\n Anna	LIPTÁKOVÁ;
\n Danica	PAULENOVÁ;
\n Ivan	PAULEN;
\n Miroslava	HOVORKOVÁ;
\narodeniny

\programna{7}{
\p  4 ; ne ;  9.30 ; Bohoslužby (J. Szőllős) ;.;;
\p  6 ; ut ; 15.15 ; Stretnutie pri Biblii (P. Pivka, Zrínskeho 2) ;.;;
\p 11 ; ne ;  9.30 ; Bohoslužby (D. Jones) ;.;;
\p 18 ; ne ;  9.30 ; Bohoslužby (D. Jones) ;.;;
\p 20 ; ut ; 15.15 ; Stretnutie pri Biblii (P. Pivka, Zrínskeho 2) ;.;;
\p 25 ; ne ;  9.30 ; Bohoslužby (S. Kráľ) ;.;;
}

\programna{8}{
\p  1 ; ne ;  9.30 ; Bohoslužby (D. Jones) ; 16.00 ; Výročné zborové členské zhromaždenie ;
\p  3 ; ut ; 15.15 ; Stretnutie pri Biblii (P. Pivka, Zrínskeho 2) ;.;;
\p  8 ; ne ;  9.30 ; Bohoslužby ;.;;
\p 15 ; ne ;  9.30 ; Bohoslužby (D. Jones) ;.;;
\p 17 ; ut ; 15.15 ; Stretnutie pri Biblii (P. Pivka, Zrínskeho 2) ;.;;
\p 22 ; ne ;  9.30 ; Bohoslužby (D. Uhrin) ;.;;
\p 29 ; ne ;  9.30 ; Bohoslužby (P. Kolárovský) ;.;;
}

\vskip3ex
\koniecprogramu


\tiraz
\bye
