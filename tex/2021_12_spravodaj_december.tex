\def\velkostpisma{10}
\def\velkostriadku{12.5}
\input makra.tex % nacitanie Ivanom pripravenych nastaveni a prikazov
\hyphenation{star-šov-stvo} % rozdelenie slov na konci riadku, treba tu uviest slova, ktore sam nepozna

\spravodaj{12}{2021}


\clanok {Vianoce sú Kristus}
Len málo vecí v~živote očakávame viac ako Vianoce. Deti počítajú dni ešte predtým, než otvoria svoj čokoládový adventný kalendár. Koncom októbra som si všimol, že v~obchodnom dome Vivo bol už otvorený vianočný obchod. Koncom novembra boli znaky Vianoc po celom meste. Veľká časť radosti z~Vianoc je očakávanie tejto nádhernej oslavy.

Dovôdom, prečo slávime adventné nedele, je, aby sme pripravili naše srdcia na Kristov príchod. Od začiatku vekov existovalo očakávanie Vianoc. Vianoce sú príbehom príchodu Vykupiteľa, Záchrancu, ktorý bol prisľúbený Adamovi a Eve ešte v~záhrade Eden. Príbeh o~Noachovej záchrane ľudstva bol takisto obrazom prichádzajúceho Mesiáša. V~starom zákone o~prísľube Syna a jeho obete nevypovedá nič lepšie, ako príbeh o~Abrahámovi a Izákovi. Očakávanie Vianoc vidíme v~celej knihe Genezis.

Roky starého zákona sa už pominuli, ale odhalili detaily o~mieste, kde sa má narodiť Mesiáš. Počas tohto mesiaca budeme sledovať príbeh Vianoc v~knihe Genezis a uvidíme, že celá táto kniha, či celé Písmo, vždy poukazovalo na narodenie Krista.

V~našom očakávaní Vianoc 2021, so všetkými opatreniami a obmedzeniami, musíme pamätať na to, že jednu vec nám nemožno vziať a táto jedna vec je hlavná. Namiesto sklamania a sťažovania sa, nech naše srdcia naplní vďaka za príchod dlho očakávaného Spasiteľa. Jeho narodenie prinieslo nádej do zlomeného a chorého sveta. Jeho narodenie a život prinášajú nádej do tohoto podobného sveta aj dnes.

Požehnaný advent nám všetkým prajem.

\autor{Danny Jones}


\clanok {Správy zo staršovstva}
V novembri sme mali tri stretnutia, ktorým dominovalo niekoľko tém.

Prvou témou bola rodinná situácia Dannyho a Clary. Spolu s~nimi prežívame zložitú a náročnú situáciu s~ich deťmi. Rozumieme, že ako rodičia cítia zodpovednosť za svoje deti a potrebu byť im blízko, keď prežívajú ťažké chvíle svojho života. Vzdialenosť medzi kontinentami robí situáciu ešte náročnejšou. Modlíme sa, aby im Pán Ježiš dával múdrosť, silu a aby bol pri nich skrze svojho Svätého Ducha. Chceme im byť nápomocní v~maximálne možnej miere modlitbami a službami, ktoré vieme prevziať namiesto nich. Veľmi si vážime Dannyho príchod na Slovensko na niekoľko dní, keď prežíval potrebu riešiť situáciu na mieste a nie na diaľku prostredníctvom techniky.

Konferencia našich zborov. Sme radi, že sme sa mohli stretnúť prostredníctvom technológií. Témy, ktoré boli rokované na diskusnej KDZ a následne schvaľované, sme predložili na zborovom členskom zhromaždení. Veľkým darom, ktorý zažívame na týchto stretnutiach, je snaha o~hľadanie jednoty v~témach, ktoré by mohli priniesť vzájomné neporozumenie. Ďakujeme všetkým, ktorí sa zúčastnili zborového členského zhromaždenia a pomohli pri rozhodovaní otázok z~konferencie.

Ako som už naznačil, ďalšou témou bolo zhromaždenie členov zboru. Sme vďační nášmu nebeskému Otcovi, že stále pridáva tých, ktorí sa chcú stať členmi Jeho duchovnej rodiny. Mohli sme prijať siedmich nových členov do nášho zboru. Každý z~nich potrebuje náš záujem o~nich a našu starostlivosť. Buďme im preto oporou na ceste za Pánom Ježišom.
Venujeme sa aj dosiahnutiu multifunkčnosti nášho priestoru na Palisádach. To, čo v~súčasnosti vieme robiť, je zmena sedenia, ktorá bola prezentovaná na ZČZ. Voláme všetkých ktorí majú čo povedať k~tejto téme, aby oslovili existujúci tím (Danny, Diana, Janko Štefko, Ľubka Kohútová a Slávo Máťuš) a predložili im svoje otázky, pripomienky alebo návrhy. Iba tak dokážeme túto vážnu tému správne uchopiť a priviesť v~pokoji do úspešného konca.

Poslednou témou, ktorú chcem spomenúť, je ďalší kazateľ nášho zboru. Túto potrebu si stále uvedomujeme. Keď prišli Jonesovci do nášho stredu, dohodli sme sa, že sa po nejakom čase začneme vážne venovať tomuto procesu. Voláme členov aj priateľov nášho spoločenstva k~modlitbám za tohto služobníka. Chceme rozumieť tomu, čo má aj v~tejto oblasti pre nás pripravené náš Pán.

S~láskou a modlitbou, aby Hospodin dal svietiť svojej tvári nad nami a bol nám milostivý.

\autor {Peter Pribula}


\clanok {Nedeľné bohoslužby v~najbližšom období}
Nedeľné zhromaždenia sa v~tomto období konajú o~9.30 hod., a to len cez online prenos, ktorý môžete sledovať po kliknutí na link: \ulink[https://bit.ly/3cgSMBG]{bit.ly/3cgSMBG}.

\clanok {Besiedka, dorast mládež}
Stretnutia besiedky sa najbližšie týždne konať nebudú.

Vzhľadom na súčasnú situáciu sme sa rozhodli začať s~online stretnutiami dorastu.
Bude sa konať cez Zoom, každý piatok o~18.00 hod. Link a prihlasovacie údaje boli zverejnené v~e-maily, ktorý obdržali rodičia dorastencov.
Mládež sa v~týchto dňoch stretávať nebude.


\clanok {Biblické hodiny}
Utorkové a štvrtkové biblické hodiny sa v~tomto období lockdown-u konať nebudú.


\clanok {Klubík}
Stretnutie mamičiek s~malými deťmi sa počas najbližších 2 týždňov neuskutoční.
V~prípade zmien v~núdzovom stave sa na mieste a čase stretnutia sestry dohodnú interne.


\clanok {Sesterské stretnutia}
Sesterské stretnutia sú takisto zrušené. Stretnutia s~Dankou Paštrnákovou zaradíme až po zlepšení epidemiologickej situácie, pravdepodobne niekedy v~novom roku. Ostatné stretnutia s~témou Blahoslavenstvá, rovnako. Zostávajme naďalej v~samoštúdiu knihy „Úvahy o~kázni na hore“, spoločne sa k~nej vrátime.


\clanok {Recenzia knihy od Johna Ortberg: Mal by som ťa radšej, keby si bol trochu viac ako ja.}
Kniha s~podnadpisom Budovanie skutočnej blízkosti.
Keď som si prečítal niekoľko prvých kapitol tejto knihy, pýtal som sa, prečo nebola táto kniha k~dispozícii v~čase mojej mladosti. Asi by som ju nedokázal čítať tak ako dnes, ale už po prvých kapitolách som ju považoval za dobrý návod, ako rozmýšľať a postupovať pri budovaní vzťahov. Z~povrchu, ale nie povrchnosti, medziľudských vzťahov postupne prechádza do hĺbky, ale nie zložitosti, vzťahu s~Bohom.

Každý z~nás túži po zdravých vzťahoch plných prijatia, uznania, podpory a lásky. Snažíme sa o~to, aby sme život prežili s~ľuďmi, na ktorých sa môžeme oprieť, spoľahnúť a zdôveriť sa im. Hľadáme tých, s~ktorými si rozdelíme svoje bremeno a ktorí znásobia našu radosť. A~dúfame, že nájdeme tých, ktorí nás príjmu takých, akí sme v~hĺbke nášho najintímnejšieho JA.

Takéto vzťahy chceme. Zranený životom, alebo skôr vzťahmi sa pýtame, či sú takéto vzťahy možné. Ak ich nevieme nájsť, držíme svoju intimitu „zamknutú na sedem zámkov“. Ortberg vo svojej knihe ukazuje na to, že vzťahy, ktoré chceme, sú možné a dokonca, že sme boli pre takýto vzťah stvorení.
Autor píše knihu akoby v~dvoch rovinách.
Prvou rovinou sú vzťahy medzi ľuďmi. Či už sú to vzťahy medzi manželmi, priateľmi, kolegami alebo rodičmi a deťmi. Na týchto vzťahoch ukazuje, aké vzťahy zažívame a zároveň ukazuje protiklad našej idealizovanej predstavy a našu túžbu zažívať vzťahy naplnené prijatím a porozumením.
Druhou rovinou je vzťah s~Bohom. Na tomto vzťahu, alebo skôr prostredníctvom vzťahu Boha k~nám, ukazuje na našu úlohu pri budovaní zdravých vzťahov. Zároveň ukazuje, že najdôležitejšou podmienkou zdravého vzťahu medzi mnou a niekým iným je dovoliť Bohu, aby bol aktívnou súčasťou týchto vzťahov.
Je úplne jedno, kto sme, čo hovorí o~nás naše okolie a aj to, ako vnímame sami seba, boli sme stvorení pre VZŤAH s~Bohom a pre vzájomné VZŤAHY.

\autor {Peter Pribula}


\clanok {Vianoce spolu}
V spolupráci s~o. z. Detská misia sme sa ako cirkevný zbor zapojili do projektu Vianoce spolu. Cieľom projektu je pred Vianocami priniesť čo najväčšiemu počtu detí Dobrú správu o~narodení Spasiteľa. Je to tiež príležitosť ako rozvíjať misijnú aktivitu zboru za múrmi kostola.

Spoločne sa modlíme za ZŠ Milana Hodžu, s~ktorou už máme isté kontakty. Prosíme za deti, učiteľov i rodičov, aby ich srdcia boli pripravené pre počutie evanjelia. Prosíme za priaznivú epidemiologickú situáciu, aby sme mohli ísť do škôl. Prosíme aj sami za seba, aby nás Pán Boh vystrojil múdrosťou, vytrvalosťou a odhodlaním byť tomuto svetu svetlom a soľou.

Do projektu ideme s~vierou, že Pán Boh má situáciu pod kontrolou. Ak by sa nám nepodarilo ísť priamo na školu, ponúkneme im vianočné video, ktoré pri praví Detská misia. Taktiež škole poskytneme vianočné letáčiky, ktoré môžu rozdať deťom. A~modliť sa môžeme vždy a za každých okolností. Pán Boh určite nenechá naše modlitby bez odozvy!

Ak budete mať k~pripravovanej akcii pripomienky, nápady alebo sa budete chcieť priamo zapojiť do vianočného programu pre deti, obráťte sa na Miriam Kešjarovú (\email{kesjarova@detskamisia.sk}). Viac info na \ulink[https://detskamisia.sk/vianoce-spolu.html]{detskamisia.sk/vianoce-spolu.html}. Ďakujeme Vám!
\vfill\break


\clanok{Zbierky za uplynulé obdobie}
Milí bratia a sestry, ďakujeme za vašu obetavosť. V~mesiaci november ste prispeli:

\vskip-1ex\begitems
* Misia: 564,00 €
* Investície: 3659,00 €
\enditems

Aj naďalej máte možnosť prispieť do „nedeľnej zbierky“, a to prevodom na účet zboru. Do poznámky pre prijímateľa, prosím, uveďte „zbierka“.

Bankové spojenie: SK36 0900 0000 0000 1147 1836, SWIFT: GIBASKBX

Ďakujeme!


\n 2.	12.	Helena	MIKLETIČOVÁ;
\n 3.	12.	Ľubica	IROVÁ;
\n 5.	12.	Tomáš 	LAURENČÍK;
\n 6.	12.	Elise	ATKINS;
\n 9.	12.	Kamila 	ZAJÍČKOVÁ;
\n 9.	12.	Ondrej	ŠKODAK;
\n 11.	12.	Vladina	LAURENČÍKOVÁ;
\n 11.	12.	Maroš	KOHÚT;
\n 13.	12.	Peter	KOLÁROVSKÝ;
\n 16.	12.	Martin	PELÍŠEK;
\n 16.	12. Pavel	KONDAČ ml.;
\n 23.	12. Eva		DOROVÁ;
\n 23.	12. Diana	DZURIAKOVÁ;
\n 25.	12. Dana	PELÍŠKOVÁ;
\n 28.	12. Dara	PLETT;
\n 29.	12. Ján		KOVÁČIK;
\n 29.	12. Daniel	ŠALING;

\narodeniny


\program{
\p  1 ; st ;.;;.;;
\p  2 ; št ;.;;.;;
\p  3 ; pi ;.;;.;;
\p  4 ; so ;.;;.;;
\p  5 ; ne ;  9.30 ; Bohoslužby (J. Szőllős);.;;
\p  6 ; po ;.;;.;;
\p  7 ; ut ;.;;.;;
\p  8 ; st ;.;;.;;
\p  9 ; št ;  6.00 ; Modlitby -- muži (Zoom);.;;
\p 10 ; pi ; 18.00 ; Dorast (Zoom) ;.;;
\p 11 ; so ;.;;.;;
\p 12 ; ne ;  9.30 ; Bohoslužby (R. Krupa) ;.;;
\p 13 ; po ;.;;.;;
\p 14 ; ut ;.;;.;;
\p 15 ; st ;  6.00 ; Modlitby -- muži (Zoom) ;.;;
\p 16 ; št ;.;;.;;
\p 17 ; pi ; 18.00 ; Dorast (Zoom) ;.;;
\p 18 ; so ;.;;.;;
\p 19 ; ne ;  9.30 ; Bohoslužby (J. Szőllős) ;.;;
\p 20 ; po ;.;;.;;
\p 21 ; ut ;.;;.;;
\p 22 ; st ;  6.00 ; Modlitby -- muži (Zoom) ;.;;
\p 23 ; št ;.;;.;;
\p 24 ; pi ; 16.00 ; Bohoslužby (D. Jones) ;.;;
\p 25 ; so ; 10.00 ; Bohoslužby (P. Kolárovský) ;.;;
\p 26 ; ne ;.;;.;;
\p 27 ; po ;.;;.;;
\p 28 ; ut ;.;;.;;
\p 29 ; st ;  6.00 ; Modlitby -- muži (Zoom) ;.;;
\p 30 ; št ;.;;.;;
\p 31 ; pi ;.; Konanie silvest. bohoslužieb oznámime v~oznamoch nášho zboru ;.;;
}
\riadokkoncaprogramu{Z bohoslužieb je zabezpečený online prenos.}
\riadokkoncaprogramu{V čase vydania spravodaja nie je osobná účasť na bohoslužbách možná.}

\tiraz
\bye
