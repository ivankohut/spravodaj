% DOKUMENTACIA:

% Prazdny riadok za textom znamena ukoncenie odstavca.
% Cierne obldzniky na konci riadku (v PDF) - to nechaj na mna (moze to o.i. znamenat, ze treba pridat nejake slovo do \hyphenation, lebo ho sam nevie rozdelit na konci riadku)

% Prikazy pre casti spravodaja:
% \spravodaj{<mesiac>}{<rok>}
% \clanok{<nazov clanku>}
% \autor{<autor clanku>}
% \n<den.mesiac.meno> - zadefinovanie oslavenca
% \narodeniny - vytvorenie tabulky s~narodeninami vsetkych zadefinovanych oslavencov
% \tiraz - ukoncenie spravodaja tirazou

% Styl fontu:
% \bf - bold, plati do konca aktualne skupiny, napr. ak mas {aaa \bf bbb} ccc, tak aaa bude normalne, bbb bude bold, ccc bude normalne
% \it - italic (pouzit rovnakym sposobom ako \bf)
% \bi - bold italic (pouzit rovnakym sposobom ako \bf)
% \rm - normalne (pouzit rovnakym sposobom ako \bf)

% Dalsie prikazy a znaky:
% \begitems - zoznam (odrazky), informacie najdes na stranke http://petr.olsak.net/ftp/olsak/opmac/opmac-u.pdf#toc%3A.5
% \ulink[<cielova adresa]{<zobrazena adresa>} - klikatelny odkaz na webstranku
% \email{<adresa>} - klikatelny odkaz na e-mailovu adresu
% ~ - nedelitelna medzera, napr. v~dome, 21.~6.~2018
% -- - pomlcka (dvakrát -)
% „ - zaciatocna uvodzovka
% “ - koncova uvodzovka
% \noindent - najblizsi odstavec nebude odsadeny
% \vskip<velkost> - vertikalna medzera, napr. \vskip3pt alebo \vskip-3ex (zaporna medzera, t.j. posun smerom hore)


\input makra.tex % nacitanie Ivanom pripravenych nastaveni a prikazov
\hyphenation{star-šov-stvo} % rozdelenie slov na konci riadku, treba tu uviest slova, ktore sam nepozna

\spravodaj{4}{2018}


\clanok{Hradby a brány}
{\it „A riekli mi: Pozostalí, ktorí pozostali zo zajatia tam v~tej krajine, sú vo veľkom súžení a v~potupe, i múr Jeruzalema je rozborený, a jeho brány sú spálené ohňom. A stalo sa, keď som počul tie slová, že som sadol a plakal som a smútil som niekoľko dní, postil som sa a modlil pred Bohom nebies.“} Nehemiáš 1, 3 

Keď sa Nehemiáš dopočul, že hradby Jeruzalema sú zborené a brány spálené ohňom, neprešiel okolo tejto skutočnosti bez pohnutia a zlomenosti. 

Prečo tisícky mužov v~Jeruzaleme a Judsku, ktorí sa vrátili a pozerali možno dennodenne na zborené hradby, ich brali ako nezmeniteľnú súčasť ich života, ich mesta a doby? Alebo tisícky židovských mužov v~Babylone, ktorí dostávali takéto správy ako Nehemiáš, s~tým proste žili ďalej, lebo sa nejednalo o~ich vlastné hradby... Proste to vzali ako fakt, s~ktorým sa treba zmieriť a naučiť žiť. 

Možno rovnako my dnes mnohé zborené hradby (nášho zboru alebo našich rodín, manželstiev, či osobného života...) nevidíme, nechceme vidieť, alebo sa s~nimi zmierujeme ako s~niečím, čo proste doba priniesla, čo sa stalo.

Ako sú na tom tvoje hradby?

Hradby predstavujú funkčné hranice. Niečo je vo vnútri – niečo naše vlastné, vzácne, jedinečné. Sme za to zodpovední. Určuje to naše hodnoty a identitu. Žijeme preto. Sme tam doma. Rastú tu naše deti. Je to niečo, čo im chceme zachovať a raz odovzdať...

No a niečo je vonku. Teda pokiaľ hradby stoja a fungujú. Niečo, čomu hovoríme nie, pretože by to ohrozilo a pokazilo všetko, čo sme a čo máme vo vnútri, čo nám Boh zveril, k čomu nás povolal.

{\it „Mesto zbúrané, bez hradieb, to je muž, čo sa neovláda.“} 	Príslovia 25, 28

Človek, ktorý nemá hranice, zábrany, ktorý sa neovláda, ktorý neovláda svojho ducha, je ako mesto bez hradieb. Ak neobnovíme svoje hradby, dávame priestor diablovi v~našom živote (rodine, manželstve, výchove detí...).

{\it „V noci som vyšiel Údolnou bránou k Dračiemu prameňu až k Hnojnej bráne a skúmal som hradby Jeruzalema, ktoré boli samá trhlina, a ktorého brány pohltil oheň...“} Nehemiáš 2, 13

Nehemiáš podnikol nočný výjazd, aby si poriadne prezrel stav hradieb. Zrejme aj od nás to bude chcieť dôkladnejšiu obhliadku, skúmanie stavu našich hradieb, identifikovanie zborených úsekov. Akokoľvek je to nepohodlné a konfrontačné, predsa vás k tomu pozývam. Začnime svojimi osobnými hradbami. Hradby mysle, srdca, očí. Hradby priorít a hodnôt. Hradby času a vôbec všetkého, čo máme v~správe. Hradby vo vzťahoch... Nemusia zostať zborené, či nefunkčné. Naberieš to odhodlanie uvidieť a identifikovať zhoreniny v~tvojich hradbách?

Keď Nehemiáš počul, že brány Jeruzalema sú spálené ohňom, úplne ho to zložilo. Sadol si, plakal, modlil sa celé dni, robil pokánie a prosil o~urgentnú zmenu toho stavu. Prečo taká radikálna reakcia? Prečo taká zlomenosť, bremeno a odhodlanie to zmeniť?

Tie brány nepredstavovali len obranu a rozsudzovanie, čo príjmeme a čo odmietneme. Zároveň predstavovali aj správu celého mesta. Tam mali sedieť starší, tam sa konali súdy a uzatvárali sa dôležité obchody a zmluvy. Navyše každá z~tých brán mala svoj dôležitý význam. Bez brán bol Jeruzalem bezbranný, bez správy, bez súdnictva, bez bezpečného trhu, bez vymožiteľnosti práva... Život na takom mieste nie je ľahký a už asi chápeme, prečo bolo málo osídlené a navrátilci sa tam nechceli vrátiť (Neh 7, 4; 11, 1 - 2).

Úlohou brány je nevpustiť dovnútra nič, čo chce ničiť a zároveň má byť otvorená veciam, ktoré môžu byť pre teba a tvojho blížneho bohatým prameňom radosti, pokoja, spoločenstva a rastu.                

\autor{Michal Kevický}


\clanok{Správy zo staršovstva}
Naše pravidelné stretnutia sme mali 6. a 20.~marca.

Nakoľko je začiatok roka, témy našich stretnutí sa niesli v~znamení prípravy Výročného zborového členského zhromaždenia (VZČZ). Pripravovali sme programovú náplň VZČZ, zapojenie jednotlivých členov staršovstva a zboru do programu, ale aj tému, ktorá je asi najviac diskutovaná, a to je hospodárenie v~minulom roku a rozpočet na nový rok. Pre zjednodušenie jednania na VZČZ sme vytvorili časový priestor pred stretnutiami staršovstva, kde všetci záujemcovia mohli prísť s~otázkami, návrhmi a požiadavkami na tvorbu rozpočtu. Požiadavky tých, ktorí túto možnosť využili, boli zapracované do predloženého návrhu.
Následne po termíne konania VZČZ sme hodnotili priebeh a diskusiu k jednotlivým bodom rokovania.

Ďalšou veľkou témou bola návšteva Dannyho a Clary Jonesovcov v~našom zbore. Sme veľmi vďační nášmu Pánovi, že sme mohli s~nimi stráviť aspoň niekoľko dní pred tým, ako sa presťahujú na Slovensko a začnú svoju prácu v~našom zbore. Nemôžem v~tejto súvislosti nespomenúť naliehavú potrebu modlitieb za úspešný predaj ich domu v~Amerike. Je to pre Dannyho a Claru veľmi dôležitá vec. Veríme tomu, že Pán Ježiš má ten najlepší čas, ale chce počuť aj naše modlitby za túto vec. Prosíme vás o~to, aby ste do svojich modlitieb zaradili aj úplne praktickú tému predaja ich domu.
Táto minca má aj druhú stranu a tou je rekonštrukcia bytu na Zrínskeho. Rekonštrukcia prebieha. Niektoré práce sú zabezpečované dodávateľským spôsobom. Ostáva však veľa práce, ktorú je potrebné zabezpečiť našimi rukami. {\bf Chceme aj touto cestou vyzvať všetkých nás k tomu, aby sme na najbližšie obdobie odložili všetko, čo nie je nutné a zapojili sa do brigádnickej práce pri rekonštrukcii bytu.} Času nie je veľa a musíme ho využiť múdro.
Danny a Clara majú kúpené letenky a do Bratislavy prídu 23. mája. Do tohto termínu je nutné, aby sme ukončili všetky práce na rekonštrukcii bytu a presune kancelárie na prízemie.

Spolu s~týmito informáciami a výzvou chceme dať do pozornosti ešte jeden termín. Ten zatiaľ nie je presne určený, ale jedná sa {\bf o~pomoc pri vykladaní kontajnera}, v~ktorom prídu ich veci. Na vyloženie je iba veľmi krátky čas, pretože, ak nebude vyložený v~stanovenom čase, je nutné platiť vysoký poplatok za prestoj. Prosíme, aby ste / sme si všetci rezervovali vo svojich termínoch čas na túto pomoc.

Nie zanedbateľnými témami boli príprava služieb v~Bratislave a na Chvojnici. Chceme sa poďakovať všetkým, ktorí sú ochotní zapojiť sa do služieb slovom, spevom, moderovaním, alebo akýmkoľvek iným spôsobom.

{\bf V apríli sa bude konať diskusná konferencia delegátov zborov.} Príprava našich tém, kandidátov na funkcie členov Rady BJB a komisií bola tiež súčasťou našich diskusií. DKDZ sa bude konať 14. apríla 2018. Prosíme o~vaše modlitby, aby sme pri diskusiách dokázali okrem múdrosti darovanej našim Pánom počúvať hlavne to, čo chce On. Bez Jeho múdrosti a Jeho požehnania to bude iba ľudské dielo, ktoré veľmi rýchlo padne. Našou túžbou je, aby to nebolo dielo človeka, ale aby sme rozumeli Božej vôli a vedeli ju prijať.

Zbor v~Podunajských Biskupiciach nás požiadal o~pomoc pri financovaní záverečných prác pri rekonštrukcii ich modlitebne. Podrobnú informáciu prinesieme v~oznamoch 15.~apríla.

{\bf Náš zbor aj jednotlivci sa zapájajú do práce s~bezdomovcami v~Bratislave.} Aktuálne je túto prácu potrebné podporiť osobnou účasťou pri vydávaní stravy a rozhovoroch s~bezdomovcami. Voláme tých, ktorým Pán Ježiš kladie na srdce túto prácu, aby sa prihlásili u~br. Mira Antalíka alebo s.~Beatky Bogárovej, prípadne u~členov staršovstva.

Ďakujeme za vašu podporu a modlitby.

\autor{Za staršovstvo zboru Peter Pribula} 


\clanok{Príprava na manželstvo}
Trojdňové pobyty určené všetkým párom, ktoré plánujú spoločnú budúcnosť a chcú pre svoje manželstvo položiť pevné základy. Kurz je založený na kresťanských princípoch a ponúka množstvo praktických rád, ako sa na spoločný život pripraviť a ako ho potom pestovať a chrániť.

Víkendový pobyt pozostáva z~5 tematických blokov:
\vskip-1.2ex
\begitems
* význam, prínos a dôsledky vzájomného záväzku
* účinná komunikácia, spoznávanie vzájomných odlišností a porozumenie
* riešenie konfliktov a uzdravovanie zranení
* budovanie priateľstva a vyjadrovanie si lásky tak, aby sa počas rokov nestratila
* stanovenie spoločných cieľov a hodnôt
\enditems
Po každom z~blokov je priestor, aby páry vo dvojiciach hovorili o~tom, čo počuli, čo by radi aplikovali, čoho sa obávajú či očakávajú,… a tiež aby si mohli vychutnať spoločný čas.

Lektormi kurzu sú manželia Vlado a Renáta Sochorovci.

Najbližší kurz: {\bf 25. -- 27. mája 2018, hotel Drotár, Hronec}

\nobreak Info a prihlášky: \email{rodinysposlanim@gmail.com}, 0904~362~439


\clanok{Dobrodružstvo s~otcom}
Nezabudnuteľný víkend pod stanom vo dvojici otec -- syn.

Dátum: 8. – 10. 6. 2018

Miesto: Lučatín
\begitems
* špeciálny čas pre otca a jeho syna vo veku 12 -- 14 rokov -- dobrodružstvo, napätie, spolupráca aj zábava
* nové podnety a výzvy
* lepšie porozumenie otcovstvu
* hlbšie poznanie Boha ako Otca
\enditems

Informácie a prihlásenie na adrese: \email{dobrosotcom@gmail.com} alebo telefón 0903~722~439 (Vlado Sochor)


\clanok{Ponuka táborov Detskej misie}
Ak premýšľate, ako by ste vašim deťom vyplnili letné prázdniny, vyberte si niektorý z~turnusov, ktoré ponúka Detská misia, o. z. v~stredisku Prameň pod hradom Červený Kameň.
\begitems
* 7. -- 14. júla 2018 – deti (7 -- 11 rokov)
* 14. -- 21. júla 2018 – športový tábor (9 -- 12 rokov)
* 21. -- 28. júla 2018 – deti (7 -- 11 rokov)
* 28. júla -- 4. augusta 2018 – starší dorast (15 -- 18 rokov)
* 4. -- 11. augusta 2018 – mladší dorast (12 -- 15 rokov)
* 11. -- 18. augusta 2018 – deti (7 -- 11 rokov)
\enditems
Viac informácií a elektronickú prihlášku nájdete na \ulink[http://www.detskamisia.sk]{www.detskamisia.sk}.


\clanok{Senior klub}

Ak dá Pán zdravia a života, stretneme sa posledný štvrtok v~mesiaci apríl, a to dňa {\bf 26.~4.~2018 od~10.00 do~14.00~hod.} na Súľovskej ul. Tému určíme dodatočne. Všetci sú srdečne vítaní!

\autor{Jana Makovíniová}


\clanok Kalendár najbližších akcií
\begitems
* Spolu v~misii, 27. -- 28. apríla 2018, chata Račkova dolina
* Deň misie BJB, 30. apríla 2018
* Konferencia sestier, 4. -- 5. mája 2018, Košice
\enditems


\clanok{Pomoc ľuďom bez domova}
Milé sestry a milí bratia, varenie polievok ľuďom v~núdzi pokračuje. Prosím, ak sa chcete zapojiť, vyberte si z~nasledujúcich termínov:

15. máj (utorok); 19. jún (utorok); 17. júl (utorok); 21. august (utorok); 18. september (utorok); 20. október (sobota); 17. november (sobota); 15. december (sobota)

Na rezervované termíny sa, prosím, nahláste u~mňa.

\autor{Beata Bogárová}


\clanok{Zbierky za marec}
Milí bratia a sestry, ďakujeme za vašu obetavosť. V mesiaci marec ste prispeli:
\vskip-1ex\begitems
* misia: 255,10 € 
* investičný fond: 270,00 €
\enditems


\clanok{Klub D.E.P.O. na Súľovskej}
D.E.P.O. junior je v~piatok od 16.00 do 18.00 hod. a je určený hlavne dorastu. D.E.P.O. je každý piatok od~18.00 do~22.00 hod. a je určený mládeži. Náplňou klubov sú moderné spoločenské hry, stolný futbal, pingpong, posedenie, rozhovory, hudba, občerstvenie, tvorivé dielne.


\clanok{Skupina anonymných alkoholikov Maják}
AA-liečebný program anonymných alkoholikov má svoje pravidelné stretnutia vo štvrtok od 17.30 do 18.30 hod. v~našich zborových priestoroch na Zrínskeho 2.

\autor{Kontakt: Želka 0903 294 927}


\n 1.	4.	Miroslav	KOLÁŘIK;
\n 4.	4.	Vierka 	ŠKODÁK; 
\n 6.	4.	Jana 	ZAJACOVÁ; 
\n 6.	4.	Jarmila	CIHOVÁ;
\n 10.	4.	Anna 	PAVLÍKOVÁ;
\n 11.	4.	Daniel 	MIKLETIČ;
\n 16.	4.	Blažena ŠKULECOVÁ; 
\n 19.	4.	Marta 	PRIBULOVÁ st.;
\n 22.	4.	Koloman ERDÉLYI;
\n 25. 4.	Elena	TALIGOVÁ;
\n 30. 4.	Jaroslav 	VOLENTIČ; 
\n 30. 4.	Ľuboš 	DZURIAK; 
\narodeniny


\program{
\p 1  ; ne ; 9.30 ; Bohoslužby (J. Szőllős);.;;
\p 2  ; po ;.;;.;;
\p 3  ; ut ; 15.00 ; Popoludnie pri Biblii (P. Pivka, Zrínskeho 2);.;;
\p 4  ; st ;.;;.;;
\p 5  ; št ; 19.00 ; Biblická hodina (J. Szőllős, Zrínskeho 2);.;;
\p 6  ; pi ; 16.00 ; D.E.P.O. (Súľovská 2);.;;
\p 7  ; so ; 18.00 ; Mládež (Súľovská 2);.;;
\p 8  ; ne ; 9.30 ; Bohoslužby (T. Valchář); 10.00; Chvojnica (P. Škulec);
\p 9  ; po ; 18.00 ; Modlitby (Zrínskeho 2);.;;
\p 10  ; ut ; 15.00 ; Popoludnie pri Biblii (P. Pivka, Zrínskeho 2);.;;
\p 11  ; st ;.;;.;;
\p 12  ; št ; 19.00 ; Biblická hodina (J. Szőllős, Zrínskeho 2);.;;
\p 13  ; pi ; 16.00 ; D.E.P.O. (Súľovská 2);.;;
\p 14  ; so ; 18.00 ; Mládež (Súľovská 2);.;;
\p 15  ; ne ; 9.30 ; Bohoslužby (M. Kolářik); 10.00; Chvojnica (P. Kolárovský);
\p 16  ; po ; 18.00 ; Modlitby (Zrínskeho 2);.;;
\p 17  ; ut ; 15.00 ; Popoludnie pri Biblii (P. Pivka, Zrínskeho 2);.;;
\p 18  ; st ;.;;.;;
\p 19  ; št ; 19.00 ; Biblická hodina (J. Szőllős, Zrínskeho 2);.;;
\p 20  ; pi ; 16.00 ; D.E.P.O. (Súľovská 2);.;;
\p 21  ; so ; 18.00 ; Mládež (Súľovská 2);.;;
\p 22  ; ne ; 9.30 ; Bohoslužby (J. Kohút); 10.00; Chvojnica (P. Pribula ml.);
\p 23  ; po ; 18.00 ; Modlitby (Zrínskeho 2);.;;
\p 24  ; ut ; 15.00 ; Popoludnie pri Biblii (P. Pivka, Zrínskeho 2);.;;
\p 25  ; st ;.;;.;;
\p 26  ; št ; 19.00 ; Biblická hodina (J. Szőllős, Zrínskeho 2);.;;
\p 27  ; pi ; 16.00 ; D.E.P.O. (Súľovská 2);.;;
\p 28  ; so ; 18.00 ; Mládež (Súľovská 2);.;;
\p 29  ; ne ; 9.30 ; Bohoslužby (I. Čonka); 10.00; Chvojnica (V. Ira);
\p 30  ; po ; 18.00 ; Modlitby (Zrínskeho 2);.;;
}

\tiraz
\bye