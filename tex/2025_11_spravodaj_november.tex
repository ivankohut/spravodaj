\def\velkostpisma{9}
\def\velkostriadku{12}
\input makra.tex % nacitanie Ivanom pripravenych nastaveni a prikazov
\hyphenation{star-šov-stvo} % rozdelenie slov na konci riadku, treba tu uviest slova, ktore sam nepozna

\spravodaj{11}{2025}


\clanok{Čas na zjednotenie}

„…vybudovaní na základe apoštolov a prorokov, pričom uholným kameňom je sám Ježiš Kristus, na ktorom každé stavanie, (príslušne) pospájané, rastie v~chrám svätý v~Pánovi.“ (Ef~2,20-21)

V cirkvi sme dosiahli bod, keď sa ľudia začínajú rozdeľovať kvôli naozaj malým veciam -- otázkam druhej a dokonca tretej kategórie. A~potom dochádza k~ďalšiemu rozdeleniu. Skupiny sa zmenšujú a zmenšujú, keďže podmienky na prijatie sú čoraz užšie a užšie. Čoskoro zistíš, že vravíš: „Zostali sme len štyria a nie som si istý, čo vy ostatní traja.“

Nie je ťažké nájsť veci, ktoré nás rozdeľujú. Sú to veci, ktoré sú na dosah ruky. Keď si mýlime svoje osobné preferencie a názory s~absolútnou pravdou, môžeme ospravedlniť oddelenie sa od kohokoľvek.

Nájsť veci, ktoré nás spájajú, si vyžaduje trochu viac úsilia. Alebo uvedomiť si, že veci, ktoré nás rozdeľujú, nás nemusia rozdeľovať. Namiesto automatického rozdeľovania ľudí do kategórií „my“ a „oni“ sa môžeme dohodnúť, že sa nezhodneme. Súd môžeme nechať na Pána.

To je smer, ktorým nás Boh chce viesť. Pamätaj, že si telom Kristovým. Máš spolupracovať, pričom každá časť hrá dôležitú úlohu. Nemôžeš rozdeliť telo alebo oddeliť jeho časti bez toho, aby si vážne poškodil celok. Preto apoštol Pavol vydal nasledovné varovanie veriacim v~Ríme: „Prosím vás však, bratia: pozor na tých, ktorí vyvolávajú rozbroje a pohoršenia v~protiklade s~učením, ktorému ste sa naučili. A~odvráťte sa od nich!“ (R~16,17)

Nebezpečné nie sú len učenia, ale aj rozdelenia, ktoré spôsobujú. Preto aj Pavol napísal veriacim v~Korinte: „A tak vás napomínam, bratia, menom nášho Pána Ježiša Krista, aby ste všetci rovnako hovorili a aby neboli roztržky medzi vami, ale aby ste jednako mysleli a zmýšľali.“ (1K~1,10)

Rozdelenia v~cirkvi nie sú len vnútornou záležitosťou. Ich dôsledky a škody siahajú ďaleko za múry cirkvi. Ľudia mimo cirkvi nás sledujú. Neveriaci, pre ktorých je posolstvo Krista takmer príliš dobré, aby bolo pravdivé. Zmenené srdcia? Premenené životy? Chcú vidieť dôkazy. Chcú vidieť, ako žijeme podľa toho, čo hlásame.
Pokiaľ ide o~mnohých neveriacich, kresťania sú, pokiaľ nie je dokázaný opak, pokrytci. Keď vidia rozdelenie v~našich radoch alebo počujú, ako sa navzájom napádame alebo spochybňujeme legitimitu viery druhých, len to potvrdzuje ich cynické podozrenia. Cítia sa slobodní ignorovať naše posolstvo a odmietnuť dobrú zvesť o~Kristovi.

Pavol povedal: „a usilujte sa zachovávať jednotu ducha vo zväzku pokoja.“ (Ef~4,3). Vyvíjaj všetko úsilie. To nenecháva priestor na polovičaté pokusy. Boh chce, aby sme uprednostňovali jednotu.
Jednota medzi veriacimi nie je len otázkou toho, aby sme vychádzali so všetkými alebo sa správali k~ostatným milo; je to tiež jedno z~najsilnejších svedectiev, ktoré môžeme dať, aby sme ukázali život meniacu silu Ježiša Krista.

Otázka na zamyslenie: Ako môžeš prekonať bariéry, ktoré ťa oddeľujú od niektorých členov tvojej cirkvi?
\autor{Greg Laurie}


\clanok{Správy zo staršovstva}
Drahí bratia a sestry, staršovstvo zboru BJB Palisády sa v~októbri stretlo dva razy, a to 14.~10. a 28.~10.~2025.

Na prvom stretnutí sme mali rozhovor s~kandidátmi na členstvo v zbore. Ďalej si členovia staršovstva rozdelili starostlivosť o~členov. Venovali sme sa aj úprave zborového poriadku, príprave zborového členského zhromaždenia a príprave oblastného vďakyvzdania.
Nasledovali rozhodnutia o bodov DKDZ. V rámci rôzneho sme riešili mzdy kazateľov, zhodnotili stretnutie organizačného tímu „vzdelávanie na Palisádach“, podporu Revúcej a deň diakónie.

Na ďalšom stretnutí 28.~10.~2025 staršovstvo zhodnotilo ZČZ. Boli sme informovaní o prípravách rekonštrukcie fasády. Ďalej sme sa zaoberali financiami: rozpočet, Súľovská, oblastné stretnutie, podpora Revúcej. Riešil sa termín zborovej víkendovky v zariadení Berea a tiež vzdelávanie na Palisádach. Na záver sme plánovali kontemplatívne bohoslužby na advent. Ďalšie stretnutie staršovstva sa uskutoční 11.~11.~2025.
\autor{Marcel Maďar}


\clanok{Projekt „Rozhovory na Palisádach“}
Milí bratia a sestry, priatelia,

radi by sme vám predstavili náš znovu oprášený projekt vzdelávania, ktorý pred rokmi organizoval náš zbor.
\vskip-1ex\begitems
* Názov: „Rozhovory na Palisádach“
* Kedy: Raz mesačne v~sobotu, v~časoch od~9.30 do~11.30 hod.
* Kde: Kostol BJB na Palisádach
* Forma: Prednášky s~možnosťou klásť otázky
* Prednášajúci: Kresťania profesionáli, ktorí tému dôverne poznajú
* Témy: Praktické témy zo života a pre život
\enditems

\noindent V tejto chvíli máme predbežne naplánované nasledovné témy a rečníkov:
\vskip-1ex\begitems \style n
* 6.~12.~2025, výnimočne v~čase od~14.00 do~16.00~hod.: Marián Možucha: „Umelá inteligencia -- nádej alebo hrozba?“
* Január 2026: „Zmierenie na Slovensku“ (Dušan Uhrin)
* Február 2026: „O manželstve“ (Staroňovci)
* Marec 2026: „Izrael a Gaza“ (Peter Švec)
\enditems

Budeme vám vďační za vaše modlitby za tento projekt a sú vítané aj vaše podnety, ktoré môžete posielať na emailovú adresu: \email{vzdelavanie@bjbpalisady.sk}

Registrácia na prednášku „Umelá inteligencia -- nádej alebo hrozba?“
je spustená: \ulink[https://forms.gle/UERQXcxuxgWMG55Q6]{forms.gle/UERQXcxuxgWMG55Q6}
\autor{organizačný tím}


\clanok{Krst}
Náš zbor pripravuje krst v~termíne nedeľa 30.~11.~2025 o~15.00~hod. v~CASD na Cablkovej~3. Kto túži vyznať svoju vieru v~krste, nech sa prihlási u~br. kazateľa Janka Szőllősa, alebo u~členov staršovstva.


\clanok{Obedy s~bratmi}
Bratia Janko Szőllős, Filip Barkóczi a Dávid Chuchút prijímajú naše pozvania na obed do našich domácností. Prosíme, zapíšte sa do online tabuľky.


\clanok{Nadácia Integra -- benefičný koncert}
Srdečne vás pozývame na benefičný koncert so skupinou Fragile, ktorý sa uskutoční v~pondelok 24.~11.~2025 o~18.30~hod. v~Zrkadlovej sieni Primaciálneho paláca v~Bratislave.
Poďme spolu dokončiť sen o~bezpečnom bývaní pre takmer 100~detí z~ulíc Nairobi.
Možnosť registrácie na tomto odkaze: \ulink[https://integra.sk/koncert-s-fragile-2025/]{integra.sk/koncert-s-fragile-2025}


\clanok{Zbierky v~októbri}
\table{lr}{
Na misiu			& 729~€ \cr
Na investičný fond 	& 362~€ \cr}
\vskip1em

Pripomíname, že zbierky sa u~nás konajú pravidelne takto:
\vskip-1ex\begitems
* každú 2. nedeľu v~mesiaci je zbierka venovaná misii a
* každá 4. nedeľa je zbierka na investície.
\enditems

Aj naďalej máte možnosť prispieť do „nedeľnej zbierky“, a to prevodom na účet zboru. Do poznámky pre prijímateľa, prosím, uveďte „zbierka“.

Bankové spojenie: SK36 0900 0000 0000 1147 1836, SWIFT: GIBASKBX


\n 2.	11.	Tomáš	VALCHÁŘ;
\n 5.	11.	Katarína	VALENTOVÁ;
\n 6.	11.	Elena	PRIBULOVÁ;
\n 6.	11.	Eva	SYČOVÁ;
\n 9.	11.	Radovan	PAULEN;
\n 15.	11.	Bohumila	ŠALINGOVÁ;
\n 19.	11.	Dávid	PRIBULA;
\n 21.	11.	Ladislav	KAMOCSAI;
\n 22.	11.	Alena	SVOBODOVÁ;
\n 22.	11.	Peter	PRIBULA;
\n 25.	11.	Petra	ŠALINGOVÁ;
\n 27.	11.	Judita	KOLÁŘIKOVÁ;
\n 29.	11.	Jaroslav	KRÁĽ;

\narodeniny


\program{
\p  1 ; so ;.;;.;;
\p  2 ; ne ;  9.30 ; Bohoslužby (V.~Pototskyi + VP) ;.;;
\p  3 ; po ; 18.00 ; Modlitebná skupinka ;.;;
\p  4 ; ut ; 15.00 ; Biblická hodina pre seniorov (P.~Pivka) ;.;;
\p  5 ; st ;.;;.;;
\p  6 ; št ; 18.00 ; Biblická hodina (F.~Barkóczi) ;.;;
\p  7 ; pi ; 17.30 ; Dorast ;.;;
\p  8 ; so ; 18.00 ; Mládež ;.;;
\p  9 ; ne ;  9.30 ; Bohoslužby (K.~Mészáros) ;.;;
\p 10 ; po ; 18.00 ; Modlitebná skupina / Skupinka „Základy viery“ ;.;;
\p 11 ; ut ; 15.00 ; Biblická hodina pre seniorov (P.~Pivka) ;.;;
\p 12 ; st ; 17.30 ; Stretnutie sestier -- SDM ;.;;
\p 13 ; št ; 18.00 ; Biblická hodina (D.~M.~Chuchút) ;.;;
\p 14 ; pi ; 17.30 ; Dorast ;.;;
\p 15 ; so ; 18.00 ; Mládež ;.;;
\p 16 ; ne ;  9.30 ; Bohoslužby (D.~M.~Chuchút) ;.;;
\p 17 ; po ; 18.00 ; Modlitebná skupinka ;.;;
\p 18 ; ut ; 15.00 ; Biblická hodina pre seniorov (P.~Pivka) ;.;;
\p 19 ; st ;.;;.;;
\p 20 ; št ; 18.00 ; Biblická hodina (F.~Barkóczi) ;.;;
\p 21 ; pi ; 17.30 ; Dorast ;.;;
\p 22 ; so ; 18.00 ; Mládež ;.;;
\p 23 ; ne ;  9.30 ; Bohoslužby (F.~Barkóczi) ;.;;
\p 24 ; po ; 18.00 ; Modlitebná skupina / Skupinka „Základy viery“ ;.;;
\p 25 ; ut ; 15.00 ; Biblická hodina pre seniorov (P.~Pivka) ;.;;
\p 26 ; st ; 17.30 ; Stretnutie sestier ;.;;
\p 27 ; št ;.;;.;;
\p 28 ; pi ; 17.30 ; Dorast ;.;;
\p 29 ; so ; 18.00 ; Mládež ;.;;
\p 30 ; ne ;  9.30 ; Bohoslužby (T.~Valchář) ;.;;
}


\tiraz
\bye
