% DOKUMENTACIA:

% Prazdny riadok za textom znamena ukoncenie odstavca.
% Cierne obldzniky na konci riadku (v PDF) - to nechaj na mna (moze to o.i. znamenat, ze treba pridat nejake slovo do \hyphenation, lebo ho sam nevie rozdelit na konci riadku)

% Prikazy pre casti spravodaja:
% \spravodaj{<mesiac>}{<rok>}
% \clanok{<nazov clanku>}
% \autor{<autor clanku>}
% \n<den.mesiac.meno> - zadefinovanie oslavenca
% \narodeniny - vytvorenie tabulky s~narodeninami vsetkych zadefinovanych oslavencov
% \tiraz - ukoncenie spravodaja tirazou

% Styl fontu:
% \bf - bold, plati do konca aktualne skupiny, napr. ak mas {aaa \bf bbb} ccc, tak aaa bude normalne, bbb bude bold, ccc bude normalne
% \it - italic (pouzit rovnakym sposobom ako \bf)
% \bi - bold italic (pouzit rovnakym sposobom ako \bf)
% \rm - normalne (pouzit rovnakym sposobom ako \bf)

% Dalsie prikazy a znaky:
% \begitems - zoznam (odrazky), informacie najdes na stranke http://petr.olsak.net/ftp/olsak/opmac/opmac-u.pdf#toc%3A.5
% \ulink[<cielova adresa]{<zobrazena adresa>} - klikatelny odkaz na webstranku
% \email{<adresa>} - klikatelny odkaz na e-mailovu adresu
% ~ - nedelitelna medzera, napr. v~dome, 21.~6.~2018
% -- - pomlcka (dvakrát -)
% „ - zaciatocna uvodzovka
% “ - koncova uvodzovka
% \noindent - najblizsi odstavec nebude odsadeny
% \vskip<velkost> - vertikalna medzera, napr. \vskip3pt alebo \vskip-3ex (zaporna medzera, t.j. posun smerom hore)


\input makra.tex % nacitanie Ivanom pripravenych nastaveni a prikazov
\hyphenation{star-šov-stvo} % rozdelenie slov na konci riadku, treba tu uviest slova, ktore sam nepozna

\spravodaj{12}{2018}


\clanok{Pár slov na úvod}
Milá rodina,

pre mňa, ako otca rodiny, sú Vianoce veľmi vzácne. Som ten, kto zabezpečuje stromček, znesie z~povaly ozdoby, zabalí kopec darčekov... Ako dobrý Američan pripravím a sám pečiem vianočného moriaka a každý rok v~obývačke staviam náš krásny taliansky Betlehem, ktorý nám prináša radosť už vyše 30 rokov. Postavy a zvieratá niekoľko týždňov čakajú vedľa prázdnych jaslí. Sú prázdne (Ježiško ešte nie je v~jasličkách) a každý rok pred postavením Betlehema bábätko Ježiša skryjem.

Aj naše deti spolu s~postavičkami z~Betlehema týždne čakali. Potom, v~noci na Štedrý večer, kým deti spali, som položil bábätko na jeho miesto v~jasliach. Deti sa ráno tešili, lebo v~noci sa narodil Pán Ježiš. Minulý rok, poprvýkrát sa mi stalo niečo neskutočné. Zabudol som, kam som Ježiša skryl. Hľadal som ho do polnoci, a nenašiel som ho. Bol som zúfalý. Vďaka Bohu, deti sú už dospelé. Inak by bolo naozaj zle. Strávili sme celé Vianoce bez Ježiška. Dokonca aj niekoľko nasledujúcich dní. Našťastie, okolo 15. januára som si spomenul, kam som ho schoval.

Som vďačný Bohu za to, že narodením Spasiteľa sa splnilo Božie zasľúbenie, dané prostredníctvom proroka Izaiáša:

{\it„Hľa, panna počne a porodí syna a dajú mu meno Emanuel, čo v~preklade znamená: Boh s~nami.“}

Pán Boh vedel, že tento svet na Neho ľahko zabudne. Zabúdame na to, že Ho máme, hlavne počas Vianoc.

Emanuel, Boh s~nami. Treba si to znovu a znovu opakovať. V~každej chvíli radosti aj sklamania, osamelosti alebo v~spoločenstve, v~čase pádov i víťazstiev, Boh je s~nami. Nemôžeme Ho stratiť, lebo On sám nám sľúbil: {\it„Nezanechám ťa, ani ťa neopustím.“} To je najlepší vianočný darček, čo môže človek v~tomto svete dostať. Emanuel, Boh s~nami. Nestrať Ho v~čase adventu. Nech nám adventné sviečky v~predvianočnom zhone pripomínajú, že Boh je prítomný, je blízko, je Emanuel, náš Ježiš s~nami.

\autor{Danny Jones}


\clanok{Voľby do staršovstva -- prieskum}
Tak ako bolo oznámené na poslednom zborovom členskom zhromaždení, vzhľadom na končiace sa volebné obdobie staršovstva zboru volebná komisia spustila proces voľby nového staršovstva zboru. Od nedele {\bf 11.~11.~2018 do nedele 2.~12.~2018} do polnoci prebieha prieskum na kandidátov členov staršovstva nášho zboru, pričom návrhy jednotlivých členov zboru môžu obsahovať ľubovoľný počet navrhnutých kandidátov na členov staršovstva, ako to samotný člen zboru uzná za vhodné.

Svoje návrhy kandidátov na členov staršovstva odovzdá člen zboru v~uvedenom termíne písomne alebo elektronicky členom volebnej komisie, ktorými sú s. Beata Bogárová, br. Ivan Kohút a br. Peter Žembery.


\clanok{Koncerty počas adventu}
\vskip-1ex\begitems
* The Young Continentals -- 2.~12.~2018, 17.00~hod., Kinosála Istropolisu, Bratislava
* Bratislava spieva GOSPEL -- 8. a 9.~12.~2018, 18.00~hod., Veľké koncertné štúdio Slovenského rozhlasu (info: \ulink[http://www.bratislavaspievagospel.sk]{www.bratislavaspievagospel.sk})
* Benefičný koncert nadácie Integra -- 11.~12.~2018, 18.30~hod., Zrkadlová sieň Primaciálneho paláca (info: \ulink[https://integra.sk]{integra.sk})
\enditems


\clanok{Detská vianočná slávnosť}
Milí rodičia, starí rodičia, priatelia, pozývame vás na detskú vianočnú slávnosť, ktorá sa uskutoční {\bf v~sobotu 8.~12.~2018 o~17.00~hod. v~modlitebni BJB Palisády}.
Bude sa niesť v~znamení myšlienky: „Každý príbeh z~Biblie šepká o~Ježišovi.“ Príďte a priveďte aj svojich kamarátov, susedov, príbuzných! Tešíme sa na vás!


\clanok{Vianočné koncerty Veľkého spevokolu}
V sobotu a v~nedeľu {\bf 15. a 16.~12.~2018} vás pozývame na Vianočný koncert. Účinkuje Veľký spevokol a komorný orchester 1. zboru BJB Bratislava Palisády.
Oba koncerty začínajú {\bf o~17.00 hod.} Všetci ste srdečne vítaní!

\break


\clanok{Nezabúdajme na spoločné modlitby}
\vskip-1ex\begitems
* Muži -- streda od 6.00~hod. do 7.00~hod., kostol na Palisádach -- okrem 26.~12.~2018
* Ženy -- pondelok od 18.00 hod., Zrínskeho 2
* Verejné modlitby za mesto a za Slovensko -- streda od 12.00 hod., kostol na Palisádach
\enditems

Priveďte na spoločné modlitby aj vašich priateľov a známych, ktorým leží na srdci naše mesto a ľudia v~ňom.


\clanok{Bohoslužby počas vianočných sviatkov}
\vskip-1ex\begitems
* 24. 12. 2018 -- 16.00 hod.
* 25. 12. 2018 -- 10.00 hod.
* 31. 12. 2018 -- 18.00 hod.
* 1. 1. 2019 -- rodinné domáce pobožnosti
\enditems

Vysvetlenie: Milý zbor, v~tomto roku vznikla výnimočná situácia. Počas 9 dní máme 6 zhromaždení po sebe. Tým, že v~zbore máme mnohé mladé rodiny s~malými deťmi a je dôležité, aby deti s~rodičmi zažili domáce bohoslužby, spolu so staršovstvom som sa rozhodol, že 1. 1. 2019 nebude zhromaždenie na Palisádach. Pripravujeme program na rodinné domáce bohoslužby, ktorý dostanete do rúk do 16. 12. 2018, aby ste sa na to mohli pripraviť. Navrhujeme, aby sme to zažili spolu s~ďalšími ľuďmi zo zboru. Znamená to, že by rodiny pozvali slobodných, starších alebo mladších, príp. novomanželov, aby v~tento deň neboli sami. Veríme, že prvý deň v~roku môže byť pre všetkých dňom povzbudenia a uctievania.

\autor{Danny Jones}


\clanok{Hokejová misia}
Matej Matušek nás pozýva do misijného projektu, ktorý sa uskutoční 3. až 12.~1.~2019. Do Bratislavy príde 24-členný hokejový tým z~East Texas Baptist University. Budú u~nás trénovať a zahrajú si aj priateľské stretnutia. V~nedeľu 6.~1.~2019 pripravujú deň na ľade pre všetky zbory a priateľov. Hokejisti chcú tiež pomáhať so službou bezdomovcom alebo s~inými potrebami zboru. Počas decembra sa dozvieme viac informácií a ako sa môžeme projektu zúčastniť.


\clanok{Decembrová séria kázní}
V decembri sa môžeme tešiť na sériu kázní, ktorá ponesie názov Kráľovstvo -- je už tu, ale nie celkom... Všetci ste srdečne pozvaní!

\break


\clanok{Verše na zapamätanie}
V nasledujúcich mesiacoch sa budeme spoločne učiť naspamäť niektoré pasáže z~Božieho slova. Prospeje to našej duši i našej mysli.

Verše na december: {\it „S radosťou budete ďakovať Otcovi, ktorý vás urobil účastnými na podiele svätých vo svetle. 13 On nás vytrhol z~moci tmy a preniesol do kráľovstva svojho milovaného Syna, 14 v~ktorom máme vykúpenie a odpustenie hriechov.“} Kolosenským~1,~12~--~14


\clanok{Zbierky za november}
Milí bratia a sestry, ďakujeme za vašu obetavosť. V~mesiaci november ste prispeli:
\vskip-1ex\begitems
* misia: 729,60 €
* investičný fond: 535,30 €
* zbierka na Svetový deň modlitieb bapt. žien venovaná rôznym projektom: 795 €
\enditems


\clanok{Služba ľuďom bez domova}
Milí bratia a sestry,

do pozornosti vám dávam zatiaľ neobsadený termín pre náš zbor na varenie polievky ľuďom v~núdzi: {\bf 15.~december}. Polievku je v~sobotu potrebné mať uvarenú už o~16.30~hod.

Nahláste sa, prosím, u~mňa. Ďakujem za ochotu poslúžiť.

\autor{Beata Bogárová}

\vfill\break


\clanok{Senior klub v~decembri a januári}
Srdečne Vás pozdravujem v~tomto adventnom období. Chcela by som Vám napísať, že v~decembri senior klub nebude.
Ak dá Pán zdravia a života, stretneme sa v~novom roku 2019, posledný štvrtok (t.~j. posledný deň) v~mesiaci, {\bf dňa 31.~januára~2019 od~10.00 hod. do 14.00~hod.} na Súľovskej ul.

Tému určíme dodatočne.

Prajeme požehnaný adventný čas v~očakávaní príchodu nášho drahého Spasiteľa.

V láske Kristovej

\autor{Jana Makovíni}


\n 2.	12.	Helena	MIKLETIČOVÁ;
\n 3.	12.	Ľubica	IROVÁ;
\n 5.	12.	Tomáš	LAURENČÍK;
\n 6.	12.	Elise	ATKINS;
\n 9.	12.	Kamila	ZAJÍČKOVÁ;
\n 11.	12.	Vladimíra	LAURENČÍKOVÁ;
\n 11.	12.	Maroš	KOHÚT;
\n 13.	12.	Peter	KOLÁROVSKÝ;
\n 16.	12.	Pavel	KONDAČ ml.;
\n 23.	12.	Diana	DZURIAKOVÁ;
\n 24.	12.	Slávka	VOLENTIČOVÁ;
\n 25.	12.	Dana	PELÍŠKOVÁ;
\n 29.	12.	Daniel	ŠALING;
\narodeniny

\program{
\p 1  ; so ; 18.00 ; Mládež (Súľovská 2);.;;
\p 2  ; ne ; 9.30  ; Bohoslužby (D. Jones); 10.00; Chvojnica (M. Antalík);
\p 3  ; po ; 18.00 ; Modlitby -- ženy (Zrínskeho 2);.;;
\p 4  ; ut ; 15.15 ; Popoludnie pri Biblii (P. Pivka, Zrínskeho 2);.;;
\p 5  ; st ; 6.00  ; Modlitby -- muži (kostol Palisády);12.00;Modlitby -- verejné (kostol Palisády);
\p 6  ; št ; 19.00 ; Biblická hodina (J. Szőllős, Zrínskeho 2);.;;
\p 7  ; pi ;.;;.;;
\p 8  ; so ; 18.00 ; Mládež (Súľovská 2);.;;
\p 9  ; ne ; 9.30  ; Bohoslužby (T. Valchář); 10.00; Chvojnica (J. Štefko);
\p 10  ; po ; 18.00 ; Modlitby -- ženy (Zrínskeho 2);.;;
\p 11  ; ut ; 15.15 ; Popoludnie pri Biblii (P. Pivka, Zrínskeho 2);.;;
\p 12  ; st ; 6.00  ; Modlitby -- muži (kostol Palisády);12.00;Modlitby -- verejné (kostol Palisády);
\p 13  ; št ; 19.00 ; Biblická hodina (J. Szőllős, Zrínskeho 2);.;;
\p 14  ; pi ;.;;.;;
\p 15  ; so ; 17.00 ; Vianočný koncert Veľkého spevokolu (Palisády);.;;
\p 16  ; ne ; 9.30  ; Bohoslužby ; 10.00; Chvojnica (P. Škulec);
\p     ;    ; 17.00 ; Vianočný koncert Veľkého spevokolu (Palisády);.;;
\p 17  ; po ; 18.00 ; Modlitby -- ženy (Zrínskeho 2);.;;
\p 18  ; ut ; 15.15 ; Popoludnie pri Biblii (P. Pivka, Zrínskeho 2);.;;
\p 19  ; st ; 6.00  ; Modlitby -- muži (kostol Palisády);12.00;Modlitby -- verejné (kostol Palisády);
\p 20  ; št ; 19.00 ; Biblická hodina (J. Szőllős, Zrínskeho 2);.;;
\p 21  ; pi ;.;;.;;
\p 22  ; so ; 18.00 ; Mládež (Súľovská 2);.;;
\p 23  ; ne ; 9.30  ; Bohoslužby (D. Jones); 10.00; Chvojnica (P. Antalík);
\p 24  ; po ; 16.00 ; Bohoslužby -- Štedrý deň (Ľ. Dzuriak);.;;
\p 25  ; ut ; 10.00 ; Bohoslužby -- 1. sviatok vianočný (T. Valchář);.;;
\p 26  ; st ;.;;.;;
\p 27  ; št ;.;;.;;
\p 28  ; pi ;.;;.;;
\p 29  ; so ; 18.00 ; Modlitby -- ženy (Zrínskeho 2);.;;
\p 30  ; ne ; 9.30  ; Bohoslužby (B. Mišina); 10.00; Chvojnica (J. Szőllős);
\p 31  ; po ; 18.00 ; Bohoslužby -- Silvester (D. Jones);.;;
}

\tiraz
\bye
