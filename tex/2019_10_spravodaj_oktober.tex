%\typosize[10/12.5]% - pouzita velkost pisma/riadku (trochu vacsie)
\input makra.tex % nacitanie Ivanom pripravenych nastaveni a prikazov
\hyphenation{star-šov-stvo} % rozdelenie slov na konci riadku, treba tu uviest slova, ktore sam nepozna

\spravodaj{10}{2019}


\clanok {Akú máme povesť?}
Zvyknem si položiť otázku, akú má náš zbor povesť alebo ja osobne? Je to dôležitá otázka. Je dobré byť známy, pokiaľ tie dôvody sú správne. V~dnešnej dobe sa o~cirkvách píše všeličo a často nie dobré. Povesť zboru je povesť ich členov a naopak. Čím si známy Ty alebo ja, tým je známy aj náš zbor. Samozrejme, naším cieľom by malo byť to, aby sme boli známi tým, čím bol aj Pán Ježiš známy. Peter sa raz rozprával o~Ježišovi s~jedným rímskym vojakom. V~Sk 10,38 Peter povedal: „Viete o~Ježišovi z~Nazareta, ako Ho Boh pomazal Duchom Svätým a mocou, takže chodil, dobre robil a liečil všetkých diablom utláčaných, pretože Boh bol s~Ním.“ Tu máme vzor dobrej povesti, čím by sme mali byť známi aj my. Sme pomazaní Duchom Svätým? Človek, ktorý je pomazaný Duchom Svätým, vie, že je pomazaný; nie kvôli daru jazykov, ale kvôli tomu, že má nepodmienečnú a neobmedzenú lásku k~ľuďom či už v~cirkvi alebo mimo nej. Pomazaný Duchom potom robí dobre a Boh je s~ním. Milujem ten výraz. Kamkoľvek Ježiš išiel, robil dobre a bolo zrejmé, že Boh je s~Ním. Stopa, ktorú po sebe zanechal, bola dobrá ako dobrá vôňa, a ľudia tým boli požehnaní. Jeho povesť bola krásna a takú povesť by sme mali mať aj my.

V okolí Palisád žije niekoľko desaťtisíc ľudí. Ak nevidia dobré skutky nášho zboru, možno sa nič o~nás nedozvedia. Možno pár ľudí občas navštívi koncert, ale väčšinou o~nás vedia málo. Zoznámil som sa s~jednou susedkou. Keď som jej povedal, čo robím, povedala: „Aha, vy ste ten žltý kostol, ktorý je stále zatvorený!“ Zľakol som sa a povedal som si: „Fíha, toto nie je povesť, ktorú by som chcel mať.“ Čo teraz? Ježiš nám dáva odpoveď. Chodil a robil dobre. Sám Ježiš nás vyzýva: „Nech tak svieti vaše svetlo pred ľuďmi, aby videli vaše dobré skutky a oslavovali vášho Otca, ktorý je v~nebesiach!“ To je náš mandát. Už vieme, čo máme robiť.

Som vďačný za tých, ktorí počas akcie {\it Milujem svoje mesto} slúžili na základnej škole. Keď sa zástupkyňa školy prišla pozrieť a videla tam 25 ľudí z~nášho zboru, bola užasnutá, že „v~dnešnej dobe sú ľudia ochotní dobrovoľne pomáhať.“ Vďaka Bohu za dobrú povesť -- povesť Ježiša, lebo sme „robili dobre“! Plánujeme organizovať aj viac takýchto „dní služby“ a verím, že ten počet dobrovoľníkov postupne narastie, lebo tým narastie aj naša povesť. Potom už nebudeme známi len ako ten žltý kostol na Palisádach so zatvorenými dverami. Susedia s~nami budú oslavovať nášho Otca v~nebesiach, lebo naše svetlo bude svietiť pred ľuďmi a budú vidieť naše dobré skutky! Veď sme Jeho dielo, stvorení v~Kristovi Ježišovi na to, aby sme konali dobré skutky, ktoré nám Boh už vopred pripravil. Buďme tým najláskavejším zborom v~Bratislave!

\autor{Danny Jones}


\clanok {Správy zo staršovstva}
Počas prázdnin sme nemali pravidelné stretnutia, ale stretávali sme sa podľa aktuálnych potrieb.

Na prvých stretnutiach v~septembri sme pripravovali program našich bohoslužieb pre najbližšie obdobie. Príprava zahŕňala okrem pravidelných stretnutí aj návštevu materského zboru vo Viedni 6.~októbra, vďakyvzdanie 3.~novembra, ale aj vianočný koncert.

Okrem programu našich spoločných bohoslužieb sme plánovali aj aktivity zamerané pre mužov a aktivity pre manželské páry.  Aktivity majú byť zamerané na budovanie vzťahu s~Bohom a budovanie vzťahov v~manželstvách.

Po skončení prázdnin a odchode do dôchodku sa práce v~stanici nášho zboru na Chvojnici ujíma brat Pavel Škulec. Ďakujeme nášmu Pánovi, že ho povolal do tejto služby a modlíme sa za múdrosť, silu a požehnanie pre neho aj pre jeho manželku.

Sme vďační nášmu Pánovi aj za Mateja Matušeka, ktorý prirodzeným spôsobom spája svoju prácu so zvestovaním evanjelia. S~jeho predstavou práce v~Srbsku sme sa mohli oboznámiť na stretnutí staršovstva a zbor v~nedeľu 8. septembra na bohoslužbách. Modlime sa naďalej za jeho službu v~Srbsku a aj za jeho manželku Kiku, ktorá ostala na Slovensku, aby dokončila školu.

Aktivita {\it Milujem svoje mesto} je medzi nami známa už niekoľko rokov. Tento rok sme mohli byť užitoční a pomôcť v~najbližšom okolí sídla nášho zboru. Je to spôsob ako ľudia okolo nás môžu vidieť našu vieru a môžeme byť pre nich „svetlom a soľou“.

Krátko po skončení prázdnin a začiatku školského roku sme sa rozlúčili s~naším bratom Radkom Hovorkom. Pán Ježiš si ho povolal k~sebe do nebeského domova. Máme v~srdci bolesť z~jeho straty, ale aj vďačnosť za všetko, čo sme skrze neho prijali od nášho Pána Ježiša. Bude nám chýbať pri stretnutiach staršovstva. Prosíme Pána Ježiša, aby sme rozumeli ceste, po ktorej vedie staršovstvo nášho zboru, ale aj za Radkovu rodinu, aby mali pokoj vo svojich srdciach, aby aj oni rozumeli tomu, čo Pán Ježiš robí v~ich životoch a aby posilňoval ich vieru v~Neho.

\autor {Za staršovstvo zboru Peter Pribula}


\clanok{Spoločné modlitby}
\vskip-1ex\begitems
* Muži -- streda {\bf od 6.00~hod. do 7.00~hod.}, kostol na Palisádach
* Ženy -- pondelok {\bf od 17.00~hod.}, Zrínskeho 2
\enditems

Priveďte na spoločné modlitby aj svojich priateľov a známych, ktorým leží na srdci naše mesto a ľudia v~ňom.


\clanok {Stretnutia sestier}
Milé sestry,

\nobreak srdečne pozývam všetkých z~vás, ktoré ste manželky, na stretnutie na Zrínskeho v~stredu 2.~10. o~17.30~hod. Téma bude {\it Dostatočnosť Božej milosti v~úlohe manželky}.

Teším sa tiež na naše ďalšie sesterské stretnutie, ktoré bude 16.~10. o~17.30~hod. v~modlitebni na Palisádach. Plánujem hovoriť o~tom, ako študovať Bibliu.

S láskou pre vás všetkých,

\autor {Clara Jones}


\clanok {Návšteva materského zboru vo Viedni}
Náš materský zbor vo Viedni na Mollardgasse si bude v~nedeľu 6.~10. pripomínať 150.~výročie svojho založenia. Vzhľadom na to, že ich priestory sú skromnejšie, budeme ich môcť navštíviť len jedným autobusom, ktorý už je plne obsadený (možnosť prihlásenia sa bola uzavretá pred niekoľkými dňami). Náklady spojené s~dopravou budú hradené zborom. Odchod autobusu je naplánovaný o~7.00 hod. spred modlitebne na Palisádach.

Okrem služby spevom a slovom by sme radi podporili náš materský zbor aj nasledovným spôsobom. Zbor vo Viedni poriada tzv. „raňajky v~parku“, kde bezdomovcom popri jedle dávajú malé balíčky, do ktorých vkladajú niektoré z~nasledovných vecí:

\vskip-1ex\begitems
* mini šampóny resp. sprchové gély na jedno použitie
* suché klobásky vo vákuovej fólii
* sladkosti (cukríky, mini čokoládky)
* kávu kapučíno balenú v~sáčkoch
* mydlo
* liek -- aspirín
* rukavice -- palčiaky
* ponožky (pre dospelých)
* hygienické potreby pre ženy (vložky, mokré utierky)
* antiperspiranty pre ženy
\enditems

{\bf Prosíme, aby ste podľa vlastného uváženia niektoré z~týchto vecí zakúpili a doniesli už túto nedeľu do zhromaždenia. V~predsieni bude na to pripravená krabica. V~prípade otázok sa obráťte na sestru Katku Valentovú.}

(Samozrejme, do tejto praktickej zbierky sa môžete zapojiť, aj keď do Viedne nepôjdete!)

Je možné k~týmto darčekom pripojiť aj osobný pozdrav na pohľadnici, môže byť po slovensky, anglicky, nemecky alebo rumunsky, lebo rôzne národnosti sa  objavujú na týchto raňajkách.

Po spoločnom zhromaždení bude v~priestoroch na prízemí podávaný obed a náš príspevok budú zákusky ku káve. Preto prosíme sestry, ktoré by mohli napiecť, aby sa ozvali sestre Katke Valentovej, od ktorej dostanú ďalšie inštrukcie.


\clanok {Služba núdznym}
Milé sestry a bratia,

ak máte chuť poslúžiť varením polievky ľuďom v~núdzi v~týchto najbližších mesiacoch, dávam vám do pozornosti voľné sobotné termíny: 19.~október, 16.~november a 14.~december.

Práve od 19.~októbra začíname so zimným režimom a opäť varíme aj v~soboty. Polievku na naše sobotné termíny je potrebné mať uvarenú do 16.30 hod., nakoľko výdaj pod mostom Lafranconi je o~17.00 hod. Vzhľadom k~tomu, že sme v~tento deň mávali menej klientov, stačí navariť 25 litrov.

Ďakujem za vaše finančné dary, modlitby a šikovným kuchárom za čas a chuť poslúžiť.

Nech vás Pán požehná a dáva silu na každý deň.

\autor {Beáta Bogárová}
\vfill\break


\clanok {Návšteva ZSS Samaritán, Tekovské Lužany}
Ak dá Pán zdravia a života, plánujeme celodennú návštevu ZSS Samaritán v~Tek. Lužanoch, a to v~stredu 2.~10.~2019. Zraz bude o~7.45~hod. na parkovisku cintorín Ružinov -- Vrakuňa. Odchod je o~8.00 hod. Predpokladaný príchod je o~18.00 hod. Predpokladaná cena dopravy vrátane obeda je 10~€. Obed je spoločný v~reštaurácii.

Téma našej služby je evanjelizačná, teda radostná so spevom. Do zariadenia Samaritán by uvítali, keby sme znova tak ako každý rok priniesli posteľnú bielizeň, paplóny, vankúše, uteráky, kuch. utierky a rôzne veci do kuchyne, ale aj knižky, bavlnu a podobné veci na ručné práce. Radi by sme im doniesli aj niečo napečené, tak prosíme sestry, aby na to nezabudli.

Uvítame, ak sa tejto návštevy zúčastnia aj tí, ktorí nenavštevujú senior klub. Srdečne sú vítaní všetci!

Prosíme všetkých, ktorí sa chcú zúčastniť tejto služby, aby sa do 20.~9. prihlásili sestrám z~prípravného výboru senior klubu v~jednotlivých zboroch, príp. e-mailom na adresu \email {makovini.jana@gmail.com} alebo telefonicky na číslo 02/4524~2592, 0903~752~974.

Prosím, keby ste na túto návštevu mysleli na svojich modlitbách.

\autor {Jana Makovíniová}


\clanok {Konferencia {\it 100 let spolu}}
Spoločná česko-slovenská konferencia BJB s~názvom {\it 100 let spolu} sa uskutoční v~termíne 25.~--~27. októbra v~Litoměřiciach v~Českej republike.
Ešte stále je možnosť prihlásiť sa, a to do konca septembra prostredníctvom webovej stránky \hbox{\ulink [https://www.stoletspolu.cz/registrace/]{stoletspolu.cz/registrace}} (pre samotné prihlásenie je po otvorení stránky potrebné kliknúť na „odkaz“ v~treťom odstavci).


\clanok {Detská misia -- dobrovoľnícka akcia {\it Partnerstvo sŕdc a rúk 2019}}
Milí bratia a sestry, priatelia Detskej misie!

Srdečne Vás pozdravujeme z~táborového strediska PRAMEŇ v~Častej. Žehnáme vám veršom: {\it No svojho služobníka Kaléba, pretože bol v~ňom iný duch a bol mi bezvýhradne oddaný, uvediem do krajiny, do ktorej už vstúpil, a jeho potomstvo ju dostane do vlastníctva.} (Num. 14,24).

Radi by sme vás pozvali na pracovnú dobrovoľnícku akciu {\it Partnerstvo sŕdc a rúk 2019}. Už tretí rok v~Častej prebieha rozširovanie ubytovacích kapacít. Rekonštruujeme budovu “C”. V~júni sme budovu bez drobných detailov dokončili. Veľká spoločenská miestnosť v~lete poslúžila niekoľkým táborom.

Ďakujeme vám, ak ste boli pri tom modlitbami, prakticky alebo finančne. Pred nami je však nemalá úloha vybudovať koreňovú čistiareň odpadových vôd pre celý táborový areál. Bez nej nemôže prebehnúť kolaudácia objektov. Pracovne táto úloha prevyšuje možnosti pracovníkov v~Častej, preto by sme vás chceli pozvať do spolupráce. Radi prijmeme vašu praktickú manuálnu pomoc, zvlášť mužov. Pomoc pri prácach prisľúbili aj niektorí dobrovoľníci zo Severného Írska, ktorí boli v~Častej už niekoľkokrát.

Miesto: Stredisko Detskej misie PRAMEŇ, Píla 27, 900 89 Častá

Termín: Od pondelka rána (7.~10.~2019) do soboty (12.~10.~2019)

Plánované práce:

\vskip-1ex\begitems
* MUŽI -- manuálne práce po výkope telesa koreňovej čistiarne bagrom; následne vytvorenie kamenného lôžka a štrkových vrstiev
* ŽENY -- bežné čistiace a upratovacie práce na ostatných budovách
\enditems

Program (pondelok -- sobota):

\vskip-1ex\begitems
* ranné stíšenia
* pracovná činnosť
* večerný duchovný program
\enditems

Pre prihlásených dobrovoľníkov bude zabezpečená strava, ubytovanie a duchovný program.

Prihlásiť sa môžete prostredníctvom
\vskip-1ex\begitems
* formulára: \ulink[https://www.detskamisia.sk/event/334/partnerstvo-ruk-a-srdc-casta-2019]{detskamisia.sk/event/334/partnerstvo-ruk-a-srdc-casta-2019}
* SMS alebo e-mailu správcovi Igorovi Pankuchovi:
\vskip-1ex\begitems
* 0908~526~161
* \email {pramen@detskamisia.sk}
\enditems
\enditems

Ak môžete, prihláste sa aj na 1 -- 2 dni najneskôr do soboty 5.~10.~2019.

Ďakujeme!

\autor {tím Detskej misie Častá}
\vfill\break


\clanok {Kniha {\it Malé veľké postavy Biblie} od J.~Pribulu}
Radi by sme vám dali do pozornosti novú knihu od zosnulého brata kazateľa Juraja Pribulu pod názvom {\it Malé veľké postavy Biblie}. Kniha približuje čitateľovi mužské a ženské postavy Písma, o~ktorých nemáme veľa informácií, no napriek tomu sú ich príbehy dôležité.

Knihu si môžete kúpiť v~nedeľu po zhromaždení, prípadne po dohode u~sestry Elenky Pribulovej. Cena je 5~€.


\clanok{Verš na zapamätanie}
Na mesiac október máme nový veršík, ktorý sa chceme spoločne učiť. Veríme, že poznanie Písma prospeje našej duši i našej mysli:

{\it „Tomu však, ktorý pôsobením svojej moci v~nás a nad to všetko môže urobiť omnoho viac, ako my prosíme alebo rozumieme, tomu sláva v~cirkvi a v~Kristovi Ježišovi po všetky pokolenia na veky vekov. Amen.“}

\autor{Kol~3,~20~--~21}


\clanok{Zbierky za uplynulé obdobie}
Milí bratia a sestry, ďakujeme za vašu obetavosť. V~uplynulom období ste prispeli:
\vskip-1ex\begitems
* investičný fond: informácia nebola v~čase uzávierky k~dispozícii
* misia: 728,--~€ (september)
\enditems


\n 2.	10.	Peter	ANTALÍK;
\n 6.	10.	Daniel	BALÁŽ;
\n 6.	10.	Michal	KAJAN;
\n 12.	10.	Barbora	PRIBULOVÁ;
\n 14.	10.	Martin	SIMON;
\n 20.	10.	Ida	PUČEKOVÁ;
\n 22.	10.	Hana	HALAMÍČKOVÁ;
\n 25.	10.	Vladimír	IRA;
\n 27.	10.	Miriam	KRÁĽOVÁ;
\n 28.	10.	František	VRABČEK;
\n 28.	10.	Ľubomír	SYČ;
\n 31.	10.	Samuel	KORIŤÁK;
\narodeniny


\program{
\p  1 ; ut ; 15.15 ; Stretnutie pri Biblii (P. Pivka, Zrínskeho 2) ;.;;
\p  2 ; st ;  6.00 ; Modlitby -- muži (kostol Palisády) ; 17.30 ; Stretn. pre manželky (Zrínskeho~2) ;
\p  3 ; št ;.;;.;;
\p  4 ; pi ;.;;.;;
\p  5 ; so ; 18.00 ; Mládež (Súľovská) ;.;;
\p  6 ; ne ;  9.30 ; Bohoslužby (T. Valchář) ; 10.00 ; Chvojnica (P. Škulec) ;
\p    ;    ; 10.00 ; Bohoslužby (D. Jones a Veľký spevokol, Mollardgasse -- Viedeň) ;.;;
\p  7 ; po ; 17.00 ; Modlitby -- ženy (Zrínskeho 2) ;.;;
\p  8 ; ut ; 15.15 ; Stretnutie pri Biblii (P. Pivka, Zrínskeho 2) ;.;;
\p  9 ; st ;  6.00 ; Modlitby -- muži (kostol Palisády) ;.;;
\p 10 ; št ; 19.00 ; Biblická hodina (J. Szőllős, Zrínskeho 2) ;.;;
\p 11 ; pi ;.;;.;;
\p 12 ; so ; 18.00 ; Mládež (Súľovská) ;.;;
\p 13 ; ne ;  9.30 ; Bohoslužby (D. Jones) ; 10.00 ; Chvojnica (J. Szőllős) ;
\p 14 ; po ; 17.00 ; Modlitby -- ženy (Zrínskeho 2) ;.;;
\p 15 ; ut ; 15.15 ; Stretnutie pri Biblii (P. Pivka, Zrínskeho 2) ;.;;
\p 16 ; st ;  6.00 ; Modlitby -- muži (kostol Palisády) ; 17.30 ; Stretnutie sestier (kostol Palisády) ;
\p 17 ; št ; 19.00 ; Biblická hodina (J. Szőllős, Zrínskeho 2) ;.;;
\p 18 ; pi ;.;;.;;
\p 19 ; so ; 18.00 ; Mládež (Súľovská) ;.;;
\p 20 ; ne ;  9.30 ; Bohoslužby (D. Jones); 10.00 ; Chvojnica (P. Škulec) ;
\p 21 ; po ; 17.00 ; Modlitby -- ženy (Zrínskeho 2) ;.;;
\p 22 ; ut ; 15.15 ; Stretnutie pri Biblii (P. Pivka, Zrínskeho 2) ;.;;
\p 23 ; st ;  6.00 ; Modlitby -- muži (kostol Palisády) ;.;;
\p 24 ; št ; 19.00 ; Biblická hodina (J. Szőllős, Zrínskeho 2) ;.;;
\p 25 ; pi ;.;;.;;
\p 26 ; so ; 18.00 ; Mládež (Súľovská) ;.;;
\p 27 ; ne ;  9.30 ; Bohoslužby (Ľ. Dzuriak); 10.00 ; Chvojnica (M. Kolářik) ;
\p 28 ; po ; 17.00 ; Modlitby -- ženy (Zrínskeho 2) ;.;;
\p 29 ; ut ; 15.15 ; Stretnutie pri Biblii (P. Pivka, Zrínskeho 2) ;.;;
\p 30 ; st ;  6.00 ; Modlitby -- muži (kostol Palisády) ; 17.30 ; Stretn. pre manželky (Zrínskeho~2) ;
\p 31 ; št ; 19.00 ; Biblická hodina (J. Szőllős, Zrínskeho 2) ;.;;
}


\tiraz
\bye
