%\typosize[10/12.5]% - pouzita velkost pisma/riadku - trochu vacsie
\input makra.tex % nacitanie Ivanom pripravenych nastaveni a prikazov
\hyphenation{star-šov-stvo} % rozdelenie slov na konci riadku, treba tu uviest slova, ktore sam nepozna

\spravodaj{2}{2020}


\clanok {29. február 2020}
Tento mesiac, február 2020, je trochu zvláštny.

Prvou zvláštnosťou je to, že má 29 dní. Túto zvláštnosť zažívame raz za štyri roky. Je to zvláštnosť, s~ktorou Stvoriteľ urobil našu slnečnú sústavu. Môže sa nám to zdať zvláštne alebo nepríjemné, ale keď sa pozrieme do Genezis 1,31, nájdeme tam jasné vyjadrenie, čo si o tom myslí samotný Stvoriteľ: {\it  „Boh videl, že všetko, čo utvoril, bolo veľmi dobré.“} To znamená, že aj nepravidelnosti v~počte dní v~roku Pán Boh považuje za veľmi dobré. V~mysliach tých, ktorí potrebujú poznať dôvody, môže táto vec vyvolávať otázky. A~je to vždy dobré, keď uvažujeme nad Božím dielom. Nás veriacich to vedie k~pokore a poznaniu toho, že nemusíme rozumieť všetkému, čo Boh robí vo svete a aj v~nás. Ale ak sa Mu odovzdáme, môžeme to prijať s~vierou, že všetko, čo robí v~našich životoch, je veľmi dobré. Teraz okamžite vznikne otázka: „Načo sú dobré ťažké alebo ‚zlé‘ situácie v~našom živote?“ Prvý list Timotejovi 2,4 úplne jednoznačne odpovedá na to, aký je Boží zámer pre naše životy: {\it „… ktorý} (teda Boh) {{\it chce, aby všetci ľudia boli {\bf spasení} a {\bf spoznali pravdu} ...“}}. Nerozumieš tomu, čo Boh robí v~tvojom živote? Pozri sa na to z~Jeho pohľadu. On chce, aby si bol spasený. Verím tomu, že ti bude ukazovať cestu, ktorou ťa vedie k~spaseniu a k~večnosti trávenej s~Ním.

Druhou zvláštnosťou 29. februára 2020 sú parlamentné voľby. Nie, nejdem agitovať. Ale téma voľby alebo výberu medzi aspoň dvoma možnosťami je jedným z~hlavných atribútov kresťanstva. V~Biblii je veľa textov, ktoré hovoria alebo vedú k~výberu medzi dvoma možnosťami. Pre mňa najznámejší text je z~knihy Jozue 24,15:  {\it „Ak sa vám nepáči slúžiť Hospodinovi, vyvoľte si dnes, komu chcete slúžiť: či bohom, ktorým slúžili vaši otcovia za Veľriekou, alebo bohom krajiny Amorejčanov, v~ktorej bývate.“} Toto považujem za vyjadrenie absolútnej slobody výberu. Jozua nikoho nenúti, aby slúžil Hospodinovi. On dáva na výber. Služba Hospodinovi alebo amorejským modlám. V~ďalších veršoch však vysvetľuje následky nesprávneho výberu. Ukazuje na život v~Božej prítomnosti v~kontraste so životom bez Boha.

Téma voľby je blízka aj v~obeti Pána Ježiša. V~Getsemanskej záhrade to vyjadril až dva razy. Prvýkrát, keď sa modlí k~Otcovi, aby vzal od Neho „kalich utrpenia“ a jedným dychom povie: {\it „… nie moja, ale Tvoja vôľa nech sa stane...“}. Druhýkrát, keď povie Petrovi: {\it „… myslíš si, že nemôžem požiadať svojho Otca a že On by mi hneď neposlal viac ako dvanásť légií anjelov?“} (Mt 26,53). Pán Ježiš sa rozhodoval medzi svojím vlastným dobrom a naším dobrom. Jeho voľba vlastného utrpenia, odsúdenia, smrti a Otcovho opustenia či prekliatia nám priniesla slobodu od hriechu a zatratenia vo večnosti bez Boha. On sa dobrovoľne rozhodol vziať na seba zlo pre naše dobro.

29. februára 2020 budeme robiť rozhodnutie, ktoré vážne ovplyvní naše životy na najbližšie roky. Prajem nám všetkým, aby toto rozhodnutie prinieslo {\it„… nerušený  a pokojný život vo všetkej zbožnosti a dôstojnosti“} (1Tim 2,1-2). Nutnou podmienkou toho sú však naše {\it „… prosby, modlitby, príhovorné modlitby, vďaky za všetkých ľudí, za kráľov a všetkých vysokopostavených ...“}.

Naproti tomu, každý deň znovu a znovu robíme rozhodnutia, ktoré majú vplyv na našu večnosť. Vidíme okolo seba ľudí, ktorí sa pre pár „šťastných rokov“ na zemi rozhodujú tak, že strácajú večnosť. Nerozumiem tomu, ale myslím si, že vymieňajú úžasnú a nepredstaviteľnú večnosť s~Bohom za úbohých 20, 30 či 50 rokov života podľa vlastných predstáv o~šťastí.

Modlím sa za to, aby sme každý deň znova a znova robili správne rozhodnutia, pri ktorých si budeme voliť večnosť v~prítomnosti Baránka a odmietali {\it „… dočasné potešenie z~hriechu ...“} (Žid 11,25).

\autor{Peter Pribula st.}


\clanok {Správy zo staršovstva}
S radosťou vám v~týchto prvých dňoch roku 2020 slúžime! Ďakujem vám všetkým za vašu dôveru. Máme veľkú radosť z~toho, že tento mesiac prijmeme jedenásť vzácnych bratov a sestier za členov zboru. V~rámci tohto procesu sme si na staršovstve vypočuli ich príbehy a svedectvá. Sme z~toho veľmi povzbudení. Niektorí z~nich prišli do nášho zboru len nedávno a porozprávali nám o~tom, ako sa na Palisádach tešili z~toho, že našli prijatie, spoločenstvo a dôraz na Božie Slovo. Prajem si, aby čoraz viac hľadajúcich našlo tú pravú Božiu lásku v~najmilšom zbore v~Bratislave.

Takisto sa tešíme z~toho, že sme do staršovstva prijali Daniela Pletta. Svedčí to o~vašej dôvere v~neho a o jeho vernosti. Prináša so sebou veľa rokov obetavej služby a lásky voči nám všetkým.

S vďačnosťou sledujeme vývoj oboch novozakladajúcich sa zborov, ktoré vychádzajú z~nášho zboru. Pozorujeme, že veľa Slovákov v~Connecte ako aj Ukrajincov Pán Ježiš zachraňuje prostredníctvom tejto služby. Sme vďační Bohu za vašu podporu a modlitby za nich.

Boli sme povzbudení vašou obetavosťou, ktorú ste prejavili v~zbierke na podporu misie na Ukrajine, služby žien a charitu v~Sýrii. Pán Boh sa o~nás hojne stará a my ako zbor z~toho štedro dávame ďalej. Je to krásny dôkaz zdravého zboru. Ako staršovstvo sme odporučili tieto zbierky, pretože vieme, že výsledok bude na Božiu slávu.

Služba spevokolu počas sviatkov bola veľkým povzbudením pre mnohých v~Bratislave či v~Bernolákove. Ďakujeme každému za obetavé hodiny cvičenia a spevu. Veríme, že aj vy sami ste prijali povzbudenie.

Prednedávnom sme starší poznamenali, že je v~zbore veľa ľudí, ktorí verne a nenápadne slúžia, a bez každého z~nás by zbor vôbec nefungoval. Ešte stále máme pred sebou veľa práce a ak sa Pán Ježiš ešte budúcu stredu nevráti, v~Božej milosti sa do nej pustíme. V~našom okolí je veľa ľudí, ktorí nepoznajú Ježiša; susedia, ktorí hľadajú lásku a prijatie, školy, ktoré potrebujú pomoc, či hladní bezdomovci, ktorí čakajú na polievku. V~Božom kráľovstve máme pred sebou nové príležitosti, do ktorých sa potrebujeme zapojiť. Na všetko z~toho sa tešíme.

S vďačnosťou voči Bohu za vás všetkých,

\autor {za staršovstvo zboru Danny Jones}


\clanok {Pozvania pre Dannyho a Claru Jonesovcov}
Br. kazateľ Danny Jones s~manželkou Clarou ešte stále radi prijmú pozvania od členov a priateľov zboru. Ak by ste mali záujem ich niekedy pozvať, prosím vás, aby ste si s~nimi dohodli termín, najlepšie priamo s~Clarou: \email {clara.m.jones@gmail.com}, príp. 0948~288~879.


\clanok {Stretnutia sestier}
Milé sestry,

srdečne vás pozývam na ďalšie stretnutia sestier, ktoré v~najbližšom období budú 12. a 26.~februára o~17.30~hod. v~modlitebni na Palisádach. Začali sme sa venovať novému štúdiu s~názvom {\it Trvalá sloboda} a budeme sa najbližší polrok stretávať každý druhý týždeň, aby sme spolu mohli otvárať Božie slovo, zdieľať sa navzájom o~tom, čo sa budeme učiť a utužovať vzťahy.

Veľmi sa teším na stretnutie s~vami!

S láskou pre vás všetkých,

\autor {Clara Jones}
\vfill\break


\clanok {Jarné prázdniny v~Račkovej doline}
Znova sú tu jarné prázdniny! Tento čas môžeme spolu stráviť v~chate Jána Amosa Komenského v~Račkovej doline v~termíne {\bf od~15.~2.~2020 do~21.~2.~2020}.

Príchod 15.~2. (sobota) ľubovoľne, odchod 21.~2. (piatok) v~dopoludňajších hodinách.

Čaká nás tam prekrásne prostredie s~krásnymi izbami a výhľadom, výborné jedlo, spoločné hry, čas strávený s~priateľmi pri rozhovoroch, vzájomné zdieľanie sa, ale hlavne lyžovanie, bežkovanie, sánkovanie a šantenie v~snehu. A~kopec detí.

Máme tiež prisľúbené, že na nejaký čas príde aj náš brat kazateľ Danny Jones s~manželkou Clarou.

Pokiaľ máte chuť takto stráviť čas, prosím prihláste sa Barbore Antalíkovej na tel.~č. 0903~255~010 {\bf do 9.~2.~2020}.

Myslite na to, prosím, že počet miest je obmedzený.

Deň príchodu aj odchodu si môžete upraviť podľa vašej potreby.

Lyžovať pri chate nie je možné, za lyžovaním sa dá ísť napr. na Podbanské (veľmi nenáročné), do Pavčinej Lehoty, na Opalisko či Čertovicu a~pod. V~okolí je aj možnosť bežkovania.
\bigskip
\noindent{\bf Cenník ubytovania:}
\bigskip
\hfil\table{lcc}{
 & Dospelý & Dieťa od 4 do 14 rokov \crl
Noc & 15~€ & 10~€ \cr
}
\bigskip
Všetky izby sú už vybavené sociálnymi zariadeniami.
\bigskip
\noindent{\bf Cenník stravovania:}
\bigskip
\hfil\table{lcc}{
 & Dospelý & Deti v~prípade detskej porcie \crl
Raňajky (švédske stoly) & 4,50~€ & 2,50~€ \cr
Obed & 6~€ & 4~€ \cr
Večera & 3,50~€ & 2~€ \cr
}
\bigskip
Treba tiež nahlásiť, ak máte nejakú diétu alebo intoleranciu.

Pokiaľ by ste potrebovali ďalšie informácie, určite mi zavolajte.

\autor {Barbora Antalíková}


\clanok {Senior klub}
Senior klub v~mesiaci február plánujeme uskutočniť posledný štvrtok v~mesiaci, t.~j. dňa 27.~2.~2020 od~10.00 do~14.00~hod. na Súľovskej ul. 2.

Prosíme seniorov, aby si nezabudli zobrať so sebou staré červené spevníky.

Téma: Význam mien Hospodina a Pána Ježiša v~Starej a v~Novej zmluve.

\autor {Jana Makovíniová}


\clanok {Mládežnícka konferencia BJB}
Mládežnícka konferencia BJB sa bude konať v~termíne 21.~--~23.~februára v~Banskej Bystrici. Viac informácií nájdete na webovej stránke jednoty: \ulink [http://mk.baptist.sk/]{mk.baptist.sk}.

Takisto je možné prihlásiť sa ako dobrovoľník.


\clanok {Víkend pre manželské páry}
Cheme vám dať do pozornosti nový termín plánovaného víkendu pre manželské páry. Nový termín je 20.~--~21.~marca. Počas týchto dvoch dní nás budú sprevádzať br. kazateľ Danny Jones s~manželkou Clarou. Začiatok bude v~piatok o~18.00~hod. v~modlitebni na Palisádach. Po krátkom príhovore pôjdu páry do mesta na rande. Zbor zabezpečí program pre deti na Zrínskeho pre tých, ktorí to budú potrebovať.

Prihlášku na túto akciu môžete nájsť tu: \ulink[https://forms.gle/V2eFzSJCXbU5R12W9]{forms.gle/V2eFzSJCXbU5R12W9}.


\clanok {Služba núdznym}
Milé sestry a bratia,

aj v~tomto roku, ak dá Pán, budeme mať príležitosť poslúžiť varením polievok pre núdznych. Rada by som Vám dala do pozornosti {\bf termíny na rok 2020}.

Ešte potrebujeme obsadiť nasledujúce termíny:

{\bf 21.~marec (sobota); 21.~apríl (utorok); 19.~máj (utorok)}.

Na rezervované termíny sa, prosím, nahláste, u~mňa.

Objem je {\bf 30 litrov} hustej výživnej polievky buď s~mäsom, s~klobásou alebo s~párkami. Prosím Vás o~dodržanie času. Výdajový tím vo vestách {\it Pomoc ľuďom v~núdzi} príde po {\bf horúcu polievku o~19.00 hod. v utorky a štvrtky a o~16.30 hod. v soboty} na vami určené miesto odberu.

Variť môžeme v~kuchynke na Zrínskeho. Máme k~dispozícii 30-litrový hrniec, taktiež aj recepty na toto množstvo a aj finančné prostriedky, za ktoré štedrým darcom ďakujeme.

Naďalej {\bf hľadáme dobrovoľníkov do výdajových tímov} -- jeden tím vydáva teplý čaj a kávu a druhý tím vydáva polievku s~pečivom. Modlíme sa, aby Pán pridal pomocníkov do tejto služby.

Buďme takýmto spôsobom požehnaním pre druhých, ktorí túto pomoc potrebujú a ako spoločenstvo preukážme ľuďom v~núdzi lásku Pána Ježiša týmto praktickým spôsobom.

Ďakujem za spoluúčasť na tejto pomoci!

\autor {Beáta Bogárová}
\vfill\break


\clanok {Víkend pre mužov}
Chceli by sme vám pripomenúť, že víkend pre mužov v~našom zbore plánujeme 24.~--~26.~apríla~2020 na Chvojnici.


\clanok {Stredisko Evanjelickej diakonie Bratislava -- ponuka práce}
Stredisko Evanjelickej DIAKONIE Bratislava hľadá opatrovateľky do Zariadenia pre seniorov (Domov dôchodcov) v~centre Bratislavy na TPP s~nástupom okamžite. Základnou kvalifikáciou pre vykonávanie tejto práce je ochotné srdce slúžiť ľuďom vo vysokom veku ako aj certifikát z~opatrovateľského kurzu. V~prípade, že máte záujem o~túto prácu a certifikát ešte nemáte, vieme sprostredkovať absolvovanie opatrovateľského kurzu počas trvania pracovného pomeru. Ide o~prácu na pravidelné 12 h zmeny (3~služby a 3~dni voľna alebo 4~služby a 4~dni voľna). Chceme povzbudiť prípadné záujemkyne, ktoré nemajú skúsenosť s~touto prácou, ale o~nej rozmýšľajú, že ponúkame možnosť si u~nás prácu opatrovateľky v~praxi vyskúšať.  Ponúkame veľmi atraktívne pracovné prostredie, dobré mzdové ohodnotenie (mzdové podmienky sú stanovené podľa Zákona č. 553/2003 Z. z. o~odmeňovaní zamestnancov pri výkone práce vo verejnom záujme + osobitné ohodnotenie + príplatky za nadčasy, víkendy a pod.), výborný kolektív, príjemnú pracovnú atmosféru v~zariadení rodinného typu. Ponúkame možnosť prespať v~zariadení medzi službami. Pre viac informácií volajte na tel. číslo: 0907~295~028.

\autor {Silvia Fáberová}
\vfill\break


\clanok{Verš na zapamätanie}
Na mesiac február máme nový veršík, ktorý sa chceme spoločne učiť. Veríme, že poznanie Písma prospeje našej duši i našej mysli:

{\it „Koniec všetkého sa priblížil. Buďte teraz rozvážni a triezvi, aby ste boli pohotoví modliť sa. Nadovšetko majte vytrvalú lásku jedni k~druhým, lebo láska prikrýva množstvo hriechov.“}

\autor{1Pt~4,~7~--~8}


\clanok{Účelové zbierky za uplynulé obdobie}
Milí bratia a sestry, ďakujeme za vašu obetavosť. V~uplynulom období ste prispeli:

\vskip-1ex\begitems
* Misia: 484,80~€
* Investície: 466,50~€
\enditems


\n 3.	2.	Vlasta	BALÁŽOVÁ;
\n 3.	2.	Margita	KRÁĽOVÁ;
\n 3.	2.	Miroslav	ANTALÍK;
\n 5.	2.	Štefánia	ANTALÍKOVÁ;
\n 5.	2.	Barbora	ANTALÍKOVÁ;
\n 11.	2.	Juraj	BALÁŽ;
\n 11.	2.	Oľga	KOVÁČOVÁ;
\n 11.	2.	Beáta	BOGÁROVÁ;
\n 12.	2.	Martin	PRIBULA;
\n 13.	2.	Zlatica	VYSKOČILOVÁ;
\n 15.	2.	Ingrid	JANČULOVÁ;
\n 16.	2.	Lenka	PRIBULOVÁ;
\n 23.	2.	Anna	PLETT;
\narodeniny


\program{
\p 1  ; so ; 18.00 ; Mládež (Súľovská 2);.;;
\p 2  ; ne ;  9.30 ; Bohoslužby (D. Jones) ; 10.00 ; Chvojnica (P. Škulec);
\p    ;    ; 16.00 ; Zborové členské zhromaždenie;.;;
\p 3  ; po ; 17.00 ; Modlitby -- ženy (Zrínskeho 2);.;;
\p 4  ; ut ; 15.15 ; Stretnutie pri Biblii (P. Pivka, Zrínskeho 2);.;;
\p 5  ; st ;  6.00 ; Modlitby -- muži (kostol Palisády);.;;
\p 6  ; št ; 19.00 ; Biblická hodina (J. Szőllős, Zrínskeho 2);.;;
\p 7  ; pi ;.;;.;;
\p 8  ; so ; 18.00 ; Mládež (Súľovská 2);.;;
\p 9  ; ne ;  9.30 ; Bohoslužby (D. Jones); 10.00 ; Chvojnica (J. Szőllős);
\p 10 ; po ; 17.00 ; Modlitby -- ženy (Zrínskeho 2);.;;
\p 11 ; ut ; 15.15 ; Stretnutie pri Biblii (P. Pivka, Zrínskeho 2);.;;
\p 12 ; st ;  6.00 ; Modlitby -- muži (kostol Palisády); 17.30 ; Stretnutie sestier (kostol Palisády);
\p 13 ; št ; 19.00 ; Biblická hodina (J. Szőllős, Zrínskeho 2);.;;
\p 14 ; pi ;.;;.;;
\p 15 ; so ;.;;.;;
\p 16 ; ne ;  9.30 ; Bohoslužby (Ľ. Dzuriak); 10.00 ; Chvojnica (J. Štefko);
\p 17 ; po ; 17.00 ; Modlitby -- ženy (Zrínskeho 2);.;;
\p 18 ; ut ; 15.15 ; Stretnutie pri Biblii (P. Pivka, Zrínskeho 2);.;;
\p 19 ; st ;  6.00 ; Modlitby -- muži (kostol Palisády);.;;
\p 20 ; št ; 19.00 ; Biblická hodina (J. Szőllős, Zrínskeho 2);.;;
\p 21 ; pi ;.;;.;;
\p 22 ; so ;.; Mládežnícka konferencia (Banská Bystrica);.;;
\p 23 ; ne ;  9.30 ; Bohoslužby (T. Valchář) ; 10.00 ; Chvojnica (P. Kohút);
\p 24 ; po ; 17.00 ; Modlitby -- ženy (Zrínskeho 2);.;;
\p 25 ; ut ; 15.15 ; Stretnutie pri Biblii (P. Pivka, Zrínskeho 2);.;;
\p 26 ; st ;  6.00 ; Modlitby -- muži (kostol Palisády); 17.30 ; Stretnutie sestier (kostol Palisády);
\p 27 ; št ; 10.00 ; Senior klub (Súľovská 2) ;19.00 ; Biblická hodina (J. Szőllős, Zrínskeho 2);
\p 28 ; pi ;.;;.;;
\p 29 ; so ; 18.00 ; Mládež (Súľovská 2);.;;
}

\tiraz
\bye