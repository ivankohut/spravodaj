%\typosize[10/12.5]% - pouzita velkost pisma/riadku - trochu vacsie
\input makra.tex % nacitanie Ivanom pripravenych nastaveni a prikazov
\hyphenation{star-šov-stvo} % rozdelenie slov na konci riadku, treba tu uviest slova, ktore sam nepozna

\spravodaj{9}{2020}


\clanok {Cirkev ako spoločenstvo}
Definícia zboru môže byť pre každého trochu iná, lebo časom sa naše chápanie cirkvi vyvíja. V~dnešnej dobe vidíme zbor viac v~spojitosti s~budovou. Ale v~dobe, keď cirkev vznikala, to tak nebolo. Vtedy sa zhromažďovali viac po domácnostiach a zbor bol viac zameraný na vzťahy, spoločenstvo, stolovanie a rodinu.

Počas zborového tábora som nad tým uvažoval, keď som pozoroval, že sme tam celý týždeň ako veľká rodina. Tešil som sa z~toho a bolo mi ľúto, že sa ho nemohlo zúčastniť viac ľudí. Viem si predstaviť, že by sme na tábore zažili cirkev ako cirkevný zbor v~Efeze či Antiochii. Musíme organizovať viac podobných zborových podujatí.

Veľmi sme obmedzení priestormi, kde by sme sa mohli všetci neformálne schádzať, stolovať a zažívať spoločenstvo. Ešte stále sa modlím za priestory na Palisádach a verím, že Boh odpovie. Ale do veľkej miery to súvisí s~naším chápaním zboru a dôležitosti spoločenstva. Potrebujeme byť viac spolu. Je možné, že znova budú opatrenia, ktoré nám obmedzia schádzanie na zhromaždení. Čo ak sa časom znova nebudeme môcť stretávať? Čo ak budeme nútení sa „stretávať“ len cez internet?

Na jar sme boli príliš izolovaní a zanechalo to na nás stopu. Hoci neviem, čo nás čaká na jeseň, musíme sa pripraviť trochu inak. Verím, že prípadné obmedzenia sa nebudú týkať stretnutí po domácnostiach. Naše zborové skupinky budú ešte dôležitejšie. Každý si musí nájsť svoje miesto. Zhromaždenia v~obývačke nemusia byť osamote, ale aj s~inými jednotlivcami či rodinami. Môžeme pritom spolu aj stolovať a zažiť hlbšie spoločenstvo. Keby k~tomu došlo, nebolo by to prvýkrát v~dejinách cirkvi. Pán Boh tieto dni koná s~istým zámerom a prináša zmenu. Buďme na to pripravení a tešme sa z~toho napriek zmenám. Nechcem, aby sme jeseň len nejako vydržali a prežili. Chcem, aby sme z~toho, čo nás čaká, mali úžitok a prospievali. Boh má pre nás prichystané len to dobré. Poďme do toho plní viery a očakávania!

\autor{Danny Jones}


\clanok {Zborové skupinky}
Si v~skupinke? Chcel by si sa do nejakej zapojiť?

Boli sme stvorení pre spoločenstvo a život mimo komunity je nebezpečný. Všetci potrebujeme miesto bezpečia a prijatia, kde môžeme spracovať svoj život a napredovať.

V našom zbore existuje niekoľko skupiniek a boli by sme radi, keby si čo najviac ľudí v~zbore našlo svoje miesto v~skupinke. Niektoré sú zamerané na biblické štúdium, štúdium nejakej knihy, diskusné skupinky k~nedeľnej kázni, či skupinky pre manželské páry. Ak by si mal záujem byť súčasťou nejakej skupinky alebo máš nejaké otázky, daj nám vedieť. Napíš e-mail na \email{zbor@bjbpalisady.sk}.


\clanok{Spoločné modlitby}
\vskip-1ex\begitems
* Muži -- streda {\bf od 6.00~hod. do 7.00~hod.}, kostol na Palisádach
* Ženy -- pondelok {\bf od 17.00~hod.}, Zrínskeho 2
\enditems


\clanok {Nedeľná besiedka}
Milí rodičia,

stretnutia nedeľnej besiedky budú v~tomto školskom roku nateraz začínať o~9.30~hod. na Zrínskeho. Vzhľadom na bezpečnostné opatrenia budú prebiehať nasledovne:

Deti nám odovzdáte pri vchodových dverách. Deti z~veľkej besiedky vo veku od~7 do~10 rokov budú mať po celý čas rúška. Deti z~malej besiedky vo veku od~3 do~7~rokov (prváci) rúška mať nemusia. Učitelia budú mať takisto rúška. Nakoľko berieme ohľad jedni na druhých, prosíme, aby ste priviedli len zdravé deti bez príznakov prechladnutia. Deti zvládnu besiedku aj bez vás, prosím, nezostávajte s~nimi na besiedke. Po skončení besiedky deti privedieme do zhromaždenia.

Veríme, že napriek nevyspytateľnosti situácie sa budeme môcť v~pokoji stretávať. Tešíme sa na deti!

\autor {za tím vedúcich Mirka Kešjarová}


\clanok {Dorast}
Stretnutia dorastu v~našom zbore budú v~novom školskom roku v~novom termíne každý piatok o~17.30~hod. na Zrínskeho. Dorasty budú končiť cca o~19.00 -- 19.30~hod. Ak by ste sa báli o~svoje deti kvôli ceste domov, dajte mi, prosím, osobne vedieť. Niekto z~vedúcich ich bude môcť priviezť domov.

\autor {za tím vedúcich Michal Vrábel}
\vfill\break


\clanok {Klubík}
Od nového školského roka zase plánujeme mať Klubík, a to každý utorok o~9.30~hod. na Zrínskeho. Prvé stretnutie bude už 8.~septembra.

Všetky mamičky na materskej so svojimi deťmi sú vítané!

\autor {Clara Jones}


\clanok {Stretnutia sestier}
Milé sestry,

srdečne vás pozývam na naše spoločné stretnutia, na ktorých by sme túto jeseň chceli dokončiť naše štúdium {\it Trvalej slobody}, ktorému sme sa začali venovať pred pandémiou. Prvé sesterské stretnutie plánujeme v~stredu 9. septembra o~17.30~hod. na Palisádach, kde budeme mať seminár a diskusiu o~3.~týždni lekcií (je teda potrebné dovtedy dokončiť domáce úlohy z~tretieho týždňa štúdia).


Ďalšie stretnutia sú potom plánované v~nasledujúcich termínoch:

\vskip-1ex\begitems
* 23. 9. 2020
* 7. 10. 2020
* 21. 10. 2020
* 4. 11. 2020
* 18. 11. 2020
\enditems

\autor {Clara Jones}


\clanok {Víkendovka pre sestry}
Drahé sestry,

srdečne vás pozývam na víkendovku, ktorú máme naplánovanú v~Častej 2.~--~4.~októbra~2020. (Začneme večerou v~piatok a skončíme obedom v~nedeľu.)

Toto víkendové stretnutie je pre všetky sestry v~zbore spolu s~ich dcérami, nevestami, mamami a svokrami, aj ak nie sú z~nášho zboru.

Téma bude {\it Rodinné vzťahy} a plánujeme sa venovať týmto konkrétnym témam:

\vskip-1ex\begitems
* Vzťah medzi matkou a dcérou
* Ako mať vzťah s~dieťaťom, keď je „iné“ (o~vzťahu s~deťmi, ktoré stratili správne smerovanie v~živote)
* Vzťah medzi svokrou a nevestou

\enditems

Tešíme sa, že Danka Paštrnáková z~Liptovského Hrádku bude našou rečníčkou. Danka je matkou 5 detí a babičkou 8 vnúčat. Venuje sa ženám na ženských skupinkách a spolu s~manželom aj manželským párom. Čoskoro vám pošleme link, cez ktorý sa budete môcť prihlásiť.

{\it „Pripomínam si tvoju úprimnú vieru, akú mala už tvoja stará mama Lóis i tvoja matka Euniké a ktorú máš, ako som presvedčený, aj ty“} (2Tim 1,5).

S radosťou a očakávaním,

\autor {Clara Jones}


\clanok {Varenie polievky pre bezdomovcov}
Máme voľné termíny na varenie polievky pre bezdomovcov, a to v~utorok 15.~septembra, v~sobotu 17.~októbra, v~sobotu 21.~novembra a v~sobotu 19.~decembra.

Tí z~vás, ktorí by boli ochotní sa zapojiť do tejto služby, kontaktujte, prosím, Beatu Bogárovu na tel. č. 0908~046~409.


\clanok {Zborový pracovný projekt}
Po minuloročnej výbornej skúsenosti sa chceme aj tento rok zapojiť do zborového pracovného projektu a poslúžiť ľuďom v~okolí nášho zboru. Túto príležitosť budeme mať v~sobotu 26.~septembra v~čase od~8.00 do~14.00~hod. na Strednej priemyselnej škole elektrotechnickej na Zochovej ul.


\clanok {Motivačný víkend pre manželov}
Radi by sme vám dali do pozornosti {\it Motivačný víkend pre manželov 1}, ktorý sa bude konať 8.~--~11.~októbra v~Račkovej doline (štvrtok 19.00~hod. -- nedeľa 13.00~hod.).

Program:

\vskip-1ex\begitems
* Intimita -- dôverní priatelia
* Vieš, čo potrebujem?
* Keď pohár pretečie
* Aby manželstvo nebolelo
\enditems

Cena kurzu je 105,--~€ na osobu. V~cene je zahrnuté 3x ubytovanie s~plnou penziou, občestvenie, materiály a pomôcky na víkend.

Viac informácií:
\vskip-1ex\begitems
* telefón: 0905~622~900
* web: \ulink[https://www.nasemanzelstvo.sk]{nasemanzelstvo.sk}
* e-mail: \email{info@nasemanzelstvo.sk}
\enditems
\vfill\break


\clanok{Verš na zapamätanie}
Na mesiac september máme nový veršík, ktorý sa chceme spoločne učiť. Veríme, že poznanie Písma prospeje našej duši i našej mysli:

{\it „Chodník spravodlivých je ako svetlo brieždenia, ktorého jas prechádza do plného dňa. Chodník bezbožníkov je sťa temnota, nevedia, na čom sa potknú.“}

\autor{Príslovia 4, 18 -- 19}


\clanok{Zbierky za uplynulé obdobie}
Milí bratia a sestry,

počas leta ste prispeli:

\vskip-1ex\begitems
* Misia: 691,50~€
* Investície: 437,20~€

\enditems

Ďakujeme vám, že napriek okolnostiam a neistým ekonomickým vyhliadkam do budúcnosti, ste mnohí prispeli na činnosť a službu zboru. Aj naďalej máte možnosť prispieť do „nedeľnej zbierky“, a to prevodom na účet zboru. Do poznámky pre prijímateľa, prosím, uveďte „zbierka“.

Bankové spojenie: SK36 0900 0000 0000 1147 1836,  SWIFT: GIBASKBX

Ďakujeme!


\n 5.	9.	Dušan	UHRIN;
\n 9.	9.	Daniel	VALENTA;
\n 14.	9.	Štefan	SYNOVEC;
\n 16.	9.	Daniel	PLETT;
\n 19.	9.	Richard	HALAMÍČEK;
\n 21.	9.	Kvetoslava	MAĎAROVÁ;
\n 21.	9.	Miroslava	SIMONOVÁ;
\n 22.	9.	Viera	KOLÁROVSKÁ;
\narodeniny


\program{
\p  1 ; ut ;.;;.;;
\p  2 ; st ;.;;.;;
\p  3 ; št ;.;;.;;
\p  4 ; pi ; 17.30 ; Dorast (Zrínskeho 2);.;;
\p  5 ; so ;.;;.;;
\p  6 ; ne ;  9.30 ; Bohoslužby (D. Jones); 10.00 ; Chvojnica (P. Škulec);
\p  7 ; po ; 17.00 ; Modlitby -- ženy (Zrínskeho 2);.;;
\p  8 ; ut ;  9.30 ; Klubík (Zrínskeho 2) ; 15.15 ; Stretnutie pri Biblii (P. Pivka, Zrínskeho 2);
\p  9 ; st ;  6.00 ; Modlitby -- muži (kostol Palisády); 17.30 ; Stretnutie sestier (kostol Palisády);
\p 10 ; št ;.;;.;;
\p 11 ; pi ; 17.30 ; Dorast (Zrínskeho 2);.;;
\p 12 ; so ;.;;.;;
\p 13 ; ne ;  9.30 ; Bohoslužby (J. Szőllős); 10.00 ; Chvojnica ;
\p 14 ; po ; 17.00 ; Modlitby -- ženy (Zrínskeho 2);.;;
\p 15 ; ut ;  9.30 ; Klubík (Zrínskeho 2) ; 15.15 ; Stretnutie pri Biblii (P. Pivka, Zrínskeho 2);
\p 16 ; st ;  6.00 ; Modlitby -- muži (kostol Palisády);.;;
\p 17 ; št ;.;;.;;
\p 18 ; pi ; 17.30 ; Dorast (Zrínskeho 2);.;;
\p 19 ; so ;.;;.;;
\p 20 ; ne ;  9.30 ; Bohoslužby (D. Jones) ; 10.00 ; Chvojnica ;
\p 21 ; po ; 17.00 ; Modlitby -- ženy (Zrínskeho 2);.;;
\p 22 ; ut ;  9.30 ; Klubík (Zrínskeho 2) ; 15.15 ; Stretnutie pri Biblii (P. Pivka, Zrínskeho 2);
\p 23 ; st ;  6.00 ; Modlitby -- muži (kostol Palisády); 17.30 ; Stretnutie sestier (kostol Palisády);
\p 24 ; št ;.;;.;;
\p 25 ; pi ; 17.30 ; Dorast (Zrínskeho 2);.;;
\p 26 ; so ;  8.00 ; Zborový pracovný projekt ;.;;
\p 27 ; ne ;  9.30 ; Bohoslužby (P. Kolárovský) ; 10.00 ; Chvojnica ;
\p 28 ; po ; 17.00 ; Modlitby -- ženy (Zrínskeho 2);.;;
\p 29 ; ut ;  9.30 ; Klubík (Zrínskeho 2) ; 15.15 ; Stretnutie pri Biblii (P. Pivka, Zrínskeho 2);
\p 30 ; st ;  6.00 ; Modlitby -- muži (kostol Palisády);.;;
}

\tiraz
\bye
