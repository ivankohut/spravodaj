%\typosize[10/12.5]% - pouzita velkost pisma/riadku - trochu vacsie
\input makra.tex % nacitanie Ivanom pripravenych nastaveni a prikazov
\hyphenation{star-šov-stvo} % rozdelenie slov na konci riadku, treba tu uviest slova, ktore sam nepozna

\spravodaj{11}{2020}

\clanok {Kairos}
Hodnota času sa mení podľa okolností. Ako tínedžer som pracoval za \$2,35 na hodinu. Nebolo to veľa, ale toľko som dokázal zarobiť. V~inom kontexte mala však hodina môjho času oveľa väčšiu hodnotu ako \$2,35. Hodina s~mládežou alebo futbalového zápolenia so svojimi bratmi mala väčšiu hodnotu ako to, čo som zarobil v~robote. Čas strávený s~otcom na rybačke mal nevyčísliteľnú hodnotu. Bolo to na nezaplatenie.

Sú veci, ktoré majú väčšiu hodnotu ako peniaze. Boh nám dal možnosť to zažiť počas koronakrízy. Definícia času sa zmenila a sme nútení ho využívať inak. Najviac sa to prejavuje v~našej rodine. Nikdy predtým sme netrávili toľko času s~rodinnými príslušníkmi ako teraz. Ale byť spolu len fyzicky nestačí. Môžeme byť v~tej istej miestnosti a nebyť prítomní, najmä vo svete technológií.

V liste Efezanom 5,16 máme príkaz od Boha: „Naplno využívajte čas, lebo dni sú zlé.“ Každá hodina je dar, ktorý sa už nevráti. Máme len jednu šancu ju využiť. To znamená, že je nesmierne hodnotná. Akú hodnotu by sme mohli prisúdiť času, ktorý strávim s~manželkou, manželom alebo deťmi? Ak by sme zarábali 30~000 € na hodinu, to by bol dobrý plat. Ale moja rodina má oveľa väčšiu hodnotu ako 30~000 € na hodinu. Žiadne peniaze na svete mi nevrátia čas, ktorý som premrhal s~rodinou. Musím zmeniť svoj pohľad a začať si čas strávený s~rodinou ceniť inak.

Grécky výraz pre „čas“ v~tomto verši je {\it kairos}. To slovo nevyjadruje normálne minúty na hodinách. Tým slovom je {\it chronos}. Kairos odkazuje na zvláštny dar okamihu v~živote. Covid je takýmto okamihom pre naše rodiny. Sme požehnaní niečím, čo normálne nemáme. A~tak Pavol píše: Naplno využívajte Covid. Nemôžem si dovoliť prísť na koniec dňa a priať si, aby som ho prežil inak. Hodiny, ktoré teraz máme – {\it kairos} týchto dní – je na nezaplatenie. Aby sme ho mohli správne využiť, musíme konať so zámerom; plánovať, čo robiť a čo nerobiť.

{\it Chronos} je teraz vďaka Covidu veľmi komplikovaný, lebo mnohí pracujeme z~domu a popritom sa snažíme usmerniť deti so školou. Uprostred stresu v~tomto novom {\it chronose} nám ľahko unikne ten {\it kairos}, ktorý nám bol daný. Pomôžu nám dve veci: Po prvé, uznať tieto dni za kairos; za zvláštne Božie požehnanie. Tvoj postoj výrazne ovplyvní, ako tento čas – {\it kairos} – využiješ. Buď Bohu vďačný za ten chaos. Ďakuj Mu za ten čas doma. Ďakuj Mu za dar rodiny a za dar Covidu. Tou druhou vecou je každý deň plánovať konkrétne spôsoby, ako naplno využívať {\it kairos}. Každý deň poteš svojho manžela či manželku nejakou maličkosťou; buď jej požehnaním. Boh ti dá kreatívne nápady, ak Ho o~to poprosíš. Oddeľ si čas na svoje deti, jednotlivo i spolu. Preruš to, čo robíš, aspoň na hodinu, sadni si na zem, a venuj im svoju pozornosť. Ak žiješ sám, zavolaj alebo napíš svojmu príbuznému. Dôležitou súčasťou tohto je duchovné vedenie. Na to, aby si poskytol duchovné vedenie, nemusíš byť teológ. Čítaj si Božie Slovo a spýtaj sa svojej rodiny: „O čom sa tu píše? Ako by sme mali žiť vo svetle tohto textu?“ Modlite sa spolu. Môže to byť divné, ale musí sa to jednoducho stať niečím prirodzeným. Opakovaním sa to zmení. Ak ti v~tom bráni nejaký hriech, kajaj sa z~neho, prijmi Božie odpustenie a využi tento kairos, ako len môžeš. Najdôležitejšie je poprosiť Ducha Svätého, aby ti dal milosť, ktorú potrebuješ pre túto príležitosť. Satan bojuje, aby ukradol {\it kairos}. Len keď dovolíme Bohu, aby nás naplnil a viedol, budeme vedieť tento čas využiť naplno. „Naplno využívajte čas, lebo dni sú zlé.“ Keď sa pre to rozhodneš, určite to nikdy neoľutuješ. Covid mení naše životy. Nech sa tieto zmeny stanú večným požehnaním pre našu rodinu a našich blízkych.

\autor{Danny Jones}


\clanok {Správy zo staršovstva}
Bratia, sestry, milí priatelia,

pamätáte si, ako vyzeral zborový život pred novembrom 1989? Starší určite áno, vy mladší ste nemali možnosť zažiť obdobie komunistickej éry. Bolo to obdobie, v~ktorom sme mohli robiť iba to, čo nám dovolili zákony a nariadenia, obmedzujúce zborový život.

Mnohí z~nás si uvedomovali, že obdobie slobody, ktoré sme mali od novembra 1989, je časovo obmedzené. Nevedeli sme termín ani spôsob, ako budeme obmedzení. Situácia, v~ktorej sa dnes nachádzame, nás neobmedzuje v~našich názoroch, ani nás nenúti popierať svoju vieru v~Pána Ježiša. To, v~čom sme obmedzení, je náš spoločný zborový -- rodinný život. Mnohí z~nás pociťujeme dopad obmedzení, ktoré zažívame. Cítime odlúčenosť od tých, s~ktorými sa máme radi, s~ktorými si rozumieme; s~tými, na ktorých nám záleží. Situácia nás núti hľadať nové spôsoby, ako spolu komunikovať, ale aj oprášiť tie staré, na ktoré máme moderné technológie.

To, čo vnímam ako hlavnú tému stretnutí staršovstva, je práve to, ako nestratiť vzájomné kontakty. Snažíme sa o~to, aby nik nezostal izolovaný, a tak isto, aby sme nezabúdali na jednotlivých členov našej zborovej rodiny.

Sme vďační za nedeľné online bohoslužby a aj za tých, ktorí stoja za ich prípravou, ale vieme, že to nestačí. Preto vás pozývame do aktivít, ktoré neodporujú súčasným nariadeniam a pritom nám pomôžu udržiavať vzájomné kontakty. Niektoré aktivity organizujeme ako zbor, iné nechávame na osobnú iniciatívu všetkých nás.

Izolácia má okrem odlúčenosti ešte jedno riziko. Tým je nebezpečenstvo, ktoré apoštol Peter opísal slovami „… váš protivník, diabol, obchádza ako revúci lev a hľadá, koho by zožral…“ (1Pt 5,8). Toto nebezpečenstvo izolácie má vplyv na našu večnosť. Potrebujeme sa navzájom aj preto, aby sme neboli stratení pre večnosť v~Božej prítomnosti.

Modlíme sa za to, aby Boží Duch pracoval v~srdci každého z~nás a aby nás „… zachoval od zlého“ (Jn 17,15).

Napriek zmenám ktoré žijeme je „Ježiš Kristus ten istý včera i dnes i naveky“ (Heb 13,8).

\autor {za staršovstvo Peter Pribula st.}


\clanok {Bohoslužby počas koronakrízy}
Vzhľadom na epidemiogickú situáciu a s~tým súvisiace bezpečnostné opatrenia nie je v~súčasnosti možné sa schádzať na nedeľných bohoslužbách. Každú nedeľu je zabezpečený online prenos o~10.30~hod., ktorý môžete sledovať tu: \ulink [https://bit.ly/3cgSMBG]{bit.ly/3cgSMBG}.

V prípade akýchkoľviek zmien vás budeme informovať.
\vfill\break


\clanok {Služba ľuďom v~núdzi -- hľadáme dobrovoľníkov}
Vnímame, že stále je okolo nás veľa ľudí, ktorí potrebujú pomoc iných, a my ako kresťania chceme na tieto ich potreby reagovať praktickou pomocou.

Preto pripravujeme v~Bratislave-Petržalke otvorenie výdajne potravín pre sociálne odkázaných ľudí. Predpokladaný termín otvorenia je január 2021, keď budú môcť núdzni ľudia získať u~nás potraviny pre domácu spotrebu. Potraviny budeme získavať z~darov jednotlivcov a organizácií a z~obchodov, ktoré nám poskytnú potraviny pred končiacou sa dobou trvanlivosti. Aby sme túto službu núdznym zvládli, chceli by sme Vás pozvať – ísť do toho spolu s~nami. Môžete si zvoliť čas, ktorý do tejto služby viete ponúknuť. Každá pomoc sa počíta!

Na prevádzku okrem dobrovoľníckej pomoci potrebujeme aj pomoc: finančnú na prevádzku výdajne a aj modlitebnú. Budeme radi, keď sa nám ozvete.

Do výdajne potravín hľadáme dobrovoľníkov do rôzneho typu služieb:
\vskip-1ex\begitems
* 10 ľudí na výdaj potravín vo výdajni (3 hodiny mesačne)
* 10 ľudí s~autom, ktorí by poobede/ podvečer mohli vyzdvihnúť potraviny v~supermarkete (1x mesačne)
* 100 ľudí, ktorí by pravidelne prispievali nad chod výdajne (5~€ mesačne)
* 20 ľudí na zbierku potravín v~Tesco Dúbravka a Tesco Podunajské Biskupice v~období 19.~11.~2020 -- 3.~12.~2020 (1x / 3 hodiny na doobedie alebo poobedie). Zbierka bude spropagovaná medzi nakupujúcimi, ktorí majú možnosť darovať do našej výdajne potraviny, ktoré nakúpia. (Dobrovoľník je pri nákupnom košíku, do ktorého nakupujúci darujú potraviny a v~prípade záujmu komunikuje s~ľuďmi.)
* 10 ľudí s~autom na odvoz potravín podvečer z~Tesco do výdajne v~termínoch 19.~11.~2020 -- 3.~12.~2020 (1x mesačne)
\enditems

Potrebujeme aj vaše ruky a ochotné srdce. Zapojíme každého. Kontaktujte sa prosím na \email{kancelaria@krestaniavmeste.sk}. Ďakujeme!

\autor {Kresťania v~meste}


\clanok {Ak potrebujete pomoc, napíšte nám!}
V našom zbore sme zriadili e-mailovú adresu \email{pomoc@bjbpalisady.sk}, na ktorú môžete napísať, ak ste sa dostali do zlej situácie alebo potrebujete nejakú pomoc. Takisto sa môžete ozvať, ak ste ochotní s~niečím pomôcť.
\vfill\break


\clanok{Verš na zapamätanie}
Tento mesiac máme nový veršík, ktorý sa chceme spoločne učiť. Veríme, že poznanie Písma prospeje našej duši i našej mysli:

{\it „Zachovajte si tieto slová v~srdci a v~mysli, priviažte si ich ako znamenie na ruky a budete ich mať ako čelenku medzi očami. Poúčajte o~nich svojich synov a hovorte im o~nich, či budeš sedieť vo svojom dome alebo pôjdeš po ceste, či budeš líhať alebo vstávať.“}

\autor{Deuteronomium~11,~18--19}


\clanok{Zbierky za uplynulé obdobie}
Milí bratia a sestry,

v októbri ste prispeli:

\vskip-1ex\begitems
* Misia: 137,00 €
* Investície: 265,00 €

\enditems

Ďakujeme vám, že napriek okolnostiam a neistým ekonomickým vyhliadkam do budúcnosti, ste mnohí prispeli na činnosť a službu zboru. Aj naďalej máte možnosť prispieť do „nedeľnej zbierky“, a to prevodom na účet zboru. Do poznámky pre prijímateľa, prosím, uveďte „zbierka“.

Bankové spojenie: SK36 0900 0000 0000 1147 1836, SWIFT: GIBASKBX

Ďakujeme!


\n 2.	11.	Tomáš	VALCHÁŘ;
\n 5.	11.	Katarína	VALENTOVÁ;
\n 6.	11.	Elena	PRIBULOVÁ;
\n 6.	11.	Eva	SYČOVÁ;
\n 9.	11.	Alžbeta	BETKOVÁ;
\n 9.	11.	Radovan	PAULEN;
\n 15.	11.	Bohumila	ŠALINGOVÁ;
\n 18.	11.	Jelka	NEVICKÁ;
\n 19.	11.	Dávid	PRIBULA;
\n 21.	11.	Ladislav	KAMOCSAI;
\n 22.	11.	Alena	SVOBODOVÁ;
\n 22.	11.	Peter	PRIBULA st.;
\n 23.  11. Danny	JONES;
\n 24.	11.	Magdaléna	ŠALLINGOVÁ;
\n 25.	11.	Petra	ŠALINGOVÁ;
\n 27.	11.	Judita	KOLAŘÍKOVÁ;
\n 29.	11.	Jaroslav	KRÁĽ;
\narodeniny


\program{
\p  1 ; ne ; 10.30 ; Bohoslužby (D. Jones, online);.;;
\p  8 ; ne ; 10.30 ; Bohoslužby (D. Jones, online);.;;
\p 15 ; ne ; 10.30 ; Bohoslužby (P. Kolárovský, online);.;;
\p 22 ; ne ; 10.30 ; Bohoslužby (J. Szőllős, online);.;;
\p 29 ; ne ; 10.30 ; Bohoslužby (D. Jones, online);.;;
}


\tiraz
\bye
