\def\velkostpisma{10}
\def\velkostriadku{12.5}
\input makra.tex % nacitanie Ivanom pripravenych nastaveni a prikazov
\hyphenation{star-šov-stvo} % rozdelenie slov na konci riadku, treba tu uviest slova, ktore sam nepozna

\spravodaj{2}{2023}


\clanok {Hľadajte najprv Božie kráľovstvo a Jeho spravodlivosť a všetko ostatné Vám bude potom pridané}

Keď hľadáme Božiu Vôľu pre svoje životy a život nášho spoločenstva, nechceme byť cirkvou, ktorá kopíruje štýl života v~tomto svete, ktorý má hojnosť, ale vo vnútri je vyprázdnený.

Ak sa budeme pripodobňovať tejto kultúre, ľahko sa môžeme stať cirkvou, ktorá si bude hovoriť: „som bohatý, zbohatol som, nič nepotrebujem“, a ktorej Ježiš odpovedá: „ale nevieš, že si biedny, úbohý, chudobný, slepý a nahý“. (Zj~3,17)

O~to viac potrebujeme vnímať našu závislosť na Bohu.
Do akej miery veríme tomuto slovu: „Hľadajte najprv Božie kráľovstvo a Jeho spravodlivosť a všetko ostatné Vám bude pridané…“? Božie kráľovstvo hľadajme v~Božej prítomnosti. Tu si uvedomme, že sa spájame v~modlitbách a pôste ako občania jedného kráľovstva, v~ktorom je Ježiš Kristus vyvýšený. Ako je napísané v~Žalme~23 -- Veď Boh je ten, ktorý nás privádza na zelené pastviny a dáva nám to, čo potrebujeme. Vedie nás po cestách spravodlivosti. A~prečo to robí? Pre česť svojho mena.

Ježiš hovorí: „Šimon, Šimon, hľa, satan si vás vyžiadal, aby vás preosial ako pšenicu. Ale ja som za teba prosil, aby tvoja viera neochabla. A~ty, až sa raz obrátiš, posilňuj svojich bratov.“ Lk~22,31-32. Kvôli Petrovej priznanej slabosti a jeho závislosti na Ježišovi sa Ježiš stavia do pozície obrancu jeho viery, aby chránil vieru Petra pred útokmi satana. A~zároveň Ježiš je pre Petra sprítomnením Božej vôle, aby bola Jeho viera zachovaná.

Aj my potrebujeme toto napojenie, aby sme mohli zažiť v~Božej blízkosti, že naše nedostatky prekrýva sám Boh. On je Ten, v~koho sile stojíme, pod koho vedením kráčame, lebo Ježiš je našou silou v~čase skúšok. On je Ten, koho milosťou sme prikrytí, aby sme žili na česť Jeho mena. Hľadajme najprv Jeho kráľovstvo, priznajme, že nie sme nezávislí…ale v~pokore prichádzajme k~Pastierovi, lebo On sa za nás prihovára u~Otca, prosí, aby naša viera neochabla -- On presne vie, čo potrebujeme.

Výhrou je žiť v~Božej prítomnosti z~každého ohľadu. Lebo v~Jeho prítomnosti sa veci dejú inak, ako keď chceme veci robiť po svojom.

Inšpirovaní Jeho prítomnosťou a závislí na Ňom, robme  všetko so zámerom, aby sme Ho oslávili. Oslavujme Ho tým, ako sprítomňujeme Krista v~našich životoch.

„Tomu však, ktorý pôsobením svojej moci v~nás a nad to všetko môže urobiť omnoho viac, ako my prosíme alebo rozumieme, tomu sláva v~cirkvi a v~Kristovi Ježišovi po všetky pokolenia na veky vekov. Amen.“ Ef~3,20-21.

\autor{Peter Šrankota}


\clanok {Správy zo staršovstva}
V januári 2023 sme mali dve stretnutia. Témy, ktorým sme sa venovali na našich stretnutiach, súvisia hlavne s~pravidelnými aktivitami na začiatku roku.

Začali sme s~prípravou výročného zborového členského zhromaždenia.

Správy za jednotlivé služby v~našom zbore sú prvou lastovičkou, ktorá hovorí o~tom, že skončil jeden rok a začína sa nový. Preto išla z~našej strany výzva k~vedúcim zborových zložiek, aby pripravili výročné správy za nimi vedené služby a odovzdali ich s. Katke Kerekréty. Prosíme vás, aby ste mysleli na túto úlohu, ktorú máme.

Druhou lastovičkou je príprava rozpočtu na nové obdobie. Táto informácia zaznela už aj v~oznamoch na spoločných bohoslužbách. Týmto chcem vyzvať všetkých, ktorí majú podnety do návrhu rozpočtu, aby svoje požiadavky predložili ekonómke s. Ľubke Kohútovej alebo členom staršovstva. Zároveň pozývam na stretnutie staršovstva tých, ktorí majú záujem diskutovať o~hospodárení zboru a plánovanom rozpočte. Tieto stretnutia sú naplánované na 7.~3. a 21.~3.~2023.

Ukrajinská misijná stanica je už nejaký čas súčasťou nášho zboru. Jednou z~ich služieb je aj misia medzi ukrajinskými študentmi na VŠ internátoch v~Mlynskej doline. Pre podporu tejto služby využívajú OZ K2, ktoré v~minulosti  založil náš zbor. O~tejto ich službe sme hovorili 17.~januára so~Zenovijom Hryhoraščukom a Dariou Loboda. Boli sme potešení a povzbudení ich zápalom a nadšením pre službu. Modlíme sa za to, aby ich Pán Ježiš požehnal v~službe na internátoch.

Rozsahom aj obsahom veľkou témou sú misijné projekty. Tie, ktoré podporujeme, aj tie, ktoré sme dostali s~ponukou na spoluprácu pre budúce obdobie. Koľko ich je a v~akom rozsahu ich aktuálne podporujeme sa môžete dozvedieť z~návrhu rozpočtu na rok 2023. Ale podpora misie nie je iba o~peniazoch. Ide hlavne o~modlitebnú podporu, osobný záujem o~to, čím žijú ľudia v~misii, a v~prípade povolania ide aj o~osobné zapojenie sa do misie. Toto je výzva pre vás, ktorí máte záujem o~osobné zapojenie sa do misie.

V~rámci príprav na zborové členské zhromaždenie rozprávame aj so záujemcami o~členstvo v~našom zbore. Niektorí už boli so staršovstvom a hovorili sme spolu, iných sme pozvali a tešíme sa na stretnutie a rozhovor s~nimi.

V minulom roku delegáti zborov na konferencii odsúhlasili nový model financovania zborov. Aj touto témou sme sa zaoberali a dlhšie obdobie budeme zaoberať. Prosíme o~vaše modlitby, aby sme vedeli nájsť múdre rozhodnutia v~rámci schválených pravidiel.

Na rok 2023 sme dostali slovo z~ev. Jána 15,16: „Nie vy ste si vyvolili mňa, ale ja som si vyvolil vás a ustanovil som vás, aby ste šli a prinášali ovocie, aby vaše ovocie zostávalo, a aby vám Otec dal všetko, o~čo ho budete prosiť v~mojom mene.“

\autor {za staršovstvo v~láske a so zasľúbením nášho Pána a Spasiteľa Peter Pribula st.}
\vfill\break


\clanok {Kairos}
Ako bolo už oznamované, tento rok máme možnosť zúčastniť sa misijného kurzu {\it Kairos}. Je to kurz, ktorý nám pomôže lepšie porozumieť tomu, čo v~tomto svete Boh robí a prečo to robí. Kairos nám pomôže žiť každodenný život s~jasným zámerom. Pomocou kurzu Kairos môžeme takisto vidieť, čo Boh koná v~národoch sveta a ako sa môžeme k~tomuto dielu pripojiť.

Kurz Kairos bude prebiehať počas dvoch víkendov, a to v~termínoch 3.~--~5. a 10.~--~12.~marca 2023. Registrovať sa môžete tu: \ulink [http://bit.ly/40rTzcz]{bit.ly/40rTzcz}


\clanok {Seminár pre pracovníkov s~deťmi a dorastom, 10.~--~12. ~februára. }
Detská misia pozýva na kurzy pre prácu s~deťmi a dorastom všetkých, ktorí majú záujem pracovať s~mladou generáciou. Kurzy sú vhodné nielen pre začínajúcich a pokročilých učiteľov, ale aj pre rodičov. Tešíme sa na Vás a veríme, že vzdelávanie v~oblasti duchovnej služby je dobre využitým časom pre každého veriaceho.

\ulink[https://www.detskamisia.sk/vzdelavanie.html]{www.detskamisia.sk/vzdelavanie.html}

\autor{za Detskú misiu Miriam Kešjarová}


\clanok {Národný týždeň manželstva}
Pre týždeň venovaného manželstvám, máme pripravené stretnutie na 16.~2.~2023 s~pozvanými hosťami. Ďalšie detaily budú zaslané v~pozvánke e-mailom.


\clanok {Plánované letné tábory}
Zborový tábor plánujeme v~termíne 20.~--~26.~augusta~2023 v~stredisku Detskej misie v~Častej-Papierničke.

Dorastenecký tábor je predbežne plánovaný v~termíne 29.~7.~--~5.~8.~2023 na Chvojnici.
\vfill\break


\clanok{Verš na zapamätanie}
Tento mesiac máme nový veršík, ktorý sa chceme spoločne učiť. Veríme, že poznanie Písma prospeje našej duši i našej mysli:

{\it „Teraz však -- znie výrok Hospodina -- vráťte sa ku mne celým srdcom a pôstom, plačom a nárekom! Roztrhite si srdcia a nie rúcha, vráťte sa k~
Hospodinovi, svojmu Bohu!“}

\autor{Joel 2,12-13}


\clanok{Zbierky za uplynulé obdobie}
Milí bratia a sestry,

v~januári ste prispeli do zbierok celkom 3035,00 €
z~toho je:
\vskip-1ex\begitems
* Misia: 208,00 €
* Investície: 332,00 €

\enditems

Ďakujeme vám, že ste mnohí prispeli na činnosť a službu zboru. Aj naďalej máte možnosť prispieť do „nedeľnej zbierky“, a to prevodom na účet zboru. Do poznámky pre prijímateľa, prosím, uveďte „zbierka“.

Bankové spojenie: SK36 0900 0000 0000 1147 1836, SWIFT: GIBASKBX

Ďakujeme!


\n 3.	2.	Vlasta	BALÁŽOVÁ;
\n 3.	2.	Margita	KRÁĽOVÁ;
\n 3.	2.	Miroslav	ANTALÍK;
\n 5.	2.	Štefánia	ANTALÍKOVÁ;
\n 5.	2.	Barbora	ANTALÍKOVÁ;
\n 11.	2.	Juraj	BALÁŽ;
\n 11.	2.	Oľga	KOVÁČOVÁ;
\n 11.	2.	Beáta	BOGÁROVÁ;
\n 12.	2.	Martin	PRIBULA;
\n 13.	2.	Zlatica	VYSKOČILOVÁ;
\n 15.	2.	Ingrid	JANČULOVÁ;

\narodeniny


\program{
\p  1 ; st ;.;;.;;
\p  2 ; št ;.;;.;;
\p  3 ; pi ;.;;.;;
\p  4 ; so ;.;;.;;
\p  5 ; ne ;  9.30 ; Bohoslužby (P. Šrankota) ;.;;
\p  6 ; po ;.;;.;;
\p  7 ; ut ; 15.15 ; Biblická hodina pre seniorov (P. Pivka) ;.;;
\p  8 ; st ;.;;.;;
\p  9 ; št ; 18.00 ; Biblická hodina (J. Szőllős) ;.;;
\p 10 ; pi ; 17.30 ; Dorast ;.;;
\p 11 ; so ; 18.00 ; Mládež ;.;;
\p 12 ; ne ;  9.30 ; Bohoslužby (T. Valchář) ;.;;
\p 13 ; po ;.;;.;;
\p 14 ; ut ; 15.15 ; Biblická hodina pre seniorov (P. Pivka) ;.;;
\p 15 ; st ; 17.30 ; Stretnutie sestier ;.;;
\p 16 ; št ; 18.00 ; Biblická hodina (J. Szőllős) ;.;;
\p 17 ; pi ; 17.30 ; Dorast ;.;;
\p 18 ; so ; 18.00 ; Mládež ;.;;
\p 19 ; ne ;  9.30 ; Bohoslužby (P. Kolárovský) ;.;;
\p 20 ; po ;.;;.;;
\p 21 ; ut ; 15.15 ; Biblická hodina pre seniorov (P. Pivka) ;.;;
\p 22 ; st ;.;;.;;
\p 23 ; št ; 18.00 ; Biblická hodina (J. Szőllős) ;.;;
\p 24 ; pi ; 17.30 ; Dorast ;.;;
\p 25 ; so ; 18.00 ; Mládež ;.;;
\p 26 ; ne ;  9.30 ; Bohoslužby (J. Rice) ;.;;
\p 27 ; po ;.;;.;;
\p 28 ; ut ; 15.15 ; Biblická hodina pre seniorov (P. Pivka) ;.;;
}
\vskip1ex
\riadokkoncaprogramu{Z bohoslužieb je zabezpečený online prenos.}


\tiraz
\bye
