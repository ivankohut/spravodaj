\def\velkostpisma{10}
\def\velkostriadku{12.5}
\input makra.tex % nacitanie Ivanom pripravenych nastaveni a prikazov
\hyphenation{star-šov-stvo} % rozdelenie slov na konci riadku, treba tu uviest slova, ktore sam nepozna

\spravodaj{10}{2025}


\clanok{Ostaň hladný}
„A za to sa modlím, aby sa vaša láska vždy viac a viac rozhojňovala v~pravom poznaní a v~každej skúsenosti.“ (F 1,9)
Večeriam o~piatej. Ale okolo tretej hodiny dostanem hlad. Tak začnem odpočítavať čas, kým sa najem. Teším sa na jedlo. Niekedy ma láka niečo zjesť, aby som vydržal. Vieš, keď sa zdá, že piata hodina nikdy nepríde a v~blízkosti je Taco Bell (americký reťazec rýchleho občerstvenia s~mexickým menu), začnem rozmýšľať o~burrito supreme (plnené mexické placky) alebo možno len o~taco (malá plnená placka), aby som vydržal.
Problém je, že mi to kazí chuť do jedla. Keď sa nasýtim niečím, čo mi neprospieva, stratím chuť na veci, ktoré mi prospievajú.

To isté sa nám môže stať aj v~duchovnej oblasti. Sú veci, ktoré robíme, miesta, kam chodíme, a ľudia, s~ktorými sa stretávame, ktoré nás duchovne otupujú. Znižujú našu chuť na Božie veci a brzdia náš rast. Keď si ich doprajeme, výsledkom je, že duchovne chceme robiť menej, a nie viac.
Na druhej strane sú ľudia, ktorí nás duchovne inšpirujú svojím príkladom a zbožnosťou svojho charakteru. Nepovažujú sa za vzory a nepýtajú sa: „Prečo nie si viac ako ja?“ Jednoducho žijú svoju vieru spôsobom, ktorý je nielen inšpirujúci, ale aj motivujúci. Byť s~nimi stimuluje našu túžbu po Ježišovi.
A to je dobré, pretože to vedie k~tomuto: ako kresťania sa vždy máme čo učiť o~tom, ako byť učeníkmi. Vždy sa máme čo učiť o~tom, ako šíriť svoju vieru a povzbudzovať druhých. V~Božom slove je vždy čo objavovať.

Apoštol Pavol napísal v~Liste Filipským 1,9: „A za to sa modlím, aby sa vaša láska vždy viac a viac rozhojňovala v~pravom poznaní a v~každej skúsenosti.“ Jeho pointou je, že ako kresťan nikdy nie si hotový. Bez ohľadu na to, ako veľmi miluješ, môžeš milovať ešte viac. Bez ohľadu na to, ako veľmi sa modlíš, môžeš sa modliť ešte viac. Bez ohľadu na to, ako veľmi poslúchaš, môžeš poslúchať ešte viac.
Ježiš povedal: „Blahoslavení, ktorí lačnia a žíznia po spravodlivosti, lebo oni nasýtení budú.“ (Mt 5,6) Neviem, ako je to s~tebou, ale myslím si, že všetci kresťania by sa mali na seba pozrieť a povedať: „Nie som spokojný s~tým, kde duchovne som. V~mojom živote je ešte veľa vecí, ktoré musím zmeniť. Musím sa stať viac podobným Ježišovi.“

V~momente, keď skrížiš ruky a povieš: „Som spokojný s~tým, kde som,“ pripravuješ sa na pád. Keď Ježiš povedal: „Ak sa neobrátite a nebudete ako deti, nikdy nevojdete do kráľovstva nebeského.“ (Mt 18,3), nemal na mysli detinskosť. Mal na mysli detskú povahu -- neustále rásť, neustále sa učiť a niekedy sa znovu učiť veci, ktoré sme zabudli.

Buď hladný po Božej pravde. Žízni po spravodlivosti. Odmietni spočívať na svojich duchovných vavrínoch. Pokiaľ dýchaš, pokračuj v~raste vo svojom vzťahu s~Kristom. Hľadaj spôsoby, ako byť účinnejším modlitebným bojovníkom a evanjelistom. Ak tak urobíš, Boh sľubuje, že budeš naplnený.
\autor{Greg Laurie}


\clanok{Správy zo staršovstva}
Okrem už v~minulomesačnom spravodaji spomínaného stretnutia 2.~9. sa staršovstvo v~septembri stretlo ešte dvakrát.
Na prvom stretnutí sme sa obšírnejšie venovali otázke úlohy starších pri starostlivosti o~„zatúlaných členov“ a s~ňou súvisiaceho napomínania tých, za ktorých máme podľa Božieho slova zodpovednosť.
Ako obyčajne sme prechádzali naše uznesenia -- otvorené úlohy a dolaďovali program bohoslužieb, ako aj aktivít zboru do konca tohto kalendárneho roka. Venovali sme sa aj príprave stretnutia s~možným kandidátom na kazateľa, bratom Timom Hanesom.

Aj druhé stretnutie bolo venované uzneseniam, príprave na diskusnú konferenciu delegátov zborov (DKDZ), príprave vzdelávacích aktivít, ako aj iných akcií v~našich priestoroch, a najmä oblastného vďakyvzdania, ktoré náš zbor organizuje.
Ako staršovstvo si uvedomujeme, že naša zodpovednosť za zbor je najmä duchovná. Nesieme vás na modlitbách a máme vás radi. Ak sa potrebujete porozprávať alebo modliť, neváhajte nás osloviť, sme tu pre vás.
\autor{Radislav Nemec}
\vfill\break


\clanok{Ako bolo na „šarkaniáde“}
Konala sa 27.~9.~2025 -- v~deň, keď bolo, vďaka Bohu, veľmi krásne počasie -- slnečno a pre šarkany vietor akurát. Pripojili sme sa ku Connect-u, a hoci nás nebolo veľa, užili sme si čas na parádne veľkom ihrisku so šarkanmi, pri rozhovoroch, grilovaní i pri vode (prítoku Malého Dunaja). Od 11-tej prekvapivo až do štvrtej. Je to krásne a ideálne miesto na takúto akciu a plánuje sa opakovať koncom septembra aj v~budúcom roku.
\autor{Eva Syčová}


\clanok{Verš na mesiac}
V~októbri sa budeme učiť verš, ktorý dostal pre rok 2025 spevokol: „Sviecou mojim nohám je Tvoje slovo a svetlo mojim chodníkom.“ (Ž 119,105)


\clanok{Modlitby v~pondelky}
V~polovici júna sme oprášili našu staršiu skupinku s~úmyslom niesť na modlitbách naše deti. V~priebehu krátkeho času sa pridali modlitebníci aj témy. Nesieme na modlitbách nielen naše malé deti, ale aj naše dospelé deti, našich drahých blízkych, ktorí nie sú s~nami v~spoločenstve. Modlíme sa pravidelne za staršovstvo, za kazateľov, za aktuálne témy v~zbore aj aktuálne potreby jednotlivcov. Stretnutia prebiehali aj počas prázdnin. Každý pondelok o~18.00~hod. sa stretávame na Zrínskeho, vždy s~vďačnosťou za dar modlitby. Skupinka nie je uzatvorená a srdečne pozývame každého, kto by sa chcel pridať.
\autor{Slávka Kráľová}


\clanok{Vďakyvzdanie BJB zborov Bratislavy a okolia}
Po dlhšej prestávke opäť plánujeme stretnutie zborov BJB z~Bratislavy a jej okolia. Náš zbor je hlavným organizátorom stretnutia.
Spoločná bohoslužba bude v~nedeľu 19.~10.~2025 od 9.30~hod. v~SÚZA.
Zhromaždenie na Palisádach v~tomto termíne nebude. Účasť prisľúbili zbory Miloslavov, Bernolákovo, Podunajské Biskupice, Viera, Connect a Nádej.
Pre deti z~besiedok je plánované divadielko od 10.00~hod.
Prosíme všetkých, ktorí sa prihlásili na obed (13~€/osoba), aby ho uhradili čo najskôr na zborový účet SK36 0900 0000 0000 1147 1836, do poznámky uveďte: „vďakyvzdanie“.

Po bohoslužbe plánujeme vytvoriť priestor pre stretnutia, rozhovory, spoznanie sa, a s~tým je spojené aj občerstvenie. Prosíme preto ochotné sestry (brat sa už prihlásil \smiley), ktoré napečiete niečo slané alebo sladké pre túto príležitosť. Prihláste sa osobne, e-mailom alebo správou u~Katky Kerekréty. Zatiaľ je nás prihlásených máličko. Ďakujeme pekne.


\clanok{Zborové členské zhromaždenie ZČZ -- 26.~10.~2025}
ZČZ sa uskutoční v~nedeľu 26.~10.~2025 na Palisádach. Jedným z~bodov programu bude prijímanie nových členov. Celý program a čas budú členom oznámené.


\clanok{Senior klub}
Senior klub sa v~septembri neuskutočnil. Veľa našich seniorov sa zúčastnilo 22. ročníka Konferencie seniorov v~Račkovej doline.
V~októbri je senior klub plánovaný tradične na posledný štvrtok v~mesiaci, a to 30.~10.~2025 od~10.00 do~14.00~hod. na Súľovskej~2. Tému ešte upresníme v~zborových oznamoch.


\clanok{Krst}
Náš zbor pripravuje krst v~termíne 30.~11.~2025, v~nedeľu poobede (miesto a čas ešte určíme). Kto túži vyznať svoju vieru v~krste, nech sa prihlási u~br. kazateľa Janka Szőllősa alebo u~členov staršovstva.


\clanok{Obedy s~bratmi}
Bratia Janko Szőllős, Filip Barkóczi a Dávid Chuchút prijímajú naše pozvania na obed do našich domácností. Prosíme, zapíšte sa do online tabuľky.
\vfill\break


\clanok{Zbierky v~septembri}
\table{lr}{
Na misiu			&   235~€ \cr
Na investičný fond 	&    71~€ \cr}
\vskip1em

Pripomíname, že zbierky sa u~nás konajú pravidelne takto:
\vskip-1ex\begitems
* každú 2. nedeľu v~mesiaci je zbierka venovaná misii a
* každá 4. nedeľa je zbierka na investície.
\enditems

Aj naďalej máte možnosť prispieť do „nedeľnej zbierky“, a to prevodom na účet zboru. Do poznámky pre prijímateľa, prosím, uveďte „zbierka“.

Bankové spojenie: SK36 0900 0000 0000 1147 1836, SWIFT: GIBASKBX


\n 2.	10.	Peter	ANTALÍK;
\n 6.	10.	Daniel	BALÁŽ;
\n 12.	10.	Barbora	PRIBULOVÁ;
\n 14.	10.	Martin	SIMON;
\n 20.	10.	Ida	PUČEKOVÁ;
\n 22.	10.	Hana	HALAMIČKOVÁ;
\n 25.	10.	Vladimír	IRA;
\n 26.	10.	Martin	HOVORKA;
\n 27.	10.	Miriam	KRÁĽOVÁ;
\n 28.	10.	František	VRABČEK;
\n 28.	10.	Ľubomír	SYČ;

\narodeniny


\programna{10}{
\p  1 ; st ;.;;.;;
\p  2 ; št ; 18.00 ; Biblická hodina (J.~Szőllős);.;;
\p  3 ; pi ; 17.30 ; Dorast ;.;;
\p  4 ; so ; 18.00 ; Mládež ;.;;
\p  5 ; ne ;  9.30 ; Bohoslužby (R.~Nagypal + VP);.;;
\p  6 ; po ; 18.00 ; Modlitby ;.;;
\p  7 ; ut ; 15.00 ; Biblická hodina pre seniorov (P.~Pivka);.;;
\p  8 ; st ; 17.30 ; Stretnutie sestier ;.;;
\p  9 ; št ;.;;.;;
\p 10 ; pi ; 17.30 ; Dorast ;.;;
\p 11 ; so ; 18.00 ; Mládež ;.;;
\p 12 ; ne ;  9.30 ; Bohoslužby (D.~Uhrin);.;;
\p 13 ; po ; 18.00 ; Modlitby / skupinka „Základy viery“;.;;
\p 14 ; ut ; 15.00 ; Biblická hodina pre seniorov (P.~Pivka);.;;
\p 15 ; st ;.;;.;;
\p 16 ; št ; 18.00 ; Biblická hodina (D.~Chuchút);.;;
\p 17 ; pi ; 17.30 ; Dorast ;.;;
\p 18 ; so ; 18.00 ; Mládež ;.;;
\p 19 ; ne ;  9.30 ; Bohoslužby (vďakyvzdanie v~SÚZA);.;;
\p 20 ; po ; 18.00 ; Modlitby ;.;;
\p 21 ; ut ; 15.00 ; Biblická hodina pre seniorov (P.~Pivka);.;;
\p 22 ; st ;.;;.;;
\p 23 ; št ; 18.00 ; Biblická hodina (F.~Barkóczi);.;;
\p 24 ; pi ; 17.30 ; Dorast ;.;;
\p 25 ; so ; 18.00 ; Mládež ;.;;
\p 26 ; ne ;  9.30 ; Bohoslužby (D.~Chuchút); . ;ZČZ;
\p 27 ; po ; 18.00 ; Modlitby / skupinka „Základy viery“;.;;
\p 28 ; ut ; 15.00 ; Biblická hodina pre seniorov (P.~Pivka);.;;
\p 29 ; st ;.;;.;;
\p 30 ; št ; 18.00 ; Biblická hodina (D.~Chuchút);.;;
\p 31 ; pi ; 17.30 ; Dorast;.;;
}


\tiraz
\bye
