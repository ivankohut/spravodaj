\def\velkostpisma{10}
\def\velkostriadku{12.5}
\input makra.tex % nacitanie Ivanom pripravenych nastaveni a prikazov
\hyphenation{star-šov-stvo} % rozdelenie slov na konci riadku, treba tu uviest slova, ktore sam nepozna

\spravodaj{4}{2023}


\clanok {Môj príbeh}
Je obdobie Veľkej noci. Príbeh Veľkej noci je známy takmer celému svetu. Ježiš svojim životom písal ten najdôležitejší príbeh všetkých čias. Je to príbeh, ktorý má dosah aj na naše životy, lebo jeho súčasťou sme aj my. Na druhej strane každý z~nás má svoj vlastný príbeh. Aký príbeh píšeš ty? Príbeh, ktorý píšeme, či si to uvedomujeme alebo nie. Dokonca tvoj príbeh sa píše, či na to myslíš alebo nie. Každé tvoje rozhodnutie určuje to, ako tvoj príbeh bude ďalej smerovať. Rozhodujeme aj o~tom, čomu v~našom príbehu priestor dáme a čomu nie.
Čo je v~jadre tvojho príbehu? Čomu nechávaš vo svojom príbehu najväčší počet strán? Určité časti príbehov sa opakujú a nachádzajú sa v~každom príbehu.

„Život som vždy vnímal predovšetkým ako príbeh. A~všade, kde je príbeh, je aj rozprávač“ (G.~K.~Chesterton). Presne tak! Do deja príbehov vstupuje rozprávač, ktorý má čo povedať každému z~nás. On pozná naše príbehy lepšie ako my sami. Je to Boh, ktorý ako rozprávač zanecháva stopu v~každom príbehu. My sa často cítime ako autor príbehu, ktorý má veci pod kontrolou. Ale opak je pravdou. Lebo rozprávač neustále vstupuje do deja a zachraňuje to, čo sme my ako takzvaní autori nedomysleli. Rozpoznal si stopu rozprávača vo svojom príbehu? „Naše hľadanie Boha je úspešné iba preto, lebo On donekonečna hľadá spôsoby, ako by sa nám ukázal“ (A.~W.~Tozer).
Boh ako rozprávač dnes opäť prehovára aj k~tebe, aby ti povedal, ako chce, aby sa tvoj príbeh vyvíjal. Chce sa s~tebou stretnúť, aby ti povedal konkrétne riešenia pre nekonečné množstvo zápletiek, ktoré sa v~jadre tvojho príbehu neustále opakujú.

O~čom hovorí tvoj príbeh...? Aký je jeho hlavný odkaz? Kto je hlavnou postavou v~tvojom príbehu? Ježiš, ako autor príbehu Veľkej noci, sa uchádza o~miesto hlavnej postavy v~tvojom príbehu. Už si obsadil miesto hlavnej postavy niekým iným? V~našom živote je mnoho vecí aj ľudí, ktorí zápasia o~našu pozornosť. Na tebe záleží, čomu ju venuješ.

Ježiš urobil všetko pre to, aby tvoj príbeh mal dobrý koniec. Veríš tomu? On každému, kto v~Neho verí, obnovil prístup k~Otcovi v~nebesiach. Obnovil prístup k~uzdraveniu, naplneniu, občerstveniu, ktoré môžeme zažívať v~Božej prítomnosti.
Je mnoho vecí, ktoré nás v~našich príbehoch ťažia, ktoré nám spôsobujú bolesti a trápia nás. Mnohé z~nich si nesieme celým dejstvom. Pod ťarchou ďalších klesáme a padáme, idúc hlavným dejstvom svojho príbehu. Ale aj za to bol zaplatená cena vykúpenia.

„Neexistuje väčšie utrpenie ako niesť nevypovedaný príbeh vo svojom vnútri“ (Maya Angelou). S~týmto výrokom súhlasím len do určitej miery. Napriek tomu dobre poukazuje na potrebu človeka zdieľať sa so svojím príbehom. Už si niekomu zdieľal svoj príbeh bez ohľadu na to, aký tvoj príbeh je? Alebo si ho píšeš sám pre seba, lebo sa bojíš, že by ťa iní za tvoj príbeh odsúdili? Ježišov životný príbeh mnohým vyzeral neakceptovateľne, iní mali od neho svoje očakávania. Napriek tomu priniesol riešenie, ktoré v~konečnom dôsledku aj ty potrebuješ na rozuzlenie každej zápletky svojho príbehu.

Daj priestor stretnutiu s~Ježišom aj počas tejto Veľkej noci, aby priniesol vzkriesenie pre nový začiatok, ktorý povedie k~trvalej zmene.

„Veď to je vôľa môjho otca, aby každý, kto vidí Syna a verí v~Neho, mal večný život a ja ho vzkriesim v~posledný deň.“ (Ján 6,~40)

\autor{Peter Šrankota}


\clanok {Správy zo staršovstva}
Staršovstvo zboru sa stretlo v~mesiaci marec dva razy, a to v~utorky 14. a 21.~3.
Prioritne sa venovalo príprave výročného zborového členského zhromaždenia konaného 26.~3.~2023.

V~rámci prípravy to bolo konkrétne:
\vskip-1ex\begitems
* stretnutia so záujemcami o~členstvo v~zbore;
* stretnutie s~členmi zboru, ktorí prejavili záujem o~diskusiu k~predloženým dokumentom, konkrétne k~návrhu rozpočtu a k~navrhovaným úpravám zborového a volebného poriadku;
* príprava návrhu stanoviska k~založeniu zborovej stanice Connect;
* príprava krokov vedúcich k~osamostatneniu zborovej ukrajinskej stanice;
* oboznámenie sa s~informáciou o~hospodárení zboru za rok 2022;
* príprava návrhu rozpočtu zboru na rok 2023.
\enditems

Ďalej to bolo najmä:
\vskip-1ex\begitems
* riešenie zefektívnenia platieb zboru za energie;
* aktualizácia zoznamu členov zboru;
* žiadosť br.~Dzuriaka o~uvoľnenie zo služby v~staršovstve zboru a riešenie personálnej náhrady;
* plánovanie víkendového výjazdového rokovania staršovstva.
\enditems

Ďakujeme za vaše modlitby a prosíme, aby ste v~nich vytrvali.

\autor {za staršovstvo Peter Antalík}


\clanok {Bohoslužby počas veľkonočných sviatkov}
Bohoslužby vo štvrtok 6. apríla, sa  budú konať o~18.00~hod., slúžiť bude br.~J.~Szőllős.
Na Veľký piatok 7.~apríla o~10.00~hod. bude spoločná veľkonočná bohoslužba bratislavských spoločenstiev v~UPC. Slovom bude slúžiť br.~M.~Tóth z~Nitry.
Podvečer o~17.00~hod. budeme mať bohoslužbu u~nás a kázať bude br.~J.~Szőllős. V~nedeľu 9.~apríla nám od 9.30~hod. poslúži náš br.~kazateľ P.~Šrankota.


\clanok {Večer v~kostole}
Bratia a sestry z~ECAV nás srdečne pozývajú na „Večer v~kostole“, ktorý sa bude konať dňa 16.~apríla~2023 o~18.00~hod. v~Novom evanjelickom kostole na Legionárskej ulici v~Bratislave. Počas podujatia odznejú úryvky z~knihy Kristíny Royovej Bez Boha na svete v~sprievode hudby a duchovných piesní. Podujatie organizuje kantorka CZ ECAV Bratislava Legionárska sestra Monika Ficzová. Vstup je voľný!


\clanok{Verš na zapamätanie}
Tento mesiac máme nový veršík, ktorý sa chceme spoločne učiť. Veríme, že poznanie Písma prospeje našej duši i našej mysli:

{\it „Veď to je vôľa môjho Otca, aby každý kto vidí Syna a verí v~Neho, mal večný život a ja ho vzkriesim v~posledný deň.“}

\autor{Ján~6,~40}


\clanok{Zbierky}
Milí bratia a sestry, ďakujeme za vašu obetavosť. V~mesiaci marec ste prispeli:
\vskip-1ex\begitems
* misia: 574,50 €
* investičný fond: 390,00 €
\enditems


\n 1.	4.	Miroslav	KOLÁŘIK;
\n 4.	4.	Vierka	ŠKODÁK;
\n 6.	4.	Jana	ZAJACOVÁ;
\n 6.	4.	Jarmila	CIHOVÁ;
\n 10.	4.	Anna	PAVLÍKOVÁ;
\n 11.	4.	Daniel	MIKLETIČ;
\n 19.	4.	Marta	PRIBULOVÁ;
\n 22.	4.	Alexander	Koloman	ERDÉLYI;
\n 25.	4.	Elena	TALIGOVÁ;
\n 30.	4.	Jaroslav	VOLENTIČ;
\n 30.	4.	Ľuboš	DZURIAK;
\narodeniny


\program{
\p  1 ; so ; 17.00 ; Veľkonočný koncert ;.;;
\p  2 ; ne ;  9.30 ; Bohoslužby (P. Šrankota) ;.;;
\p  3 ; po ;.;;.;;
\p  4 ; ut ; 15.15 ; Biblická hodina pre seniorov (P. Pivka) ;.;;
\p  5 ; st ;.;;.;;
\p  6 ; št ; 18.00 ; Bohoslužby (J. Szőllős) ;.;;
\p  7 ; pi ; 17.00 ; Bohoslužby (J. Szőllős) ;.;;
\p  8 ; so ;.;;.;;
\p  9 ; ne ;  9.30 ; Bohoslužby (P. Šrankota) ;.;;
\p 10 ; po ;.;;.;;
\p 11 ; ut ; 15.15 ; Biblická hodina pre seniorov (P. Pivka) ;.;;
\p 12 ; st ; 17.30 ; Stretnutie sestier ;.;;
\p 13 ; št ; 18.00 ; Biblická hodina (J. Szőllős) ;.;;
\p 14 ; pi ; 17.30 ; Dorast ;.;;
\p 15 ; so ; 18.00 ; Mládež ;.;;
\p 16 ; ne ;  9.30 ; Bohoslužby (P. Kolárovský) ;.;;
\p 17 ; po ;.;;.;;
\p 18 ; ut ; 15.15 ; Biblická hodina pre seniorov (P. Pivka) ;.;;
\p 19 ; st ;.;;.;;
\p 20 ; št ; 18.00 ; Biblická hodina (J. Szőllős) ;.;;
\p 21 ; pi ; 17.30 ; Dorast ;.;;
\p 22 ; so ; 18.00 ; Mládež ;.;;
\p 23 ; ne ;  9.30 ; Bohoslužby (F. Barkóczi) ;.;;
\p 24 ; po ;.;;.;;
\p 25 ; ut ; 15.15 ; Biblická hodina pre seniorov (P. Pivka) ;.;;
\p 26 ; st ;.;;.;;
\p 27 ; št ; 18.00 ; Biblická hodina (J. Szőllős) ;.;;
\p 28 ; pi ; 17.30 ; Dorast ;.;;
\p 29 ; so ; 18.00 ; Mládež ;.;;
\p 30 ; ne ;  9.30 ; Bohoslužby (D. Kraljik) ;.;;
}
\vskip1ex
\riadokkoncaprogramu{Z bohoslužieb je zabezpečený online prenos.}


\tiraz
\bye
