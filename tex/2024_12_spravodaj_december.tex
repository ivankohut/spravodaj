\def\velkostpisma{9}
\def\velkostriadku{12}
\input makra.tex % nacitanie Ivanom pripravenych nastaveni a prikazov
\hyphenation{star-šov-stvo} % rozdelenie slov na konci riadku, treba tu uviest slova, ktore sam nepozna

\spravodaj{12}{2024}


\clanok {Čo by chcel Ježiš na Vianoce?}
Čo by chcel Ježiš na Vianoce? Odpoveď môžeme vidieť v~Jeho modlitbách. O~čo prosí Boha? Jeho najdlhšia modlitba je v~Jánovi, v~17.~kapitole. Toto je jeho najväčšia túžba:

„Otče, chcem, aby aj tí, ktorých si mi dal, boli so mnou, kde som ja.“ (v. 24)

Medzi všetkými nehodnými hriešnikmi na svete sú niektorí, ktorých Boh „dal Ježišovi“. Sú to tí, ktorých Boh pripravil pre Ježiša (Ján 6,44.65). Sú to kresťania -- ľudia, ktorí „prijali“ Ježiša ako ukrižovaného a vzkrieseného Spasiteľa a Pána a Poklad ich životov (Ján 1,12; 10,11.17–18; 20,28; 6,35; 3,17). Ježiš hovorí, že chce, aby boli s~Ním.

Niektorí ľudia tvrdia, že Boh stvoril človeka, pretože sa cítil osamelo. Inak povedané, „Boh nás stvoril, a preto by sme mali byť s~Ním“. Súhlasil by s~tým Ježiš? Dobre, skutočne chce, aby sme boli s~Ním. Áno, ale prečo? Posúď zvyšok verša. Prečo Ježiš chce, aby sme s~Ním boli?

„…a videli moju slávu, ktorú mi (Otec) dal, pretože si ma miloval ešte pred založením sveta.“

Toto veľmi nesúvisí s~osamelosťou. „Chcem, aby videli moju slávu.“ V~skutočnosti to nevyjadruje osamelosť. Ježiš nie je osamelý. Ježiš, Jeho Otec a Duch sú hlboko uspokojení priateľstvom Trojice. Nie On, ale my niečo potrebujeme. Čo Ježiš chce na Vianoce, je, aby sme mohli zažiť to, kvôli čomu sme boli stvorení -- vidieť a zakúsiť Jeho slávu.

Boh to chce vliať do našich duší! Ježiš nás stvoril (Ján 1,3) na to, aby sme videli Jeho slávu. Práve pred ukrižovaním vyznáva Otcovi svoje najhlbšie túžby: „Otče, chcem -- chcem! -- aby oni… boli so mnou, kde som ja, aby videli moju slávu.“

Ale toto je iba polovica toho, o~čo Ježiš žiada v~záverečných vrcholiacich veršoch jeho modlitby. Práve som povedal, že sme boli stvorení, aby sme videli a zakúsili jeho slávu. Je to to, čo chcel -- aby sme nielen videli jeho slávu, ale ju aj zažili, mali v~nej záľubu, tešili sa z~nej, vážili si ju a milovali? Pouvažujme nad veršom 26 (posledný verš):

„Oznámil som im Tvoje meno a ešte oznámim, aby láska, ktorou si ma miloval, bola v~nich a ja aby som bol v~nich.“

Toto je záver modlitby. Čo je Ježišovým konečným zámerom s~nami? Nie, aby sme iba videli jeho slávu, ale aby sme ho milovali rovnakou láskou, akou ho miluje Otec:
„…aby láska, ktorou si ma miloval, bola v~nich.“ Ježišovou túžbou je, aby sme videli Jeho slávu a aby sme boli schopní milovať to, čo vidíme láskou, ktorú má Otec pre Syna. A~nemyslí tým, že máme iba imitovať lásku Otca k~Synovi. Otcova veľká láska sa stáva našou láskou k~Synovi. Teda milujeme Syna láskou Otca k~Synovi. A~to je to, čo udeľuje a prináša Duch do našich životov: Lásku k~Synovi od Otca cez Ducha.

To, čo Ježiš chce na Vianoce najviac je, aby jeho vyvolení boli zhromaždení a dostali, po čom najviac túžia -- vidieť Jeho slávu a zažiť ju láskou, ktorú má Otec pre Syna.

To, čo si ja najviac tento rok želám na Vianoce, je spojiť sa s vami (a mnohými ďalšími) vo vnímaní Ježiša v jeho plnosti, a tak budeme spolu schopní milovať to, čo vidíme láskou, ktorá prekračuje našu vlastnú ľudskú polovičatú schopnosť milovať.

Toto je to, o čo Ježiš prosí na Vianoce: „Otče, ukáž im moju slávu a daj im moju radosť, ktorú mám od teba.“ Ach, nech uvidíme Krista očami Boha a zažijeme Krista cez srdce Boha. To je podstata neba. Je to dar, pre ktorý Kristus prišiel vykúpiť hriešnikov vlastnou smrťou.

Nech Ho vidíte a zakúsite!

\autor {John Piper, prevzaté z \ulink[https://chcemviac.com/clanky/co-by-chcel-jezis-na-vianoce/]{chcemviac.com}}


\clanok {Správy zo staršovstva}
V~novembri sa staršovstvo zboru stretlo trikrát.

Jedno celé stretnutie bolo venované vyhodnoteniu pôsobenia brata kazateľa Petra Šrankotu v~našom zbore. Našim cieľom bolo nájsť príčiny vzniku napätej situácie v~zbore a poučiť sa z~nich a tiež sme sa modlili za zbor a za brata kazateľa a jeho rodinu.

„Starší v~príprave“ bola téma mimoriadneho stretnutia staršovstva s~bratmi Timom Hanesom a Zdenkom Švejkovským (ktorý túto prípravu absolvoval) zo zboru BJB Revúcka Lehota. Ako aj názov naznačuje, jedná sa o~prípravu bratov na službu staršieho zboru a má pomôcť nádejným starším, aby mohli rozsúdiť, či ich Pán Boh skutočné povoláva do služby starších v~zbore. Starší v~príprave sú riadne volení a po zvolení sa začnú teoreticky (pod vedením skúseného staršieho zboru alebo kazateľa) a prakticky (účasťou na stretnutiach staršovstva) pripravovať k~službe staršieho zboru. Počas tohto obdobia majú v~staršovstve poradný hlas avšak nenesú zodpovednosť pred zborom za jeho rozhodnutia. Je to čas, kedy sa môže v~nich vybudovať túžba byť starším a presvedčenie, že je to Božie povolanie pre nich. Môžu sa kedykoľvek rozhodnúť, že to nie je miesto, kde ich Pán chce použiť a z~pozície odstúpiť. S~touto myšlienkou sa ako staršovstvo stotožňujeme a chceli by sme ZČZ predložiť návrh na zavedenie starších v~príprave aj v~našom zbore.

Napokon na treťom stretnutí sme riešili prípravu volieb staršovstva zboru, prípadne starších v~príprave, termíny ZČZ a výročného ZČZ a program počas sviatkov.
Ďalšie, posledné, oficiálne stretnutie staršovstva v~tomto roku sa uskutoční 10.~12.~2024.

\autor {za staršovstvo R. Nemec}


\clanok {Verš na mesiac}
„Zjavila sa totiž Božia milosť prinášajúca spásu všetkým ľuďom, ktorá nás vychováva, aby sme sa zriekli bezbožnosti a svetských žiadostí a žili v terajšom veku rozvážne, spravodlivo a nábožne.“ (Tít 2,11-12)

\clanok {Stretnutie sestier}
Sestry sa stretnú v~stredu 11.~12. o~17.30~hod. na Zrínskeho. Pozvaným hosťom je sestra Táňa Trúsiková z~BJB Viera. Všetky sestry sú srdečne vítané.


\clanok {Pre členov zboru}
Členovia zboru majú odovzdať návrhy na kandidátov na správcu zboru a členov staršovstva volebnej komisii do konca kalendárneho roka 2024, takisto posielajú návrhy do rozpočtu na \email{starsovstvo@bjbpalisady.sk} do 7.~1.~2025.


\clanok {Celkové plnenie rozpočtu k~27.~11.~2024}
\table{lrrr}{
Príjem				& Plán		& Skutočnosť	& podiel z~ročného plánu \crli
Nedeľné zbierky		& 28 000 €	& 23 324 €	 	& 83,30 \% \cr
Dary a desiatky		& 34 000 €	& 30 333 €	 	& 89,21 \% \cr
Misijný fond 		&  5 500 €	&  5 489 €	 	& 99,80 \% \cr
Investičný fond		&  4 000 €	&  1 696 €	 	& 42,40 \% \cr
Záväzky na fasádu	& 92 700 €	& 81 381 €	 	& 87,79 \% \cr
Plat J. Szőllősa	&  7 493 €	&  4 374 €	 	& 58,37 \% \cr}
\vskip1ex

Príjmy zo zbierok a darov sú nižšie ako plánované v~zborovom rozpočte. Podobne príspevky na plat druhého kazateľa sú nižšie ako vyplatená mzda. Odhadovaný príjem na konci roka je nižší o~cca 5~000~€, z~toho vyplýva, že na bežné výdavky zboru si budeme musieť požičať z~prostriedkov fondov určených na iné účely.

\autor{Ľubomíra Kohútová}


\n 2.	12.	Helena	MIKLETIČOVÁ;
\n 3.	12.	Ľubica	IROVÁ;
\n 5.	12.	Tomáš	LAURENČÍK;
\n 6.	12.	Elise	ATKINS;
\n 9.	12.	Ondrej  ŠKODÁK;
\n 9.	12.	Kamila	ZAJÍČKOVÁ;
\n 11.	12.	Vladimíra	LAURENČÍKOVÁ;
\n 11.	12.	Maroš	KOHÚT;
\n 13.	12.	Peter	KOLÁROVSKÝ;
\n 16.	12.	Pavel	KONDAČ ml.;
\n 16.	12.	Martin	PELÍŠEK;
\n 23.	12.	Diana	DZURIAKOVÁ;
\n 23.	12.	Eva Rudy DOROVÁ;
\n 24.	12.	Slávka	VOLENTIČOVÁ;
\n 25.	12.	Dana	PELÍŠKOVÁ;
\n 26.	12.	Jana	KRÁĽOVÁ;
\n 28.	12.	Dara	PLETT;
\n 29.	12.	Ján	KOVÁČIK;
\narodeniny


\program{
\p  1 ; ne ;  9.30 ; Bohoslužby (J. Szőllős + VP);.;;
\p  2 ; po ; 18.00 ; Skupinka „Základy viery“;.;;
\p  3 ; ut ; 14.00 ; Biblická hodina pre seniorov (P. Pivka);.;;
\p  4 ; st ;.;;.;;
\p  5 ; št ; 18.00 ; Biblická hodina (J. Szőllős);.;;
\p  6 ; pi ; 17.30 ; Dorast;.;;
\p  7 ; so ; 17.00 ; The Hope Gospel Singers koncert ; 18.00 ; Mládež ;
\p  8 ; ne ;  9.30 ; Bohoslužby (P. Pribula);.;;
\p  9 ; po ;.;;.;;
\p 10 ; ut ; 14.00 ; Biblická hodina pre seniorov (P. Pivka);.;;
\p 11 ; st ; 17.30 ; Stretnutie sestier;.;;
\p 12 ; št ; 18.00 ; Biblická hodina (J. Szőllős);.;;
\p 13 ; pi ; 17.30 ; Dorast;.;;
\p 14 ; so ; 17.00 ; Vianočný koncert spevokolu ; 18.00 ; Mládež ;
\p 15 ; ne ;  9.30 ; Bohoslužby (F. Barkóczi + besiedka) ; 17.00 ; Vianočný koncert spevokolu ;
\p 16 ; po ; 18.00 ; Skupinka „Základy viery“;.;;
\p 17 ; ut ; 14.00 ; Biblická hodina pre seniorov (P. Pivka);.;;
\p 18 ; st ;.;;.;;
\p 19 ; št ; 18.00 ; Biblická hodina (J. Szőllős);.;;
\p 20 ; pi ; 17.30 ; Dorast vianočný;.;;
\p 21 ; so ;.;;.;;
\p 22 ; ne ;  9.30 ; Bohoslužby (J. Szőllős);.;;
\p 23 ; po ;.;;.;;
\p 24 ; ut ; 15.00 ; Bohoslužby vianočné (P. Kolárovský);.;;
\p 25 ; st ; 10.00 ; Bohoslužby -- 1. sviatok vianočný (S. Baláž);.;;
\p 26 ; št ;.;;.;;
\p 27 ; pi ;.;;.;;
\p 28 ; so ;.;;.;;
\p 29 ; ne ;  9.30 ; Bohoslužby (Stano Kráľ);.;;
\p 30 ; po ;.;;.;;
\p 31 ; ut ; 18.00 ; Bohoslužby silvestrovské;.;;
}

\tiraz
\bye
