\def\velkostpisma{9}
\def\velkostriadku{12}
\input makra.tex % nacitanie Ivanom pripravenych nastaveni a prikazov
\hyphenation{star-šov-stvo} % rozdelenie slov na konci riadku, treba tu uviest slova, ktore sam nepozna

\spravodaj{11}{2024}

\def\sekcia#1{\vskip0.5em\noindent #1}

\clanok{JAHVE ŠALOM}

Základom mena „Jahve Šalom“ je hebrejské slovo {\em šalom}, označujúce omnoho viac ako len „pokoj“, ako ho my poznáme. V~našom chápaní sa zvyčajne spája so stavom bez vonkajšieho konfliktu alebo vyjadrenie vnútornej vyrovnanosti. Slovo šalom obsahuje aj takéto významy, ale ide ďalej a znamená tiež úplnosť, kompletnosť, dokonalosť, bezpečie, dobrotu. Šalom pochádza zo života v~súlade s~Bohom. Ovocie tohto súladu je harmónia s~inými, zdar, zdravie, spokojnosť, vernosť, úplnosť a požehnanie. Keď sa modlíš k~„Jahve Šalom“, modlíš sa k~Bohu, ktorý je zdrojom všetkého pokoja. Nie je div, že jeho Syn sa nazýva Knieža pokoja.

\sekcia{PÁN JE POKOJ -- JAHVE SHALOM}

\sekcia{KĽÚČOVÉ VERŠE}

„Gideón tam postavil oltár na počesť Hospodinovi a nazval ho: ‚Hospodin je pokoj.‘ V~abíezerovskej Ofre je dodnes.“ (Sud~6,24)

„Jeruzalemu vyprosujte pokoj. Kiež v~pokoji žijú tí, čo Ťa milujú! ‚Kiež je pokoj na Tvojich hradbách, istota v~Tvojich palácoch!‘ Pre svojich bratov a priateľov hovorím: ‚Pokoj v~tebe!‘“ (Ž~122,6-8)

\sekcia{ZAMYSLI SA}

Porozmýšľaj o~období svojho života keď si sa cítil zavalený rôznymi okolnosťami. Čo tieto ťažkostí spôsobilo a ako si na ne reagoval? Čo ti zíde na um, keď počuješ slovo „POKOJ“?

\sekcia{MODLITBA}

{\em Pane, túžim po pokoji, ktorý mi môžeš dať len Ty. Pomôž mi modliť sa za pokoj vo svete a~takisto za pokoj v~mojom srdci. Chcem žiť Tvoj pokoj ako znak Tvojej prítomnosti v~mojom živote. Pomôž mi konať veci tak, aby som mohol nie len prežívať skutočný šalom, ale ho aj šíriť navôkol. Amen.}

\sekcia{PÁN JE POKOJ -- JAHVE SHALOM}

„Pre nič nebuďte ustarostení, ale vo všetkom s~vďakou predkladajte Bohu svoje žiadosti vo svojich modlitbách a prosbách. A~pokoj Boží, ktorý prevyšuje každý rozum, uchráni vaše srdcia a vaše mysle v~Kristovi Ježišovi.“ (Fil~4,6-7)

\vskip-1ex\begitems
* {\it Chváľ ho}: Pretože je pre nás bohatým zdrojom milosti a pokoja v~každom čase.
* {\it Ďakuj mu}: Za to, že Boží Syn je náš pokoj.
* {\it Vyznaj mu}: Všetok strach a úzkosť vo svojom živote. Všetky zvyky, ktoré ti bránili prežívať Boží pokoj.
* {\it Pros ho}: Aby ti dal pokoj, ktorý prevyšuje každý rozum/chápanie.
\enditems

\sekcia{PRISĽÚBENIA SPOJENÉ S~MENOM JAHVE SHALOM}
 
Žiť v~Božej prítomnosti skrze Ducha Svätého znamená žiť v~pokoji, v~pokoji s~Bohom, v~pokoji s~inými, v~pokoji so sebou samým. On je zdrojom skutočného pokoja, požehnania, harmónie zdravia, bezpečia a naplnenia. Bez ohľadu na to, aký búrlivý môže náš svet byť, my máme pokoj, ktorý pochádza od Boha, nehľadiac na okolnosti.

\sekcia{PRISĽÚBENIA V~PÍSME}

„Blažený človek, čo našiel múdrosť, a muž, čo získal rozumnosť! Jej cesty sú príjemné a všetky jej chodníky sú pokoj.“ (Pr~3,13.17)
„Otvorte brány, nech vojde spravodlivý ľud, čo zachováva vernosť a má nezlomnú myseľ. Daruj mu pokoj, pokoj, veď v~Teba dúfa.“ (Iz~26,2-3)

PÁN JE POKOJ -- JAHVE SHALOM, ktorý vyplýva pre každého z~nás, aj zo slov: „Veď ja poznám zámer, ktorý mám s~vami -- hovorí Pán. Sú to myšlienky pokoja a nie súženia: dám vám budúcnosť a nádej. Keď budete volať ku mne, keď prídete a budete sa ku mne modliť, vyslyším vás. Budete ma hľadať a nájdete ma: ak ma budete hľadať celým svojím srdcom, dám sa vám nájsť -- hovorí Pán.“ (Jer~29,11-14)

\autor{na základe knihy Božie mená, Peter Šrankota}


\clanok {Správy zo staršovstva za október 2024}

Staršovstvo zboru sa v~októbri stretlo tri razy.

Na prvom stretnutí 1.~10.~2024 boli pozvaní záujemcovia o~členstvo v~zbore, pripravovali sme sa na zborové členské zhromaždenie, s~Adamom Alexajom sme riešili budúcnosť fungovania chválospevovej skupiny a rozdelenie služieb na najbližšie obdobie.

Na druhom mimoriadnom stretnutí v~sobotu 12.~10.~2024 sme diskutovali s~hosťami Danielom Mikletičom, Želkou Praženicovou a Ľubošom Kešjarom o~budúcnosti chvojnickej chalupy, diskutovali sme o~správcovi zboru a riešili technické detaily ukončenia služby Petra Šrankotu vo funkcii kazateľa zboru.

Na treťom stretnutí 15.~10.~2024 sme vyhodnotili ZČZ z~13.~10.~2024, prešli sme body DKDZ, riešili prosbu zboru Nádej, misijnú kampaň BJB, program a zbierku Dobrodina a manuál prevedenia zborovej hodiny. Ďalšie stretnutie staršovstva sa uskutoční 29.~10.~2024.

\autor {za staršovstvo M. Maďar}
   

\clanok {Verš na mesiac}
„Ale kto zachováva jeho slovo, v~tom sa Božia láska stala naozaj dokonalou. A~podľa toho poznávame, že sme v~ňom.“ (1J 2,5)


\clanok {Zborové členské zhromaždenie}
Staršovstvo zboru zvoláva zborové členské zhromaždenie na nedeľu 10.~11.~2024. Čas a program bude ešte upresnený.


\clanok {Stretnutie sestier}
Sestry sa stretnú v~stredu 13.~11.~2024 o~17.30~hod. na Zrínskeho~2. Témou bude Svetový deň modlitieb baptistických žien.


\clanok {Rekonštrukcia fasády}
V~októbri sme poslali už osloveným piatim firmám žiadosť o~aktualizáciu cenových ponúk s~tým, aby tam boli zahrnuté výsledky z~reštaurátorského výskumu. Zatiaľ máme odpoveď aj s~novou cenovou ponukou od dvoch firiem. Po obdržaní všetkých cenových ponúk sa stretne hospodársky výbor, aby cenové ponuky vyhodnotil a vybral pre nás najvhodnejšieho dodávateľa.

Informácia ohľadom plnenia finančných záväzkov k~30.~9.~2024:

\vskip1ex
\table{lrr}{
Príjmy -- záväzky na fasádu & & \crli
plán            & 92 700 € & \cr
skutkový stav   & 63 159 € & (68,13 \%) \cr}
\vskip1ex

Sme vďační, že z~Božej milosti môžeme takto pravidelne dopĺňať potrebné financie.

\autor {za hospodársky výbor Ľubomír Syč}


\clanok {Celkové plnenie rozpočtu k~22.~10.~2024}

\table{lrrr}{
Príjem				& Plán		& Skutočnosť	& podiel z~ročného plánu \crli
Nedeľné zbierky		& 28 000 €	& 20 820 €		& 74,36 \% \cr
Dary a desiatky		& 34 000 €	& 27 403 €		& 80,60 \% \cr
Misijný fond 		&  5 500 €	&  4 946 €		& 89,93 \% \cr
Investičný fond		&  4 000 €	&  1 696 €		& 42,40 \% \cr
Záväzky na fasádu	& 92 700 €	& 75 090 €		& 81,00 \% \cr
Plat J. Szőllősa	&  7 493 €	&  2 956 €		& 39,45 \% \cr}
\vskip1ex
 
Zbierka na Dobrodinu z~20.~10.~2024 vo výške 1~225~€ bola prevedená na bankový účet Dobrodiny.

\n 2.	11.	Tomáš	VALCHÁŘ;
\n 5.	11.	Katarína	KRÁĽOVÁ;
\n 6.	11.	Elena	PRIBULOVÁ;
\n 6.	11.	Eva	SYČOVÁ;
\n 9.	11.	Alžbeta	BETKOVÁ;
\n 9.	11.	Radovan	PAULEN;
\n 15.	11.	Bohumila	ŠALINGOVÁ;
\n 19.	11.	Dávid	PRIBULA;
\n 21.	11.	Ladislav	KAMOCSAI;
\n 22.	11.	Alena	SVOBODOVÁ;
\n 22.	11.	Peter	PRIBULA;
\n 25.	11.	Petra	ŠALINGOVÁ;
\n 27.	11.	Judita	KOLÁŘIKOVÁ;
\n 29.	11.	Jaroslav	KRÁĽ;
\narodeniny




\program{
\p  1 ; pi ; 17.30 ; Dorast;.;; 
\p  2 ; so ; 18.00 ; Mládež;.;;
\p  3 ; ne ;  9.30 ; Bohoslužby (V. Potockij + VP);.;; 
\p  4 ; po ;.;;.;;    
\p  5 ; ut ; 15.15 ; Biblická hodina pre seniorov (P. Pivka);.;;  
\p  6 ; st ;.;;.;;   
\p  7 ; št ; 18.00 ; Biblická hodina (J. Szőllős);.;; 
\p  8 ; pi ; 17.30 ; Dorast;.;; 
\p  9 ; so ; 18.00 ; Mládež;.;;
\p 10 ; ne ;  9.30 ; Bohoslužby (P. Pribula);.;;
\p 11 ; po ;.;;.;;  
\p 12 ; ut ; 15.15 ; Biblická hodina pre seniorov (P. Pivka);.;;
\p 13 ; st ; 17.30 ; Stretnutie sestier;.;;
\p 14 ; št ; 18.00 ; Biblická hodina (J. Szőllős);.;;
\p 15 ; pi ; 17.30 ; Dorast;.;; 
\p 16 ; so ; 18.00 ; Mládež;.;;
\p 17 ; ne ;  9.30 ; Bohoslužby (J. Szőllős);.;; 
\p 18 ; po ;.;;.;; 
\p 19 ; ut ; 15.15 ; Biblická hodina pre seniorov (P. Pivka);.;; 
\p 20 ; st ;.;;.;;  
\p 21 ; št ; 18.00 ; Biblická hodina (J. Szőllős);.;; 
\p 22 ; pi ; 17.30 ; Dorast;.;;
\p 23 ; so ; 18.00 ; Mládež;.;;
\p 24 ; ne ;  9.30 ; Bohoslužby (T. Hanes);.;;
\p 25 ; po ;.;;.;; 
\p 26 ; ut ; 15.15 ; Biblická hodina pre seniorov (P. Pivka);.;;
\p 27 ; st ;.;;.;; 
\p 28 ; št ; 18.00 ; Biblická hodina (J. Szőllős);.;;  
\p 29 ; pi ; 17.30 ; Dorast;.;; 
\p 30 ; so ; 18.00 ; Mládež;.;; 
}


\tiraz
\bye
