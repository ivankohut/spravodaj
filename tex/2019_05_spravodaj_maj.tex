% DOKUMENTACIA:

% Prazdny riadok za textom znamena ukoncenie odstavca.
% Cierne obldzniky na konci riadku (v PDF) - to nechaj na mna (moze to o.i. znamenat, ze treba pridat nejake slovo do \hyphenation, lebo ho sam nevie rozdelit na konci riadku)

% Prikazy pre casti spravodaja:
% \spravodaj{<mesiac>}{<rok>}
% \clanok{<nazov clanku>}
% \autor{<autor clanku>}
% \n<den.mesiac.meno> - zadefinovanie oslavenca
% \narodeniny - vytvorenie tabulky s~narodeninami vsetkych zadefinovanych oslavencov
% \tiraz - ukoncenie spravodaja tirazou

% Styl fontu:
% \bf - bold, plati do konca aktualne skupiny, napr. ak mas {aaa \bf bbb} ccc, tak aaa bude normalne, bbb bude bold, ccc bude normalne
% \it - italic (pouzit rovnakym sposobom ako \bf)
% \bi - bold italic (pouzit rovnakym sposobom ako \bf)
% \rm - normalne (pouzit rovnakym sposobom ako \bf)

% Dalsie prikazy a znaky:
% \begitems - zoznam (odrazky), informacie najdes na stranke http://petr.olsak.net/ftp/olsak/opmac/opmac-u.pdf#toc%3A.5
% \ulink[<cielova adresa]{<zobrazena adresa>} - klikatelny odkaz na webstranku
% \email{<adresa>} - klikatelny odkaz na e-mailovu adresu
% ~ - nedelitelna medzera, napr. v~dome, 21.~6.~2018
% -- - pomlcka (dvakrát -)
% „ - zaciatocna uvodzovka
% “ - koncova uvodzovka
% \noindent - najblizsi odstavec nebude odsadeny
% \vskip<velkost> - vertikalna medzera, napr. \vskip3pt alebo \vskip-3ex (zaporna medzera, t.j. posun smerom hore)

%\typosize[9/12]% - standard% - pouzita velkost pisma/riadku
\input makra.tex % nacitanie Ivanom pripravenych nastaveni a prikazov
\hyphenation{star-šov-stvo} % rozdelenie slov na konci riadku, treba tu uviest slova, ktore sam nepozna

\spravodaj{5}{2019}


\clanok {Radosť zo vzkriesenia}
Prišla Veľká noc, plná svojich zvyklostí a tradícií, ako aj povinností s~ňou spojených; a už aj prešla. Škoda. Sviatky sú dobré a mám dojem, že ako evanjelikáli ich nemáme dosť. Možno je to reakcia voči tradičným cirkvám a snaha byť triezvy a vážny, lebo to považujeme za niečo zbožnejšie. Zdá sa mi, že málo oslavujeme. Izrael oslavoval pravidelne a často. Bolo bežné radovať sa, spievať a tancovať. Radosť z~oslavy sa musí vrátiť do našich životov. Veľká noc a vzkriesenie Ježiša Krista sú k~tomu úžasnou motiváciou. Náš Pán žije a každý deň sa teší zo spoločného života. Neexistuje lepšia motivácia k~tancu.

Po veľkonočnom zhromaždení som vás vyzval k~tomu, aby ste cestou z~modlitebne radostne tancovali. Pokiaľ viem, urobila to len jedna vzácna a odvážna sestra. Nepoviem, o~koho ide, ale musím uznať, že vie dobre tancovať. Mal som z~toho veľkú radosť. Verím, že tá sestra žije v~radosti deň čo deň.

Ale všetci by sme mali neustále prebývať v~tej radosti zo vzkriesenia. Jeden zo spôsobom, ako v~tej radosti spočívať je, že sa k~tomu budeme stále povzbudzovať a si tú pravdu pripomínať. Hovoriť o~zmŕtvychvstaní raz do roka je málo. Rýchlo zabudneme, čo to v~našom živote spôsobilo. Potrebujeme mať okolo seba ľudí, ktorí nám budú neustále pripomínať, že Ježiš žije a že aj my sme vďaka tomu vzkriesení k~novému životu. Už nie sme otrokmi hriechu, ale vyslobodení spod Zákona, a teraz z~Kristovej milosti žijeme život viery. V~každom bežnom rozhovore by sme mali o~tom hovoriť. Možno vždy trochu inak, aby nám to nezovšednelo.
Pavol sa veľmi tešil na návštevu Rimanov. Páči sa mi, ako o~tom píše v~R~1,11~--~12: {\it „Veď vás túžim vidieť, aby som vám odovzdal duchovný dar, aby ste ním boli posilnení,} {\bi alebo lepšie: aby sme sa navzájom vašou a mojou vierou povzbudili.“} O~to ide! Je oveľa lepšie, keď sa navzájom vierou povzbudzujeme. Ja potrebujem tvoju vieru a ty moju. Keď sa spolu stretneme a pripomenieme si vzkriesenie, dúfam, že od radosti budeme tancovať okolo kuchyne -- tak ako Izrael na druhej strane Červeného mora. Nech je tento rok rokom tancom slobody. Veľká noc nám dala na to dôvod.

\autor{Danny Jones}


\clanok {Mosty k~lidem}
V aprílovom Spravodaji som informoval o~plánovanom evanjelizačnom tréningu {\it Mosty k~lidem}, ktorý zastrešuje Evanjelická cirkev metodistická. Je to výborná príležitosť, ako si osvojiť evanjelizačné zručnosti -- osloviť neznámych, neveriacich, pozvať ich na neformálne čítanie Biblie, kde sa môžu stretnúť so živým Pánom Ježišom a poznávať Ho osobne, pohotovo povedať svoj príbeh alebo Boží plán spasenia. Prihlásiť sa treba už do {\bf 29. apríla 2019}, najlepšie elektronicky cez stránku \ulink[https://www.facebook.com/events/297430134487659/]{https://www.facebook.com/events/297430134487659/}.

\vskip-1ex\begitems
* Prvá časť tréningu: 11. -- 12. mája v~čase 14.00 -- 18.30 h, Klub Apollo (Dúhový salón)
* Druhá časť tréningu: 1. -- 2. júna v~čase 14.00 -- 18.30 h (miesto bude upresnené)
\enditems

Modlím sa za to, aby od nás na tento tréning išlo 25 ľudí. Modli sa za to so mnou! Verím, že je to dôležité pre rast nášho zboru. Takisto chcem pripomenúť výzvu, aby si sa modlil za konkrétneho človeka alebo konkrétnych ľudí. Maj zoznam týchto ľudí stále so sebou, aby si sa za nich nezabudol modliť.

Teším sa na to, čo bude náš Pán konať!

\autor{Danny Jones}


\clanok {Zborové členské zhromaždenie}
Všetkých členov cirkevného zboru BJB Palisády pozývame na zborové členské zhromaždenie, ktoré sa bude konať v~nedeľu 5. mája hneď po dopoludňajšom zhromaždení.

Program:
\vskip-1ex\begitems
1. Prejednanie uzáverov Diskusnej konferencie delegátov zborov

2. Rôzne
\enditems


\clanok{Spoločné modlitby}
\vskip-1ex\begitems
* Muži -- streda {\bf od 6.00~hod. do 7.00~hod.}, kostol na Palisádach
* Ženy -- pondelok {\bf od 17.00~hod.}, Zrínskeho 2
\enditems

Priveďte na spoločné modlitby aj svojich priateľov a známych, ktorým leží na srdci naše mesto a ľudia v~ňom.


\clanok{Verše na zapamätanie}
Na mesiac máj máme nový veršík, ktorý sa chceme spoločne učiť. Veríme, že poznanie Písma prospeje našej duši i našej mysli:

{\it „Môj nárek si zmenil na tanec. Vyzliekol si mi smútočný šat a odel si ma radosťou, aby nikdy neutíchla oslavná pieseň pre Teba. Hospodin, môj Bože, navždy Ťa chcem velebiť.“}

\autor{Ž~30,~12~--~13}


\clanok{Stretnutia sestier}
Májové stretnutia sestier sa uskutočnia  {\bf 8.~a~22.~mája o~17.30~hod.} v~modlitebni na Palisádach.

Ženy všetkých vekových kategórií sú srdečne vítané!


\clanok{Senior klub v~máji}
Ak dá Pán zdravia a života, v~mesiaci máj sa stretneme {\bf posledný štvrtok, t.j.~dňa 30.~mája~2019 na Súľovskej ul. od 10.00~hod. do 14.00~hod.} Téma: Moc slova.

Všetci sú srdečne vítaní!

\autor{Jana Makovíniová}


\clanok{Služba ľuďom bez domova}
Hľadáme dobrovoľníkov na varenie polievky pre ľudí v~núdzi na nasledovné termíny: 18.~júna, 16.~júla a 20.~augusta (všetky termíny sú utorkové). Po polievku príde výdajový tím o~19.00~hod.

Máme takisto nedostatok ľudí vo výdajových tímoch. Ak by ste boli ochotní pomôcť a zapojiť sa do výdaja teplej polievky, veľmi nám to pomôže.

Dobrovoľníci sa môžu hlásiť u~sestry Beaty Bogárovej.


\clanok {Prosba o~poskytnutie nocľahu}
Brat Pavol Šinko sa obracia na náš zbor s~prosbou o~poskytnutie nocľahu na jednu noc v~sobotu 18. mája pre medzinárodných delegátov konferencie {\it International Needs}, ktorá sa bude konať v~Račkovej doline. Avšak z~niektorých krajín musia delegáti pricestovať už o~deň skôr (konferencia začne v~nedeľu 19.~mája). Ide približne o~osem ľudí (manželských párov aj jednotlivcov) z~rôznych krajín sveta. V~tom čase budú v~Bratislave prebiehať MS v~ľadovom hokeji a ubytovacie kapacity v~meste sú plne vyťažené resp. niekoľkonásobne predražené. Ak by ste boli ochotní niekomu poskytnúť nocľah, dajte vedieť Petrovi Kolárovskému, prípadne sa ohláste cez e-mail: \email{pkolarovsky@bjbpalisady.sk}.


\clanok{Zbierky za uplynulé obdobie}
Milí bratia a sestry, ďakujeme za vašu obetavosť. V~uplynulom období ste prispeli:
\vskip-1ex\begitems
* investičný fond: 498 € (marec)
* misia: 551 € (apríl)
\enditems

\vskip1ex

\n 1. 5.	Milica	MALÁ;
\n 1. 5.	Andrea	ČURILLOVÁ;
\n 3. 5.	Dárius	KRÁĽ;
\n 4. 5.	Peter	BUZÁŠ, ml.;
\n 8. 5.	Vladimír	KRAJČI;
\n 8. 5.	Jana	ŠEBO;
\n 11. 5.	Želmíra	PRAŽENICOVÁ;
\n 16. 5.	Mária	ŠEĎOVÁ;
\n 16. 5.	Ján	SZŐLLŐS;
\n 17. 5.	Lenka	KOVÁČOVÁ;
\n 17. 5.	Lívia	KOLÁŘIKOVÁ;
\n 18. 5.	Anna	DANTEROVÁ;
\n 19.	5.	Oľga	VALCHÁŘOVÁ;
\n 20.	5.	Rastislav	PAULEN;
\n 26.	5.	Radovan	HOVORKA;
\n 30.	5.	Miluška	BAŽALOVÁ;
\narodeniny


\program{
\p 1  ; st ;  6.00 ; Modlitby -- muži (kostol Palisády) ;.;;
\p 2  ; št ; 19.00 ; Biblická hodina (J. Szőllős, Zrínskeho 2) ;.;;
\p 3  ; pi ;.;;.;;
\p 4  ; so ; 18.00 ; Mládež (Súľovská 2) ;.;;
\p 5  ; ne ;  9.30 ; Bohoslužby (D. Jones); 10.00 ; Chvojnica (V. Ira) ;
\p 6  ; po ; 17.00 ; Modlitby -- ženy (Zrínskeho 2) ;.;;
\p 7  ; ut ; 15.15 ; Stretnutie pri Biblii (P. Pivka, Zrínskeho 2) ;.;;
\p 8  ; st ;  6.00 ; Modlitby -- muži (kostol Palisády) ; 17.30 ; Stretnutie sestier ;
\p 9  ; št ; 19.00 ; Biblická hodina (J. Szőllős, Zrínskeho 2) ;.;;
\p 10 ; pi ;.;;.;;
\p 11 ; so ; 18.00 ; Mládež (Súľovská 2) ;.;;
\p 12 ; ne ;  9.30 ; Bohoslužby (Dawson Jones) ; 10.00 ; Chvojnica (J. Szőllős) ;
\p 13 ; po ; 17.00 ; Modlitby -- ženy (Zrínskeho 2) ;.;;
\p 14 ; ut ; 15.15 ; Stretnutie pri Biblii (P. Pivka, Zrínskeho 2) ;.;;
\p 15 ; st ;  6.00 ; Modlitby -- muži (kostol Palisády) ;.;;
\p 16 ; št ; 19.00 ; Biblická hodina (J. Szőllős, Zrínskeho 2) ;.;;
\p 17 ; pi ;.;;.;;
\p 18 ; so ; 18.00 ; Mládež (Súľovská 2) ;.;;
\p 19 ; ne ;  9.30 ; Bohoslužby (R. Nagypal) ; 10.00 ; Chvojnica (P. Kolárovský) ;
\p 20 ; po ; 17.00 ; Modlitby -- ženy (Zrínskeho 2) ;.;;
\p 21 ; ut ; 15.15 ; Stretnutie pri Biblii (P. Pivka, Zrínskeho 2) ;.;;
\p 22 ; st ;  6.00 ; Modlitby -- muži (kostol Palisády) ; 17.30 ; Stretnutie sestier ;
\p 23 ; št ; 19.00 ; Biblická hodina (J. Szőllős, Zrínskeho 2) ;.;;
\p 24 ; pi ;.;;.;;
\p 25 ; so ; 18.00 ; Mládež (Súľovská 2) ;.;;
\p 26 ; ne ;  9.30 ; Bohoslužby (Ľ. Dzuriak) ; 10.00 ; Chvojnica (P. Antalík) ;
\p 27 ; po ; 17.00 ; Modlitby -- ženy (Zrínskeho 2) ;.;;
\p 28 ; ut ; 15.15 ; Stretnutie pri Biblii (P. Pivka, Zrínskeho 2) ;.;;
\p 29 ; st ;  6.00 ; Modlitby -- muži (kostol Palisády) ;.;;
\p 30 ; št ; 10.00 ; Senior klub (Súľovská 2) ; 19.00 ; Biblická hodina (J. Szőllős, Zrínskeho 2) ;
\p 31 ; pi ;.;;.;;
}

\tiraz
\bye
