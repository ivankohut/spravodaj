\input makra.tex % nacitanie Ivanom pripravenych nastaveni a prikazov
\hyphenation{star-šov-stvo} % rozdelenie slov na konci riadku, treba tu uviest slova, ktore sam nepozna

\vyrocnespravy{2017}

\clanok{Zbor}
\cast{Úvod}

Koniec starého a začiatok nového obdobia je čas, kedy väčšinou pozeráme na to, čo bolo, hodnotíme, aké to bolo, a zároveň pozeráme do budúcna možno aj s~túžbou, keby som tak vopred vedel, čo ma čaká.

Pred rokom sme dostali verš: „Kto zvíťazí, tomu dám sedieť so mnou na mojom tróne, tak ako aj ja som zvíťazil a sedím so svojím Otcom na Jeho tróne. Kto má uši, nech počuje, čo Duch hovorí cirkevným zborom!“ Zjavenie Jána 3, 21 -- 22. Víťazstvo vyvoláva predstavu súťaže alebo boja. Denne sa dostávame do situácií, v~ktorých bojujeme. Náš boj nie je s~niečím telesným. Bojujeme s~hriechom a tento boj nemôžeme vyhrať vlastnými silami. Boj s~hriechom môžeme vyhrať iba tak, že sami seba odovzdáme Bohu. Cieľom tohto boja je sedieť na tróne s~Pánom Ježišom Kristom. Znamená to, inak povedané, zvíťaziť -- získať večný život.

V minulom roku sa veľká časť našich snáh, bojov, sústreďovala na voľby kazateľa.  Hľadali sme kandidátov, ktorí by boli ochotní prijať službu kazateľa v~našom zbore a vnímali Božie povolanie do tejto služby. Po dlhom  hľadaní nám Pán Boh daroval kazateľa, ktorý prijal Božie povolanie. Verím tomu, že všetkým členom aj priateľom nášho zboru ležala táto vec na srdci a prinášali ju na svojich modlitbách pred Božiu tvár. Rozhodnutie Dannyho a Clary bolo aj pre nich dlhým duchovným zápasom. Vnímam, že sme vďační za kazateľa, ktorého sme dostali. Jeden z~duchovných bojov, ktoré sme bojovali, sme v~Božej sile a z~Jeho milosti mohli vyhrať.

Počas obdobia bez kazateľa sme sa museli, dúfam, že aj chceli, viac ako po iné roky zapojiť do práce v~zbore. To viditeľné sme skonštatovali už na záver roku. V minulom roku stálo za našou kazateľňou veľké množstvo slúžiacich. Bol to dôsledok toho, že sme nemali zvoleného kazateľa. Brat Dušan Uhrin počas zastupovania kazateľa pracoval, ako keby bol riadne zvoleným kazateľom. Neskôr už bolo nutné, aby sme viacerí priložili ruku k~službe slovom. Nejedná sa iba o~Bratislavu, ale aj o~službu na Chvojnici. To, čo bolo menej, možno až vôbec viditeľné bola práca v~„zákulisí“.  Správcovstvo zboru súvisí aj s~chodom kancelárie. Až bez brata kazateľa Janka Szőllősa sme videli, koľko práce urobil bez toho, aby o~nej rozprával. Preto, aby sme nahradili Jankove nasadenie, prácu v~kancelárii zboru prevzala Miriam Kešjarová. Brat kazateľ Janko Szőllős je pre nás aj naďalej veľkou pomocou. Biblické hodiny sú plne v~jeho réžii.

Na jednom zborovom členskom zhromaždení zaznelo „Nebojme sa byť bez kazateľa!“.  Tento rok som zažil jednu úžasnú vec. Neviem, koľkí z~nás sa báli byť bez kazateľa a koľkí nie. Tento rok mnohí z~nás bez toho, aby videli alebo vedeli, čo je pred nami, s~vierou vykročili a urobili krok viery. Krok viery v~službe tým druhým. Všetkým nám.

Ďakujem nášmu nebeskému Otcovi za Vás všetkých a za to, že v~dohľadnej dobe budeme mať brata kazateľa Dannyho medzi nami.

\cast{Štatistika}

K 1. 1. 2018 mal náš zbor 171 členov, z~toho 8 na Chvojnici. Čo sa týka vekovej štruktúry spoločenstva, približne dve tretiny členov zboru tvoria bratia a sestry do 65 rokov. Hoci počet riadnych členov zboru mierne klesol (zo 174 na 171), pribudli do našej zborovej rodiny noví bratia a sestry:  Kajan Michal, Škodák Vierka a Ondrej, Štefko Ján, Taliga Ladislav a Taligová Elena. O ukončenie členstva v~zbore a preregistráciu do iného zboru požiadala sestra Ružena Uhrinová a manželia Stanislav a Andrea Balážovci.

V uplynulom roku nás opustili viacerí bratia a sestry z~nášho stredu. Do nebeskej vlasti nás predišli: Pribula Juraj, Bednárik Tomáš, Kachaňáková Janka, Vyskočil Kamil, Kvačková Katarína a Dvořáková Ružena.

Rok 2017 bol výnimočný počtom detí, ktoré pribudli do našich rodín: Simon Daniel, Laurenčík Tobiáš, Kováčová Nora, Kamocsai Gabriel, Maďar Michal, Kohút Ondrej, Syčová Eunika a začiatkom roku 2018 sa narodila do rodiny Pribulovcov dcérka Abigail. Viaceré z~týchto bábätiek sme už požehnávali a privítali sme ich v~našom strede.

Svoje „Áno” na spoločnú cestu životom si v~uplynulom roku povedali 3 páry: Danka a Martin Pelíškovci, Andrea a Pavol Čurillovci (sobáš mali na Chvojnici) a Janka a Štefan Šebovci. Aj minulý rok sme priestory našej modlitebne prenajímali na sobášne zhromaždenia iných kresťanských cirkví.

Priemerná návštevnosť nedeľných bohoslužieb sa pohybuje okolo 120 – 130 návštevníkov. Viacerí seniori nemôžu prichádzať na bohoslužby, nakoľko im to nedovoľuje zdravotný stav. Starostlivosť o~týchto bratov a sestry zabezpečujú diakoni a sestry, ktorí ich pravidelne navštevujú. V posledných mesiacoch je možnosť sledovať bohoslužby cez internet, čo oceňujú nielen starší bratia a sestry. Do nášho spoločenstva prichádza viacero nových ľudí, preto je nevyhnutné, aby sme si ich všímali a vybudovali sme si s~nimi priateľstvá. Len tak môže naše spoločenstvo rásť a priťahovať ďalších.

\autor{Peter Pribula a Miriam Kešjarová}

\clanok{Staršovstvo}
Staršovstvo dostalo na rok 2017 verš  zo Žalmu 46, 2: „Boh nám je útočišťom a silou, pomocou v~súžení vždy osvedčenou.” Hovorili sme o~tom, že máme mať starosť o~zborové spoločenstvo a potrebujeme to vedieť uvaliť na Hospodina. Našou úlohou je služba zboru.  Sme v~prvej línii, preto na nás útočí satan a potrebujeme útočisko a ochranu, kde budeme chránení. Musíme spoliehať na to, že Boh nám je útočiskom. Na základe prísľubu tohto verša sme vedeli, že sa vo všetkom môžeme spoľahnúť na nášho Pána. Z ľudského pohľadu sa niektoré situácie, ktorým sme sa venovali, mohli zdať bezvýchodiskové. Na základe prísľubu z~Božieho slova sme sa spoľahli na to, že aj v~tých najťažších otázkach môžeme počítať s~Jeho vedením, múdrosťou a spoľahli sme sa na Jeho vôľu.

Staršovstvo zboru pracovalo v~roku 2017 v~zložení Peter Antalík, Radovan Hovorka, Vladimír Ira, Pavel Kohút, Peter Kolárovský, Miroslav Kolářik, Peter Pribula ako zvolení členovia a Dušan Uhrin ako zvolený správca zboru.

Hlavné témy, ktorým sme sa venovali boli:
\begitems
* príprava a návrh rozpočtu zboru,
* zabezpečenie služieb v~zbore v~období bez kazateľa zboru,
* v rámci Národného týždňa manželstva sme pripravovali program a aktivity zamerané pre manželské páry,
* rozprávali sme s~viacerými záujemcami o~členstvo v~zbore, na zborových členských zhromaždeniach sme viacerých z~nich aj prijali za členov,
* s bratom kazateľom Benjamínom Uhrinom rozprávali o~našich predstavách o~fungovaní Rady BJB a možných zmenách v~štruktúre vedenia BJB. V súvislosti s~tým sme navrhovali kandidátov do Rady BJB a komisií.
* viackrát sme sa venovali pastoračným otázkam a aj otázkam niektorých vzťahov v~zbore,
* veľkú časť roka sme sa venovali voľbám kazateľa zboru,
* práca na Chvojnici, budúcnosť našej zborovej stanice a celková situácia bola a stále je jednou z~dôležitých tém,
* radostnou udalosťou, plnou vďačnosti a nádeje do ďalších rokov bola pripomienka 130-teho výročia založenia zboru.
\enditems

Veľmi radostnou udalosťou zboru bolo zvolenie Dannyho C. Jonesa za kazateľa. V súčinnosti sa venujeme príprave príchodu Jonesovcov na Slovensko.

Sme vďační nášmu nebeskému Otcovi za Jeho vedenie, múdrosť a požehnanie pri všetkom, čomu sme sa počas roku 2017 venovali.

\autor{Peter Pribula}

\clanok{Hospodársky výbor}

Hneď začiatkom roka mráz preveril šikovnosť nášho brata Daniela, ktorý na Chvojnici odstraňoval jeho škody. Zamrzol rozvod vody a potrhal viaceré ventily, ktoré bolo treba vymeniť. Po tejto skúsenosti sme sa rozhodli, že našu chalupu nebudeme používať v~zimnom období počas obdobia výrazných mrazov.

V jarných mesiacoch sme zahájili práce na dokončení rekonštrukcie interiéru kostolíka na Chvojnici. Firma Chrenka vykonala položenie podlahy vo veľkej sále (izolácia, dlažba) a vykonali druhý náter stien. Ostatné práce sme realizovali svojpomocne. Položili podlahové kúrenie,  dokončili zapojenie el. rozvodov a  pustili sa do opravy náterov okien,  dverí, lavíc, galérie. Jednou z~posledných robôt bolo privezenie mobiliáru a jeho montáž. V exteriéri sme opravili WC, vrátane prístupu k~nim. Pred Letnicami sa pridali k~prácam mládežníci, ktorí poupratovali chalupu a stodolu a vonku pokosili jej okolie ako aj okolie kostolíka.
Na Letnice sme mohli ďakovať Bohu.

V letných mesiacoch boli prevedené bežné prace na údržbe kostola na Palisádach. Bola vykonaná obnova náterov okien z~vonkajšej strany, oprava nášľapnej platne pred vchodom, oprava oderov, obnova náterov zadnej steny v~sále, predsiene a schodiska na galériu. Bola vykonaná aretácia lavicových dosiek a v~exteriéri vyklčované kríky okolo obvodových múrov kostola.

V jesenných mesiacoch bola dokončená oprava plotu na Zrínskeho. Bol prevedený náter plotových blokov a oprava betónového sokla.

Začiatkom leta sa pokazil projektor na Palisádach. Oprava by bola nerentabilná a z~toho dôvodu bola vykonaná výmena za nový s~podobnými parametrami. Zvýšila sa jeho svietivosť a pribudla možnosť posielať dáta na projektor z~akéhokoľvek zariadenia prostredníctvom Wi-Fi pripojenia.

Špeciálna vďaka patrí bratovi Jarovi Bánovi, ktorý zabezpečil na vlastné náklady a sprevádzkoval online video vysielanie z~našich bohoslužieb cez Youtube.

Vďaka patrí nášmu bratovi Danielovi Mikletičovi, ktorý vedie hospodársky výbor v~zložení Ľ. Kešjar, M. Kolářik, M. Maďar,  Ľ. Syč, J. Szőllős. Podieľal sa na väčšine prác po koordinačnej a zároveň aj výkonnej stránke. Vďaka patrí aj všetkým ostatným bratom a sestrám, ktorí priložili ruky k~dielu, pomohli s~mnohými prácami, ktoré potrebujeme vykonávať pre praktické fungovanie zboru.

Vďaka patrí predovšetkým nášmu Pánovi, že nám dal silu a chuť pre prácu.

\autor{Ľuboš Kešjar}

\clanok{Diakonia}
Verš na rok 2017 -- Matúš 6, 19 -- 20: „Nezhromažďujte si poklady na zemi, kde ich moľ a hrdza ničí a kde sa zlodeji vlamujú a kradnú. Ale zhromažďujte si poklady v~nebi, kde ich ani moľ ani hrdza neničí a kde sa zlodeji nevlamujú a nekradnú."

\cast{Pracovné stretnutia}

Tím diakonov sa v~roku 2017 pravidelne stretával na pracovných stretnutiach raz mesačne na Zrínskeho ulici v~kancelárii zboru. (Zápisnice z~pracovných stretnutí boli pravidelne zaslané všetkým členom e-mailom, prípadne osobne odovzdané.) Pozvánky na pracovné stretnutia pre členov tímu diakonov boli zasielané e-mailom spravidla tri dni vopred. Taktiež boli vyhlasované v~oznamoch v~rámci  nedeľného  zhromaždenia.

\def\aktivita#1{{\it #1\par}\firstnoindent}
\cast{I. Vnútrozborové aktivity}

\begitems \style n
* \aktivita{Návštevná služba}
Pravidelne pokračovala návšteva našich imobilných členov v~domácnostiach, ktorú vykonávajú jednotlivé sestry a bratia, ktorí majú s~navštevovanými bratmi a sestrami vytvorený niekoľkoročný blízky vzťah.  Chcel by som aj menovite spomenúť aspoň niektorých  členov tímu diakonov, ktorí pravidelne navštevovali našich imobilných seniorov. Sú to sestry Lenka Gubová, Vladka Laurenčíková, Juditka Kolářiková, Gitka Kráľová a manželia Valentovci. Taktiež sa návštev zúčastňovali aj bratia kazatelia J.~Szőllős, D.~Uhrin a brat diakon P.~Pivka (väčšinou pri vysluhovaní VP).
Okrem našich seniorov boli navštevovaní aj naši nemocní bratia a sestry či už v~domácnostiach alebo v~nemocniciach. Sestry navštevovali aj mladé mamičky s~bábätkami z~nášho zboru.

* \aktivita{Služba núdznym}
Bratovi Romanovi Žiaranovi, sa venuje brat kazateľ D.~Uhrin spolu s~bratmi zo zboru, ktorý majú s~ním blízky vzťah. Tohto roku sme sa venovali na tíme diakonov aj sestre Sépovej, ktorá tiež potrebuje našu pomoc.
Taktiež na decembrovom stretnutí sme rozdelili sociálny fond na „vianočnú výpomoc“ sociálne slabším členom nášho zboru. Ako každý rok.

* \aktivita{Zborové pohostenie}
V nedeľu 28. 5. 2017 sa uskutočnil obed pre nových členov zboru (hlavne pre Slovákov zo Srbska). Na Zrínskeho ulici bolo okolo 30 hostí.

3. 12. 2017 sme mali už tradičný „vianočný obed pre seniorov“. Varil brat Lacko Taliga. Zákusky zabezpečili sestry. V rámci obeda boli odovzdané aj vianočné darčeky našim seniorom.

* \aktivita{Svätodušné sviatky na Chvojnici}
Aj v~roku 2017 sme na Letnice v~nedeľu 4. 6.  boli poslúžiť bratom a sestrám na Chvojnici. Bol to pre mnohých z~nás celodenný zborový výlet, na ktorý nás odviezol už tradične „Barnybus“. Obed varili manželia Taligovci.

* \aktivita{Vysluhovanie Večere Pánovej}
Večera Pánova sa vysluhovala pravidelne každú prvú nedeľu v~mesiaci (podľa rozpisu). Okrem toho sa Večera Pánova vysluhovala aj v~domácnostiach. Brat V.~Krajčí nás informoval, že potrebujeme nových mladých bratov -- služobníkov k~vysluhovaniu VP.

* \aktivita{Slávnosť 130. výročia nášho zboru}
15. októbra sme mali 130. narodeniny nášho zboru. Táto slávnosť nahradila tradičný zborový deň, ktorý pravidelne v~našom zbore organizujeme. V hoteli Tatra sme mali objednaný slávnostný obed.
\enditems

\cast{II. Aktivity zboru smerom von}

\begitems \style n
* \aktivita{Služba v~domovoch sociálnej starostlivosti}

Okrem služby v~našom zbore sa venujeme aj službe mimo zboru v~domovoch dôchodcov pod vedením brata P.~Pivku za vernej pomoci sestier L.~Gubovej a V.~Laurenčíkovej. Pravidelne navštevujeme „Domovy sociálnej starostlivosti“ v~Starom meste a v~Dúbravke.

* \aktivita{Služba bezdomovcom}
Aj tento rok sme podporovali službu varenia pre bezdomovcov v~rámci spoločenstva Kresťania v~meste. Túto službu koordinujú sestry, t. č. pod vedením sestry B.~Bogárovej.
\enditems

\autor{Pavel Pivka}

\clanok{Biblické a iné vzdelávanie}

Spoločné štúdium Svätého Písma prebiehalo vo viac-menej pravidelnom rytme a nezmeneným spôsobom. Takmer každý utorok popoludní preberal brat kazateľ Pavel Pivka so skupinou záujemcov najmä z~radov seniorov texty z~Božieho slova. V prvom polroku do júna študovali spolu Evanjelium Lukáša a od septembra, po letných prázdninách, Listy apoštola Petra. Počas štvrtkových večerných biblických hodín pokračoval brat kazateľ Ján Szőllős do konca februára vo vykladaní Listu Rímskym a od marca, okrem letnej prestávky v~mesiacoch júl – september sa preberalo Evanjelium Jána. Na každom z~uvedených vzdelávaní, ktoré prebiehajú na Zrínskeho ulici, sa zúčastňovalo v~priemere len okolo 10 bratov a sestier, čo je veľmi malý počet.  Okrem uvedených celozborových príležitostí bolo zamyslenie nad Písmom a diskusia aj súčasťou stretávania viacerých skupiniek. Na skupinkách bol aj priestor na tematické štúdium Písma.

Napriek existencii širokej škály ponúk rôznych typov vzdelávania v~rámci našej cirkvi aj na naddenominačnej úrovni a napriek ponúkanej podpore zo strany zboru sú tieto možnosti málo využívané našimi členmi. Niektorí členovia nášho zboru sa aj minulý rok zúčastnili v~rámci BJB v~januári a v~júni Školy pastorálneho poradenstva, v~novembri Víkendu pre vedúcich mládeží či Duchovných cvičení. Na medzidenominačnej úrovni sa tím pracujúci s~mládežou zúčastnil na Konferencii pre pracovníkov s~mládežou (KPM) v~Žiline. Do kurzov Detskej misie sa mohli zapojiť niektorí jednotlivci spomedzi vedúcich besiedky.
Pokračovali aj stretnutia mamičiek malých detí v~našich priestoroch raz týždenne na Klubíku, čo je veľmi cenené zo strany účastníčok.

\autor{Ján Szőllős}

\clanok{Sestry}

Do nového roku sme vstupovali s~biblickým odkazom J 15,16, ktorý dostali sestry ako novoročný verš: „Nie vy ste si mňa vyvolili, ale ja som si vyvolil vás a ustanovil som vás, aby ste šli a prinášali ovocie a vaše ovocie aby zostávalo, tak aby vám Otec dal, čokoľvek by ste v~mojom mene prosili od Neho.“

{\bf ...aby ste šli a prinášali ovocie (napr. aj navštevovaním chorých):}

Prvé stretnutie sme mali 7. 2. 2017 s~účasťou 17 sestier a na tému „navštevovanie chorých“ nám poslúžila svojimi skúsenosťami a podnetmi sestra Gitka Kráľová. Záverom sme si zaumienili, že by bolo dobré, aby sa k~sestrám (diakonkám), ktoré navštevujú našich chorých a nevládnych bratov a sestry, pripojili občas aj mladšie sestry a nabrali skúsenosti aj v~tejto oblasti služby.

{\bf ...aby vám Otec dal, čokoľvek by ste v~mojom mene prosili od Neho (v modlitbe):}

Prvý marcový piatok (3. 3. 2017) sa Svetového dňa modlitieb zúčastnili žiaľ len štyri sestry z~nášho zboru. Je málo príležitostí, kedy môžeme zotrvávať na modlitbách spolu s~bratmi a sestrami z~iných denominácií (a v~duchu byť spojení aj s~bratmi a sestrami na celom svete) a uvedomiť si, že cirkev tvoríme spolu, že sme údmi jedného tela. Svetový modlitebný deň je zameraný na modlitebné potreby konkrétnej krajiny. Tento rok to bude Surinam a na spoločné modlitby za túto krajinu, ktorá bude na modlitebnom zhromaždení v~prvý marcový piatok aj informačne predstavená, ste všetci srdečne pozvaní.

Prvý aprílový utorok (4. 4. 2017) sme sa opäť stretli na Zrínskeho ulici o~17.00~hod. a povedali sme si pár pohľadov na modlitbu, čím je, aký má význam, aký úžitok z~nej máme.

{\bf ...a vaše ovocie aby zostávalo (na svedectvo ďalším sestrám a ďalším generáciám):}

5. 5. 2017 sa začala trojdňová sesterská konferencia v~atraktívnom meste Karlove Vary v~ČR, ktorej sa zúčastnilo 10 sestier z~nášho zboru. Konferencia sa pre nás začala už na spoločnej ceste autobusom, do ktorého sme pristúpili v~Bratislave k~sestrám z~Banskej Bystrice a Bernolákova. V Karlových Varoch sme prežili požehnané chvíle počúvaním svedectiev, napr. aj o~misijnej práci priamo v~tomto meste, pri prehliadke mesta, ale najmä pri prednáškach o~„múdrosti veriacej ženy v~rodine, zbore a spoločnosti“. 28. 5. 2017 na večernom nedeľnom zhromaždení sme sa o~našich najhlbších dojmoch z~konferencie zdieľali slovom, piesňou, videoprezentáciou. Verím, že táto forma odovzdávania prijatého požehnania bude pokračovať aj tento rok a že sestry pôjdu na tohtoročnú májovú konferenciu do Košíc v~hojnejšom počte.

Medzitým, 21. 5. 2017, som na nedeľnom dopoludňajšom zhromaždení na požiadanie brata kazateľa Dušana Uhrina v~krátkosti predstavila prácu našich sestier.

Na júnovom stretnutí (20. 6. 2017) sme mali tak ako minulý rok jedno tematické stretnutie s~hosťom. Tentoraz prišla z~Banskej Bystrice sestra Mgr. Ivana Ladomerská, veriaca detská psychologička, ktorá poradila najmä mladým mamičkám prostredníctvom prednášky, ale aj otázok a odpovedí, ako viesť v~dnešnej dobe deti k~Bohu, ako ich chrániť pred nástrahami sveta, čo pre to môžu urobiť rodičia a ako sa naučiť zvládať rôzne druhy temperamentu.

{\bf Nie vy ste si mňa vyvolili, ale ja som si vyvolil vás a ustanovil som vás, aby ste šli... (do služby):}

Naše posledné stretnutie sa uskutočnilo 5. 9. 2017, na ktorom sme sa zdieľali s~tým, čo sme prežili s~Pánom počas prázdninových mesiacov. Zároveň som sestrám oznámila moje ukončenie služby vedúcej sestier z~dôvodu, že sa s~manželom sťahujeme od októbra 2017 do Žatca, kde bude Stanko vykonávať službu kazateľa. Verím, že tak ako si nás vyvolil pre konkrétnu službu v~konkrétnom čase, vyvolí si a ustanoví ďalšiu vedúcu sestier v~roku 2018.

S vďačnosťou Pánovi za získané skúsenosti, lásku a spoluprácu sestier

\autor{Andrea Balážová}

\clanok{Mládež}

V roku 2017 sa naša mládež početne nerozrástla, ani pravidelných návštevníkov nie je viac. Sme však radi, že väčšina z~tých, ktorí chodia, chodia pravidelne a majú záujem o~to, čo sa v~mládeži deje a o~to, čo mládež robí.

V dnešnej dobe už nie je samozrejmosť, že si mladý človek spraví v~sobotu poobede čas, aby prišiel na mládež, a to platí aj o~mladých z~kresťanských kruhov, z~nášho zboru a našich rodín. Mladí ľudia majú dnes nespočetné množstvo možností, majú mnoho záujmov a záľub. Možno je to aj na nás, aby sme ich naučili milovať spoločenstvo a vybrať si práve mládež napriek rôznym možnostiam, ktoré sa ponúkajú a sú ponúkané. Nad tým sme sa aj ako výbor častokrát zamýšľali a modlili sme sa za to, aby sme vedeli spraviť naše mládeže zaujímavejšie, príťažlivejšie a lákavejšie. Snažili sme sa a stále sa snažíme vytvoriť si čo najpevnejšie a najlepšie puto a vzťahy s~mládežníkmi najmä z~nášho zboru. Preto sme častokrát, možno aj na úkor kvalitnej témy, radšej zvolili možnosť sedieť pri jednom stole a len sa rozprávať o~tom, čo ľudia prežívajú každý deň. Takéto mládeže sa v~minulom roku vyskytli pomerne často, a pri počte účastníkov od päť do desať ľudí sú takéto rozhovory priam ideálne na to, aby sme sa navzájom spoznali, mali jeden k~druhému bližšie a aby ľudia cítili prijatie a osobný vzťah.

Stále sme vďační za priestory nášho zboru na Chvojnici, kde sme spolu prežili niekoľko víkendov, a napriek silnej zime a nedostatku vody aj decembrová víkendovka mládeže, kde sa k~nám pridali aj niekoľkí bývalí mládežníci, dopadla veľmi dobre. Ďakujem Bohu, že nás sprevádza aj takýmito situáciami a dokáže aj zlé veci obrátiť na dobré.

Počas roka 2017 sme viackrát mali možnosť byť s~mládežníkmi z~iných krajov. Ako každý rok bola aj v~r. 2017 mládežnícka konferencia, ktorej sme sa zúčastnili aj my.

Okrem konferencie sme boli aj na letnom tábore na Muránskej Zdychave spolu s~mládežou z~okolia Revúcej, ale boli tam aj účastníci z~viacerých kútov Slovenska.
Neskôr, začiatkom školského roka, sme dostali pozvanie na oblastné stretnutie do Lučenca, ktoré sa konalo v~októbri 2017, kde sa stretávajú mladí z~Lučenca a okolia, ale aj oblasti južného Slovenska.

Aj na tábore, aj na oblastnom stretnutí sme mali možnosť spoznať nových ľudí, nadviazať kontakt s~ľuďmi z~iných zborov a miest. Sme vďační za prijatie, ktorého sa nám dostalo vo všetkých zboroch a spoločenstvách, ktoré sme navštívili. Verím a teším sa na to, že aj v~budúcnosti táto spolupráca bude pokračovať, a chcem, aby aj naša mládež bola schopná pripraviť program a prijať ľudí či už na tábore alebo pripraviť oblastné stretnutie pre západnú oblasť.

V radoch nášho mládežníckeho výboru sme veľké zmeny nepocítili. Zostali sme štyria: Marta Pribulová ml., momentálne nespolupracuje na sto percent, ale je vždy ochotná pomôcť. Takže zostava je taká, že vedúcim ostáva Peter Antalík a na príprave mládeží a iných akcií sa podieľame ja (Dávid Pribula), Barbora Volentičová a Anička Plett.

Našou túžbou je aj to, aby sa aj rady nášho výboru rozrástli, aby momentálni mládežníci cítili zodpovednosť a potrebu pracovať s~mládežou. Veríme a dúfame, že aj novozvolený kazateľ brat Danny Jones má predstavu a plán pre každú zo zložiek zboru, aj pre mládež. Chceme počuť jeho stanovisko, predstavy a plány, ale predovšetkým nasledovať Božie vedenie, plniť Jeho vôľu a oslavovať Jeho meno.

Prosím, myslite na nás, aby aj naše životy mohli poukazovať na Krista, viesť mládež bližšie k~Bohu a aby sa tak naplnilo aj Božie slovo, ale aj naše motto:  {\it Mládež je živé spoločenstvo v~Ježišovi Kristovi, ktorého cieľom je uctievanie Boha, budovanie sa navzájom a početný rast.}

\autor{Dávid Pribula}

\clanok{Dorast}

So skupinou dorastencov vo veku od 11 do 15 rokov sme sa stretávali každú nedeľu počas bohoslužieb. Preberali sme témy z~materiálu Detskej misie „Božia cesta“. Učili sme sa, aký je Boh, čo je most spásy alebo aké je ovocie Ducha a mnohé ďalšie témy. Práca s~touto knižkou sa nám páčila, lebo všetky tvrdenia sú podložené textami z~Biblie. Núti nás to pracovať s~Bibliami na každom stretnutí -- a to je dobre. Máme k~tomu aktivitu: „tasenie mečov“ -- keď chceme vyhľadať biblický verš, požiadame dorastencov, aby si „pripravili meče“ (Biblie držia nad hlavou.), zadáme verš a potom zavelíme „Útok!“ Súťažia, kto nájde verš najrýchlejšie. Víťaz potom  text nahlas prečíta. Ďalšou aktivitou je „NIT“ -- {\bf n}ajdôležitejšia {\bf i}nformácia {\bf t}ýždňa -- v~rámci nej si hovoríme, čo sme uplynulý týždeň prežili.

Nestretávali sme sa iba počas nedieľ, ale podarila sa nám aj víkendovka na Chvojnici, tábor v~Častej, niekoľko spoločných obedov a samozrejme vianočná kapustnica s~darčekmi, koláčikmi a hrami.

Práca s~dorastencami nás vedúcich veľmi obohacuje. Máme radosť z~ich pozornosti a trpezlivosti pri čítaní textov. Sú otvorení a komunikatívni. Našou túžbou je, aby svoje životy odovzdali Pánovi Ježišovi, ale tiež aby si medzi sebou vytvorili vzťahy.

\autor{Rišo Halamíček, Martin Simon, manželia Čurillovci, manželia Hovorkovci}

\clanok{Besiedka}
\begitems
* \noindent{\bf Cieľ:} Viesť deti za Pánom Ježišom – vyučovaním Božieho slova a osobným príkladom

* \noindent{\bf Vyučovacie materiály:} Materiály vydané Detskou misiou a CD z~Čiech z~organizácie Tim 2,2

* \noindent{\bf Témy: } Každý školský rok sa snažíme striedať starozákonné a novozákonné témy (plus špeciálne lekcie na Vianoce, Veľkú noc alebo ku Dňu matiek).
\enditems
\cast{Štruktúra besiedky}

Deti sa stretávajú na Palisádach a sú prítomné na začiatku bohoslužieb (zvyčajne si pre ne moderátor bohoslužieb pripraví krátky príhovor).
Väčšinou po chválach odchádzajú na Zrínskeho v~doprovode učiteľov besiedky.
Besiedka začína cca o~10.15 hod.

\cast{Program besiedky}
\begitems
* Hry na úvod
* Modlitba (krátke stíšenie)
* Biblická lekcia
* Opakovanie, učenie biblického verša
* Občerstvenie (pripravujú rodičia detí)
* Ručná práca, príp. hry
* Spev (začína o~11.00 hod.) – vedie ho Diana Dzuriaková
* Odchod detí: 11.20 – 11.30 hod. (deti si odvádzajú rodičia)
\enditems
Na malej besiedke sú prítomní dvaja učitelia (učiteľ a pomocník). Na veľkej besiedke učí jeden učiteľ.

\cast{Štatistika}

V šk. roku 2017/2018 máme 47 detí vo veku 0 – 15 rokov:
\vskip-1ex\begitems
* 0 – 3 roky – 12 detí  (Klubík – vedie Lívia Kolářiková – štvrtok o~9.00 hod.)
* 3 – 7 rokov – 18 detí (Predškolská (malá) besiedka)
* 7 – 11 rokov – 8 detí (Veľká besiedka)
* 12 – 15 rokov – 9 detí (Mladší dorast)
\enditems
Tieto počty zahŕňajú pravidelných účastníkov besiedok. K nim nám často prichádzajú aj nepravidelní návštevníci – kamaráti alebo rodinní príslušníci našich detí.

\cast{Mimoriadne stretnutia}

Okrem pravidelných besiedok pripravujeme s~deťmi a dorastencami program na Vianoce a Deň matiek (príležitostne deti slúžia aj pri iných príležitostiach).
Stretávame sa na rodinnom tábore, ktorý pripravujeme pre rodiny s~deťmi už 16~rokov v~Stredisku Detskej misie pri obci Častá (v letných mesiacoch). Cez jarné prázdniny (február, marec) sa niektoré rodiny stretávajú na spoločnej dovolenke v~Račkovej doline.

V minulých rokoch sme organizovali spoločné rodinné obedy (jednu nedeľu za 2 – 3 mesiace). Tento rok ich vystriedali spoločné obedy pre celý zbor (raz za štvrťrok).

\cast{Modlitby za deti}

Uvedomujeme si, že deti sú veľmi krehké nádoby. Snažíme sa ich niesť na modlitbách – na začiatku šk. roka deti rozdelíme medzi učiteľov a myslíme na ne (a~ich rodiny) v~priebehu šk. roka.

\cast{Učitelia v~r. 2017}

\begitems
* malá besiedka: Hovorková Mirka, Kešjarová Mirka, Kešjarová Kristína, Máťušová Janka, Vulić Rada
* pomocník: Máťuš Lukáš
* veľká besiedka: Šandorová Hanka, Volentičová Barbora, Kováčik Janko, Kráľová Ľubka, Matušek Matej
\enditems
\autor{Miriam Kešjarová}

\clanok{Klubík}

Stretnutia pre mamičky s~detičkami pokračovali aj druhý rok od vzniku Klubíku a stretávali sme sa v~obvyklom čase, vo štvrtok doobeda, na Zrínskeho. V minulom školskom roku sme boli v~menšom počte ako rok predtým, no osobne som mohla aj napriek tomu nabrať mnoho inšpirácie od ostatných mamičiek a verím, že to bol požehnaný čas. Chcem sa poďakovať Lynn Plettovej za jej obetavú a vytrvalú prácu, ktorú pre nás robí, a každý týždeň je pripravená s~novým programom pre detičky. Aj tento rok sme s~mamičkami pokračovali v~preberaní knižky „Moc modlitieb ženy“. Som vďačná Pánu Bohu za to, že nám dáva podmienky, aby sme sa mohli takto stretávať a požehnávať sa navzájom. Chcem pozvať všetky mamičky, ktoré majú chuť a čas, aby sa k~nám pripojili a zažili vzácny čas v~kruhu priateliek a sestier v~Pánovi.

\autor{Lívia Kolářiková}

\clanok{Hudobné večery}

V uplynulom roku bolo hudobných večerov pomenej. Keď skúmam archív, vychádza mi, že prvý sa uskutočnil v~apríli a hral Juraj Alexander (violončelo) s~Filipom Jarom (kontrabas). Bolo to výborné. Májový koncert moderoval namiesto mňa brat kazateľ Dušan Uhrin (vďaka!) a o~hudobnú produkciu sa postaral Ensemble Moscheles – zdvorilo sme odmietli záštitu koncertu poslancom NR SR Marekom Krajčím. Dôvodom bolo, že ako organizátori hudobných večerov nepoznáme politickú orientáciu všetkých našich hudobných hostí (ani nám na nej, myslím, veľmi nezáleží – ide o~peknú hudbu) a nepovažujem za korektné spájať naše produkcie s~politikmi, či politickými stranami. Koncert sa nakoniec uskutočnil ku Dňu matiek a počul som naň len pozitívne referencie. Jarnú dramaturgiu sme zakončili koncertom čerstvej doktorky umenia Barbory Gálovej (flauta). Zhodou okolností jesennú dramaturgiu uviedla tiež flautistka Tünde Jakab a na klavíri ju sprevádzal Peter Valentovič. Koncert bol na špičkovej úrovni. Posledný večer cyklu patril dvojici s~názvom Duo Altera (Belinda Sandiová, Zuzka Bednárová) – mal krásnu atmosféru.

Hodnotenie koncertov pojmem z~hľadiska hudobného a misijného. K tomu prvému, myslím si, že hudobné večery prinášali už niekoľko rokov po sebe koncerty vynikajúcich slovenských umelcov. Túto úroveň sa nám darilo zachovať, povesť našich koncertov rástla (aj vďaka propagácii Zuzky Godárovej) a častejšie sa stávalo, že nás kontaktovali interpreti so záujmom o~hranie, než sme my museli kontaktovať hudobníkov.

Z hľadiska misie sme usporadúvanie koncertov definovali ako priestor vytvorený cirkvou pre osobnú evanjelizáciu. Hudobné večery nie sú primárne evanjelizačné akcie, ale ponúkajú priestor, aby členovia cirkevného zboru pozývali svojich priateľov, či známych do cirkevných priestorov. Môžu tak napr. odprezentovať, kam v~nedeľu chodia a hlavne prečo. Osobne nedokážem povedať, nakoľko sme v~tomto úspešní. Určitou vizitkou misijného úsilia je tiež to, aké publikum dokážeme ako zbor vytvoriť umeleckým hosťom – spravidla býva najlepšia návštevnosť pred blížiacimi sa koncertami spevokolu, a tak si občas pomyslím, či by nebolo fajn presunúť veľkonočný koncert spevokolu na jún. Každopádne, koncerty mali už svojich stálych návštevníkov aj z~mimozborového prostredia, dokonca sa uskutočnili koncerty, kde bolo týchto ľudí viac ako cirkevníkov. Ak sa k~organizovaniu večerov vrátime, bude namieste sa týmto stavom zaoberať a prebrať evanjelizačné možnosti – možno aspoň vo forme evanjelizačných letákov v~laviciach, alebo opätovného otvárania a ukončovania koncertov spevokolom. Myslím si, že aj vďaka týmto koncertom sa dá povedať, že BJB Palisády sú súčasťou komunity tvoriacej (kvalitnú) kultúru hlavného mesta.

Aby som nás len nechválil, musím si priznať, že tomu, aby boli koncerty ešte kvalitnejšie, bráni moja lenivosť usilovať sa o~granty pre umelecké podujatia. Mám tiež nápady ohľadom založenia občianskeho združenia – tak by bolo asi ľahšie tie granty získať a prípadnej spolupráce s~kresťanskými médiami. Mimochodom, zopár sme ich na koncertoch už mali (nie kresťanských). Pozvali si ich hosťujúci umelci.

Na organizovaní koncertu sa najčastejšie podieľajú nasledovní ochotníci: Roberta Krmášková – tvorba plagátov, Peter Žembery – roznášanie plagátov, Ľubo Kešjar, Daniel Plett, Marcel Maďar – zvukári, s~moderovaním občas pomáha Šimon Hovorka a Dušan Uhrin. Patrí im moja veľká vďaka (aj tým, ktorých som zabudol menovať).

Ak sa majú hudobné večery začať znova organizovať, budem potrebovať stálu moderátorskú pomoc, s~ktorou si to podelíme fifty-fifty. Produkciu večerov som od novembra pozastavil, pretože som sa rozhodol dať prednosť svojej rodine, s~ktorou trávim málo času.

\autor{Laco Kamocsai}

\clanok{Matuzalem}

Matuzalemci sme  aj tohto roku boli pozývaní a boli sme zakaždým radi, keď sme mohli na pozvanie, hlavne kvôli zdravotnej spôsobilosti, odpovedať kladne. Preto sa poslednú dobu ani sami pozývateľom nepripomíname a nehľadáme možnosti na službu, lebo v~nedávnom čase sme práve zo zdravotných dôvodov museli niekoľko koncertov zrušiť.

Nakoľko nás zo začiatku roka nikto nepozýval, okrem občasného spievania na domácej pôde, sme sa prvý polrok mohli venovať utužovaniu zdravia. No ku koncu školského roka sme naraz dostali pozvania hneď štyri. V máji sme išli do zboru ECM v~Seredi a v~júni sme dostali zaujímavú ponuku z~Rádia~7. Chceli, aby sme urobili pre nich koncert, ale v~našej modlitebni na Palisádach. Hneď týždeň nato sme absolvovali náročnú cestu do západných Čiech do nášho zboru v~Žatci. Aj keď v~ostatnom čase v~Česku ľudia všeobecne prestávajú rozumieť slovenčine, v~Žatci nám nielen dobre rozumeli, ale ako prezradili, doslova slovenčinu milujú.

Začiatkom júla sme išli do nášho zboru na juh Slovenska do Svätého Petra. Ich kazateľ brat Bálint Döczé mi povedal, že u nich sú bohoslužby celý rok v~maďarčine, len na Petra a Pavla v~slovenčine. Preto u nich spievať môžeme len raz do roka.

Ďakujeme Pánu Bohu za možnosť slúžiť a prosíme o~ochotné a čisté srdcia.

\autor{Slávo Kráľ}

\clanok{Spevokol}

Sme vďační Pánu Bohu za možnosť spevom oslavovať nášho Stvoriteľa. Ďakujeme za každý nový hlas, ktorý je ochotný sa do tejto náročnej služby zapojiť.
Ako každý rok, tak aj tento sme začali službou na Novoročnom koncerte v~evanjelickom kostole v~Petržalke. Pravidelne sa tam stretávame s~viacerými spevokolmi i skupinami z~Bratislavy a blízkeho okolia.

V tomto roku sme sa snažili viacej ako v~minulosti slúžiť na domácej pôde počas  nedeľných bohoslužieb.  No hlavný dôraz našej služby vedome prikladáme na naše koncerty pre širokú verejnosť. Vianoce a Veľká noc sú vynikajúcou príležitosťou osloviť aj takých ľudí, ktorí by k~nám inokedy neprišli. Účasť cudzích ľudí je taká vysoká, že našich členov medzi nimi treba hľadať.  Nakoľko v~predchádzajúcom roku na vianočný koncert prišlo tak veľa ľudí, že asi päťdesiati sa nedostali ani do sály, rozhodli sme sa urobiť koncerty dva. Keďže na obidvoch bola miestnosť úplne zaplnená, rozhodli sme sa tak urobiť aj v~tomto roku. Nechodíme evanjelizovať do mesta, preto vítame aj túto možnosť evanjelizácie nášmu mestu.

Aby sme sa na koncerty čo najlepšie pripravili, snažíme sa urobiť generálky týždeň pred našimi termínmi, niekde mimo nášho zboru. Tento rok nás pozvali s~veľkonočným koncertom do maďarského baptistického zhromaždenia do obce Albertirsa. U nás potom bol koncert v~nedeľu pred Veľkou nocou.
Na generálku vianočného koncertu sme išli dva týždne pred našimi termínmi na Palisádach do nášho zboru v~Budapešti.

Uvedomujeme si dôležitosť tejto služby, ktorá má dosah nielen na poslucháčov, ale aj na nás samotných, a preto prosíme o~silu a požehnanie na túto prácu.

\autor{Slávo Kráľ}

\clanok{Služba ľuďom bez domova}

V spolupráci s~občianskym združením Kresťania v~meste (ďalej ako „KvM“) sa náš zbor aj v~roku 2017 zapojil do pomoci ľuďom bez domova, či už varením polievok, službou vo výdajových tímoch alebo dobrovoľnými finančnými darmi na nákup surovín na polievku. Jedlo sa vydávalo  pod mostom Lafranconi počas chladných mesiacov (október až marec) hromadne vždy od 19.30 hod. každý utorok a štvrtok a od 17.00 hod v~sobotu. Výdaj počas teplejších mesiacov (apríl až september) bol 2x do týždňa od 19.30 hod. (utorok, štvrtok).

Vo výdajovom tíme máme Štefku a Mira Antalíkovcov.

Náš zbor slúžil počas uplynulého roka varením polievok každý tretí utorok v~mesiaci okrem mesiaca júl. Dokopy sme za uplynulý rok varili 14x – okrem  dohodnutých mesačných termínov aj navyše, a to v~mesiaci november 2x a v~mesiaci december 3x.

Do prípravy a varenia polievok sa tento rok zapojili bratia a sestry: Vladina a Janko Laurenčíkovci, Marta a Peter Pribulovci, Jana Zajacová a Kamila Zajíčková, Laco a Elenka Taligovci, Rút Bednáriková, Marta Račičová, Jozef Doba, Danka Kotmanová, Helenka Uhrinová z~Komárna, Tomáš Bednárik, Slávka Volentičová s~kamarátkou Majkou, Mária, Jarka a David Lomenovci a Beata Bogárová.

Dobrovoľné finančné dary od darcov z~nášho zboru na nákup surovín na polievky boli v~roku 2017 celkovo vo výške 315~€. Zostatok z~roku 2016 bol 129,98~€.    V r. 2017 sa na varenie polievok použilo 214,50~€. Zostatok do ďalšieho roku je 145,48~€.

Zo srdca ďakujem všetkým Vám, ktorí ste ochotne a s~láskou darovali svoj čas, financie, schopnosti, poskytli ste svoje autá do tejto užitočnej služby núdznym,  spolupodieľali ste sa na prípravách polievok, či už nákupom surovín, krájaním zeleniny a mäsa, či samotným varením. Vďaka Pánu Bohu za Vás! Verím, že sa ešte mnohí z~nášho zboru zapoja, či už do varenia polievky alebo inak.

\autor{Beata Bogárová}

\clanok{Anonymní Alkoholici}

{\bf Rebélia a prijatie}

Každý z~nás prechádza obdobím, keď sa môžeme jedine úpenlivo modliť. Niekedy však zájdeme aj ďalej. Zmocní sa nás taká odpudzujúca rebélia, že sa jednoducho ani nemodlíme. Keď sa to stane, nemali by sme zmýšľať o~sebe príliš zle. Jednoducho by sme sa mali čo najskôr vrátiť k~modlitbe a robiť to, čo považujeme pre seba za dobré.

Človek, ktorý zotrváva v~modlitbe, zisťuje, že vlastní úžasné dary. Keď musí riešiť ťažké okolnosti, zisťuje, že im dokáže čeliť. Dokáže prijať sám seba i svet okolo seba.

Dokáže to robiť, lebo teraz prijíma Boha, ktorý je Všetkým – a miluje všetkých. Keď povie: „Otče náš, ktorý si v~nebesiach, nech sa posvätí Tvoje meno,“ myslí to úprimne a pokorne. Keď medituje oslobodený od hluku sveta, vie, že je v~Božích rukách a že jeho konečný cieľ je naozaj istý, teraz alebo potom, nech sa stane čokoľvek.

\autor{Citované z~knižky „Ako to vidí Bill“}

Anonymní Alkoholici (AA) sú spoločenstvo mužov a žien, ktorí sa navzájom delia o~svoje skúsenosti, silu a nádej, aby mohli riešiť svoj spoločný problém a pomáhať ostatným uzdraviť sa z~alkoholizmu. Jedinou podmienkou členstva je túžba prestať piť... Naším prvoradým cieľom je zostať triezvym a pomáhať k~triezvosti ostatným alkoholikom.

Naša skupina AA máva stretnutie každý štvrtok o~18.30 hod. na Zrínskeho. Máš problém s~alkoholom? Príď medzi nás.

\autor{Želka Praženicová}

\clanok{Revízia hospodárenia}

Revízna komisia v~zložení Miroslav Antalík, Helena Mikletičová, Katarína Valentová za prítomnosti účtovníčky zboru Ľubky Kohútovej vykonala revíziu hospodárenia za rok 2017. Boli prekontrolované nasledovné doklady:
\begitems \style -
* výpisy z~bežného účtu vedeného v~Sl. sporiteľni za mesiace 1, 3, 5, 7, 9 a 11
* výdavkové pokladničné doklady za mesiace 2, 4, 6, 8, 10 a 12
* príjmové pokladničné doklady za mesiace 2, 4, 6, 8, 10 a 12
\enditems
Revízna komisia konštatuje, že uvedené doklady sú vedené prehľadne v~súlade s~účtovnými predpismi. Pokladničná kniha je vedená mesačne a založená priamo pri pokladničných dokladoch.

Neboli zistené žiadne nedostatky.

Stav finančnej hotovosti ku dňu 31. 12. 2017 bol:

\vskip1em\hskip1cm\table{lr}{
pokladňa & 7~978,23~€ \cr
bankový účet & 23~846,42~€ \crl
spolu & 31~824,65~€ \cr
}\vskip1em

Tento stav súhlasí so stavom v~účtovnej evidencii k~uvedenému dátumu.

\autor{Katarína Valentová}

\tiraz
\bye
