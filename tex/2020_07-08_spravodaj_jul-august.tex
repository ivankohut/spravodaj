%\typosize[9/12]% - pouzita velkost pisma/riadku - standard
\input makra.tex % nacitanie Ivanom pripravenych nastaveni a prikazov
\hyphenation{star-šov-stvo} % rozdelenie slov na konci riadku, treba tu uviest slova, ktore sam nepozna

\spravodaj{7-8}{2020}


\clanok {Leto s~Pánom Ježišom}
Leto pre mňa znamená záhrada. Je to miesto kľudu a pokoja. Netreba veľa miesta. Stačí kúsok pôdy a pár semienok a všetko sa zmení. Asi nie každý inklinuje k~záhrade, ale príroda zohráva v~ľudstve kľúčovú rolu. Nazdávam sa, že všetci sa potrebujeme vrátiť k~tomu pôvodnému. Všetko sa začalo v~záhrade. Podľa Zjavenia Jána v~záhrade aj skončíme. Stvorenie začína aj končí so Stvoriteľom v~záhrade. Zdá sa, že pre Ježiša sú záhrady vzácne. Záhradu nielen stvoril, ale sa v~nej často aj modlil a takisto v~nej vstal aj z~mŕtvych. Ale najkrajšia vec na tých záhradách je Pán Ježiš sám. Nechce, aby sme si Jeho dary vážili viac ako Jeho samého.

Prichádza leto. Aspoň časť tohto leta strávime v~záhrade alebo v~prírode. Tešíme sa na to, lebo sme boli stvorení na to, aby sme si odpočinuli v~prírode. No nezabúdajme na to, že tou najlepšou vecou na záhrade alebo prírode je náš Boh stvorenstva. Všimnime si Ho tam. Stretnime sa s~Ním a odpočiňme si v~Jeho prítomnosti. Pána Ježiša potom zažijeme v~prírode inak. Podobne ako v~príbehu s~Adamom a Evou, aj teraz prichádza do záhrady a hľadá ťa. Volá ťa a teší sa na teba.

Nech je toto leto letom oddychu a občerstvenia v~Jeho prítomnosti. Nech počujeme Jeho hlas, lebo určite nám má čo povedať.

Prajem vám požehnané leto!

\autor{Danny Jones}


\clanok {Správy zo staršovstva}
Bratia, sestry, milí priatelia,

sme veľmi radi, brat kazateľ Danny to vyjadril slovami „teším sa veľmi“, keď po nejakom čase sme sa mohli znovu stretnúť v~našej modlitebni na Palisádach. Čas, kedy sme sa mohli stretávať iba virtuálne -- nie skutočne, je prerušený. Dnes si môžeme vybrať, či sa pri našich zborových aktivitách stretneme osobne, alebo zostaneme doma a naďalej budeme využívať možnosť virtuálnych stretnutí.

Aj stretnutia staršovstva sme mávali prostredníctvom internetu. Bolo to dobré, že sme mohli využívať techniku. Posledné stretnutie však bolo opäť v~kancelárii na Zrínskeho. Ja osobne som toto stretnutie vnímal ako osobnejšie a intenzívnejšie som prežíval to, že sme spolu. Nie iba to, že spolu hovoríme a pracujeme, ale najmä osobnú blízkosť mojich bratov.

Novým spôsobom som si uvedomil význam výzvy z~Božieho slova, aby sme neopúšťali spoločenstvo bratov a sestier. Porozumel som tomu, že nejde iba o~to nebyť „single“, nebyť sám, ale že ide o~to zažívať spoločenstvo tých druhých, našich bratov a sestier. Pochopil som, že nejde iba o~to „nebyť“ sám, ale ide hlavne o~to „byť“ s~niekým, mať spoločenstvo.

Spoločné osobné stretnutia sú to, čo sme na stretnutiach staršovstva, virtuálnych aj osobných, v~poslednej dobe diskutovali. Hľadali sme možnosti, ako dodržať v~našej modlitebni bezpečnostné opatrenia a zároveň vytvoriť priestor pre čo najväčší počet záujemcov o~osobné stretnutie pri bohoslužbách.

Veľmi rád som sa stretol s~tými, ktorí našli odvahu a prišli osobne na Palisády. A~dovolím si parafrázovať slová apoštola Pavla: „Pevne dúfam v~Pánovi, že čoskoro prídeme všetci na Palisády.“ Chcem povzbudiť nás všetkých k~tomu, aby sme využili čas, ktorý nám Pán dáva, a znova sa spoločne stretávali, so zachovaním maximálnej zdravotnej bezpečnosti. Určite to bude na vzájomné duchovné povzbudenie nás všetkých.

Chcem ešte vyjadriť vďaku za jedno konkrétne stretnutie. Ide o~naše výročné celozborové členské zhromaždenie. Som vďačný nášmu nebeskému Otcovi, že nám umožnil organizačne ukončiť rok~2019 a nastaviť rok~2020. Rovnako som Mu vďačný za vás všetkých, ktorí ste prišli na naše VCZČZ. Na jedno konkrétne schválenie sme potrebovali, aby hlasovala polovica členov zboru. Vďaka Pánovi nás bolo dosť na to, aby sme mohli potvrdiť voľbu bratov a sestier z~Chvojnickej stanice a brata Paľka Škuleca uviesť do služby správcu zborovej stanice na Chvojnici. Teší nás jeho záujem o~túto našu stanicu, nasadenie, ktoré dáva do služby, a nápady, ktoré má pre šírenie evanjelia vo svojom okolí. Na VCZČZ sme pre neho aj pre jeho manželku vyprosili Božie požehnanie a zmocnenie pre službu, do ktorej bol povolaný Pánom Ježišom.

Prvý list Korinťanom 1,3: {\it „Milosť vám a pokoj od Boha, nášho Otca, a od Pána Ježiša Krista.“}

\autor {za staršovstvo Peter Pribula st.}


\clanok {Spoločné bohoslužby so zborom z~Podunajských Biskupíc}
19.~júla o~10.00~hod. budeme mať spoločné bohoslužby so zborom v~Podunajských Biskupiciach na Nákovnej~34.

Brat Lacko Taliga nám všetkým pripraví guláš; bolo by dobré, keby sme my ostatní prispeli šalátmi či zákuskami.


\clanok {Zborový tábor v~Častej-Papierničke}
Tešíme sa, že aj tento rok sa budeme môcť stretnúť na zborovom tábore v~Častej-Papierničke. Vzhľadom na uvoľňovanie bezpečnostných opatrení v~súvislosti s~novým koronavírusom môžeme tento tábor organizovať bez väčších obmedzení.

Viac informácií môžete získať u~br.~Petra Kolárovského.


\clanok{Dorastenecký tábor}
Dorastenecký tábor sa tento rok uskutoční v~termíne 16.~--~18.~júla a to formou denného tábora. Čas budeme tráviť v~Bratislave a blízkom okolí. Pripravujeme mestskú hru s~turistikou, bicyklovačku spojenú s~návštevou galérie a kúpaním a splav rieky Morava.

\autor {za dorastenecký tím Mišo Vrábel}


\clanok {Tábory Detskej misie}
Aj tento rok Detská misia poriada tábory pre školákov a dorastencov v~Častej--Papierničke. Niektoré turnusy sú ešte voľné. Ak by ste chceli poslať deti do Častej, kliknite na \ulink[https://www.detskamisia.sk/letne-tabory-2020.html]{detskamisia.sk/letne-tabory-2020.html}. Viac informácií získate u~sestry M. Kešjarovej.


\clanok {Pomoc ľuďom v~núdzi}
Obnovujeme zbierky pre ľudí bez domova. Teraz potrebujeme tieto veci: pánske nohavice, tričká, spodnú bielizeň, topánky, ľahké bundy, hygienické potreby (šampóny, mydlá), kovové (polievkové) lyžice a knihy.

V rámci služby ľuďom bez domova sme sa stretli so záujmom o~čítanie kníh. Keďže by sme radi našich klientov povzbudili, potešili a zároveň donútili rozmýšľať o~dôležitých veciach v~živote, obraciame sa na Vás s~prosbou o~darovanie jednej knihy, ktorá vás nejakým pozitívnym spôsobom ovplyvnila, prípadne vám pomohla, alebo vás jednoducho rozveselila a pobavila. Túto knižku si budú môcť ľudia bez domova požičať v~rámci našich pravidelných výdajov.

Všetky veci je možné prinášať vždy v~pondelok medzi 17.00~--~19.00~hod. na Ambroseho~6 v~BA. Dopredu sa prosím dohodnite s~našou koordinátorku zbierok Sylviou Vaniherovou 0905~484~675. Ďakujeme!

Máme aj voľné termíny na varenie polievky -- utorok 21.~júla a utorok 18.~augusta. Nahlásiť sa môžete u~Beaty Bogárovej na telefónnom čísle 0908~046~409. Vopred ďakujem za ochotu poslúžiť.


\clanok{Zbierky za uplynulé obdobie}
Milí bratia a sestry, ďakujeme za vašu obetavosť. V~uplynulom období ste prispeli:
\vskip-1ex\begitems
* investičný fond: 719,--~€
* misia: 294,--~€
\enditems


\n 1.	7.	Ľudovít	BETKO;
\n 4.	7.	Margita	ELISCHEROVÁ;
\n 4.	7.	Ľubomíra	KOHÚTOVÁ;
\n 10.	7.	Slavomír	MÁŤUŠ;
\n 10.	7.	Katarína	KEREKRÉTY;
\n 11.	7.	Milada	KREJČOVÁ;
\n 16.	7.	Rút	BEDNÁRIKOVÁ;
\n 20.	7.	Mária	KOHÚTOVÁ;
\n 27.	7.	Lenka	KOHÚTOVÁ;
\n 28.	7.	Elena	ŠALINGOVÁ;
\n 28.	7.	Pavlína	SYNOVCOVÁ;
\n 31.	7.	Marína	CIHOVÁ;
\n 1.	8.	Dana	KEŠJAROVÁ;
\n 1.	8.	Zuzana	HRAŠKOVÁ;
\n 1.	8.	Vlasta	ŠALINGOVÁ;
\n 2.	8.	Marta	RAČIČOVÁ;
\n 7.	8.	Anna	KOPČOKOVÁ;
\n 11.	8.	Šimon	HOVORKA;
\n 16.	8.	Radovan	JANČULA;
\n 18.	8.	Anna	LIPTÁKOVÁ;
\n 23.	8.	Danica	PAULENOVÁ;
\n 25.	8.	Ivan	PAULEN;
\n 31.	8.	Miroslava	HOVORKOVÁ;
\narodeniny


\programna{7}{
\p  5 ; ne ;  9.30 ; Bohoslužby (D. Jones); 10.00 ; Chvojnica (J. Szőllős) ;
\p 12 ; ne ;  9.30 ; Bohoslužby (P. Kolárovský) ; 10.00 ; Chvojnica (P. Škulec) ;
\p 19 ; ne ; 10.00 ; Bohoslužby (D. Jones, Pod. Biskupice) ; 10.00 ; Chvojnica (P. Škulec) ;
\p 26 ; ne ;  9.30 ; Bohoslužby ; 10.00 ; Chvojnica (P. Pribula) ;
}
\vskip3ex
\programna{8}{
\p  2 ; ne ;  9.30 ; Bohoslužby (D. Jones) ; 10.00 ; Chvojnica (P. Škulec) ;
\p  9 ; ne ;  9.30 ; Bohoslužby (J. Szőllős); 10.00 ; Chvojnica ;
\p 16 ; ne ;  9.30 ; Bohoslužby (D. Jones) ; 10.00 ; Chvojnica (M. Kolářik) ;
\p 23 ; ne ;  9.30 ; Bohoslužby (D. Uhrin) ; 10.00 ; Chvojnica (P. Škulec) ;
\p 30 ; ne ;  9.30 ; Bohoslužby (P. Mozola) ; 10.00 ; Chvojnica ;
}
\vskip3ex
\koniecprogramu

\tiraz
\bye
