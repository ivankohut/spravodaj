\def\velkostpisma{10}
\def\velkostriadku{12.5}
\input makra.tex % nacitanie Ivanom pripravenych nastaveni a prikazov
% \hyphenation{star-šov-stvo} % rozdelenie slov na konci riadku, treba tu uviest slova, ktore sam nepozna

\vyrocnespravy{2024}

\clanok{Zbor}

\cast{Úvod}

Pripraviť správu kazateľa za zbor je pre mňa vždy novou výzvou. Tohtoročná správa je iná v~tom, že ju pripravujem po návrate do pozície kazateľa zboru. Hoci som počas svojej dlhoročnej služby v~našom zbore pripravoval už viacero správ kazateľa a mohlo by sa to stať pre mňa aj určitou rutinou, vždy si s~pokorou uvedomujem, že nedokážem zachytiť všetko, čo Pán Boh konal a koná v~našom zbore. Z~tej našej, ľudskej stránky sú súčasťou diania v~našom zbore všetky naše činy, ktoré sme vykonali podľa Božej vôle a na Božiu slávu, ale aj tie, ktoré boli vedené sebeckými, hriešnymi motívmi, túžbami, či žiadosťami, ktoré nepriniesli požehnanie a úžitok ľuďom ani slávu Bohu. Pán Boh je ten, ktorý to všetko vidí a dokáže správne posúdiť. Moje pohľady, hodnotenia toho, čo sa udialo sú len čiastočné a nedokonalé, ľudské a rovnako je to aj s~výhľadmi, plánmi do budúcnosti. Pri tom všetkom ma povzbudzuje vedomie, že cirkev a v rámci nej aj náš zbor patrí Pánovi.
On vždy koná nad naše očakávania, viac, ako my dokážeme zachytiť, koná aj v~skrytosti a vidí všetky radosti aj starosti a aj skutočný stav nášho zborového spoločenstva. Vie použiť aj naše slabosti a obrátiť na dobré aj to, čo sme my pokazili. Zároveň jeho plány aj s~našim spoločenstvom, tak ako aj s~nami jednotlivcami, sú tie najlepšie. Toto vedomie, potvrdené aj Božím Slovom, nás nezbavuje zodpovednosti za naše konanie a rozhodnutia, ktoré robíme. Všetci si uvedomujeme, že žijeme v~zložitých časoch. Preto bol povzbudením pre náš zbor verš, ktorý sme dostali na rok 2024: „Ak máš útočisko v~Hospodinovi, u~Najvyššieho svoj príbytok, nič zlé sa ti nestane, nijaká pohroma sa k~tvojmu stanu nepriblíži. Veď on svojim anjelom prikáže, aby ťa chránili na všetkých tvojich cestách.“ (Ž 91,9-10) Božiu ochranu a vedenie sme mohli prežívať po celý rok v~náročných situáciách, ktorými sme ako zbor prechádzali.

\cast{Najdôležitejšie udalosti v~živote zboru}

Pravdepodobne budete so mnou súhlasiť, že to bol náročný rok v~živote nášho zboru a na jeho začiatku sme nečakali zmeny, ktoré počas roku nastali. Určujúcou udalosťou roka bola zmena na mieste kazateľa zboru. Zmena kazateľa je pre zbor vždy veľkou výzvou o~to viac, ak je neplánovaná a neočakávaná. Môj návrat a nástup do pozície ďalšieho kazateľa zboru bol plánovaný. Vnímal som Božie povolanie, aby som sa po viac ako siedmich rokoch vrátil aj oficiálne do služby kazateľa v~zbore. Zo zboru som počas tých rokov neodišiel a snažil som sa slúžiť v rámci mojich obdarovaní a síl ako jeden zo starších zboru aj bez oficiálnej pozície kazateľa. Vnímal som a vnímam, že môžem v~zbore plnšie rozvinúť svoju službu ako zvolený a ustanovený kazateľ zboru a vnímal som aj potrebu zboru. Ďakujem všetkým členom zboru, ktorí to rovnako rozpoznali a vyjadrili to aj vo voľbách v~máji a júni uplynulého roku. V~zbore som začal pracovať na polovičný úväzok od 1.~7. a slávnostná inštalácia sa uskutočnila 8.~9. Túžim slúžiť s~plným nasadením a naďalej, rovnako ako pri prvej inštalácii, vnímam ako svoju úlohu „pripravovať svätých na dielo služby, na budovanie Kristovho tela.“ (Ef 4,12).

Pôvodný plán bol, že aj z~pozície ďalšieho kazateľa budem podporovať a pomáhať v~službe kazateľovi a správcovi zboru bratovi Petrovi Šrankotovi. Žiaľ, v~priebehu prvého polroka sa čoraz viac prehlboval nesúlad medzi očakávaniami viacerých členov zboru a aj vedúcich zborových zložiek od služby kazateľa zboru v~oblasti služby slovom, pastorácie aj spolupráce so zložkami zboru ako aj správy zboru. Služba kazateľa zboru nebola určitou časťou zboru prijímaná. Nespokojnosť so službou kazateľa bola komunikovaná aj verejne pri niekoľkých príležitostiach. Staršovstvo zboru riešilo situáciu na stretnutiach a rozhovoroch s~kazateľom zboru P.~Šrankotom počas letných mesiacov júl a august.
Po modlitbách a diskusiách dospelo staršovstvo k~záveru, „že brat kazateľ Peter Šrankota aktuálne nedisponuje potrebnými skúsenosťami, na základe ktorých by mohol situáciu v~zbore riešiť a napĺňať potreby členov zboru.“ (uznesenie staršovstva zboru prezentované 1.~9.). Na základe toho staršovstvo vnímalo ako najlepšie možné riešenie pre život a službu zboru ako aj pre ďalšiu službu Šrankotovcov na poli Božieho kráľovstva, predčasne ukončiť ku koncu roku 2024 službu P.~Šrankotu ako kazateľa a správcu nášho zboru BJB Palisády. Brat kazateľ a jeho manželka tento návrh akceptovali a dohodli sme sa na ukončení ich služby v~našom zbore. Veríme, že Pán má pripravenú pre brata kazateľa P.~Šrankotu a jeho rodinu iné miesto služby na poli Božieho kráľovstva. Je dôležité uviesť, že za krátky dvojročný čas služby v~našom zbore priniesli a vykonali aj mnoho pozitívneho (napr. na poli podpory misie, modlitebného zápasu a praktickej pomoci), za čo sme im 22.~12. vyslovili aj vďaku. Celý proces, ktorým sme prešli, a prijatie uvedeného rozhodnutia o~predčasnom ukončení služby bolo náročné zo strany zboru aj brata kazateľa a jeho manželky. Som vďačný, že prebehol viac-menej v~pokoji, bez hádok, ale to neznamená, že všetko bolo v~poriadku. Je veľmi dôležité, aby sme aj ako zbor, aj ako jednotlivci urobili reflexiu toho, prečo a ako nastala táto situácia, kde sme urobili chyby a hriechy, vyznali ich a urobili z~nich pokánie, odpustili a prijali odpustenie a vyvodili aj poučenia pre budúcnosť. Čo je zrejmé už teraz, budeme musieť zrejme venovať väčšiu pozornosť výberu kandidátov na kazateľa zboru a zvážiť zmeny v~procese výberu a volieb. Staršovstvo zboru pracuje na takejto reflexii za zbor. Na individuálnej úrovni vyzývam, aby si ju každý, kto bol nejakým spôsobom zainteresovaný, urobil sám za seba.

\cast{Štatistika}

V~počte členov nášho zboru nenastali v~uplynulom roku výrazné pohyby.
Počet registrovaných členov sa pohyboval okolo 160 členov vrátane zborovej stanice Connect.
Z~Božej milosti sa do rodín v~našom zbore narodili v~uplynulom roku dve deti -- Nina Kohútová a Joshua Smolka. Vo februári (18.~2.) brat kazateľ P.~Šrankota požehnával Mathewa Henryho Smolku a v~marci (24.~3.) Riška Kolářika, ktorý sa narodil ešte v~roku 2023.

Na rozdiel od roku 2023 sme mali v~uplynulom roku v~zbore aj dva sobáše. V~auguste (29.~8.) som mal to privilégium sobášiť Katku Valentovú a Sláva Kráľa a v~novembri (16.~11.) Svitlanu Didkovskú a Feriho Vondenu.

Do členstva zboru sme prijali 3 členov, Svitlanu Didkovskú, Filipa Barkócziho (prešiel z~členstva v~zborovej stanici Connect) a zo zboru BJB Viera sa k~nám vrátila sestra Slávka Volentičová (Kráľová). O~ukončenie členstva v~našom zbore požiadala sestra Ľudmila Vida, ktorá sa odsťahovala a chcela požiadať o~členstvo v~zbore BJB Tekovské Lužany.

Z~Božej milosti sme mali aj dva krsty, ktoré vykonal brat kazateľ Peter Šrankota. V~júni sa na jazere Kuchajda krstil Daniel Vrábel a koncom septembra sa na Zlatých pieskoch krstili tri sestry Evka Syčová, Miriam Kolářiková a Antónia Fábriová.

Do nebeskej vlasti si Pán v~uplynulom roku povolal 4 členov nášho zboru a jedného bývalého, ale dlhoročného člena nášho zboru. S~poďakovaním Pánovi za ich život a požehnanie, ktoré sme skrze nich prijali, sme sa v~marci rozlúčili s~bratmi Mirkom Kolářikom (9.~3.) a Julkom Betkom (19.~3.), v~júni s~bratom Lajkom (Ľudovítom) Betkom (19.~6.), v~auguste s~bratom Jurajom Balážom (23.~8.) a v~októbri so sestrou Jelkou Nevickou (31.~10.).

Účasť na hlavných nedeľných bohoslužbách ostala v~roku 2024 približne na rovnakej úrovni okolo 100 fyzicky prítomných na bohoslužbách a v~nedeľných besiedkach. Zároveň prebieha aj priame online vysielanie, ktoré sledujú najmä členovia, ktorí sa zo zdravotných, alebo iných dôvodov nemôžu zúčastňovať našich bohoslužieb a ľudia z~iných miest na Slovensku aj v~zahraničí.

Okolo 50 členov zboru evidujeme ako vzdialených členov, ktorí sa dlhodobo z~dôvodu zdravotného stavu, veku, vzdialeného bydliska, alebo z~pastoračných dôvodov dlhodobo nezúčastňujú na bohoslužbách a neparticipujú na živote zboru. Bude potrebné sa rozhodnúť, najmä v~prípade asi 20 členov, s~ktorými už dlhodobo nemáme žiaden kontakt, či nepozastavíme a následne neukončíme ich členstvo v~našom zbore. Je však dôležité, aby všetci vzdialení členovia ostávali predmetom našich modlitieb, spolu s~tými členmi našich rodín, ktorí ešte neprijali Krista za svojho Spasiteľa, alebo sa vzdialili od Pána.

Našich nedeľných bohoslužieb aj rôznych iných zborových aktivít sa viac-menej pravidelne zúčastňujú aj viacerí priatelia, ktorí nie sú, alebo sa nechcú stať z~rôznych dôvodov členmi nášho zboru. Nemáme ich presne spočítaných, ale môže ich byť okolo 25 až 30, z~ktorých asi pätina sa aj aktívne zapája do služby v~našom zbore.

Za veľmi dôležité považujem, aby sme prehlbovali svoj záujem o~ľudí, ktorí prichádzajú do nášho spoločenstva ako nepravidelní alebo noví návštevníci a majú záujem o~naše spoločenstvo. Nie je to len úloha uvítacej služby pri dverách, ktorej služba sa žiaľ ešte úplne neobnovila v~organizovanej forme, ďakujem bratovi S.~Máťušovi, ktorý ju „ťahá“ s~občasnou pomocou niektorých členov staršovstva. Je to však úlohou v~podstate každého z~nás „domácich“, aby sme prejavovali záujem o~ľudí, ktorí k~nám prídu, aby si mohli u~nás nájsť priateľov, vybudovať kontakty a nájsť v~našom spoločenstve svoju duchovnú rodinu.

\cast{Modlitby}

O~modlitbe sa hovorí a prirovnáva sa k~dýchaniu v~duchovnom živote. Je to veľká Božia milosť a zároveň príležitosť, že môžeme predkladať naše vďaky, chvály aj prosby všemocnému Bohu a môžeme mať skúsenosť, že On viditeľne odpovedá. Modlitebné zázemie a podpora, príhovorné modlitby za našich bratov a sestry, za priateľov a príbuzných, ktorí ešte nepoznajú Pána Ježiša ako svojho Spasiteľa, ako aj modlitebná podpora všetkých aktivít, celého života nášho spoločenstva je jednou z~podstatných vecí, ktoré nás odlišujú od akéhokoľvek ľudského spolku.

Popri našich individuálnych modlitbách v~komôrke sú pridanou hodnotou spoločenstva spoločné modlitby, ktoré majú špeciálne prisľúbenie od Pána. Som rád, že súčasťou veľkej väčšiny našich stretnutí, či už skupín, pracovných výborov aj spoločných nedeľných bohoslužieb sú aj hlasné spoločné modlitby. Žiaľ počuteľne sa do nich zapájajú väčšinou tí istí niekoľkí bratia a sestry, ale verím, že ostatní sa pripájajú v~tichých modlitbách.
Možno by bolo dobré obnoviť modlitebnú chvíľu ako pravidelnú súčasť našich nedeľných bohoslužieb aj s~úvodom k~modlitbám.
Okrem nedele však máme príležitosť prinášať neustále v~našich príhovorných modlitbách a prosbách pred nášho Pána všetkých našich bratov a sestry, priateľov a aj neznovuzrodených členov našich rodín, o~ktorých som hovoril aj v~predchádzajúcom odseku, ako aj všetky problémy. Modlitebný život je jedným zo základov života jednotlivca aj spoločenstva veriacich.
Špeciálne spoločné modlitebné stretnutia, ktoré sme v~minulosti mali, počas pandemických rokov zanikli a už sa neobnovili.
O~to dôležitejšie je udržiavať modlitby na ostatných spoločných stretnutiach.

Celý uplynulý rok pokračovali v~našich priestoroch na Zrínskeho ulici aj pravidelné modlitby (najmä počas schôdzí NR SR) kresťanov z~rôznych spoločenstiev z~Bratislavy a okolia s~poslancami NR SR. Od roku 2012 sa vytvorila skupina modlitebníkov, ktorí aj týmto spôsobom napĺňajú výzvu Božieho Slova (1Tim 2,2), aby sme sa modlili za tých v~moci postavených. Je to jedna z~príležitostí, ako sa konkrétne zapojiť do modlitebného zápasu za našu krajinu. V~súčasnosti bývajú tieto modlitby v~dohodnuté stredy od~7.45 do~8.45~hod.
Minulý rok 2024 sa nepodarilo obnoviť pomôcku -- modlitebný kalendár nášho zboru a ani jeho aktualizáciu uverejňovanú v~týždenných oznamoch. V~tomto čase, keď píšem správu, je už kalendár na 2025 k~dispozícii, ale musíme ďalej vytvárať vhodné formy, ako pravidelne prinášať aktuálne modlitebné predmety do zboru.
Využívajme výsadu a príležitosť, ktorú máme, že sa môžeme modliť k~všemohúcemu Bohu, Stvoriteľovi neba a zeme a prichádzať k~Nemu kedykoľvek ako milované deti k~nášmu Nebeskému Otcovi.

\cast{Nedeľné bohoslužby}

Hlavnou príležitosťou, kde sa mohla stretnúť k~spoločnej oslave nášho Pána celá naša zborová rodina, boli aj v~roku 2024 nedeľné dopoludňajšie bohoslužby. Vysoko si cením službu moderátorov, ktorí pripravia a zorganizujú nielen celé bohoslužby, ale väčšinou majú aj obohacujúce sprievodné slovo a síce nie pravidelne, ale aj úvody k~spoločným modlitbám či príhovor a modlitbu za deti v~besiedke. Túto dôležitú službu však potrebujeme doplniť o~ďalších bratov a sestry ochotných a schopných túto službu konať.

Brat Slávo Kráľ roky verne pripravuje a doprevádza na organe spev klasických spoločných piesní, čo vnímam ako veľké obohatenie nedeľných bohoslužieb nielen kvôli hudbe, ale aj výpovednej hodnote textov, ktoré vo väčšine prípadov vhodne dopĺňajú a umocňujú zvestované Božie Slovo.

Naša vďaka patrí aj bratom a sestrám, ktorí sa aj v~uplynulom roku zapájali do služby vedenia chvál v~našom zbore. Hoci sa najmä kvôli personálnym problémom nedarilo mať chvály každú nedeľu dopoludnia, tie, ktoré boli, priniesli svoje požehnanie, keď sme mohli aj vďaka hudbe a spevu plnšie prežívať Božiu prítomnosť. Považujem za dôležité, aby sa aj táto služba v~zbore rozvíjala, aby spievané chvály boli súčasťou našich bohoslužieb každú nedeľu, možno aj v~spojení s~modlitebnou chvíľou, o~ktorej som písal vyššie.

Som vďačný Bohu, že máme v~našom zbore viacerých obdarovaných bratov a sestry, ktorí sú zároveň väčšinou aj ochotní uplatniť svoje obdarovanie a slúžia v~našom zbore kázaním Slova. Kázanie a výklad Božieho Slova je ústrednou časťou našich nedeľných bohoslužieb. Záujem o~službu Slovom v~našom zbore je aj zo strany hostí, čo nás teší.

Božím Slovom v~uplynulom roku slúžil pravidelne raz až dvakrát do mesiaca až do októbra brat kazateľ Peter Šrankota. Počas celého roka, okrem mesiaca jún, som slúžil v~zbore Slovom minimálne raz do mesiaca a v~druhom polroku, keď som nastúpil do pozície ďalšieho kazateľa zboru, v~prípade potreby aj dvakrát do mesiaca. Zvyšné nedele slúžili bratia zo zboru a pozvaní hostia. Z~ordinovaných kazateľov, členov a priateľov nášho zboru na Vianoce nám poslúžil brat kazateľ Stanislav Baláž. Ostatní ordinovaní kazatelia, členovia nášho zboru, br. D. Uhrin a T. Valchář v~minulom roku nedostali priestor pre službu, čo je mi ľúto. Z~ďalších kazateľov BJB slúžil viackrát V.~Potockij z~nášho ukrajinského zboru Nádej. Vo februári nám v~rámci Národného týždňa manželstva slúžil brat Ivan Staroň s~manželkou, v~marci brat Dávid Cekov z~Nesvád a v~júni brat Andrew Hayes (v~súčasnosti zastupujúci kazateľ medzinárodného baptistického zboru). Brat Timotej Hanes zo zboru BJB Revúcka Lehota nám okrem služby Slovom zároveň predstavil v~novembri na staršovstve koncept „staršieho v~príprave“, ktorý u~nich v~zbore úspešne uplatnili. Z~členov zboru viackrát slúžili bratia Peter Pribula, Filip Barkóczi, Peter Kolárovský a Mirek Ira. Teším sa, že sa do tejto služby zapojili aj ďalší bratia: Rado Nemec, Daniel Plett, Ľubo Syč, Slávo Kráľ a Stano Kráľ. Náš zbor začiatkom septembra navštívil a slúžil Slovom pri príležitosti inštalácie ďalšieho kazateľa aj predseda Rady BJB Benjamín Uhrin. Z~medzinárodných návštev sa uskutočnila v~marci návšteva mládežníckeho tímu z~USA na čele s~misijným kazateľom Sky Prattom, ktorí slúžili s~mládežou nášho aj ukrajinského zboru.

Veľkým prínosom je, že naše bohoslužby sú vysielané priamo cez internet a sú tam aj archivované. Patrí za to veľká vďaka našim technikom, ktorí prenosy a archiváciu zabezpečujú.

\cast{Vzťahy so zbormi v~západnej oblasti BJB a ostatnými cirkvami a medzinárodná spolupráca}

Hneď začiatkom roku sme sa už tradične ako zbor zapojili do Aliančného modlitebného týždňa. Na stretnutí vo štvrtok u~nás slúžil úvodom k~modlitbám brat kazateľ Peter Šrankota.
Plánované spoločné oblastné vďakyvzdanie zborov západnej oblasti BJB sa minulý rok nekonalo kvôli tomu, že sme nenašli, alebo sa nepodarilo prenajať vhodné, kapacitou postačujúce priestory. Zo spoločných zhromaždení sa konala v~rámci oblasti jedine tradičná Záhradná slávnosť v~júni v~Podunajských Biskupiciach. V~uplynulom roku som sa zúčastnil ako zástupca BJB v~januári ekumenických bohoslužieb vrámci Týždňa modlitieb za jednotu kresťanov. Bol som pozvaný slúžiť Slovom v~marci a júni do zboru BJB Komárno, na Veľkú noc do zboru BJB Klenovec, v~máji do zboru BJB Lučenec, v~októbri som slúžil na vďakyvzdaní v~ukrajinskom zbore BJB Nádej a v~začiatkom novembra na zborovej víkendovke BJB Blansko. Počas vianočných a novoročných sviatkov som slúžil na misijnej stanici zboru BJB Revúcka Lehota. Brat kazateľ P.~Šrankota slúžil úvodným slovom k~Večeri Pánovej na spoločnom zhromaždení KvM na Veľký Piatok. O~ďalšej jeho službe v~iných zboroch nemám informáciu.
Z~iných denominácií v~našom zbore slúžil brat Jaro Tomašovský (CB).

Tak ako v~predchádzajúcich rokoch pokračovala aj v~roku 2024 spolupráca so zbormi vrámci platformy Kresťania v~meste najmä pri varení polievky bezdomovcom. Dobrovoľníci z~nášho zboru pokrývajú značnú časť tejto služby, ktorá sa koná 2- až 3-krát do týždňa a varia niekoľkokrát do mesiaca, za čo patrí vďaka Slávke Volentičovej za organizáciu varenia a všetkým, čo sa zapájajú a aj financujú túto veľmi potrebnú službu.

\cast{Služba zborových zložiek a život zboru}

Som veľmi vďačný Pánovi za obetavú službu viacerých bratov a sestier v~rôznych zložkách nášho zboru počnúc od besiedok, dorastu, mládeže, cez diakoniu, hospodársky výbor až po spevokol, či skupinky, ktoré dotvárajú obraz o~živote nášho zboru. O~ich službe nájdete informáciu v~správach za jednotlivé zložky, tu uvediem len niektoré vybrané udalosti.
Zvlášť ma teší, že spevokol nabral po pandemickej prestávke nový dych a opäť mohol byť veľkonočný a dva vianočné koncerty. Priestory nášho kostola nestačia kapacitne pokryť záujem, preto je potrebné uvažovať uskutočniť aspoň jeden z~koncertov vo väčších priestoroch. V~tomto smere sa už vytvára spolupráca so zborom ECAV v~Petržalke, čomu som rád. Za veľmi dôležité pre formovanie budúcich hudobníkov a spevákov, ale aj pre budovanie rodinného spoločenstva v~našom zbore považujem aj rodinný koncert, ktorý bol v~uplynulom roku koncom mája.
K~utuženiu nášho zborového života prispievajú aj spoločné obedy, ktoré organizujú diakoni a uskutočnili sa v~januári a septembri v~hoteli Plus, a tiež obed pre nových pravidelných návštevníkov nášho zboru, ktorý bol v~júni.

Biblické vzdelávanie je tiež spomenuté v~samostatnej správe, ale v~minulom roku nám v~tejto oblasti poslúžil aj „biblickým hĺbaním“ v~januári brat Vlado Boško, ktorého sa v~januári zúčastnilo asi 25 ľudí. Plánované boli viaceré takéto stretnutia, ale z~rôznych dôvodov sa ich nepodarilo uskutočniť. Brat V.~Boško nám slúžil podnetným spôsobom aj pri zborových víkendovkách v~marci a v~septembri, ktoré sa konali v~zariadení Berea. Žiaľ záujem bol nízky a ledva sme pokryli minimálnu kapacitu potrebnú pre poskytnutie služieb v~tomto zariadení. Jesenný termín kolidoval s~inými akciami. Myslím si, že je to výborná príležitosť mať ďalšiu formu praktizovania spoločenstva, kde sa môžeme spoznávať a aj načerpať.

Výnimočnou aktivitou sestier boli aj dámske raňajky, na ktoré bol vždy pozvaný hosť a ktoré sa konali v~zariadení Evanjelickej diakonie vedľa nášho kostola.
Kým v~minulosti sme organizovali v~zbore počas zimných a letných prázdnin viaceré tábory, v~minulom roku sa uskutočnila zborová lyžovačka v~Račkovej doline a letný spoločný tábor dorastu a mládeže. Účastníci z~neho prišli nadšení a povzbudení. Som veľmi vďačný tým, ktorý už roky organizujú tieto tábory, či pobyty, ale aj tým, ktorí sa zapojili len nedávno. Žiaľ po dlhých rokoch sa minulý rok nepodarilo zorganizovať z~organizačných dôvodov a následného malého záujmu potenciálnych účastníkov rodinný zborový tábor a museli sme platiť penále za nevyužitie rezervácie. Som rád, že pre rok 2025 to vyzerá nádejne, že tento tábor sa opäť uskutoční.
V~minulom roku sa udiali aj jednorazové podujatia, ktoré však boli návratom v~inovovanej forme po rokoch k~niečomu, čo sme v~našom zbore robili.
V~prvý júnový deň sa uskutočnil v~našej modlitebni koncert modlitieb a chvál, kde brat kazateľ P.~Šrankota zorganizoval a pozval bratov zo zboru Nové Zámky, aby sme spolu nielen chválili Pána, ale sme mali aj krátke vyučovanie o~chválach a uctievaní od brata Dávida Cekova. V~prvý júnový piatok sme sa po rokoch opäť zapojili do kampane Noc kostolov. Nerobili sme špeciálny program, len sme otvorili náš kostol a rozprávali sme sa s~ľuďmi, ktorí sa k~nám prišli pozrieť. Obe podujatia, aj keď rozdielne vo forme aj obsahu boli pre zúčastnených inšpirujúce, povzbudzujúce a verím, že priniesli svoje požehnanie. Výnimočnou udalosťou bol aj koncert Hope Gospel Singers v~našom kostole na úvod adventu.

Veľmi dôležitou službou, ktorú je potrebné zabezpečovať, je služba na parkovisku pri jeho otváraní a zatváraní. Som vďačný Bohu, že máme k~dispozícii toto parkovisko počas našich bohoslužieb a akcií. Ďakujem všetkým, ktorí ju obetavo robia. Bolo by potrebné, aby aj v~tejto službe našli svoje uplatnenie ďalší ochotní ľudia, aby sa bremeno tejto pravidelnej služby rozložilo na väčší počet zapojených.
V~minulom roku sme na staršovstve hodnotili aj výsledky dotazníka NCD (prirodzeného rastu zboru), ktorý sme robili už po štvrtý krát a je takým zrkadlom, ktoré ukazuje silné a slabé stránky života nášho zboru. Práve efektívne pracujúce štruktúry a zmocňujúce vodcovstvo, motivácia služobníkov a odovzdávanie služby spolu s~inšpiratívnymi bohoslužbami sú oblasti, kde máme čo zlepšovať.

Veľmi dôležitou udalosťou z~hľadiska zboru bolo minulý rok aj rozbehnutie procesu rekonštrukcie vonkajšej fasády. Hospodársky výbor a najmä brat Ľ.~Syč vykonali veľké množstvo práce pri zadaní súťaže, vyhodnotení ponúk a pri vybavovaní potrebných dokumentov na úradoch. Vyzdvihnúť je potrebné aj finančnú obetavosť členov a priateľov zboru, že sme nielen prijali záväzky potrebné na pokrytie nákladov na rekonštrukciu, ale že sme ich dokázali za stanovený čas aj naplniť, takže rekonštrukcia sa môže v~roku 2025 uskutočniť.

Dôležité bolo v~minulom roku aj uskutočnenie niekoľkohodinovej diskusie o~zborovej chalupe na Chvojnici a jej využívaní a budúcnosti. V~poslednom období ju využívali najmä skupiny nie z~nášho zboru. Brat Daniel Mikletič roky obetavo vykonáva úlohu správcu tohto objektu a potrebuje pomocné ruky. Vidíme ďalej perspektívu využívania objektu na Chvojnici na rôzne stretnutia a výlety rodín, skupín zo zboru.

\cast{Záver}

Za všetko, čo sa aj v~uplynulom roku v~našej službe a v~živote nášho zboru udialo, patrí v~prvom rade vďaka nášmu Bohu, ktorý dával silu, zmocnenie, svoje požehnanie, a ktorý nás viedol aj v~roku 2024. Tieto výročné správy nemajú slúžiť na naše vychvaľovanie sa, ale majú povzbudzovať oslavu nášho Pána. Jemu jedinému patrí naša chvála za čokoľvek, čo sa podarilo uskutočniť. Bez neho by sme nič neboli schopní urobiť. Naša vďaka však patrí aj ľuďom, všetkým, ktorí ste sa aktívne zapojili do života a služby v~našom zbore, všetkým, ktorí ste túto službu podopierali svojimi modlitbami. Nech je povzbudením a výzvou do našej ďalšej spoločnej služby v~Pánovej prítomnosti a k~jeho nasledovaniu aj v~tomto roku verš, ktorý sme dostali na rok 2025 pre náš zbor: „Vyuč ma, Hospodin, svojej ceste a budem žiť podľa Tvojej pravdy. Upriam moju myseľ na bázeň pred Tvojím menom.“ (Ž 86,11)

Aj v~otázkach zborového života máme „upriamiť svoju myseľ“, teda sa máme sústrediť na Boha a rešpektovať Jeho samého. Jeho Slovo, Jeho pravdu, samotného Pána Ježiša. Prežili sme v~uplynulom roku ako zbor náročné obdobie a do toho nového potrebujeme vykročiť s~novým odhodlaním dať sa viesť Pánom, učiť sa Jeho cestám.

Čakajú nás dôležité rozhodnutia -- voľby do staršovstva zboru, voľba správcu, hľadanie ďalšieho kazateľa. Chceme sa rozhodovať podľa Božej vôle. Tiež nás čaká rekonštrukcia fasády našej modlitebne, ktorú je tiež potrebné modlitebne pripraviť a sprevádzať. Je dôležité predkladať tieto veci Pánovi na modlitbách a pravidelne sa modliť za rast Pánovho diela v~našom zbore.

Táto konkrétna modlitebná prosba je veršom pre náš zbor na tento rok. Modlime sa ju počas roka a pracujme na jej každodennom uplatnení v~našich osobných životoch a v~živote nášho spoločenstva. Učme sa spolu kráčať v~bázni po Božích cestách a žiť podľa Božej pravdy sústredení na samotného Boha, Pána Ježiša Krista v~našich každodenných zápasoch a rozhodovaniach. To prinesie Božie požehnanie do našich osobných životov aj do života nášho zborového spoločenstva, po ktorom určite všetci túžime.

\autor{Ján Szőllős}


\clanok{Staršovstvo}

Staršovstvo dostalo pre rok 2024 slovo z~listu Rimanom 8, 14-16: „Lebo všetci, ktorých vedie Boží Duch, sú Boží synovia. Veď ste neprijali ducha otroctva, aby ste opäť žili v~strachu, ale prijali ste Ducha synovstva, v~ktorom voláme: ‚Abba, Otče.‘ Tento Duch sám dosvedčuje nášmu duchu, že sme Božie deti.“

V~roku 2024 sme zažívali turbulentné obdobie. Častokrát sme mohli mať pocit strachu, obavy a neistoty. Pýtali sme sa, či cesta, ktorou ideme, je správna a či sme neodbočili inam, ako nás Boží Duch viedol. Mohli sme mať srdcia naplnené strachom, ktorý paralyzuje a nedovolí pokračovať ďalej. Jednotlivo aj spoločne sme volali: „Otče, ukáž nám cestu a veď nás po nej!“ S~vďakou a pokorou môžem vyznať, že napriek náročnému obdobiu sme zažívali to, že sme boli vedení našim Nebeským Otcom.

V~roku 2024 sme pracovali v~zložení: kazatelia zboru -- Peter Šrankota a Ján Szőllős -- od zvolenia za kazateľa, členovia staršovstva -- Peter Antalík, Marcel Maďar, Radislav Nemec, Peter Pribula, Ľubomír Syč a Ján Szőllős -- do zvolenia za kazateľa.
Ľudská slabosť nás neobchádzala ani tento rok. Napriek rôznym zdravotným problémom sme v~Božej sile dokázali stáť v~službe, do ktorej sme boli postavení a slúžiť tak, ako nás viedol Boží Duch.

Prelomom rokov 2023 a 2024 rezonovala téma voľby ďalšieho kazateľa. Prešli sme procesom návrhov kandidátov, ich predstavenia a v~polovici roka konečnou voľbou.
Po niekoľkoročnej pauze nám Pán Boh znova daroval za kazateľa brata Janka Szőllősa. Brat Janko Szőllős je kazateľom na polovičný pracovný úväzok, nakoľko pracuje ešte aj pre rádio TWR Európa. Inštaláciu za kazateľa zboru sme mali po prázdninách 8.~9.~2024. Pri tejto príležitosti nás navštívili predseda Rady BJB na Slovensku br.~Benjamín Uhrin a zástupca Rady BJB za západnú oblasť br. Darko Kraljik.

Od jarnej zborovej víkendovky, ktorá sa uskutočnila koncom apríla, sme intenzívne komunikovali tému ohľadom potrebných skúseností brata kazateľa Petra Šrankotu pre vedenie zboru a napĺňanie potrieb jeho členov. Po mnohých diskusiách, hľadaní a zvažovaní možných riešení, týždňoch modlitebných zápasov sme dospeli k~poznaniu potreby odobrať z~jeho pliec nadmernú tiaž bremena služby kazateľa zboru. Toto rozhodnutie sme komunikovali aj s~niektorými bratmi mimo nášho spoločenstva. Brat Benjamín Uhrin sa zúčastnil dvoch mimoriadnych stretnutí, ktoré sme mali počas prázdnin a pomáhal nám pri vzájomnej komunikácii. Toto rozhodnutie nás neteší a vnímame ťažkosť situácie, v~ktorej sa nachádza zbor aj rodina brata Petra Šrankotu. Urobili sme ho pre dobro všetkých a zachovanie vzťahu Lásky, ktorá je centrom evanjelia, ktorému sme uverili a chceme, aby zostal aj centrom našich vzájomných vzťahov. Voláme vás k~tomu, aby ste Šrankotovcom prejavovali lásku rôznym spôsobom, ku ktorému vás vedie Pán Ježiš.

V~priebehu roku sme mali krst tých, ktorí sa rozhodli vyznať vieru v~Pána Ježiša Krista. S~pokorou vyjadrujeme vďaku nášmu Nebeskému Otcovi za tých, ktorí sa rozhodli takto verejne Ho vyznať ako svojho Pána a Spasiteľa.
Koncom roka sme sa stretli s~bratom kazateľom Timotejom Hanesom. Témou stretnutia bola príprava mužov na službu starších zboru. Túto tému sme komunikovali už aj na zborovom členskom zhromaždení. Sme Bohu vďační za tých bratov, ktorí sa rozhodli zapojiť do tohto učeníckeho programu.

Veľkou (rozsahom) udalosťou, ktorú sme prijali je organizovanie konferencie odborov sestier z~Čiech a Slovenska. Pri tomto projekte prosíme o~vaše intenzívne modlitby, aby všetko mohlo byť na slávu Božiu. Konferencia sa uskutoční v~máji 2025 a budeme pri jej organizovaní potrebovať veľa vašich, našich rúk a hlavne naše srdcia.
Ďalšou témou, ktorá časovo presahuje rok 2024 je oprava fasády našej modlitebne, či kostola. Projekt sme odovzdali hospodárskemu výboru. Chcem aj touto cestou poďakovať za ich prácu pri plánovaní a realizácii rekonštrukcie. Vám, nám všetkým chcem poďakovať za finančné záväzky, ktoré sme dali a mnohí už aj zrealizovali. Zároveň sa modlím o~Božie vedenie pri ďalších krokoch, ktoré smerujú k~ukončeniu rekonštrukcie modlitebne.

Významnou časťou našich stretnutí sú modlitby. Modlitby za nás, aby sme boli poslušní Božiemu vedeniu, múdri pri zverenej službe a aj modlitby za vás, aby sme spoločne napĺňali poslanie, pre ktoré nás Pán Boh postavil do Bratislavy a na Palisády.

\autor{Peter Pribula}


\clanok{Diakonia}

V~uplynulom roku mala diakonia nášho zboru štyri pracovné stretnutia, konkrétne 24.~2., 27.~5., 30.~9. a 18.~11. Predmetom stretnutí bola príprava a následne aj zabezpečovanie a vykonávanie nasledovných činností:
\begitems
* materiálne a personálne zabezpečovanie pravidelného mesačného vysluhovania Večere Pánovej -- materiálne zabezpečenie a prípravu stolovania verne zabezpečovali manželia Miroslav a Štefánia Antalíkovci a manželia Dávid a Barbora Pribulovci,
* organizovanie a zabezpečenie spoločných zborových obedov a občerstvení pre rodiny členov a priateľov ako aj seniorov nášho zboru -- 9.~6. obed pre nových pravidelných návštevníkov našich bohoslužieb, 8.~9. obed pri príležitosti inštalácia kazateľa Jána Szőllősa, 8.~12. pravidelný vianočný obed pre seniorov nášho zboru -- organizačne a materiálne verne zabezpečovali Katarína Kráľová, Vlasta Šalingová a Barbora Antalíková,
* pravidelná návšteva seniorov, chorých a imobilných členov zboru v~ich domácnostiach a zariadeniach pre seniorov s~cieľom zistenia ich potrieb a budovania a udržiavania prirodzených bratsko-sesterských vzťahov v~niektorých prípadoch spojená aj s~vysluhovaním Večere Pánovej -- Alžbeta Dudášová (Dúbravská oáza), Jaruška Krišková (doma), Margita Betková a Alžbeta Betková (Stredisko Evanjelickej DIAKONIE Bratislava) -- Večeru Pánovu vysluhuje kazateľ Pavel Pivka a niekto z~členov diakonie nášho zboru,
* kontaktovanie členov zboru, ktorí už dlhší čas zo zdravotných, ale aj neznámych dôvodov nenavštevujú pravidelné zhromaždenia s~cieľom zistenia príčin a následne zaradenia do predmetov modlitieb,
* pravidelné návštevy v~zariadení pre seniorov BETANIA a HESTIA so službou piesňami a Božím Slovom -- kazateľ Pavel Pivka, Elenka Gubová a Vlaďka Laurenčíková,
* zabezpečovanie dopravy sestier a bratov seniorov na ich pravidelné stretnutia a služby -- Chata Komenského v~Račkovej doline, Zariadenie sociálnych služieb Samaritán v~Tekovských Lužanoch,
* pravidelné biblické vyučovanie pre seniorov nášho zboru a iných spoločenstiev v~našom zborovom dome na Zrínskeho, ktoré vedie kazateľ Pavel Pivka,
* zabezpečovanie návštev a gratulácií pri príležitosti vzácnych životných jubileí -- Juraj Kvačka (80 rokov), Peter Lichanec (70 rokov), Jana Perknovská (70 rokov), Jelka Nevická (75 rokov),
* finančná podpora varenia obedov pre bezdomovcov,
* vianočná finančná podpora členov zboru, ktorí z~rôznych dôvodov majú finančnú núdzu.
\enditems

Touto cestou ďakujeme všetkým členom diakonie za aktívnu a obetavú prácu a zároveň aj staršovstvu zboru za spoluprácu a podporu.

\autor{Ján Štefko}


\clanok{Hospodársky výbor}

Hospodársky výbor v~roku 2024 pracoval v~tomto zložení: P.~Šrankota, Ľ.~Kešjar, Ľ.~Syč, J.~Štefko, M.~Maďar, D.~Mikletič. Externí pracovníci HV: J.~Szőllős, Ľ.~Kohútová, K.~Kerekréty.

V~prvom rade by som chcel poďakovať za obetavú prácu menovaným sestrám a bratom, ktorí sa na úkor svojich rodinných povinností obetovali pre pre potreby zboru.

Nový rok nám prináša nové výzvy a povinnosti, s~ktorými sa chceme popasovať. Sme si vedomí, že bez Božieho požehnania a Jeho prítomnosti v~našich životoch to nebude možné zrealizovať. Preto chceme pokračovať v~práci s~Jeho požehnaním aj tento nadchádzajúci rok 2025. Pán nás vyzbrojil požehnaním z~Božieho slova, veršom z~Písma: „Nie vy ste si vyvolili mňa, ale ja som si vyvolil vás, aby ste išli a prinášali ovocie a aby vaše ovocie zostalo; aby vám Otec dal všetko, o~čo budete prosiť v~mojom mene.“ (J 15,16)

A tak v~pokore a očakávaní vstupujeme do nového roku a túžime s~Jeho požehnaním pokračovať v~práci na Jeho diele. Ak to bude Jeho vôľa, chceli by sme v~prvom rade zrealizovať dlhodobo pripravovanú opravu fasády kostola. Ďakujeme bratovi Syčovi, že sa podujal na túto neľahkú úlohu a komunikuje s~vybranou firmou, ktorá bude vykonávať práce obnovy fasády. Ďakujeme všetkým darcom, ktorí prispievajú na realizáciu obnovy fasády.

Náš zborový majetok na Chvojnici, kostolík a chalupa, je udržiavaný v~prevádzky schopnom stave. Chalupa je využívaná na rekreačné pobyty, ktoré dostatočne pokrývajú náklady na prevádzku bez záťaže na zborový rozpočet. V~roku 2024 sa na chalupe uskutočnilo počas roku šesť brigád aj za účasti dorastencov, ktorí odviedli peknú prácu. V~tomto roku bolo uskutočnených osemnásť rekreačných pobytov a dve svadby. Pán Boh požehnáva i túto prácu, ktorú náš zbor ponúka k~budovaniu vzťahov a poznania Božej vôle v~prostredí prírody, ktorú stvoril pre nás. Ďakujeme a dúfame, že sa Chvojnica bude ďalej využívať na povzbudenie a zrekreovanie duchovných a telesných potrieb členov a priateľov nášho zboru.

\autor{Daniel Mikletič}


\clanok{Biblické a iné vzdelávanie}

Správa o~biblickom a inom vzdelávaní obsahuje len vzdelávanie formou stretnutí na biblických hodinách organizovaných vo štvrtok večer na Palisádach (Zrínskeho) pod mojím vedením. Ostatné formy biblického vzdelávania organizované jednotlivými zložkami zboru môžu byť zahrnuté v~správach za zložky, alebo v~správe kazateľa zboru. V~tejto správe nie sú zahrnuté ani vzdelávacie aktivity na nadzborovej úrovni, ktorých sa zúčastnili členovia nášho zboru a ani vzdelávanie v~skupinkách.

Spoločné štúdium Svätého Písma prebiehalo viac-menej pravidelne raz týždenne od začiatku júla až do konca septembra. Pravidelnosť bola občas narušená, keď som kvôli pracovným povinnostiam bol nútený stretnutie zrušiť.
Online vzdelávanie vzhľadom na charakter nášho vzdelávania na biblických hodinách a zloženie účastníkov a aj moje kapacitné možnosti sme nezrealizovali.

Počas prvej polovice uplynulého roku až do konca júna sme preberali postupne všetky tri Jánove listy. Vzdelávanie sa konalo vo štvrtky v~priestoroch na Zrínskeho 2. Po prázdninách sme od začiatku októbra začali preberať knihu proroka Ozeáša a pokračoval som v~jej výklade až do vianočných prázdnin. Prebrali sme 6~kapitol a od januára pokračujeme od 7.~kapitoly. Myslím si, že štúdium tak Jánových listov ako aj knihy Ozeáš prinieslo a prináša nielen nové vedomosti a objavy, ale prežívame pri tom aj živý dotyk Božieho Slova, ktoré hovorí do našej situácie a životov. Vzdelávania vo štvrtky sa zúčastňovalo v~priemere 10 až 12 členov a priateľov nášho zboru.

\autor{Ján Szőllős}




\clanok{Sestry}

Rok 2024 bol pre nás sestry rokom, keď sme sa usilovali hľadať Božiu tvár, v~súlade s~veršom, ktorý sme si vytiahli: „Zamerajte teda svoju myseľ i srdce na hľadanie Hospodina, svojho Boha! Vstaňte a postavte svätyňu Hospodina“ (1Kron 22,19)

Na našom prvom januárovom stretnutí sme sa zamýšľali nad tým, že Božiu prítomnosť nachádzame cez vďačné a uctievajúce srdce a že Bohu vzdávame slávu, keď milujeme druhých ľudí, bratov, sestry, ale aj nepriateľov. Zdieľali sme sa navzájom o~tom, aký verš sme si vytiahli samy pre seba a čo hovorí do nášho života.

Vo februári nám poslúžila sestra Natalija Elijas na tému: Neste si navzájom bremená, a tak naplníte Kristov zákon (G 6,2). Pripomenula nám okrem mnohých iných dôležitých právd aj to, že potrebujeme hovoriť do života druhých -- nie o~druhých, ale druhým, a to s~múdrosťou, milosťou, s~láskou, s~pokorou a s krotkosťou, a musíme sa milovať navzájom dosť na to, aby sme mohli ponúknuť silu, nádej a nápravu.

Prvý piatok v~marci sme oslavovali ekumenický Svetový deň modlitieb, ktorý v~Bratislave organizuje vždy iná cirkev. V~roku 2024 to bol cirkevný zbor Evanjelickej cirkvi v~Dúbravke. Bohoslužbu Svetového dňa modlitieb pripravili kresťanské ženy z~Palestíny na biblický text: „Prosím vás, znášajte sa navzájom v~láske“ (Ef 4,1-7). V~čase zúriaceho vojnového konfliktu to bola veľká výzva a pokladali sme za Božiu prozreteľnosť (o~tejto téme a krajine sa rozhodlo na svetovej konferencii v~roku 2017), že sa kresťanské ženy i muži na celom svete môžu modliť spolu s~palestínskymi ženami za mier.
Na našom pravidelnom sesterskom stretnutí v~marci si sestry z~vedúceho tímu -- Aďka, Baka, Barbi a Mirka pripravili svedectvo o~tom, čo prežívajú s~Pánom.

V~apríli nás navštívila sestra Dáša Leeder a hovorila o~téme, ktorá jej veľmi leží na srdci: „Život v~milosti“.

V~máji mnohé naše sestry vycestovali spoločným autobusom na Konferenciu sestier ČR a SR do Litoměříc. Témou konferencie bolo Srdce ženy: „Kde je tvoj poklad, tam bude i tvoje srdce“ (Mt 6,21). Boli sme všetky veľmi povzbudené a požehnané.

Na stretnutí v~júni nás navštívili sestry z~výboru Odboru sestier, témou bolo: Za čo sme sa modlili a modlíme: „A máme k~Nemu pevnú dôveru, že nás počuje, kedykoľvek o~niečo prosíme podľa Jeho vôle“ (1J 5,14).
Po prázdninách~Jeremiáša 18, 6: „Ako je hlina v~ruke hrnčiara, tak ste vy, dom Izraela, v~mojej ruke.“

V~októbri 2024 som sa spolu s~ďalšími sestrami zo slovenského i českého výboru Odboru sestier zúčastnila konferencie Únie baptistických žien Európy (EBWU), ktorá sa konala od 10. do 13. 10. v~Lisabone. Stretlo sa nás približne 60 žien z~21 krajín Európy a Blízkeho východu, a tiež z~Azerbajdžanu a Uzbekistanu. Na našom októbrovom stretnutí som referovala o~tejto konferencii, ktorej ústrednou témou bol Víťazný život.

V~novembri sme si pripomenuli Svetový deň modlitieb baptistických žien. Pripojili sme sa k~ženám vo viac ako sto krajinách, ktoré sa v~modlitbách prihovárajú za svoje komunity a usilujú sa zmeniť svet k~lepšiemu, a to aj vyzbieranými finančnými prostriedkami. Témou roka 2024 bol „Život podľa Božieho zámeru“ na základe verša: „Vieme, že všetky veci slúžia na dobro tým, čo milujú Boha, ktorí sú povolaní podľa Jeho predsavzatia“ (R 8,28).

V~decembri si pre nás pripravila adventnú tému sestra Táňa Trusiková zo zboru Viera o~očakávaní príchodu Krista, ktorý bol prorokovaný už na začiatku Biblie.

Na našich stretnutiach sa zúčastňovalo pravidelne približne 20 žien rôzneho veku. Stretávali sme sa v~útulnom prostredí na Zrínskeho a vnímali sme zakaždým Božiu prítomnosť a Jeho požehnanie.

\autor{Jarmila Cihová}
\vfill\break


\clanok{Dorast}

Na doraste sme v~roku 2024 pokračovali v~tradičnom stretávaní každý piatok medzi 17.30 a 19.00 hod. na Súľovskej 2. Priemerný počet ľudí na stretnutiach bol cca 15 až 20. Do letných prázdnin sme sa venovali hlavne knihe Nehemiáš. Nehemiášom sme ukončili trojročný systematický cyklus zamyslení nad Starým zákonom. Chronologicky sme prešli obdobie od Judských kráľov až po koniec Starej zmluvy. Popri bežných témach sme privítali aj niekoľko hostí a dvakrát sme sa stretli s~dorastencami zo zboru Viera.

V~lete sme mali už tradičný dorastenecko-mládežnícky tábor, tentoraz v~Novej Lehote. Hlavné témy tábora boli vystavané okolo Ježišovej kázne na hore. Pri večerných táborákoch sme sa snažili reflektovať aktuálne témy, ktoré riešia mladí ľudia v~dnešnom svete. Napr. vzťahy, materializmus, sebahodnota či dôvody, pre ktoré je kresťan kresťanom. Zachovali sme aj spoločné stíšenia na izbách, počas ktorých sa mohli dorastenci inšpirovať tým, ako si vedúci robia svoje osobné stíšenia. To platilo aj pre skupinky po hlavných témach, počas ktorých sme sa v~osobnejšom počte mohli zamýšľať nad témou, modliť sa a diskutovať o~otázkach, ku ktorým nás vyprovokovala téma. V~štyroch tímoch sme mohli vypustiť paru, súťaživosť a vzájomne súperiť a merať sily hlavne v~športových disciplínach a rôznych hrách. Zažili sme spolu, vďaka Bohu, veľmi požehnaný týždeň.

Počas prázdnin mali dorastenci tradičnú letnú výzvu. Tento rok to bolo naučiť sa naspamäť názvy všetkých biblických kníh a prečítať evanjelium podľa Matúša. Za odmenu sme s~nimi šli na aktivitu „team up“, počas ktorej sme mohli pri rôznych úlohách zažiť a prehĺbiť vzájomnú spoluprácu, športového ducha a prelamovanie záhad.

Nový školský rok sme začali s~generačnou obmenou, keďže väčšina dorastencov odišla do mládeže. Piatkových stretnutí sa pravidelne zúčastňovalo 4 až 5 dorastencov. V~témach sme sa preklopili do Novej zmluvy -- k~štúdiu Matúšovho evanjelia. Témy z~Matúša sa snažíme prepájať aj s~osobnými svedectvami, aby sme dorastencom ukázali, že Písmo a Boh nie sú odtrhnutí od každodenného života, ale majú naň bezprostredný efekt.

Dorastový tím pokračoval v~zložení Janko Kováčik, Rado Nemec, Tamarka Syčová, Mišo a Angie Vráblovci a Filip Barkóczi. Našim hlavným cieľom je aj v~tomto roku ukazovať dorastencom Slovom aj životom na Pána Boha. Túžime po tom, aby ho mohli osobne zažívať, spoznávať a rásť vo vzťahu s~Ním. Prosíme Vás o~modlitby za to, aby sme vedeli verne žiť a komunikovať Božie slovo zrozumiteľným spôsobom pre dorastencov. Ďakujeme, že aj touto formou sa zapojíte do sprevádzania dorastencov na ich ceste k~Pánu Bohu.

\autor{Filip Barkóczi}


\clanok{Besiedka}

V~1.~polroku roku 2024 sme mali v~malej besiedke (3 -- 7 rokov) zapísaných 11~detí, vo veľkej besiedke 7 detí (8 -- 11 rokov). V~2.~polroku nastali presuny a momentálne máme v~malej aj veľkej besiedke po 9 detí. K~pravidelným návštevníkom besiedky prichádzajú občas aj hostia alebo sa k~nám pripoja rodičia, či starí rodičia niektorých besiedkárov (v~malej besiedke).

Pozitívne hodnotíme, že deti prichádzajú v~nedeľu ráno do zhromaždenia. Cítia sa tak súčasťou palisádskej rodiny. Zo zhromaždenia ich potom odvádzame do besiedky. Niektoré menšie deti občas zaslzia, ale v~podstate žiadne z~detí nemá vážnejšie problémy s~trávením času bez rodičov. Väčšina našich besiedkárov sú škôlkari alebo školáci, takže sú na prostredie medzi deťmi zvyknutí.
Po príchode na Zrínskeho sa deti rozdelia na veľkú a malú besiedku a každá skupina má následne svoj program. Malá besiedka začína pohybovými aktivitami, ktoré súvisia s~preberanou biblickou lekciou. Potom sa pomodlíme, prípadne si zaspievame a prejdeme k~hlavnému bodu programu, ktorým je biblické vyučovanie. Snažíme sa ho robiť interaktívne, aby deti boli zapojené do rozprávania. Využívame rôzne názorné pomôcky, rolové hry, kladieme deťom otázky, snažíme sa Božie slovo aplikovať do každodenných životov detí.
V~1.~polroku sme na malej besiedke preberali starozákonné príbehy (Jozef, Mojžiš, Daniel). Po Veľkej noci sme prešli 10~Božích prikázaní. Od nového školského roku sme sa zamerali na Nový zákon a na zázraky Pána Ježiša. Súčasťou našich besiedok je aj učenie sa biblických veršov naspamäť. Keďže sa nám neosvedčil systém, keď deti dostávali každú nedeľu nový verš, rozhodli sme sa, že sa budeme učiť vybrané väčšie odseky z~Písma. V~minulom roku sme sa tak naučili 23. žalm. Opakovali sme ho dlhšie, ale vďaka tomu sme si ho mohli lepšie zapamätať (nielen deti, ale aj učitelia).

Okrem už vyššie spomínaných bodov programu je súčasťou malej besiedky opakovanie toho, čo deti počuli, a samozrejme tvorivá aktivita. Deti si veľmi radi vyrábajú a tešia sa, keď si môžu svoj výrobok zobrať domov.

Priebeh veľkej besiedky je podobný: úvodné privítanie, krátke zdieľanie toho, čo deti cez týždeň zažili, modlitby za konkrétne veci, úvodná aktivita do témy (hra, scénka atď.), biblická téma, práca s~Bibliou a ak je dostatok času, deti si na záver vyrábajú niečo k~téme, príp. si opakujú verš na zapamätanie.
Vo veľkej besiedke deti preberali v~prvom polroku Skutky apoštolov (materiál Ako šlo evanjelium do sveta) a po prázdninách sériu lekcií Boh sa stará, keď sú deti smutné (z~Detskej misie).
V~predvianočnom období sme opäť obnovili nácviky spevu s~Diankou Dzuriakovou, z~čoho mali viaceré deti radosť.

Pokračujeme v~zbierke pre 7-ročného chlapca z~Etiópie (Bereket Daniel Tadese). Cez nadáciu Integra sme tak zapojení do projektu podpory detí z~chudobných rodín v~meste Yabelo.

Deti poslúžili v~zbore v~minulom roku krátkym programom na Deň matiek aj počas adventu. Deti z~veľkej besiedky sú veľmi tvorivé a aktívne sa podieľali na tvorbe scenárov. Už tradične veľkí i malí besiedkári vítali návštevníkov prichádzajúcich do zboru na Kvetnú nedeľu.
Istý brat povedal: „Aká bude naša budúcnosť, závisí od toho, čo vkladáme do detí v~súčasnosti.” Ako veriaci vieme, že našu budúcnosť má v~rukách náš nebeský Otec, ale zároveň tiež vieme, že sme zodpovední práve pred Bohom, ako vedieme naše deti \hbox{(Ž~78,1-7)}. Využime čas, ktorý nám Pán Boh ešte dáva, aby sme deti viedli k~správnym hodnotám a aby sme im pomáhali pochopiť Božie pravdy.

Učiteľky v~malej besiedke: Kika Horvátiková, Mirka Hovorková, Miriam Kešjarová, pomocníci: Katka Kerekréty, Martina Javorníková (1. polrok), Martin Hovorka, Tamarka Syčová
Učitelia vo veľkej besiedke: Kvetka Maďarová, Baka Pribulová, Slávka Volentičová, Filip Barkóczi (2. polrok)

\autor{Miriam Kešjarová}


\clanok{Spevokol}

Náš spevokol sa v~ostatných rokoch postupne stáva stále silnejšou a známejšou zložkou nášho zboru hlavne mimo nášho zboru. Je to tým, že na bežných nedeľných bohoslužbách slúžime naozaj len výnimočne. Vnútorne sme sa preorientovali na službu pre návštevníkov našich koncertov. Na vianočné koncerty chodí toľko ľudí, že už pravidelne musíme robiť koncerty dva. Modlitebňa je aj tak naplnená do posledného miesta a predsa sa stane, že niekto sa dnu ani nedostane. Tieto koncerty môžeme robiť aj preto, že členmi nášho spevokolu sa stávajú aj speváci, ktorí sú členmi iných zborov, kde nemajú možnosť slúžiť touto hrivnou. V~uplynulom roku ma dokonca požiadala jedna sestra z~B.~Bystrice, či by mohla chodiť na naše nácviky a s~nami spievať. Prekvapilo ma, že by chodila do Bratislavy len kvôli službe v~našom spevokole. Dvesto kilometrov do Bratislavy a po nácviku zas dvesto kilometrov domov. Takýto postoj zahanbuje hlavne nás Bratislavčanov, ktorí takéto prekážky prekonávať vôbec nemusíme a aj tak spevu nevenujeme potrebnú dôležitosť.

V~minulom roku sme mali Veľkonočný koncert 23.~3. o~17.00~hod. Po ňom sme zaspievali ešte niekoľko piesní počas veľkonočných bohoslužieb v~našom zbore. Zo Srbska sme dostali pozvanie poslúžiť spevom na ich konferenciu 13.~10. Túto návštevu sme museli odvolať pre rôzne prekážky.

Potom sme sa už venovali príprave na vianočné koncerty. Zase sme urobili koncerty dva: 14. a 15.~12. Koncerty mali veľmi dobrú odozvu. Poslúžili sme niekoľkými piesňami aj počas vianočných sviatkov v~našom zbore. Z~nedeľného koncertu sme urobili aj video, ktoré je dostupné na verejnej sieti.

Uvedomujeme si, že sme pominuteľní a čas na tejto zemi sa nám kráti, tak chceme spievať a hrať na Božiu slávu, spievať o~Jeho láske a povzbudzovať tých, čo na Božiu cestu už vykročili.

\autor{Slávo Kráľ}


\clanok{Chválospevy}

Podobne ako v~roku 2023, aj v~roku 2024 by som pôsobenie služby chvál rozdelil na niekoľko období.
Prvé z~nich bolo od januára, do konca školského roku. Počas tohto obdobia celkom úspešne fungovala štruktúra rozbehnutá ešte v~roku 2023. Za 6 mesiacov chvály chýbali iba na 6 bohoslužbách. Vždy v~priebehu týždňa sme sa dohodli, kto sa môže v~danú nedeľu zapojiť, a postupne sa takto prestriedali v~rôznych zloženiach.

Od júla do septembra bývali chvály iba občasne, z~vlastnej iniciatívy slúžiacich. Veľkú úlohu v~tom zohrali tábory, dovolenky a neskorý štart organizovania tejto služby.

V~októbri sme mali spoločné stretnutie našej zložky na Súľovskej, na ktorom sme diskutovali a dohodli sa na spôsobe fungovania na nasledujúci školský rok. Keďže systém z~minulého roku sa celkom osvedčil, zhodli sme sa na pokračovaní v~ňom. Až do Vianoc sme mohli spoločne chváliť takmer každý týždeň. Počas Vianoc za nás zaskočil spevokol, za čo im ďakujeme.

Verím, že naša služba však nebola iba o~tom, či bolo na bohoslužbe viac alebo menej hudby, ale o~spoločnej chvále nášho Pána. Týmto by sme chceli poďakovať za všetky povzbudenia a spätnú väzbu, ktoré ste nám po bohoslužbách vyjadrili. Podporu a povzbudenie do služby nám vyslovilo aj staršovstvo, za čo im taktiež ďakujeme. Osobne ďakujem všetkým, ktorí sa do našej služby zapojili. Som veľmi vďačný za Vašu ochotu zapojiť sa, odvahu postaviť sa pred obecenstvo a radosť, ktorú ste do služby priniesli.

Veľmi by nás potešilo, keby sa ku nám pridalo viac ľudí, či už hrou na hudobný nástroj alebo spevom. Budeme však vďační za každú podporu prejavenú či už povzbudením alebo modlitbou za našu službu.

\autor{Adam Alexaj}


\clanok{Služba ľuďom v~núdzi}

Na úvod mi dovoľte trochu objasniť štruktúru tejto pomoci ľuďom v~núdzi. Tento projekt vedie a zastrešuje organizácia Kresťania v~meste (KvM) a členovia a priatelia nášho zboru sa aktívne zapájajú svojim dielom do tejto práce. Takisto aj v~uplynulom roku 2024 sme sa znovu aktívne zapájali do tejto služby. V~našom zbore sa vytvorilo 6 až 7 spoľahlivých tímov, ktoré skoro na pravidelnej báze pomohli zasýtiť výbornou polievkou ľudí bez domova.

Naše poďakovanie znova patrí mnohým z~vás, či už za samotné varenie alebo finančnú výpomoc. Pre úplnosť mi dovoľte pripomenúť, že náklady na prípravu jednej 25~l polievky nám postupne vzrastajú. Dnes sa pohybujeme na sume okolo 80-100~€. S~veľkou vďačnosťou môžeme poďakovať, lebo všetky náklady sú hradené z~Vašich dobrovoľných darov.

Nedá sa však nevidieť, že záujem o~túto formy pomoci klesá. Náš zbor je verný v~tejto službe, ale vidíme, že celkové pokrytie tejto služby KvM dobrovoľníkmi je čoraz náročnejšie.

Verím tomu, že služba tým najzraniteľnejším je dôležitá. Pre mnohých môže byť naša teplá polievka jediným teplým jedlom za niekoľko dní. Táto služba je spojená aj s~výdajom šatstva a otvoreným priestorom pre zvestovanie Slova Božieho.

\autor{Slávka Kráľová}


\clanok{Zakladania nového zboru -- Connect}

Rok 2024 sme venovali hlavne vnútornému budovaniu Connectu. Absolvovali sme tri víkendy na chate, kde sme prehlbovali svoje poznanie základných princípov zdravej cirkvi.
Jednotlivci sa podieľali na pravidelných pouličných evanjelizáciách v~perióde raz za mesiac. Narástol aj tím služobníkov a služobníčok zabezpečujúcich nedeľné Connecty.
Starostlivosť o~deti sme rozšírili o~Junior Connect, ktorý je súčasťou našich nedeľných Connectov. Neosvedčilo sa nám osobitné stretnutie detí počas týždňa.
Ustanovili sme staršovstvo v~počte troch osôb a pokračujeme v~práci na zborovom poriadku.
V~roku 2025 plánujeme jednodňové akcie zamerané na oslovovanie nových ľudí.
Modlíme sa za záchranu ľudí a aby Pán Boh požehnával svoju cirkev a vrámci nej aj Connect novými ľuďmi.

\autor{Tomáš Valchář}
\vfill\break


\clanok{Revízia hospodárenia}

Revízna komisia v~zložení Miroslav Antalík, Helena Mikletičová, Barbora Antalíková za spolupráce účtovníčky zboru Ľubomíry Kohútovej vykonala revíziu hospodárenia za rok 2024.

Boli prekontrolované nasledovné doklady:
\begitems
* výpisy z~bežného účtu vedeného v~Slovenskej sporiteľni za mesiace 2, 3, 6, 8, 10 a 12
* výdavkové a príjmové pokladničné doklady za mesiace 2, 3, 6, 8, 10 a 12
\enditems
Revízna komisia konštatuje, že uvedené doklady sú vedené prehľadne v~súlade s~účtovnými predpismi. Pokladničná kniha je vedená mesačne a založená priamo pri pokladničných dokladoch.

Neboli zistené žiadne nedostatky.

Stav finančnej hotovosti ku dňu 31.~12.~2024 bol:

\vskip1em\hskip1cm\table{lr}{
pokladňa 		&   2~783,09~€ \cr
bankový účet 	& 198~395,35~€ \crl
spolu 			& 201~178,44~€ \cr
}\vskip1em

Tento stav súhlasí so stavom v~účtovnej evidencii k~uvedenému dátumu.


\autor{Miroslav Antalík}

\tiraz
\bye
