\def\velkostpisma{9}
\def\velkostriadku{12}
\input makra.tex % nacitanie Ivanom pripravenych nastaveni a prikazov
\hyphenation{star-šov-stvo} % rozdelenie slov na konci riadku, treba tu uviest slova, ktore sam nepozna

\spravodaj{2}{2022}


\clanok {Pastier stáda}

Keď som mal 20 rokov, viedol som spolu s~inými tím ľudí, ktorý slúžil v~zboroch v~USA. Jedného dňa mi vedúci povedal, že odchádza slúžiť do inej organizácie. Trochu som sa zľakol, hoci som vedel, že raz tento deň príde. Poznal som organizáciu a vedel som, čo je potrebné na vedenie tejto služby. Nevedel som si však predstaviť, že by som vedel robiť službu bez vedúceho. V~tej chvíli Boh mi potvrdil, že ma povoláva do novej úrovne zodpovednosti a odovzdania sa. Prijal som túto výzvu a vstúpil som do plodnej služby, ktorá trvala niekoľko rokov. Tento scenár sa potom v~mojom živote opakoval znova a znova.

Je to normálne, pretože Boh zo svojej večnej perspektívy koordinuje všetko tak, aby sa naplnila Jeho vôľa, a to nielen v~širšom kontexte cirkvi, ale aj v~životoch jednotlivcov, ktorých sa to týka. Boh to opäť robí v~mojom živote, hoci ja som ten, kto odchádza.
Ako pastier tohto stáda prežívam obavy z~toho, čo sa v~našom zbore stane a kto ho bude viesť.
Pán cirkvi mi ale pripomína, že nie som sám pastierom tohto stáda. Ježiš, hlavný pastier, neodchádza. Zbor Palisády som viedol len podľa toho, čo odo mňa chcel Najvyšší pastier. Tento zbor patrí Jemu. On zostane Pánom nadchádzajúcich dní, mesiacov a rokov.
Tak isto viem, že som len jedným zo skupiny pastierov, ktorú nazývame staršovstvo. Mal som svoju úlohu, ale každý zo starších tiež. Sú to muži, ktorých som si zamiloval a vážil. Pevne verím, že budú aj naďalej tento zbor viesť tak, ako ho viesť treba. Nakoniec si pripomínam, že som len jedným z~celého tela, ktoré Boh spojil, aby s~milosťou a krásou fungovalo. On je hlava a my sme len časti. Ako hlava sa stará o~každú z~častí, stavia ich, lieči ich, prinúti ich spolupracovať na vykonaní krásnej práce, ktorú možno vykonať len vtedy, keď každá časť pracuje na svojej úlohe. On je hlava, z~Neho rastie celé telo, pevne spojené vzájomne sa podporujúcimi kĺbmi a buduje sa v~láske podľa toho, ako je dané každej časti (Ef. 4,16).

To najdôležitejšie, čo čaká zbor Palisády, nie je to, kto bude ďalším kazateľom alebo ako dopadnú voľby starších. Keď sa pozeráme do budúcnosti, najdôležitejšou vecou je vedieť, k~čomu každého z~vás Boh volá. Ste súčasťou tohto cirkevného zboru z~večných dôvodov. Boh si ťa chce použiť na uskutočnenie svojho cieľa a to: vytvoriť ten najláskavejší kostol
v~Bratislave. Bez teba to nepôjde. Si palcom, či polovicou pľúc alebo kolenom a bez toho, aby si fungoval naplno, sa Božia vôľa nedosiahne. Niekto mi nedávno povedal o~tom, do čoho si myslí, že ho Boh v~rámci služby v~zbore povoláva. Minulý týždeň mi poslal správu, že urobil prvý krok k~realizácii tohto plánu. Moje srdce bolo naplnené radosťou.
Ak každý z~vás, ako časť tela, požiada Boha, aby mu ukázal, čo má robiť, potom celé telo bude prosperovať a naplní Boží plán pre tento zbor. Ani jeden z~nás to nedokáže sám.
Ježiš povedal, že bez Neho nemôžeme nič urobiť (Ev. Jána 15). On nám dáva kľúč k~uskutočneniu svojho plánu. Kľúčom je denne prebývať v~Ňom tak, ako konárik zostáva spojený s~viničom. Robíme to denne prostredníctvom našich rozhodnutí, ako trávime čas a kam dávame svoju energiu. Je to určené úrovňou, do akej sa živíme Božím Slovom a trávime s~Ním čas v~modlitbe. Je to určené úrovňou, do ktorej dovolíme Duchu, aby nás viedol cez našu každodennú rutinu. Je to určené úrovňou, v~ktorej žijeme v~závislosti od Ježiša. A~keď sa to stane, Pán Ježiš koná. Ev. Jána 15,5 -- Ja som vinič a vy ratolesti. Kto zostáva vo mne a ja v~Ňom, prináša veľa ovocia, pretože bezo mňa nemôžete nič urobiť. Prinesieme veľa ovocia. Je to sľub od Krista. Toto je mojou výzvou pre každého z~vás v~zbore na Palisádach. Môžeš zohrať novú úlohu pri vytváraní najláskavejšieho zboru v~Bratislave. Zbor, kde ľudia, ktorí sú ďaleko od Boha, nachádzajú nádej, lásku a uzdravenie z~hriechu, ktorý im ničí život.

Najlepšie dni pre Palisády sú pred nami. Ak je náš odchod to najlepšie pre Elliota, je to zároveň aj to najlepšie pre zbor. Verím tomu. Nepremeškaj svoju kľúčovú úlohu v~tom, čo má Boh v~zásobe. Chce ťa použiť na svoju slávu. Nikdy o~tom nepochybujte a dajte Mu to, po čom najviac túži -- po vás všetkých. Povzbudzujem vás teda bratia, pre Božie milosrdenstvo, aby ste odovzdávali svoje telá ako živú, svätú, Bohu príjemnú obetu, ako vašu rozumnú službu Bohu.

Na Jeho slávu v~tomto meste,


\autor{Danny Jones}


\clanok {Správy zo staršovstva -- december 2021}

V decembri sme mali dve stretnutia. Na týchto stretnutiach sme:
\begitems
* Pokračovali v~krokoch súvisiacich s~procesom hľadania ďalšieho kazateľa. Oslovili sme kandidátov, ktorí boli navrhnutí členmi zboru. Ich rozhodnutie kandidovať je v~procese hľadania Božej vôle, modlitieb a rozhodovania.
* Venovali sme sa situácii v~rodine Dannyho a Clary. Nakoľko Danny pravidelne píše listy, sme všetci rovnako informovaní o~ich situácii a hľadaní cesty, ktorú pre nich pripravil náš Pán.
* Spolu s~Viktorom Potockým sme diskutovali o~práci s~Ukrajincami. Táto skupina sa skladá z~dvoch skupín. Jednou sú stabilní členovia, ktorí sú na Slovensku dlhodobo a plánujú svoju budúcnosť v~našom regióne. Druhou sú tí, ktorí k~nám prídu na krátke obdobie práce a potom sa vracajú späť na Ukrajinu. Z~tohto dôvodu je aj práca medzi nimi podriadená týmto faktorom. Tí, ktorí sú na Slovensku krátko, vďaka obmedzenému stretávaniu sa v~posledných rokoch nevnímajú náš zbor ako svoj domáci. Je to pochopiteľné, a preto je na nás, aby sme aj pre nich vytvorili miesto prijatia s~pocitom domova.
* Venovali sme sa výsledkom hlasovania KDZ a pripravovaným voľbám do Rady a orgánov BJB.
* S Ľubošom Kešjarom sme hovorili o~službe technikou v~zbore. Najmä posledné dva roku ukazujú, že táto služba je viac ako službou. Vyžaduje veľa času, nasadenia, učenia sa pracovať s~technikou, ale aj pochopenia zo strany nás prijímateľov tejto služby, že všetci, čo slúžia týmto spôsobom, sú značne vyťažení. Spolu s~technikmi preto hľadáme najvhodnejší model tejto práce.
* Pripravovali sme zabezpečenie služby kostolníka. Primárne sme túto otázku posunuli na diakonov zboru. Je však zrejmé, že v~preklenovacom období budeme spoločne všetci členovia zboru musieť priložiť ruku k~tejto práci.
* Venovali sme sa aj sedeniu v~modlitebni. Táto téma je v~procese tak, ako sme boli na ZČZ informovaní zo strany pracovnej skupiny. Od nás očakávajú návrhy a pripomienky k~tejto téme.
\enditems

\autor {Peter Pribula}


\clanok {Správy zo staršovstva -- január 2022}

Je dobrým zvykom v~našom zbore si posledný deň starého roku na večernom spoločnom stretnutí, alebo v~prvých dňoch nového roku vybrať veršík zo Slova Božieho, ktorý môže byť pre nás určeným svetlom na cestách nového roku, pričom staršovstvo dostalo na rok 2022 text zo Žalmu 50, 23.~verš: Ten, kto obetuje chválu, ma ctí, a tomu, kto pozoruje na svoju cestu, ukážem spasenie Božie.

V~januári malo staršovstvo svoje dve stretnutia, pričom prvé z~nich bolo cez Zoom, ale druhé už mohlo byť z~dôvodu lepšieho vývoja pandemickej situácie organizované prezenčne. Venovali sme sa dôležitým otázkam najbližšieho obdobia, ktoré budú od nás vyžadovať zvýšenú angažovanosť, nasadenie a spoluprácu všetkých členov nášho zboru. Čaká nás spoločne opäť po nie dlhej dobe proces voľby kazateľa, ale takisto aj pravidelné voľby staršovstva zboru a taktiež niektoré procesy v~zborovom živote, s~ktorými sa obvykle často nestretávame. Pôjde najmä o~nie príjemný a vôbec nie chcený proces zániku desiatky rokov trvajúcej práce na našej zborovej stanici Chvojnica a na druhej strane proces vzniku zborovej stanice ukrajinskej služby, ktorá sa v~našom zbore v~ostatnom čase sľubne rozvíja. Plánujeme aj k~týmto témam v~blízkej dobe veľmi dôležité dve zborové členské zhromaždenia, prvé na 20.~2. a druhé výročné na 27.~3.~2022, na ktorých bude veľmi potrebná osobná účasť čo najväčšieho počtu členov zboru.

Nemenej dôležitou oblasťou života zboru podliehajúcou zvýšenej pozornosti staršovstva sú pravidelné spoločné zhromaždenia, ktoré sa za ostatné dva roky z~dôvodu pandémie zúžili iba na nedeľné dopoludňajšie bohoslužby a to mnoho ráz iba online formou. Zvýšená pozornosť je nasmerovaná najmä na snahu staršovstva opätovne vzbudiť a zvýšiť návštevnosť našich spoločných zhromaždení, ktorá nedosahuje ani platné úradne dovolené možnosti počtu zúčastnených. V~mene staršovstva chcem aj týmto povzbudiť všetkých členov a priateľov nášho zboru k~modlitbám za upevnenie našej viery v~Pána Ježiša Krista aj v~tomto inom období, ako aj
k~prosbám za posilnenie našej spolupatričnosti k~zborovej rodine.


\autor {Miroslav Kolářik}


\clanok {Nedeľné bohoslužby v~súčasnom období}

Vyhláška k~organizácii hromadných podujatí -- 2/2022 V.v.~SR s~účinnosťou od 12.~1.~2022 nám umožnila začať s~prezenčnými bohoslužbami. Konajú sa v~režime OP, kde majú účastníci prekryté horné dýchacie cesty a obsadia najviac 25 percent priestoru alebo sa zúčastní maximálne 100 osôb. Nemusíme sa vopred zapisovať do prezenčných tabuliek. Pred začiatkom je však stále potrebné podpísať vyhlásenie o~OP. Naďalej odporúčame nezúčastnovať sa bohoslužby, ak máte akékoľvek príznaky. Odkaz pre sledovanie živého prenosu je pravidelne zaradený v~oznamoch.


\clanok {Besiedka, dorast,  mládež}

Malá aj veľká besiedka začala svoju činnosť 23.~1.~2022. Deti sa pred stretnutím testujú v~domácom prostredí a okrem malých detí, majú prekryté horné dýchacie cesty respirátorom.

Dorast sa stretol online, na spoločnej prechádzke a už aj prezenčne na Súľovskej~2. Rodičia dorastencov dostávajú pravidelné informácie o~stretnutiach e-mailom.

Mládež sa začala s~prezenčnými stretnutiami 22.~1.~2022 na Súľovskej.


\clanok {Biblické hodiny}
Vo februári začali fungovať aj obe biblické hodiny vo svojich tradičných časoch a miestach.


\clanok {Klubík}
Mamičky s~malými deťmi mali svoje posledné stretnutie s~Clarou v~utorok 1.~2.~2022 na Zrínskeho~2.
Ostatné stretnutia si zatiaľ budú organizovať podľa potreby samé.


\clanok {Sesterské stretnutia}
Po pár týždňoch prestávky kvôli súčasnej situácii sa sestry prvýkrát stretli po novom roku 2.~2.~2022. Spolu s~Dankou Paštrnákovou a jej nevestatmi hovorili na tému: Vzťah medzi svokrou a nevestou. V~stredu 9.~2.~2022 pokračujú so sestrou Dankou témou: Svedectvo matky a syna, ktorý mal problém so závislosťou.


\clanok{Verš na zapamätanie}
Na mesiac február máme nový veršík, ktorý sa chceme spoločne učiť. Veríme, že poznanie Písma prospeje našej duši i~našej mysli:

{\it „Kto je múdry, nech tieto veci pochopí a kto je chápavý, nech ich spozná. Lebo priame sú cesty Hospodinove; spravodliví po nich kráčajú, no hriešnici sa na nich potkýnajú.“}

\autor{Ozeáš 14,10}
\vfill\break


\clanok {Služba núdznym}
Milí bratia a sestry, do pozornosti Vám dávam voľné termíny na varenie polievky núdznym: 22.~2. a 26.~2.~2022. V~prípade záujmu ma môžete kontaktovať telefonicky 0908~046~409.

\cast{Pridajte sa k nám!}
Ak by ste mali čas a chuť slúžiť núdznym, veľmi uvítame vašu pomoc a zapojenie najmä pri večerných výdajoch stravy pod mostom Lafranconi. Ak sa vy osobne chcete zapojiť, alebo viete o~niekom, kto by sa rád zapojil, budeme vďační za zdieľanie tejto správy. V~prípade záujmu nás neváhajte kontaktovať na t.č. 0948~115~515 alebo e-mailom: \email{antalikova@krestaniavmeste.sk}

\cast{Zbierka šatstva}
Šatstvo preberáme každý štvrtok od~16.00 do~19.00~hod. na Ambroseho~6. Pred prinesením vecí, prosím, kontaktujte Sylviu Vaniherovú, koordinátorku zbierok šatstva, na t.č. 0905~484~675.
Prosím, nenoste nám šatstvo priamo pod most Lafranconi. Ďakujeme!

Zbierame:
\vskip-1ex\begitems
* pánske oblečenie: teplé zimné bundy, tenisky, zimné topánky, mikiny, tričká, spodné prádlo, rifle, deky, spacáky, čiapky, rukavice;
* hygienické potreby: sprchové gély, šampóny, pracie prášky, zubné pasty a kefky, jednorázové žiletky na holenie, hygienické vreckovky;
* suché potraviny: paštéty, rybie konzervy, konzervovanú zeleninu a strukoviny, čaje, kávu, cukor, napolitánky, instatné polievky a iné.
\enditems
\vskip-1ex
\autor{Beáta Bogárová}


\clanok{Zbierky za uplynulé obdobie}
Milí bratia a sestry, ďakujeme za vašu obetavosť. V~mesiaci december a január ste prispeli:

\vskip-1ex\begitems
* Misia: december 2021: 256,00 € / január 2022: 581,30 €
* Investície: december 2021: 256,00 € / január 2022: 581,30 €
\enditems

Aj naďalej máte možnosť prispieť do „nedeľnej zbierky“, a to prevodom na účet zboru. Do poznámky pre prijímateľa, prosím, uveďte „zbierka“.

Bankové spojenie: SK36 0900 0000 0000 1147 1836, SWIFT: GIBASKBX

Ďakujeme!


\n  3.	2.	Vlasta	BALÁŽOVÁ;
\n  3.	2.	Margita	KRÁĽOVÁ;
\n  3.	2.	Miroslav	ANTALÍK;
\n  5.	2.	Štefánia	ANTALÍKOVÁ;
\n  5.	2.	Barbora	ANTALÍKOVÁ;
\n 11.	2.	Juraj	BALÁŽ;
\n 11.	2.	Oľga	KOVÁČOVÁ;
\n 11.	2.	Beáta	BOGÁROVÁ;
\n 12.	2.	Martin	PRIBULA;
\n 13.	2.	Zlatica	VYSKOČILOVÁ;
\n 15.	2.  Inrid	JANČULOVÁ;
\n 16.	2.  Lenka	PRIBULOVÁ;
\n 23.	2.  Anna	PLETT;
\narodeniny


\program{
\p  1 ; ut ;.;;.;;
\p  2 ; st ; 17.30 ; Stretnutie sestier;.;;
\p  3 ; št ; 18.00 ; Biblická hodina (J. Szőllős);.;;
\p  4 ; pi ; 17.30 ; Dorast (Súľovská 2);.;;
\p  5 ; so ; 18.00 ; Mládež (Súľovská 2);.;;
\p  6 ; ne ;  9.30 ; Bohoslužby (D. Jones);.;;
\p  7 ; po ;.;;.;;
\p  8 ; ut ; 15.15 ; Biblická hodina pre seniorov (P. Pivka);.;;
\p  9 ; st ; 17.30 ; Stretnutie sestier ;.;;
\p 10 ; št ; 18.00 ; Biblická hodina (J. Szőllős);.;;
\p 11 ; pi ; 17.30 ; Dorast (Súľovská 2);.;;
\p 12 ; so ; 18.00 ; Mládež (Súľovská 2);.;;
\p 13 ; ne ;  9.30 ; Bohoslužby (S. Baláž);.;;
\p 14 ; po ;.;;.;;
\p 15 ; ut ; 15.15 ; Biblická hodina pre seniorov (P. Pivka);.;;
\p 16 ; st ;.;;.;;
\p 17 ; št ; 18.00 ; Biblická hodina (J. Szőllős);.;;
\p 18 ; pi ; 17.30 ; Dorast (Súľovská 2);.;;
\p 19 ; so ; 18.00 ; Mládež (Súľovská 2);.;;
\p 20 ; ne ;  9.30 ; Bohoslužby (P. Pribula);.;;
\p 21 ; po ;.;;.;;
\p 22 ; ut ; 15.15 ; Biblická hodina pre seniorov (P. Pivka);.;;
\p 23 ; st ;.;;.;;
\p 24 ; št ; 18.00 ; Biblická hodina (J. Szőllős);.;;
\p 25 ; pi ; 17.30 ; Dorast (Súľovská 2);.;;
\p 26 ; so ; 18.00 ; Mládež (Súľovská 2);.;;
\p 27 ; ne ;  9.30 ; Bohoslužby (T. Hanes);.;;
\p 28 ; po ;.;;.;;
}


\tiraz
\bye
