%\typosize[10/12.5]% - pouzita velkost pisma/riadku (trochu vacsie)
%v programe je vela poloziek, musel som znizit vysku riadku: \vrule height2.4ex% -> \vrule height2.2ex%
\input makra.tex % nacitanie Ivanom pripravenych nastaveni a prikazov
\hyphenation{star-šov-stvo} % rozdelenie slov na konci riadku, treba tu uviest slova, ktore sam nepozna

\spravodaj{11}{2019}


\clanok {Kľučka na dverách}
Slúžili sme s~Clarou na misijnej konferencii v~Modre.  Hlavná myšlienka víkendu bola, že Boh otvára a zatvára dvere, a tým nás vedie. Jedna z~tém, ktorú som mal, bola dobrodružstvo v~službe. Keď žijeme odovzdaní Bohu, vôbec nevieme, čo nás čaká. Nečakane sa nám otvoria nejaké dvere -- musíme s~tým počítať a byť pripravení prispôsobiť sa. Apoštol Pavol mal známu kázeň v~Aténach na Aeropágu. V~tomto príbehu vidím dva dôležité momenty. V~Sk 17,16 sa píše: „Medzitým, čo Pavol čakal na nich...“ Neviem, či ostatní meškali alebo mal voľný deň, ale otvorili sa mu dvere času. Nevyčkával, čo má robiť, ale hľadal možnosti. Druhý moment je, že cestou po meste sa obzeral a všímal si, čo je naokolo; rozmýšľal a hľadal kľučku, ktorá by mu otvorila dvere pre službu. Zbadal oltár Neznámemu bohu. Verím, že Duch Svätý mu pošepkal, že toto je tá kľučka, ktorou otvorí dvere. Som presvedčený, že aj my zažívame takéto momenty ako Pavel a musíme byť na ne citliví. Tak ako vtedy v~Aténach, aj v~Bratislave sú hľadajúci, na ktorých Pán Boh pracuje. Pracujeme s~nimi. Športujeme s~nimi. Nakupujeme s~nimi. Bývame vedľa nich. Boh ich pripravuje, ale omylom predpokladáme, že to tak nie je alebo že na to nemáme.

Russ, misionár, ktorý nedávno navštívil náš zbor, sa už presťahoval s~rodinou do Jordánska. Prežil, že niektorému zo susedov sa sníval sen, ktorý by mohol otvoriť dvere evanjeliu. Začal sa pýtať susedov, či sa im nesnívalo niečo zvláštne. Jeden mu povedal: „Áno, snívalo sa mi, že jordánsky princ ma stretol a pozval ma ku kráľovi do paláca. Spýtal som sa ho, či by som mohol so sebou zobrať aj rodinu a súhlasil. A~tak sme všetci išli ku kráľovi. Neviem, čo to znamená.“ Russ mu vysvetlil, že nebeský Princ prišiel na zem a ho pozýva aj s~rodinou do paláca k~nebeskému Kráľovi. Niekoľkokrát tú rodinu navštívili a nedávno s~nimi začali doma biblické štúdium. Tou kľučkou bol sen. Russ sa nechal viesť Duchom Svätým a použil kľučku, aby sa dostal dnu.

Aké sú v~našom okolí dvere a ktorými kľučkami by sme sa mohli dostať dnu? Pán má na túto otázku odpoveď a keď budeme hľadať a budeme pripravení do toho ísť, ukáže nám ich. Jednu kľučku do dvoch škôl budeme môcť použiť 14.~--~15.~novembra, keď s~Petrom Makovínim z~Dánska príde skupina študentov. Modlime sa za otvorené dvere do škôl, k~študentom a ich rodinám. A~medzitým sa pýtajme Pána, či by nám neotvoril dvere, aby sme mali príležitosť svedčiť. Určite na to odpovie! Nebuďme prekvapení, ale radšej pripravení!

\autor{Danny Jones}


\clanok {Správy zo staršovstva}
Bratia a sestry, chcem vás pozdraviť aj touto cestou tým pozdravom, ktorý povedal Pán Ježiš, keď sa po vzkriesení stretol so svojimi učeníkmi: „Pokoj vám“. Pán Ježiš vedel, že v~situácii, v~ktorej sa nachádzali, potrebovali nadprirodzene zažiť to, čo im aktuálna situácia nedovoľovala. Boli plní neistoty a otázok, čo bude ďalej. Ale aj my dnes potrebujeme zažívať, že v~dobe, ktorá je rýchla a v~mnohých ohľadoch neistá, nájdeme miesto plné nadprirodzeného pokoja; pokoja, ktorý môže dať iba Pán Ježiš.

V uplynulých týždňoch sme s~kazateľom a staršími uvažovali a modlili sa za múdrosť v~mnohých témach. Niektoré z~nich spomeniem.

Hľadáme spôsob ako najvhodnejšie podporiť misiu, ktorá je zameraná minimálne dvomi smermi. Jedným je zakladanie nového zboru v~Ružinove. Rozprávali sme s~bratom Tomášom Valchářom o~službe, ktorú robia, o~ich práci a aj o~našej podpore tejto služby. Sme vďační nášmu Pánovi, že služba v~Ružinove napreduje a sú ľudia, ktorí sa pýtajú na cestu za Pánom Ježišom a na to, ako byť poslušný Jeho slovu. Druhým smerom je podpora služby medzi Ukrajincami žijúcimi na Slovensku, konkrétne v~Bratislave, a s~tým súvisiaca podpora brata Viktora Potockého v~tejto službe.

Zároveň myslíme aj na členov nášho zboru. Hľadáme spôsob ako podporiť manželské páry, aby zdravo budovali svoje vzťahy na jedinom trvalom kameni, na Pánovi Ježišovi. Pripravujeme víkend pre manželské páry, ktorý by sme chceli realizovať vo februári budúci rok. Rovnako pripravujeme víkend aj pre mužov z~nášho zboru, ktorý by sa mal uskutočniť v~apríli 2020.

Na november sme pripravili viacero podujatí rôzneho zamerania. Pre záujemcov o~apologetiku, obranu viery pred neveriacimi, sa bude 8. a 9. novembra konať prednáška brata T.~Woodwarda v~našej modlitebni. Pre tých, ktorí majú záujem poznávať korene, na ktorých je postavený aj baptizmus, bude 10.~novembra prednáška brata K.~Mészárosa o~anabaptizme a anabaptistoch. Opakovane už po niekoľkýkrát prídu mladí bratia a sestry z~biblickej školy v~Dánsku a budú rozprávať o~škole a svojej práci. Termín ich návštevy je plánovaný na 16.~--~17.~novembra.

Okrem programových záležitostí sme sa venovali aj otázkam členov a zložiek zboru. Rodičia dorastencov a vedúci dorastu sa dohodli na zmene termínu stretnutia. Dorastenci sa stretávajú v~nedeľu poobede. Niekoľkokrát sme sa venovali téme doplnenia staršovstva o~jedného člena. Pastoračné otázky sú pravidelne na našom jednacom stole rovnako ako na našich modlitbách. Nakoniec spomeniem, ale s~rovnakou vážnosťou ako iné témy, situáciu vo vedení tímu diakonov. Tejto téme sa venujeme opakovane už niekoľkokrát.

Všetko, čo som spomenul, má spoločného menovateľa. Považujeme tieto témy za dôležité a vieme, že sami sme neschopní ich vyriešiť. Pre ich riešenie potrebujeme Božie vedenie a Jeho múdrosť. Preto prosím aj vás, aby ste sa modlili za múdrosť a Božie vedenie pre kazateľa, staršovstvo ale aj pre každého, kto s~úprimným srdcom hľadá cestu za Pánom Ježišom.

\autor {za staršovstvo zboru Peter Pribula}


\clanok {Doplňujúce voľby do staršovstva}
Staršovstvo zboru sa rozhodlo pristúpiť k~doplňujúcej voľbe do staršovstva. Prosíme vás, aby ste sa za túto voľbu modlili, aby nás Pán viedol v~tomto rozhodovaní. Prvé novembrové nedele 3. a 10.~novembra bude prebiehať prieskum, kedy členovia volebnej komisie budú rozdávať lístočky, na ktoré budete môcť napísať svoje návrhy. Návrhy môžete posielať aj elektronicky, a to na emailovú adresu \email {starsovstvo@bjbpalisady.sk}. Počet navrhovaných kandidátov nie je obmedzený. 24.~novembra plánujeme zborové členské zhromaždenie, na ktorom okrem iného členovia zboru schvália konečnú kandidátku.

\autor {staršovstvo zboru}


\clanok {Krst v~zbore}
V našom zbore plánujeme krst na vyznanie viery, ktorý sa bude konať 17.~novembra o~15.00~hod. v~Miloslavove.


\clanok{Spoločné modlitby}
\vskip-1ex\begitems
* Muži -- streda {\bf od 6.00~hod. do 7.00~hod.}, kostol na Palisádach
* Ženy -- pondelok {\bf od 17.00~hod.}, Zrínskeho 2
\enditems

Priveďte na spoločné modlitby aj svojich priateľov a známych, ktorým leží na srdci naše mesto a ľudia v~ňom.


\clanok {Evolúcia alebo inteligentný dizajn?}
V piatok 8. novembra v~čase 19.00~--~21.00~hod. a v~sobotu 9. novembra v~čase 14.00~--~18.00~hod. sa v~našej modlitebni na Palisádach budú konať prednášky Dr.~Toma Woodwarda, profesora z~Trinity College na Floride a prezidenta C.~S.~Lewis Society. Tom už niekoľko rokov vedie požehnanú službu v~oblasti kresťanskej apologetiky a v~rámci nej robí prednášky po celom svete o~inteligentnom dizajne a nedostatkoch evolučnej teórie.

Viac informácii o~Tomovi Woodwardovi a jeho službe si môžete prečítať na webovej stránke \ulink[https://apologetics.org/about-us/dr.-tom-woodward/]{apologetics.org/about-us/dr.-tom-woodward/} (v~angličtine).


\clanok {Príbeh anabaptistov -- mučeníkov}
V nedeľu 10.~novembra o~15.00~hod. na Palisádach sa uskutoční prednáška cirkevného historika Kálmána Mészárosa z~Budapešti na tému {\it Príbeh anabaptistov -- mučeníkov}. Je to príležitosť dozvedieť sa viac o~novokrstencoch, ktorí sú aj našimi duchovnými predchodcami.


\clanok {Stretnutie sestier}
Milé sestry,

srdečne vás pozývam na sesterské stretnutie, ktoré bude 13.~novembra o~17.30~hod. v~modlitebni na Palisádach. Toto stretnutie bude výnimočné v~tom, že Ester Jankovičová mi pomôže s~prekladom a takisto spolu predstavíme, ako viesť diskusnú skupinu.

Témou stretnutia bude {\it Oddychujeme pri spoznávaní Božieho vedenia}.

S láskou pre vás všetkých,

\autor {Clara Jones}


\clanok {Služba núdznym}
Milé sestry a bratia,

momentálne pociťujeme naliehavý nedostatok výdajových tímov / alebo aj jednotlivcov, ktorí by boli ochotní pravidelne sa do služby zapojiť. Pod pojmom pravidelne myslím aj pravidelnosť 1x za mesiac, ako to komu vyhovuje. Verím, že služba núdznym je niečo, čím nás Pán poveril a čo nám dal za úlohu. Je to služba, v~ktorej ovocie nie je veľmi vidieť, ale ktorú potrebujeme ľuďom na ulici priniesť. Oni prišli o~nádej, o~vieru, o~dôstojnosť, ale zúfalo túžia po láske a po prijatí. Stačí len prísť a ponúknuť im milosť prijatia tým, že ich neodsúdime a ešte im poslúžime jedlom a pitím.

Ktokoľvek by mal záujem si to vyskúšať, kontaktujte, prosím, Lenku Antalíkovú \email {antalikova@krestaniavmeste.sk}, všetko vysvetlí. Vďaka!

\autor {Beáta Bogárová}


\clanok {Senior klub}
Posledný senior klub západnej oblasti v~tomto roku, z~Božej milosti, plánujeme uskutočniť posledný štvrtok v~mesiaci, t.~j. 28.~11. v~čase 10.00~--~14.00~hod. na Súľovskej ul.

Téma je vianočná -- narodenie nášho drahého Pána Ježiša Krista.

Všetci sú srdečne pozvaní.

Boží pokoj a Jeho milosť nech Vás sprevádza každý deň.

\autor {Jana Makovíniová}


\clanok{Verš na zapamätanie}
Na mesiac november máme nový veršík, ktorý sa chceme spoločne učiť. Veríme, že poznanie Písma prospeje našej duši i našej mysli:

\vskip0em\par% Tieto prikazy su ako workaround kvoli zvlastnej veci - bez nich je font paticky predoslej strany ovplyvneny fontovym prikazom z nasledujuceho riadku. Neviem, cim je to sposobene, ale nemam cas to riesit, takze mi staci tento workaround.
{\it „Pre nič nebuďte ustarostení, ale vo všetkom s~vďakou predkladajte Bohu svoje žiadosti vo svojich modlitbách a prosbách. A~pokoj Boží, ktorý prevyšuje každý rozum, uchráni vaše srdcia a vaše mysle v~Kristovi Ježišovi.“}

\autor{Filipanom~4,~6~--~7}


\clanok{Zbierky za uplynulé obdobie}
Milí bratia a sestry, ďakujeme za vašu obetavosť. V~uplynulom období ste prispeli:
\vskip-1ex\begitems
* investičný fond: 698,50~€ (september), 303,00~€ (október)
* misia: 1555,00~€ (október)
\enditems


\n 2.	11.	Tomáš	VALCHÁŘ;
\n 5.	11.	Katarína	VALENTOVÁ;
\n 6.	11.	Elena	PRIBULOVÁ;
\n 6.	11.	Eva	SYČOVÁ;
\n 9.	11.	Alžbeta	BETKOVÁ;
\n 9.	11.	Radovan	PAULEN;
\n 15.	11.	Bohumila	ŠALINGOVÁ;
\n 18.	11.	Jelka	NEVICKÁ;
\n 19.	11.	Dávid	PRIBULA;
\n 21.	11.	Ladislav	KAMOCSAI;
\n 22.	11.	Alena	SVOBODOVÁ;
\n 22.	11.	Peter	PRIBULA;
\n 24.	11.	Vojtech	PAULEN;
\n 24.	11.	Magdaléna	ŠALLINGOVÁ;
\n 25.	11.	Petra	ŠALINGOVÁ;
\n 27.	11.	Judita	KOLÁŘIKOVÁ;
\n 29.	11.	Jaroslav	KRÁĽ;
\narodeniny


\program{
\p  1 ; pi ;.;;.;;
\p  2 ; so ; 18.00 ; Koncert hudobnej skupiny Revival ;.;;
\p  3 ; ne ;  9.30 ; Bohoslužby (Revival); 10.00 ; Chvojnica (J. Szőllős) ;
\p  4 ; po ; 17.00 ; Modlitby -- ženy (Zrínskeho 2) ;.;;
\p  5 ; ut ; 15.15 ; Stretnutie pri Biblii (P. Pivka, Zrínskeho 2) ;.;;
\p  6 ; st ;  6.00 ; Modlitby -- muži (kostol Palisády) ;.;;
\p  7 ; št ; 19.00 ; Biblická hodina (J. Szőllős, Zrínskeho 2) ;.;;
\p  8 ; pi ; 19.00 ; Evolúcia alebo inteligentný dizajn? (Dr. Tom Woodward) ;.;;
\p  9 ; so ; 14.00 ; Evolúcia alebo inteligentný dizajn? (Dr. Tom Woodward) ; 18.00 ; Mládež (Súľovská) ;
\p 10 ; ne ;  9.30 ; Bohoslužby (K. Mészáros) ; 10.00 ; Chvojnica (P. Škulec) ;
\p    ;    ; 15.00 ; Príbeh anabaptistov -- mučeníkov (K. Mészáros) ;.;;
\p 11 ; po ; 17.00 ; Modlitby -- ženy (Zrínskeho 2) ;.;;
\p 12 ; ut ; 15.15 ; Stretnutie pri Biblii (P. Pivka, Zrínskeho 2) ;.;;
\p 13 ; st ;  6.00 ; Modlitby -- muži (kostol Palisády) ; 17.30 ; Stretnutie sestier (C. Jones) ;
\p 14 ; št ; 19.00 ; Biblická hodina (J. Szőllős, Zrínskeho 2) ;.;;
\p 15 ; pi ;.;;.;;
\p 16 ; so ; 18.00 ; Mládež (Súľovská) ;.;;
\p 17 ; ne ;  9.30 ; Bohoslužby (D. Jones) ; 10.00 ; Chvojnica ;
\p    ;    ; 15.00 ; Krst (BJB Miloslavov) ;.;;
\p 18 ; po ; 17.00 ; Modlitby -- ženy (Zrínskeho 2) ;.;;
\p 19 ; ut ; 15.15 ; Stretnutie pri Biblii (P. Pivka, Zrínskeho 2) ;.;;
\p 20 ; st ;  6.00 ; Modlitby -- muži (kostol Palisády) ;.;;
\p 21 ; št ; 19.00 ; Biblická hodina (J. Szőllős, Zrínskeho 2) ;.;;
\p 22 ; pi ;.;;.;;
\p 23 ; so ;.;;.;;
\p 24 ; ne ;  9.30 ; Bohoslužby (V. Potocký); 10.00 ; Chvojnica (P. Škulec) ;
\p    ;    ; 16.00 ; Zborové členské zhromaždenie ;.;;
\p 25 ; po ; 17.00 ; Modlitby -- ženy (Zrínskeho 2) ;.;;
\p 26 ; ut ; 15.15 ; Stretnutie pri Biblii (P. Pivka, Zrínskeho 2) ;.;;
\p 27 ; st ;  6.00 ; Modlitby -- muži (kostol Palisády) ;.;;
\p 28 ; št ; 10.00 ; Senior klub (Súľovská) ; 19.00 ; Biblická hodina (J. Szőllős, Zrínskeho 2) ;
\p 29 ; pi ;.;;.;;
\p 30 ; so ; 18.00 ; Mládež (Súľovská) ;.;;
}

\tiraz
\bye
