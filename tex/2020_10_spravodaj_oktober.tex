%\typosize[10/12.5]% - pouzita velkost pisma/riadku - trochu vacsie
%v programe je vela poloziek, musel som znizit vysku riadku: \vrule height2.4ex% -> \vrule height2.2ex%
\input makra.tex % nacitanie Ivanom pripravenych nastaveni a prikazov
\hyphenation{star-šov-stvo} % rozdelenie slov na konci riadku, treba tu uviest slova, ktore sam nepozna

\spravodaj{10}{2020}


\clanok {Počúvame?}
Čítam teraz Prvú knihu Samuelovu. V~tretej kapitole je ten známy príbeh, keď Boh v~noci osloví Samuela. Trikrát ho Boh oslovil a zakaždým bol Samuel zmätený. Nečakal Boží hlas. Až potom, keď Elimu napadlo, že ho zrejme volá Boh, vedel ako reagovať. Ešte pred tým je napísané, že „Slovo Hospodina bolo v~tých časoch vzácne a prorocké videnie zriedkavé“. Neboli na to zvyknutí, že im Boh niečo zjaví alebo povie. Preto ani Elimu hneď nenapadlo, že to hovorí sám Boh.

Keď som si to prečítal, hneď som uvažoval a rozmýšľal nad tým, ako často aj mňa Boh volá – v~noci či cez deň – a čosi mi hovorí, ale nerozoznávam Jeho hlas, lebo to neočakávam a neverím, že mi niečo povie. Samozrejme, komunikuje s~nami cez Písmo, ale Duch Boží v~nás prebýva a vždy nám má čo povedať. Aj Ježiš sám nás na to pripravil: „Keď však príde on, Duch pravdy, {\it uvedie} vás do celej pravdy, lebo nebude {\it hovoriť} sám od seba, ale bude {\it hovoriť}, čo bude počuť. Bude {\it zvestovať} aj to, čo príde“ (Jn~16,13, kurzíva moje).

Ešte stále to platí? Samozrejme! Som presvedčený o~tom, že zvlášť v~tejto čudnej dobe máme očakávať Božie vedenie a nasmerovanie; že k nám bude hovoriť. Ale počúvame s~očakávaním? Niekedy potrebujeme svojho Eliho, ktorý rozozná Boží hlas a bude nám vedieť povedať, ako na to reagovať. Tým mojím Elim môže byť zrelší priateľ alebo niekto z~vedenia zboru. Môže ním byť rodič alebo príbuzný, ktorý žije s~Pánom. Ak sa ti zdá, že ti Pán niečo ukazuje alebo hovorí, neopovrhuj tým a nespochybňuj, že ti niečo dôležité komunikuje. Poraď sa s~niekým, modli sa a odpovedz Bohu ako Samuel: „Hovor, tvoj služobník počúva.“

\autor{Danny Jones}


\clanok {Správy zo staršovstva}
Po letných prázdninách sa zborový život vrátil do starých koľají. Rozbehli sa stretnutia väčšiny zborových zložiek a po letnej prestávke sa staršovstvo stretlo na štyroch stretnutiach. Okrem pravidelných aktivít sme plánovali aj niekoľko mimoriadnych stretnutí (víkendovka sestier, víkend pre manželské páry, zborový pracovný projekt, či víkendovka pre mužov). Narastajúci počet pozitívne testovaných na Covid-19 však zasiahol aj do našich plánov a momentálne sa veľa z~nich nesie v~duchu „uvidíme“. Napriek momentálnej situácii a reštrikčným opatreniam sme vďační, že sa môžeme schádzať a budovať sa navzájom v~láske a poznaní Pána Ježiša.

Na našich stretnutiach sme sa venovali viacerým témam, ktoré sa týkajú života nášho zboru. V~polovici novembra plánujeme krst v~zbore a tešíme sa, že už niektorí prejavili záujem verejne takto vyznať svoju vieru v~Pána Ježiša. Tí z~vás, ktorí by sa ešte chceli dať pokrstiť, môžete dať vedieť br. kazateľovi Dannymu Jonesovi alebo ktorémukoľvek členovi staršovstva. Ešte predtým, 18.~októbra, plánujeme zborové členské zhromaždenie, na ktorom budeme okrem iného prijímať nových členov zboru. Sme radi, že aj napriek zložitej súčasnej situácii si viacero nových ľudí nachádza cestu do nášho zboru.

Na ostatnom stretnutí staršovstva sme sa venovali aj obsadeniu služieb v~poslednom štvrťroku. Veľkú „dieru“ v~súčasnosti prežívame v~oblasti vedenia a moderovania nedeľných zhromaždení ako aj v~oblasti vedenia chvál. Modlíme sa, aby Pán pridal pracovníkov a doplnil rady slúžiacich.

V ostatnom období sme sa intenzívnejšie zaoberali aj komunikáciou nášho zboru smerom navonok. V~spolupráci s~Ľubkou Kováčikovou a Marošom Kohútom pripravujeme novú webovú stránku zboru ako aj nové logo. Veríme, že v~najbližších dňoch ich budeme môcť predstaviť zboru.

Uvedomujeme si, že viacerí členovia a návštevníci zboru sa v~dôsledku epidemiologickej situácie dostali takpovediac na perifériu zborového života. Máme tu druhú vlnu pandémie, od ktorej nevieme, čo čakať a ako nás zasiahne. V~mene staršovstva vám chcem dať na srdce, aby ste nám dali o~sebe vedieť, ak prechádzate ťažším obdobím alebo čelíte nejakým výzvam. Sila spoločenstva sa neprejavuje len v~tom, že sa navzájom vidíme v~zhromaždení, ale aj v~tom, že sa jeden za druhého modlíme a nesieme bremená jeden druhého.

\autor {za staršovstvo Peter Kolárovský}


\clanok {Bohoslužby počas koronakrízy}
Vzhľadom na epidemiologickú situáciu budeme mať každú nedeľu dve zhromaždenia v~čase o~9.00~hod. a 10.30~hod. V~súčasnosti platí obmedzenie účasti na 50~ľudí. Aby sme predišli tomu, že by sa niekto z~kapacitných dôvodov nemohol dostať do modlitebne, je potrebné sa pre účasť na zhromaždení zapísať do tabuľky: \ulink [https://bit.ly/2M5WYbZ]{bit.ly/2M5WYbZ} (každá osoba do samostatného riadku).

Živý prenos bude zabezpečený zo zhromaždenia o~10.30~hod.: \ulink [https://bit.ly/3cgSMBG]{bit.ly/3cgSMBG}.

Zároveň vás chceme vyzvať na dodržiavanie opatrení vydaných Úradom verejného zdravotníctva SR a pri vstupe do modlitebne sa riadiť príslušnými pokynmi. Chceme vás takisto poprosiť, aby ste v~prípade akýchkoľvek príznakov ochorenia do zhromaždenia nechodili (pri vstupe budeme merať teplotu).


\clanok{Spoločné modlitby}
\vskip-1ex\begitems
* Muži -- streda {\bf od 6.00~hod. do 7.00~hod.}, kostol na Palisádach
* Ženy -- pondelok {\bf od 17.00~hod.}, Zrínskeho 2
\enditems


\clanok {Nedeľná besiedka, dorast a mládež}
Stretnutia nedeľnej besiedky budú v~tomto školskom roku nateraz začínať o~10.30~hod. na Zrínskeho. Vzhľadom na bezpečnostné opatrenia budú prebiehať nasledovne:

Deti nám odovzdáte pri vchodových dverách. Deti z~veľkej besiedky vo veku od 7 do 10 rokov budú mať po celý čas rúška. Deti z~malej besiedky vo veku od~3 do~7 rokov (prváci) rúška mať nemusia. Učitelia budú mať takisto rúška. Nakoľko berieme ohľad jedni na druhých, prosíme, aby ste priviedli len zdravé deti bez príznakov prechladnutia. Deti zvládnu besiedku aj bez vás, prosím, nezostávajte s~nimi na besiedke. Po skončení besiedky deti privedieme do zhromaždenia.

Pokiaľ nám to súčasná situácia v~súvislosti s~koronavírusom dovolí, dorastenci sa budú stretávať v~piatok o~17.30~hod. na Zrínskeho a mládež v~sobotu o~18.00~hod. na Súľovskej.


\clanok {Klubík}
Stretnutia mamičiek na materskej so svojimi deťmi sú každý útorok o~9.30~hod. na Zrínskeho.


\clanok {Stretnutia sestier}
Milé sestry,

srdečne vás pozývam na naše spoločné stretnutia, na ktorých budeme pokračovať v~štúdiu {\it Trvalej slobody}. V~októbri plánujeme dve stretnutia, a to 7.~10. a 21.~10., vždy o~17.30~hod.

\autor {Clara Jones}


\clanok {Zborový pracovný projekt}
Zborový pracovný projekt na Strednej priemyselnej škole elektrotechnickej, ktorý sme plánovali v~septembri, sa kvôli zlému počasiu neuskutočnil. Náhradný termín je predbežne určený na 10. októbra.

Kvôli plánovaniu práce vás chceme požiadať, aby ste nám dali vedieť, či sa do tohto projektu zapojíte, a to zapísaním sa do nasledovnej tabuľky: \ulink [https://bit.ly/2zcrzC4]{bit.ly/2zcrzC4}.


\clanok {Krst v~zbore}
V našom zbore plánujeme krst ponorením na vyznanie viery (predbežne 15.~novembra). Tí, ktorí by sa chceli dať pokrstiť, sa môžu prihlásiť u~kazateľa alebo starších zboru.


\clanok {Varenie polievky pre bezdomovcov}
Máme voľné termíny na varenie polievky pre bezdomovcov, a to v~sobotu 17.~októbra, v~sobotu 21.~novembra a v~sobotu 19.~decembra.

Tí z~vás, ktorí by boli ochotní sa zapojiť do tejto služby, kontaktujte, prosím, Beatu Bogárovu na tel. č. 0908~046~409.


\clanok {Služba ľuďom v~núdzi -- hľadáme dobrovoľníkov}
Už dlhšie vnímame potrebu pomáhať sociálne odkázaným ľuďom, preto pre nich pripravujeme otvorenie výdajne potravín v~Bratislave, v~Petržalke.

Do výdajne potravín hľadáme dobrovoľníkov do rôzneho typu služieb:
\vskip-1ex\begitems
* na výdaj potravín na 2 hodiny týždenne (doobeda alebo poobede) -- pravidelne alebo sporadicky
* ľudí s~autom, ktorí by poobede/podvečer mohli vyzdvihnúť potraviny v~supermarkete
* ľudí s~autom, ktorí by zaviezli potraviny do konkrétnej rodiny
* na zbierku potravín v~Tesco Dúbravka a Tesco Podunajské Biskupice v~čase 19.~11.~2020 -- 21.~11.~2020 medzi 8.00 -- 18.00 hod. na doobedie alebo poobedie. Zbierka bude spropagovaná medzi nakupujúcimi, ktorí majú možnosť darovať do našej výdajne potraviny, ktoré nakúpia. (Dobrovoľník je pri nákupnom košíku, do ktorého ľudia darujú potraviny a v~prípade záujmu komunikuje s~ľuďmi).
\enditems

Potrebujeme aj vaše ruky a ochotné srdce. Zapojíme každého. Kontaktujte sa prosím na \email {kancelaria@krestaniavmeste.sk}. Ďakujeme!

\autor {Kresťania v~meste}


\clanok {Bratislava spieva gospel}
Bratislavský zbor BJB Viera spúšťa tretí ročník projektu Bratislava spieva gospel. Tohtoročná sezóna je unikátna v~tom, že k~„živým“ spevákom sa môžu pridať speváci z~celého Slovenska cez tzv. virtuálny spevokol.

Ak rád spievaš, pridaj sa k~nám! Nauč sa pieseň „I Smile“ podľa podkladov, ktoré ti pošleme, natoč sa a pošli nám svoju nahrávku. Zaradíme ju do finálneho videoklipu, ktorý (dúfame) obehne celé Slovensko.

Aj v~roku 2020 máme dôvod na úsmev – zdieľaj túto správu ďalej svojim veriacim i neveriacim priateľom, kolegom, či susedom! Slovensko, smile! :-)

Viac informácií nájdeš na stránke \ulink [https://bratislavaspievagospel.sk/]{bratislavaspievagospel.sk}.
\vfill\break


\clanok{Verš na zapamätanie}
Na mesiac október máme nový veršík, ktorý sa chceme spoločne učiť. Veríme, že poznanie Písma prospeje našej duši i našej mysli:

{\it „Pán mu povedal: ‚Správne, dobrý a verný sluha! Bol si verný nad málom, ustanovím ťa nad mnohým. Vojdi do radosti svojho pána!‘“}

\autor{Matúš~25,~21}


\clanok{Zbierky za uplynulé obdobie}
Milí bratia a sestry,

v septembri ste prispeli:

\vskip-1ex\begitems
* Misia: 260,75~€
* Investície: 443,25~€

\enditems

Ďakujeme vám, že napriek okolnostiam a neistým ekonomickým vyhliadkam do budúcnosti, ste mnohí prispeli na činnosť a službu zboru. Aj naďalej máte možnosť prispieť do „nedeľnej zbierky“, a to prevodom na účet zboru. Do poznámky pre prijímateľa, prosím, uveďte „zbierka“.

Bankové spojenie: SK36 0900 0000 0000 1147 1836, SWIFT: GIBASKBX

Ďakujeme!


\n 2.	10.	Peter	ANTALÍK;
\n 6.	10.	Daniel	BALÁŽ;
\n 12.	10.	Barbora	PRIBULOVÁ;
\n 14.	10.	Martin	SIMON;
\n 20.	10.	Ida	PUČEKOVÁ;
\n 22.	10.	Hana	HALAMIČKOVÁ;
\n 25.	10.	Vladimír	IRA;
\n 27.	10.	Miriam	KRÁĽOVÁ;
\n 28.	10.	František	VRABČEK;
\n 28.	10.	Ľubomír	SYČ;
\n 31.	10.	Samuel	KORIŤÁK;
\narodeniny


\program{
\p 1  ; št ;.;;.;;
\p 2  ; pi ; 17.30 ; Dorast (Zrínskeho 2);.;;
\p 3  ; so ; 15.00 ; Mládež (Hodžovo nám.);.;;
\p 4  ; ne ;  9.00 ; Bohoslužby (D. Jones); 10.30 ; Chvojnica (J. Szőllős);
\p    ;    ; 10.30 ; Bohoslužby (D. Jones) ; 10.30 ; Detská besiedka (Zrínskeho 2);
\p 5  ; po ; 17.00 ; Modlitby -- ženy (Zrínskeho 2);.;;
\p 6  ; ut ;  9.30 ; Klubík (Zrínskeho 2) ; 15.15 ; Stret. pri Biblii (P. Pivka, Zrín. 2);
\p 7  ; st ;  6.00 ; Modlitby -- muži (kostol Palisády); 17.30 ; Stretnutie sestier (kostol Palisády);
\p 8  ; št ; 19.00 ; Biblická hodina (J. Szőllős, Zrínskeho 2);.;;
\p 9  ; pi ; 17.30 ; Dorast (Zrínskeho 2);.;;
\p 10 ; so ;  8.00 ; Zborový pracovný projekt ; 18.00 ; Mládež (Zrínskeho 2);
\p 11 ; ne ;  9.00 ; Bohoslužby (J. Szőllős); 10.00 ; Chvojnica (P. Antalík) ;
\p    ;    ; 10.30 ; Bohoslužby (J. Szőllős); 10.30 ; Detská besiedka (Zrínskeho 2);
\p 12 ; po ; 17.00 ; Modlitby -- ženy (Zrínskeho 2);.;;
\p 13 ; ut ;  9.30 ; Klubík (Zrínskeho 2) ; 15.15 ; Stret. pri Biblii (P. Pivka, Zrín. 2);
\p 14 ; st ;  6.00 ; Modlitby -- muži (kostol Palisády);.;;
\p 15 ; št ; 19.00 ; Biblická hodina (J. Szőllős, Zrínskeho 2);.;;
\p 16 ; pi ; 17.30 ; Dorast (Zrínskeho 2);.;;
\p 17 ; so ; 18.00 ; Mládež (Súľovská 2);.;;
\p 18 ; ne ;  9.00 ; Bohoslužby (D. Jones) ; 10.00 ; Chvojnica (P. Škulec);
\p    ;    ; 10.30 ; Bohoslužby (D. Jones) ; 10.30 ; Detská besiedka (Zrínskeho 2);
\p    ;    ; 11.30 ; Zborové členské zhromaždenie;.;;
\p 19 ; po ; 17.00 ; Modlitby -- ženy (Zrínskeho 2);.;;
\p 20 ; ut ;  9.30 ; Klubík (Zrínskeho 2) ; 15.15 ; Stret. pri Biblii (P. Pivka, Zrín. 2);
\p 21 ; st ;  6.00 ; Modlitby -- muži (kostol Palisády); 17.30 ; Stretnutie sestier (kostol Palisády);
\p 22 ; št ; 19.00 ; Biblická hodina (J. Szőllős, Zrínskeho 2);.;;
\p 23 ; pi ; 17.30 ; Dorast (Zrínskeho 2);.;;
\p 24 ; so ; 18.00 ; Mládež (Súľovská 2);.;;
\p 25 ; ne ;  9.00 ; Bohoslužby (P. Kolárovský) ; 10.00 ; Chvojnica (L. Taliga) ;
\p    ;    ; 10.30 ; Bohoslužby (P. Kolárovský) ; 10.30 ; Detská besiedka (Zrínskeho 2);
\p 26 ; po ; 17.00 ; Modlitby -- ženy (Zrínskeho 2);.;;
\p 27 ; ut ;  9.30 ; Klubík (Zrínskeho 2); 15.15 ; Stret. pri Biblii (P. Pivka, Zrín. 2);
\p 28 ; st ;  6.00 ; Modlitby -- muži (kostol Palisády);.;;
\p 29 ; št ; 19.00 ; Biblická hodina (J. Szőllős, Zrínskeho 2);.;;
\p 30 ; pi ; 17.30 ; Dorast (Zrínskeho 2);.;;
\p 31 ; so ; 18.00 ; Mládež (Súľovská 2);.;;
}

\tiraz
\bye
