%\typosize[9/12]% - pouzita velkost pisma/riadku - standard
\input makra.tex % nacitanie Ivanom pripravenych nastaveni a prikazov
\hyphenation{star-šov-stvo} % rozdelenie slov na konci riadku, treba tu uviest slova, ktore sam nepozna

\spravodaj{4}{2020}


\clanok {Nádej a dobrá správa uprostred koronavírusovej epidémie}
Úplne nová, pre naše generácie donedávna nepredstaviteľná situácia, nás núti, aby sme sa zastavili, prehodnotili naše plány a venovali čas poctivému hodnoteniu našich životov. Celý svet, celá Európa i naša krajina prechádza hlbokou krízou, ktorá dramaticky zasahuje nielen systém zdravotnej starostlivosti a hospodárstvo, ale veľmi osobne ovplyvňuje životy nás všetkých. Obmedzenia v~celej krajine, v~každodennom živote v~našich domovoch a na pracoviskách nás nútia, aby sme sa zastavili a pozreli sa do zrkadla našich životov. Naši politici vo svojich televíznych a rozhlasových vystúpeniach hovoria nielen o~všeobecných opatreniach proti šíreniu koronavírusu spôsobujúceho ochorenie Covid-19, ale často spomínajú aj obavy a úzkosti obyvateľstva. V~médiách zmizlo mnoho predvolebných a povolebných kontroverzných diskusií, ktoré sa javili ako veľmi dôležité ešte pred niekoľkými týždňami. Zrazu sú to lekári, vedci z~mnohých vedných odborov, psychológovia, psychiatri, ekonómovia, sociológovia, prognostici, filozofi, teológovia, pedagógovia a ďalší odborníci, ktorí dostali príležitosť, aby odpovedali na základnú otázku: Čo je potrebné robiť v~týchto neistých a zložitých časoch? Možno naša post-kresťanská spoločnosť si uvedomuje, že je čas začať prehodnocovať spôsob nášho života. Náš životný štýl, do veľkej miery založený na nadmernej spotrebe, neprimeranej závislosti na práci a niekedy bezbrehom individualizme, čelí „testu“ autentickosti.

My, ktorí vyznávame, že veríme v~Krista, nie sme oslobodení od ťažkostí a utrpenia tejto krízy. Z~Písma je zrejmé, že nás Pán Boh povoláva k~tomu, aby sme prehodnotili svoje priority a spôsoby služby a boli šíriteľmi nádeje. Vývoj šírenia epidémie, stúpajúca krivka nakazených koronavírusom a počtov úmrtí sa stáva čoraz naliehavejšou výzvou pre cirkev prosiť nášho Pána za našich blízkych, susedov a spolupracovníkov. Je možné, že mnohí stratia zamestnanie a prídu o~zdroje príjmov, a tak bude načase ukázať, ako sa hodnoty evanjelia prejavujú v~praxi pri riešení ťažkých situácií. Keď sa osobné a skupinové vzťahy zredukujú na minimum, ako kresťania v~našom meste, ako miestny zbor sme povolaní, aby sme spoluvytvárali spoločenstvo, ktoré je aktívne aj za hranicami našich zborov, našich modlitební a kostolov. Mnohí ľudia sa dnes pýtajú, či existuje „niekto“, kto má kontrolu nad touto veľmi zložitou situáciou. Odpoveď je: áno. Je tu Niekto a je k~dispozícii a je blízko, uprostred búrky koronavírusovej epidémie. Modlime sa za to, aby nás Pán Boh používal pre šírenie tejto dobrej správy. Výzva je mimoriadne naliehavá. My máme však nádej, pretože si uvedomujeme, že slúžime Bohu, ktorý stvoril tento vesmír a drží všetko vo svojich rukách.

Blížiace sa veľkonočné sviatky nám pripomínajú význam nádeje. Ak nemá byť odtrhnutá od bežného života, musíme ju vidieť v~kontexte toho, čím žijeme. Vzkriesením Krista nám Pán Boh dáva jasne najavo, že nás neopustil. Veľká noc je sviatok vzkriesenia Krista, ktorý nám pripomína, že Božia moc premáha všetku ťažobu, kriesi nádej a prebúdza život.

\autor{Vladimír Ira}


\clanok {Správy zo staršovstva}
Pozdravujeme vás bratia a sestry. Aj v~tomto čase izolácie sa chceme s~vami podeliť o~to, čím žijeme.

Situácia v~súvislosti s~koronavírusom a zákaz spoločného stretávania sa nás zastihla nepripravených. Počas doby komunizmu naše zbory zažili množstvo obmedzení. Boli to napríklad zbory bez kazateľov, zákaz kázať Slovo Božie alebo množstvo iných obmedzení. To, s~ktorým sme sa stretli v~roku 2020, je úplne jedinečné. Nie je tu preto, že si niekto niečo vymyslel z~dôvodu ideológie.  Obmedzenia sú tu pre našu ochranu.

V tomto čase rozmýšľame nad tým, ako zabezpečiť možnosť spoločného kontaktu. To, čo bolo bežné vo svete biznisu, sme v~cirkvi poznali iba veľmi okrajovo. Telekonferencie alebo porady na diaľku sme v~zboroch nepotrebovali. Uprednostňovali sme osobný kontakt. Dnes ak chceme zachovať vzájomný kontakt, potrebujeme v~maximálnej miere využívať technológie, ktoré nám Pán Boh daroval. A~toto je téma, ktorej sme venovali časť diskusie na staršovstve. Máme záujem o~to, aby podobným spôsobom mohli fungovať biblické hodiny, či „stretnutia“ skupiniek alebo zborových zložiek.

Sme vďační Bohu za tých vedúcich, ktorí už podobný spôsob komunikácie využívajú.

Zabezpečenie niektorých činností je ešte o~niečo náročnejšie. Pre zhodnotenie práce v~minulom roku a nastavenie pravidiel pre ďalší rok je potrebné sa stretnúť na výročnom zborovom členskom zhromaždení. V~tomto roku to asi nebudeme vedieť urobiť. Preto spolu s~volebnou komisiou hľadáme možnosti, ako zabezpečiť tieto rozhodnutia. O~výsledku vás budeme informovať a takisto vás zapojíme do spoločného rozhodovania.

Všetko to vyzerá príliš technicky a moderne. No napriek tomu je to naša snaha o~to „staré“, aby sme niesli evanjelium a bremená tých druhých, aby sme získavali učeníkov, alebo aby sme neopúšťali naše spoločenstvo v~tejto zmenenej dobe.

Výsledky parlamentných volieb zostávajú v~tieni vírusu a opatrení pre obmedzenie jeho šírenia. Napriek tomu som presvedčený, že všetci, ktorých sme zvolili do parlamentu a ktorí sa dostali do vlády, potrebujú naše modlitby. Sú to ľudia, ktorí sú v~prvej línii riadenia štátu a sú na nich vyvíjané veľké požiadavky. Modlime sa za ich múdrosť, silu, odvahu a aj vieru tých, ktorí vyznávajú svoju vieru v~Pána Ježiša Krista.

Modlime sa za nás, aby Pán Ježiš zachoval našu aj vašu vieru aj v~tomto čase obmedzení.

\autor {za staršovstvo zboru Peter Pribula st.}


\clanok {Bohoslužby počas veľkonočných sviatkov}
Bohoslužby na Zelený štvrtok v~našom zbore nebudú.

Ekumenické bohoslužby na Veľký piatok, ktoré tradične bývajú v~budove Istropolisu, bude možné sledovať naživo o~10.00 hod. prostredníctvom YouTube kanála Rádia 7 -- \ulink[https://bit.ly/2UHeloF]{https://bit.ly/2UHeloF}, resp. na Facebook-u Rádia 7. Zvukový prenos bude možné sledovať na frekvenciách Rádia 7: 103,6 MHz (Bratislava) a 91,7 MHz (Dúbravka). Božím Slovom poslúži náš brat kazateľ Danny Jones.

Bohoslužby na Veľký piatok v~našom zbore budú o~17.00 hod. a slúžiť bude br. farár Boris Mišina.

Bohoslužby na Veľkonočnú nedeľu budú v~riadnom čase o~9.30 hod. a slúžiť bude br. kaz. Ján Szőllős.

Prenosy bohoslužieb z~nášho zboru môžete sledovať ako obyčajne prostredníctvom živého prenosu nášho zboru: \ulink[https://bit.ly/3aL34co]{https://bit.ly/3aL34co}.


\clanok{Verš na zapamätanie}
Na mesiac apríl máme nový veršík, ktorý sa chceme spoločne učiť. Veríme, že poznanie Písma prospeje našej duši i našej mysli:

{\it „Nech povedia tí, čo sa boja Hospodina: Naveky trvá Jeho milosť. V~tiesni som vzýval Hospodina; Hospodin ma vypočul a vyslobodil ma.“}

\autor{Ž~118,~4~--~5}


\clanok{Zbierky za uplynulé obdobie}
Milí bratia a sestry, ďakujeme za vašu obetavosť. V~uplynulom období boli zbierky výrazne ovplyvnené skutočnosťou, že sme sa od polovice marca nemohli stretávať v~modlitebni na Palisádach. Už účasť na bohoslužbách 8. marca bola značne poznačená súčasnou situáciou. Z~toho pramenia aj nižšie čiastky venované na misiu a investície zboru:

\vskip-1ex\begitems
* Misia: 15,00~€
* Investície: 0~€

\enditems

Ďakujeme vám, že napriek okolnostiam a neistým ekonomickým vyhliadkam do budúcnosti ste mnohí prispeli na činnosť a službu zboru. Aj naďalej máte možnosť prispieť do „nedeľnej zbierky“, a to prevodom na účet zboru. Do poznámky pre prijímateľa, prosím, uveďte „zbierka“.

Bankové spojenie: IBAN: SK36 0900 0000 0000 1147 1836, SWIFT: GIBASKBX

Ďakujeme!


\n 1.	4.	Miroslav	KOLÁŘIK;
\n 4.	4.	Vierka	ŠKODÁKOVÁ;
\n 6.	4.	Jana	ZAJACOVÁ;
\n 6.	4.	Jarmila	CIHOVÁ;
\n 10.	4.	Anna	PAVLÍKOVÁ;
\n 11.	4.	Daniel	MIKLETIČ;
\n 16.	4.	Blažena	ŠKULECOVÁ;
\n 16.	4.	Clara	JONES;
\n 19.	4.	Marta	PRIBULOVÁ;
\n 22.	4.	Koloman	ERDÉLYI;
\n 25.	4.	Elena	TALIGOVÁ;
\n 30.	4.	Jaroslav	VOLENTIČ;
\n 30.	4.	Ľuboš	DZURIAK;
\narodeniny


\program{
\p 1 ; st ;.;;.;;
\p 2 ; št ;.;;.;;
\p 3 ; pi ;.;;.;;
\p 4 ; so ; 14.00 ; Sobáš Marty Pribulovej a Tomáša Barkóciho (online);.;;
\p 5 ; ne ;  9.30 ; Bohoslužby (D. Jones, online);.;;
\p 6 ; po ;.;;.;;
\p 7 ; ut ;.;;.;;
\p 8 ; st ; .;;.;;
\p 9 ; št ;.;;.;;
\p 10 ; pi ; 10.00 ; Ekumenické bohoslužby, Veľký piatok (D. Jones, online Radio 7); 17.00 ; Bohoslužby, Veľký piatok (B. Mišina, online);
\p 11 ; so ;.;;.;;
\p 12 ; ne ; 9.30 ; Bohoslužby, Veľkonočná nedeľa (J. Szőllős, online);.;;
\p 13 ; po ;.;;.;;
\p 14 ; ut ;.;;.;;
\p 15 ; st ;.;;.;;
\p 16 ; št ;.;;.;;
\p 17 ; pi ;.;;.;;
\p 18 ; so ;.;;.;;
\p 19 ; ne ; 9.30 ; Bohoslužby (S. Kráľ, online);.;;
\p 20 ; po ;.;;.;;
\p 21 ; ut ;.;;.;;
\p 22 ; st ;.;;.;;
\p 23 ; št ;.;;.;;
\p 24 ; pi ;.;;.;;
\p 25 ; so ;.;;.;;
\p 26 ; ne ; 9.30 ; Bohoslužby (T. Valchář, online);.;;
\p 27 ; po ;.;;.;;
\p 28 ; ut ;.;;.;;
\p 29 ; st ;.;;.;;
\p 30 ; št ;.;;.;;
}

\tiraz
\bye
